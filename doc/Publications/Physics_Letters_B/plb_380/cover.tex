
\documentstyle[aps,epsf,psfig]{revtex}


\thispagestyle{empty}
\newcommand{\HRule}{\rule{\linewidth}{1mm}}
\setlength{\parindent}{0mm}
\setlength{\parskip}{0mm}

\begin{document}
   \HRule
     \begin{center}
         \Huge {\bf ECT* PREPRINT}
     \end{center}
   \HRule
%  here you can put in the date

   \vspace*{\stretch{2}}
%   here you can have the title of the preprint
   \begin{center}
      \Large {\bf  Massive Quark Matter in Neutron stars}
   \end{center}
%  here you can put your name and those of possible friends
   \begin{center}
      \large A.\ Drago$^a$, U.\ Tambini$^a$ and
      M.\ Hjorth--Jensen$^b$
    \end{center}
%   The addresses
    \begin{center}
     \large $^a$Dipartimento di Fisica, Universit\'{a} di Ferrara, and INFN,
    Sezione di Ferrara, I-44100 Ferrara, Italy 
    \end{center}
    \begin{center}
      \large   $^b$ECT*, European Centre for Theoretical
        Studies in Nuclear Physics and Related Areas,
        Trento, Italy
    \end{center}

    \vspace*{\stretch{2}}
%   This may be optional , you could for instance plug the title
%   of a talk given somewhere, a dedication to your mum etc.
    \begin{center}
      \large   Physics Letters {\bf B}, in press
    \end{center}
    \vspace*{\stretch{4}}
%    the preprint number and the ect* address
    \begin{center}
        \Large {\bf ECT* preprint $\#$: ECT*/MAY/95--03}
    \end{center}
    \begin{figure}[hbtp]
        \begin{center}
        {\centering\mbox{\psfig{figure=villa2.ps,height=7cm,width=10cm}}}
        \end{center}
     \end{figure}
    \HRule
    \begin{center}
        \Large {\bf European Centre for Theoretical Studies in Nuclear
        Physics and Related Areas}
    \end{center}
    \begin{center}
        {\bf Strada delle Tabarelle 286, I--38050 Villazzano (TN),
        Italy}
    \end{center}
    \begin{center}
        {\bf tel.\ +39--461--314730, fax.\ +39--461--935007}
    \end{center}
     \begin{center}
        {\bf e--mail: ectstar@ect.unitn.it, www: http://www.ect.unitn.it}
    \end{center}
    \HRule

\clearpage




\draft
\title{Massive Quarks in Neutron Stars}

\author{A.\ Drago and U.\ Tambini}

\address{Dipartimento di Fisica, Universit\'{a} di Ferrara, and INFN,
Sezione di Ferrara, Via Paradiso 12, I-44100 Ferrara, Italy}

\author{M.\ Hjorth-Jensen}

\address{ECT*, European Centre for Theoretical
Studies in Nuclear Physics and Related Areas, Trento, Italy}



\maketitle

\begin{abstract}

We study various neutron star properties
using the Color--Dielectric model 
to describe quark matter. For the baryon
sector at low densities we employ 
the Walecka model.
Applying Gibbs criteria to this composite system, we find
that, for matter at $\beta$-equilibrium, the pure hadronic phase ends at 
$0.11$ fm$^{-3}$ and that the mixed quark and hadronic phase  
extends to  
$0.31$ fm$^{-3}$. 
The resulting equation of state yields a maximum neutron
star mass of $1.59 M_{\odot}$.
A neutron star with total mass
of $1.4 M_{\odot}$
will consist of a crust made of hadronic matter only, 
a $\sim$ 1 km thick region of mixed phase and 
a core composed of pure  
quark matter. Implications for the cooling of neutron stars
are discussed.

\end{abstract}

\pacs{PACS numbers: 97.60.Jd, 12.39.-x, 24.85.+p}


\clearpage


The equation of state (EOS) for dense matter is central to calculations of
neutron--star properties, such as the mass range, the mass--radius
relationship, the crust thickness
and the cooling rate,
see
e.g.\  Refs.\ \cite{prl95,lrp93}. The same EOS is also crucial
in calculating the energy released in a supernova explosion.

The typical density range of a neutron star
stretches from central densities of the order of 5 to 10
times the nuclear matter saturation density $n_0=0.17$
fm$^{-3}$ to very small values
at the edge of the star. Clearly,
the relevant degrees of freedom will not be the same in the crust,
where the density is much smaller than $n_0$, and in the center
of the star where the density is so high that models based
solely on interacting
nucleons are questionable.
Data from
neutron stars indicate that the EOS  
should probably be moderately stiff in order to
support maximum neutron star masses in a range
from approximately 
$1.4 M_{\odot}$ to $1.9 M_{\odot}$ \cite{thorsett93}. In addition,
simulations of supernovae explosions seem to require an EOS which is 
even softer. A combined analysis of the data coming from binary pulsar
systems and from neutron star formation scenarios can be found
in Ref. \cite{finn}, where it is shown that neutron star masses should fall
predominantly in the range $1.3\le M/M_{\odot}\le 1.6 $.

The aim of this work is to 
study properties of neutron stars like cooling rate,
total mass and radius employing a massive quark model, the
so--called Color--Dielectric model (CDM) \cite{pirner92,birse90,dfb95}.
The CDM is a confinement model which has been used with success 
to study 
properties of 
single nucleons, such as structure functions \cite{barone}
and form factors \cite{ff}, or to describe the
interaction potential between two nucleons \cite{kurt}, or to investigate
quark matter \cite{dfb95,mitja}.
In particular,
it is possible, using the same set of parameters,
both to describe the single nucleon properties 
and to obtain meaningful results
for the deconfinement phase transition \cite{dfb95}.
The latter happens at a density of the order
of 2--3 times $n_0$ when symmetric nuclear matter is considered, 
and at even smaller densities for matter in $\beta$--equilibrium,
as discussed below in this work.

Another important feature is 
that  effective quark masses in the CDM are always larger than
a value of the order of 100 MeV, hence chiral symmetry is broken and
the Goldstone bosons are relevant degrees of freedom. This is to be
contrasted with models like the MIT bag, 
where quarks have masses of a few MeV.
We therefore expect the CDM to be relevant 
for computing the cooling rate of neutron stars {\it via} 
the URCA
mechanism, as suggested by Iwamoto \cite{iwa}.



The Lagrangian\footnote{
Throughout this paper we set $G=c=\hbar=1$, where $G$ is the gravitational 
constant.}
of the model is given by:
\begin{eqnarray}
     {\cal L} &=& i\bar \psi \gamma^{\mu}\partial_{\mu} \psi \nonumber\\
     &+&\!\!\!\sum_{f=u,d} {g_f\over f_\pi \chi} \, \bar \psi_f\left(\sigma
     +i\gamma_5\vec\tau\cdot\vec\pi\right) \psi_f 
     +{g_s \over \chi} \, \bar \psi_s \psi_s       \nonumber
     \\
      &+&{1\over 2}{\left(\partial_\mu\chi\right)}^2
      -V \left(\chi\right) \\
     \label{eq:in1}
     &+&{1\over 2}{\left(\partial_\mu\sigma \right)}^2
     +{1\over 2}{\left(\partial_\mu\vec\pi\right)}^2
     -U\left(\sigma ,\vec\pi\right)   \nonumber\, ,
\end{eqnarray}
where $U(\sigma ,\vec\pi)$ is the ``mexican-hat'' potential, as in
Ref.\ \cite{NF93}.
The lagrangian ${\cal L}$  
describes a system of interacting $u$, $d$ and $s$ quarks, pions, sigmas and
a scalar--isoscalar chiral singlet field $\chi$
whose potential $U(\chi)$ is given by
\begin{equation}
    V(\chi)={1\over 2}{\cal M}^2\chi^2.
    \label{eq:in2}
\end{equation}
 
The coupling constants are given by $g_{u,d}=g (f_{\pi}\pm \xi_3)$
and $g_s= g(2 f_K -f_{\pi})$, where $f_{\pi}=93$ MeV and $f_{K}=113$ MeV 
are the pion and the kaon 
decay constants, respectively, and $\xi_3=f_{K^\pm}-f_{K^0}=-0.75$ MeV. 
These coupling constants depend 
only on a single parameter $g$.
The $SU(3)_f$ version of the model has been
introduced by Birse and McGovern \cite{su3,su3col}.

When considering a single hadron,
confinement is obtained {\it via} the effective quark masses
$m_{u,d}=g_{u,d} \bar\sigma/(\bar\chi f_\pi)$ and 
$m_s=g_s / \bar\chi$,
which diverge outside the nucleon.
Indeed, the classical fields
$\bar \chi$ and $\bar\sigma$ are solutions of the Euler--Lagrange equations
and
$\bar\chi$ goes asymptotically to zero at large distances.

The model parameters $g$ and ${\cal M}$ are fixed so as to 
reproduce the experimental mass and radius of the nucleon.
In order to describe the single nucleon state, a double projection on linear
and angular momentum eigenstates has to be performed, see Ref.\
\cite{NF93}.
We will use the parameters $g=0.023$ GeV and
${\cal M}=1.7$ GeV, giving a nucleon isoscalar radius of 0.80 fm 
(exp.val.=0.79 fm) and an average
delta--nucleon mass of 1.129 GeV (exp.val.=1.085 GeV). A similar set
of parameters has been used to compute structure functions \cite{barone}
and form factors \cite{ff}.

The quark matter (QM) phase is characterized by a constant value of the 
scalar fields and by using plane waves to describe the quarks.
The total energy of QM in the mean field approximation reads
\begin{equation}
      E_{QM}=6 V \!\!\sum_{f=u,d,s}\int\!\!\!
      {d{\bf k} \over (2\pi)^3}
      \sqrt{{\bf k}^2 + m_f^2}
      \theta (k_F^f-k)
      + V U(\bar \chi) + V W(\bar \sigma,\vec\pi=0),
      \label{eq:in4}
\end{equation}
where $k_{F}^f$ is the Fermi momentum of quarks with flavour $f$.

We will employ the CDM model to describe the deconfined quark
matter phase.
The high--density matter 
in the interior of neutron stars is described by
requiring the system to be locally neutral
\begin{equation} 
    \label{eq:neut}
    (2/3)n_u -(1/3)n_d - (1/3)n_s - n_e = 0,
\end{equation}
where $n_{u,d,s,e}$ 
are the densities of the $u$, $d$ and $s$ quarks and of the
electrons, respectively. 
Morover, the system must be in $\beta$--equilibrium, i.e.\
the chemical potentials have to satisfy the following equations:
\begin{equation}
      \label{eq:ud}
      \mu_d=\mu_u+\mu_e,
\end{equation}
and
\begin{equation}
      \label{eq:us}
      \mu_s=\mu_u+\mu_e .
\end{equation}
Eqs.\ (\ref{eq:neut})--(\ref{eq:us}) have to be solved 
self--consistently together with
field equations, at a fixed baryon density $n=n_u+n_d+n_s$.

To describe the hadronic phase, we employ a relativistic field
theoretic model of the Walecka type \cite{sw86}, 
including protons and neutrons
only. The parameters used to define the Lagrangian of the hadronic part
are given in the work of Horowitz and Serot \cite{hs81} and
used recently by Knorren {\em et al.} \cite{kpe95}, labelled HS81
in Ref.\ \cite{kpe95}.
The hadronic phase is also required to be in $\beta$--equilibrium,
and the equations corresponding to eqs.\
(\ref{eq:neut})--(\ref{eq:us}) at a fixed baryon density are 
\begin{equation} 
    \label{eq:neut_np}
    n_p= n_e,
\end{equation}
and 
\begin{equation}
      \label{eq:np_chem}
      \mu_n=\mu_p+\mu_e,
\end{equation}
where the subscripts $p$  and $n$ 
refer to protons and neutrons, respectively.

For the mixed phase, we treat the 
multi--component system
following  recents works of 
Glendenning \cite{glen}
and M\"uller and Serot \cite{serot}. This gives a mixed phase of quarks
and hadrons 
which extends from $0.11$ fm$^{-3}$ to $0.31$ fm$^{-3}$, whereas for
higher densities matter is described by a deconfined quark phase only.
In Fig.\ \ref{fig:fig1} we notice that the pressure exhibits a
monotonic increase in the density region corresponding to the 
mixed phase. This should be contrasted to the case where only
one conserved charge is present, as discussed in depth by Glendenning
\cite{glen}. In the present work, we need to obey conservation
of electric charge and baryon number.  


From the general theory of relativity,
the structure of a static neutron star is determined through the
Tolman--Oppenheimer--Volkov equation, i.e.\
\begin{equation}
   \frac{dP}{dr}=
    - \frac{ \left\{\rho (r)+P(r) \right\}
   \left\{M(r)+4\pi r^3 P(r)\right\}}{r^2- 2r M(r)},
   \label{eq:tov}
\end{equation}
where $P(r)$ is the pressure and $M(r)$ is
the gravitational mass inside a radius $r$. The equation
of state discussed in Fig.\ \ref{fig:fig1} is then used
to evaluate the total mass and radius of the neutron star. At very low
densities ($0\leq \rho \leq 0.08$ fm$^{-3}$), the model
for the pure hadronic phase gives a negative pressure. At these
densities, we link therefore our EOS with that of Malone {\em et al.}
\cite{mjb75}. The resulting mass and radius exhibit only a weak dependence
on the structure of the EOS at very low densities.
The resulting mass           
is shown in Fig.\ \ref{fig:fig2}.
With the CDM we obtain a maximum mass
$M_{\rm{max}}\approx 1.59 M_{\odot}$
and a radius of 10.5 km at a central density
corresponding to approximately 7 times nuclear matter
saturation density, in good agreement with the experimental values
for the mass \cite{thorsett93,finn}. However, considerations about
the maximum mass of a neutron star are not sufficient to discriminate
between various equations of state, such as those derived
within non--relativistic and relativistic 
baryonic many--body theories \cite{behoo94,wff88}
or those employing approaches similar to the present work \cite{glen}.
It is worth comparing with the work of 
Glendenning \cite{glen}, which differs from ours in the 
model used for the pure quark phase. In Ref.\ \cite{glen} the MIT bag model
is used, with masses for $u$ and $d$ quarks set to zero, while
the mass of the $s$ quark is set to $150$ MeV. In the CDM,
the masses of the $u$ and $d$ quarks are of the order of $100$ MeV
for all densities of interest. Another important point differentiating
the two models is the vacuum pressure, which in the MIT model is of the
order of $150$ MeV and in the CDM is $\sim 50$ MeV. 
This yields a
rather different composition of a neutron star compared with the
results of Ref.\ \cite{glen}. In Fig.\ \ref{fig:fig3}
we show the baryon and quark composition of a neutron star
with central density $0.7$ fm$^{-3}$, mass of 
$1.41 M_{\odot}$ and radius $R=10.52$ km, obtained with the CDM model.
One observes that most of the star is
composed of quark matter only, with a mixed phase which
extends from $8.4$ km to  $10.1$ km, and a crust region of pure
baryonic matter. The structure of our star is therefore substantially
different from that obtained using the MIT model, where
the mixed phase encompasses most of the interior of the star, see
Fig.\ 8 of Ref.\ \cite{glen}, and the pure  quark matter phase is
never reached. The deconfinement transition turns out to be much
smoother in the CDM than in the MIT. Actually, there is the possibility that
the transition in the CDM becomes a smooth crossover, if one
considers correlations beyond the mean-field approximation\cite{rinaldo}.

A more stringent test to ascertain the validity of the various models
is the computation of the cooling time of the neutron star.
The composition of the star is crucial for
neutrino and antineutrino emission,
which can be responsible for the rapid cooling of neutron stars (URCA
mechanism). 
In traditional scenarios of non-relativistic and relativistic 
nuclear physics the 
so-called direct URCA process can start at large densities only, of the
order of $0.5$--$0.7$fm$^{-3}$ \cite{behoo94,elg96}. 
Various modified URCA processes have therefore been considered in 
more traditional studies
of neutron star cooling. 
It is however an
open question
whether approaches based on baryonic
degrees of freedom  only are applicable at densities
of the order of $0.5$--$0.7$fm$^{-3}$.
It is therefore of interest to see whether the direct URCA process
can start if the interior of the star consists of quark matter.
We discuss now this process considering quark matter
described by the CDM.
In the interior of the star shown in 
Fig.\ \ref{fig:fig3}, where also strange quark matter
is present, the relevant reactions are:
\begin{equation}
     d\rightarrow u + e^- + \bar\nu_e,
\end{equation}
\begin{equation}
     e^- + u \rightarrow  d + \nu_e, 
\end{equation}
\begin{equation}
     s\rightarrow u + e^- + \bar\nu_e, 
\end{equation}
\begin{equation}
     e^- + u \rightarrow  s + \nu_e .
\end{equation}
To conserve momentum in the reactions, the following inequalities have to
be satisfied:
\begin{equation}
    \label{kud}
    \vert k_F^u - k_F^e \vert \leq k_F^d \leq  k_F^u + k_F^e,
\end{equation}   
\begin{equation}
    \vert k_F^u - k_F^e \vert \leq  k_F^s \leq  k_F^u + k_F^e. 
    \label{kus}
\end{equation}   

For densities larger than 0.53 fm$^{-3}$, 
conditions (\ref{kud}) are satisfied
and the direct URCA mechanism involving only $u$ and $d$ quarks can start. 
Conditions (\ref{kus}) are satisfied
only for densities larger than 1.4 fm$^{-3}$. 
At lower densities, we have various modified URCA processes with quarks,
and at densities below $0.31$ fm$^{-3}$, also 
the corresponding modified URCA processes with protons and neutrons.
The direct URCA mechanism
involving strange quarks is 
suppressed by a factor $\sin ^2\theta_c\simeq 0.05$ with respect to the
previous process,
where $\theta_c$ is the Cabibbo angle.

Here we focus on the direct URCA process for quarks.
We compute the neutrino and antineutrino luminosity, using the
ultrarelativistic expansion for the chemical potentials \cite{iwa}. This
approximation is reasonable
because the ratio $m_q/k_F$ is approximately equal
to 0.25.
One gets for the total luminosity \cite{iwa}
\begin{equation} 
   \epsilon={457\over 1680} G_F^2 \cos^2\theta_c
    m_d^2 f k_F^u (k_B T)^6,
\end{equation}
where $f=1-(m_u/m_d)^2 (k_F^d/k_F^u) - (m_e/m_d)^2 (k_F^d/k_F^e)$.
To obtain the characteristic cooling time we equate the energy loss per
unit volume to the rate of change of thermal energy per unit volume
$\tau=c_V T/\epsilon$. Here
the heat capacity of the QM is $c_V=\sum_{f=u,d,s} m_f k_F^f k_B^2 T$.
We get $\tau=C\,\, 1$day$/T_9^4$, where $T_9$ is the temperature measured in
units of $10^9 \,^{\rm o} {\rm K}$ 
and $C$ is a constant ranging from 0.5 to $\sim$ 10
going from heavy to light neutron stars.
The cooling time obtained considering quark matter described by the
MIT model is roughly one order of magnitude smaller, $\tau\simeq 1$hour$/T_9^4$
\cite{iwa}.
However, in order to compare with observation,
the structure of the star has to be computed in details.
In particular the possible presence of superfluidity
in the interior has to be
considered. The superfluid would suppress the $\nu,\bar{\nu}$ emissivity and
would allow for 
reheating through friction with the crust.
There are also  
indications \cite{umeda,vanriper}
that temperatures of young ($\sim 10^4$ years old) neutron stars lie
below that obtained through the so--called modified  URCA processes.
One has also to note \cite{Page94}
that the modified URCA processes are 
weakly dependent on the mass of the star,
i.e. on the central density, while
faster cooling mechanisms like the above direct
URCA processes are in general strongly
dependent on it. Thus,
the detection of two coeval stars, whose temperatures differ by a factor
of the order of 2 or larger
would allow to distinguish between traditional cooling
scenarios, like those discussed by Page \cite{Page94}, and more exotic
ones like the above direct quark URCA reactions.
Such a huge variation in the
temperature of coeval stars would indicate the presence of a threshold
in the cooling mechanism, triggered by the density of the star \cite{SAX}.

In summary, the principal properties of a neutron star, described using
the CDM, are in agreement with observations. When comparing with the
MIT model, one should also consider the very recent result from 
lattice QCD studies
\cite{lattice}, indicating that at finite density the deconfinement 
transition is a smooth crossover and that at low temperatures chiral
symmetry remains broken at all densities. Both these indications
points in the direction of the CDM.

This work has been supported by the Istituto Nazionale di Fisica
Nucleare (INFN) , Italy, the Istituto Trentino di Cultura, Trento, Italy, 
and the Research Council of Norway. We are also indebted to Profs.\
L.\ Caneschi, D.\ Mukhopadyay and E.\ \O stgaard for useful
comments on the manuscript.

\clearpage

%           References

\begin{thebibliography}{99}
\bibitem{prl95} C.J.\ Pethick, D.G.\ Ravenhall and C.P.\ Lorenz,
Nucl.\ Phys.\  A584 (1995) 675.
\bibitem{lrp93} C.P.\ Lorenz, D.G.\ Ravenhall and C.J.\ Pethick,
Phys.\ Rev.\ Lett.\  70 (1993) 379.
\bibitem{thorsett93} S.E.\ Thorsett, Z.\ Arzoumanian, M.M.\ McKinnon
and J.H.\ Taylor, Astrophys.\ J.\  405 (1993) L29.
\bibitem{finn} L.S.\ Finn,\ Phys.\ Rev.\ Lett.  73 (1994) 1878.
\bibitem{pirner92} H.J.\ Pirner, Prog.\ Part.\ Nucl.\ Phys.\  29
(1992) 33.
\bibitem{birse90} M.C.\ Birse, Prog.\ Part.\ Nucl.\ Phys.\  25
(1990) 1.
\bibitem{dfb95} A.\ Drago, A.\ Fiolhais and U.\ Tambini,
Nucl.\ Phys.\  A588 (1995) 801.
\bibitem{barone}V.\ Barone\ and A.\ Drago,\ Nucl.\ Phys.\  A552
(1993) 479;  A560 (1993) 1076;
V.\ Barone,\ A.\ Drago and M.\ Fiolhais,\ Phys.\ Lett.\ B 338 (1994) 433.
\bibitem{ff} M.\ Fiolhais, T.\ Neuber and K.\ Goeke,\ Nucl.\ Phys.\
 A570 (1994) 782; A.\ Drago, M.\ Fiolhais,\ U.\ Tambini, in preparation.
\bibitem{kurt}K.\ Br\"auer,\ A.\ Drago
and A.\ Faessler, Nucl.\ Phys.\  A511 (1990) 558.
\bibitem{mitja}W.\ Broniowski,\ M.\ \^Cibej,\ 
M.\ Kutschera\ and M.\ Rosina,\
Phys.\ Rev.\ D  41 (1990) 285.
\bibitem{iwa} N.\ Iwamoto,\ Ann.\ Phys.\  141 (1982) 1.
\bibitem{NF93} T.\ Neuber, M.\ Fiolhais, K.\ Goeke and J.N.\ Urbano,\
Nucl.\ Phys.\  A560 (1993) 909.
\bibitem{su3}J.A.\ McGovern\ and M.C.\ Birse,\ Nucl.\ Phys.\  A506
(1990) 367;  A506 (1990) 392.
\bibitem{su3col}J.A.\ McGovern,\ Nucl.\ Phys.\  A533 (1991) 553.
\bibitem{sw86} B.D.\  Serot and J.D.\ Walecka, Adv.\ Nucl.\ Phys.\
 16 (1986) 1.
\bibitem{hs81} C.J.\ Horowitz and B.D.\ Serot, Nucl.\ Phys.\
A368 (1981) 503.
\bibitem{kpe95} R.\ Knorren, M.\ Prakash and P.J.\ Ellis,
Phys.\ Rev.\ C 52 (1995) 3470.
\bibitem{glen}N.K.\ Glendenning, \ Phys.\ Rev.\  D46 (1992) 1274.
\bibitem{serot}H.\ M\"uller and B.D.\ Serot,\ Phys.\ Rev.\ C52 
(1995) 2072
\bibitem{mjb75} R.C.\ Malone, M.B.\ Johnson and H.A.\ Bethe,
Astrophys.\ J.\ 199 (1975) 741
\bibitem{behoo94} L.\ Engvik, M.\ Hjorth-Jensen, E.\ Osnes, G.\ Bao
and E.\ \O stgaard, \ Phys.\ Rev.\ Lett.\  73 (1994) 2650; Astrophys.\
J.\ in press.
\bibitem{wff88} R.B.\ Wiringa, V.\ Fiks and A.\ Fabrocini,
Phys.\ Rev.\ C 38 (1988) 1010.
\bibitem{rinaldo} R.\ Cenni, A.\ Drago, P.\ Saracco and U.Tambini, 
work in progress.
\bibitem{elg96} \O.\ Elgar\o y, L.\ Engvik. E.\ Osnes, F.V.\ De Blasio,
M.\ Hjorth--Jensen and G.\ Lazzari, Phys.\ Rev.\ Lett., in press.
\bibitem{umeda}H.\ Umeda, N.\ Shibazaki, K.\ Nomoto and S.\ Tsuruta,
Astr.\ Journ.\  408 (1993) 186.
\bibitem{vanriper}K.\ VanRiper, R.\ Epstein and B.\ Link,
 Astrophys.\ J.\ (1995) 448.
294.
\bibitem{Page94} D.\ Page, Astrophys.\ J.\  428 (1994) 250.
\bibitem{SAX} A.\ Drago et al., Proposal for SAX core program,
November 1995.
\bibitem{lattice} T.\ Blum, J.E.\ Hetrick and D.\ Toussaint, 
Phys.\ Rev.\ Lett.\ 76 (1996) 1019.
\end{thebibliography}

\clearpage

\listoffigures




\begin{figure}[hbtp]
       \setlength{\unitlength}{1mm}
       \begin{picture}(100,180)
       \put(25,0){\epsfxsize=12cm \epsfbox{fig1.eps}}
       \end{picture}
       \caption{Energy density and pressure
                as function of density $n$.
                The mixed phase 
                begins at $0.11$ fm$^{-3}$ and ends at 
                $0.31$ fm$^{-3}$.}
                \label{fig:fig1}
\end{figure}



\begin{figure}[hbtp]
       \setlength{\unitlength}{1mm}
       \begin{picture}(100,180)
       \put(25,0){\epsfxsize=12cm \epsfbox{fig2.eps}}
       \end{picture}
       \caption{$M/M_{\odot}$ as function of central density.}
                \label{fig:fig2}
\end{figure}



\begin{figure}[hbtp]
       \setlength{\unitlength}{1mm}
       \begin{picture}(100,180)
       \put(25,0){\epsfxsize=12cm \epsfbox{fig3.eps}}
       \end{picture}
       \caption{Baryon and quark composition of a neutron star
                with central density $0.7$ fm$^{-3}$, mass of 
                $1.41 M_{\odot}$ and radius $R=10.52$ km
                as function of the distance from the center.
                To help reading the figure, a different length's scale
                has been used where the mixed phase is formed.}
       \label{fig:fig3}
\end{figure}


\end{document}



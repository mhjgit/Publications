\documentclass{article}

\begin{document}

\large Comments to the referee\normalsize

\vspace*{0.5cm}
It is true that the results for the Fe nuclei in the resubmission differ 
somewhat from the first submission at high excitation energies. This is due to
an improved error estimate of the first-generation matrix and does not affect 
the observed step structures under discussion at low excitation energies.

The referee's claim that statistical assumptions are not justified at low
excitation energies (i.e.\ at around $2\Delta$) is unsubstantiated and, we 
believe incorrect. We have given several references from other authors in the 
scientific literature in our previous reply and in the resubmission where 
statistical assumptions have been applied with success at similarly low 
excitation energies, and for the same nuclei as in the present work. We will 
give three additional arguments here:
\begin{itemize}
\item The good agreement between our data and the curve from counting of 
discrete levels at around $2\Delta$ gives {\em experimental} support for our
assumption of the validity of statistical methods in the present case. As we 
have pointed out in our previous reply, experimental evidence rather than 
{\em a priori} assumptions should be the deciding factor.
\item In numerous instances of our previous work (see, e.g., nucl-ex/0307009 
and ref.\ [2-12,21,22] therein) we have been able to extract consistent level 
density curves around $2\Delta$ for many rare-earth nuclei such as 
$^{148,149}$Sm, $^{160,161,162}$Dy, $^{166,167}$Er, and $^{170,171,172}$Yb.
\item In ref.\ [8] of the present work, we have pushed our method with success 
to the very light nuclei $^{27,28}$Si and covering all excitation energies 
between 0 and the neutron binding energy. Again, this constitutes an 
{\em experimental} counterargument against the referee's assumptions.
\end{itemize}
We would like to clarify our views on the limits of the statistical model.
Statistical assumptions are not justified to calculate any specific transition 
matrix element between any two explicit states. In this case, using the 
statistical model, one can only get an average expectation value for the 
partial transition width (around which the experimental value is distributed, 
most often obeying a Porter-Thomas distribution). Thus, for these questions,
more detailed nuclear structure calculations are preferred. If one, on the 
other hand, averages over a sufficient number of states within a given 
excitation energy bin, one can expect that the average partial decay width for 
the ensemble of states with equal spin and parity to any given, explicit state 
(including the ground state and any other discrete low-lying level) is well 
described by the statistical model. This is an {\em experimental} observation. 
Further, we will discuss the inverse reaction, namely photoexcitation on the 
ground state. With this method, one is, e.g., able to observe the Giant Dipole 
Resonance (GDR) built on the ground state of even-even, odd-mass, and odd-odd 
nuclei on and off closed shells, all of them having very different types of 
ground-state wave functions. Essentially, one cannot see any effect of the type
of ground-state wave function on the applicability of this type of 
{\em experiment}. The reason is again, that the average partial radiative width
connecting the different types of ground states and any ensemble of excited 
states within an energy bin of sufficient width (i.e.\ containing a sufficient 
number of states) is well described by the statistical model. We can therefore 
conclude that the applicability of statistical assumptions does neither hinge 
on a certain minimal value of the level density around the initial or final 
states, nor on both states being complex states (i.e.\ one- or more 
quasiparticle states, compound-like states, or states in nuclei far from closed
shells) as implied by the referee. The applicability of the statistical model 
is simply governed by averaging over a sufficient number of states. The physics
discussed in this paragraph is expertly reviewed in the 30-year old ref.\ [3] 
of the present manuscript.

We agree with the referee that for the lowest excitation energies (where the 
curve from counting discrete levels is not drawn in Fig.\ 4), the agreement 
between the present data and the discrete level scheme is not good. There are
several well-understood reasons for this :
\begin{itemize}
\item The level density is so low, and the spacing so high that the binning 
into 240~keV bins is too fine, thus giving rise to huge fluctuations in the 
level density curve determined from counting of discrete levels. However, on 
average, the present data, when integrated over this energy region gives 
approximately the right number of levels in the discrete level scheme. Thus, 
with larger bins, (of the order of 1~MeV), the comparison is favorable.
\item The discrete levels at these excitation energies have very biased spin 
and parity distributions. If we assume that, by the initial reaction, we 
populate mainly spin 4--6 states, and that the primary-$\gamma$ spectrum is 
dominated by dipole transitions, one simply cannot populate the dominating spin
0 or 2 states at very low excitation energies. This can lead to an 
underestimation of the level density by our method.
\item For the inelastic reaction (leading to the odd nuclei in the present 
work), we have observed in the past [A.~Schiller \sl et al.\rm, Phys.\ Rev.\ C
\bf 61\rm, 044324 (2000)] that the level density for collective excitations 
near the ground state can be overestimated by our method.
\end{itemize}
We note that the referee states that 'the discrete level curve should be 
continued to zero and this would indicate the limits of the method'. We fully 
agree with this statement and would like to comment that in all four nuclei 
under study in the present work, the level density curve from counting of 
discrete levels agrees very well with the present data down to energies far 
below $2\Delta$. Thus, following the referee's own prescription the method 
works well even around excitation energies of $2\Delta$.

We agree with the referee that the method cannot produce sharper level density
steps than the resolution of the detectors. However, this has never been 
claimed in our manuscript. The resolution of our detectors is given in the 
manuscript (roughly 250~keV for the particle detectors and 6\% at 1.3~MeV, 
i.e., 80~keV for the $\gamma$ detectors). This is smaller than the widths of 
the claimed step structures (roughly 480-720~keV, i.e., 2-3 data points).

We agree with the referee on his comment regarding the counting of discrete 
levels. It is true that the limit of completeness of a level scheme is often 
determined in the manner described by the referee. However, this is not 
necessarily a good technique, since structures in the level-density curve at 
low excitation energies like those discussed in the present work might prevent 
one from finding the right limit. In any case, the present method of 
{\em measuring} the level density curve between the complete level scheme of 
discrete levels and the value determined from neutron resonance spacing should 
certainly take precedence over the {\em assumption} of an essentially 
exponential curve, as promoted by the referee in his second report. Further, 
the argument of the referee could be turned around and applied to the present 
work in the following way: the lower limit of validity of the present method 
should be determined by the onset of disagreement between the present data and 
the discrete level scheme, which, as shown in Fig.\ 4, is far below $2\Delta$ 
(see also our previous comments).

The theory in the present work is an entirely microscopic (though very 
schematic and limited) calculation of a nuclear level density beyond the 
pairing gap using seniority conserving and non-conserving interactions. It is
included to illustrate the interplay between these two kinds of interactions in
creating step structures of different shapes and magnitudes in the 
level-density curve. Devoting one page of six to this calculation does not seem
to deserve the description 'lengthy theory part'.

\end{document}


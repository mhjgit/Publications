 \documentstyle[12pt]{article}
\topmargin=0cm
\headheight=0cm
\headsep=10mm
\footheight=2.5cm
\hoffset -1cm
\voffset -10mm
\parindent=10mm
\textheight=245mm
\textwidth=165mm
\begin{document}
\begin{titlepage}
\begin{center}
            {\Large\bf
The spin-neutrino correlation revisited in $^{28}$Si muon-capture:
                  a new determination of the induced pseudoscalar
coupling ${\bf g_P/g_A}$}
             \footnote{The work was supported in part
             by the International Science Foundation} \\ [1cm]

{\large V.Brudanin, V.Egorov, T.Filipova, T.Mamedov, A.Salamatin, \\
   Yu.Shitov, Ts.Vylov, I.Yutlandov, Sh.Zaparov  \\
 {\normalsize\it  ( Joint Institute for Nuclear Research, 141980 Dubna,
Russia )}\\
           J.Deutsch, R.Prieels \\

 {\normalsize\it ( Universit{\'e} Catholique de Louvain, B-1348
Louvain-la-Neuve, Belgium )}\\
                          Ch.Brian{\c c}on  \\
{\normalsize\it (Centre de Spectrom{\'e}trie Nucl{\'e}aire et de
Spectrom{\'e}trie de Masse, 91405 Orsay, France)} \\ 

                        M.Kudoyarov, V.Lobanov, A.Pasternak \\
{\normalsize\it (A.F.Ioffe Physical Technical Institute RAS,
St.-Peterburg, Russia)}} \\ [1cm]
\end{center}


\begin{abstract}
We describe an improved follow-up of our previous spin-neutrino
correlation experiment, where the 1229 and 2171~keV gamma-rays emitted 
after the $^{28}$Si($\mu ,\nu $)$^{28}$Al(1$^+$, 2202~keV)
reaction were observed
by high-resolution HP Ge detectors at different angles with respect to
the
muon spin\cite{WEIN95,Brudanin95}. In the experiment described
here, a magnetic field
was used both to select events according to the spin-gamma angle and to
measure the
residual muon polarization by $\mu$SR-method.

In addition of using the new data from the improved setup, we further
increase the
precision of our  result making use of new experimental quantities which
in the previous analysis were considered as unknown  free parameters.
We obtain for 
the parameter  $x \equiv$ M(2)/M(-1) the value of $ x = 0.238 \pm 0.023$ 
(to be compared to $x = 0.254 \pm 0.034$ as reported
in \cite{Brudanin95}). This value can be compared to  $x = 0.315 \pm
0.080$ 
obtained recently  at TRIUMF with different technic \cite{Moftah97}. 
The mean-value  $x = 0.244 \pm 0.022$  can be confronted
to its most up-to-date evaluation as a function of the
coupling-constant ratio $g_P/g_A$, which makes use of renormalized 
transition operators calculated recently in \cite{Siiskonen98}. 
Depending on the effective interaction, one obtains for
$g_P/g_A$ the solution-range of  2.0 - 3.5 to be compared to the 
value of  7  predicted by {\it PCAC} for the free nucleon.
The comparison of our result with the most
recent theoretical evaluations still indicates a  50\%
quenching at least of $g_P/g_A$ as reported in \cite{Brudanin95}.
\end{abstract}
\vspace*{3cm}

\begin{center}
\fbox{
\parbox{145mm}{
{\bf Keywords:}\\
NUCLEAR REACTIONS $^{28}$Si(polarized~$\mu ^-,\nu$);
measured~$E_{\gamma}(\theta )$;
deduced nuclear state alignment; parameter~$g$(P)/$g$(A);$\mu$SR-method;
$^{28}$Al~deduced~levels. Natural~target.
}
}
\end{center}

\end{titlepage}

\rm

\section{  Introduction}
        We discussed in \cite{Brudanin95} the interest and the method to
measure the
induced pseudoscalar coupling in nuclear $\mu$-capture observing various
spin-neutrino correlations by the precision-determination of the
Doppler-pattern of gamma-rays produced in the capture of polarized
muons \cite{Grenacs68}.
We
refer the reader to these discussions and shall list here only the new
developments which made the revisiting we report here both timely and
useful. For
easy reference we recall in Fig.1 the relevant features of the $^{28}$Al
decay-
scheme fed in the $\mu$-capture reaction we investigated.

The novel elements which will be discussed are:
\begin{enumerate}
\item a new experiment was performed installing a magnetic field
perpendicular to the muon-spin and observing the evolution of the gamma
Doppler-shape induced by the spin-precession (see Fig.2).
We considered both the 1229 keV and the 2171 keV gamma lines.
\item the multipolarity mixing-ratio of the 2171 keV gamma-ray was
measured in a dedicated experiment using the 
$^{26}$Mg($^3$He,p) reaction \cite{Kudoyarov96};
\item the negligibility of the background from $\mu$-capture on
the $^{29}$Si-impurity and from cascade feeding was assessed;
\item the specificity of the TRIUMF-experiment \cite{Moftah97} will be
recalled;
\item in the framework of  a model, which describes the slowing-down
of the recoiling nucleus before $\gamma$-emission,
experimentally extracted slowing-down parameters were used
instead of theoretical values;
\item an important new theoretical development took  place
recently, showing that the use of renormalized transition operators
reduces
the quenching of the induced pseudoscalar coupling 
deduced from the experiments \cite{Siiskonen98}.
\end{enumerate}
        Finally we shall terminate giving the up-to-date conclusions
which
result from this study.

\section{The spin-precession experiment}

        Let us recall that the muon spin dependence of the nuclear
recoil was
determined in our first experiment keeping the muon-spin aligned on the
beam 
axis and {\em changing} time-to-time {\em the position} of the
gamma-detector. 
In the present experiment, which is also performed at  the JINR
phasotron, 
the muon-spin dependence is determined applying  a magnetic field
transverse
to the muon-spin  and observes the {\em time-dependence} of the 
Doppler-shape introduced by the muon-spin precession (see Fig.2). This
new 
approach allowed us:
\begin{enumerate}
\item to measure simultaneously the residual polarization of the muon in 
the {\it1s}--state
for the target used instead of relying on
some other material-dependent data;
\item to eliminate the source of systematic errors due to
the displacement of the
detectors e.g. eventual modification of the setup geometry and the
Ge-response function between
the measurements performed at two different positions.
\end{enumerate} 
The experimental set-up used in the present work is shown on Fig.2.
Incoming
muons \\ (125$\pm$6~MeV/c, $\vec{P}\simeq$ --0.7) are detected by
counters 1..3
and stopped in the natural silicon target ($\oslash$65x25~mm).
The energy of the photon and its arrival time relatively
to the $\mu$-stop are measured. The arrival time is also measured for
the
muon decay electrons. Comparing with our previous
apparatus\cite{Brudanin95}
the setup is complemented with a pair of
coils which provide a transverse  magnetic field $B \approx 320 $ Gauss
in the target 
region with inhomogeneity less than 0.1 \%.  
The cascade of Auger and radiative transitions of the muon from the
exited states
to the 
ground state leads to its strong depolarization, but in case
of {\em metallic} silicon, residual polarization of the muon in the
{\it1s}--state is
still significant.  The muon-spin precession is monitored by the
time-evolution
of the muon decay electrons which is detected by telescopes
(5$\cdot$7$\cdot$A) 
and (6$\cdot$8$\cdot$B) during 2~$\mu$s ($\approx$~7 muon spin
precession 
($\mu$SR) periods) 
after the $\mu$-stop. Analysis of the time-differential electron spectra
(see Fig.3) 
allows to extract the precise value  of the $\mu$SR period 
{\it T$_{rot}^i$}, the phase\footnote{With respect to zero time
which is determined by detecting the prompt beam electrons (9\% of
incoming beam
particles) scattered on the
target and on the counter 3.} $\phi_i^0$ and the electron 
asymmetry coefficients A$_i$, presented in the Table~\ref{msrTable}. 

\begin{table}[h]
\begin{center}
\caption{\it The $\mu$SR parameters.} \label{msrTable}
\begin{tabular}{|c|ccc|ccc|} \hline
 \multicolumn{1}{|c}{} &
 \multicolumn{3}{|c|}{Detector A} &
 \multicolumn{3}{c|}{Detector B} \\ \cline{2-7}
%
 \multicolumn{1}{|c|}{Run} &
 \multicolumn{1}{c}{\it T$_{rot.}^A$} (ns)  \hspace{3mm}&
 \multicolumn{1}{c}{$\phi _A^0$ (deg.) \hspace{3mm}} &
 \multicolumn{1}{c|}{A$_A$} &
 \multicolumn{1}{c}{\it T$_{rot.}^B$} (ns) \hspace{3mm} &
 \multicolumn{1}{c}{$\phi _B^0$ (deg.) \hspace{3mm} }  &
 \multicolumn{1}{c|}{A$_B$} \\ \hline
%
  1--6 & $232.9$(4) & $+77$(2)  & $0.052$(1)
       & $232.0$(3) & $-96$(2)  & $0.045$(1) \\ \hline
  7--8 & $232.7$(5) & $+72$(3)  & $0.050$(1)
       & $232.6$(7) & $-94$(3)  & $0.044$(1) \\ \hline
 9--14 & $233.2$(3) & $+75$(2)  & $0.051$(1)
       & $233.0$(4) & $-91$(2)  & $0.046$(1) \\ \hline
\end{tabular}
\end{center}
\end{table}

The value of the residual muon-polarization in the {\it1s}--state is
determined using
the electron asymmetry coefficients\footnote{$A_i = C_i \cdot P \cdot
\frac{1}{3}$,
where factor $C_i$ depends on the setup geometry, the scattering in the
target and the
low-energy threshold.
GEANT setup simulation  is used in order to 
calculate $C_i$. A comparable value of the muon polarization $\vec P$
 is obtained in additional separate 
measurements on a {\em thin tilted} Si-Target, where $C_i \approx 1$.}:
\begin{equation}
  |\vec{P}| = 0.120(2) \; .
\end{equation}
To take into account the energy dependence of the timing and
to correlate the time scales of the gamma- and electron- spectra,
prompt muonic X-rays from a special composite Si-Fe-Zn-Zr-Pb
target are used. As a result a time
resolution of $\Delta t \approx 5$ ns is obtained.

The same telescopes ($\stackrel{-}{5} \cdot \stackrel{-}{7} \cdot $A)
and
($\stackrel{-}{6} \cdot \stackrel{-}{8} \cdot $B) are used to detect
$\gamma$-quanta following ordinary $\mu$-capture in the target.
In order to suppress the bremsstrahlung,
neutron-induced gammas and other external correlated background in 
$\gamma$-spectrum, two selection cuts are applied:
\begin{enumerate}
\item  the $\mu$-stop should not be preceded by another incoming
muons during 2~$\mu$s;
\item no counters 1-8 should be fired after the
$\mu$-stop within the measurement window (also 2~$\mu$s).
\end{enumerate}
In addition to the coincidence spectra, uncorrelated spectra
are also stored. Non-broadened 941~keV, 1173~keV, 1332~keV and 2614~keV
$\gamma$-lines of these uncorrelated spectra are used to extract
precisely
the Ge-response function required
for the analysis of the $\gamma$-line shape.

In order to reduce the long-term drift of the acquisition electronics,
amplifiers and ADCs, are installed into a
thermo-stabilized box ($T=32.0\pm 0.2$~$^{\circ}C$).

\section{Analysis of the results}

All "good" $\gamma$-events are sorted according to the angle $\phi \pm
22.5^o$ between
the rotating axial vector $\vec{P}$ and the momentum $\vec{k}$ of the
$\gamma$-quantum detected by each Ge detector. As a result, we obtain
eight $\gamma$-spectra for each line (see Fig.4) corresponding to
\begin{equation}
\phi = (-135^o, -90^o, -45^o, 0^o, +45^o, +90^o, +135^o, +180^o) 
\end{equation}
The  analysis of data is carried out in a similar way 
to the one described in  Subsection~4.4 of our previous work
\cite{Brudanin95}.
The result is in good agreement with the one obtained in the first
experiment.
Using the multipolarity mixing-ratio $\delta$ still as a free parameter,
we
obtain
\begin{equation}
x = 0.249 \pm 0.027 \; ,
\end{equation}
to be compared to  $x = 0.254 \pm 0.034$.

In the common fit to our previous and present experiment data sets
(see~3.6)
we take into account
all the new information appeared recently. These several new features 
will be discussed below in detail.

\subsection{The multipolarity mixing ratio of the 2171~keV transition}

This ratio, which was left as a free parameter in our original analysis,
is now
determined  measuring the angular distribution of the 2171~keV
$\gamma$-rays
produced in the $^{26}$Mg($^3$He,p)$^{28}$Al reaction at
E$_{He}$=6.9~MeV, using
the
PTI cyclotron (St.-Peterburg) \cite{Kudoyarov96}. For the evaluation of
the alignment, the transition 975~keV of $^{25}$Mg produced in the
($^3$He,$\alpha$) reaction have been used.
As a result a value  $\delta = 0.37 \pm 0.11$ was obtained. 
This mixing-ratio is compatible with the value $\delta = 0.74 \pm 0.29 $
obtained in the fits to our first muon-capture data where this
parameter was left free.

In the fit to our data, the ratio $\delta$ is still used as a variable
considering 
its experimental value $\delta_{exp}$ and error
$\Delta\delta_{exp}$
as an additional data point
\footnote{This is done in
the standard way. 
The corresponding contribution to $\chi^2$: 
$\Delta \chi^2 = \frac{(\delta - \delta_{exp})^2}{\Delta
\delta_{exp}^2}$ 
is added to the $\chi^2$ of our data-points so as
to take into account, in the fit, the known experimental value and error
of $\delta$.}.
The same procedure is applied for the slowing down  parameter $\tau$,
using the experimental value $\tau = 65 \pm 35$~fs~\cite{Endt90}.

\subsection{The slowing-down parameters}
The multiple scattering and slowing-down (SD) of the recoiling $^{28}$Al
nucleus  
in the Si-target leads to a significant attenuation of a Doppler-effect
and narrows
the observed $\gamma$-line. In theoretical calculations this process is
simulated by the theory developed by Lindhard, Scharff and 
Schiott (LSS-theory), which describes the SD of ions in the 
medium~\cite[see also the ref. in~\ref{brud}]{Lindhard63}. The nuclear
spectroscopy applications of the LSS-theory, such as the analysis of the 
Doppler-broadened SD-distorted $\gamma$-lines, were developed 
in~\cite{Lemberg85,Georgieva89}. The theoretical
values of the electron SD coefficient  $f_e$, the nuclear SD
coefficients
$f_n$ and the $\phi_n$ introduced in the LSS-theory were used in the
previous work. 
In principle, this
theoretical values can deviate from the real one up to 5\% for $f_e$ and
up to 30\% for $f_n$ and $\phi_n$\footnote{Fortunately, in our case the
electron
SD mode  prevails under the nuclear SD one.}~\cite{Lemberg85}.
The LSS-parameters $f_e = 1.27(6)$, $f_n = 0.77(7)$,
$\phi_n = 0.90(15)$  are now extracted from experimental data obtained
in the
reaction $^{27}$Al(n,pn$\gamma$)$^{26}$Mg at E$_n$ = 14.9~MeV, where the
SD of $^{26}$Mg ions in the $^{27}$Al were
investigated~\cite{Pasternak92}. 
The obtained values
are in good agreement
with theoretical predictions. As the LSS-parameters depend
only slightly on the nuclear charge Z and the mass number A, 
  we use
% we extrapolate
those experimental values in our evaluation of SD.
 
\subsection{Background from $\mu$-capture on $^{29}$Si and cascade
feeding}
As discussed in \cite{Brudanin95}, the shape of the two gamma-lines of
interest
could be distorted if, in addition to the direct production of the
2001.5~keV 
state in $\mu$-capture on $^{28}$Si, this level would be populated
also from $\mu$-capture
on $^{29}$Si (abundance 4.67\%) accompanied by the emission of a
neutron.

The value of this eventual contribution was tested in an ancillary
$\mu$-capture
experiment at PSI using an enriched $^{29}$Si-target. It was found to be
less than 
10\%, negligible for our purposes. This conclusion was corroborated even
more 
precisely in earlier experiments, where the upper limit of $\leq$ 5\%
was established 
\cite{Miller72}.

Another possible source of line-shape distortion is the possible cascade
feeding 
of 2201.5~keV $^{28}$Al-state from higher levels. The calculations
showed, that
the effect caused by a 5\% cascade feeding from the higher levels
(reported as an upper
bound in \cite{Moftah97}) is negligible in comparison with the
statistical error.

\subsection{The TRIUMF-experiment}
We stressed in our previous article~\cite[see Fig.4]{Brudanin95} 
that, if the 2171~keV line
had a smooth underlying background, this was not the case for the
1229~keV one. The 2171~keV line is about 3.7 times more intense than the
1229~keV one,
but being a transition of mixed multipolarity, its sensitivity to the
parameter $x$
depends also on the mixing ratio $\delta$ discussed above.

In order to exclude the influence of the $\delta$ on the final result,
our colleagues 
at TRIUMF decided to concentrate their attention on the 1229~keV line
and to purify it from background requesting coincidence with the 942~keV
$\gamma$-line
which de-excites the 974.4~keV state (Fig.1 in \cite{Moftah97}). The
relatively
small count rate was counterbalanced by the cleanness of spectrum
obtained. 
The systematic errors can be expected to be complementary to ours.
As mentioned in the introduction, they obtained the value $x = 0.315 \pm
0.080$.

\subsection{New theoretical developments}

A new approach was made by a Finnish
group~\cite{Siiskonen98,Siiskonen98_2}, 
continuing the considerable amount of theoretical works
devoted to the process of interest \cite{Fujii59}-\cite{BKOT94}. Their
main
observation was that the use of renormalized effective operators
diminishes the
quenching of $g_A/g_P$ deduced from the experimental value of the
parameter $x$.
This conclusion holds for configuration-mixing evaluated with different
two-body
forces and 
gives also good agreement 
with the observed partial $\mu$-capture rates
in $^{28}$Si \cite{Siiskonen98_2}. 
The impact on the quenching found in this new theoretical approach
gives new impetus
to the experimental efforts to investigate the long-standing issue of
the quenching
of the induced pseudo-scalar coupling $g_P$ in finite nuclei (see also
references
in \cite{WEIN95,Brudanin95,Moftah97,Siiskonen98}).


\section{Results and discussion} 

Using both our old and new $\mu$-capture data sets and all the mentioned
assumptions
and corrections, we obtain for the parameter $x$ the value:
\begin{equation} \label{xval}
  x = 0.238 \pm 0.023 \;.
\end{equation}

Combining our result (Eq.\ref{xval}) with the one obtained at
TRIUMF~\cite{Moftah97},
we obtain the world-average value:
\begin{equation} \label{xsum}
  x = 0.244 \pm 0.022 \; ,
\end{equation}
which differs only slightly from our result.

The recent evaluation of T.Siiskonen et al. (see Fig.4
in~\cite{Siiskonen98}) allow
us to extract from this world-average our best estimate for the induced
pseudoscalar
coupling in $^{28}$Si. After due renormalization of the transition
operators, and
using for the determination of the involved configuration-mixing the
USD- or CD-Born
interactions we obtain:
\begin{equation} \label{gpga}
   2. < g_P/g_A < 3.5
\end{equation}
As the unquenched value of this ratio is expected to be around 7, the 
world-average of the $^{28}$Si correlation data indicates
a quenching of at least 50\% compared to the free-nucleon value.

Although the recent theoretical calculation~\cite{Siiskonen98} allows
to raise the experimental $g_P/g_A$ value above the catastrophic
negative
one, the quenching
of $g_P/g_A$ still remains considerable. Further improvements of
the experimental method and in the accuracy of a measurement can be
reached by two ways:
\begin{enumerate}
\item a more accurate search for as yet unknown background lines
under the investigated 1229~keV one, which can 
distort the $\gamma$-line shape.
This seems to be the last possible source of a
systematic errors, which was not totally eliminated;
\item very precise measurement of lifetime $\tau$ of 2201.5~keV level in
$^{28}$Al,
which is strongly correlated with the parameter $x$. The accuracy of the
present value $\tau = 65 \pm 35$~fs is insufficient 
to constrain significantly our results.
The knowledge of a more precise $\tau$ value would allow us i) to
decrease
significantly the errors for the parameter $x$ and 
ii) to fit separately the 1229~keV and 2171~keV lines in order to
compare the two
results for $x$ and thus to test the validity of our background
assumptions.
This measurement can be carried
out on the Si-containing compound with very small density (so called
``aerogel'' 
with density $\approx$ 0.2~g/cm$^3$)
for which the influence of slowing-down processes
is greatly reduced.
%This measurement could improve
%moreover the evaluation of cascade feeding of 2201.5 $^{28}Al$-state
from the 
%higher levels.
\end{enumerate}

Similar measurements on other target nuclei would also be needed
for a further clarification of the situation. The most appropriate
candidate is  $^{20}$Ne, the $\mu$-capture on this nucleus
leads to the {\bf 1$^+$} ($\tau = 7.4 \pm 1.6$~fs) 1056.85~keV state of 
$^{20}$F de-exciting to the ground state by a pure M1 transition. 
The investigation of this new candidate  will allow us to extend the 
test to a new case.

\section{Acknowledgments}
We would like to thank Dr.J.Rak, V.Vorobel, the staff of  JINR phasotron
and
Beam group for their support during the experiment, and Dr.V.Zinov for
useful remarks.
We are indebted to A.Kachalkin for the setup preparation and for its
 maintenance, 
Dr.V.Sandukovsky, A.Revenko for the HP Ge preparation.

\begin{thebibliography}{99}
\bibitem{WEIN95}
Ch.Briancon et al.,
Abstracts of WEIN'95 - The IV Int.Symp on Weak and Electromagnetic
Interactions in Nuclei, June 12-16, 1995, Osaka, Japan, p.185. \\
WEIN'95 - Yamada Conf. XLIV, Proc. of IV Int.Symp on Weak and
Electromagnetic Interactions in Nuclei, June 12-16, 1995, Osaka,
Japan, World Scientific 1995, p.390.
\bibitem{Brudanin95} \label{brud}
  V.Brudanin {\it et al.}, Nucl. Phys. {\bf A587} (1995) 577.

\bibitem{Moftah97}
B.A.Moftah {\it et al.}, Phys.Lett. {\bf B398} (1997) 157.

\bibitem{Siiskonen98}
T.Siiskonen {\it et al.}, nucl-th/9806052 18 Jun 1998 and to be
published.

\bibitem{Grenacs68}
L. Grenacs {\it el al.}, Nucl. Intr. and Meth. {\bf 58} (1968) 164.

\bibitem{Kudoyarov96}
M.F.Kudoyarov {\it et al.}, Proc.46th
Ann.Conf.Nucl.Spectrosc.Struct.At.Nuclei, 
Moscow, (1996) 235 and to be published.
\bibitem{Lindhard63}
 J.Lindhard, M.Scharff, H.E.Schiott, Dansk. Vid. Selsk. Mat.-Fys. Medd.
    {\bf 33} (1963) 3.
\bibitem{Lemberg85}
 I.Kh.Lemberg, A.A.Pasternak,  Sovremennye metody yadernoi
    spektroskopii, Leningrad, "Nauka", 1985, 3.
\bibitem{Georgieva89}
 M.K.Georgieva {\it et al.}, Elem. Chast. i At. Yadra, {\bf 20} (1989)
9307.
\bibitem{Endt90}
P.M.Endt, Nucl. Phys. {\bf A521} (1990) 1.

\bibitem{Pasternak92}
A.A.Pasternak, D.Khongiu Bull.Rus.Acad.Sci.Phys. 56 (1992) 1811.\\
Izv.of RAS, Phys. ser., v.56, n.11 (1992) 179.

\bibitem{Miller72}
G.H.Miller {\it et al.} Phys.Rev.Lett. {\bf 29} (1972) 1194\\
G.H.Miller Ph.D. thesis, College of William and Mary (1972).
\bibitem{Siiskonen98_2}
Siiskonen {\it et al.}, Nucl. Phys. A635(1998) 446.
\bibitem{Fujii59}
 A.Fujii and H.Primakoff, Nuo. Cim. {\bf 12} (1959) 327.
\bibitem{Ciechanowicz76}
 S.Ciechanowicz, Nucl. Phys. {\bf A267} (1976) 472.
\bibitem{Parthasarathy78}
 R.Parthasarathy and V.N.Sridhar, Phys. Rev. {\bf C18} (1978) 1796.
\bibitem{Parthasarathy81}
 R.Parthasarathy and V.N.Sridhar, Phys. Rev. {\bf C23} (1981) 861.
\bibitem{BKOT94}
 E.Boschitz et al., {\it Communication of JINR,} Dubna, {\bf P4-94-427}
(1994).
\end{thebibliography}

\section{Figure captions}

{\bf Figure.1} The relevant part of the de-excitation scheme of
$^{28}$Al, 
produced in  $\mu$-capture by $^{28}$Si.

{\bf Figure 2.} Experimental setup of the spin-precession experiment.

{\bf Figure 3.} Measured $\mu$SR-curves from $\mu$-capture by $^{28}$Si
.


{\bf Figure 4.} Measured and expected $\gamma$-line shapes.


\end{document}


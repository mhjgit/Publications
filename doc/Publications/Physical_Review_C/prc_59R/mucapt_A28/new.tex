\documentstyle[aps,multicol]{revtex}
\draft

\begin{document}

\title{Towards the solution of the $C_{\rm P}/C_{\rm A}$ anomaly in
shell-model calculations of muon capture}

\author{T.\ Siiskonen, J.\ Suhonen}

\address{Department of Physics, University of Jyv\"{a}skyl\"{a},
     P.O.B.\  35, FIN-40351 Jyv\"{a}skyl\"{a}, Finland}

\author{M.\ Hjorth-Jensen}

\address{Department of Physics, University of Oslo, N-0316 Oslo, Norway}

\maketitle

\begin{abstract}
Recently many authors have performed shell-model calculations of
nuclear matrix elements
determining the rates of the ordinary muon
capture in light nuclei.
These calculations have employed well-tested
effective interactions
in large scale shell-model studies.
For one of the nuclei of interest, namely
$^{28}$Si, there exists recent experimental data which can be used to
deduce the value of the ratio $C_{\rm P}/C_{\rm A}$ by using the calculated
matrix elements. Surprisingly enough, all the shell-model results
suggest a very small value ($\simeq 0$) for $C_{\rm P}/C_{\rm A}$, quite
far from the PCAC prediction and recent data on muon capture in
hydrogen. We show that this rather
disturbing anomaly is solved by employing
effective transition operators. This finding
is also very important for realistic muon-capture calculations
where one explores the strength of the scalar coupling in the weak
charged current of leptons and hadrons.
\end{abstract}

\pacs{PACS numbers: 23.40.Bw, 23.40.Hc, 21.60.Cs}

\begin{multicols}{2}

The calculation of the nuclear matrix elements involved in the
ordinary (non-radiative) capture of stopped negative muons by atomic nuclei has
been of considerable interest because they enable to access the
structure of the effective weak baryonic current. Due to the large mass
of the captured muon the process involves a large energy release
(roughly 100 MeV) and thus surveys the baryonic current deeper than the
ordinary beta decay or the electron capture.
In particular, the role of the induced pseudoscalar coupling $C_{\mathrm P}$
becomes prominent. Furthermore, the same nuclear matrix elements can be
used in the context of studies of the fundamental structure of the weak
charged current of leptons and hadrons, in particular concerning
contributions coming from the yet undetected scalar coupling of the current.

In the past there have been many calculations of nuclear matrix
elements involved in the muon-capture processes. These calculations
have been either very schematic ones \cite{mor,GIL65,PAR78} or more
realistic ones using various truncations of the nuclear shell model
\cite{GIL65,PAR78,ERI64}. Ultimately, all these
calculations have aimed at predicting the ratio $C_{\rm P}/C_{\rm A}$
of the induced pseudoscalar and axial-vector
coupling strengths of the weak baryonic current by exploiting the
scarce experimental information
on muon-capture rates in light nuclei.
These calculations seem to suggest wide ranges of values as can be seen from
Table \ref{values}.
For reference, we also give the nuclear-model independent Goldberger-Treiman
value $C_{\rm P}/C_{\rm A}=6.8$,
which is obtained using the partially conserved axial current hypothesis
(PCAC). It should be
reasonable to assume that this relation between the induced pseudoscalar
and axial-vector coupling constants will survive in finite nuclei
although corrections may occur e.g., due to mesonic corrections in the weak
vertices. In particular, this result should be roughly recovered using
nuclear-structure calculations assuming the impulse approximation.
Renormalizations within the impulse approximation have been extensively
discussed for the $C_{\rm A}$ coefficient in the context of beta decay
and electron capture.

Recently, also the nuclear shell model has been used to
calculate the needed nuclear matrix elements for muon capture
\cite{GMI90,KUZ94,JOH96,MOF97,sii}.
In \cite{GMI90} and \cite{KUZ94} no definitive conclusions about
the $C_{\rm P}/C_{\rm A}$ ratio could be reached
on the basis of their computed matrix elements.
In \cite{JOH96} and \cite{sii} the $C_{\rm P}/C_{\rm A}$ ranges shown in
Table \ref{values}
were extracted from the available experimental muon capture rates.

Very recently two important measurements of correlation coefficients of
$\gamma$-radiation anisotropy in the capture of a polarized negative muon
have been reported \cite{MOF97,bru}. For the allowed muon capture the
angular correlation between the emitted $\gamma$-radiation and the neutrino
is scaled by the coefficient $\alpha$ \cite{PAR78,MOF97} which is related
to the coefficient
        \begin{equation}
       x\equiv M_1(2)/M_1(-1)
       \label{xdef}
        \end{equation}
of \cite{bru} by
        \begin{equation}
          \alpha = {\sqrt{2}x - x^2/2 \over 1+x^2}\ .
          \label{alphax}
        \end{equation}
Here the quantities $M_1(2)$ and $M_1(-1)$ are given by
        \begin{eqnarray}
        M_1(-1)&=&\sqrt{2\over3}\left\{\left({1\over3}G_{\rm P}-G_{\rm A}
                \right)[101]+G_{\rm P}{\sqrt2\over3}[121]\right.\cr
                &-& \left.{C_{\rm A}
                \over M}[011p]+{C_{\rm V}\over M}\sqrt{2\over3}[111p]\right\},\\
        M_1(2)&=&\sqrt{2\over3}\left\{\left(G_{\rm A}-{2\over3}G_{\rm P}\right)
                [121]-G_{\rm P}{\sqrt2\over3}[101]\right.\cr &+&
                \left.{C_{\rm A}\over M}\sqrt2
                [011p]+{C_{\rm V}\over M}\sqrt{2\over3}[111p]\right\},
        \end{eqnarray}
where $M$ is the nucleon mass. The definitions of the involved reduced nuclear
matrix elements are given in Table \ref{operators}. The constants $G_{\rm P}$
and $G_{\rm A}$ are defined as (with $C_{\rm T}=0$)
        \begin{eqnarray}
        G_{\rm P}&=&(C_{\rm P}-C_{\rm A}-C_{\rm V}-C_{\rm M}){E_\nu\over2M},\\
        G_{\rm A}&=&C_{\rm A}-(C_{\rm V}+C_{\rm M}){E_\nu\over2M}.
        \end{eqnarray}

Using the
expressions of $M_1(2)$ and $M_1(-1)$ with Eqs.\ (\ref{xdef}) and (\ref{alphax})
leads to rather contradictory results
between different realistic nuclear models. In \cite{bru} the following values
of $C_{\rm P}/C_{\rm A}$ were extracted using the different realistic
nuclear models: $C_{\rm P}/C_{\rm A} = 3.4\pm 1.0$ and
$C_{\rm P}/C_{\rm A} = 2.0\pm 1.6$ for the matrix elements of \cite{CIE76}
and \cite{PAR81}, respectively. In the measurement of \cite{MOF97}
the corresponding values are $C_{\rm P}/C_{\rm A} = 5.3\pm 2.0$ \cite{CIE76}
and $C_{\rm P}/C_{\rm A} = 4.2\pm 2.5$ \cite{PAR81}. These results would
indicate a small quenching of the $C_{\rm P}/C_{\rm A}$ ratio with respect
to the PCAC value. However, using the more realistic nuclear matrix elements
obtained from the full $1s0d$ shell-model calculation using
the $sd$-shell effective interaction (USD)
of Wildenthal {\em et al.} \cite{wil}, would lead to the
very contradictory result of $C_{\rm P}/C_{\rm A} = 0.0\pm 3.2$ \cite{MOF97}
by using the computed matrix elements of Junker et {\it al.} \cite{JUN95}.
The same situation occurs for the calculation of \cite{sii} verifying
thereby  the
calculation of \cite{JUN95}. This result is the more surprising considering
the fact that the USD interaction is found to be very succesful
for the $1s0d$ nuclei in reproducing various spectroscopic quantities like
energy spectra, Gamow--Teller decay properties, electromagnetic moments and
transitions (see e.g., \cite{car,bro}) as well as strength functions of
charge-exchange reactions \cite{JOH96}.

The above anomaly in the $C_{\rm P}/C_{\rm A}$ predictions from
state-of-the-art shell-model calculations is rather disturbing when
contrasted with the experimental data. To give a deeper insight, we investigate
in the present Letter the capture reaction
${}^{28}_{14}{\mathrm Si}(0^+_{\mathrm gs})+\mu^-
\rightarrow{}^{28}_{13}{\mathrm Al}(1^+_3)+\nu_\mu$ within
the full shell-model framework and try to evaluate
the ratio $C_{\rm P}/C_{\rm A}$ through the quantity $x$ and its measured
values reviewed above. The needed muon-capture formalism is developed
in \cite{mor} and reviewed in the case
of shell-model calculations in \cite{KUZ94,sii}. The two-body interaction
matrix elements used in the shell-model calculation are given by the
USD interaction \cite{wil} and a microscopic effective
interaction based on the recent charge dependent nucleon-nucleon
interaction of Machleidt and co-workers,
the CD-Bonn interaction \cite{mac96}. This is a meson-exchange
potential model. Based on this nucleon-nucleon
interaction, we derive a $G$-matrix appropriate for
the $1s0d$ shell. This $G$-matrix is in turn used in a
perturbative summation of higher-order terms using the
so-called $\hat{Q}$-box approach described in e.g., Ref.\ \cite{hko95}.
All diagrams through third-order in perturbation
theory were used to define the $\hat{Q}$-box, while folded
diagrams were summed to infinite order, see Ref.\  \cite{hko95}
for further details. Below we will label results obtained with this
effective interaction by CD-Bonn. We have also performed similar
calculations with the Nijmegen I \cite{nim94}
potential. The results were very close to those obtained with the CD-Bonn
potential and hence skipped in the discussion below.
These effective interactions
are defined within the
$1s0d$ shell, using $^{16}$O as closed shell core,
and we perform a full shell-model calculation
using the code OXBASH \cite{oxb}.

It is important to keep in mind that the USD
interaction is fitted to reproduce several properties of
$1s0d$ shell nuclei, whereas the effective interaction based on the
CD-Bonn meson-exchange potential model starts from the
bare nucleon-nucleon interaction. The effects of the
nuclear medium are then introduced through various terms
in the many-body expansion. However, microscopic effective interactions
like the above derived from the CD-Bonn and Nijmegen I interactions
typically fail in reproducing properties of nuclei like $^{28}$Si
or e.g., shell-closure in $^{48}$Ca \cite{hko95}.
Therefore, Zuker and co-workers \cite{zuk94,zuk98} have adopted a slightly
different scheme where the effective interaction is
rewritten in terms of a multipole expansion \cite{zuk94}.
The monopole part, which is proportional to the energy centroids,
can then, see e.g., Ref.\
\cite{zuk98}, be extracted from the spectra of particle
and hole states of doubly magic cores. We have thus employed this
prescription to our CD-Bonn and Nijmegen I effective interactions
correcting thereby our diagonal terms with the phenomenological
ones derived by Duflo and Zuker \cite{zuk98}. In addition, single-particle
energies appropriate for $^{28}$Si as  
predicted from the approach of Ref.\ \cite{zuk98} are 
employed in the calculations with the CD-Bonn and Nijmegen interactions.
These energies are $\varepsilon_{0d3/2}-\varepsilon_{0d5/2}=5.66$ MeV and
$\varepsilon_{1s1/2}-\varepsilon_{0d5/2}=0.68$ MeV. For the USD interaction
the experimental energies are used.
As discussed below, the modification of the effective interactions according
to Ref.\ \cite{zuk98} 
improves considerably the spectroscopy of $^{28}$Si and $^{28}$Al
compared with the unmodified interaction.

The single-particle
matrix elements were evaluated in the harmonic-oscillator basis
by numerical integration of the radial
part containing overlap of the initial and final harmonic-oscillator
wave functions and the muonic $s$-state wave function.
A harmonic oscillator basis was also employed in our calculation
of the $G$-matrix which enters the computation of the effective
interaction using an oscillator parameter of $1.72$ fm.

In addition to using the
traditional bare transition operators in the evaluation of the nuclear
matrix elements for muon capture we will also employ
effective transition operators. Such a calculation
will be referred to as the renormalized one in the discussion
below. To obtain effective one-body
transition operators for muon capture, we evaluate all
effective operator diagrams through second-order in the
$G$-matrix obtained with  the CD-Bonn and Nijmegen I interactions,
including folded diagrams. Such diagrams
are discussed in the reviews by Towner \cite{towner87}
and Ellis and Osnes \cite{eo77}. The abovementioned $G$-matrix
obtained with the two potentials has been used in the evaluation
of the various diagrams.
Terms arising from meson-exchange currents have
been neglected, similarly, also the possibility
of having isobars $\Delta$ as intermediate states are omitted
since the focus here is  on nucleonic degrees
of freedom only. Moreover, the nucleon-nucleon potentials
we are employing do already contain such intermediate states.
Including $\Delta$ degrees freedom may thus lead to a possible
double-counting.
Intermediate state excitations in each diagram
up to $6\hbar\omega$ in oscillator energy were included
in order to achieve a converged result. This is also in line
with studies of effective interactions with weak tensor
forces \cite{sommerman},
such as the CD-Bonn potential employed here.

As such, the determination of both the microscopic
shell-model effective interaction and the effective
one-body operators are calculated at the same level of many-body
approach. Further details will be presented elsewhere \cite{ssh98}.
In general, higher order terms introduce a correction of the
order of $30\%$ compared with the bare transition operator.
Since the USD interaction is an effective one acting
within the $1s0d$ shell only, it is not possible to calculate
a corresponding effective operator employing perturbative
many-body methods. We will therefore employ the effective operators obtained
from the CD-Bonn interaction for the USD calculation as well.

Our results are summarized in Figs.\ \ref{nme}, and \ref{x} with
$C_{\mathrm A}/C_{\mathrm V}=-1.251$. The coefficient $x$ is not sensitive
to changes in the ratio $C_{\rm A}/C_{\rm V}$. Using the renormalized value
$-1.0$ for $C_{\mathrm A}/C_{\mathrm V}$ changes $x$ only by a few percent for
$C_{\rm P}/C_{\rm A}=7$. In fact, using the renormalized axial vector coupling
may lead to double counting as the renormalization effects are already included
in the renormalized single-particle matrix elements.

Inspection of the nuclear matrix elements for muon
capture of Fig.\ \ref{nme} shows
that the USD and the CD-Bonn predictions are very similar to each other. The
same is true also for the Gamow--Teller matrix elements $B({\rm GT})$, which are
within 25\% from each other for the $1^+_3$ final state in ${}^{28}$Al.
The effects of renormalization are some 10\% for the Gamow--Teller type
matrix element $[101]$ and roughly 30\% for $[011p]$,  $[111p]$ and $[121]$
matrix elements. In all cases the magnitude of the matrix element reduces
when compared to the bare interaction.
The differences are reflected also in the capture
rates where we also get reduction.
However, in the capture rate formula \cite{mor} we have products of the reduced nuclear
matrix elements and thus the capture rate is very sensitive to changes in the
magnitudes of the involved matrix elements. In the expression for $x$ (see
Eq.\ (\ref{xdef})) we have only ratios of them: dividing
$M_1(-1)$ and $M_1(2)$ by $[101]$,
we get ratios of the matrix elements, where the renormalization effects
give roughly a 30\% reduction for all of them.
Thus $x$ might be less sensitive to systematical errors of the involved matrix
elements than the capture rate.

The leading configurations of the involved states in $^{28}$Si and $^{28}$Al
are given in Table \ref{occ}.
The calculated spectrum of $^{28}$Si is in good agreement with experiment for
both interactions. The spectrum of $^{28}$Al is well reproduced by both
interactions (see Fig.\ \ref{spec}).

In spite of the different structures of the initial and final states of muon
capture, emerging from the use of the two interactions, the renormalization
of the muon capture matrix elements changes the value of $x$ to the same
direction in both cases. Thus the renormalization effects on the value of
$x$ seem to be rather interaction independent.

To extract an estimate for the ratio $C_{\rm P}/C_{\rm A}$, we have
plotted $x$ of Eq.\ (\ref{xdef}) as a function of $C_{\rm P}/C_{\rm A}$. The
experimental value $x=0.315\pm0.08$ was taken
from \cite{MOF97}. With this choice  we obtain from Fig.\thinspace \ref{x}
the range $0.4\le C_{\rm P}/C_{\rm A}\le 2.7$
for the bare USD and CD-Bonn calculations
in agreement with the USD result of \cite{JUN95} cited in \cite{MOF97}. Thus,
for both of the adopted interactions the bare result is almost the
same in spite of the basic difference in the origin
of the used interactions, hinting to a strong
suppression of the $C_{\rm P}/C_{\rm A}$ ratio for the studied
muon-capture transition in the framework of the nuclear shell-model.
As discussed before, this contradicts the
shell-model calculation of the partial capture rates in
$^{23}$Na (with a fitted $C_{\rm P}/C_{\rm A}$ ratio)
as well as experimental data on hydrogen.

Also the USD and CD-Bonn calculations with renormalized transition
probabilities agree almost exactly and both yield a very different value for
$C_{\rm P}/C_{\rm A}$ than the calculation using bare operators.
Dividing $M_1(-1)$ and $M_1(2)$ by $[101]$, the ratios of the reduced matrix
elements change roughly
by the same amount, so the most important factors are the ones in front of
the matrix element $[101]$.
The renormalized result is $3.4\le C_{\rm P}/C_{\rm A}\le 5.0$,
closer to the PCAC value.
This result
encourages us to believe that the anomaly in the $C_{\rm P}/C_{\rm A}$ ratio,
inherent in the sophisticated shell-model calculations, has been lifted
by introducing effective renormalized transitions operators acting in
the muon-capture process. In this way the renormalized shell-model
calculations yield a value for $C_{\rm P}/C_{\rm A}$ in $^{28}$Si
compatible with data and expectations coming from other theoretical
approaches.

In conclusion, it is found that the renormalization of the transition
operators is essential in the ordinary muon capture
even in the case of an empirical effective interaction, which otherwise
reproduces the spectroscopy of the involved nuclei extremely well. This is
confirmed by the anisotropy data
through the quantity $x=M_1(2)/M_1(-1)$. The effects of the renormalization
seem to be rather interaction independent. Thus the renormalization has helped
to reduce the $C_{\rm P}/C_{\rm A}$ anomaly inherent in the shell-model
calculations leading to $C_{\rm P}/C_{\rm A}$ ratios closer to data coming
from measurements of muon capture in hydrogen.
Further analysis of this observation is in progress for other light
nuclei in terms of the capture rates since anisotropy data is only
available for $^{28}$Si. For other nuclei of interest the
theoretical situation is more complicated than is the case in
$^{28}$Si since they are either situated at the interface of the $0p$ and
$1s0d$ shells or $1s0d$ and $1p0f$ shells, complicating thereby the evaluation
of an effective interaction and increasing the dimensionality
of the shell-model calculation. Research along such lines is
in progress \cite{ssh98}.\newline

We are much indebted to Prof.\
A.\ P.\ Zuker for explaining the use of a recently derived
correction scheme for effective interactions.
J.\ S.\ thanks the Academy of
Finland for financial support.
This research has also been supported by
the Nordic Academy of Advanced Studies (NorFA).

\begin{thebibliography}{100}
\bibitem{mor} M.\ Morita and A.\ Fujii, Phys.\ Rev.\ {\bf 118}, 606 (1960).
\bibitem{GIL65} V.\ Gillet and D.\ Jenkins, Phys.\ Rev.\ {\bf 140}, B32 (1965).
\bibitem{PAR78} R.\ Parthasarathy and V.\ N.\ Sridhar, Phys.\ Rev.\
        C {\bf 18}, 1796 (1978).
\bibitem{ERI64} T.\ Ericson, J.\ C.\ Sens, and H.\ P.\ C.\ Rood,
        Nuovo Cim. {\bf 24}, 51 (1964).
\bibitem{bar} G.\ Bardin, J.\ Duclos, A.\ Magnon, J.\ Martino, A.\ Richter,
        E.\ Zavattini, A.\ Bertin, M.\ Piccinini, and A.\ Vitale, Phys.\ Lett.\
        B {\bf 104}, 320 (1981).
\bibitem{jon} G.\ Jonkmans {\it et al.}, Phys.\ Rev.\ Lett.\ {\bf 77}, 4512
        (1996).
\bibitem{GMI90} M.\ Gmitro, S.\ S.\ Kamalov, F.\ \v{S}imkovic, and A.\ A.\
        Ovchinnikova, Nucl.\ Phys.\ {\bf A507}, 707 (1990).
\bibitem{KUZ94} V.\ A.\ Kuz'min, A.\ A.\ Ovchinnikova and T.\ V.\ Tetereva,
        Physics of Atomic Nuclei {\bf 57}, 1881 (1994).
\bibitem{JOH96} B.\ L.\ Johnson, T.\ P.\ Gorringe, D.\ S.\ Armstrong,
        J.\ Bauer, M.\ D.\ Hasinoff, M.\ A.\ Kovash, D.\ F.\ Measday,
        B.\ A.\ Moftah, R.\ Porter, and D.\ H.\ Wright, Phys.\ Rev.\ C {\bf 54},
        2714 (1996).
\bibitem{MOF97} B.\ A.\ Moftah, E.\ Gete, D.\ F.\ Measday, D.\ S.\ Armstrong,
        J.\ Bauer, T.\ P.\ Gorringe, B.\ L.\ Johnson, B.\ Siebels, and
        S.\ Stanislaus, Phys.\ Lett.\ B {\bf 395}, 157 (1997).
\bibitem{sii} T.\ Siiskonen, J.\ Suhonen, V.\ A.\ Kuz'min, and T.\ V.\ Tetereva,
        Nucl.\ Phys.\ {\bf A635}, 446 (1998).
\bibitem{bru} V.\ Brudanin {\it et al.}, Nucl.\ Phys.\ {\bf A587}, 577 (1995).
%\bibitem{FUJ59} A.\ Fujii and H.\ Primakoff, Nuovo Cim.\ {\bf 12}, 327 (1959).
\bibitem{CIE76} S.\ Ciechanowicz, Nucl. Phys.\ {\bf A267}, 472 (1976).
\bibitem{PAR81} R.\ Parthasarathy and V.\ N.\ Sridhar, Phys.\ Rev.\
        C {\bf 23}, 861 (1981).
\bibitem{wil} B.\ H.\ Wildenthal, Prog.\ Part.\ Nucl.\ Phys.\ {\bf 11}, 5 (1984).
\bibitem{JUN95} K.\ Junker, V.\ A.\ Kuz'min, A.\ A.\ Ovichinnikova, and
        T.\ V.\ Tetereva, Proc.\ IV Int.\ Symp.\ on Weak and
        Electromagnetic Interactions in Nuclei (WEIN'95), Osaka, Japan, 1995,
        eds. H.\ Ejiri, T.\ Kishimoto, and T.\ Sato (World Scientific, Singapore, 1996)
        p. 394.
\bibitem{car} M.\ Carchidi, B.\ H.\ Wildenthal, and B.\ A.\ Brown, Phys.\ Rev.\
        C {\bf 34}, 2280 (1986).
\bibitem{bro} B.\ A.\ Brown and B.\ H.\ Wildenthal, Nucl.\ Phys.\ {\bf A474}, 290
        (1987).
\bibitem{mac96} R.\ Machleidt, F.\ Sammarruca, and Y.\ Song,
        Phys.\ Rev.\ C {\bf 53}, R1483 (1996).
\bibitem{hko95} M.\ Hjorth-Jensen, T.\ T.\ S.\ Kuo, and E.\ Osnes,
        Phys.\ Rep.\ {\bf 261}, 125 (1995).
\bibitem{nim94} V.\ G.\ J.\ Stoks, R.\ A.\ M.\ Klomp,
        C.\ P.\ F.\ Terheggen, and J.\ J.\
        de Swart, Phys.\ Rev.\ C {\bf 49},   2950 (1994).
\bibitem{oxb} B.\ A.\ Brown, A.\ Etchegoyen, and W.\ D.\ M.\ Rae,
        The computer code OXBASH, MSU-NSCL report 524 (1988).
\bibitem{zuk94} A.\ P.\ Zuker, Nucl.\ Phys.\ {\bf A576}, 65 (1994).
\bibitem{zuk98} J.\ Duflo and A.\ P.\ Zuker, submitted to Phys.\ Rev.\ Lett.
\bibitem{towner87} I.\ S.\ Towner, Phys.\ Rep.\ {\bf 155}, 263 (1987); B.\ Castel
        and I.\ S.\ Towner, {\em Modern Theories
        of Nuclear Moments}, (Clarendon Press, Oxford, 1990) pp.\ 55.
\bibitem{eo77} P.\ J.\ Ellis and E.\ Osnes, Rev.\ Mod.\ Phys.\
        {\bf 49}, 777 (1977).
\bibitem{sommerman}H.\ M.\ Sommerman, H.\ M\"uther, K.\ C.\
Tam, T.\ T.\ S.\ Kuo, and A.\ Faessler, Phys.\ Rev.\
C {\bf 23}, 1765 (1981).
\bibitem{ssh98} T.\ Siiskonen, J.\ Suhonen, and M.\ Hjorth-Jensen,
        (unpublished).
\bibitem{lede} C.\ Lederer and V.\ S.\ Shirley (Editors) {\em Table of Isotopes},
        7th edition (Wiley, New York, 1978).
\end{thebibliography}

\end{multicols}

\begin{figure}
        \caption{The values of all the relevant
         reduced nuclear matrix
        elements for muon capture computed using the USD and CD-Bonn
        interactions with and
        without renormalization of the involved transition operators.
        The recoil matrix elements $[\dots p]$ are
        in units of fm$^{-1}$.}
        \label{nme}
\end{figure}

\begin{figure}
        \caption{The ratio $x=M(2)/M(-1)$ as a function of the ratio
        $C_{\mathrm P}/C_{\mathrm A}$. The experimental
        value \protect\cite{MOF97} with the error
        limits is indicated by the horizontal lines.}
        \label{x}
\end{figure}

\begin{figure}
        \caption{Calculated and experimental \protect\cite{lede} spectrum of
        $^{28}$Al.}
        \label{spec}
\end{figure}

%%%%%%%%%%%%%%%%%%%%%%%%%%%%%%%%%%%%%%%%%%%%%%%%%%%%%%%%%%%%%%%%%%%%%%%%%%

\begin{table}
\caption{Experimental data and earlier estimates exploiting the measured
        muon-capture rates for the ratio $C_{\mathrm P}/C_{\mathrm A}$.}
\begin{tabular}{lll}
        Ref.\ & Value & Type of extraction \\
        \hline
        \cite{mor} & $4\le C_{\mathrm P}/C_{\mathrm A}\le 39$ & Schematic \\
        \cite{GIL65,PAR78} & $3\le C_{\mathrm P}/C_{\mathrm A}\le 20$ &
                Schematic, realistic \\
        \cite{ERI64} & $13\le C_{\mathrm P}/C_{\mathrm A}$ & Realistic \\
        \cite{bar} & $6.8\le C_{\mathrm P}/C_{\mathrm A}\le 10.6$ & Exp. \\
        \cite{jon} & $8.8\le C_{\mathrm P}/C_{\mathrm A}\le 10.8$ & Exp. \\
        \cite{JOH96} & $4.1\le C_{\mathrm P}/C_{\mathrm A}\le8.9$ & Shell-model \\
        \cite{sii} & $-3.0\le C_{\mathrm P}/C_{\mathrm A}\le2.5$ & Shell-model \\
         & $6.8$ & Goldberger--Treiman relation \\
\end{tabular}
\label{values}
\end{table}

\begin{table}
\caption{Definition of the reduced nuclear matrix elements
$\int U_{J_fM_f}^\dagger\sum_{s=1}^A{\rm e}^{-\alpha Zm_\mu'r_s}\Psi_s
\tau_-^sU_{J_iM_i}d{\bf r}_1\cdots d{\bf r}_A={\cal M}[k\,w\,u\,(
{\pm\atop p})](J_i\,M_i\,u\,M_f-M_i|J_f\,M_f)$. Functions $\cal Y$ are given by
        ${\cal Y}^{\mu'-\mu}_{0\nu u}(\Omega)={1\over\sqrt{4\pi}}Y_{\nu,
        \mu'-\mu}(\Omega)$, ${\cal Y}^{\mu'-\mu}_{1\nu u}(\Omega,\sigma)=
        \sum_m(1\ -m\ \nu\ m-\mu+\mu'|u\ \mu'-\mu)Y_{\nu,m+\mu'-\mu}(\Omega)
        {\cal Y}_{1,-m}(\sigma)$,
where ${\cal Y}_{1m}(\sigma)=\sqrt{3\over4\pi}\sigma_m$.
The functions $Y$ are the spherical harmonics, and $j_w(qr_s)$
are spherical Bessel functions.}
\begin{tabular}{ll}
        Matrix element & $\Psi_s$\cr
        \hline
        $[0wu]$ & $j_w(qr_s){\cal Y}_{0wu}^{M_f-M_i}(\hat r_s)\delta_{wu}$\cr
        $[1wu]$ & $j_w(qr_s){\cal Y}_{1wu}^{M_f-M_i}(\hat r_s,\sigma_s)$\cr
        $[0wu\pm]$ & $\left[j_w(qr_s)\pm\alpha Z(m_\mu'/p_\nu
           )j_{w\mp1}(qr_s)\right]{\cal Y}_{0wu}^{M_f-M_i}
           (\hat r_s)\delta_{wu}$\cr
        $[1wu\pm]$ & $\left[j_w(qr_s)\pm\alpha Z(m_\mu'/p_\nu
           )j_{w\mp1}(qr_s)\right]{\cal Y}_{1wu}^{M_f-M_i}
           (\hat r_s,\sigma_s)$\cr
        $[0wup]$ & $ij_w(qr_s){\cal Y}_{0wu}^{M_f-M_i}(\hat
           r_s)\sigma_s\cdot{\bf p}_s\delta_{wu}$\cr
        $[1wup]$ & $ij_w(qr_s){\cal Y}_{1wu}^{M_f-M_i}(\hat r_s,{\bf p}_s)$\cr
\end{tabular}
\label{operators}
\end{table}

\begin{table}
\caption{The five largest configurations of the indicated states. The partitions
are given by $[n_1({\rm d}_{5/2})\ n_2({\rm s}_{1/2})\ n_3({\rm d}_{3/2})]$,
where $n_1+n_2+n_3=12$.}
\begin{tabular}{crcr}
\multicolumn{4}{c}{$^{28}$Si ($0^+_{\rm g.s.}$)} \\
\hline
\multicolumn{2}{c}{USD} & \multicolumn{2}{c}{CD-Bonn} \\
Partition & Amount (\%) & Partition & Amount (\%)\\
\hline
12 0 0 & 19.7 & 10 2 0 & 11.2 \\
10 2 0 & 13.0 & 8 2 2 & 10.5 \\
10 0 2 & 12.7 & 7 3 2 & 7.2 \\
8 2 2  & 9.8 & 8 3 1 & 6.8 \\
10 1 1 & 6.3 & 9 2 1 & 6.5 \\
\hline
\multicolumn{4}{c}{$^{28}$Al ($1^+_3$)} \\
\hline
10 1 1 & 17.4 & 8 3 1 & 12.8 \\
9 2 1 & 10.9 & 9 2 1 & 12.3 \\
8 2 2 & 9.2 & 8 2 2 & 11.7 \\
8 3 1 & 8.0 & 10 1 1 & 9.5 \\
10 2 0 & 7.5 & 7 3 2 & 7.0 \\
\hline
\end{tabular}
\label{occ}
\end{table}

\end{document}


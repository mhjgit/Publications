\documentstyle[aps,preprint]{revtex}

\draft

\begin{document}

\title{Effective three-body interactions and the nuclear shell model}

\author{T.\ Engeland, M.\ Hjorth-Jensen, and E.\ Osnes}
 
\address{Department of Physics, University of Oslo, N-0316 Oslo, Norway}

\maketitle

\begin{abstract}

We perform large-scale shell-model calculations
with realistic two-body  and three-body effective interactions
for Sn isotopes from $A=100$ to $A=132$.
The effective interactions are derived from recent
nucleon-nucleon potentials. 

\end{abstract}




\section{Introduction}\label{sec:sec1}


Traditional shell-model studies 
have recently received a kind of renewed
interest through large-scale shell model calculations
in both the $0f1p$ shell and the $0g1d2s$ shells with
the inclusion of the $0h_{11/2}$ intruder state as well. 
It is now fully possible to perform large-scale 
shell-model investigations
and study the excitation spectra 
for systems with 
some 10 million basis states. With recent advances in
Monte Carlo methods one is also 
able to enlarge the dimensionality
of the systems under study considerably,
and important information on e.g., ground state properties
has thereby been obtained.

An important feature of such large scale calculations
is that it allows one to probe the underlying many-body
physics in a hitherto unprecedented way.
The crucial starting point in all such shell-model 
calculations is
the derivation of an effective interaction, be it
either an approach based on a microscopic theory
starting from the free $NN$ interaction or a more 
phenomenologically determined interaction. 
In shell-model studies of e.g., the Sn isotopes, one may have
up to 31 valence particles or holes interacting via e.g.,
an effective two-body interaction. The results of such 
calculations can therefore yield, when compared with 
the availiable body of experimental data, critical
inputs to the underlying theory of the effective interaction.
Until very recently, realistic shell-model effective interactions have
mainly been applied to nuclei with two or a few valence particles beyond
closed shells, such as the oxygen and calcium isotopes. Thus, by going to the
tin isotopes, in which the major neutron shell between neutron numbers 50 and
82 is being filled beyond the $^{100}$Sn closed shell core, we have the opportunity
of testing the potential of large-scale 
shell-model calculations as well as the realiability of
realistic effective interactions in systems with many valence particles. It should
be noted that in many current shell-model calculations the effective interaction
is frequently either parametrized or adjusted in order to optimize the fit to the
data. As a matter of principle we shall refrain from making any such
adjustments and stick to the interaction obtained by a 
rigorous calculation consistent with the many-body scheme chosen. 
Only then may one 
be able to assess the quality and reliability of the interaction 
obtained and the possible
needs for improvement. 
It is our firm belief that 
one of the important  aims  
behind many-body based derivations of effective
interactions for the shell model is namely
to provide a link between e.g., the free
nucleon-nucleon interaction and properties of
finite nuclei. 
Clearly, although the nucleon-nucleon interaction is of short
but finite range, with typical interparticle
distances of the order of $1\sim 2$ fm, there are  
indications from both studies of few-body systems such as the triton and
infinite nuclear matter, that at least three-body
interactions, both real and effective ones, may be of
importance. 
Thus, with many valence nucleons present, such
large-scale shell-model calculations may
tell us how well e.g., an effective interaction
which only includes two-body terms does in
reproducing properties such as excitation spectra and
binding energies. 

The aim of this work is therefore to present 
results of shell model calculations 
for the Sn isotopes from $^{100}$Sn to $^{132}$Sn
with a special focus on the role of effective 
three-body interactions. 
To motivate the study of three-body effective interactions and 
set the scene for the present investigation, we mention 
two facets of recent works on
the heavy Sn isotopes, see Refs.\ \cite{tin98a,tin98b}. 
In general, 
excitation spectra for the chain of both odd and
even isotopes from  
$^{116}$Sn to $^{130}$Sn exhibit an excellent agreement
with the data, see e.g., Ref.\ \cite{tin98a}.
The  results reported in Refs.\ \cite{tin98a,tin98b} were 
based on shell-model calculations starting with
a two-body effective interaction determined from the 
recent charge dependent $NN$ potential of Machleidt
and co-workers \cite{cdbonn96}. This $NN$-interaction
was renormalized for the given nuclear medium, using $^{132}$Sn as closed
shell core, through
the introduction of the so-called reaction matrix $G$,
which corresponds to solving the Lippmann-Schwinger
equation for a finite nucleus. The $G$-matrix formed then
the basis for a perturbative calculation of more complicated
Feynman-Goldstone diagrams (see the discussion in
the next section). 
However, if one studies  \ref{tab:sec1table1},
one sees that the binding energy relative to $^{131}$Sn  
is clearly at askance with the data. 
The binding energy is defined as
\begin{equation}
      BE[^{132 - n}Sn)] = BE[^{132 - n}Sn] - BE[^{132}Sn] 
      - n  \left (BE[^{131}Sn] -  BE([^{132}Sn] \right ).
\end{equation}
Experiment indicates a minimum around $^{124}$Sn-$^{122}$Sn and consequently
a shell closure around $^{116}$Sn whereas theoretical 
binding energies increase linearly all the way 
down to $^{116}$Sn. 

The above is just an example of one of the problems which beset the theory
of effective interactions for the shell model. In this case, as also
pointed out by Zuker and co-workers \cite{andres94,andres96,andres98},
one is not able to obtain simultaneously a good reproduction
of both the excitation spectra and the binding energy. Similar problems
have also been discussed in connection with large scale shell-model
calculations of $1f0p$ shell nuclei. To give an example, effective
interactions derived from two-body $NN$ interactions which fit 
nucleon-nucleon scattering data, are not able to reproduce the 
well-known shell clousure in $^{48}$Ca or the excitation spectra 
of $^{47}$Ca and $^{49}$Ca, see e.g., the discussion in Ref.\ \cite{hko95}.

In this work we wish to address the discrepancy between theory and experiment
shown in Table \ref{tab:sec1table1} by including effective three-body
forces in our shell-model calculations. The reason for this follows
from the observation that the introduction of a global monopole correction
of the form $Wn(n-1)/2$, with $n$ being the number of valence particles and
$W$ a quantity which is related to the energy centroids, 
can, when added to our theoretical binding energy, partly cure the 
discrepancy seen in Table \ref{tab:sec1table1}. Three-body forces
typically yield repulsive corrections which scale as $n(n-1)$.  
In Table \ref{tab:sec1table1} we adjusted the term $W$ by simply requiring
it to equal the difference in binding energy between the calculated and
experimental values for $^{130}$Sn. This resulted in $W=0.15$ MeV.
Adding such a global correction to even nuclei shown in Table  
\ref{tab:sec1table1} results in the column labelled Mod.\ shell model.
Clearly this improves the theoretical results in the correct direction
and such a global modification of the matrix elements has no
effect on the excitation spectra. 

These observations form thus the starting point for our investigation
of three-body effective interactions. 
More explicitely, we aim at seeing whether three-body interactions
may yield a more microscopic understanding of the above problem. 
The theory for such effective interactions is thus revisited  
in the next section.
There we also show how one can obtain a fully valence-linked 
three body interaction when folded diagrams are included. In addition,
we show how one can obtain valence-space effective interactions
through the solution of the Bethe-Faddev equations.
Shell-model results with these two-body and three-body interactions
for the Sn isotopes are then presented in Sec.\ \ref{sec:sec3}.
Our concluding remarks are given in Sec.\ \ref{sec:sec4}. 

\section{Effective interactions and the shell model}\label{sec:sec2}

In this section we discuss the theory of effective interactions
for open-shell systems. We will especially pay attention
to the cancellation of so-called unlinked diagrams which arise
due to the summation of
folded diagrams. In Sec.\ \ref{subsec:sec21} we revisit time-dependent
perturbation and its connection to a valence-particle effective
interaction containing linked Feynman-Goldstone diagrams only.
The cancellation of unlinked diagrams is demonstrated through the 
application of the decomposition theorem and applied to the 
case of effective interactions based on two-body interactions 
only.
In  Sec.\ \ref{subsec:sec22} we discuss the evaluation of 
effective non-folded three-body diagrams 
through the solution of the Bethe-Faddeev equations 
and present how to 
obtain a folded-diagram expansion which contains only linked
diagrams. 


\subsection{Effective interactions for valence nucleon systems}
\label{subsec:sec21}

In this work we will primarily be interested in valence-linked
perturbative expansions for selected operators. Such expansions
can be conveniently expressed within the framework of time-dependent
perturbation theory. In this section we will review some of the
properties of the time-development operator $U(t,t')$ and its
connection to the decomposition theorem, which yields a
valence-linked expression for the actual operator. However, some
of the divergencies which occur can only be handled by the
introduction of the so-called folded diagrams, to be
discussed below. The folded diagrams
arise due to the removal of the dependence on the exact
energy of the perturbative expansion. This is the price one has
to pay when introducing the Rayleigh-Schr\"{o}dinger expansion.
In this section we derive valence-linked expressions for 
the effective interaction.

\subsection{Inclusion of three-body interactions}\label{subsec:sec22}

\section{Results and discussion}\label{sec:sec3}


\section{Conclusions}\label{sec:sec4}


\begin{thebibliography}{99}

\bibitem{tin98a} A.\ Holt, T.\ Engeland, M.\ Hjorth-Jensen, and E.\ Osnes,
Nucl.\ Phys.\ {\bf A634}, 41 (1998).
\bibitem{tin98b} T.\ Engeland, M.\ Hjorth-Jensen, A.\ Holt, and E.\ Osnes,
in proceedings of ``Highlights of modern nuclear structure'', Ed.\ A.\ Covello,
(World Sientific, Singapore, 1999), in press.
\bibitem{cdbonn96}  R.\ Machleidt, F.\ Sammarruca, and Y.\ Song,
Phys.\ Rev.\ C {\bf 53}, 1483 (1996).

\end{thebibliography}

%    table 1, section 1
\begin{table}[htbp]
     \caption{Binding energies for even Sn isotopes.}
     \label{tab:sec1table1}
     \begin{center}
     \begin{tabular}{lrrrrrrrr}
      &$^{130}$Sn &$^{128}$Sn &$^{126}$Sn &$^{124}$Sn &$^{122}$Sn
                              &$^{120}$Sn &$^{118}$Sn &$^{116}$Sn\\\hline
      Experiment       &-2.09&-3.64&-4.79&-5.47& -5.64& -5.26& -4.28& -2.61\\
      Shell Model      &-2.24&-4.60&-6.99&-9.39&-11.77&-14.11&-16.39&-18.58\\
      Mod.\ Shell Model &-2.09&-3.72&-4.81&-5.32& -5.22& -4.51& -3.15& -1.12\\
      \end{tabular}
     \end{center}
\end{table}


\end{document}





















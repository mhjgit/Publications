



In this work we will primarily be interested in valence-linked
perturbative expansions for selected operators. Such expansions
can be conveniently expressed within the framework of time-dependent
perturbation theory. In this section we will review some of the
properties of the time-development operator $U(t,t')$ and its
connection to the decomposition theorem, which yields a
valence-linked expression for the actual operator. However, some
of the divergencies which occur can only be handled by the
introduction of the so-called folded diagrams, to be
discussed below. The folded diagrams
arise due to the removal of the dependence on the exact
energy of the perturbative expansion. This is the price one has
to pay when introducing the Rayleigh-Schr\"{o}dinger expansion.
In this section we derive valence-linked expressions for both
the effective interaction and the effective transition operator.

The time-development operator $U$ has the
properties that 
\begin{equation}
	       U^{\dagger}(t,t')U(t,t')=U(t,t')U^{\dagger}(t,t')=1,
\end{equation}
which implies that $U$ is unitary
\begin{equation}
	       U^{\dagger}(t,t')=U^{-1}(t,t').
\end{equation}
Further,
\begin{equation}
	      U(t,t')U(t't'')=U(t,t'')
\end{equation}
and
\begin{equation}
	      U(t,t')U(t',t)=1,
\end{equation}
which leads to
\begin{equation}
	      U(t,t')=U^{\dagger}(t',t).
\end{equation}
We can then construct the true eigenvectors $\Psi_{i,f}$, where $i$ and
$f$ indicate the initial and final states, respectively, in
terms of the unperturbed wave functions $\Phi_{i,f}$  through the
action of the time-development operator $U$.
In the present discussion of the time-dependent theory we will make
use of the so-called complex-time approach to describe the time
evolution operator $U$ \cite{ko90}
\footnote{The time-development operator used here is determined by use
of the interaction picture. The suffix $I$, which is commonly
used in the literature to distinguish $U_I$ in the interaction
picture from $U$ in the
Schr\"{o}dinger or Heisenberg pictures, has been omitted.}.
This means that we
allow the time $t$ to be rotated by a small angle $\epsilon$
relative to the real time axis. The complex time $t$ is then
related to the real time $\tilde{t}$ by
\begin{equation}
	t=\tilde{t}(1-i\epsilon ).
\end{equation}
Let us first study the true eigenvector $\Psi_{i}$ which evolves
from the unperturbed eigenvectors $\Phi_{i}$ through the action of the
time development operator
\begin{eqnarray}
	       U(t,t')&=\lim_{\epsilon \rightarrow 0}
	      \lim_{t'\rightarrow -\infty (1-i\epsilon )}
	      {\displaystyle\sum_{n=0}^{\infty}\frac{(-i)^n}{n!}
	      \int_{t'}^{t}dt_1  \int_{t'}^{t}dt_2
	      \dots  \int_{t'}^{t}dt_n}
		\\ \nonumber
	     &  \times T\left[H_1(t_1)H_1(t_2)\dots H_1(t_n)\right],
	     \label{eq:timeu}
\end{eqnarray}
where $T$ stands for the correct time-ordering \cite{no88,fw71}.

In time-dependent
perturbation theory we let $\Psi_i$ develop from $\Phi_i$ in the
remote past to a given time $t$
\begin{equation}
	      \frac{\ket{\Psi_i}}
	      {\left\langle\Phi_i | \Psi_i \right\rangle}=
	      \lim_{\epsilon \rightarrow 0}
	      \lim_{t'\rightarrow -\infty (1-i\epsilon )}
	      \frac{U(t,t' )\ket{\Phi_i} }
	      {\bra{\Phi_i} U(t,t' )\ket{\Phi_i} },
	      \label{eq:psii}
\end{equation}
and similarly, we let
$\Psi_f$ develop from $\Phi_f$ in the remote future
\begin{equation}
	       \frac{\bra{\Psi_f}}{\left\langle
	       \Phi_f | \Psi_f \right\rangle}=
	       \lim_{\epsilon \rightarrow 0}
	       \lim_{t'\rightarrow \infty (1-i\epsilon )}
	       \frac{\bra{\Phi_f}U(t' ,t) }
	      {\bra{\Phi_f} U(t' ,t)\ket{\Phi_f} }.
	      \label{eq:psif}
\end{equation}

Here we are interested in the expectation value of a given
operator ${\cal O}$ acting at a time $t=0$. This can be achieved
from eqs.\ (\ref{eq:psii}) and (\ref{eq:psif}) defining
\begin{equation}
       \ket{\Psi_{f,i}'}=
       \frac{\ket{\Psi_{f,i}}}{\left\langle
       \Phi_{f,i} | \Psi_{f,i} \right\rangle},
\end{equation}
we obtain
\begin{equation}
	       {\displaystyle  \frac{
	       \bra{\Psi_{\lambda}'}H\ket{\Psi_{\lambda}'} }
	       {\left\langle\Psi_{\lambda}' |
	       \Psi_{\lambda}' \right\rangle} }
	       \label{eq:hexpect}.
\end{equation}

At this stage, {\em it is important to observe} that our
expression for the expectation value of a given operator $H$
{\em is hermitian} insofar H^{\dagger}=H$. This is readily
demonstrated. Eq.\ (\ref{eq:expect}) is of the general form
\begin{equation}
U(t,t_0)HU(t_0,-t),
\end{equation}
and noting that
\begin{equation}
U^{\dagger}(t,t_0)=
\left({\displaystyle e^{iH_0t}e^{-iH(t-t_0)}e^{-iH_0t}}\right)^{\dagger}
=U(t_0,-t),
\end{equation}
since $H^{\dagger}=H$ and $H_0^{\dagger}=H_0$, we have that
\begin{equation}
    \left(U(t,t_0)HU(t_0,-t)\right)^{\dagger}
    =U(t,t_0)HU(t_0,-t).
\end{equation}
This property of eq.\ (\ref{eq:expect})  explains the
reason why we are interested to express e.g.\ the expectation value of
$H$ as in eq.\ (\ref{eq:hexpect}).
In this form, the expression for the effective interaction differs
from that given by either the folded-diagram  method of eq.\ (\ref{eq:fd})
or the
Lee-Suzuki method eq.\ (\ref{eq:ls}). To define these perturbation
expansions we employed the similarity transformation eq.\
(\ref{eq:htrans})
\begin{equation}
     \Omega^{-1}H\Omega = e^{-\chi}He^{\chi},
\end{equation}
where
\begin{equation}
     \Omega =1+\chi = 1+Q\chi P = e^{\chi}.
\end{equation}
The important point to note here is that $e^{\chi}$ does not in general
fulfill the condition $\chi^{\dagger}=-\chi$, which in turn leads to
non-hermitian expansions for the effective interaction defined by
the folded-diagram method or the Lee-Suzuki expansion. {\em
This non-hermiticity
is sought to be removed in this work by using
eq.\ (\ref{eq:hexpect})}.

Eq.\ (\ref{eq:hexpect}) is our basic matrix element for
$H$, though in its present form it is not suitable
for computation, since both the numerator and denominator contain
divergencies. These divergencies have to be removed in order to
obtain a meaningful expression for $H$. The final expression
we are seeking, will be a valence-linked expansion were
every term in the expansion for $H$ is finite.
In order to discuss the removal of
these divergencies, we will use the
definition of the time-development operator in eq.\
(\ref{eq:timeu}), such that the numerator and the
denominators can be expressed as collections
of time-ordered Goldstone diagrams.  The aim of the development here is
to factorize
\begin{equation}
\lim_{\epsilon \rightarrow 0}
   \lim_{t'\rightarrow -\infty (1-i\epsilon )}
\frac{U(t,t' )\ket{\Phi_i} }
{ \bra{\Phi_i} U(t,t' )\ket{\Phi_i} },
\end{equation}
as a whole. To achieve this, we will make use of the decomposition
theorem.


\begin{equation}
	       U(0,-\infty )\ket{\Phi_i} =
	       U_Q(0,-\infty )\ket{\tilde{c}}
	       \bra{\tilde{c}}U(0,-\infty )\ket{\tilde{c}}
	       {\displaystyle \sum_{j=1}^{d}
	       U_{VQ}(0,-\infty )\ket{\Phi_j}
	       \bra{\Phi_j}U_V(0,-\infty )\ket{\Phi_i}}
	       \label{eq:decomp}
\end{equation}

The decomposition theorem applies equally well to eq.\ (\ref{eq:psif}).


Starting with the decomposition theorem as given by
eq.\ (\ref{eq:decomp}), we will in the next subsubsection
detail the derivation of a valence-linked expression
for the effective interaction appropriate for finite nuclei. Subsequently,
the techniques used to derive an effective interaction, are applied
to effective transition operators as well.


Let us for illustrative purposes
assume that the operator ${\cal O}$ corresponds to the
familiar two-body hamiltonian $H=H_0 +H_1$. The unperturbed
part $H_0$ is
\begin{equation}
H_0(t) ={\displaystyle \sum_{\alpha\beta}\bra{\alpha}(t+u)\ket{\beta}
                       a_{\alpha}^{\dagger}(t)a_{\beta}(t)},
    \label{eq:unpert}
\end{equation}
with $t$ and $u$ the sp kinetic energy and auxiliary potential, 
respectively. The eigenfunctions of $H_0$ are the unperturbed 
wave 
functions given by eq.\ (\ref{eq:twowf}).
The eigenvalues
correspond to the sum of the unperturbed sp energies. The interaction
$H_1$ is given by a two-body term
\begin{equation}
H_1(t) = {\displaystyle \frac{1}{2}
           \sum_{\alpha\beta\gamma\delta}\bra{\alpha\beta}H_1
	   \ket{\gamma\delta}a_{\alpha}^{\dagger}(t)
	   a_{\beta}^{\dagger}(t)
	   a_{\delta}(t)a_{\gamma}(t)}.
     \label{eq:inter}
\end{equation}
Here we let the indices $\alpha\beta\gamma\delta$ run over both valence-
and $Q$-space sp states. 

The effective interaction we will derive should have the following
properties:
\begin{itemize}
\item The original eigenvalue problem given by eq.\ (\ref{eq:schro})
 reduces to a model-space eigenvalue problem
\begin{equation}
    PH_{\mathrm{eff}}P\Psi_i = E_i P\Psi_i.
    \label{eq:mspacee1}
\end{equation}
 The main purpose is to derive an effective interaction from the 
  original hamiltonian. Moreover, we wish our final model-space
  eigenvalue problem to reproduce the empirical shell-model
  secular equation. More explicitely, we expect that eq.\ (\ref{eq:mspacee1})
  can be separated into a valence-space part and a core contribution
\begin{equation}
    PH_{\mathrm{eff}}'P\Psi_i = \left( E_i -E_C\right) P\Psi_i,
    \label{eq:mspacee}
\end{equation}
  where $E_C$ is the true energy of the core. If one considers a nucleus
 like $^{18}$O, $E_C$ corresponds then to the energy of $^{16}$O.
 In the empirical shell model we have
\begin{equation}
  PH_{\mathrm{eff}}'P = PH_0'P + PH_1'P,
\end{equation}
 where the sp energies of $H_0'$ in eq.\ (\ref{eq:unpert}) are taken
 to be the binding difference between a state in the appropriate nucleus
 with one nucleon in addition to closed shells and the ground state of
the corresponding closed-shell nucleus. Following this prescription,
 $PH_1'P$ should contain two-body interactions only. Note well that 
 $H_1'$ is different from $H_1$ since the latter includes both one-
and two-body terms. Similarly, $H_0'$ is normally approximated with the
experimental sp energies.
\item
  The model-space eq.\ (\ref{eq:mspacee}) is supposed to give
  $D$ solutions, $D$ being the dimension of the model space. We need therefore
  a scheme which gives us a one-to-one correspondence between $D$ parent
states $\ket{\rho_{\lambda}}$ in the model space and the true eigenstates 
$\ket{\Psi_{\lambda}}$. 
We define the parent states as the projections of the true eigenvectors
onto the model space. Further, we assume that they are linearly independent
and expand them in terms of the model-space basis states $\ket{\Phi_i}$
\begin{equation}
\ket{\rho_{\lambda}}=\sum_{i=1}^{D}C_i^{(\lambda)}\ket{\Phi_i},
\label{eq:parent}
\end{equation}
where
\begin{equation}
{\left\langle\rho_{\lambda} | P\Psi_{\mu} \right\rangle}=0,
\end{equation}
for $\lambda\neq\mu =1,2,\dots ,D$. Recall that the unperturbed
eigenfunctions $\ket{\Phi_i}$ are the eigenfunctions of $H_0$, with
corresponding unperturbed eigenvalues $\varepsilon_i$.
The latter equality holds because we
have assumed that the parent states are linearly independent.
The parent states should only be regarded as a mathematical tool in order
to obtain the effective interaction, since the construction of the parent
states depends on the projection of the true eigenfunctions onto the model
space. This projection is not available until one knows the efffective
interaction. The final expression for eq. (\ref{eq:mspacee}) should
therefore not depend on the knowledge of $\ket{\rho_{\lambda}}$, as will
be demonstrated below.
Thus we wish to have a one-to-one correspondence between
\begin{equation}
\lim_{\epsilon \rightarrow 0}
   \lim_{t'\rightarrow -\infty (1-i\epsilon )}
\frac{U(0,t' )\ket{\rho_{\lambda}} }
{ \bra{\rho_{\lambda}} U(0,t' )\ket{\rho_{\lambda}} }=
\frac{\ket{\Psi_{\lambda}}}
{\left\langle\rho_{\lambda} | \Psi_{\lambda} \right\rangle},
\end{equation}
such that 
\begin{equation}
H\frac{U(0,-\infty )\ket{\rho_{\lambda}} }
{ \bra{\rho_{\lambda}} U(0,-\infty )\ket{\rho_{\lambda}} }=
E_{\lambda}\frac{U(0,-\infty )\ket{\rho_{\lambda}} }
{ \bra{\rho_{\lambda}} U(0,-\infty )\ket{\rho_{\lambda}} },
\label{eq:truee}
\end{equation}
This equation states that the parent states $\rho_{\lambda}$ 
should give the true eigenvalues $E_{\lambda}$. In eq.\ (\ref{eq:truee})
we have suppressed the complex time-limit. Such a one-to-one correspondence
can only be proven in the complex-time approach, as demonstrated in ref.\ 
\cite{ko90}\footnote{We will not repeat the proof leading to the above statement.
For the details, we refer the reader to ref.\ \cite{ko90}, pp.\ 40.}. 
According to the complex-time approach, the eigenvalues
reproduced by eq.\ (\ref{eq:mspacee})
are the lowest $D$ eigenvalues with eigenvectors $\ket{\Psi_{\lambda}}$,
$\lambda =1,2,\dots ,D$, with non-zero projection $P\ket{\Psi_{\lambda}}$ onto
the model space. In actual calculations however, we may not obtain the 
lowest $D$ eigenstates of $H$ with non-zero model-space overlaps since we are
are not able to compute the effective interaction exactly. In the present
work we will approximate the effective hamiltonian with certain classes of 
diagrams, thus only qualitative arguments about e.g.\ the convergence
property of the effective interaction can be made. 
\end{itemize}

We wish now to derive an expression for the model-space effective 
interaction which has the structure of eq.\ (\ref{eq:mspacee}). 
Employing the definition of the parent state in eq.\ (\ref{eq:parent}),
we can rewrite eq.\ (\ref{eq:truee}) as
\begin{equation}
{\displaystyle \sum_{i=1}^{D}C_i^{(\lambda )}
H\frac{U(0,-\infty )\ket{\Phi_i} }
{ \bra{\rho_{\lambda}} U(0,-\infty )\ket{\rho_{\lambda}} }=
\sum_{j=1}^{D}C_j^{(\lambda )}E_{\lambda}\frac{U(0,-\infty )\ket{\Phi_j} }
{ \bra{\rho_{\lambda}} U(0,-\infty )\ket{\rho_{\lambda}} }},
\label{eq:truee1}
\end{equation}
which, through use of the decomposition 
theorem in eq.\ (\ref{eq:decomp}) reads
\begin{equation}
{\displaystyle \sum_{k=1}^{D}b_k^{(\lambda )}
HU_Q(0,-\infty )\ket{\tilde{c}}U_{VQ}(0,-\infty )\ket{\Phi_k}=
\sum_{l=1}^{D}b_l^{(\lambda )}E_{\lambda}
U_Q(0,-\infty )\ket{\tilde{c}}U_{VQ}(0,-\infty )\ket{\Phi_l}},
\label{eq:truee2}
\end{equation}
where we have defined 
\begin{equation}
b_k^{(\lambda )}=
{\displaystyle \sum_{i=1}^{D}C_i^{(\lambda )}
\frac{\bra{\Phi_k}U_V(0,-\infty )\ket{\Phi_i} 
\bra{\tilde{c}}U(0,-\infty )\ket{\tilde{c}}}
{ \bra{\rho_{\lambda}} U(0,-\infty )\ket{\rho_{\lambda}} }}.
\label{eq:bk}
\end{equation}
Note that both the terms $\bra{\Phi_k}U_V(0,-\infty )\ket{\Phi_i} $ and
$\bra{\tilde{c}}U(0,-\infty )\ket{\tilde{c}}$ contain
divergencies, these are however cancelled by corresponding terms in the
denominator of eq.\ (\ref{eq:bk}). The quantity $b_k$ is in turn proportional
to the projection of the true eigenstate onto the model space, a property
which follows by multiplying 
\begin{equation}
\frac{\ket{\Psi_{\lambda}}}
{\left\langle\rho_{\lambda} | \Psi_{\lambda} \right\rangle},
\end{equation}
with the model-space basis state $\bra{\Phi_k}$
\begin{equation}
\frac{\left\langle\Phi_k | \Psi_{\lambda} \right\rangle }
{\left\langle\rho_{\lambda} | \Psi_{\lambda} \right\rangle}=
{\displaystyle \sum_{i=1}^{D}C_i^{(\lambda )}
\frac{\bra{\Phi_k}U_V(0,-\infty )\ket{\Phi_i} 
\bra{\tilde{c}}U(0,-\infty )\ket{\tilde{c}}}
{ \bra{\rho_{\lambda}} U(0,-\infty )\ket{\rho_{\lambda}} }}=
b_k^{(\lambda )}.
\label{eq:bk1}
\end{equation}
Thus, the only dependence of the model-space eigenvalue problem
on the parent state $\ket{\rho_{\lambda}}$ is through the
coefficient $b_k^{(\lambda )}$, and, as we will 
demonstrate below, we may solve the model-space eigenvalue
problem for $b_k^{(\lambda )}$ only. For a known $b_k^{(\lambda )}$, 
eq.\ (\ref{eq:bk1}) serves to establish that the parent state 
really can be constructed as the projection of the true eigenstate. We can
then construct a model-space eigenstate $\ket{b_{\lambda}}$ as
\begin{equation}
\ket{b_{\lambda}}=\sum_{i=1}^{D}b_i^{(\lambda )}\ket{\Phi_i}=
\frac{P\ket{\Psi_{\lambda}}}
{\left\langle\rho_{\lambda} | \Psi_{\lambda} \right\rangle}.
\label{wfb}
\end{equation}
Before we construct the final expression 
for $H_{\mathrm{eff}}$ we observe that in general
\begin{equation}
{\left\langle b_{\lambda} | b_{\lambda} \right\rangle}\neq 1,
\end{equation}
since the projections of the true orthogonal eigenvectors onto the 
model space, do not in general preserve 
orthogonality. This deficiency, which
may lead to a non-hermitian $H_{\mathrm{eff}}$ 
can be overcome by introducing the 
biorthogonal wave function
\begin{equation}
\ket{\overline{b}_{\lambda}}=\sum_{i=1}^{D}\overline{b}_i^{(\lambda )}
\ket{\overline{\Phi}_i},
\label{wfbi}
\end{equation}
such that 
\begin{equation}
{\left\langle \overline{b}_{\lambda} | b_{\mu} \right\rangle}=
\delta_{\lambda\mu}.
\end{equation}

With these preliminaries we are now able to write a model-space equation
for the effective interaction of the form given by eq.\
(\ref{eq:expect}). Since we wish the final expression to be hermitian,
it may useful first to reproduce the 
folded-diagram effective interaction of Kuo and co-workers, which 
yields a non-hermitian expansion for $H_{\mathrm{eff}}$. 
Let us first introduce the shorthands
\begin{equation}
\ket{\Psi_{\tilde{c}}}=U_Q(t,t' )\ket{\tilde{c}}
\end{equation}
and 
\begin{equation}
U_L(t,t')\ket{\Phi_n}=U_{VQ}(t,t' )\ket{\Phi_n}
\end{equation}
which can be written as, employing the folding operation,
\begin{equation}
U_L(t,t')\ket{\Phi_n}=\ket{\Phi_n}+\left( {\displaystyle W-W\int\hat{Q} 
+W\int\hat{Q}\int\hat{Q} -\dots}\right)\ket{\Phi_n}
=\ket{\Phi_n} +\tilde{W}(t,t')\ket{\Phi_n},
\label{eq:wavefold}
\end{equation}
where the $\int$ signs represent the folding operation and the structure
of the wave function operator $\tilde{W}$
is shown in fig.\ \ref{fig:waveopfig}.
Observe also that 
\begin{equation}
U_L^{\dagger}(0,-\infty)=U_L(\infty ,0).
\end{equation}
The first term in eq.\ (\ref{eq:wavefold}) is just the free propagation 
of a state. The remaining terms include at least one nuclear interaction
and end always in a passive state. The folding operation $\int$ 
means that the states
connecting the $\hat{Q}$-box and the wave operator $W$  must be active states.
\begin{figure}[hbtp]
      \setlength{\unitlength}{1mm}
      \begin{picture}(140,60)
     %\put(25,10){\epsfxsize=12cm \epsfbox{fig1.eps}}
      \end{picture}
\caption{{\em The structure of the wave operator $\tilde{W}$. Note that it
always ends in a $Q$-space state.}}
\label{fig:waveopfig}
\end{figure}

Multiplying eq.\ (\ref{eq:truee2}) from the 
left with $\bra{\Phi_k}$ we obtain
\begin{equation}
{\displaystyle
\sum_{l=1}^{D}b_l^{(\lambda )}\bra{\Phi_k}
HU_L(0,-\infty )\ket{\Phi_l}\ket{\Psi_{\tilde{c}}} } =
E_{\lambda}b_k^{(\lambda )},
\label{eq:truee3}
\end{equation}
or
\begin{equation}
PH_{\mathrm{eff}}\ket{P\Psi_{\lambda}}=
E_{\lambda}\ket{P\Psi_{\lambda}},
\label{eq:truee4}
\end{equation}
where
\begin{equation}
H_{\mathrm{eff}}=
\bra{\Phi_k}
HU_L(0,-\infty )\ket{\Phi_l}\ket{\Psi_{\tilde{c}}}
\label{eq:heff1}
\end{equation}
Eqs.\ (\ref{eq:truee3}) and (\ref{eq:truee4}) have indeed the form of a
model space effective interaction equation. 


The hamiltonian $H$ is written as the sum of an unperturbed part $H_0$ and an
interaction term $H_1$.
In order to understand the structure of eq.\
(\ref{eq:heff1}) and obtain a secular model-space equation which resembles
that of the empirical shell model, we study first the contributions from
$H_0$ to eq.\ (\ref{eq:heff1}). This contribution can written as
\begin{equation}
\bra{\Phi_k}
H_0U_L(0,-\infty )\ket{\Phi_l}\ket{\Psi_{\tilde{c}}} =
\bra{\Phi}_kH_0\ket{\Phi_l}
\end{equation}
The term on the rhs.\ of the
latter equation comes through due to the fact that the only $P$-space
component in the wave operator $U_L$ is the unperturbed
wave function $\Phi$. This term represents therefore nothing but the
unperturbed energies and is conventionally \cite{ko90}
split into a valence
component and a core contribution
\begin{equation}
 \bra{\Phi_k}H_0\ket{\Phi_l}=\delta_{mk}\left(
\varepsilon_V + \varepsilon_C \right),
\end{equation}
where $\varepsilon_V$ is the eigenvalue of the valence part while 
$\varepsilon_C$ is the core contribution. In our example 
$^{18}$O, $\varepsilon_V$ is the unperturbed energy of the valence
particles while $\varepsilon_C$ is the unperturbed energy of the
$^{16}$O core.  The effective interaction can then 
conveniently be expressed in terms of eq.\ (\ref{eq:fd}).

In a similar way, we can decompose the interaction term $H_1$
into a valence part $H_1(V)$ and 
into a part $H_1(C)$ where $H_1$ at the time $t=0$ is not linked to any valence
line. The latter contributions give rise to contributions to the true core
energy $E_C$. Examples of such terms are exhibited in the upper part
of fig.\ \ref{fig:heff12}. Typical contributions to $H_1(V)$ are shown
in the lower part of fig.\ \ref{fig:heff12}. Thus, summarizing, we may write
\begin{figure}[hbtp]
      \setlength{\unitlength}{1mm}
      \begin{picture}(140,60)
     %\put(25,10){\epsfxsize=12cm \epsfbox{fig1.eps}}
      \end{picture}
\caption{{\em Examples of terms which contribute to $H_1(C)$,
upper part, and $H_1(V)$, lower part.}}
\label{fig:heff12}
\end{figure}
\begin{equation}
\bra{\Phi_k}
HU_L(0,-\infty )\ket{\Phi_l}\ket{\Psi_{\tilde{c}}} =
\delta_{mk}\left(
\varepsilon_V + \varepsilon_C \right)+
\bra{\Phi_k}
\left(H_1(V)+H_1(C)\right) U_L(0,-\infty )\ket{\Phi_l}\ket{\Psi_{\tilde{c}}},
\end{equation}
which finally yields
\begin{equation}
{\displaystyle
\sum_{l=1}^{D}b_l^{(\lambda )}\bra{\Phi_k}
\left(H_0(V)+H_1(V)\right) U_L(0,-\infty )\ket{\Phi_l}\ket{\Psi_{\tilde{c}}} } =
\left(E_{\lambda}-E_C\right) b_k^{(\lambda )},
\label{eq:truee5}
\end{equation}
with $E_C=\varepsilon_C +H_1(C)$. Clearly, eq.\ (\ref{eq:truee5}) has the 
form of the empirical shell-model secular eq.\ (\ref{eq:mspacee}). 
Due to the inclusion of folded-diagrams, eq.\ (\ref{eq:truee3}) is 
in general non-hermitian. 
The non-hermiticity of the folded-diagram method
is not unphyscal since we deal with projection of the true 
eigenvector to the left of $H_1$ and the unperturbed wave function
$\Phi$ to the left. This asymmetry is reflected in the fact that there
is no anologue in the FD method of fig.\ \ref{fig:nonhermit} , where
the folded block is attached to the final valence lines 
instead  of the initial ones. The effective interaction depends therefore 
on which states are the final and initial ones. The resulting effective 
interaction is therefore non-hermitian.
\begin{figure}[hbtp]
      \setlength{\unitlength}{1mm}
      \begin{picture}(140,60)
     %\put(25,10){\epsfxsize=12cm \epsfbox{fig1.eps}}
      \end{picture}
\caption{{\em Examples of folded diagrams which contritbute to the
folded-diagrams effective interaction.}}
\label{fig:nonhermit}
\end{figure}



\begin{thebibliography}{200}
%\addcontentsline{toc}{section}{References}
% references to chap 1 & 2
\bibitem{lindgren91} I.\ Lindgren, J.\ Phys.\ B: At.\ Mol.\ Opt.\ Phys.
{\bf 24} (1991) 1143
\bibitem{suzuki93} K.\ Suzuki, R.\ Okamoto, P.J.\ Ellis, J.\ Hao, Z.\ Li 
and T.T.S.\ Kuo,
Phys.\ Lett.\ {\bf B308} (1993) 1; Nucl.\ Phys.\ {\bf A560} (1993)
\bibitem{kuo93} P.\ Navratil, H.B.\ Geyer and T.T.S.\ Kuo, Phys.\ Lett.\ 
{\bf B315} (1993) 1
\bibitem{eo77} P.J.\ Ellis and E.\ Osnes, Rev. Mod. Phys. {\bf 49} (1977) 777
\bibitem{lm85} I.\ Lindgren and J.\ Morrison, Atomic Many-Body Theory,
(Springer, Berlin, 1985)
\bibitem{bran67} B.H.\ Brandow, Rev.\ Mod.\ Phys.\ {\bf 39} (1967) 771

\bibitem{ko90} T.T.S.\ Kuo and E.\ Osnes, Folded-Diagram Theory
of the Effective Interaction in Atomic Nuclei, Springer Lecture
Notes in Physics, (Springer, Berlin, 1990) Vol. 364

\bibitem{kuo81} T.T.S.\ Kuo, Lecture Notes in
Physics; Topics in Nuclear Physics, ed. T. T. S. Kuo and S. S.
M. Wong, (Springer, Berlin, 1981) Vol. 144, p. 248
\bibitem{hom92} M.\ Hjorth-Jensen, E.\ Osnes and H.\ M\"{u}ther,
Ann. of Phys.  {\bf 213} (1992) 102
\bibitem{no88} J.W.\ Negele and H.\ Orland, Quantum Many-Particle
Systems, (Addison-Wesley, Redwood City, 1988)

\bibitem{mhe87} R.\ Machleidt, K.\ Holinde and C.\ Elster, Phys.\ Reports
{\bf 149} (1987) 1
\bibitem{mac89} R.\ Machleidt, Adv.\ Nucl.\ Phys.\ {\bf 19} (1989) 189
\bibitem{paris80} M. Lacombe {\em et al.} , Phys. Rev. {\bf C21} (1980)
861
\bibitem{reid68} R.V.\ Reid, Ann.\ of Phys.\ {\bf 50} (1968) 411
\bibitem{ms91} C.\ Mahaux and R.\ Sartor, Adv.\ Nucl.\ Phys.\ {\bf 20}
(1991) 1
\bibitem{dm92} W.H. Dickhoff and H. M\"{u}ther, Rep.\ Prog.\ Phys.\
{\bf 55} (1992) 1947
\bibitem{pand93} V.J.\ Pandharipande, Nucl.\ Phys.\ {\bf A553} (1993) 191c
\bibitem{bloch59} C.\ Bloch, Nucl.\ Phys.\ {\bf 6} (1958) 329;
Nucl.\ Phys.\ {\bf 7} (1958) 451
\bibitem{klz76} H.\ K\"{u}mmel, K.H.\ L\"{u}hrman and J.G.\
Zabolitzky, Phys. Reports {\bf 36C} (1978) 1
\bibitem{ls80} S.Y.\ Lee and K.\ Suzuki, Phys.\ Lett.\ {\bf B91} (1980) 79;
K.\ Suzuki and S.Y.\ Lee, Prog.\ Theor.\ Phys.\
{\bf 64} (1980) 2091
\bibitem{des} J.\ Des Cloizeaux, Nucl.\ Phys.\ {\bf 20} (1960) 321
\bibitem{sw72} T.H.\ Schucan and H.A.\ Weidenm\"{u}ller, Ann. of Phys.
{\bf 73} (1972) 108; {\bf 76} (1973) 483
\bibitem{lk75} J.M.\ Leinaas and T.T.S.\ Kuo, Ann. of Phys.
{\bf 98} (1975) 177
\bibitem{lk77} J.M.\ Leinaas and T.T.S.\ Kuo, Ann. of Phys.
{\bf 111} (1977) 19
\bibitem{sm90} L.D.\ Skouras and H.\ M\"{u}ther, Nucl. Phys. {\bf A515}
(1990) 93
\bibitem{ee70} P.J.\ Ellis and T.\ Engeland, Nucl. Phys. {\bf A144} (1970) 161




\end{thebibliography}






\end{document}













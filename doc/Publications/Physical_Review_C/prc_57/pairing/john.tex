
%% ReVTeX 3.0:  10 pages %%%%
\documentstyle[prl,aps,preprint]{revtex}
\begin{document}
\draft
\title{Universalities of Triplet Pairing in Neutron Matter}
\author{V. A. Khodel, V. V. Khodel, and J. W. Clark}
\address{McDonnell Center for the Space Sciences
and Department of Physics,\\
Washington University, St. Louis, MO 63130 USA}
\date{\today , Submitted to Physical Review Letters}
\maketitle
\begin{abstract}

The fundamental structure of the set of unitary solutions of
the BCS $^3{\rm P}_2$ pairing problem in neutron matter is established.
The relations between different spin-angle components in these 
solutions are shown to be practically independent of density,
temperature, and the specific form of the pairing interaction.
The spectrum of pairing energies is found to be highly degenerate.
\end{abstract}

\vskip 1cm
\pacs{67.20+k, 74.20.Fg, 97.60.Jd}

Since the discovery of superfluidity in liquid $^3$He \cite{osh}, 
great advances have been made in understanding the properties of 
superfluid systems with triplet pairing.  In addition to the well-studied
case of liquid $^3$He below 2.6 mK~[2-7], triplet pairing is 
expected to occur in neutron matter in the quantum fluid interior of
a neutron star~[8-13].  A number of common features of superfluid, 
thermodynamic, and magnetic properties of different systems have been
revealed by analyses based on symmetry principles \cite{wol,vol}. 
However, as we shall see, dealing directly with the BCS gap equation 
permits one to uncover new universalities in  triplet pairing. 

The purpose of this letter is to identify fundamental solutions of the
triplet pairing problem in neutron matter and elucidate their structure
and their relationships.  If two identical spin-${1\over 2}$ fermions 
are paired with a nonzero total momentum ${\bf J}={\bf L}+{\bf S}$, 
the ordinary $S$-wave gap equation is converted into a system 
of coupled integral equations.  In the standard notation \cite{tak,ost}, 
we have the expansion
$\Delta({\bf p}) =\sum\Delta_{LJ}^M(p)G_{LJ}^M({\bf n})$ 
in the spin-angle matrices 
$G_{LJ}^M({\bf n};s_1,s_2)=\sum C^{1M_S}_{{1\over 2}
{1\over 2}s_1s_2}C^{JM}_{1LM_sM_L}Y_{LM_L}({\bf n})$, 
with $(\Delta_{LJ}^M(p))^*=(-1)^{J-M}\Delta_{LJ}^{-M}(p)$ assuming
time-reversal invariance.  In systems with  tensor forces, the
$pp$ interaction has the corresponding expansion 
$V({\bf p},{\bf p}')=\sum\langle p|V_{LJ}^{L'J}|p'
\rangle G_{LJ}^M({\bf n})G_{L'J}^{M*}({\bf n}')$.
The generalized BCS system then reads
\begin{equation}
 \Delta_{LJ}^M(p)
= \sum_{L'L_1J_1M_1}(-1)^{\Lambda} \int 
  \langle p|V_{LJ}^{L'J}|p'\rangle S^{MM_1}_{L'JL_1J_1}({\bf n}')
  {\Delta_{L_1J_1}^{M_1}(p')\over 2E({\bf p}')}
  \tanh {E({\bf p}') \over 2T}d\tau' \, ,
\end{equation}
where $\Lambda=L-L_1+1$, $d\tau= p^2dpd{\bf n} \equiv d\tau_0d{\bf n}$, 
and $S^{MM_1}_{LJL_1J_1}({\bf n})={\rm Tr}
\left[G_{LJ}^{M*}({\bf n})G_{L_1J_1}^{M_1}({\bf n})\right]$ 
represents the summation over spin variables.  The energy denominator
$E({\bf p})=[\xi^2(p)+D^2({\bf p})]^{1/2}$ involves
the single-particle excitation energy $\xi(p)$ of the
normal system measured from the Fermi surface and a gap quantity
whose square is constructed as
\begin{equation}
 D^2({\bf p})={1\over 2}\sum_{LJML_1J_1M_1}\Delta_{LJ}^{M*}(p)
 \Delta_{L_1J_1}^{M_1}(p) S^{MM_1}_{LJL_1J_1}({\bf n}) \, .
\end{equation}

For the present, we retain only diagonal terms with $L_1= L'=L,J_1=J$ 
on the r.h.s. of (1), thus allowing for superposition of spin-angle
components only in the magnetic quantum number $M$.  The contributions 
from nondiagonal terms can be incorporated with the aid a standard 
renormalization procedure that alters the interaction $V$ but does 
not substantially affect the results we shall derive.  The analysis 
is greatly facilitated by a generalization of the separation method 
developed for $S$-wave pairing in \cite{kkc}.   Thus, defining 
$\phi_{LJ}(p)=\langle p|V_{LJ}^{LJ}|p_F\rangle /v_F$ and 
$v_F=\langle p_F|V_{LJ}^{LJ}|p_F\rangle $, we employ the decomposition
\begin{equation}
  4\pi\langle p| V_{LJ}^{LJ}| p'\rangle 
= v_F\phi_{LJ}(p)\phi_{LJ}(p')+W_{LJ}(p,p') \, ,
\end{equation}
of the relevant pairing matrix into a separable portion and a
remainder $W_{LJ}(p,p')$ that vanishes identically when either
argument is on the Fermi surface.   Due to the latter property, 
integrals containing $W_{LJ}$ are guaranteed to receive their overwhelming
contributions some distance from the Fermi surface where the 
replacements $E\rightarrow |\xi|$ and $\tanh (E/2T)\rightarrow 1$ 
are justified to high accuracy.  Substituting (3) into (1), 
the gap equations are recast as
\begin{equation}
  \Delta_{LJ}^M(p)+\int {W_{LJ}(p,p')\over 2|\xi(p')|}
  \Delta_{LJ}^M(p')d\tau_0'=v_FB_{LJ}^M\phi_{LJ}(p) \, ,
\end{equation}
\begin{equation}
  B^M_{LJ}
= -\sum_{M_1} \int \phi_{LJ}(p)S^{MM_1}_{LJLJ}({\bf n})
  {\Delta_{LJ}^{M_1}(p)\over 2E({\bf p})}
  \tanh {E({\bf p})\over 2T}d\tau \, .
\end{equation}
The quantities $B^M_{LJ}$ are merely numerical factors.  Consequently,
the $p$ dependence  of all gap components is seen to be identical.
Specifically, we may write $\Delta_{LJ}^M(p)=D_{LJ}^M\chi_{LJ}(p)$, 
where the shape factor $\chi_{LJ}(p)$ obeys an integral equation 
\begin{equation}
   \chi_{LJ}(p) + \int W_{LJ}(p,p')
   {\chi_{LJ}(p') \over 2|\xi(p')|}d\tau_0'=\phi_{LJ}(p) 
\end{equation}
of the same form as in the singlet case \cite{kkc}.  To determine
the amplitude $D_{LJ}^M$, we note that $\chi_{LJ}(p_F)=\phi_{LJ}(p_F)=1$  
since $W_{LJ}(p_F,p')=0$.  Therefore $\Delta_{LJ}^M(p_F)=D_{LJ}^M$,
and Eq.~(4) implies $D_{LJ}^M=v_FB^M_{LJ}$ while Eq.~(5) gives
\begin{equation}
   D^M_{LJ}
= -v_F\sum_{M_1} \int \phi_{LJ}(p)S^{MM_1}_{LJLJ}({\bf n})
  D^{M_1}_{LJ} {\chi_{LJ}(p)\over 2E({\bf p})} 
  \tanh {E({\bf p})\over 2T}d\tau \ .
\end{equation}
The system (6)--(7) is more convenient for solution than the 
original equations (1), since the problem has been decomposed into
(i)~evaluation of the $M$-independent shape factor 
$\chi_{LJ}(p)$ from the nonsingular linear integral Eq.~(6), and
(ii)~determination of the structure coefficients $D^M_{LJ}$ from the
nonlinear equation (7), where the log singularity has been isolated.

Henceforth we specialize to the case $L=S=1$, $J=2$, this being 
the most favored uncoupled channel for pairing in neutron matter 
at densities prevailing in the quantum fluid interior of a neutron
star ($k_F\sim 2$~fm$^{-1}$), where the $^1$S$_0$ gap has already 
closed \cite{tak,kkc}.  The arguments are simplified if we adopt 
$D^0_{12}\equiv\delta$ as a scale factor and introduce the 
``structure function'' 
\begin{eqnarray}
    d^2({\bf n}) = 
&& 16\pi D^2({\bf p})/\chi_{12}^2(p)
   = \delta^2[(1+\lambda_2)^2+\kappa^2_1
     +(\lambda_1^2-4\lambda_2-\kappa^2_1)x^2 \nonumber\\
&-& 2(\lambda_1+ \lambda_1\lambda_2+\kappa_1\kappa_2) xz 
    +(3+\lambda^2_1-\lambda^2_2-2\lambda_2)z^2\nonumber\\
&+& 2(2\kappa_2-\kappa_1\lambda_1)xy
    +2(\kappa_1+ \lambda_1\kappa_2-\lambda_2\kappa_1)yz] \, ,
\end{eqnarray}
where $x=\sin\theta\cos\varphi$, 
$y=\sin\theta\sin\varphi$, and $z=\cos\theta$.  
After the separation of the real and imaginary parts of 
$D^{M\neq 0}_{12}\equiv (\lambda_M+i\kappa_M)\delta\sqrt{6}$ in (7), 
we arrive at the set of equations
\begin{eqnarray}
    \lambda_2
&=& -v_F[\lambda_2(J_0+J_5) -\lambda_1 J_1 -\kappa_1J_2-J_3]\, ,\nonumber \\
    \kappa_2
&=& -v_F[\kappa_2(J_0+J_5) -\kappa_1 J_1 +\lambda_1J_2+J_4] \, ,\nonumber\\
    \lambda_1
&=& -v_F[\lambda_1J_6-(\lambda_2+1)J_1+\kappa_2J_2-\kappa_1J_4/2]\, ,\nonumber\\
     \kappa_1
&=& -v_F[\kappa_1J_7-\kappa_2J_1-(\lambda_2-1)J_2-\lambda_1 J_4/2]\, ,\nonumber\\
    1
&=&-v_F[-(\lambda_1 J_1-\kappa_1 J_2+\lambda_2 J_3-\kappa_2 J_4)/3+J_5] \,,
\end{eqnarray}
which, in angular content, is consistent with the corresponding set 
in \cite{ost}.  The integrals $J_k$ are given by $J_6=(J_0+4J_5+2J_3)/4$, 
$J_7=(J_0+4J_5-2J_3)/4$, and, for $k=1,\cdots,5$, by
\begin{equation}
  J_k = \int f_k(\theta,\varphi){\phi_{12}(p)\chi_{12}(p)
        \over 2E({\bf p})}\tanh{E({\bf p})\over 2T}d\tau \, ,
\end{equation}
with
$f_0=1-3z^2$, $f_1=3xz/2$, $f_2=3yz/2$, $f_3=3(2x^2+z^2-1)/2$, $f_4=3xy$, 
and $f_5=(1+3z^2)/2$.
Substitution of $\partial d^2(\theta,\varphi)/\partial\varphi$ 
for $f_k$ in (10) evidently yields zero upon integration over $\varphi$. 
This identity implies a relation 
\begin{equation}
  \sum_{k=1}^4 c_kJ_k=0
\end{equation}
between the integrals $J_k$,
with $c_1=\kappa_1+\lambda_1\kappa_2- \lambda_2\kappa_1$, 
$c_2=\lambda_1+\lambda_1\lambda_2+\kappa_1\kappa_2$, 
$c_3=2\kappa_2-\kappa_1\lambda_1$, and 
$c_4=2\lambda_2-(\lambda_1^2-\kappa_1^2)/2$.
But note that if we multiply the first equation of (9) 
by $\kappa_2$, the second by $\lambda_2$, the third by $2\kappa_1$, 
and the fourth by $-2\lambda_1$, and subtract the results of the last
three operations from that of the first, relation (11) is reproduced.
Thus only four of the five equations in (9) are truly independent and 
hence any one of the parameters 
$\lambda_1$, $\lambda_2$, $\kappa_1$, $\kappa_2$ 
can be chosen arbitrarily.  We take $\kappa_1=0$.  
With this choice, solutions of (9) are necessarily even functions of 
$\lambda_1$, so attention may be focused on the sector $\lambda_1\geq 0$. 

The set (9) has three one-component solutions \cite{tak,ost} with 
$|M|=0,~1$, and 2.  We start the search for multicomponent solutions
with the case $\kappa_2=0$, for which $d^2({\bf n})$ is independent 
of $y$. In Eq.~(10), the integration of $f_k(\theta,\varphi)$ 
over $y$ is then carried out using the formula 
$\sin\theta d\theta d\varphi =2\delta(x^2+y^2+z^2-1)dx~dy~dz$.  
For $k=1,3$, this yields $2f_k(x,z)(1-x^2-z^2)^{-1/2}$ 
since $f_1$ and $f_3$ are independent of $y$, while for $k=2,4$ 
one finds $J_2=J_4=0$ because $f_2$ and $f_4$ are odd in $y$ 
and the $y$ integral has symmetric limits.  As a result, 
there remain only three independent equations,
\begin{eqnarray}
\lambda_2 &=& -v_F[\lambda_2(J_0+J_5) -\lambda_1 J_1-J_3] \, ,\\
\lambda_1 &=& -v_F[\lambda_1(J_0/4+J_5)-(\lambda_2+1)J_1
              +\lambda_1 J_3/2] \, ,\\
        1 &=& -v_F[-(\lambda_1 J_1+\lambda_2 J_3)/3+J_5] \, .
\end{eqnarray}
We first identify and verify particular solutions of this system,
beginning with ($\lambda_2=-1,\lambda_1~{\rm arbitrary}$).
The structure function (8) is seen to take the factorized form
$d^2({\bf n};\lambda_1,\lambda_2=-1)=3\Delta^2(x^2+z^2)$, where 
$\Delta^2=\delta^2(\lambda^2_1+4)/3$ is a pairing energy parameter. 
Integration of the combination $3f_0+2f_3=6(x^2-z^2)$ over $x,z$ 
then produces 0, hence $3J_0+2J_3=0$, since this quantity changes 
sign on interchange of $x$ and $z$ while $d^2$ and other factors 
within the integrand of (10) are left unchanged.  By the same reasoning, 
integration of $f_1=xz$ over $x$ gives 0 and hence $J_1=0$.  
It follows that Eqs.~(12)--(14) {\it coincide} and provide an 
equation $1=-v_F[J_3/3+J_5]$ that determines $\Delta^2$ rather 
than $\lambda_1$ and $\delta$ individually.  Here we have a striking 
example of the universal structure of solutions of the $^3{\rm P}_2$ 
pairing problem.  Two other particular solutions of Eqs.~(12)--(14) 
are ($\lambda_1=0$, $\lambda_2=3$) and ($\lambda_1=0$, $\lambda_2=1$). 
In both cases, Eq.~(13) vanishes identically.  The symmetry 
of $d^2(x,z;\lambda_1=0,\lambda_2=3)=\Delta^2[4-3(x^2+z^2)]$
with respect to $x$ and $z$ again implies the relation $3J_0+2J_3=0$.  
Again Eqs.~(12) and (14) coincide and the resulting equation, 
$1=-v_F(J_5-J_3)$, determines only $\Delta^2$.  On the other hand, 
$d^2(x,z;\lambda_1=0,\lambda_2=1)=3\Delta^2(1-x^2)$ depends on $x$ 
alone, allowing us to integrate (10) over $z$, for any form of
$d^2\equiv d^2(x)$, to obtain the relation $3J_0-2J_3=0$.  Yet
again Eqs.~(12) and (14) merge, in this case to $1=-v_F(J_5-J_3/3)$, 
which again only determines $\Delta^2$.

Other solutions of the system (12)--(14) are found by implementing 
a rotation \\
$R=\bigl( x=t\cos\alpha+u\sin\alpha,\, z=t\sin\alpha-u\cos\alpha\bigr)$ 
with $\tau=\tan\alpha$.  Expressing $d^2$ 
in terms of $t$ and $u$, one easily finds conditions
\begin{equation}
  (\lambda^2_1-4\lambda_2)\tau^2
+ 2\lambda_1(1+\lambda_2)\tau
+ \lambda^2_1-\lambda^2_2-2\lambda_2 +3=0 \, ,~~~
  \lambda_1\tau^2-(\lambda_2-3)\tau -\lambda_1=0 \, ,
\end{equation}
under which  $d^2$ becomes a function of $t$ only.  The choice 
$\tau(\lambda_1,\lambda_2)=\tau_0(\lambda_1,\lambda_2) =\\
\lambda_1(1+\lambda_2)/(4\lambda_2-\lambda^2_1)$ 
meets both conditions provided
\begin{equation}
(\lambda^2_1-2\lambda_2+2)(\lambda^2_1-2\lambda^2_2-6\lambda_2)=0 \, .
\end{equation}
Equation~(16) embodies three branches of $\lambda_2$ versus $\lambda_1$, 
which start as parabolas from $\lambda_1=0$ and $\lambda_2=1$, 0, and $-3$.  
The structure function factorizes as $d^2(t)\propto\Delta^2_1(1-t^2)$ 
when the first factor of (16) vanishes and as 
$d^2(t)\propto\Delta^2_2(1+3t^2)$ when the second is zero, with
appropriate different values of the pairing parameter.
Calculating the integrals $J_1$ and $J_3$ by rotation of the 
$x,z$ plane under $R$, we are led to the relations
\begin{eqnarray}
    (\lambda^2_1-4\lambda_2)J_1+\lambda_1(\lambda_2+1)J_3
&=& 0 \, ,\nonumber\\
    3\lambda_1(1+\lambda_2)J_0-2(\lambda^2_1-2\lambda^2_2+6)J_1
&=& 0 \, .
\end{eqnarray}
The first relation (for example) is verified as follows, noting
that its l.h.s. is proportional to 
$[(\lambda^2_1-4\lambda_2)f_1+\lambda_1(\lambda_2+1)f_3]/(1-t^2-u^2)^{1/2}$. 
Substituting $f_1(t,u)$ and $f_3(t,u)$ and integrating over $u$, 
which can be done freely for any shape of $d^2(t)$, 
we obtain a result that is proportional to 
$[(\lambda^2_1-4\lambda_2)\tau+\lambda_1(\lambda_2+1)]$ and therefore
vanishes when $\tau_0(\lambda_1,\lambda_2)$ is substituted.  What is 
remarkable is that Eqs.~(12)--(14) coincide when relations (17) 
are inserted, and once again these equations determine only 
$\Delta^2$. Thus the rotation $R$ permits us ``to kill two birds with 
one stone,'' and the attendant condition (16) specifies another
set of solutions of our system. 

We now allow $\kappa_2$ to have a nonzero value, thus bringing in
the second of Eqs.~(9).  At $\lambda_1=0$, this equation becomes identical
with the first of the set, as is seen with the aid of Eq.~(11).  
The particular solution $(\lambda_1=0,~\lambda_2=3)$ is then replaced 
by one with $\lambda_1=0$ and $(\lambda_2^2+\kappa_2^2)^{1/2}=3$, but 
the pairing energy remains unaltered, the relevant quantities being
independent of the phase of the coefficient $D^2_{12}(\lambda_1=0)$. 
To find the other multicomponent solutions in the general case with 
$\kappa_2\neq 0$ and $\lambda_1\neq 0$, we may extend our previous 
tactic and apply a rotation in {\it three}-dimensional
space so as to eliminate four terms in expression (8) and cast
$d^2$ into a one-dimensional form.  The three Euler angles are 
thereby fixed, implying a relation between the parameters $\lambda_2$,
$\kappa_2$, and $\lambda_1$, namely 
\begin{equation}
  \kappa^2_2=(1+\lambda_2)(\lambda^2_1/2-\lambda_2+1) \, ,
\end{equation}
which can be shown to satisfy all of Eqs.~(9).  This relation defines
two branches of $\kappa_2(\lambda_1,\lambda_2)$.  Starting at the plane 
$\kappa_2=0$, one branch grows out of the particular solution 
$\lambda_2=-1$ while the other grows out of the quasiparabola 
$\lambda^2_1/2=\lambda_2-1$ contained in Eq.~(16).  (Accordingly,
$\lambda_2=-1$ and the latter quasiparabola cannot be counted
as independent solutions.)  The branches (18) complete the set 
of unitary states of the $^3{\rm P}_2$ problem.

At $T=0$, the solutions we have identified divide into two groups, 
the states within a group being essentially degenerate in energy.  
This behavior is consistent with the numerical calculations reviewed 
in \cite{tak}.  The group with lowest energy, having structure function 
$d^2(t)=\Delta^2(1+3t^2)$, contains only nodeless states and consists 
of (i)~the particular state ($\lambda_1=0$, $\lambda_2=3$) and 
(ii)~the states (including the one-component state with $M=0$) 
belonging to the branches of Eq.~(16) starting at the points 
$\lambda_2=0$ and $\lambda_2=-3$.  The upper group contains the 
remaining states, having $d^2(t)=3\Delta^2(1-t^2)$ and one node.
The energy splitting between the groups is a universal function of 
the ratio $T/T_c$, the critical temperature $T_c$ being determined
by Eq.~(14) reduced to $1=-v_FJ_5(T_c)$.
The splitting between the groups can be calculated (for example) 
from the ratio 
$\eta(T)=\Delta^2_1(T)/\Delta^2_0(T)=\delta^2_1(T)/3\delta^2_0(T)$ 
for the particular states $\lambda_2=-1$ and ($\lambda_1=0$, 
$\lambda_2=3$), denoted 1 and 0 respectively.
The ratio $\delta_1^2(T)/\delta^2_0(T)$ of the scale factors is also 
found from Eq.~(14). It can be shown that the dominant
contributions to this ratio come from the integration region adjacent 
to the Fermi surface where $\chi_{12}(p)=\phi_{12}(p)=1$. 
In doing so at $T=0$ while neglecting unimportant strong-coupling
corrections to the interaction $V$ involving the gap itself, one finds
\begin{equation}
 \ln\eta(T=0)={2\pi\over 9\sqrt{3}}+{2\over 3}-\ln 3\simeq -0.028 \, ,  
\end{equation}
in agreement with the calculations of Takasuka and Tamagaki \cite{tak}.
Analytical results are also available in the vicinity of the critical 
temperature $T_c$.  Straightforward evaluation of the integrals 
$J_k$ in Eq.~(14) near $T_c$ yields
$\Delta^2_k(T\rightarrow T_c)\sim {5\over 6}\Delta_S^2(T\rightarrow T_c)$,
where $ \Delta_S^2$ is the corresponding BCS $S$-wave pairing value.  
>From this formula, which applies to {\it all} solutions of 
Eqs.~(12)--(14), we infer that the specific heat jumps at $T_c$ 
are also universal, i.e., independent of the density $\rho$. 
Thus, if mixing of the different $L,J$ channels is neglected, 
the $^3{\rm P}_2$ gap spectrum turns out to be essentially degenerate 
and independent of any input parameters including the density $\rho$ 
and the temperature $T$.  The concrete form of the $pp$ interaction 
was not used anywhere, so the structure and relations we have established 
retain their validity even when fluctuation contributions to $V$ 
(or other dressings of the pairing interaction) are taken into account.  
Under heating or pressure only the scale of the pairing energy is 
altered, and transitions between different states are first-order 
phase transitions.

Now let us consider the non-diagonal contributions to Eq.~(1), which 
can be accommodated by a renormalization of the interaction $V$, 
following a well-known procedure employed in Fermi-liquid 
theory \cite{mig}.  Schematically, Eq.~(1) is rewritten in the 
form $\Delta=V(A_d+A_n)\Delta$, where $A_d$ [respectively, $A_n$] 
collects the diagonal [nondiagonal] contributions, 
and the operator $(1-VA_n)^{-1}$ applied to both the 
sides of this revised equation.  Equation~(1) is thereby replaced 
by a new equation, $\Delta=V_{\rm eff}A_d\Delta$, 
in which only diagonal terms appear but $V$ has been replaced 
by an effective interaction $V_{\rm eff}=V+VA_nV_{\rm eff}$.  
A complication is that the structure function $d^2$ of the 
renormalized equation involves higher harmonics and is no longer 
factorizable.  Therefore the connection (11) and Eqs.~(14)--(16) 
are modified, with a consequent lifting of degeneracy that entails 
full determination of the parameters $\lambda_i$, $\kappa_i$.  
However, simple estimation and analysis of available numerical 
results\cite{and} indicate that the resulting corrections 
to the universal relations we have derived for triplet pairing 
in neutron matter are small, of order one percent.

The situation changes drastically in the case of superfluid $^3$He,
where the states with $L=S=1$ and $J=0,1,2$ contribute on 
equal footing and hence the number of equations to be solved rises 
to nine.  Nevertheless, we expect the same methods of analysis 
to prove illuminating for this system as well.

We acknowledge helpful discussions with R.~Fisch, P.~Schuck, and 
G.~E.~Volovik.  This research was supported in part by the Russian 
Foundation for Basic Research under Grant No.~96-02-19292 and by 
the U. S. National Science Foundation under Grant No.~PHY-9602127. 

\begin{references}
\bibitem{osh} 
D.~D.~Osheroff, R.~C.~Richardson, and D.~M.~Lee, 
Phys. Rev. Lett. {\bf 28}, 885 (1972). 
\bibitem{and}
P.~W.~Anderson and P.~Morel, Phys. Rev. {\bf 123}, 1911 (1961). 
\bibitem{leg} 
A.~J.~Leggett, Rev. Mod. Phys. {\bf 47}, 331 (1975).
\bibitem{wheat} 
J.~C.~Wheatley, Rev. Mod. Phys. {\bf 47}, 415 (1975). 
\bibitem{has} 
Y.~Hasegawa, T.~Usagawa, and F.~Iwamoto, 
Prog. Theor. Phys. {\bf 63}, 1458  (1979).
\bibitem{wol} 
D.~Vollhardt and P.~W\"olfle, {\it The Superfluid Phases of Helium 3} 
(Taylor \& Francis, London, 1990).
\bibitem{vol} 
G.~E.~Volovik, in {\it Helium Three}, ed. W.~P.~Halperin
and L.~P.~Pitaevskii (North Holland, Amsterdam, 1990).
\bibitem{pin} 
D.~Pines and M.~A.~Alpar, Nature {\bf 316}, 27 (1985).
\bibitem{hoff} 
M. Hoffberg, A. E. Glassgold, R. W. Richardson, and
M. Ruderman, Phys. Rev. Lett. {\bf 24}, 775 (1970).
\bibitem{tak} 
T.~Takatsuka and R.~Tamagaki, 
Prog. Theor. Phys. Suppl. {\bf 112}, 27 (1993). 
\bibitem{ost} 
L.~Amundsen and E.~\O stgaard, Nucl. Phys. {\bf A442}, 163 (1985). 
\bibitem{bal} 
M.~Baldo, J.~Gugnon, A.~Lejeune, and U.~Lombardo, 
Nucl. Phys. {\bf A515}, 409 (1990).
\bibitem{oslo} 
{\O}. Elgar{\o}y, L. Engvik, M. Hjorth-Jensen, and E. Osnes,
Nucl. Phys. {\bf A607}, 425 (1996).
\bibitem{kkc} 
V.~A.~Khodel, V.~V.~Khodel, and J.~W.~Clark, 
Nucl. Phys. {\bf A598}, 390 (1996).
\bibitem{mig} 
A. B.~Migdal, {\it Theory of Finite Fermi Systems and 
the Properties of the Atomic Nucleus} (Interscience, New York, 1967).
\end{references}

\end{document}




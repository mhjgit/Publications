\documentstyle[aps,psfig,multicol]{revtex}

\begin{document}

\draft

\title{Nucleon-Nucleon Phase Shifts and Pairing in 
Neutron Matter and Nuclear Matter}
 
\author{\O.\ Elgar\o y$^a$ and M.\ Hjorth-Jensen$^b$}

\address{$^a$Department of Physics, University of Oslo, N-0316 Oslo, Norway}

\address{$^b$Nordita, Blegdamsvej 17, DK-2100 K\o benhavn \O, Denmark}

\maketitle

\begin{abstract}

We consider $^1S_0$ pairing in infinite neutron matter 
and nuclear matter 
and show that in the lowest order approximation, where the pairing 
interaction is taken to be the bare nucleon-nucleon (NN) interaction 
in the $^1S_0$ channel, the pairing interaction and the energy gap 
can be determined directly from the $^1S_0$ phase shifts.  
This is due to the almost separable 
character of the nucleon-nucleon interaction in this partial wave. 
Since the most recent NN interactions are charge-dependent, we 
have to solve coupled gap equations for proton-proton, neutron-neutron, 
and neutron-proton pairing in nuclear matter.  The results are, 
however, found to be close to those obtained with charge-independent 
potentials.


\end{abstract}

\pacs{PACS number(s): 21.30.-x, 21.65.+f, 26.60.+c }

\begin{multicols}{2}

Recently, there has been renewed interest in the pairing problem in 
neutron matter and neutron-rich nuclei.  The superfluid properties 
of neutron matter is of importance in the study of neutron stars 
\cite{petra95}, while pairing in neutron-rich systems is of relevance 
for the study of heavy nuclei close to the drip line \cite{mulshe93} 
and the light halo nuclei \cite{riis94}.  Much effort has gone into 
calculating the superfluid energy gap in dilute neutron matter 
\cite{baldo90,chen93,tak93,elg96,khodel96}.  
Most of these studies,  
e.g., those of Refs.\ \cite{baldo90,tak93,elg96,khodel96} have been carried 
out using pairing matrix elements given by the bare nucleon-nucleon 
(NN) interaction.  Many of the same authors have calculated the 
$^1S_0$ gap in nuclear matter, which has also been the subject of  
recent relativistic formulations of the pairing problem 
\cite{ring90,guim96,matera97}.  

Even though it is a long time since  
Clark et al.\ \cite{clark76} showed that the effects of density and 
spin-density fluctuations must be included in the pairing interaction, 
and there 
has been much progress in that direction recently \cite{wam93,schulze96}, 
we will here focus on the situation at the level of the 
bare interaction.  In this lowest-order approximation to the 
problem it has been found that results for the $^1S_0$ energy 
gap in neutron matter and in nuclear matter are almost independent 
of the choice of NN interaction.  
We aim at explaining how this can 
be understood directly from the measured properties of the free NN 
interaction.  
Moreover, although a relation between the pairing gap and
NN phase shifts was obtained almost forty years ago by Emery and
Sessler \cite{es60} (see also Hoffberg et al.\ \cite{hoffberg70}),  
in this work we wish to focus on the near 
interaction independence of the results for the energy gap at the 
Fermi level, and try to explain this from the NN scattering data directly.  
Our investigation is similar in spirit to the work of Refs.\ 
\cite{khodel96,carlson97} where the relation between the $^1S_0$ 
scattering amplitude and the gap function in momentum space was 
clarified.  In this paper, however, the focus is on the size 
of the energy gap at the Fermi momentum and how well this quantity is 
determined by NN scattering data.  

The energy gap in infinite matter is obtained by solving the BCS equation 
for the gap function $\Delta(k)$.  
\begin{equation}
      \Delta(k)=-\frac{1}{\pi}\int_{0}^{\infty}dk'k'^2 
                 V(k,k')\frac{\Delta(k')}{E(k')}, 
      \label{eq:eq1}
\end{equation}
where $V(k,k')$ is the bare momentum-space NN interaction in the 
$^1S_0$ channel, and $E(k)$ is the quasiparticle energy given by 
$E(k)=\sqrt{(\epsilon(k)-\epsilon(k_F))^2+\Delta(k)^2}$, where 
$\epsilon(k)$ is the single-particle energy of a neutron with 
momentum $k$, and $k_F$ is the Fermi momentum.  
Medium effects should 
be included in $\epsilon(k)$, but we will use free single-particle 
energies $\epsilon(k)=k^{2}/2m$, where $m$ is the neutron rest mass,  
to avoid unnecessary complications.  
And in neutron matter, at least at the densities 
considered here,  Brueckner-type calculations \cite{elg96} 
indicate that in-medium single-particle energies do not 
differ much from the free ones. 
The energy gap is defined as $\Delta_F\equiv \Delta(k_F)$.  
Eq.\ (\ref{eq:eq1}) can be 
solved by various techniques, some of which are described in 
Refs.\ \cite{elg96,khodel96}.  
In Fig.\ \ref{fig:fig1} we show 
the results for $\Delta_F$ obtained with the CD-Bonn potential (full line) 
\cite{mach96},  
the Nijmegen I and Nijmegen II potentials (long-dashed line and 
short-dashed line, respectively) \cite{nijm94}. 
The results are virtually identical, with the maximum value 
of the gap varying from 2.98 MeV for the Nijmegen I potential to 3.05 MeV 
for the Nijmegen II potential.  The same insensitivity of the energy gap 
with respect to 
the choice of NN interaction was found in Refs.\ 
\cite{baldo90,elg96,khodel96}. 
We will now discuss how these results can be understood 
from the properties of the NN interaction in the $^1S_0$ channel.

A characteristic feature of $^1S_0$ NN scattering is the large, negative 
scattering length, indicating the presence of  
a virtual bound state at $\approx$ 140 keV scattering energy.  
This state shows up as a pole in the NN $T$-matrix, 
which then  can be written in separable form, 
and this implies that the NN interaction itself to a good approximation is 
rank-one separable near this pole \cite{brown76}.   
Thus, at low energies we can write 
\begin{equation}
       V(k,k')=\lambda v(k)v(k'),
       \label{eq:eq2}
\end{equation}
where $\lambda$ is a constant.  Then it is easily seen from 
Eq.\ (\ref{eq:eq1}) that the gap function can be written as $\Delta_F v(k)$, 
where $\Delta_F$ is the energy gap.  Inserting this form of 
$\Delta(k)$ into Eq.\ (\ref{eq:eq1}) one obtains 
\begin{equation}
      1=-\frac{1}{\pi}\int_{0}^{\infty}dk'k'^2\frac{\lambda v^2(k')}{E(k')}, 
      \label{eq:eq3}
\end{equation}
which shows that the energy gap $\Delta_F$ is 
determined by the diagonal elements $\lambda v^2(k)$ of the NN interaction.  
The crucial point is that in scattering theory it can be shown that 
the inverse scattering problem, that is, the determination of a 
two-particle potential from the knowledge of the phase shifts at all 
energies, is exactly, and uniquely, solvable for rank-one 
separable potentials \cite{brown76,chadan92}.  Following the notation 
of Ref.\ \cite{brown76} we have 
\begin{equation}
       \lambda v^2(k)=-\frac{k^2+\kappa_B^2}{k^2}
                       \frac{\sin \delta(k)}{k}e^{-\alpha(k)},
       \label{eq:eq4}
\end{equation}
for an attractive potential with a bound state at energy $E=-\kappa_B^2$. 
In our case we take  $\kappa_B\approx 0$.    
Here $\delta(k)$ is the $^1S_0$ phase shift as a function of momentum $k$, 
while $\alpha(k)$ is given by a principle value integral: 
\begin{equation}
       \alpha(k)=\frac{1}{\pi}{\rm P}\int_{-\infty}^{+\infty}dk'
                 \frac{\delta(k')}{k'-k},
       \label{eq:eq5}
\end{equation}
where the phase shifts are extended to negative momenta through 
$\delta(-k)=-\delta(k)$.  Eqs.\ (\ref{eq:eq4}) and (\ref{eq:eq5}) 
can also be rewritten in terms of the Jost function \cite{chadan92}  
as done in Ref.\ \cite{kk97}.

 From this discussion we see that $\lambda v^2(k)$, and therefore also 
the energy gap $\Delta_F$, is completely determined by the $^1S_0$ 
phase shifts.  However, there are two obvious limitations on the 
practical validity of this statement.  First of all, the separable 
approximation can only be expected to be good at low energies, near the 
pole in the $T$-matrix.  Secondly, we see from Eq.\ (\ref{eq:eq5}) that 
knowledge of the phase shifts $\delta(k)$ at all energies is required.  
This is, of course, impossible, and most phase shift 
analyses stop at a laboratory energy $E_{\rm lab}=350$ MeV.  
Strictly speaking, the rank-one separable  approximation to the 
$^1S_0$ interaction breaks down already where the 
$^1S_0$ phase shift changes sign from positive to negative at 
$E_{\rm lab}\approx 248$ MeV, corresponding to a single-particle momentum 
of $k\approx1.73\;{\rm fm}^{-1}$.  However, at low values of $k_F$, knowledge 
of $v(k)$ up to this value of $k$ may actually be enough to determine 
the value of $\Delta_F$, as the integrand in Eq.\ (\ref{eq:eq5}) is 
strongly peaked around $k_F$.  We therefore found it worthwhile to 
try to calculate the energy gap directly from the $^1S_0$ phase shifts 
using Eqs. (\ref{eq:eq3})-(\ref{eq:eq5}). 
A possible improvement to the rank-one
separable approach for potentials which change sign is discussed by
Kwong and K\"ohler \cite{kk97}. 

The input in our calculation is the $^1S_0$ phase shifts taken from  
the recent Nijmegen phase shift analysis \cite{nijm93}. 
We then evaluated $\lambda v^2(k)$ from Eqs. (\ref{eq:eq4}) and 
(\ref{eq:eq5}), using methods described in Ref.\ \cite{davies91} to 
evaluate the principle value integral in Eq.\ (\ref{eq:eq5}). 
Finally, we evaluated the energy gap $\Delta_F$ for various values 
of $k_F$ by solving Eq.\ (\ref{eq:eq3}).  
Numerically the integral on the right-hand side of this equation 
depended very weakly on the momentum structure of $\Delta(k)$, so 
in our calculations we could take $\Delta(k)\approx \Delta_F$ in 
Eq.\ (\ref{eq:eq3}), and thus it became an algebraic equation 
for the energy gap $\Delta_F$.  
The resulting energy gap is plotted in Fig.\ 
\ref{fig:fig2} (dashed line) together with the gap obtained with the 
CD-Bonn potential (full line). 
As the reader can see, the agreement 
between the direct calculation from the phase shifts and the CD-Bonn 
calculation of $\Delta_F$ is very good, even 
at densities as high as $k_F=1.4\;{\rm fm}^{-1}$.  The energy gap 
is to a great extent determined by the available $^1S_0$ phase shifts.  
This can also be understood from the fact that for a rank-one separable
potential, the equations for the scattering state and the pair 
state become identical, as also pointed out by Carlson et al.\ 
\cite{carlson97}.
In the same figure we also report the results (dot-dashed line) 
obtained using the effective range approximation to the phase shifts: 
\begin{equation}
       k\cot \delta(k)=-\frac{1}{a_0}+\frac{1}{2}r_0 k^2,
       \label{eq:eq6}
\end{equation}
where $a_0=-18.8\pm 0.3$ fm and $r_0=2.75\pm 0.11$ fm are the singlet 
neutron-neutron scattering length and effective range, respectively.  
In this case an analytic expression can be obtained for $\lambda v^2(k)$, as 
shown in Ref.\ \cite{chadan92}:
\begin{equation}
       \lambda v^2(k)=-\frac{1}{\sqrt{k^2+
                       \frac{r_0^2}{4}(k^2+\alpha^2)^2}}
                       \sqrt{\frac{k^2+\beta_2^2}{k^2-\beta_1^2}},
       \label{eq:eq7}
\end{equation}
with $\alpha^2=-2/a_0 r_0$, and $\beta_1\approx-0.0498\;{\rm fm}^{-1}$, 
and $\beta_2\approx 0.777\;{\rm fm}^{-1}$ are the two roots of the 
quadratic equation 
\begin{equation}
   \beta^2-\frac{2}{r_0}\beta-\alpha^2=0.
   \label{eq:eq8}
\end{equation}  
The phase shifts using this approximation are positive at all energies, 
and this is reflected in Eq.\ (\ref{eq:eq7}) where $\lambda v^2(k)$ 
is attractive for all $k$.  From Fig.\ \ref{fig:fig1} we see that 
below $k_F=0.5\;{\rm fm}^{-1}$ the energy gap can with reasonable 
accuracy be calculated with the interaction obtained directly from 
the effective range approximation.  
One can therefore say that 
at densities below $k_F=0.5\;{\rm fm}^{-1}$, and at the crudest level 
of sophistication in many-body theory,  the superfluid properties 
of neutron matter are determined by just two parameters, namely 
the free-space scattering length and effective range. At such densities,
more complicated many-body terms are also less important.
Also interesting is the fact that the phase shifts predict the position 
of the first zero of $\Delta(k)$ in momentum space, since we see from 
Eq.\ (\ref{eq:eq4}) that $\Delta(k)=\Delta_F v(k)=0$ first for $\delta(k)=0$, 
which occurs at $E_{\rm lab}\approx 248$ MeV (pp scattering) 
corresponding to $k\approx 
1.73\;{\rm fm}^{-1}$.  This is in good agreement with the results of 
Khodel et al.\ \cite{khodel96}.  In Ref.\ \cite{khodel96} it is 
also shown that this first zero of the gap function determines the 
Fermi momentum at which $\Delta_F=0$.  Our results therefore indicate 
that this Fermi momentum is in fact given by the energy at which 
the $^1S_0$ phase shifts become negative. 

The calculation of the $^1S_0$ gap in symmetric nuclear matter is  
closely related to the one for neutron matter.  In fact, with 
charge-independent forces, like the older Bonn potentials, and 
free single--particle energies one would, of course, obtain 
exactly the same results.  However, the new potentials 
on the market are charge-dependent, in order to achieve high quality fits  
to both np and pp scattering data, and therefore we must solve three 
coupled gap equations for neutron-neutron (nn), proton-proton (pp), 
and neutron-proton (np) pairing \cite{good79}:
\begin{equation}
   \Delta_i(k)=-\frac{1}{\pi}\int_0^{\infty}dk'k'^2 V_i(k,k')
   \frac{\Delta_i(k')}{E(k')},
\label{eq:eq9}
\end{equation}
where $i$=nn, pp and np, and the quasiparticle energy is still given by 
$E(k)=\sqrt{(\epsilon(k)-\epsilon(k_F))^2+\Delta(k)^2}$, but the energy 
gap is now given by 
\begin{equation}
    \Delta(k)^2=\Delta_{\rm nn}(k)^2+\Delta_{\rm pp}(k)^2+\Delta_{\rm np}(k)^2.
    \label{eq:eq10}
\end{equation}
Thus the equations are coupled through their common energy denominator.  
The $^1S_0$ pp and nn interactions are very nearly identical, so  
the set of equations above can be reduced to two: one for the nn 
(or pp) gap and one for 
the np gap.  Solving these equations, both with the CD-Bonn potential and 
with the phase shift approximations we get the results shown in Fig.\ 
\ref{fig:fig3}.  For comparison we have in the same figure plotted the 
results for pure neutron matter with the CD-Bonn potential (dashed line). 
 From the figure it is clear that the phase shift approximation works well 
also in this case, and that the gap in symmetric matter is not very different 
from the gap in neutron matter.   As could be expected, the results 
are very close to those obtained earlier with charge-independent 
interactions \cite{baldo90,chen93,tak93,elg96}. 


In summary, we have shown that in infinite neutron and nuclear matter, 
owing to the near rank-one separability of the NN interaction in 
the $^1S_0$ partial wave,  
we are able to compute the $^1S_0$ pairing gap directly from the NN 
phase shifts. This explains why all NN potentials which fit 
the scattering data result in almost identical $^1S_0$ pairing gaps.  
Our findings conform with the conclusions of Khodel et al. \cite{khodel96} 
and Carlson et al. \cite{carlson97}: 
The virtual bound state in $^1S_0$ NN scattering determines  
the features of nucleon pairing in that partial wave.    
This is the main result of this work. 
However, it should be mentioned that this result    
is not likely to survive in a more refined calculation, for instance 
if one includes density and spin-density fluctuations in the 
effective pairing interaction like in e.g., Refs.\ \cite{wam93,schulze96}.  
Other partial waves will then be involved, and the simple arguments 
employed here will no longer apply.  Our reasoning here
applies also only
to a partial wave with bound or virtual bound state, that is, where  
$T$-matrix has a pole, and we have
neglected the fact that the phase shifts become negative at higher
energies.  
As a curiosity, we have found that at Fermi momenta below
$0.5$ fm$^{-1}$ the pairing gap is even determined by two parameters 
only, the effective range and the scattering length.
Also, we have pointed out that since the new NN interactions are charge 
dependent, one has to consider three coupled gap equations for 
$^1S_0$ pairing in nuclear matter.  The final result though is very 
nearly the same as what one obtains with charge-independent interactions.


We are much indebted to  B.\ V.\ Carlson, 
J.\ W.\ Clark, and E.\ Osnes for many 
valuable comments and discussions.  
  
\begin{references}
\bibitem{petra95} C.\ J.\ Pethick and D.\ G.\ Ravenhall, Annu.\ Rev.\ 
Nucl.\ Part.\ Sci.\ {\bf 45}, 429 (1995).
\bibitem{mulshe93} A.\ C.\ M\"{u}ller and B.\ M.\ Sherril, Annu.\ Rev.\ Nucl.\ 
Part.\ Sci.\ {\bf 43}, 529 (1993).
\bibitem{riis94} K.\ Riisager, Rev.\ Mod.\ Phys.\ {\bf 66}, 1105 (1994).
\bibitem{baldo90} M.\ Baldo, J.\ Cugnon, A.\ Lejeune, and U.\ Lombardo, 
Nucl.\ Phys.\ A {\bf 515}, 409 (1990).
\bibitem{chen93} J.\ M.\ C.\ Chen, J.\ W.\ Clark, R.\ D.\ Dav\'{e}, and 
V.\ V.\ Khodel, Nucl.\ Phys.\ A {\bf 555}, 59 (1993).
\bibitem{tak93} T.\ Takatsuka and R.\ Tamagaki, Prog.\ Theor.\ Phys.\ 
Suppl.\ {\bf 112}, 27 (1993).
\bibitem{elg96} \O.\ Elgar\o y, L.\ Engvik, M.\ Hjorth-Jensen, and E.\ Osnes, 
Nucl.\ Phys.\ A {\bf 604}, 466 (1996).
\bibitem{khodel96} V.\ A.\ Khodel, V.\ V.\ Khodel, and J.\ W.\ Clark, 
Nucl.\ Phys.\ A {\bf 598}, 390 (1996).
\bibitem{ring90} H.\ Kucharek and P.\ Ring, Z.\ Phys. {\bf 339}, 23 (1990).
\bibitem{guim96} F.\ B.\ Guimar\~{a}es, B.\ V.\ Carlson, and T.\ Frederico, 
Phys.\ Rev.\ C {\bf 54}, 2385 (1996).
\bibitem{matera97} F.\ Matera, G.\ Fabbri, and A.\ Dellafiore, 
Phys.\ Rev.\ C {\bf 56}, 228 (1997).
\bibitem{clark76} J.\ W.\ Clark, C.\ G.\ K\"{a}llman, C.\-H.\ Yang, and 
D.\ A.\ Chakkalakal, Phys.\ Lett.\ B {\bf 61}, 331 (1976).
\bibitem{wam93} J.\ Wambach, T.\ L.\ Ainsworth, and D.\ Pines, Nucl.\ Phys.\ 
A {\bf 555}, 128 (1993).
\bibitem{schulze96} H.\-J.\ Schulze, J.\ Cugnon, A.\ Lejeune, M.\ Baldo, 
and U.\ Lombardo, Phys.\ Lett.\ B {\bf 375}, 1 (1996).
\bibitem{es60} V.\ J.\ Emery and A.\ M.\ Sessler,  Phys.\ Rev.\ {\bf 119},
248 (1960) and references therein.
\bibitem{hoffberg70} M.\ Hoffberg, A.\ E.\ Glassgold, R.\ W.\ Richardson, 
and M.\ Ruderman, Phys.\ Rev.\ Lett.\ {\bf 24}, 775 (1970).
\bibitem{carlson97} B.\ V.\ Carlson, T.\ Frederico, 
 and F.\ B.\ Guimar\~{a}es, nucl-th/9706071, submitted to Phys.\ Rev.\ C. 
\bibitem{mach96} R.\ Machleidt, F.\ Sammarruca, and Y.\ Song, Phys.\ Rev.\ C 
{\bf 53}, 1483 (1996).
\bibitem{nijm94} V.\ G.\ J.\ Stoks, R.\ A.\ M.\ Klomp, C.\ P.\ F.\ Terheggen, 
and J.\ J.\ de Swart, Phys.\ Rev.\ C {\bf 49}, 2950 (1994);
NN-Online facility, (URL: http://NN-OnLine.sci.kun.nl).
\bibitem{brown76} G.\ E.\ Brown and A.\ D.\ Jackson, {\it The Nucleon-Nucleon 
Interaction} (North-Holland, Amsterdam, 1976).
\bibitem{chadan92} K.\ Chadan and P.\ C.\ Sabatier, {\it Inverse Problems in 
Quantum Scattering Theory}, 2nd ed.\ (Springer, New York, 1992).
\bibitem{kk97} N.\ H.\ Kwong and H.\ S.\ K\"ohler, Phys.\ Rev.\ C {\bf 55},
1650 (1997).
\bibitem{nijm93} V.\ G.\ J.\ Stoks, R.\ A.\ M.\ Klomp, 
M.\ G.\ M.\ Rentmeester, and 
J.\ J.\ de Swart, Phys.\ Rev.\ C {\bf 48}, 792 (1993). 
\bibitem{davies91} K.\ T.\ R.\ Davies, G.\ D.\ White, and R.\ W.\ Davies, 
Nucl.\ Phys.\ A {\bf 524}, 743 (1991).
\bibitem{good79} A.\ L.\ Goodman, Adv.\ Nucl.\ Phys.\ {\bf 11}, 263 (1979).
\end{references}


\end{multicols}

\clearpage

\begin{figure}
    
%\documentstyle[pst-plot]{article}

%\begin{document}
\pagestyle{empty}
\begin{figure}[htbp]
\setlength{\unitlength}{1cm}
\begin{center}
\setlength{\unitlength}{1cm}
\thicklines
\Cartesian(1.6cm,3cm)
\pspicture(0,0)(10,7)
\psline[linewidth=1pt]{->}(0.5,0.0)(0.5,5.0)
\psline[linewidth=1pt]{-}(0.4,0.0)(0.5,0.0)
\psline[linewidth=1pt]{-}(0.4,1)(0.5,1)
\psline[linewidth=1pt]{-}(0.4,2)(0.5,2)
\psline[linewidth=1pt]{-}(0.4,3)(0.5,3)
\psline[linewidth=1pt]{-}(0.4,4)(0.5,4)
\uput[0](0,0.0){0}
\uput[0](0,1.0){1}
\uput[0](0,2.0){2}
\uput[0](0,3.0){3}
\uput[0](0,4.0){4}
\uput[0](-0.5,5){Energy (MeV)}
%expt
\psline{-}(1,0.0)(2,0.0) 
\psline{-}(1,0.1)(2,0.1) 
\psline{-}(1,0.32)(2,0.32) 
\psline{-}(1,0.44)(2,0.44) 
\psline{-}(1,0.89)(2,0.89) 
\psline{-}(1,1.53)(2,1.53) 
\psline{-}(1,1.95)(2,1.95) 
\psline{-}(1,2.26)(2,2.26) 
\psline{-}(1,2.76)(2,2.76) 
\psline{-}(1,3.99)(2,3.99) 
\uput[0](2.5,0.0){\footnotesize{$(2^{+}$)}}
\uput[0](2.5,0.1){\footnotesize{$(4^{+}$})}
\uput[0](2.5,0.32){\footnotesize{$(5^{+}$})}
\uput[0](2.5,0.44){\footnotesize{$(6^{+}$})}
\uput[0](2.5,0.89){\footnotesize{$(7^{+}$})}
\uput[0](2.5,1.53){\footnotesize{$(8^{+}$})}
\uput[0](2.5,2.26){\footnotesize{$(10^{+}$})}
%theory1
\psline{-}(5,0.0)(6,0.0) 
\psline{-}(5,0.14)(6,0.14) 
\psline{-}(5,0.19)(6,0.19) 
\psline{-}(5,0.39)(6,0.39) 
\psline{-}(5,0.54)(6,0.54) 
\psline{-}(5,0.66)(6,0.66) 
\psline{-}(5,1.34)(6,1.34) 
\psline{-}(5,1.78)(6,1.78) 
\psline{-}(5,2.34)(6,2.34) 
\psline{-}(5,2.57)(6,2.57) 
\psline{-}(5,3.1)(6,3.1) 
\psline{-}(5,3.6)(6,3.6) 
\psline{-}(5,4.2)(6,4.2) 
\uput[0](6.5,0.0){\footnotesize{$2^{+}$}}
\uput[0](6.5,0.14){\footnotesize{$1^{+}$}}
\uput[0](6.5,0.25){\footnotesize{$4^{+}$}}
\uput[0](6.5,0.39){\footnotesize{$3^{+}$}}
\uput[0](6.5,0.54){\footnotesize{$5^{+}$}}
\uput[0](6.5,0.66){\footnotesize{$6^{+}$}}
\uput[0](6.5,1.34){\footnotesize{$7^{+}$}}
\uput[0](6.5,1.78){\footnotesize{$8^{+}$}}
\uput[0](6.5,2.34){\footnotesize{$9^{+}$}}
\uput[0](6.5,2.57){\footnotesize{$10^{+}$}}
\uput[0](6.5,3.1){\footnotesize{$11^{+}$}}
\uput[0](6.5,3.6){\footnotesize{$12^{+}$}}
\uput[0](6.5,4.2){\footnotesize{$13^{+}$}}
%theory2
\psline{-}(8,0.0)(9,0.0) 
\psline{-}(8,0.14)(9,0.14) 
\psline{-}(8,0.24)(9,0.24) 
\psline{-}(8,0.40)(9,0.40) 
\psline{-}(8,0.46)(9,0.46) 
\psline{-}(8,0.74)(9,0.74) 
\psline{-}(8,1.1)(9,1.1) 
\psline{-}(8,1.54)(9,1.54) 
\psline{-}(8,1.86)(9,1.86) 
\psline{-}(8,2.23)(9,2.23) 
\psline{-}(8,2.59)(9,2.59) 
\psline{-}(8,3.86)(9,3.86) 
\psline{-}(8,4.32)(9,4.32) 
\uput[0](9.5,0.0){\footnotesize{$1^{+}$}}
\uput[0](9.5,0.14){\footnotesize{$2^{+}$}}
\uput[0](9.5,0.24){\footnotesize{$4^{+}$}}
\uput[0](9.5,0.40){\footnotesize{$3^{+}$}}
\uput[0](9.5,0.52){\footnotesize{$5^{+}$}}
\uput[0](9.5,0.74){\footnotesize{$6^{+}$}}
\uput[0](9.5,1.1){\footnotesize{$7^{+}$}}
\uput[0](9.5,1.54){\footnotesize{$8^{+}$}}
\uput[0](9.5,1.86){\footnotesize{$9^{+}$}}
\uput[0](9.5,2.23){\footnotesize{$10^{+}$}}
\uput[0](9.5,2.59){\footnotesize{$11^{+}$}}
\uput[0](9.5,3.86){\footnotesize{$12^{+}$}}
\uput[0](9.5,4.32){\footnotesize{$13^{+}$}}
\uput[0](1.5,-0.2){EXPT}
\uput[0](6,-0.2){$1d_{5/2}0g_{7/2}1d_{3/2}2s_{1/2}0h_{11/2}$}
\uput[0](9,-0.2){$1d_{5/2}0g_{7/2}1d_{3/2}2s_{1/2}$}

\endpspicture

 \end{center}
 \end{figure}



    \caption{$^1S_0$ energy gap in neutron matter with the CD-Bonn, 
             Nijmegen I and Nijmegen II potentials.}
    \label{fig:fig1}
\end{figure}

\begin{figure}
    % GNUPLOT: LaTeX picture with Postscript
\setlength{\unitlength}{0.1bp}
\special{!
%!PS-Adobe-2.0
%%Creator: gnuplot
%%DocumentFonts: Helvetica
%%BoundingBox: 50 50 770 554
%%Pages: (atend)
%%EndComments
/gnudict 40 dict def
gnudict begin
/Color false def
/Solid false def
/gnulinewidth 5.000 def
/vshift -33 def
/dl {10 mul} def
/hpt 31.5 def
/vpt 31.5 def
/M {moveto} bind def
/L {lineto} bind def
/R {rmoveto} bind def
/V {rlineto} bind def
/vpt2 vpt 2 mul def
/hpt2 hpt 2 mul def
/Lshow { currentpoint stroke M
  0 vshift R show } def
/Rshow { currentpoint stroke M
  dup stringwidth pop neg vshift R show } def
/Cshow { currentpoint stroke M
  dup stringwidth pop -2 div vshift R show } def
/DL { Color {setrgbcolor Solid {pop []} if 0 setdash }
 {pop pop pop Solid {pop []} if 0 setdash} ifelse } def
/BL { stroke gnulinewidth 2 mul setlinewidth } def
/AL { stroke gnulinewidth 2 div setlinewidth } def
/PL { stroke gnulinewidth setlinewidth } def
/LTb { BL [] 0 0 0 DL } def
/LTa { AL [1 dl 2 dl] 0 setdash 0 0 0 setrgbcolor } def
/LT0 { PL [] 0 1 0 DL } def
/LT1 { PL [4 dl 2 dl] 0 0 1 DL } def
/LT2 { PL [2 dl 3 dl] 1 0 0 DL } def
/LT3 { PL [1 dl 1.5 dl] 1 0 1 DL } def
/LT4 { PL [5 dl 2 dl 1 dl 2 dl] 0 1 1 DL } def
/LT5 { PL [4 dl 3 dl 1 dl 3 dl] 1 1 0 DL } def
/LT6 { PL [2 dl 2 dl 2 dl 4 dl] 0 0 0 DL } def
/LT7 { PL [2 dl 2 dl 2 dl 2 dl 2 dl 4 dl] 1 0.3 0 DL } def
/LT8 { PL [2 dl 2 dl 2 dl 2 dl 2 dl 2 dl 2 dl 4 dl] 0.5 0.5 0.5 DL } def
/P { stroke [] 0 setdash
  currentlinewidth 2 div sub M
  0 currentlinewidth V stroke } def
/D { stroke [] 0 setdash 2 copy vpt add M
  hpt neg vpt neg V hpt vpt neg V
  hpt vpt V hpt neg vpt V closepath stroke
  P } def
/A { stroke [] 0 setdash vpt sub M 0 vpt2 V
  currentpoint stroke M
  hpt neg vpt neg R hpt2 0 V stroke
  } def
/B { stroke [] 0 setdash 2 copy exch hpt sub exch vpt add M
  0 vpt2 neg V hpt2 0 V 0 vpt2 V
  hpt2 neg 0 V closepath stroke
  P } def
/C { stroke [] 0 setdash exch hpt sub exch vpt add M
  hpt2 vpt2 neg V currentpoint stroke M
  hpt2 neg 0 R hpt2 vpt2 V stroke } def
/T { stroke [] 0 setdash 2 copy vpt 1.12 mul add M
  hpt neg vpt -1.62 mul V
  hpt 2 mul 0 V
  hpt neg vpt 1.62 mul V closepath stroke
  P  } def
/S { 2 copy A C} def
end
%%EndProlog
}
\begin{picture}(3600,2160)(0,0)
\special{"
%%Page: 1 1
gnudict begin
gsave
50 50 translate
0.100 0.100 scale
0 setgray
/Helvetica findfont 100 scalefont setfont
newpath
-500.000000 -500.000000 translate
LTa
600 1490 M
2817 0 V
LTb
600 251 M
63 0 V
2754 0 R
-63 0 V
600 561 M
63 0 V
2754 0 R
-63 0 V
600 870 M
63 0 V
2754 0 R
-63 0 V
600 1180 M
63 0 V
2754 0 R
-63 0 V
600 1490 M
63 0 V
2754 0 R
-63 0 V
600 1799 M
63 0 V
2754 0 R
-63 0 V
600 2109 M
63 0 V
2754 0 R
-63 0 V
600 251 M
0 63 V
0 1795 R
0 -63 V
913 251 M
0 63 V
0 1795 R
0 -63 V
1226 251 M
0 63 V
0 1795 R
0 -63 V
1539 251 M
0 63 V
0 1795 R
0 -63 V
1852 251 M
0 63 V
0 1795 R
0 -63 V
2165 251 M
0 63 V
0 1795 R
0 -63 V
2478 251 M
0 63 V
0 1795 R
0 -63 V
2791 251 M
0 63 V
0 1795 R
0 -63 V
3104 251 M
0 63 V
0 1795 R
0 -63 V
3417 251 M
0 63 V
0 1795 R
0 -63 V
600 251 M
2817 0 V
0 1858 V
-2817 0 V
600 251 L
LT0
1114 1946 M
180 0 V
600 1480 M
31 -4 V
32 -3 V
31 -3 V
31 -3 V
31 -4 V
32 -3 V
31 -3 V
31 -4 V
32 -3 V
31 -3 V
31 -4 V
32 -3 V
31 -3 V
31 -3 V
32 -3 V
31 -3 V
31 -3 V
31 -3 V
32 -3 V
31 -3 V
31 -2 V
32 -3 V
31 -2 V
31 -3 V
32 -2 V
31 -3 V
31 -2 V
31 -3 V
32 -2 V
31 -3 V
31 -2 V
32 -2 V
31 -2 V
31 -3 V
32 -2 V
31 -2 V
31 -2 V
31 -1 V
32 -1 V
31 -1 V
31 -1 V
32 -1 V
31 0 V
31 -1 V
32 0 V
31 -1 V
31 0 V
31 -1 V
32 0 V
31 0 V
31 1 V
32 1 V
31 1 V
31 1 V
31 1 V
32 1 V
31 1 V
31 1 V
32 1 V
31 0 V
31 0 V
32 1 V
31 0 V
31 1 V
31 1 V
32 1 V
31 1 V
31 1 V
32 2 V
31 3 V
31 3 V
32 3 V
31 3 V
31 4 V
32 3 V
31 3 V
31 3 V
31 3 V
32 2 V
31 2 V
31 2 V
32 1 V
31 2 V
31 1 V
31 1 V
32 2 V
31 1 V
31 1 V
32 2 V
LT1
1114 1846 M
180 0 V
600 1499 M
31 4 V
32 4 V
31 5 V
31 4 V
31 5 V
32 5 V
31 5 V
31 4 V
32 5 V
31 6 V
31 5 V
32 6 V
31 6 V
31 7 V
32 6 V
31 6 V
31 7 V
31 6 V
32 6 V
31 7 V
31 6 V
32 6 V
31 6 V
31 6 V
32 7 V
31 6 V
31 7 V
31 7 V
32 7 V
31 7 V
31 8 V
32 7 V
31 7 V
31 8 V
32 7 V
31 8 V
31 6 V
31 6 V
32 6 V
31 6 V
31 6 V
32 5 V
31 6 V
31 5 V
32 6 V
31 5 V
31 5 V
31 4 V
32 4 V
31 4 V
31 2 V
32 1 V
31 2 V
31 2 V
31 3 V
32 4 V
31 4 V
31 6 V
32 6 V
31 7 V
31 9 V
32 10 V
31 9 V
31 9 V
31 9 V
32 9 V
31 7 V
31 7 V
32 6 V
31 5 V
31 5 V
32 5 V
31 5 V
31 5 V
32 5 V
31 6 V
31 5 V
31 6 V
32 5 V
31 6 V
31 6 V
32 5 V
31 6 V
31 5 V
31 5 V
32 4 V
31 5 V
31 4 V
32 4 V
LT2
1114 1746 M
180 0 V
600 1466 M
31 -7 V
32 -8 V
31 -8 V
31 -8 V
31 -9 V
32 -10 V
31 -11 V
31 -10 V
32 -12 V
31 -12 V
31 -15 V
32 -14 V
31 -14 V
31 -13 V
32 -13 V
31 -12 V
31 -11 V
31 -11 V
32 -10 V
31 -11 V
31 -9 V
32 -9 V
31 -10 V
31 -11 V
32 -11 V
31 -11 V
31 -13 V
31 -12 V
32 -14 V
31 -14 V
31 -14 V
32 -15 V
31 -15 V
31 -15 V
32 -15 V
31 -16 V
31 -15 V
31 -14 V
32 -15 V
31 -14 V
31 -14 V
32 -14 V
31 -15 V
31 -14 V
32 -14 V
31 -13 V
31 -14 V
31 -14 V
32 -13 V
31 -13 V
31 -12 V
32 -11 V
31 -12 V
31 -11 V
31 -12 V
32 -11 V
31 -11 V
31 -12 V
32 -11 V
31 -11 V
31 -12 V
32 -11 V
31 -11 V
31 -11 V
31 -10 V
32 -10 V
31 -10 V
31 -9 V
32 -9 V
31 -9 V
31 -9 V
32 -9 V
31 -9 V
31 -9 V
32 -9 V
31 -9 V
31 -9 V
31 -9 V
32 -10 V
31 -9 V
31 -9 V
32 -9 V
31 -9 V
31 -9 V
31 -9 V
32 -8 V
31 -8 V
31 -8 V
32 -8 V
LT3
1114 1646 M
180 0 V
600 1477 M
31 -4 V
32 -4 V
31 -4 V
31 -4 V
31 -3 V
32 -4 V
31 -3 V
31 -4 V
32 -3 V
31 -4 V
31 -3 V
32 -3 V
31 -3 V
31 -3 V
32 -3 V
31 -2 V
31 -3 V
31 -2 V
32 -2 V
31 -3 V
31 -2 V
32 -2 V
31 -2 V
31 -2 V
32 -2 V
31 -2 V
31 -2 V
31 -1 V
32 -2 V
31 -2 V
31 -2 V
32 -2 V
31 -2 V
31 -2 V
32 -1 V
31 -2 V
31 -2 V
31 -1 V
32 -1 V
31 -2 V
31 -1 V
32 -1 V
31 -2 V
31 -1 V
32 -1 V
31 -2 V
31 -1 V
31 -1 V
32 -1 V
31 -1 V
31 0 V
32 0 V
31 0 V
31 -1 V
31 0 V
32 -1 V
31 -1 V
31 -1 V
32 -2 V
31 -1 V
31 -2 V
32 -3 V
31 -2 V
31 -2 V
31 -2 V
32 -2 V
31 -2 V
31 -2 V
32 -2 V
31 -2 V
31 -2 V
32 -2 V
31 -2 V
31 -3 V
32 -2 V
31 -3 V
31 -3 V
31 -3 V
32 -3 V
31 -3 V
31 -3 V
32 -3 V
31 -3 V
31 -4 V
31 -2 V
32 -3 V
31 -3 V
31 -2 V
32 -1 V
stroke
grestore
end
showpage
}
\put(1054,1646){\makebox(0,0)[r]{$\mu_{\Lambda}$}}
\put(1054,1746){\makebox(0,0)[r]{$\mu_{\Sigma^-}$}}
\put(1054,1846){\makebox(0,0)[r]{$\mu_p$}}
\put(1054,1946){\makebox(0,0)[r]{$\mu_n$}}
\put(2008,51){\makebox(0,0){$n$ (fm$^{-3}$)}}
\put(100,1180){%
\special{ps: gsave currentpoint currentpoint translate
270 rotate neg exch neg exch translate}%
\makebox(0,0)[b]{\shortstack{$\mu_i$ (MeV)}}%
\special{ps: currentpoint grestore moveto}%
}
\put(3417,151){\makebox(0,0){1.2}}
\put(3104,151){\makebox(0,0){1.1}}
\put(2791,151){\makebox(0,0){1}}
\put(2478,151){\makebox(0,0){0.9}}
\put(2165,151){\makebox(0,0){0.8}}
\put(1852,151){\makebox(0,0){0.7}}
\put(1539,151){\makebox(0,0){0.6}}
\put(1226,151){\makebox(0,0){0.5}}
\put(913,151){\makebox(0,0){0.4}}
\put(600,151){\makebox(0,0){0.3}}
\put(540,2109){\makebox(0,0)[r]{100}}
\put(540,1799){\makebox(0,0)[r]{50}}
\put(540,1490){\makebox(0,0)[r]{0}}
\put(540,1180){\makebox(0,0)[r]{-50}}
\put(540,870){\makebox(0,0)[r]{-100}}
\put(540,561){\makebox(0,0)[r]{-150}}
\put(540,251){\makebox(0,0)[r]{-200}}
\end{picture}

    \caption{$^1S_0$ energy gap in neutron matter calculated with  
             the CD-Bonn potential compared with the direct calculation 
             from $^1S_0$ phase shifts.}
    \label{fig:fig2}
\end{figure} 

\begin{figure}
    % GNUPLOT: LaTeX picture with Postscript
\begingroup%
  \makeatletter%
  \newcommand{\GNUPLOTspecial}{%
    \@sanitize\catcode`\%=14\relax\special}%
  \setlength{\unitlength}{0.1bp}%
{\GNUPLOTspecial{!
%!PS-Adobe-2.0 EPSF-2.0
%%Title: fig3.tex
%%Creator: gnuplot 3.7 patchlevel 1
%%CreationDate: Sat Mar 10 16:08:37 2001
%%DocumentFonts: 
%%BoundingBox: 0 0 360 216
%%Orientation: Landscape
%%EndComments
/gnudict 256 dict def
gnudict begin
/Color false def
/Solid false def
/gnulinewidth 5.000 def
/userlinewidth gnulinewidth def
/vshift -33 def
/dl {10 mul} def
/hpt_ 31.5 def
/vpt_ 31.5 def
/hpt hpt_ def
/vpt vpt_ def
/M {moveto} bind def
/L {lineto} bind def
/R {rmoveto} bind def
/V {rlineto} bind def
/vpt2 vpt 2 mul def
/hpt2 hpt 2 mul def
/Lshow { currentpoint stroke M
  0 vshift R show } def
/Rshow { currentpoint stroke M
  dup stringwidth pop neg vshift R show } def
/Cshow { currentpoint stroke M
  dup stringwidth pop -2 div vshift R show } def
/UP { dup vpt_ mul /vpt exch def hpt_ mul /hpt exch def
  /hpt2 hpt 2 mul def /vpt2 vpt 2 mul def } def
/DL { Color {setrgbcolor Solid {pop []} if 0 setdash }
 {pop pop pop Solid {pop []} if 0 setdash} ifelse } def
/BL { stroke userlinewidth 2 mul setlinewidth } def
/AL { stroke userlinewidth 2 div setlinewidth } def
/UL { dup gnulinewidth mul /userlinewidth exch def
      10 mul /udl exch def } def
/PL { stroke userlinewidth setlinewidth } def
/LTb { BL [] 0 0 0 DL } def
/LTa { AL [1 udl mul 2 udl mul] 0 setdash 0 0 0 setrgbcolor } def
/LT0 { PL [] 1 0 0 DL } def
/LT1 { PL [4 dl 2 dl] 0 1 0 DL } def
/LT2 { PL [2 dl 3 dl] 0 0 1 DL } def
/LT3 { PL [1 dl 1.5 dl] 1 0 1 DL } def
/LT4 { PL [5 dl 2 dl 1 dl 2 dl] 0 1 1 DL } def
/LT5 { PL [4 dl 3 dl 1 dl 3 dl] 1 1 0 DL } def
/LT6 { PL [2 dl 2 dl 2 dl 4 dl] 0 0 0 DL } def
/LT7 { PL [2 dl 2 dl 2 dl 2 dl 2 dl 4 dl] 1 0.3 0 DL } def
/LT8 { PL [2 dl 2 dl 2 dl 2 dl 2 dl 2 dl 2 dl 4 dl] 0.5 0.5 0.5 DL } def
/Pnt { stroke [] 0 setdash
   gsave 1 setlinecap M 0 0 V stroke grestore } def
/Dia { stroke [] 0 setdash 2 copy vpt add M
  hpt neg vpt neg V hpt vpt neg V
  hpt vpt V hpt neg vpt V closepath stroke
  Pnt } def
/Pls { stroke [] 0 setdash vpt sub M 0 vpt2 V
  currentpoint stroke M
  hpt neg vpt neg R hpt2 0 V stroke
  } def
/Box { stroke [] 0 setdash 2 copy exch hpt sub exch vpt add M
  0 vpt2 neg V hpt2 0 V 0 vpt2 V
  hpt2 neg 0 V closepath stroke
  Pnt } def
/Crs { stroke [] 0 setdash exch hpt sub exch vpt add M
  hpt2 vpt2 neg V currentpoint stroke M
  hpt2 neg 0 R hpt2 vpt2 V stroke } def
/TriU { stroke [] 0 setdash 2 copy vpt 1.12 mul add M
  hpt neg vpt -1.62 mul V
  hpt 2 mul 0 V
  hpt neg vpt 1.62 mul V closepath stroke
  Pnt  } def
/Star { 2 copy Pls Crs } def
/BoxF { stroke [] 0 setdash exch hpt sub exch vpt add M
  0 vpt2 neg V  hpt2 0 V  0 vpt2 V
  hpt2 neg 0 V  closepath fill } def
/TriUF { stroke [] 0 setdash vpt 1.12 mul add M
  hpt neg vpt -1.62 mul V
  hpt 2 mul 0 V
  hpt neg vpt 1.62 mul V closepath fill } def
/TriD { stroke [] 0 setdash 2 copy vpt 1.12 mul sub M
  hpt neg vpt 1.62 mul V
  hpt 2 mul 0 V
  hpt neg vpt -1.62 mul V closepath stroke
  Pnt  } def
/TriDF { stroke [] 0 setdash vpt 1.12 mul sub M
  hpt neg vpt 1.62 mul V
  hpt 2 mul 0 V
  hpt neg vpt -1.62 mul V closepath fill} def
/DiaF { stroke [] 0 setdash vpt add M
  hpt neg vpt neg V hpt vpt neg V
  hpt vpt V hpt neg vpt V closepath fill } def
/Pent { stroke [] 0 setdash 2 copy gsave
  translate 0 hpt M 4 {72 rotate 0 hpt L} repeat
  closepath stroke grestore Pnt } def
/PentF { stroke [] 0 setdash gsave
  translate 0 hpt M 4 {72 rotate 0 hpt L} repeat
  closepath fill grestore } def
/Circle { stroke [] 0 setdash 2 copy
  hpt 0 360 arc stroke Pnt } def
/CircleF { stroke [] 0 setdash hpt 0 360 arc fill } def
/C0 { BL [] 0 setdash 2 copy moveto vpt 90 450  arc } bind def
/C1 { BL [] 0 setdash 2 copy        moveto
       2 copy  vpt 0 90 arc closepath fill
               vpt 0 360 arc closepath } bind def
/C2 { BL [] 0 setdash 2 copy moveto
       2 copy  vpt 90 180 arc closepath fill
               vpt 0 360 arc closepath } bind def
/C3 { BL [] 0 setdash 2 copy moveto
       2 copy  vpt 0 180 arc closepath fill
               vpt 0 360 arc closepath } bind def
/C4 { BL [] 0 setdash 2 copy moveto
       2 copy  vpt 180 270 arc closepath fill
               vpt 0 360 arc closepath } bind def
/C5 { BL [] 0 setdash 2 copy moveto
       2 copy  vpt 0 90 arc
       2 copy moveto
       2 copy  vpt 180 270 arc closepath fill
               vpt 0 360 arc } bind def
/C6 { BL [] 0 setdash 2 copy moveto
      2 copy  vpt 90 270 arc closepath fill
              vpt 0 360 arc closepath } bind def
/C7 { BL [] 0 setdash 2 copy moveto
      2 copy  vpt 0 270 arc closepath fill
              vpt 0 360 arc closepath } bind def
/C8 { BL [] 0 setdash 2 copy moveto
      2 copy vpt 270 360 arc closepath fill
              vpt 0 360 arc closepath } bind def
/C9 { BL [] 0 setdash 2 copy moveto
      2 copy  vpt 270 450 arc closepath fill
              vpt 0 360 arc closepath } bind def
/C10 { BL [] 0 setdash 2 copy 2 copy moveto vpt 270 360 arc closepath fill
       2 copy moveto
       2 copy vpt 90 180 arc closepath fill
               vpt 0 360 arc closepath } bind def
/C11 { BL [] 0 setdash 2 copy moveto
       2 copy  vpt 0 180 arc closepath fill
       2 copy moveto
       2 copy  vpt 270 360 arc closepath fill
               vpt 0 360 arc closepath } bind def
/C12 { BL [] 0 setdash 2 copy moveto
       2 copy  vpt 180 360 arc closepath fill
               vpt 0 360 arc closepath } bind def
/C13 { BL [] 0 setdash  2 copy moveto
       2 copy  vpt 0 90 arc closepath fill
       2 copy moveto
       2 copy  vpt 180 360 arc closepath fill
               vpt 0 360 arc closepath } bind def
/C14 { BL [] 0 setdash 2 copy moveto
       2 copy  vpt 90 360 arc closepath fill
               vpt 0 360 arc } bind def
/C15 { BL [] 0 setdash 2 copy vpt 0 360 arc closepath fill
               vpt 0 360 arc closepath } bind def
/Rec   { newpath 4 2 roll moveto 1 index 0 rlineto 0 exch rlineto
       neg 0 rlineto closepath } bind def
/Square { dup Rec } bind def
/Bsquare { vpt sub exch vpt sub exch vpt2 Square } bind def
/S0 { BL [] 0 setdash 2 copy moveto 0 vpt rlineto BL Bsquare } bind def
/S1 { BL [] 0 setdash 2 copy vpt Square fill Bsquare } bind def
/S2 { BL [] 0 setdash 2 copy exch vpt sub exch vpt Square fill Bsquare } bind def
/S3 { BL [] 0 setdash 2 copy exch vpt sub exch vpt2 vpt Rec fill Bsquare } bind def
/S4 { BL [] 0 setdash 2 copy exch vpt sub exch vpt sub vpt Square fill Bsquare } bind def
/S5 { BL [] 0 setdash 2 copy 2 copy vpt Square fill
       exch vpt sub exch vpt sub vpt Square fill Bsquare } bind def
/S6 { BL [] 0 setdash 2 copy exch vpt sub exch vpt sub vpt vpt2 Rec fill Bsquare } bind def
/S7 { BL [] 0 setdash 2 copy exch vpt sub exch vpt sub vpt vpt2 Rec fill
       2 copy vpt Square fill
       Bsquare } bind def
/S8 { BL [] 0 setdash 2 copy vpt sub vpt Square fill Bsquare } bind def
/S9 { BL [] 0 setdash 2 copy vpt sub vpt vpt2 Rec fill Bsquare } bind def
/S10 { BL [] 0 setdash 2 copy vpt sub vpt Square fill 2 copy exch vpt sub exch vpt Square fill
       Bsquare } bind def
/S11 { BL [] 0 setdash 2 copy vpt sub vpt Square fill 2 copy exch vpt sub exch vpt2 vpt Rec fill
       Bsquare } bind def
/S12 { BL [] 0 setdash 2 copy exch vpt sub exch vpt sub vpt2 vpt Rec fill Bsquare } bind def
/S13 { BL [] 0 setdash 2 copy exch vpt sub exch vpt sub vpt2 vpt Rec fill
       2 copy vpt Square fill Bsquare } bind def
/S14 { BL [] 0 setdash 2 copy exch vpt sub exch vpt sub vpt2 vpt Rec fill
       2 copy exch vpt sub exch vpt Square fill Bsquare } bind def
/S15 { BL [] 0 setdash 2 copy Bsquare fill Bsquare } bind def
/D0 { gsave translate 45 rotate 0 0 S0 stroke grestore } bind def
/D1 { gsave translate 45 rotate 0 0 S1 stroke grestore } bind def
/D2 { gsave translate 45 rotate 0 0 S2 stroke grestore } bind def
/D3 { gsave translate 45 rotate 0 0 S3 stroke grestore } bind def
/D4 { gsave translate 45 rotate 0 0 S4 stroke grestore } bind def
/D5 { gsave translate 45 rotate 0 0 S5 stroke grestore } bind def
/D6 { gsave translate 45 rotate 0 0 S6 stroke grestore } bind def
/D7 { gsave translate 45 rotate 0 0 S7 stroke grestore } bind def
/D8 { gsave translate 45 rotate 0 0 S8 stroke grestore } bind def
/D9 { gsave translate 45 rotate 0 0 S9 stroke grestore } bind def
/D10 { gsave translate 45 rotate 0 0 S10 stroke grestore } bind def
/D11 { gsave translate 45 rotate 0 0 S11 stroke grestore } bind def
/D12 { gsave translate 45 rotate 0 0 S12 stroke grestore } bind def
/D13 { gsave translate 45 rotate 0 0 S13 stroke grestore } bind def
/D14 { gsave translate 45 rotate 0 0 S14 stroke grestore } bind def
/D15 { gsave translate 45 rotate 0 0 S15 stroke grestore } bind def
/DiaE { stroke [] 0 setdash vpt add M
  hpt neg vpt neg V hpt vpt neg V
  hpt vpt V hpt neg vpt V closepath stroke } def
/BoxE { stroke [] 0 setdash exch hpt sub exch vpt add M
  0 vpt2 neg V hpt2 0 V 0 vpt2 V
  hpt2 neg 0 V closepath stroke } def
/TriUE { stroke [] 0 setdash vpt 1.12 mul add M
  hpt neg vpt -1.62 mul V
  hpt 2 mul 0 V
  hpt neg vpt 1.62 mul V closepath stroke } def
/TriDE { stroke [] 0 setdash vpt 1.12 mul sub M
  hpt neg vpt 1.62 mul V
  hpt 2 mul 0 V
  hpt neg vpt -1.62 mul V closepath stroke } def
/PentE { stroke [] 0 setdash gsave
  translate 0 hpt M 4 {72 rotate 0 hpt L} repeat
  closepath stroke grestore } def
/CircE { stroke [] 0 setdash 
  hpt 0 360 arc stroke } def
/Opaque { gsave closepath 1 setgray fill grestore 0 setgray closepath } def
/DiaW { stroke [] 0 setdash vpt add M
  hpt neg vpt neg V hpt vpt neg V
  hpt vpt V hpt neg vpt V Opaque stroke } def
/BoxW { stroke [] 0 setdash exch hpt sub exch vpt add M
  0 vpt2 neg V hpt2 0 V 0 vpt2 V
  hpt2 neg 0 V Opaque stroke } def
/TriUW { stroke [] 0 setdash vpt 1.12 mul add M
  hpt neg vpt -1.62 mul V
  hpt 2 mul 0 V
  hpt neg vpt 1.62 mul V Opaque stroke } def
/TriDW { stroke [] 0 setdash vpt 1.12 mul sub M
  hpt neg vpt 1.62 mul V
  hpt 2 mul 0 V
  hpt neg vpt -1.62 mul V Opaque stroke } def
/PentW { stroke [] 0 setdash gsave
  translate 0 hpt M 4 {72 rotate 0 hpt L} repeat
  Opaque stroke grestore } def
/CircW { stroke [] 0 setdash 
  hpt 0 360 arc Opaque stroke } def
/BoxFill { gsave Rec 1 setgray fill grestore } def
end
%%EndProlog
}}%
\begin{picture}(3600,2160)(0,0)%
{\GNUPLOTspecial{"
gnudict begin
gsave
0 0 translate
0.100 0.100 scale
0 setgray
newpath
1.000 UL
LTb
350 300 M
63 0 V
3037 0 R
-63 0 V
350 593 M
63 0 V
3037 0 R
-63 0 V
350 887 M
63 0 V
3037 0 R
-63 0 V
350 1180 M
63 0 V
3037 0 R
-63 0 V
350 1473 M
63 0 V
3037 0 R
-63 0 V
350 1767 M
63 0 V
3037 0 R
-63 0 V
350 2060 M
63 0 V
3037 0 R
-63 0 V
350 300 M
0 63 V
0 1697 R
0 -63 V
970 300 M
0 63 V
0 1697 R
0 -63 V
1590 300 M
0 63 V
0 1697 R
0 -63 V
2210 300 M
0 63 V
0 1697 R
0 -63 V
2830 300 M
0 63 V
0 1697 R
0 -63 V
3450 300 M
0 63 V
0 1697 R
0 -63 V
1.000 UL
LTb
350 300 M
3100 0 V
0 1760 V
-3100 0 V
350 300 L
1.000 UP
1.000 UL
LT0
350 593 Pls
725 837 Pls
728 655 Pls
730 604 Pls
733 600 Pls
736 653 Pls
738 599 Pls
741 624 Pls
744 627 Pls
746 638 Pls
749 605 Pls
751 672 Pls
754 639 Pls
757 621 Pls
759 681 Pls
762 684 Pls
764 824 Pls
767 701 Pls
772 747 Pls
777 789 Pls
783 1019 Pls
1043 1213 Pls
1046 966 Pls
1048 883 Pls
1051 762 Pls
1054 746 Pls
1056 737 Pls
1059 693 Pls
1061 618 Pls
1064 684 Pls
1067 607 Pls
1069 600 Pls
1072 603 Pls
1074 558 Pls
1077 592 Pls
1080 547 Pls
1082 542 Pls
1085 546 Pls
1088 546 Pls
1090 548 Pls
1093 549 Pls
1095 547 Pls
1098 565 Pls
1101 542 Pls
1103 588 Pls
1106 578 Pls
1108 574 Pls
1111 616 Pls
1114 592 Pls
1116 641 Pls
1119 620 Pls
1121 675 Pls
1124 701 Pls
1127 699 Pls
1129 763 Pls
1132 734 Pls
1134 739 Pls
1137 859 Pls
1140 843 Pls
1142 870 Pls
1145 1023 Pls
1147 917 Pls
1150 1021 Pls
1153 996 Pls
1155 1027 Pls
1158 1462 Pls
1161 1199 Pls
1163 1296 Pls
1168 1302 Pls
1171 1478 Pls
1288 1703 Pls
1291 1533 Pls
1293 1450 Pls
1296 1393 Pls
1299 1319 Pls
1301 1231 Pls
1304 1203 Pls
1306 1094 Pls
1309 1060 Pls
1312 1152 Pls
1314 978 Pls
1317 1055 Pls
1319 896 Pls
1322 815 Pls
1325 853 Pls
1327 802 Pls
1330 753 Pls
1333 784 Pls
1335 768 Pls
1338 716 Pls
1340 700 Pls
1343 693 Pls
1346 639 Pls
1348 644 Pls
1351 641 Pls
1353 601 Pls
1356 607 Pls
1359 598 Pls
1361 584 Pls
1364 588 Pls
1366 566 Pls
1369 553 Pls
1372 570 Pls
1374 550 Pls
1377 544 Pls
1379 549 Pls
1382 538 Pls
1385 539 Pls
1387 547 Pls
1390 540 Pls
1392 540 Pls
1395 548 Pls
1398 541 Pls
1400 556 Pls
1403 559 Pls
1405 560 Pls
1408 579 Pls
1411 575 Pls
1413 585 Pls
1416 602 Pls
1419 609 Pls
1421 602 Pls
1424 633 Pls
1426 651 Pls
1429 648 Pls
1432 676 Pls
1434 686 Pls
1437 700 Pls
1439 722 Pls
1442 729 Pls
1445 768 Pls
1447 792 Pls
1450 790 Pls
1452 818 Pls
1455 832 Pls
1458 921 Pls
1460 895 Pls
1463 912 Pls
1465 941 Pls
1468 995 Pls
1471 1023 Pls
1473 1043 Pls
1476 1049 Pls
1478 1147 Pls
1481 1251 Pls
1484 1128 Pls
1486 1252 Pls
1489 1235 Pls
1491 1248 Pls
1494 1236 Pls
1497 1225 Pls
1499 1214 Pls
1502 1174 Pls
1505 1145 Pls
1507 1127 Pls
1510 1103 Pls
1512 1060 Pls
1515 1045 Pls
1518 1088 Pls
1520 1034 Pls
1523 1023 Pls
1525 1013 Pls
1528 1001 Pls
1531 977 Pls
1533 928 Pls
1536 980 Pls
1538 953 Pls
1541 881 Pls
1544 907 Pls
1546 914 Pls
1549 860 Pls
1551 829 Pls
1554 806 Pls
1557 798 Pls
1559 777 Pls
1562 768 Pls
1564 749 Pls
1567 723 Pls
1570 729 Pls
1572 742 Pls
1575 695 Pls
1577 686 Pls
1580 700 Pls
1583 679 Pls
1585 643 Pls
1588 663 Pls
1591 657 Pls
1593 638 Pls
1596 625 Pls
1598 633 Pls
1601 627 Pls
1604 607 Pls
1606 606 Pls
1609 613 Pls
1611 594 Pls
1614 583 Pls
1617 590 Pls
1619 593 Pls
1622 572 Pls
1624 580 Pls
1627 583 Pls
1630 578 Pls
1632 566 Pls
1635 587 Pls
1637 583 Pls
1640 565 Pls
1643 585 Pls
1645 584 Pls
1648 575 Pls
1650 581 Pls
1653 592 Pls
1656 591 Pls
1658 598 Pls
1661 602 Pls
1663 608 Pls
1666 610 Pls
1669 612 Pls
1671 624 Pls
1674 634 Pls
1677 641 Pls
1679 642 Pls
1682 661 Pls
1684 659 Pls
1687 673 Pls
1690 677 Pls
1692 687 Pls
1695 705 Pls
1697 711 Pls
1700 715 Pls
1703 726 Pls
1705 733 Pls
1708 766 Pls
1710 761 Pls
1713 731 Pls
1716 761 Pls
1718 824 Pls
1721 799 Pls
1723 735 Pls
1726 778 Pls
1729 864 Pls
1731 835 Pls
1734 736 Pls
1736 774 Pls
1739 909 Pls
1742 850 Pls
1744 741 Pls
1747 763 Pls
1749 929 Pls
1752 832 Pls
1755 743 Pls
1757 746 Pls
1760 859 Pls
1763 819 Pls
1765 750 Pls
1768 734 Pls
1770 798 Pls
1773 788 Pls
1776 727 Pls
1778 710 Pls
1781 753 Pls
1783 737 Pls
1786 706 Pls
1789 688 Pls
1791 722 Pls
1794 705 Pls
1796 677 Pls
1799 679 Pls
1802 695 Pls
1804 673 Pls
1807 668 Pls
1809 668 Pls
1812 685 Pls
1815 663 Pls
1817 660 Pls
1820 666 Pls
1822 671 Pls
1825 659 Pls
1828 665 Pls
1830 665 Pls
1833 675 Pls
1835 661 Pls
1838 673 Pls
1841 668 Pls
1843 673 Pls
1846 671 Pls
1849 673 Pls
1851 674 Pls
1854 678 Pls
1856 684 Pls
1859 676 Pls
1862 675 Pls
1864 681 Pls
1867 689 Pls
1869 680 Pls
1872 685 Pls
1875 693 Pls
1877 693 Pls
1880 694 Pls
1882 691 Pls
1885 702 Pls
1888 710 Pls
1890 707 Pls
1893 715 Pls
1895 713 Pls
1898 719 Pls
1901 719 Pls
1903 735 Pls
1906 733 Pls
1908 720 Pls
1911 749 Pls
1914 760 Pls
1916 742 Pls
1919 738 Pls
1921 758 Pls
1924 748 Pls
1927 776 Pls
1929 761 Pls
1932 747 Pls
1935 765 Pls
1937 780 Pls
1940 767 Pls
1942 755 Pls
1945 767 Pls
1948 770 Pls
1950 797 Pls
1953 760 Pls
1955 767 Pls
1958 782 Pls
1961 806 Pls
1963 793 Pls
1966 775 Pls
1968 793 Pls
1971 813 Pls
1974 823 Pls
1976 786 Pls
1979 808 Pls
1981 815 Pls
1984 846 Pls
1987 803 Pls
1989 807 Pls
1992 806 Pls
1994 833 Pls
1997 809 Pls
2000 792 Pls
2002 789 Pls
2005 810 Pls
2007 813 Pls
2010 776 Pls
2013 775 Pls
2015 796 Pls
2018 806 Pls
2021 772 Pls
2023 774 Pls
2026 797 Pls
2028 804 Pls
2031 777 Pls
2034 781 Pls
2036 797 Pls
2039 807 Pls
2041 786 Pls
2044 794 Pls
2047 807 Pls
2049 813 Pls
2052 803 Pls
2054 808 Pls
2057 817 Pls
2060 829 Pls
2062 819 Pls
2065 821 Pls
2067 830 Pls
2070 845 Pls
2073 835 Pls
2075 836 Pls
2078 842 Pls
2080 860 Pls
2083 848 Pls
2086 856 Pls
2088 868 Pls
2091 870 Pls
2093 863 Pls
2096 869 Pls
2099 889 Pls
2101 888 Pls
2104 873 Pls
2106 876 Pls
2109 895 Pls
2112 883 Pls
2114 880 Pls
2117 891 Pls
2120 885 Pls
2122 891 Pls
2125 902 Pls
2127 893 Pls
2130 896 Pls
2133 901 Pls
2135 904 Pls
2138 920 Pls
2140 904 Pls
2143 912 Pls
2146 926 Pls
2148 927 Pls
2151 921 Pls
2153 931 Pls
2156 936 Pls
2159 944 Pls
2161 938 Pls
2164 945 Pls
2166 948 Pls
2169 950 Pls
2172 949 Pls
2174 954 Pls
2177 940 Pls
2179 948 Pls
2182 962 Pls
2185 943 Pls
2187 953 Pls
2190 964 Pls
2193 973 Pls
2195 967 Pls
2198 976 Pls
2200 984 Pls
2203 1000 Pls
2206 998 Pls
2208 990 Pls
2211 1013 Pls
2213 1024 Pls
2216 1012 Pls
2219 1016 Pls
2221 1018 Pls
2224 1047 Pls
2226 1040 Pls
2229 968 Pls
2232 1038 Pls
2234 1080 Pls
2237 983 Pls
2239 1014 Pls
2242 1018 Pls
2245 1013 Pls
2247 1040 Pls
2250 1016 Pls
2252 1026 Pls
2255 1046 Pls
2258 1043 Pls
2260 1035 Pls
2263 1050 Pls
2266 1047 Pls
2268 1067 Pls
2271 1061 Pls
2273 1064 Pls
2276 1079 Pls
2279 1096 Pls
2281 1084 Pls
2284 1085 Pls
2286 1098 Pls
2289 1115 Pls
2292 1090 Pls
2294 1117 Pls
2297 1112 Pls
2299 1118 Pls
2302 1124 Pls
2305 1127 Pls
2307 1116 Pls
2310 1140 Pls
2312 1140 Pls
2315 1143 Pls
2318 1136 Pls
2320 1153 Pls
2323 1158 Pls
2325 1152 Pls
2328 1156 Pls
2331 1166 Pls
2333 1175 Pls
2336 1169 Pls
2338 1177 Pls
2341 1178 Pls
2344 1188 Pls
2346 1191 Pls
2349 1184 Pls
2351 1190 Pls
2354 1210 Pls
2357 1208 Pls
2359 1205 Pls
2362 1203 Pls
2365 1223 Pls
2367 1230 Pls
2370 1217 Pls
2372 1221 Pls
2375 1244 Pls
2378 1248 Pls
2380 1243 Pls
2383 1239 Pls
2385 1254 Pls
2388 1269 Pls
2391 1273 Pls
2393 1248 Pls
2396 1271 Pls
2398 1287 Pls
2401 1293 Pls
2404 1270 Pls
2406 1289 Pls
2409 1300 Pls
2411 1314 Pls
2414 1304 Pls
2417 1290 Pls
2419 1316 Pls
2422 1340 Pls
2425 1338 Pls
2427 1319 Pls
2430 1334 Pls
2432 1357 Pls
2435 1370 Pls
2438 1348 Pls
2440 1349 Pls
2443 1358 Pls
2445 1407 Pls
2448 1376 Pls
2451 1371 Pls
2453 1373 Pls
2456 1408 Pls
2458 1419 Pls
2461 1394 Pls
2464 1387 Pls
2466 1407 Pls
2469 1449 Pls
2471 1424 Pls
2474 1415 Pls
2477 1420 Pls
2479 1463 Pls
2482 1458 Pls
2484 1447 Pls
2487 1430 Pls
2490 1462 Pls
2492 1494 Pls
2495 1479 Pls
2497 1466 Pls
2500 1468 Pls
2503 1505 Pls
2505 1517 Pls
2508 1503 Pls
2510 1477 Pls
2513 1516 Pls
2516 1545 Pls
2518 1545 Pls
2521 1519 Pls
2523 1517 Pls
2526 1553 Pls
2529 1586 Pls
2531 1564 Pls
2534 1524 Pls
2536 1578 Pls
2539 1593 Pls
2542 1625 Pls
2544 1567 Pls
2547 1563 Pls
2550 1623 Pls
2552 1663 Pls
2555 1625 Pls
2557 1575 Pls
2560 1636 Pls
2563 1663 Pls
2565 1699 Pls
2568 1622 Pls
2570 1612 Pls
2573 1702 Pls
2576 1724 Pls
2578 1680 Pls
2581 1633 Pls
2583 1706 Pls
2586 1736 Pls
2589 1771 Pls
2591 1682 Pls
2594 1684 Pls
2597 1756 Pls
2599 1795 Pls
2602 1748 Pls
2604 1704 Pls
2607 1768 Pls
2610 1797 Pls
2612 1849 Pls
2615 1743 Pls
2617 1767 Pls
2620 1817 Pls
2623 1873 Pls
2625 1819 Pls
2628 1794 Pls
2630 1828 Pls
2633 1862 Pls
2636 1928 Pls
2638 1812 Pls
2641 1859 Pls
2643 1880 Pls
2646 1951 Pls
2649 1898 Pls
2651 1898 Pls
2654 1894 Pls
2656 1932 Pls
2659 2021 Pls
2662 1898 Pls
2664 1949 Pls
2667 1943 Pls
3218 1947 Pls
stroke
grestore
end
showpage
}}%
\put(3037,1947){\makebox(0,0)[r]{$T=0.85$}}%
\put(1900,50){\makebox(0,0){Excitation energy $E$}}%
\put(100,1180){%
\special{ps: gsave currentpoint currentpoint translate
270 rotate neg exch neg exch translate}%
\makebox(0,0)[b]{\shortstack{Free energy $F(E)$}}%
\special{ps: currentpoint grestore moveto}%
}%
\put(3450,200){\makebox(0,0){25}}%
\put(2830,200){\makebox(0,0){20}}%
\put(2210,200){\makebox(0,0){15}}%
\put(1590,200){\makebox(0,0){10}}%
\put(970,200){\makebox(0,0){5}}%
\put(350,200){\makebox(0,0){0}}%
\put(300,2060){\makebox(0,0)[r]{10}}%
\put(300,1767){\makebox(0,0)[r]{8}}%
\put(300,1473){\makebox(0,0)[r]{6}}%
\put(300,1180){\makebox(0,0)[r]{4}}%
\put(300,887){\makebox(0,0)[r]{2}}%
\put(300,593){\makebox(0,0)[r]{0}}%
\put(300,300){\makebox(0,0)[r]{-2}}%
\end{picture}%
\endgroup
\endinput

	\caption{$^1S_0$ energy gap in nuclear matter calculated with  
                 the CD-Bonn potential  
                 compared with the direct calculation from the 
                 $^1S_0$ np and pp phase shifts.  Also shown are the results 
     for neutron matter with the CD-Bonn potential.}
    \label{fig:fig3}
\end{figure}	

\end{document}















\documentstyle[aps,psfig,multicol]{revtex}

\begin{document}

\draft

\title{Nucleon-Nucleon Phase Shifts and Pairing in 
Neutron Matter and Nuclear Matter}
 
\author{\O.\ Elgar\o y$^a$ and M.\ Hjorth-Jensen$^b$}

\address{$^a$Department of Physics, University of Oslo, N-0316 Oslo, Norway}

\address{$^b$Nordita, Blegdamsvej 17, DK-2100 K\o benhavn \O, Denmark}

\maketitle

\begin{abstract}

We consider $^1S_0$ pairing in infinite neutron matter 
and nuclear matter 
and show that in the lowest order approximation, where the pairing 
interaction is taken to be the bare nucleon-nucleon (NN) interaction 
in the $^1S_0$ channel, the pairing interaction and the energy gap 
can be determined directly from the $^1S_0$ phase shifts.  
This is due to the almost separable 
character of the nucleon-nucleon interaction in this partial wave. 
Since the most recent NN interactions are charge-dependent, we 
have to solve coupled gap equations for proton-proton, neutron-neutron, 
and neutron-proton pairing in nuclear matter.  The results are, 
however, found to be close to those obtained with charge-independent 
potentials.


\end{abstract}

\pacs{PACS number(s): 21.30.-x, 21.65.+f, 26.60.+c }

\begin{multicols}{2}

Recently, there has been renewed interest in the pairing problem in 
neutron matter and neutron-rich nuclei.  The superfluid properties 
of neutron matter is of importance in the study of neutron stars 
\cite{petra95}, while pairing in neutron-rich systems is of relevance 
for the study of heavy nuclei close to the drip line \cite{mulshe93} 
and the light halo nuclei \cite{riis94}.  Much effort has gone into 
calculating the superfluid energy gap in dilute neutron matter 
\cite{baldo90,chen93,tak93,elg96,khodel96}.  
Most of these studies,  
e.g., those of Refs.\ \cite{baldo90,tak93,elg96,khodel96} have been carried 
out using pairing matrix elements given by the bare nucleon-nucleon 
(NN) interaction.  Many of the same authors have calculated the 
$^1S_0$ gap in nuclear matter, which has also been the subject of  
recent relativistic formulations of the pairing problem 
\cite{ring90,guim96,matera97}.  

Even though it is a long time since  
Clark et al.\ \cite{clark76} showed that the effects of density and 
spin-density fluctuations must be included in the pairing interaction, 
and there 
has been much progress in that direction recently \cite{wam93,schulze96}, 
we will here focus on the situation at the level of the 
bare interaction.  In this lowest-order approximation to the 
problem it has been found that results for the $^1S_0$ energy 
gap in neutron matter and in nuclear matter are almost independent 
of the choice of NN interaction.  
We aim at explaining how this can 
be understood directly from the measured properties of the free NN 
interaction.  
Moreover, although a relation between the pairing gap and
NN phase shifts was obtained almost forty years ago by Emery and
Sessler \cite{es60} (see also Hoffberg et al.\ \cite{hoffberg70}),  
in this work we wish to focus on the near 
interaction independence of the results for the energy gap at the 
Fermi level, and try to explain this from the NN scattering data directly.  
Our investigation is similar in spirit to the work of Refs.\ 
\cite{khodel96,carlson97} where the relation between the $^1S_0$ 
scattering amplitude and the gap function in momentum space was 
clarified.  In this paper, however, the focus is on the size 
of the energy gap at the Fermi momentum and how well this quantity is 
determined by NN scattering data.  

The energy gap in infinite matter is obtained by solving the BCS equation 
for the gap function $\Delta(k)$.  
\begin{equation}
      \Delta(k)=-\frac{1}{\pi}\int_{0}^{\infty}dk'k'^2 
                 V(k,k')\frac{\Delta(k')}{E(k')}, 
      \label{eq:eq1}
\end{equation}
where $V(k,k')$ is the bare momentum-space NN interaction in the 
$^1S_0$ channel, and $E(k)$ is the quasiparticle energy given by 
$E(k)=\sqrt{(\epsilon(k)-\epsilon(k_F))^2+\Delta(k)^2}$, where 
$\epsilon(k)$ is the single-particle energy of a neutron with 
momentum $k$, and $k_F$ is the Fermi momentum.  
Medium effects should 
be included in $\epsilon(k)$, but we will use free single-particle 
energies $\epsilon(k)=k^{2}/2m$, where $m$ is the neutron rest mass,  
to avoid unnecessary complications.  
And in neutron matter, at least at the densities 
considered here,  Brueckner-type calculations \cite{elg96} 
indicate that in-medium single-particle energies do not 
differ much from the free ones. 
The energy gap is defined as $\Delta_F\equiv \Delta(k_F)$.  
Eq.\ (\ref{eq:eq1}) can be 
solved by various techniques, some of which are described in 
Refs.\ \cite{elg96,khodel96}.  
In Fig.\ \ref{fig:fig1} we show 
the results for $\Delta_F$ obtained with the CD-Bonn potential (full line) 
\cite{mach96},  
the Nijmegen I and Nijmegen II potentials (long-dashed line and 
short-dashed line, respectively) \cite{nijm94}. 
The results are virtually identical, with the maximum value 
of the gap varying from 2.98 MeV for the Nijmegen I potential to 3.05 MeV 
for the Nijmegen II potential.  The same insensitivity of the energy gap 
with respect to 
the choice of NN interaction was found in Refs.\ 
\cite{baldo90,elg96,khodel96}. 
We will now discuss how these results can be understood 
from the properties of the NN interaction in the $^1S_0$ channel.

A characteristic feature of $^1S_0$ NN scattering is the large, negative 
scattering length, indicating the presence of  
a virtual bound state at $\approx$ 140 keV scattering energy.  
This state shows up as a pole in the NN $T$-matrix, 
which then  can be written in separable form, 
and this implies that the NN interaction itself to a good approximation is 
rank-one separable near this pole \cite{brown76}.   
Thus, at low energies we can write 
\begin{equation}
       V(k,k')=\lambda v(k)v(k'),
       \label{eq:eq2}
\end{equation}
where $\lambda$ is a constant.  Then it is easily seen from 
Eq.\ (\ref{eq:eq1}) that the gap function can be written as $\Delta_F v(k)$, 
where $\Delta_F$ is the energy gap.  Inserting this form of 
$\Delta(k)$ into Eq.\ (\ref{eq:eq1}) one obtains 
\begin{equation}
      1=-\frac{1}{\pi}\int_{0}^{\infty}dk'k'^2\frac{\lambda v^2(k')}{E(k')}, 
      \label{eq:eq3}
\end{equation}
which shows that the energy gap $\Delta_F$ is 
determined by the diagonal elements $\lambda v^2(k)$ of the NN interaction.  
The crucial point is that in scattering theory it can be shown that 
the inverse scattering problem, that is, the determination of a 
two-particle potential from the knowledge of the phase shifts at all 
energies, is exactly, and uniquely, solvable for rank-one 
separable potentials \cite{brown76,chadan92}.  Following the notation 
of Ref.\ \cite{brown76} we have 
\begin{equation}
       \lambda v^2(k)=-\frac{k^2+\kappa_B^2}{k^2}
                       \frac{\sin \delta(k)}{k}e^{-\alpha(k)},
       \label{eq:eq4}
\end{equation}
for an attractive potential with a bound state at energy $E=-\kappa_B^2$. 
In our case we take  $\kappa_B\approx 0$.    
Here $\delta(k)$ is the $^1S_0$ phase shift as a function of momentum $k$, 
while $\alpha(k)$ is given by a principle value integral: 
\begin{equation}
       \alpha(k)=\frac{1}{\pi}{\rm P}\int_{-\infty}^{+\infty}dk'
                 \frac{\delta(k')}{k'-k},
       \label{eq:eq5}
\end{equation}
where the phase shifts are extended to negative momenta through 
$\delta(-k)=-\delta(k)$.  Eqs.\ (\ref{eq:eq4}) and (\ref{eq:eq5}) 
can also be rewritten in terms of the Jost function \cite{chadan92}  
as done in Ref.\ \cite{kk97}.

 From this discussion we see that $\lambda v^2(k)$, and therefore also 
the energy gap $\Delta_F$, is completely determined by the $^1S_0$ 
phase shifts.  However, there are two obvious limitations on the 
practical validity of this statement.  First of all, the separable 
approximation can only be expected to be good at low energies, near the 
pole in the $T$-matrix.  Secondly, we see from Eq.\ (\ref{eq:eq5}) that 
knowledge of the phase shifts $\delta(k)$ at all energies is required.  
This is, of course, impossible, and most phase shift 
analyses stop at a laboratory energy $E_{\rm lab}=350$ MeV.  
Strictly speaking, the rank-one separable  approximation to the 
$^1S_0$ interaction breaks down already where the 
$^1S_0$ phase shift changes sign from positive to negative at 
$E_{\rm lab}\approx 248$ MeV, corresponding to a single-particle momentum 
of $k\approx1.73\;{\rm fm}^{-1}$.  However, at low values of $k_F$, knowledge 
of $v(k)$ up to this value of $k$ may actually be enough to determine 
the value of $\Delta_F$, as the integrand in Eq.\ (\ref{eq:eq5}) is 
strongly peaked around $k_F$.  We therefore found it worthwhile to 
try to calculate the energy gap directly from the $^1S_0$ phase shifts 
using Eqs. (\ref{eq:eq3})-(\ref{eq:eq5}). 
A possible improvement to the rank-one
separable approach for potentials which change sign is discussed by
Kwong and K\"ohler \cite{kk97}. 

The input in our calculation is the $^1S_0$ phase shifts taken from  
the recent Nijmegen phase shift analysis \cite{nijm93}. 
We then evaluated $\lambda v^2(k)$ from Eqs. (\ref{eq:eq4}) and 
(\ref{eq:eq5}), using methods described in Ref.\ \cite{davies91} to 
evaluate the principle value integral in Eq.\ (\ref{eq:eq5}). 
Finally, we evaluated the energy gap $\Delta_F$ for various values 
of $k_F$ by solving Eq.\ (\ref{eq:eq3}).  
Numerically the integral on the right-hand side of this equation 
depended very weakly on the momentum structure of $\Delta(k)$, so 
in our calculations we could take $\Delta(k)\approx \Delta_F$ in 
Eq.\ (\ref{eq:eq3}), and thus it became an algebraic equation 
for the energy gap $\Delta_F$.  
The resulting energy gap is plotted in Fig.\ 
\ref{fig:fig2} (dashed line) together with the gap obtained with the 
CD-Bonn potential (full line). 
As the reader can see, the agreement 
between the direct calculation from the phase shifts and the CD-Bonn 
calculation of $\Delta_F$ is very good, even 
at densities as high as $k_F=1.4\;{\rm fm}^{-1}$.  The energy gap 
is to a great extent determined by the available $^1S_0$ phase shifts.  
This can also be understood from the fact that for a rank-one separable
potential, the equations for the scattering state and the pair 
state become identical, as also pointed out by Carlson et al.\ 
\cite{carlson97}.
In the same figure we also report the results (dot-dashed line) 
obtained using the effective range approximation to the phase shifts: 
\begin{equation}
       k\cot \delta(k)=-\frac{1}{a_0}+\frac{1}{2}r_0 k^2,
       \label{eq:eq6}
\end{equation}
where $a_0=-18.8\pm 0.3$ fm and $r_0=2.75\pm 0.11$ fm are the singlet 
neutron-neutron scattering length and effective range, respectively.  
In this case an analytic expression can be obtained for $\lambda v^2(k)$, as 
shown in Ref.\ \cite{chadan92}:
\begin{equation}
       \lambda v^2(k)=-\frac{1}{\sqrt{k^2+
                       \frac{r_0^2}{4}(k^2+\alpha^2)^2}}
                       \sqrt{\frac{k^2+\beta_2^2}{k^2-\beta_1^2}},
       \label{eq:eq7}
\end{equation}
with $\alpha^2=-2/a_0 r_0$, and $\beta_1\approx-0.0498\;{\rm fm}^{-1}$, 
and $\beta_2\approx 0.777\;{\rm fm}^{-1}$ are the two roots of the 
quadratic equation 
\begin{equation}
   \beta^2-\frac{2}{r_0}\beta-\alpha^2=0.
   \label{eq:eq8}
\end{equation}  
The phase shifts using this approximation are positive at all energies, 
and this is reflected in Eq.\ (\ref{eq:eq7}) where $\lambda v^2(k)$ 
is attractive for all $k$.  From Fig.\ \ref{fig:fig1} we see that 
below $k_F=0.5\;{\rm fm}^{-1}$ the energy gap can with reasonable 
accuracy be calculated with the interaction obtained directly from 
the effective range approximation.  
One can therefore say that 
at densities below $k_F=0.5\;{\rm fm}^{-1}$, and at the crudest level 
of sophistication in many-body theory,  the superfluid properties 
of neutron matter are determined by just two parameters, namely 
the free-space scattering length and effective range. At such densities,
more complicated many-body terms are also less important.
Also interesting is the fact that the phase shifts predict the position 
of the first zero of $\Delta(k)$ in momentum space, since we see from 
Eq.\ (\ref{eq:eq4}) that $\Delta(k)=\Delta_F v(k)=0$ first for $\delta(k)=0$, 
which occurs at $E_{\rm lab}\approx 248$ MeV (pp scattering) 
corresponding to $k\approx 
1.73\;{\rm fm}^{-1}$.  This is in good agreement with the results of 
Khodel et al.\ \cite{khodel96}.  In Ref.\ \cite{khodel96} it is 
also shown that this first zero of the gap function determines the 
Fermi momentum at which $\Delta_F=0$.  Our results therefore indicate 
that this Fermi momentum is in fact given by the energy at which 
the $^1S_0$ phase shifts become negative. 

The calculation of the $^1S_0$ gap in symmetric nuclear matter is  
closely related to the one for neutron matter.  In fact, with 
charge-independent forces, like the older Bonn potentials, and 
free single--particle energies one would, of course, obtain 
exactly the same results.  However, the new potentials 
on the market are charge-dependent, in order to achieve high quality fits  
to both np and pp scattering data, and therefore we must solve three 
coupled gap equations for neutron-neutron (nn), proton-proton (pp), 
and neutron-proton (np) pairing \cite{good79}:
\begin{equation}
   \Delta_i(k)=-\frac{1}{\pi}\int_0^{\infty}dk'k'^2 V_i(k,k')
   \frac{\Delta_i(k')}{E(k')},
\label{eq:eq9}
\end{equation}
where $i$=nn, pp and np, and the quasiparticle energy is still given by 
$E(k)=\sqrt{(\epsilon(k)-\epsilon(k_F))^2+\Delta(k)^2}$, but the energy 
gap is now given by 
\begin{equation}
    \Delta(k)^2=\Delta_{\rm nn}(k)^2+\Delta_{\rm pp}(k)^2+\Delta_{\rm np}(k)^2.
    \label{eq:eq10}
\end{equation}
Thus the equations are coupled through their common energy denominator.  
The $^1S_0$ pp and nn interactions are very nearly identical, so  
the set of equations above can be reduced to two: one for the nn 
(or pp) gap and one for 
the np gap.  Solving these equations, both with the CD-Bonn potential and 
with the phase shift approximations we get the results shown in Fig.\ 
\ref{fig:fig3}.  For comparison we have in the same figure plotted the 
results for pure neutron matter with the CD-Bonn potential (dashed line). 
 From the figure it is clear that the phase shift approximation works well 
also in this case, and that the gap in symmetric matter is not very different 
from the gap in neutron matter.   As could be expected, the results 
are very close to those obtained earlier with charge-independent 
interactions \cite{baldo90,chen93,tak93,elg96}. 


In summary, we have shown that in infinite neutron and nuclear matter, 
owing to the near rank-one separability of the NN interaction in 
the $^1S_0$ partial wave,  
we are able to compute the $^1S_0$ pairing gap directly from the NN 
phase shifts. This explains why all NN potentials which fit 
the scattering data result in almost identical $^1S_0$ pairing gaps.  
Our findings conform with the conclusions of Khodel et al. \cite{khodel96} 
and Carlson et al. \cite{carlson97}: 
The virtual bound state in $^1S_0$ NN scattering determines  
the features of nucleon pairing in that partial wave.    
This is the main result of this work. 
However, it should be mentioned that this result    
is not likely to survive in a more refined calculation, for instance 
if one includes density and spin-density fluctuations in the 
effective pairing interaction like in e.g., Refs.\ \cite{wam93,schulze96}.  
Other partial waves will then be involved, and the simple arguments 
employed here will no longer apply.  Our reasoning here
applies also only
to a partial wave with bound or virtual bound state, that is, where  
$T$-matrix has a pole, and we have
neglected the fact that the phase shifts become negative at higher
energies.  
As a curiosity, we have found that at Fermi momenta below
$0.5$ fm$^{-1}$ the pairing gap is even determined by two parameters 
only, the effective range and the scattering length.
Also, we have pointed out that since the new NN interactions are charge 
dependent, one has to consider three coupled gap equations for 
$^1S_0$ pairing in nuclear matter.  The final result though is very 
nearly the same as what one obtains with charge-independent interactions.


We are much indebted to  B.\ V.\ Carlson, 
J.\ W.\ Clark, and E.\ Osnes for many 
valuable comments and discussions.  
  
\begin{references}
\bibitem{petra95} C.\ J.\ Pethick and D.\ G.\ Ravenhall, Annu.\ Rev.\ 
Nucl.\ Part.\ Sci.\ {\bf 45}, 429 (1995).
\bibitem{mulshe93} A.\ C.\ M\"{u}ller and B.\ M.\ Sherril, Annu.\ Rev.\ Nucl.\ 
Part.\ Sci.\ {\bf 43}, 529 (1993).
\bibitem{riis94} K.\ Riisager, Rev.\ Mod.\ Phys.\ {\bf 66}, 1105 (1994).
\bibitem{baldo90} M.\ Baldo, J.\ Cugnon, A.\ Lejeune, and U.\ Lombardo, 
Nucl.\ Phys.\ A {\bf 515}, 409 (1990).
\bibitem{chen93} J.\ M.\ C.\ Chen, J.\ W.\ Clark, R.\ D.\ Dav\'{e}, and 
V.\ V.\ Khodel, Nucl.\ Phys.\ A {\bf 555}, 59 (1993).
\bibitem{tak93} T.\ Takatsuka and R.\ Tamagaki, Prog.\ Theor.\ Phys.\ 
Suppl.\ {\bf 112}, 27 (1993).
\bibitem{elg96} \O.\ Elgar\o y, L.\ Engvik, M.\ Hjorth-Jensen, and E.\ Osnes, 
Nucl.\ Phys.\ A {\bf 604}, 466 (1996).
\bibitem{khodel96} V.\ A.\ Khodel, V.\ V.\ Khodel, and J.\ W.\ Clark, 
Nucl.\ Phys.\ A {\bf 598}, 390 (1996).
\bibitem{ring90} H.\ Kucharek and P.\ Ring, Z.\ Phys. {\bf 339}, 23 (1990).
\bibitem{guim96} F.\ B.\ Guimar\~{a}es, B.\ V.\ Carlson, and T.\ Frederico, 
Phys.\ Rev.\ C {\bf 54}, 2385 (1996).
\bibitem{matera97} F.\ Matera, G.\ Fabbri, and A.\ Dellafiore, 
Phys.\ Rev.\ C {\bf 56}, 228 (1997).
\bibitem{clark76} J.\ W.\ Clark, C.\ G.\ K\"{a}llman, C.\-H.\ Yang, and 
D.\ A.\ Chakkalakal, Phys.\ Lett.\ B {\bf 61}, 331 (1976).
\bibitem{wam93} J.\ Wambach, T.\ L.\ Ainsworth, and D.\ Pines, Nucl.\ Phys.\ 
A {\bf 555}, 128 (1993).
\bibitem{schulze96} H.\-J.\ Schulze, J.\ Cugnon, A.\ Lejeune, M.\ Baldo, 
and U.\ Lombardo, Phys.\ Lett.\ B {\bf 375}, 1 (1996).
\bibitem{es60} V.\ J.\ Emery and A.\ M.\ Sessler,  Phys.\ Rev.\ {\bf 119},
248 (1960) and references therein.
\bibitem{hoffberg70} M.\ Hoffberg, A.\ E.\ Glassgold, R.\ W.\ Richardson, 
and M.\ Ruderman, Phys.\ Rev.\ Lett.\ {\bf 24}, 775 (1970).
\bibitem{carlson97} B.\ V.\ Carlson, T.\ Frederico, 
 and F.\ B.\ Guimar\~{a}es, nucl-th/9706071, submitted to Phys.\ Rev.\ C. 
\bibitem{mach96} R.\ Machleidt, F.\ Sammarruca, and Y.\ Song, Phys.\ Rev.\ C 
{\bf 53}, 1483 (1996).
\bibitem{nijm94} V.\ G.\ J.\ Stoks, R.\ A.\ M.\ Klomp, C.\ P.\ F.\ Terheggen, 
and J.\ J.\ de Swart, Phys.\ Rev.\ C {\bf 49}, 2950 (1994);
NN-Online facility, (URL: http://NN-OnLine.sci.kun.nl).
\bibitem{brown76} G.\ E.\ Brown and A.\ D.\ Jackson, {\it The Nucleon-Nucleon 
Interaction} (North-Holland, Amsterdam, 1976).
\bibitem{chadan92} K.\ Chadan and P.\ C.\ Sabatier, {\it Inverse Problems in 
Quantum Scattering Theory}, 2nd ed.\ (Springer, New York, 1992).
\bibitem{kk97} N.\ H.\ Kwong and H.\ S.\ K\"ohler, Phys.\ Rev.\ C {\bf 55},
1650 (1997).
\bibitem{nijm93} V.\ G.\ J.\ Stoks, R.\ A.\ M.\ Klomp, 
M.\ G.\ M.\ Rentmeester, and 
J.\ J.\ de Swart, Phys.\ Rev.\ C {\bf 48}, 792 (1993). 
\bibitem{davies91} K.\ T.\ R.\ Davies, G.\ D.\ White, and R.\ W.\ Davies, 
Nucl.\ Phys.\ A {\bf 524}, 743 (1991).
\bibitem{good79} A.\ L.\ Goodman, Adv.\ Nucl.\ Phys.\ {\bf 11}, 263 (1979).
\end{references}


\end{multicols}

\clearpage

\begin{figure}
    
%\documentstyle[pst-plot]{article}

%\begin{document}
\pagestyle{empty}
\begin{figure}[htbp]
\setlength{\unitlength}{1cm}
\begin{center}
\setlength{\unitlength}{1cm}
\thicklines
\Cartesian(1.6cm,3cm)
\pspicture(0,0)(10,7)
\psline[linewidth=1pt]{->}(0.5,0.0)(0.5,5.0)
\psline[linewidth=1pt]{-}(0.4,0.0)(0.5,0.0)
\psline[linewidth=1pt]{-}(0.4,1)(0.5,1)
\psline[linewidth=1pt]{-}(0.4,2)(0.5,2)
\psline[linewidth=1pt]{-}(0.4,3)(0.5,3)
\psline[linewidth=1pt]{-}(0.4,4)(0.5,4)
\uput[0](0,0.0){0}
\uput[0](0,1.0){1}
\uput[0](0,2.0){2}
\uput[0](0,3.0){3}
\uput[0](0,4.0){4}
\uput[0](-0.5,5){Energy (MeV)}
%expt
\psline{-}(1,0.0)(2,0.0) 
\psline{-}(1,0.1)(2,0.1) 
\psline{-}(1,0.32)(2,0.32) 
\psline{-}(1,0.44)(2,0.44) 
\psline{-}(1,0.89)(2,0.89) 
\psline{-}(1,1.53)(2,1.53) 
\psline{-}(1,1.95)(2,1.95) 
\psline{-}(1,2.26)(2,2.26) 
\psline{-}(1,2.76)(2,2.76) 
\psline{-}(1,3.99)(2,3.99) 
\uput[0](2.5,0.0){\footnotesize{$(2^{+}$)}}
\uput[0](2.5,0.1){\footnotesize{$(4^{+}$})}
\uput[0](2.5,0.32){\footnotesize{$(5^{+}$})}
\uput[0](2.5,0.44){\footnotesize{$(6^{+}$})}
\uput[0](2.5,0.89){\footnotesize{$(7^{+}$})}
\uput[0](2.5,1.53){\footnotesize{$(8^{+}$})}
\uput[0](2.5,2.26){\footnotesize{$(10^{+}$})}
%theory1
\psline{-}(5,0.0)(6,0.0) 
\psline{-}(5,0.14)(6,0.14) 
\psline{-}(5,0.19)(6,0.19) 
\psline{-}(5,0.39)(6,0.39) 
\psline{-}(5,0.54)(6,0.54) 
\psline{-}(5,0.66)(6,0.66) 
\psline{-}(5,1.34)(6,1.34) 
\psline{-}(5,1.78)(6,1.78) 
\psline{-}(5,2.34)(6,2.34) 
\psline{-}(5,2.57)(6,2.57) 
\psline{-}(5,3.1)(6,3.1) 
\psline{-}(5,3.6)(6,3.6) 
\psline{-}(5,4.2)(6,4.2) 
\uput[0](6.5,0.0){\footnotesize{$2^{+}$}}
\uput[0](6.5,0.14){\footnotesize{$1^{+}$}}
\uput[0](6.5,0.25){\footnotesize{$4^{+}$}}
\uput[0](6.5,0.39){\footnotesize{$3^{+}$}}
\uput[0](6.5,0.54){\footnotesize{$5^{+}$}}
\uput[0](6.5,0.66){\footnotesize{$6^{+}$}}
\uput[0](6.5,1.34){\footnotesize{$7^{+}$}}
\uput[0](6.5,1.78){\footnotesize{$8^{+}$}}
\uput[0](6.5,2.34){\footnotesize{$9^{+}$}}
\uput[0](6.5,2.57){\footnotesize{$10^{+}$}}
\uput[0](6.5,3.1){\footnotesize{$11^{+}$}}
\uput[0](6.5,3.6){\footnotesize{$12^{+}$}}
\uput[0](6.5,4.2){\footnotesize{$13^{+}$}}
%theory2
\psline{-}(8,0.0)(9,0.0) 
\psline{-}(8,0.14)(9,0.14) 
\psline{-}(8,0.24)(9,0.24) 
\psline{-}(8,0.40)(9,0.40) 
\psline{-}(8,0.46)(9,0.46) 
\psline{-}(8,0.74)(9,0.74) 
\psline{-}(8,1.1)(9,1.1) 
\psline{-}(8,1.54)(9,1.54) 
\psline{-}(8,1.86)(9,1.86) 
\psline{-}(8,2.23)(9,2.23) 
\psline{-}(8,2.59)(9,2.59) 
\psline{-}(8,3.86)(9,3.86) 
\psline{-}(8,4.32)(9,4.32) 
\uput[0](9.5,0.0){\footnotesize{$1^{+}$}}
\uput[0](9.5,0.14){\footnotesize{$2^{+}$}}
\uput[0](9.5,0.24){\footnotesize{$4^{+}$}}
\uput[0](9.5,0.40){\footnotesize{$3^{+}$}}
\uput[0](9.5,0.52){\footnotesize{$5^{+}$}}
\uput[0](9.5,0.74){\footnotesize{$6^{+}$}}
\uput[0](9.5,1.1){\footnotesize{$7^{+}$}}
\uput[0](9.5,1.54){\footnotesize{$8^{+}$}}
\uput[0](9.5,1.86){\footnotesize{$9^{+}$}}
\uput[0](9.5,2.23){\footnotesize{$10^{+}$}}
\uput[0](9.5,2.59){\footnotesize{$11^{+}$}}
\uput[0](9.5,3.86){\footnotesize{$12^{+}$}}
\uput[0](9.5,4.32){\footnotesize{$13^{+}$}}
\uput[0](1.5,-0.2){EXPT}
\uput[0](6,-0.2){$1d_{5/2}0g_{7/2}1d_{3/2}2s_{1/2}0h_{11/2}$}
\uput[0](9,-0.2){$1d_{5/2}0g_{7/2}1d_{3/2}2s_{1/2}$}

\endpspicture

 \end{center}
 \end{figure}



    \caption{$^1S_0$ energy gap in neutron matter with the CD-Bonn, 
             Nijmegen I and Nijmegen II potentials.}
    \label{fig:fig1}
\end{figure}

\begin{figure}
    % GNUPLOT: LaTeX picture with Postscript
\begingroup%
  \makeatletter%
  \newcommand{\GNUPLOTspecial}{%
    \@sanitize\catcode`\%=14\relax\special}%
  \setlength{\unitlength}{0.1bp}%
{\GNUPLOTspecial{!
%!PS-Adobe-2.0 EPSF-2.0
%%Title: fig2.tex
%%Creator: gnuplot 3.7 patchlevel 1
%%CreationDate: Sat Mar 10 16:08:37 2001
%%DocumentFonts: 
%%BoundingBox: 0 0 360 216
%%Orientation: Landscape
%%EndComments
/gnudict 256 dict def
gnudict begin
/Color false def
/Solid false def
/gnulinewidth 5.000 def
/userlinewidth gnulinewidth def
/vshift -33 def
/dl {10 mul} def
/hpt_ 31.5 def
/vpt_ 31.5 def
/hpt hpt_ def
/vpt vpt_ def
/M {moveto} bind def
/L {lineto} bind def
/R {rmoveto} bind def
/V {rlineto} bind def
/vpt2 vpt 2 mul def
/hpt2 hpt 2 mul def
/Lshow { currentpoint stroke M
  0 vshift R show } def
/Rshow { currentpoint stroke M
  dup stringwidth pop neg vshift R show } def
/Cshow { currentpoint stroke M
  dup stringwidth pop -2 div vshift R show } def
/UP { dup vpt_ mul /vpt exch def hpt_ mul /hpt exch def
  /hpt2 hpt 2 mul def /vpt2 vpt 2 mul def } def
/DL { Color {setrgbcolor Solid {pop []} if 0 setdash }
 {pop pop pop Solid {pop []} if 0 setdash} ifelse } def
/BL { stroke userlinewidth 2 mul setlinewidth } def
/AL { stroke userlinewidth 2 div setlinewidth } def
/UL { dup gnulinewidth mul /userlinewidth exch def
      10 mul /udl exch def } def
/PL { stroke userlinewidth setlinewidth } def
/LTb { BL [] 0 0 0 DL } def
/LTa { AL [1 udl mul 2 udl mul] 0 setdash 0 0 0 setrgbcolor } def
/LT0 { PL [] 1 0 0 DL } def
/LT1 { PL [4 dl 2 dl] 0 1 0 DL } def
/LT2 { PL [2 dl 3 dl] 0 0 1 DL } def
/LT3 { PL [1 dl 1.5 dl] 1 0 1 DL } def
/LT4 { PL [5 dl 2 dl 1 dl 2 dl] 0 1 1 DL } def
/LT5 { PL [4 dl 3 dl 1 dl 3 dl] 1 1 0 DL } def
/LT6 { PL [2 dl 2 dl 2 dl 4 dl] 0 0 0 DL } def
/LT7 { PL [2 dl 2 dl 2 dl 2 dl 2 dl 4 dl] 1 0.3 0 DL } def
/LT8 { PL [2 dl 2 dl 2 dl 2 dl 2 dl 2 dl 2 dl 4 dl] 0.5 0.5 0.5 DL } def
/Pnt { stroke [] 0 setdash
   gsave 1 setlinecap M 0 0 V stroke grestore } def
/Dia { stroke [] 0 setdash 2 copy vpt add M
  hpt neg vpt neg V hpt vpt neg V
  hpt vpt V hpt neg vpt V closepath stroke
  Pnt } def
/Pls { stroke [] 0 setdash vpt sub M 0 vpt2 V
  currentpoint stroke M
  hpt neg vpt neg R hpt2 0 V stroke
  } def
/Box { stroke [] 0 setdash 2 copy exch hpt sub exch vpt add M
  0 vpt2 neg V hpt2 0 V 0 vpt2 V
  hpt2 neg 0 V closepath stroke
  Pnt } def
/Crs { stroke [] 0 setdash exch hpt sub exch vpt add M
  hpt2 vpt2 neg V currentpoint stroke M
  hpt2 neg 0 R hpt2 vpt2 V stroke } def
/TriU { stroke [] 0 setdash 2 copy vpt 1.12 mul add M
  hpt neg vpt -1.62 mul V
  hpt 2 mul 0 V
  hpt neg vpt 1.62 mul V closepath stroke
  Pnt  } def
/Star { 2 copy Pls Crs } def
/BoxF { stroke [] 0 setdash exch hpt sub exch vpt add M
  0 vpt2 neg V  hpt2 0 V  0 vpt2 V
  hpt2 neg 0 V  closepath fill } def
/TriUF { stroke [] 0 setdash vpt 1.12 mul add M
  hpt neg vpt -1.62 mul V
  hpt 2 mul 0 V
  hpt neg vpt 1.62 mul V closepath fill } def
/TriD { stroke [] 0 setdash 2 copy vpt 1.12 mul sub M
  hpt neg vpt 1.62 mul V
  hpt 2 mul 0 V
  hpt neg vpt -1.62 mul V closepath stroke
  Pnt  } def
/TriDF { stroke [] 0 setdash vpt 1.12 mul sub M
  hpt neg vpt 1.62 mul V
  hpt 2 mul 0 V
  hpt neg vpt -1.62 mul V closepath fill} def
/DiaF { stroke [] 0 setdash vpt add M
  hpt neg vpt neg V hpt vpt neg V
  hpt vpt V hpt neg vpt V closepath fill } def
/Pent { stroke [] 0 setdash 2 copy gsave
  translate 0 hpt M 4 {72 rotate 0 hpt L} repeat
  closepath stroke grestore Pnt } def
/PentF { stroke [] 0 setdash gsave
  translate 0 hpt M 4 {72 rotate 0 hpt L} repeat
  closepath fill grestore } def
/Circle { stroke [] 0 setdash 2 copy
  hpt 0 360 arc stroke Pnt } def
/CircleF { stroke [] 0 setdash hpt 0 360 arc fill } def
/C0 { BL [] 0 setdash 2 copy moveto vpt 90 450  arc } bind def
/C1 { BL [] 0 setdash 2 copy        moveto
       2 copy  vpt 0 90 arc closepath fill
               vpt 0 360 arc closepath } bind def
/C2 { BL [] 0 setdash 2 copy moveto
       2 copy  vpt 90 180 arc closepath fill
               vpt 0 360 arc closepath } bind def
/C3 { BL [] 0 setdash 2 copy moveto
       2 copy  vpt 0 180 arc closepath fill
               vpt 0 360 arc closepath } bind def
/C4 { BL [] 0 setdash 2 copy moveto
       2 copy  vpt 180 270 arc closepath fill
               vpt 0 360 arc closepath } bind def
/C5 { BL [] 0 setdash 2 copy moveto
       2 copy  vpt 0 90 arc
       2 copy moveto
       2 copy  vpt 180 270 arc closepath fill
               vpt 0 360 arc } bind def
/C6 { BL [] 0 setdash 2 copy moveto
      2 copy  vpt 90 270 arc closepath fill
              vpt 0 360 arc closepath } bind def
/C7 { BL [] 0 setdash 2 copy moveto
      2 copy  vpt 0 270 arc closepath fill
              vpt 0 360 arc closepath } bind def
/C8 { BL [] 0 setdash 2 copy moveto
      2 copy vpt 270 360 arc closepath fill
              vpt 0 360 arc closepath } bind def
/C9 { BL [] 0 setdash 2 copy moveto
      2 copy  vpt 270 450 arc closepath fill
              vpt 0 360 arc closepath } bind def
/C10 { BL [] 0 setdash 2 copy 2 copy moveto vpt 270 360 arc closepath fill
       2 copy moveto
       2 copy vpt 90 180 arc closepath fill
               vpt 0 360 arc closepath } bind def
/C11 { BL [] 0 setdash 2 copy moveto
       2 copy  vpt 0 180 arc closepath fill
       2 copy moveto
       2 copy  vpt 270 360 arc closepath fill
               vpt 0 360 arc closepath } bind def
/C12 { BL [] 0 setdash 2 copy moveto
       2 copy  vpt 180 360 arc closepath fill
               vpt 0 360 arc closepath } bind def
/C13 { BL [] 0 setdash  2 copy moveto
       2 copy  vpt 0 90 arc closepath fill
       2 copy moveto
       2 copy  vpt 180 360 arc closepath fill
               vpt 0 360 arc closepath } bind def
/C14 { BL [] 0 setdash 2 copy moveto
       2 copy  vpt 90 360 arc closepath fill
               vpt 0 360 arc } bind def
/C15 { BL [] 0 setdash 2 copy vpt 0 360 arc closepath fill
               vpt 0 360 arc closepath } bind def
/Rec   { newpath 4 2 roll moveto 1 index 0 rlineto 0 exch rlineto
       neg 0 rlineto closepath } bind def
/Square { dup Rec } bind def
/Bsquare { vpt sub exch vpt sub exch vpt2 Square } bind def
/S0 { BL [] 0 setdash 2 copy moveto 0 vpt rlineto BL Bsquare } bind def
/S1 { BL [] 0 setdash 2 copy vpt Square fill Bsquare } bind def
/S2 { BL [] 0 setdash 2 copy exch vpt sub exch vpt Square fill Bsquare } bind def
/S3 { BL [] 0 setdash 2 copy exch vpt sub exch vpt2 vpt Rec fill Bsquare } bind def
/S4 { BL [] 0 setdash 2 copy exch vpt sub exch vpt sub vpt Square fill Bsquare } bind def
/S5 { BL [] 0 setdash 2 copy 2 copy vpt Square fill
       exch vpt sub exch vpt sub vpt Square fill Bsquare } bind def
/S6 { BL [] 0 setdash 2 copy exch vpt sub exch vpt sub vpt vpt2 Rec fill Bsquare } bind def
/S7 { BL [] 0 setdash 2 copy exch vpt sub exch vpt sub vpt vpt2 Rec fill
       2 copy vpt Square fill
       Bsquare } bind def
/S8 { BL [] 0 setdash 2 copy vpt sub vpt Square fill Bsquare } bind def
/S9 { BL [] 0 setdash 2 copy vpt sub vpt vpt2 Rec fill Bsquare } bind def
/S10 { BL [] 0 setdash 2 copy vpt sub vpt Square fill 2 copy exch vpt sub exch vpt Square fill
       Bsquare } bind def
/S11 { BL [] 0 setdash 2 copy vpt sub vpt Square fill 2 copy exch vpt sub exch vpt2 vpt Rec fill
       Bsquare } bind def
/S12 { BL [] 0 setdash 2 copy exch vpt sub exch vpt sub vpt2 vpt Rec fill Bsquare } bind def
/S13 { BL [] 0 setdash 2 copy exch vpt sub exch vpt sub vpt2 vpt Rec fill
       2 copy vpt Square fill Bsquare } bind def
/S14 { BL [] 0 setdash 2 copy exch vpt sub exch vpt sub vpt2 vpt Rec fill
       2 copy exch vpt sub exch vpt Square fill Bsquare } bind def
/S15 { BL [] 0 setdash 2 copy Bsquare fill Bsquare } bind def
/D0 { gsave translate 45 rotate 0 0 S0 stroke grestore } bind def
/D1 { gsave translate 45 rotate 0 0 S1 stroke grestore } bind def
/D2 { gsave translate 45 rotate 0 0 S2 stroke grestore } bind def
/D3 { gsave translate 45 rotate 0 0 S3 stroke grestore } bind def
/D4 { gsave translate 45 rotate 0 0 S4 stroke grestore } bind def
/D5 { gsave translate 45 rotate 0 0 S5 stroke grestore } bind def
/D6 { gsave translate 45 rotate 0 0 S6 stroke grestore } bind def
/D7 { gsave translate 45 rotate 0 0 S7 stroke grestore } bind def
/D8 { gsave translate 45 rotate 0 0 S8 stroke grestore } bind def
/D9 { gsave translate 45 rotate 0 0 S9 stroke grestore } bind def
/D10 { gsave translate 45 rotate 0 0 S10 stroke grestore } bind def
/D11 { gsave translate 45 rotate 0 0 S11 stroke grestore } bind def
/D12 { gsave translate 45 rotate 0 0 S12 stroke grestore } bind def
/D13 { gsave translate 45 rotate 0 0 S13 stroke grestore } bind def
/D14 { gsave translate 45 rotate 0 0 S14 stroke grestore } bind def
/D15 { gsave translate 45 rotate 0 0 S15 stroke grestore } bind def
/DiaE { stroke [] 0 setdash vpt add M
  hpt neg vpt neg V hpt vpt neg V
  hpt vpt V hpt neg vpt V closepath stroke } def
/BoxE { stroke [] 0 setdash exch hpt sub exch vpt add M
  0 vpt2 neg V hpt2 0 V 0 vpt2 V
  hpt2 neg 0 V closepath stroke } def
/TriUE { stroke [] 0 setdash vpt 1.12 mul add M
  hpt neg vpt -1.62 mul V
  hpt 2 mul 0 V
  hpt neg vpt 1.62 mul V closepath stroke } def
/TriDE { stroke [] 0 setdash vpt 1.12 mul sub M
  hpt neg vpt 1.62 mul V
  hpt 2 mul 0 V
  hpt neg vpt -1.62 mul V closepath stroke } def
/PentE { stroke [] 0 setdash gsave
  translate 0 hpt M 4 {72 rotate 0 hpt L} repeat
  closepath stroke grestore } def
/CircE { stroke [] 0 setdash 
  hpt 0 360 arc stroke } def
/Opaque { gsave closepath 1 setgray fill grestore 0 setgray closepath } def
/DiaW { stroke [] 0 setdash vpt add M
  hpt neg vpt neg V hpt vpt neg V
  hpt vpt V hpt neg vpt V Opaque stroke } def
/BoxW { stroke [] 0 setdash exch hpt sub exch vpt add M
  0 vpt2 neg V hpt2 0 V 0 vpt2 V
  hpt2 neg 0 V Opaque stroke } def
/TriUW { stroke [] 0 setdash vpt 1.12 mul add M
  hpt neg vpt -1.62 mul V
  hpt 2 mul 0 V
  hpt neg vpt 1.62 mul V Opaque stroke } def
/TriDW { stroke [] 0 setdash vpt 1.12 mul sub M
  hpt neg vpt 1.62 mul V
  hpt 2 mul 0 V
  hpt neg vpt -1.62 mul V Opaque stroke } def
/PentW { stroke [] 0 setdash gsave
  translate 0 hpt M 4 {72 rotate 0 hpt L} repeat
  Opaque stroke grestore } def
/CircW { stroke [] 0 setdash 
  hpt 0 360 arc Opaque stroke } def
/BoxFill { gsave Rec 1 setgray fill grestore } def
end
%%EndProlog
}}%
\begin{picture}(3600,2160)(0,0)%
{\GNUPLOTspecial{"
gnudict begin
gsave
0 0 translate
0.100 0.100 scale
0 setgray
newpath
1.000 UL
LTb
350 300 M
63 0 V
3037 0 R
-63 0 V
350 551 M
63 0 V
3037 0 R
-63 0 V
350 803 M
63 0 V
3037 0 R
-63 0 V
350 1054 M
63 0 V
3037 0 R
-63 0 V
350 1306 M
63 0 V
3037 0 R
-63 0 V
350 1557 M
63 0 V
3037 0 R
-63 0 V
350 1809 M
63 0 V
3037 0 R
-63 0 V
350 2060 M
63 0 V
3037 0 R
-63 0 V
350 300 M
0 63 V
0 1697 R
0 -63 V
970 300 M
0 63 V
0 1697 R
0 -63 V
1590 300 M
0 63 V
0 1697 R
0 -63 V
2210 300 M
0 63 V
0 1697 R
0 -63 V
2830 300 M
0 63 V
0 1697 R
0 -63 V
3450 300 M
0 63 V
0 1697 R
0 -63 V
1.000 UL
LTb
350 300 M
3100 0 V
0 1760 V
-3100 0 V
350 300 L
1.000 UP
1.000 UL
LT0
350 803 Pls
725 981 Pls
728 797 Pls
730 745 Pls
733 741 Pls
736 794 Pls
738 739 Pls
741 763 Pls
744 766 Pls
746 777 Pls
749 743 Pls
751 810 Pls
754 777 Pls
757 758 Pls
759 818 Pls
762 821 Pls
764 962 Pls
767 837 Pls
772 882 Pls
777 924 Pls
783 1154 Pls
1043 1303 Pls
1046 1054 Pls
1048 970 Pls
1051 848 Pls
1054 831 Pls
1056 822 Pls
1059 776 Pls
1061 700 Pls
1064 766 Pls
1067 689 Pls
1069 681 Pls
1072 683 Pls
1074 638 Pls
1077 671 Pls
1080 626 Pls
1082 621 Pls
1085 624 Pls
1088 623 Pls
1090 625 Pls
1093 625 Pls
1095 623 Pls
1098 640 Pls
1101 617 Pls
1103 663 Pls
1106 652 Pls
1108 648 Pls
1111 689 Pls
1114 665 Pls
1116 714 Pls
1119 692 Pls
1121 747 Pls
1124 773 Pls
1127 770 Pls
1129 834 Pls
1132 805 Pls
1134 809 Pls
1137 930 Pls
1140 913 Pls
1142 940 Pls
1145 1094 Pls
1147 987 Pls
1150 1091 Pls
1153 1066 Pls
1155 1096 Pls
1158 1535 Pls
1161 1268 Pls
1163 1366 Pls
1168 1371 Pls
1171 1548 Pls
1288 1754 Pls
1291 1582 Pls
1293 1498 Pls
1296 1440 Pls
1299 1365 Pls
1301 1275 Pls
1304 1247 Pls
1306 1137 Pls
1309 1102 Pls
1312 1194 Pls
1314 1018 Pls
1317 1095 Pls
1319 935 Pls
1322 853 Pls
1325 891 Pls
1327 838 Pls
1330 789 Pls
1333 819 Pls
1335 803 Pls
1338 750 Pls
1340 733 Pls
1343 726 Pls
1346 671 Pls
1348 676 Pls
1351 672 Pls
1353 631 Pls
1356 637 Pls
1359 628 Pls
1361 612 Pls
1364 617 Pls
1366 593 Pls
1369 580 Pls
1372 597 Pls
1374 576 Pls
1377 569 Pls
1379 574 Pls
1382 562 Pls
1385 563 Pls
1387 570 Pls
1390 563 Pls
1392 562 Pls
1395 571 Pls
1398 562 Pls
1400 577 Pls
1403 580 Pls
1405 581 Pls
1408 599 Pls
1411 594 Pls
1413 604 Pls
1416 621 Pls
1419 627 Pls
1421 620 Pls
1424 651 Pls
1426 668 Pls
1429 664 Pls
1432 692 Pls
1434 702 Pls
1437 716 Pls
1439 737 Pls
1442 744 Pls
1445 783 Pls
1447 807 Pls
1450 804 Pls
1452 833 Pls
1455 845 Pls
1458 936 Pls
1460 908 Pls
1463 926 Pls
1465 954 Pls
1468 1008 Pls
1471 1035 Pls
1473 1055 Pls
1476 1061 Pls
1478 1159 Pls
1481 1263 Pls
1484 1139 Pls
1486 1264 Pls
1489 1246 Pls
1491 1259 Pls
1494 1246 Pls
1497 1235 Pls
1499 1224 Pls
1502 1182 Pls
1505 1152 Pls
1507 1134 Pls
1510 1110 Pls
1512 1066 Pls
1515 1050 Pls
1518 1093 Pls
1520 1037 Pls
1523 1026 Pls
1525 1015 Pls
1528 1003 Pls
1531 978 Pls
1533 928 Pls
1536 980 Pls
1538 953 Pls
1541 880 Pls
1544 906 Pls
1546 912 Pls
1549 858 Pls
1551 826 Pls
1554 802 Pls
1557 793 Pls
1559 772 Pls
1562 762 Pls
1564 742 Pls
1567 716 Pls
1570 721 Pls
1572 734 Pls
1575 686 Pls
1577 677 Pls
1580 690 Pls
1583 669 Pls
1585 632 Pls
1588 652 Pls
1591 646 Pls
1593 626 Pls
1596 612 Pls
1598 619 Pls
1601 613 Pls
1604 592 Pls
1606 590 Pls
1609 598 Pls
1611 578 Pls
1614 566 Pls
1617 573 Pls
1619 575 Pls
1622 554 Pls
1624 561 Pls
1627 564 Pls
1630 558 Pls
1632 546 Pls
1635 567 Pls
1637 562 Pls
1640 544 Pls
1643 563 Pls
1645 562 Pls
1648 552 Pls
1650 558 Pls
1653 568 Pls
1656 566 Pls
1658 573 Pls
1661 577 Pls
1663 582 Pls
1666 585 Pls
1669 586 Pls
1671 597 Pls
1674 607 Pls
1677 614 Pls
1679 614 Pls
1682 633 Pls
1684 631 Pls
1687 644 Pls
1690 648 Pls
1692 657 Pls
1695 675 Pls
1697 681 Pls
1700 684 Pls
1703 694 Pls
1705 701 Pls
1708 734 Pls
1710 728 Pls
1713 698 Pls
1716 728 Pls
1718 791 Pls
1721 765 Pls
1723 700 Pls
1726 743 Pls
1729 829 Pls
1731 800 Pls
1734 699 Pls
1736 737 Pls
1739 873 Pls
1742 813 Pls
1744 702 Pls
1747 724 Pls
1749 891 Pls
1752 793 Pls
1755 703 Pls
1757 705 Pls
1760 819 Pls
1763 778 Pls
1765 708 Pls
1768 691 Pls
1770 755 Pls
1773 745 Pls
1776 682 Pls
1778 665 Pls
1781 708 Pls
1783 691 Pls
1786 659 Pls
1789 641 Pls
1791 675 Pls
1794 657 Pls
1796 628 Pls
1799 630 Pls
1802 645 Pls
1804 623 Pls
1807 617 Pls
1809 617 Pls
1812 633 Pls
1815 611 Pls
1817 607 Pls
1820 613 Pls
1822 617 Pls
1825 605 Pls
1828 611 Pls
1830 611 Pls
1833 620 Pls
1835 605 Pls
1838 617 Pls
1841 612 Pls
1843 616 Pls
1846 614 Pls
1849 615 Pls
1851 615 Pls
1854 619 Pls
1856 625 Pls
1859 617 Pls
1862 615 Pls
1864 621 Pls
1867 628 Pls
1869 619 Pls
1872 623 Pls
1875 630 Pls
1877 630 Pls
1880 630 Pls
1882 627 Pls
1885 638 Pls
1888 645 Pls
1890 642 Pls
1893 650 Pls
1895 647 Pls
1898 653 Pls
1901 653 Pls
1903 668 Pls
1906 666 Pls
1908 652 Pls
1911 681 Pls
1914 691 Pls
1916 673 Pls
1919 668 Pls
1921 687 Pls
1924 678 Pls
1927 705 Pls
1929 690 Pls
1932 675 Pls
1935 693 Pls
1937 707 Pls
1940 694 Pls
1942 681 Pls
1945 693 Pls
1948 695 Pls
1950 722 Pls
1953 684 Pls
1955 691 Pls
1958 705 Pls
1961 729 Pls
1963 715 Pls
1966 697 Pls
1968 714 Pls
1971 735 Pls
1974 743 Pls
1976 706 Pls
1979 727 Pls
1981 735 Pls
1984 765 Pls
1987 721 Pls
1989 725 Pls
1992 723 Pls
1994 750 Pls
1997 726 Pls
2000 708 Pls
2002 705 Pls
2005 725 Pls
2007 728 Pls
2010 690 Pls
2013 689 Pls
2015 710 Pls
2018 719 Pls
2021 685 Pls
2023 686 Pls
2026 709 Pls
2028 715 Pls
2031 687 Pls
2034 691 Pls
2036 707 Pls
2039 716 Pls
2041 695 Pls
2044 702 Pls
2047 715 Pls
2049 720 Pls
2052 710 Pls
2054 715 Pls
2057 723 Pls
2060 735 Pls
2062 724 Pls
2065 726 Pls
2067 734 Pls
2070 748 Pls
2073 739 Pls
2075 739 Pls
2078 745 Pls
2080 762 Pls
2083 750 Pls
2086 757 Pls
2088 769 Pls
2091 770 Pls
2093 763 Pls
2096 769 Pls
2099 788 Pls
2101 786 Pls
2104 771 Pls
2106 774 Pls
2109 792 Pls
2112 780 Pls
2114 776 Pls
2117 787 Pls
2120 781 Pls
2122 786 Pls
2125 796 Pls
2127 787 Pls
2130 790 Pls
2133 794 Pls
2135 797 Pls
2138 812 Pls
2140 796 Pls
2143 804 Pls
2146 817 Pls
2148 818 Pls
2151 812 Pls
2153 821 Pls
2156 826 Pls
2159 833 Pls
2161 827 Pls
2164 833 Pls
2166 836 Pls
2169 837 Pls
2172 836 Pls
2174 840 Pls
2177 825 Pls
2179 834 Pls
2182 847 Pls
2185 827 Pls
2187 837 Pls
2190 847 Pls
2193 856 Pls
2195 850 Pls
2198 858 Pls
2200 866 Pls
2203 882 Pls
2206 879 Pls
2208 871 Pls
2211 893 Pls
2213 903 Pls
2216 892 Pls
2219 895 Pls
2221 896 Pls
2224 926 Pls
2226 917 Pls
2229 844 Pls
2232 915 Pls
2234 956 Pls
2237 858 Pls
2239 890 Pls
2242 893 Pls
2245 887 Pls
2247 914 Pls
2250 889 Pls
2252 899 Pls
2255 918 Pls
2258 915 Pls
2260 907 Pls
2263 922 Pls
2266 918 Pls
2268 937 Pls
2271 931 Pls
2273 934 Pls
2276 948 Pls
2279 964 Pls
2281 953 Pls
2284 953 Pls
2286 965 Pls
2289 982 Pls
2292 956 Pls
2294 983 Pls
2297 977 Pls
2299 983 Pls
2302 989 Pls
2305 991 Pls
2307 980 Pls
2310 1003 Pls
2312 1003 Pls
2315 1005 Pls
2318 999 Pls
2320 1015 Pls
2323 1019 Pls
2325 1013 Pls
2328 1016 Pls
2331 1026 Pls
2333 1034 Pls
2336 1028 Pls
2338 1036 Pls
2341 1036 Pls
2344 1046 Pls
2346 1048 Pls
2349 1041 Pls
2351 1046 Pls
2354 1066 Pls
2357 1063 Pls
2359 1060 Pls
2362 1058 Pls
2365 1077 Pls
2367 1084 Pls
2370 1071 Pls
2372 1074 Pls
2375 1096 Pls
2378 1101 Pls
2380 1095 Pls
2383 1091 Pls
2385 1105 Pls
2388 1120 Pls
2391 1123 Pls
2393 1098 Pls
2396 1120 Pls
2398 1136 Pls
2401 1141 Pls
2404 1118 Pls
2406 1137 Pls
2409 1147 Pls
2411 1161 Pls
2414 1150 Pls
2417 1135 Pls
2419 1161 Pls
2422 1186 Pls
2425 1183 Pls
2427 1163 Pls
2430 1178 Pls
2432 1200 Pls
2435 1213 Pls
2438 1191 Pls
2440 1191 Pls
2443 1199 Pls
2445 1248 Pls
2448 1217 Pls
2451 1211 Pls
2453 1213 Pls
2456 1247 Pls
2458 1258 Pls
2461 1233 Pls
2464 1225 Pls
2466 1245 Pls
2469 1287 Pls
2471 1261 Pls
2474 1251 Pls
2477 1256 Pls
2479 1299 Pls
2482 1293 Pls
2484 1282 Pls
2487 1265 Pls
2490 1296 Pls
2492 1328 Pls
2495 1313 Pls
2497 1299 Pls
2500 1301 Pls
2503 1337 Pls
2505 1349 Pls
2508 1334 Pls
2510 1308 Pls
2513 1346 Pls
2516 1375 Pls
2518 1375 Pls
2521 1348 Pls
2523 1345 Pls
2526 1382 Pls
2529 1414 Pls
2531 1391 Pls
2534 1351 Pls
2536 1405 Pls
2539 1419 Pls
2542 1451 Pls
2544 1392 Pls
2547 1388 Pls
2550 1447 Pls
2552 1487 Pls
2555 1449 Pls
2557 1398 Pls
2560 1459 Pls
2563 1486 Pls
2565 1522 Pls
2568 1444 Pls
2570 1433 Pls
2573 1523 Pls
2576 1545 Pls
2578 1500 Pls
2581 1453 Pls
2583 1525 Pls
2586 1555 Pls
2589 1589 Pls
2591 1500 Pls
2594 1501 Pls
2597 1573 Pls
2599 1612 Pls
2602 1564 Pls
2604 1519 Pls
2607 1584 Pls
2610 1612 Pls
2612 1664 Pls
2615 1557 Pls
2617 1581 Pls
2620 1631 Pls
2623 1686 Pls
2625 1632 Pls
2628 1606 Pls
2630 1640 Pls
2633 1674 Pls
2636 1740 Pls
2638 1622 Pls
2641 1669 Pls
2643 1690 Pls
2646 1761 Pls
2649 1707 Pls
2651 1707 Pls
2654 1702 Pls
2656 1740 Pls
2659 1829 Pls
2662 1705 Pls
2664 1756 Pls
2667 1749 Pls
3218 1947 Pls
stroke
grestore
end
showpage
}}%
\put(3037,1947){\makebox(0,0)[r]{$T=1.0$}}%
\put(1900,50){\makebox(0,0){Excitation energy $E$}}%
\put(100,1180){%
\special{ps: gsave currentpoint currentpoint translate
270 rotate neg exch neg exch translate}%
\makebox(0,0)[b]{\shortstack{Free energy $F(E)$}}%
\special{ps: currentpoint grestore moveto}%
}%
\put(3450,200){\makebox(0,0){25}}%
\put(2830,200){\makebox(0,0){20}}%
\put(2210,200){\makebox(0,0){15}}%
\put(1590,200){\makebox(0,0){10}}%
\put(970,200){\makebox(0,0){5}}%
\put(350,200){\makebox(0,0){0}}%
\put(300,2060){\makebox(0,0)[r]{10}}%
\put(300,1809){\makebox(0,0)[r]{8}}%
\put(300,1557){\makebox(0,0)[r]{6}}%
\put(300,1306){\makebox(0,0)[r]{4}}%
\put(300,1054){\makebox(0,0)[r]{2}}%
\put(300,803){\makebox(0,0)[r]{0}}%
\put(300,551){\makebox(0,0)[r]{-2}}%
\put(300,300){\makebox(0,0)[r]{-4}}%
\end{picture}%
\endgroup
\endinput

    \caption{$^1S_0$ energy gap in neutron matter calculated with  
             the CD-Bonn potential compared with the direct calculation 
             from $^1S_0$ phase shifts.}
    \label{fig:fig2}
\end{figure} 

\begin{figure}
    % GNUPLOT: LaTeX picture with Postscript
\setlength{\unitlength}{0.1bp}
\special{!
%!PS-Adobe-2.0
%%Creator: gnuplot
%%DocumentFonts: Helvetica
%%BoundingBox: 50 50 770 554
%%Pages: (atend)
%%EndComments
/gnudict 40 dict def
gnudict begin
/Color false def
/Solid false def
/gnulinewidth 5.000 def
/vshift -33 def
/dl {10 mul} def
/hpt 31.5 def
/vpt 31.5 def
/M {moveto} bind def
/L {lineto} bind def
/R {rmoveto} bind def
/V {rlineto} bind def
/vpt2 vpt 2 mul def
/hpt2 hpt 2 mul def
/Lshow { currentpoint stroke M
  0 vshift R show } def
/Rshow { currentpoint stroke M
  dup stringwidth pop neg vshift R show } def
/Cshow { currentpoint stroke M
  dup stringwidth pop -2 div vshift R show } def
/DL { Color {setrgbcolor Solid {pop []} if 0 setdash }
 {pop pop pop Solid {pop []} if 0 setdash} ifelse } def
/BL { stroke gnulinewidth 2 mul setlinewidth } def
/AL { stroke gnulinewidth 2 div setlinewidth } def
/PL { stroke gnulinewidth setlinewidth } def
/LTb { BL [] 0 0 0 DL } def
/LTa { AL [1 dl 2 dl] 0 setdash 0 0 0 setrgbcolor } def
/LT0 { PL [] 0 1 0 DL } def
/LT1 { PL [4 dl 2 dl] 0 0 1 DL } def
/LT2 { PL [2 dl 3 dl] 1 0 0 DL } def
/LT3 { PL [1 dl 1.5 dl] 1 0 1 DL } def
/LT4 { PL [5 dl 2 dl 1 dl 2 dl] 0 1 1 DL } def
/LT5 { PL [4 dl 3 dl 1 dl 3 dl] 1 1 0 DL } def
/LT6 { PL [2 dl 2 dl 2 dl 4 dl] 0 0 0 DL } def
/LT7 { PL [2 dl 2 dl 2 dl 2 dl 2 dl 4 dl] 1 0.3 0 DL } def
/LT8 { PL [2 dl 2 dl 2 dl 2 dl 2 dl 2 dl 2 dl 4 dl] 0.5 0.5 0.5 DL } def
/P { stroke [] 0 setdash
  currentlinewidth 2 div sub M
  0 currentlinewidth V stroke } def
/D { stroke [] 0 setdash 2 copy vpt add M
  hpt neg vpt neg V hpt vpt neg V
  hpt vpt V hpt neg vpt V closepath stroke
  P } def
/A { stroke [] 0 setdash vpt sub M 0 vpt2 V
  currentpoint stroke M
  hpt neg vpt neg R hpt2 0 V stroke
  } def
/B { stroke [] 0 setdash 2 copy exch hpt sub exch vpt add M
  0 vpt2 neg V hpt2 0 V 0 vpt2 V
  hpt2 neg 0 V closepath stroke
  P } def
/C { stroke [] 0 setdash exch hpt sub exch vpt add M
  hpt2 vpt2 neg V currentpoint stroke M
  hpt2 neg 0 R hpt2 vpt2 V stroke } def
/T { stroke [] 0 setdash 2 copy vpt 1.12 mul add M
  hpt neg vpt -1.62 mul V
  hpt 2 mul 0 V
  hpt neg vpt 1.62 mul V closepath stroke
  P  } def
/S { 2 copy A C} def
end
%%EndProlog
}
\begin{picture}(3600,2160)(0,0)
\special{"
%%Page: 1 1
gnudict begin
gsave
50 50 translate
0.100 0.100 scale
0 setgray
/Helvetica findfont 100 scalefont setfont
newpath
-500.000000 -500.000000 translate
LTa
600 251 M
2817 0 V
600 251 M
0 1858 V
LTb
600 251 M
63 0 V
2754 0 R
-63 0 V
600 437 M
63 0 V
2754 0 R
-63 0 V
600 623 M
63 0 V
2754 0 R
-63 0 V
600 808 M
63 0 V
2754 0 R
-63 0 V
600 994 M
63 0 V
2754 0 R
-63 0 V
600 1180 M
63 0 V
2754 0 R
-63 0 V
600 1366 M
63 0 V
2754 0 R
-63 0 V
600 1552 M
63 0 V
2754 0 R
-63 0 V
600 1737 M
63 0 V
2754 0 R
-63 0 V
600 1923 M
63 0 V
2754 0 R
-63 0 V
600 2109 M
63 0 V
2754 0 R
-63 0 V
600 251 M
0 63 V
0 1795 R
0 -63 V
976 251 M
0 63 V
0 1795 R
0 -63 V
1351 251 M
0 63 V
0 1795 R
0 -63 V
1727 251 M
0 63 V
0 1795 R
0 -63 V
2102 251 M
0 63 V
0 1795 R
0 -63 V
2478 251 M
0 63 V
0 1795 R
0 -63 V
2854 251 M
0 63 V
0 1795 R
0 -63 V
3229 251 M
0 63 V
0 1795 R
0 -63 V
600 251 M
2817 0 V
0 1858 V
-2817 0 V
600 251 L
LT0
3114 1946 M
180 0 V
788 278 M
2 1 V
2 1 V
4 2 V
5 2 V
6 3 V
8 3 V
8 4 V
10 5 V
11 6 V
12 7 V
13 8 V
14 9 V
16 10 V
16 12 V
18 13 V
18 14 V
20 16 V
21 18 V
21 19 V
23 21 V
23 22 V
25 24 V
25 26 V
26 27 V
27 29 V
28 30 V
29 32 V
29 33 V
31 34 V
31 34 V
31 36 V
33 37 V
32 37 V
34 38 V
34 37 V
35 38 V
35 37 V
35 37 V
36 36 V
36 35 V
37 33 V
37 33 V
37 31 V
38 29 V
37 27 V
38 25 V
38 22 V
38 21 V
38 18 V
39 15 V
38 13 V
38 9 V
38 6 V
38 2 V
37 0 V
38 -2 V
37 -5 V
37 -6 V
37 -8 V
36 -11 V
36 -13 V
35 -15 V
35 -22 V
35 -26 V
34 -29 V
34 -32 V
32 -33 V
33 -35 V
31 -37 V
31 -37 V
31 -38 V
29 -39 V
29 -38 V
28 -38 V
27 -38 V
26 -38 V
25 -36 V
25 -36 V
23 -34 V
23 -33 V
21 -32 V
21 -31 V
20 -29 V
18 -27 V
18 -26 V
16 -24 V
16 -22 V
14 -20 V
13 -19 V
12 -18 V
11 -15 V
10 -14 V
8 -12 V
8 -10 V
6 -9 V
5 -7 V
4 -5 V
2 -4 V
2 -2 V
LT1
3114 1846 M
180 0 V
788 282 M
2 1 V
2 1 V
4 2 V
5 3 V
6 3 V
8 3 V
8 5 V
10 5 V
11 7 V
12 7 V
13 8 V
14 10 V
16 11 V
16 12 V
18 13 V
18 15 V
20 16 V
21 18 V
21 19 V
23 22 V
23 22 V
25 25 V
25 26 V
26 27 V
27 29 V
28 30 V
29 32 V
29 32 V
31 34 V
31 35 V
31 36 V
33 36 V
32 36 V
34 37 V
34 37 V
35 37 V
35 37 V
35 36 V
36 35 V
36 34 V
37 33 V
37 32 V
37 30 V
38 28 V
37 26 V
38 24 V
38 22 V
38 19 V
38 17 V
39 13 V
38 12 V
38 8 V
38 4 V
38 2 V
37 -2 V
38 -4 V
37 -8 V
37 -10 V
37 -14 V
36 -17 V
36 -19 V
35 -22 V
35 -25 V
35 -27 V
34 -28 V
34 -31 V
32 -32 V
33 -34 V
31 -36 V
31 -36 V
31 -37 V
29 -37 V
29 -36 V
28 -37 V
27 -37 V
26 -36 V
25 -35 V
25 -34 V
23 -32 V
23 -31 V
21 -29 V
21 -28 V
20 -26 V
18 -24 V
18 -22 V
16 -21 V
16 -19 V
14 -17 V
13 -15 V
12 -14 V
11 -12 V
10 -10 V
8 -10 V
8 -7 V
6 -7 V
5 -5 V
4 -4 V
2 -3 V
2 -1 V
LT4
3114 1746 M
180 0 V
788 327 M
2 0 V
2 0 V
4 0 V
5 -1 V
6 0 V
8 0 V
8 1 V
10 1 V
11 1 V
12 2 V
13 2 V
14 4 V
16 4 V
16 5 V
18 7 V
18 8 V
20 10 V
21 12 V
21 13 V
23 15 V
23 17 V
25 20 V
25 21 V
26 24 V
27 26 V
28 30 V
29 32 V
29 33 V
31 36 V
31 37 V
31 39 V
33 38 V
32 38 V
34 38 V
34 39 V
35 38 V
35 37 V
35 35 V
36 35 V
36 34 V
37 33 V
37 32 V
37 34 V
38 33 V
37 32 V
38 30 V
38 28 V
38 24 V
38 22 V
39 19 V
38 16 V
38 13 V
38 10 V
38 6 V
37 4 V
38 0 V
37 -3 V
37 -8 V
37 -11 V
36 -13 V
36 -17 V
35 -19 V
35 -18 V
35 -21 V
34 -23 V
34 -28 V
32 -31 V
33 -39 V
31 -50 V
31 -53 V
31 -53 V
29 -53 V
29 -51 V
28 -49 V
27 -40 V
26 -36 V
25 -35 V
25 -32 V
23 -31 V
23 -29 V
21 -28 V
21 -26 V
20 -25 V
18 -23 V
18 -22 V
16 -21 V
16 -20 V
14 -18 V
13 -17 V
12 -16 V
11 -14 V
10 -13 V
8 -11 V
8 -10 V
6 -9 V
5 -7 V
4 -5 V
2 -3 V
2 -3 V
stroke
grestore
end
showpage
}
\put(3054,1746){\makebox(0,0)[r]{Phase Shifts}}
\put(3054,1846){\makebox(0,0)[r]{CD-Bonn, Neutron Matter}}
\put(3054,1946){\makebox(0,0)[r]{CD-Bonn Nuclear Matter}}
\put(2008,21){\makebox(0,0){$k_F$ (fm$^{-1}$)}}
\put(100,1180){%
\special{ps: gsave currentpoint currentpoint translate
270 rotate neg exch neg exch translate}%
\makebox(0,0)[b]{\shortstack{Pairing gap $\Delta (k_F)$ (MeV)}}%
\special{ps: currentpoint grestore moveto}%
}
\put(3229,151){\makebox(0,0){1.4}}
\put(2854,151){\makebox(0,0){1.2}}
\put(2478,151){\makebox(0,0){1}}
\put(2102,151){\makebox(0,0){0.8}}
\put(1727,151){\makebox(0,0){0.6}}
\put(1351,151){\makebox(0,0){0.4}}
\put(976,151){\makebox(0,0){0.2}}
\put(600,151){\makebox(0,0){0}}
\put(540,2109){\makebox(0,0)[r]{5}}
\put(540,1923){\makebox(0,0)[r]{4.5}}
\put(540,1737){\makebox(0,0)[r]{4}}
\put(540,1552){\makebox(0,0)[r]{3.5}}
\put(540,1366){\makebox(0,0)[r]{3}}
\put(540,1180){\makebox(0,0)[r]{2.5}}
\put(540,994){\makebox(0,0)[r]{2}}
\put(540,808){\makebox(0,0)[r]{1.5}}
\put(540,623){\makebox(0,0)[r]{1}}
\put(540,437){\makebox(0,0)[r]{0.5}}
\put(540,251){\makebox(0,0)[r]{0}}
\end{picture}

	\caption{$^1S_0$ energy gap in nuclear matter calculated with  
                 the CD-Bonn potential  
                 compared with the direct calculation from the 
                 $^1S_0$ np and pp phase shifts.  Also shown are the results 
     for neutron matter with the CD-Bonn potential.}
    \label{fig:fig3}
\end{figure}	

\end{document}















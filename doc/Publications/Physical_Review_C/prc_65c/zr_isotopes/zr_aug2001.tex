%To: prc@aps.org
%%%%%%%%%%%%%%%%%%%%%%%%%%%%%%%%%%%%%%%%%%%%%%%%%%%%%%%%%%%%%%%%
%
%   Submission The Physical Review C
%   Author: Fotiades et al.
%   Date: 
% 
%   This File contains the LaTeX version of our paper.
% 
%   We will e-mail in separate postscript files our figures.
%
%   All correspondence should be addressed to:
%   
%        Jolie A Cizewski
%        Department of Physics and Astronomy
%        Rutgers University
%        136 Frelinghuysen Road
%        Piscataway, New Jersey 08854-8019  USA
%
%   Fax:        732-445-4343
%   Phone:      732-445-3884
%   E-mail:     cizewski@physics.rutgers.edu
%
%   Thank you
%%%%%%%%%%%%%%%%%%%%%%%%%%%%%%%%%%%%%%%%%%%%%%%%%%%%%%%%%%%%%%%%
\documentstyle[preprint,aps]{revtex}
\begin{document}
\draft
\preprint{RU}
\title{ High-spin excitations in $^{92,93,94,95}$Zr }
\author{N.~Fotiades,$^{1,2}$ J.~A.~Cizewski,$^2$ J.~A.~Becker,$^3$
 L.~A.~Bernstein,$^3$ D.~P.~McNabb,$^3$ W.~Younes,$^3$ R.~M.~Clark,$^4$
 P.~Fallon,$^4$ I.~Y.~Lee,$^4$ and A.~O.~Macchiavelli,$^4$  }

\address{$^{1}$Los Alamos National Laboratory, Los Alamos, New Mexico 87545 }
\address{$^{2}$Department of Physics and Astronomy, Rutgers University, New Brunswick, New Jersey 08903}
\address{$^{3}$Lawrence Livermore National Laboratory, Livermore, California 94550}
\address{$^{4}$Nuclear Science Division, Lawrence Berkeley National Laboratory, Berkeley, California 94720}

\date{\today}

\maketitle

\begin{abstract}
The level structure of several Zr isotopes near $A$~=~94 has been studied in
the fission of the compound nucleus $^{197}$Pb, formed in the $^{24}$Mg +
$^{173}$Yb reaction at 134.5 MeV. Sequences of transitions, observed in
coincidence with known transitions in the complementary Mo fragments, have been
assigned to $^{93,95}$Zr. The previously known level scheme of
$^{94}$Zr has been considerably extended to higher excitations, and exhibits
structural similarities to the level scheme of $^{92}$Zr up to spin 10$\hbar$.
The level schemes of $^{93}$Zr and $^{95}$Zr can be generally interpreted as
the coupling of a $d_{5/2}$ neutron to the levels of $^{92}$Zr and/or
$^{94}$Zr, and of $^{96}$Zr, respectively. The observed experimental states are
compared with theoretical shell-model calculations.
\end{abstract}

\pacs{PACS number(s): 23.20.Lv; 27.60+j}

\narrowtext

\section{ INTRODUCTION }

The Zr isotopes between $^{90}$Zr and $^{96}$Zr are expected to be 
spherical with structural similarities which can be interpreted in terms of simple
excitations across the subshell closures at $Z=40$ and $N=50,56$. On the other
hand, the comparison of the relative excitations of certain states between these
nuclei can give information on the interplay between proton and neutron 
excitations across these gaps.

 Many properties of these nuclei, including energy spectra, have been
well reproduced in shell-model calculations using a $\pi [ p_{1/2}, g_{9/2} ]
\nu [ d_{5/2}, s_{1/2} ]$ space~\cite{Glo}, which support the picture of simple
excitations across the subshells. A more extended shell-model space, comprising
the $\pi f_{5/2}$ and $\pi p_{3/2}$ orbitals, has been used to reproduce better
the measured $B(E2)$ values for the first positive-parity states in
$^{92,94}$Zr~\cite{Mac}. More recently, $g$-factor measurements~\cite{Jak} 
have confirmed the expected dominance of the neutron
$(d_{5/2})^{2}$ particle/hole configuration in the 2$^+$ and 4$^+$
states in $^{92}$Zr / $^{94}$Zr, respectively. Calculations in a large shell-model 
space by Zhang, Wang, and Gu~\cite{Cha} support a 70-80$\%$ 
contribution of the $d_{5/2}$ neutron configuration in these excitations.
There have also been calculations by A. Holt and coworkers 
in a large space~\cite{holt} which predict high-spin states.  

It is difficult to study Zr isotopes with $A>90$ to moderate
and high spins because they are too close to the line of stability to be readily
populated in reactions which bring in high angular momentum. Alternatively,
these isotopes can be studied via the prompt $\gamma$-ray spectroscopy of
fission fragments following fusion reactions of much heavier nuclei. Such
methods have been used recently to collect information on high-spin
states of nuclei near the line of stability~\cite{Fot}. Identification of
high-spin excitations in Zr isotopes near stability is important to 
determine what extensions of the shell-model space are needed 
in this mass region.

In the present work excitations in $^{92-95}$Zr have been studied as
fission products from a fusion reaction with a $^{197}$Pb compound nucleus.
This method enabled the use of coincidences with transitions in the
complementary Mo fragments to assign transitions to $^{93,95}$Zr and construct
level schemes up to $\sim$5~MeV excitation energy. Extensions of the level
schemes of $^{92,94}$Zr to higher excitations were also possible.

\section{ EXPERIMENT}

 The $^{197}$Pb compound nucleus was formed in the $^{24}$Mg + $^{173}$Yb
reaction with a 134.5-MeV beam from the 88-Inch Cyclotron Facility at Lawrence
Berkeley National Laboratory. The target was 1 mg/cm$^{2}$ in areal density and
consisted of isotopically enriched $^{173}$Yb evaporated on a 7 mg/cm$^{2}$ gold
backing.

The Gammasphere array (92 Ge detectors) was used for $\gamma$-ray spectroscopy.
A symmetrized, three-dimensional cube was constructed to investigate coincidence
relationships between the transitions. All previously known~\cite{TOI} Zr
isotopes, from $^{92}$Zr to $^{98}$Zr, and Mo isotopes, from $^{94}$Mo to
$^{102}$Mo, were identified in the present analysis. Additional information on
the experiment and the analysis of the data are given in~\cite{Foti}.

\section{ EXPERIMENTAL RESULTS}

High-spin excitations in $^{92}$Zr have been previously studied in heavy-ion
fusion-evaporation reactions and the level scheme extended up to $\sim$8~MeV
excitation energy~\cite{Bro,Kor,NDS2}. Additional states at low excitations
are also known from light-ion induced reactions and $\beta$-decay
studies~\cite{NDS2}. In the present work only three new weak 
transitions have been
added to the level scheme in Fig.~1(a): the 471.3-keV ($6^{+} \rightarrow
5^{-}$), 559.6-keV ($6^{+} \rightarrow 4_{2}^{+}$) and 1681.3-keV ($18^{+}
\rightarrow 16^{+}$) transitions. With the addition of the
471.3- and 559.6-keV transitions, the deexcitation path in $^{92}$Zr now
resembles the one observed in $^{94}$Zr. Although in the present
data the 1463-keV ($4_2^{+} \rightarrow 2_{1}^{+}$) transition in
Fig.~1(a) could not be separated from the strong 1461.8-keV 
($6^{+} \rightarrow 4^{+}$) transition, the placement of the 
1463-keV transition had been previously established~\cite{TOI}.

The only levels established previously in $^{93}$Zr were low-spin states
observed in light-ion induced reactions or $\beta$-decay studies~\cite{NDS3}.
Although transitions had been assigned to $^{93}$Zr in an earlier study of
fission fragments following a heavy-ion fusion evaporation reaction~\cite{Por},
no levels were established because of the limited $\gamma$-ray energy range,
$E_{\gamma} <$ 2~MeV. The assignments of these transitions to $^{93}$Zr are
confirmed in the present work by coincidences with Mo fragments and 
the level scheme in Fig.~2 was
constructed. The present data provide information up to 
$E_{\gamma} =$ 2.7~MeV. No
additional transition could be assigned to $^{93}$Zr in the 
2.0~MeV $< E_{\gamma} <$ 2.7~MeV
region. 
%The presence of more than one
%parallel path in the proposed level scheme of $^{93}$Zr connecting the same
%levels supports the absence of any transition with $E_{\gamma} >$ 2~MeV
%in-between these levels. 

 In $^{94}$Zr the levels below the 5$^{-}$ state at 2605~keV 
were previously observed in light-ion reactions~\cite{NDS4}. In the present
work the yrast levels above this state were established up to $\sim$9~MeV
in excitation energy, as shown Fig.~1(b). 
Of particular interest are the 6$^{+}$,
8$^{+}$, 10$^{+}$ and (7$^{-}$) states, which can be directly compared to the
corresponding states in  $^{92}$Zr and to theoretical shell model calculations.

The low-spin levels in $^{95}$Zr were established from 
light-ion induced reactions and $\beta$-decay
studies~\cite{NDS5}.   
Transitions were assigned to $^{95}$Zr in the previous 
fission fragment studies~\cite{Por}. These transitions were 
observed in coincidence with known
transitions in Mo complementary fragments in the present work 
and a comprehensive level scheme,
based on coincidence relations and relative intensities, is proposed in Fig.~3.

The $\gamma$-ray transitions assigned to $^{93,94,95}$Zr are summarized in 
Table~I; intensities reported in Table~I were not corrected for internal
conversion because of limited knowledge of the multipolarities. However, since
the internal conversion coefficients for low-multipolarity, low-energy
transitions in Zr isotopes are small, this correction is not expected to change
significantly the intensities reported here. The possible exceptions are the
65.6-keV transition in $^{93}$Zr and the 103.0-keV transition in $^{95}$Zr.
The intensities of the 774.3- and 1159.0-keV
transitions of $^{95}$Zr in Table~I are equal within error; hence, their
placement in the level scheme in Fig.~3 could be interchanged. The intensities
reported in Table~I for transitions of $^{95}$Zr are in general agreement with
those quoted in Ref.~\cite{Por}. However, there are significant differences
between the intensities of the transitions of $^{93}$Zr observed in the present
work and those in Ref.~\cite{Por}.  Given the limited details presented 
in Ref.~\cite{Por}, the present measurements are adopted.

Spin and parity assignments of the new levels assigned to $^{92-95}$Zr
in the present work are difficult to deduce because of the lack of directional
correlation information for the fission products. However, the tentative spin
assignments in Figs.~1, 2, and 3 are supported by a comparison with known
states in the neighboring Zr isotopes, as well as the results of theoretical
shell model calculations, as discussed below.

\section{ DISCUSSION}

 The resemblance between the levels in $^{92}$Zr and $^{94}$Zr up to spin
10$\hbar$ in Fig.~1 is striking, which suggests that similar orbitals 
are involved.

The dominant neutron $(d_{5/2})^2$ particle/hole parentage of the $0^+, 2^+, 4^+$ states in $^{92}$Zr / $^{94}$Zr, respectively,
has been experimentally established by measurement of the $g$-factors of these
states~\cite{Jak} and is supported by shell-model calculations~\cite{Cha}.

 A large gap is expected between the first 4$^{+}$ and 6$^{+}$ states, as 
previously observed in $^{92}$Zr~\cite{Bro}, due to the energy needed to excite a
pair of protons from the $p_{1/2}$ to the $g_{9/2}$ orbitals to form the 
$(\pi g_{9/2})^2_{6^+,8^+}$ states.  The same gap is
present in $^{94}$Zr. The 6$^{+}$ and 8$^{+}$ states which originate from this 
proton excitation are pushed to slightly higher excitations in 
$^{94}$Zr compared to $^{92}$Zr. The energy spacings between the
8$^{+}$, 10$^{+}$ and 12$^{+}$ states in $^{92}$Zr are 
analogous to those between the 0$^{+}$, 2$^{+}$ and 4$^{+}$ states, supporting
the interpretation of these excitations as the coupling of the 
two $g_{9/2}$ protons to the
neutron $d_{5/2}$ pair.  A similar structure is observed in 
$^{94}$Zr up to the 10$^{+}$
state. However, none of the levels above this state in $^{94}$Zr is a good candidate for a
12$^{+}$ state, which corresponds to full alignment of the spins of the 
nucleons which form the $\pi (g_{9/2})^2 \nu (d_{5/2})^2$ configuration. 
Based on the energy systematics of the $6^+$ to $10^+$ states, 
the $12^+$ state in $^{94}$Zr would be about 150-350~keV higher 
in excitation than the 4947-keV 12$^{+}$ state in
$^{92}$Zr. In contrast, the sequence of levels above the 
10$^{+}$ state in $^{94}$Zr
is entirely different from the corresponding ones in $^{92}$Zr, 
which suggests excitations
of a different nature. 

A large gap in energy between the 4$^+$ and 5$^-$ states is also observed in 
$^{90}$Zr~\cite{TOI} and $^{92}$Zr, 
since the 5$^-$ state includes a proton ($g_{9/2} p_{1/2}$)
excitation~\cite{Bro}.
The 5$^{-}$ state in $^{94}$Zr is the 
analog of the 5$^{-}$
states in $^{90}$Zr~\cite{TOI} and $^{92}$Zr. 
The spacing between the 5$^{-}$ and (7$^{-}$) states in $^{94}$Zr is
similar to the spacing between the 0$^{+}$ and 2$^{+}$ states in this isotope, 
as is also the case for the 5$^{-}$ and 7$^{-}$ states in $^{92}$Zr and 
$^{90}$Zr.  These spacings suggest that the 7$^{-}$ states in all three
isotopes arise from the coupling of the neutron $d_{5/2}$ pair to
the $\pi (g_{9/2} p_{1/2})$ configuration.

The spin assignments for $^{93}$Zr are suggested by a weak coupling 
of the valence $d_{5/2}$ neutron to the excitations in the 
$^{92,94}$Zr cores, as displayed in Fig.~4.  The $5/2^+$ ground state 
and ($9/2^+$) state at 950~keV can be readily associated with the 
$0^+$ and $2^+$ states of the cores, which are 
predominantly~\cite{Cha} $(\nu d_{5/2})^2$ configurations.  
Similarly, the proposed $(17/2^+_1)$ and $(21/2^+_1)$ states 
can be associated with coupling of the valence $d_{5/2}$ neutron 
to the $6^+$ and $8^+$ $(\pi g_{9/2})^2$ states of the core.  
The decay pattern of the $6_1^+$ state in $^{92}$Zr to 
lower-lying $5^-$, $4_1^+$, and $4_2^+$ states is similar 
to that observed for the $(17/2_1^+)$ state, which supports a 
$(15/2^-)$ assignment for the 2485-keV level and $(13/2^+)$ 
assignments for the states at 1655 and 2774~keV.  
However, the reduced branching ratios for the decay of the 
$6_1^+$ state in $^{92}$Zr and the $(17/2^+)$ state in $^{93}$Zr 
%summarized in Table~II, 
are different.  This is not unexpected 
since the $4^+$ state in the core is predominantly 
$(\nu d_{5/2})^2$, while the $13/2_1^+$ state in $^{93}$Zr 
must include components of the $4^+$ core state outside of the 
$(\nu d_{5/2})^2$ configuration, since a $13/2^+$ state cannot 
be generated from $(\nu d_{5/2})^3$.  This could explain the 
higher excitation energy of the $(13/2_1^+)$ state in $^{93}$Zr 
compared to the $4_1^+$ states in $^{92,94}$Zr.  
That the energy of the proposed $(13/2_2^+)$ state in 
$^{93}$Zr is also higher than the $4_2^+$ states in the cores 
also suggests that core components outside of $(\nu d_{5/2})^2$ 
are important in this excitation in $^{93}$Zr.  
The proposed $(11/2^-)$ state at 2374-keV could be associated 
with the 2363(10)~keV, $9/2^-,11/2^-$ state observed previously in 
light-ion transfer reactions~\cite{NDS3}.  

$^{95}$Zr has only one neutron less than the subshell closure at 
$Z=40$ and $N=56$. Hence, the low-lying states are expected 
to originate from the coupling of a $d_{5/2}$ neutron hole 
to the levels of the $^{96}$Zr core. However, the lack of 
directional correlations of the $\gamma$~rays makes even 
tentative spin-parity assignments difficult. The coupling of 
the $d_{5/2}$ neutron hole to the $0^+, 2^+, 3^-$ levels of the 
$^{96}$Zr core is a likely assignment only for the 
5/2$^{+}$, (9/2$^{+}$) and (11/2$^{-}$) states, respectively, of $^{95}$Zr.  
The proposed $(11/2^-)$ state at 2022~keV could be the 
$9/2^-,11/2^-$ state at 2025~keV previously observed in 
light-ion transfer reactions~\cite{NDS5} as a candidate for 
$\nu h_{11/2}$ strength.

Shell-model calculations for all Zr isotopes discussed in the present work have
been carried out by Gloeckner using a $\pi [ p_{1/2}, g_{9/2} ] \nu [ d_{5/2},
s_{1/2} ]$ space~\cite{Glo}, as well as Holt and coworkers~\cite{holt} 
using a larger neutron $s_{1/2}d_{5/2}g_{7/2}h_{11/2}$ space, 
which provide predictions for high-spin states in these nuclei. At the
time that these calculation were performed only high-spin states in $^{91}$Zr and
$^{92}$Zr were known for a direct comparison to the calculation~\cite{Glo,holt,Bro}.
In general, the comparison to the experimental energies showed that these
calculations succeed in reproducing the positive-parity states, while
underestimating the excitation of the negative-parity states. With many high-spin
states established in $^{93}$Zr, $^{94}$Zr and $^{95}$Zr isotopes in the present
work this comparison can be pursued further. 

In Fig.~5 the experimentally
observed states of $^{94}$Zr are compared to the theoretical calculations of refs.~\cite{Glo,holt}.
The yrast positive-parity states up to 6$^{+}$ are reproduced 
very well by the calculations of ref.~\cite{Glo}. 
In particular, the observed 6$^{+}$ state lies exactly where 
predicted. The 8$^{+}$ and 10$^{+}$ states, as well as all 
negative-parity states, are underestimated by the calculations, 
although the spacing between the 8$^+$ and 10$^+$ states is reproduced.  
In contrast, essentially all of the excitations are under predicted by the 
calculations of ref.~\cite{holt}, including the spacing between the 
8$^+$ and 10$^+$ states. 

In Fig.~6 the
experimentally observed states of $^{93,95}$Zr, for which tentative 
spin-parity assignments have been proposed, are compared to the 
theoretical calculations of ref.~\cite{holt}.
For $^{93}$Zr the agreement between experiment and theory is 
comparable to that for the $^{94}$Zr core.  
Unfortunately, the theoretical predictions for $^{95}$Zr above 2~MeV 
can provide no guidance in proposing tentative spin-parity values 
to the levels displayed in Fig.~3.
The spins and parities need to be determined for the extensive level schemes 
deduced for $^{93,95}$Zr before additional comparison between 
theory and experiment would be fruitful.

Finally, we would like to highlight the high excitation energies and 
angular momenta to
which the $^{92}$Zr and $^{94}$Zr isotopes have been observed in this study.
Studies of fission fragments produced in fusion-evaporation reactions in general
allow up to spin $\sim$14$\hbar$ and up to $\sim$6~MeV excitation energy to be 
populated in a fission fragment (see, for example,
Refs.~\cite{Por,JPG}). Of course, the population of 
high-spin states depends strongly
on the particular reaction and the fissioning compound nucleus.
Nevertheless, the observation of a  (18$^{+}$) state at
more than 9~MeV excitation energy in the fission fragment $^{92}$Zr is quite unusual.

\section{ SUMMARY}

 In conclusion, transitions have been assigned to $^{93}$Zr and $^{95}$Zr based on
coincidences with known transitions in Mo isotopes, produced as fission products
in a fusion-evaporation reaction. Level schemes up to $\sim$4.5~MeV excitation
energy have been constructed for these isotopes and the previously known level schemes
of $^{92}$Zr and $^{94}$Zr have been enriched and extended to higher excitations.
The similarities in the levels of $^{92}$Zr and $^{94}$Zr up to $\sim$3.5~MeV
excitation energy are striking, while the states in $^{93}$Zr up to the same
energy can be interpreted as originating from the coupling of the odd $d_{5/2}$
neutron to the levels of $^{92}$Zr and $^{94}$Zr. The states in $^{95}$Zr up
to $\sim$2~MeV excitation energy can also be interpreted as the coupling of
the odd $d_{5/2}$ neutron hole to the levels of the $^{96}$Zr core. 
The breaking of
the subshell closure in $^{95}$Zr results in a steep drop of the excitation 
energies of the rest of the states compared to those in $^{96}$Zr. The observed
experimental states are compared with theoretical 
shell-model calculations, which
reproduce well the excitation energies of the positive-parity states in the
$^{92,93,94}$Zr isotopes and underestimate the negative-parity states.

\acknowledgments

We thank A. Holt and M. Hjorth-Jensen for providing calculations 
of high-spin states in Zr isotopes, prior to publication.
This work has been supported in part by the 
National Science Foundation (Rutgers) and U.S. Department of Energy under
Contract Nos. W-7405-ENG-36 (LANL), W-7405-ENG-48 (LLNL), and AC03-76SF00098
(LBNL).

\begin{references}
\bibitem{Glo} D. H. Gloeckner, Nucl. Phys. {\bf A253}, 301 (1975).
\bibitem{Mac} H. Mach, E. K. Warburton, W. Krips, R. L. Gill, and M.
 Moszy\'{n}ski, Phys. Rev. C {\bf 42}, 568 (1990).
\bibitem{Jak} G. Jakob, N. Benczer-Koller, J. Holden, G. Kumbartzki, T. J.
 Mertzimekis, K.-H. Speidel, C. W. Beausang, and R. Kr\"ucken, Phys. Lett.
 B {\bf 468}, 13 (1999).
\bibitem{Cha} C. Zhang, S. Wang, and J. Gu, Phys. Rev. C {\bf 60}, 054316 (1999).
\bibitem{holt} A. Holt, T. Engeland, M. Hjorth-Jensen, and E. Osnes, Phys. Rev. C {\bf 61}, 064318 (2000) and private communication.
\bibitem{Fot} N. Fotiades {\it et al.}, Phys. Rev. C {\bf 58}, 1997 (1998), and
 references therein.
\bibitem{TOI} R. B. Firestone, V. S. Shirley, C. M. Baglin, S. Y. Frank Chu,
 and J. Zipkin,  {\it Table of Isotopes}, (Wiley, New York, 1996), and 
 references therein.
\bibitem{Foti} N. Fotiades {\it et al.}, Phys. Rev. C {\bf 57}, 1624 (1998).
\bibitem{Bro} B. A. Brown, D. B. Fossan, P. M. S. Lesser, and A.R. Poletti,
 Phys. Rev. C {\bf 14}, 602 (1976).
\bibitem{Kor} G. Korschinek, M. Fenzl, H. Hick, A. J. Kreiner, W. Kutschera,
 E. Nolte, and H. Morinaga, in {\it Proc. Intern. Conf. Nucl. Structure},
 Tokyo, Vol. 1, p. 326 (1977).
\bibitem{NDS2} C. M. Baglin, Nucl. Data Sheets {\bf 66}, 347 (1992). %% 92Zr
\bibitem{NDS3} C. M. Baglin, Nucl. Data Sheets {\bf 80},   1 (1997). %% 93Zr
\bibitem{Por} M.-G. Porquet {\it et al.}, Acta Phys. Pol. B {\bf 27}, 179 (1996).
\bibitem{NDS4} J. K. Tuli, Nucl. Data Sheets {\bf 66},   1 (1992). %% 94Zr
\bibitem{NDS5} T. W. Burrows, Nucl. Data Sheets {\bf 68}, 635 (1993). %% 95Zr
\bibitem{JPG} N. Fotiades {\it et al.}, Physica Scripta {\bf T88}, 127 (2000).
\end{references}

\begin{table}
\caption[]{ Energies and relative intensities of transitions assigned to $^{93,94,95}$Zr.}
\begin{tabular}{cccccccc}
\hspace{1.2cm} & \hspace{1.2cm} & \hspace{0.5cm} & \hspace{1.2cm} & \hspace{1.2cm}
 & \hspace{0.5cm} & \hspace{1.2cm}  & \hspace{1.2cm} \\
\multicolumn{2}{c}{$^{93}$Zr} & & \multicolumn{2}{c}{$^{94}$Zr} &&
\multicolumn{2}{c}{$^{95}$Zr} \\
Energy$^{a}$  &  Intensity &  & Energy$^{a}$  &  Intensity &  &
Energy$^{a}$  &  Intensity \\
 (keV) & ($^{\circ}/_{\circ\circ}$) & & (keV) & ($^{\circ}/_{\circ\circ}$) & &
 (keV) & ($^{\circ}/_{\circ\circ}$) \\
 & & & & & & & \\  \hline
 & & & & & & &  \\
  65.6 & 250(100)& & 145.7 &  19(3)  & & 103.0 & 202(80)  \\ [0.2cm]
 111.2 & 360(70) & & 152.3 &  66(15) & & 115.9 & $\equiv$1000 \\ [0.2cm]
 115.4 & 162(50) & & 202.3 &  85(20) & & 177.9 & 104(40)  \\ [0.2cm]
 214.8 &  41(5)  & & 313.6 &  88(20) & & 208.1 & 302(60)  \\ [0.2cm]
 274.9 & 530(70) & & 333.1 & 101(30) & & 229.4 & 710(50)  \\ [0.2cm]
 325.7 & 369(60) & & 364.9 & 112(30) & & 240.8 & 276(50)  \\ [0.2cm]
 391.6 & 126(30) & & 489.2 & 560(90) & & 270.7 & 131(30)  \\ [0.2cm]
 503.4 & 203(40) & & 537.2 &  70(20) & & 425.3 & 102(30)  \\ [0.2cm]
 646.9 &  60(5)  & & 550.6 & 581(80) & & 556.2 & 193(50)  \\ [0.2cm]
 705.0 & 191(50) & & 629.3 &  17(3)  & & 561.4 & 240(60)  \\ [0.2cm]
 949.8 & $\equiv$1000 & & 683.3 & 58(10) & & 603.6 & 241(40) \\ [0.2cm]
1333.9 & 155(30) & & 736.8 &  54(10) & & 607.5 & 242(80)  \\ [0.2cm]
1424.1 & 382(50) & & 782.0 &  21(4)  & & 774.3 &  68(20)  \\ [0.2cm]
1823.8 &  54(10) & & 812.5 & 391(80) & & 815.4 & 158(30)  \\ [0.2cm]
       &         & & 837.4 & 276(50) & & 836.8 & 220(40)  \\ [0.2cm]
       &         & & 847.7 & 354(60) & & 877.0 & 129(30)  \\ [0.2cm]
       &         & & 860   & $<$10   & &1045.3 & 161(30)  \\ [0.2cm]
       &         & & 918.7 & $\equiv$1000 & &1056.4 &  60(10)      \\ [0.2cm]
       &         & &1011.6 &  99(30) & &1159.0 &  62(20)  \\ [0.2cm]
       &         & &1135.5 & 287(50) & &1676.3 & 800(100) \\ [0.2cm]
       &         & &1188.8 & $<$15   & &1792.3 &  90(30)  \\ [0.2cm]
       &         & &1194.4 &  80(20) & &       &          \\ [0.2cm]
       &         & &1410.8 & 351(50) & &       &          \\ [0.2cm]
       &         & &1672.9 &  15(3)  & &       &          \\ [0.2cm]
\end{tabular}
\end{table}
\noindent
 $^{a}$The uncertainties of the $\gamma$-ray energies vary from 0.2 to
 0.4 keV for the strong transitions and from 0.6 to 0.8 keV for the
 weakest ones.\\
\noindent
$^{b}$ Most likely the 65.6~keV transition is of $M1$ or $E1$ character; 
internal conversion correction for $M1$ multipolarity could 
increase its intensity by a factor of
$\sim$1.5. The 103.0~keV transition is likely of $E2$ multipolarity, because no
cross-over transitions bypassing the 3955- and/or 4058-keV levels were observed.
In this case the correction for internal conversion would approximately double
the intensity reported in Table~1, which supports its placement below the
4058-keV level in Fig.~3.

\begin{figure}
\caption[]{Level schemes of $^{92}$Zr and $^{94}$Zr as obtained in the present
work. Transition and excitation energies are in keV. The widths of the
arrows are proportional to the intensities of the transitions.} 
%relative to the $2^{+} \rightarrow 0^{+}$ transitions. }
\end{figure}

\begin{figure}
\caption[]{Level scheme assigned to $^{93}$Zr in the present work.
 Transition and excitation energies are in keV.} 
%The widths of the arrows are proportional to the intensities of the transitions. }
\end{figure}

\begin{figure}
\caption[]{Level scheme assigned to $^{95}$Zr in the present work.
 Transition and excitation energies are in keV.} 
%The widths of the arrows are proportional to the intensities of the transitions. }. 
\end{figure}

\begin{figure}
\caption[]{ Comparison between the excitations in $^{92,93,94}$Zr.}
% (from Figs.~1 and 2). The excitation energy (in keV) and the spin of the states are given. }
\end{figure}

%\begin{figure}
%\caption[]{ Comparison between excitations in the lower positive parity (left) and negative
% parity (right) states in $^{95}$Zr and $^{96}$Zr (from Fig.~3 and
% Ref.~\cite{TOI}, respectively). The excitation energy (in keV) and the spin of
% the states are given. }
%\end{figure}

\begin{figure}
\caption[]{ Comparison between $^{94}$Zr states observed experimentally in the present
 work and theoretical shell-model predictions of Ref.~\cite{Glo}, Figure 6 and Ref.~\cite{holt}.} 
%The spin and parity of the states are given. }
\end{figure}

\begin{figure}
\caption[]{ Comparison between $^{93,95}$Zr states observed experimentally in the present
 work and theoretical shell-model predictions~\cite{holt}.}
%The spin and parity of the states are given. }
\end{figure}

\end{document}



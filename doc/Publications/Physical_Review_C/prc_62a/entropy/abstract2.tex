\documentclass[12pt]{article}
\renewcommand{\thefootnote}{\fnsymbol{footnote}}

\title{Thermodynamical properties of rare earth nuclei: the melting of pair 
correlations}
\author{A. Schiller\footnotemark[1] \footnotemark[2],
M. Guttormsen\footnotemark[2], M. Hjorth-Jensen\footnotemark[2],
E. Melby\footnotemark[2],\\ 
J. Rekstad\footnotemark[2], S. Siem\footnotemark[2]}

\date{}

\begin{document}

\maketitle

\footnotetext[1]{Lawrence Livermore National Laboratory}
\footnotetext[2]{University of Oslo, Oslo, Norway}

The level density at low spin in the $^{161,162}$Dy and $^{171,172}$Yb nuclei
has been extracted from primary $\gamma$ rays following a ($^3$He,$\alpha$) 
reaction \cite{SB01}. To our knowledge, the experimental method \cite{SB00} is 
by far the most accurate and least model dependent method to obtain level 
densities (and radiative strength functions) at energies below the neutron 
binding energy. Using the level density as a basis, partition functions for the
nuclei can be calculated within the microcanonical and the canonical ensemble.

Observed structures in the microcanonical caloric curve and heat capacity can
be interpreted as fingerprints of the breaking of Cooper pairs 
\cite{MB99,MG01}. On the other hand, these structures are completely smeared
out in the canonical partition function, due to the Laplace transformation
involved. However, a new feature emerges: the S-shape in the canonical heat
capacity curve indicating a second order phase transition from a strongly 
paired to an unpaired phase \cite{SB01}. This feature has also been obtained
theoretically in shell model Monte Carlo calculations \cite{LA00} as well as by
a simple model \cite{GH01}. Furthermore, theoretical calculations and 
experiments on small superconducting particles yield quite similar signals for
the phase transition from a superconductive to a normal conductive state
\cite{LA93}.

Differences in the microcanonical and canonical description of phase 
transitions in small systems will be discussed.

\begin{thebibliography}{99}
\bibitem{SB01}A. Schiller, A. Bjerve, M. Guttormsen, M. Hjorth-Jensen, F.
Ingebretsen, E. Melby, S. Messelt, J. Rekstad, S. Siem, S. W. {\O}deg{\aa}rd,
Phys.\ Rev.\ C \bf 63\rm, 021306(R) (2001).  
\bibitem{SB00}A. Schiller, L. Bergholt, M. Guttormsen, E. Melby, J. Rekstad, S.
Siem, Nucl.\ Instrum.\ Methods Phys.\ Res.\ A \bf 447\rm, 498 (2000).
\bibitem{MB99}E. Melby, L. Bergholt, M. Guttormsen, M. Hjorth-Jensen, F.
Ingebretsen, S. Messelt, J. Rekstad, A. Schiller, S. Siem, S. W. 
{\O}deg{\aa}rd, Phys.\ Rev.\ Lett.\ \bf 83\rm, 3150 (1999).
\bibitem{MG01}E. Melby, M. Guttormsen, J. Rekstad, A. Schiller, S. Siem, A.
Voinov, preprint nucl-ex/0010019 and accepted for publication in Phys.\ Rev.\ 
C. 
\bibitem{LA00}S. Liu, Y. Alhassid, preprint nucl-th/0009006.
\bibitem{GH01}M. Guttormsen, M. Hjorth-Jensen, E. Melby, J. Rekstad, 
A. Schiller, S. Siem, Phys.\ Rev.\ C \bf 63\rm, 044301 (2000).
\bibitem{LA93}B. Lauritzen, A. Anselmino, P. F. Bortignon, R. A. Broglia, Ann.\
Phys.\ \bf 223\rm, 216 (1993).
\end{thebibliography}
\end{document}


\title{Generalized seniority scheme in light Sn isotopes }


\author {N.\ Sandulescu $^{a,b}$, J.\ Blomqvist $^b$,T.\ Engeland $^c$, 
	M.\ Hjorth-Jensen $^d$, A.\ Holt $^c$ , R.\ J.\ Liotta $^b$ and 
        E.\ Osnes $^c$ }


\address{$^a$Institute of Atomic Physics, P.O.Box MG-6, Bucharest, Romania}

\address{$^b$Royal Institute of Technology, Physics Department
Frescati, S-10405, Stockholm,Sweden}

\address{$^c$Department of Physics, 
University of Oslo, N-0316 Oslo, Norway}

\address{$^d$Nordita, Blegdamsvej 17, DK-2100 K\o benhavn \O, Denmark}

\maketitle



\begin{abstract}
In search of a possible truncation scheme for shell model calculations,
the  yrast generalized seniority  states  are compared with  the   
corresponding  shell model  states for the case of the Sn
isotopes $^{104-112}$Sn. For most of the cases the energies 
agree within a few hundred keV. For the $0^+$ ($2^+$) states the 
overlaps decrease from 97\% (93\%) in $^{104}$Sn to 91\% (78\%) 
in $^{112}$Sn when the coefficients of the pairs in the $S$ and $D$
boson operators are allowed to vary with the number of particles. 
For constant
pairing coefficients throughout the entire isotope range, the overlaps
are considerably smaller. It is concluded, with the
realistic effective interaction applied here, that a truncation scheme
based on seniority zero and two states is inadequate when the 
number of valence particles gets large and 
that configurations of higher seniority
should be included.

\end{abstract}


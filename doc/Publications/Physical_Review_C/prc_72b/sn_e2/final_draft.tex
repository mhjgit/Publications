
%\documentstyle[epsfig,aps,twocolumn,amssymb,floats,tighten]{revtex}
\documentclass[aps,prc,twocolumn,amssymb,showpacs]{revtex4}
\usepackage{epsfig,subfigure}

\def\Journal#1#2#3#4{{#1} {\bf #2}, #3 (#4)}
\def\NIMR{Nucl. Instr. and Meth. in Phys. Res. A}
\def\NIMB{Nucl. Instr. and Meth. in Phys. B}
\def\ADNDT{Atomic Data and Nuclear Data Tables}
\def\NPA{Nucl. Phys. A}
\def\CPC{Computer Physics Communications}
\def\PLB{Phys. Lett.  B}
\def\PRL{Phys. Rev. Lett.}
\def\PRep{Phys. Rep.}
\def\PRC{Phys. Rev. C}
\def\PRD{Phys. Rev. D}
\def\PR{Phys. Rep.}
\def\ZPA{Z. Phys. A}
\def\EPJA{Eur. J. Phys. A}
\def\MPL{Mod. Phys. Lett. A}
\def\ARNPS{Ann. Rev. Nucl. Part. Sc.}
\def\RPP{Rep. on Progr. in Phys.}
\def\Prep{Phys. Rep.}
\def\MFMDVS{Mat. Fys. Medd. Dan. Vid. Selsk.}
\def\Acta{Acta Phys.}

\begin{document}

\title{Study of $^{108}$Sn by intermediate-energy Coulomb
excitation}

\author{A.~Banu$^{1,2}$, J.~Gerl$^{1}$, C.~Fahlander$^{3}$,
M.~G\'{o}rska$^{1}$, H.~Grawe$^{1}$, T.R.~Saito$^{1}$,
H.-J.~Wollersheim$^{1}$, A.~Gniady$^{4}$, M.~Hjorth-Jensen$^{5}$,
T.~Beck$^{1}$, F.~Becker$^{1}$, P.~Bednarczyk$^{1,6}$,
M.A.~Bentley$^{7}$, A.~B\"{u}rger$^{8}$, E.~Caurier$^{4}$,
F.~Cristancho$^{3}$, G.~de Angelis$^{9}$, Zs.~Dombradi$^{10}$,
P.~Doornenbal$^{1,13}$, T.~Engeland$^{5}$, H.~Geissel$^{1}$,
J.~Gr\c{e}bosz$^{1,6}$, G.~Hammond$^{7}$,
M.~Hellstr\"{o}m$^{1,\footnote{Present address: Department of
Physics, University of Lund, Sweden}}$, I.~Kojouharov$^{1}$,
N.~Kurz$^{1}$, R.~Lozeva$^{1,\footnote{Pressent address: Faculty
of Physics, University of Sofia, Sofia, Bulgaria}}$,
S.~Mandal$^{1,\footnote{Present address: University of Delhi, New
Delhi, India}}$, N.~M\u{a}rginean$^{9}$, S.~Muralithar$^{11}$,
F.~Nowacki$^{4}$, J.~Nyberg$^{12}$, J.~Pochodzalla$^{2}$,
W.~Prokopowicz$^{1,6}$, P.~Reiter$^{13}$, D.~Rudolph$^{3}$,
C.~Rusu$^{9}$, N.~Saito$^{1}$, H.~Schaffner$^{1}$,
D.~Sohler$^{10}$, H.~Weick$^{1}$,
C.~Wheldon$^{1,\footnote{Pressent address: Hahn-Meitner-Institut
Berlin, Berlin, Germany}}$, M.~Winkler$^{1}$}

\affiliation{
$^{1}$ Gesellschaft f\"ur Schwerionenforschung (GSI), Darmstadt, Germany\\
$^{2}$ Institut f\"ur Kernphysik, Universit\"at Mainz, Mainz, Germany\\
$^{3}$ Department of Physics, University of Lund, Sweden\\
$^{4}$ IReS, Strasbourg, Cedex 2, France\\
$^{5}$ Department of Physics and Center of Mathematics for Applications, University of Oslo, Oslo, Norway\\
$^{6}$ The Henryk Niewodnicza\'{n}ski Institute of Nuclear Physics, Krakow, Poland\\
$^{7}$ Department of Physics, University of York, UK\\
$^{8}$ Helmholtz-Institut f\"{u}r Strahlen- und Kernphysik, Universit\"{a}t Bonn\\
$^{9}$ INFN Laboratori Nazionali di Legnaro, Legnaro, Italy\\
$^{10}$ Institute for Nuclear Research, Debrecen, Hungary\\
$^{11}$ Nuclear Science Center, New Delhi, India\\
$^{12}$ Department of Radiation Sciences, University of Uppsala,
Sweden\\
$^{13}$ Institut f\"{u}r Kernphysik, Universit\"{a}t zu
K\"oln\\
}



\date{\today}

\begin{abstract}
The unstable neutron-deficient $^{108}$Sn isotope has been studied
by intermediate-energy Coulomb excitation in inverse kinematics,
using the newly developed RISING/FRS experimental set-up at
GSI-Darmstadt heavy ion research laboratory. This is the highest-Z
nucleus studied to date with this method and its reduced
transition probability
B(E2;0$_{\text{g.s.}}^{+}$$\rightarrow$2$_{1}^{+})$ has been
measured for the first time. The extracted B(E2) value of 0.230
(50) e$^2$b$^2$ has been determined relative to the known value in
the stable isotope $^{112}$Sn. The result is in good agreement
with recent large scale shell model calculations performed with
realistic effective interactions, and can be understood
phenomenologically within a generalized seniority scheme  model.
\end{abstract}

%\pacs{}
\maketitle

The structure of nuclei far from $\beta$-stability is currently a
key topic of research both experimentally and theoretically with
emphasis on phenomena such as shell evolution, proton-neutron
interaction and change of collective properties. One of the
burning questions in nuclear structure physics is whether the
shell closures known close to stability are preserved towards the
limits of nuclear existence. Due to the soft neutron potential and
decoupling of neutrons from protons, topics like shell quenching,
new shell closures and new collective modes are of main interest
\cite{dob94,naz95} towards the neutron drip line. On the other
hand, towards the proton drip line due to the confinement of
protons by the Coulomb barrier and the vicinity of the N=Z line
changes in shell structure and collectivity are expected to be
driven exclusively by monopole migration \cite{ots01,gra04} and
proton-neutron interaction in identical shell model orbitals
\cite{naz95}. Therefore, at the N=Z and proton drip line, core
polarization as studied in spin (M1, Gamow-Teller) and shape (E2)
response, proton-neutron pairing and isospin symmetry are key
topics of nuclear structure investigations. In this respect, the
heaviest isospin symmetric doubly-magic nucleus,$^{100}$Sn, which
is proton bound, is an important test ground. Information on E2
polarization and shape response of the magic core could be
inferred from the energy of its first excited 2$^+$ state and its
E2 transition rate to the ground state. Experimentally, the
nuclear properties of this nucleus are only indirectly known,
although its existence has already been confirmed
\cite{lew94,sch94}. To gain insight into its structure, the nuclei
in its vicinity are being studied.\\
%In the past, in a series of heavy-ion fusion reaction experiments,
%delayed $\gamma$-rays following the de-excitation of high spin
%isomers were measured. Lifetime measurements of the isomeric 6$^+$
%state in the even tin isotopes \cite{lip98,gor98}, and of the
%isomeric 8$^+$ state in $^{98}$Cd \cite{gor97,bla04} provided
%B(E2) values of the 6$^+$$\rightarrow$4$^+$ and
%8$^+$$\rightarrow$6$^+$ transitions, respectively. Core excited
%states have been identified in $^{98}$Cd and the $^{100}$Sn shell
%gap was inferred \cite{bla04}.
The tin isotopes between the N = 50 and 82 shell closures provide
the longest chain of semi-magic nuclei accessible to nuclear
structure studies both in the neutron valence space of a full
major shell and with emphasis of excitations of the Z = 50 proton core.\\
The B(E2) values yield neutron and proton effective charges, which
are measures of the polarization effects on the core induced by
the valence nucleons. The reduced transition probabilities most
sensitive to the detail of the shell structure and quadrupole
collective effects are the
B(E2;0$^+_{\text{g.s.}}$$\rightarrow$2$^+_1$), since they probe E2
correlations related to core polarization better. Until recently
only the B(E2;0$^+_{\text{g.s.}}$$\rightarrow$2$^+_1$) values of
the stable $^{112-124}$Sn nuclei were measured \cite{ram01}. The
existence of higher lying isomeric states hampers a direct
measurement of the lifetime of these states by standard Doppler
methods (DSAM, RDM), and their very short lifetimes also make it
difficult to apply electronic timing methods. Therefore, a Coulomb
excitation measurement is the only way to obtain this important
piece of nuclear structure information.\\
In the following, we report on the first intermediate-energy
Coulomb excitation experiment performed on the $^{108,112}$Sn
isotopes with the RISING/FRS set-up \cite{wol05}. A primary beam
of $^{124}$Xe$^{47+}$ with an energy of 700 A$\cdot$MeV and an
average intensity of 6 $\times$ 10$^7$ s$^{-1}$ was delivered by
the SIS heavy ion synchrotron at GSI-Darmstadt. The light tin
isotopes of interest were produced via projectile fragmentation
reactions of the primary beam on a 4 g/cm$^2$ $^9$Be primary
target located at the entrance of the fragment separator (FRS).
Secondary beams were separated by FRS and identified on an
event-by-event basis by coincidence measurements of energy loss in
an ionization chamber (MUSIC), magnetic rigidity analysis
($B\rho-\Delta E-B\rho\ technique$), and time-of-flight with the
help of two scintillator detectors (SCI1, SCI2), as illustrated in
Fig.~\ref{fig:RISING}. Projectile fragment trajectories were
tracked event-by-event with two multiwire proportional chambers
(MW1, MW2). An achromatic $^{27}$Al wedge degrader with variable
thicknesses of 4.83 g/cm$^2$ and 4.59 g/cm$^2$ corresponding to
the two fragment settings $^{112}$Sn respectively $^{108}$Sn, was
placed at the middle focal plane of FRS to obtain an optimized
selection of the fragment beams of interest amounting in both
cases to $\simeq 60 \%$ of the incoming fragment beams. A
secondary $^{197}$Au target with a thickness of 386 mg/cm$^2$ was
located at the final focal plane of the fragment separator. The
projectile fragments $^{112,108}$Sn at energies before the
reaction target of 147 A$\cdot$MeV and 142 A$\cdot$MeV,
respectively were excited by means of electromagnetic interaction
with the gold target. Gamma rays of the reaction products were
detected by the RISING Germanium cluster detectors in coincidence
with projectile fragments. The position sensitive detector array
CATE \cite{loz05} consisting of 3 $\times$ 3 Si-CsI(Tl) modular
$\Delta$E-E telescopes, and covering a solid angle of 58 mrad was
used at 1426 mm behind the secondary target for the reaction
channel selection as well as for measuring the scattering angle of
the projectile fragments. A schematic layout of the RISING/FRS
set-up is shown in Fig.~\ref{fig:RISING}.\\
\begin{figure*}[ht]
\centering\mbox{\epsfig{file=NIM_FRS_setup.eps,width=0.6\textwidth}}
\caption{\small Schematic layout of the RISING/FRS set-up. The
beam spectrometer consists of two multiwire proportional chambers
(MW1, MW2), an ionization chamber (MUSIC), and two plastic
scintillation detectors (SCI1, SCI2). The $\gamma$ rays are
measured with an array of Ge-Cluster detectors. The fragments
scattered off the Coulex target are identified by the calorimeter
telescope CATE.} \label{fig:RISING}
\end{figure*}
At intermediate energy Coulomb excitation is an experimental
challenge caused by intense atomic background radiation and
relativistic effects such as Doppler broadening that have to be
accounted for. To date the method has been applied to nuclei with
Z $\leq$ 30 only. Therefore the Z = 50 Sn isotopes, having large
transition energies and short halflives, with respect to these
challenges provide a new methodological bench mark. The RISING 15
Ge-cluster detectors used in the experiment were part of the
former EUROBALL spectrometer \cite{sim97}. Due to relativistic
projectile fragment beams, the Ge detectors were positioned at
forward angles with a small opening angle of 3$^{\circ}$ in order
to maximize the effective solid angle affected by the Lorentz
boost, and to minimize the Doppler broadening. The detector
cluster geometry allows to achieve a good energy resolution of 2
$\%$, and moreover, to apply the add-back procedure for the
Compton scattered gamma events. Each Ge-cluster detector was
surrounded on the side by a lead sheet of 2 mm thickness, and was
shielded in front by a combination of Pb, Sn and Al absorbers of 5
mm thickness, in order to suppress $\gamma$ rays with energies
below 500 keV in the laboratory frame. In the off-line analysis an
event-by-event Doppler-shift correction was performed. The top
panel in Fig.~\ref{fig:Sn_peaks} shows the Doppler corrected
energy spectrum of the Coulomb excitation $\gamma$-ray line in
$^{112}$Sn at 1257 keV. The bottom panel shows the corresponding
spectrum for $^{108}$Sn with the Coulomb excitation
$\gamma$-ray line at 1206 keV.\\
The conditions applied in the data analysis to obtain the spectra
in Fig.~\ref{fig:Sn_peaks} are the following:
\begin{itemize}
\item fragment selection before the target by the FRS; \item
fragment selection after the target by CATE; \item scattering
angle cut between 1$^{\circ}$--2$^{\circ}$; \item prompt gamma
time condition; \item $\text{M}_{\gamma}(\text{E}_{\gamma}\ \ge\
500\ \text{keV}) = 1$.
\end{itemize}
The requested scattering angle range between 1$^{\circ}$ -
2$^{\circ}$ corresponds in the case of $^{112}$Sn to impact
parameters between 10.61 fm (below which nuclear reactions start
to play a role) and 21.22 fm (above which the elastic channel
dominates, and so the atomic background by chance coincidences).
In the case of $^{108}$Sn, the above mentioned scattering angle
range translates to an impact parameter range between 11.37 fm and
22.74 fm.\\
\begin{figure}[!h]%
\centering
\subfigure{%
\resizebox{0.42\textwidth}{!}{%
%\includegraphics{dummy.eps}}
\includegraphics{112Sn_coulex.eps}}}\\
\vspace{-5pt}
\subfigure{%
\resizebox{0.42\textwidth}{!}{%
%\includegraphics{dummy.eps}}
\includegraphics{108Sn_coulex.eps}}}%
\vspace{5pt} \caption[Coulomb excitations $\gamma$-ray lines
observed in $^{112,108}$Sn]{\small Coulomb excitation $\gamma$-ray
lines corresponding to the $0^+ \rightarrow 2^+_1$ transition
clearly observed in $^{112,108}$Sn isotopes, respectively. The
insets illustrate the error-wise distribution of the two Coulomb excitation peaks.}%
\label{fig:Sn_peaks}%
\end{figure}The intermediate-energy Coulomb excitation reaction is a
\textit{one-step process} which implies a gamma hit multiplicity
equal to one. However, there is a large probability that the
Coulomb de-excitation event is by chance accompanied by a gamma
event originating from atomic processes. Therefore, for an
appropriate Coulomb excitation event selection, it was required in
the data analysis that the condition of single gamma hit
multiplicity is satisfied only for those prompt $\gamma$ rays at
energies larger than 500 keV (in the laboratory frame), which was
the threshold for the atomic background radiation fulfilled by the
RISING experimental set-up.\\
The observation of the Doppler corrected $\gamma$ line
corresponding to the 0$^+_{\text{g.s.}}$$\rightarrow$2$^+_1$
transition in the $^{108}$Sn isotope allows for a measurement of
the Coulomb excitation cross section, which is directly
proportional to the reduced transition probability B(E2)
connecting the two levels. However, in order to avoid possible
systematic errors, such as excitations of higher-lying 2$^+$
states (feeding contributions) or unknown gamma angular
distribution effects, the
B(E2;0$^+_{\text{g.s.}}$$\rightarrow$2$^+_1$) value in $^{108}$Sn
was measured relative to the B(E2) value in $^{112}$Sn of 0.240
(14) e$^2$b$^2$ \cite{ram01} known from previous work measured in
Coulomb excitation below the barrier. The Coulomb excitation
measurement performed on the stable $^{112}$Sn isotope was used as
calibration point. Experimentally, the Coulomb excitation cross
section $\sigma_{E2}^{exp}$ is determined as follows:
\begin{displaymath}
\sigma_{E2}^{exp} = \frac{N_{E2}}{N_p \times N_t} =
\frac{N_{\gamma}/\epsilon_{\gamma}^{abs}}{N_p \times N_t}
\end{displaymath}
where N$_p$, N$_t$, and N$_{\gamma}$ represent, respectively the
number of incoming beam particles interacting with target, the
number of target atoms per unit area, and  the total number of
photons detected with an absolute efficiency,
$\epsilon_{\gamma}^{abs}$. Taking into account the direct
proportionality relationship between the Coulomb excitation cross
section and the reduced transition probability (see ref.
\cite{win79}), the following expression is deduced for the
relative determination of the B(E2) value in $^{108}$Sn:
\begin{displaymath}
\scriptsize [B(E2;0^+_{\text{g.s.}} \rightarrow
2^+_1)]_{\textbf{108}} = [B(E2;0^+_{\text{g.s.}} \rightarrow
2^+_1)]_{\textbf{112}}\times\frac{\sigma^{\textbf{108}}_{E2}}{\sigma^{\textbf{112}}_{E2}}
\times\frac{f(b^{\textbf{112}}_{\text{min}})}{f(b^{\textbf{108}}_{\text{min}})},
\end{displaymath}
where the proportionality factor
f(b$^{\text{projectile}}_{\text{min}})$ depends on the projectile
minimum impact parameter \cite{win79}. The ratio of the
proportionality factors has been determined with the code DWEIKO
\cite{ber03}, a computer program applicable to nuclear scattering
at intermediate and high energies (E$_{\text{lab}}$ $\geqslant$ 50
MeV/nucleon), amounting to a value of 0.88. In Table~\ref{tab-1}
the experimental values of the variables needed to calculate the
B(E2) value in $^{108}$Sn are listed. \vspace{-10pt}
\begin{table}[!h]
\caption{The photon yield N$_{\gamma}$ and incoming particle flux
N$_{p}$ for $^{112,108}$Sn isotopes.}\label{tab-1}
\begin{tabular}[t]{cccc}
\hline \hline
           &               &              &  ~~~ \\
Isotope ~~~& $^{112}$Sn ~~~& $^{108}$Sn   &  ~~~ \\
           &               &              &  ~~~ \\
\hline \hline
           &               &              &  ~~~ \\
N$_{\gamma}$\footnote{\vspace{5pt}The photon yield was determined with the particle$\otimes$$\gamma$ trigger.} ~~~& 106 (20) ~~~ & 174 (26) &  ~~~ \\
           &               &              &  ~~~ \\
N$_{p}$\footnote{The incoming particle flux was determined with a down-scaled particle-singles trigger.}~~~& 101217280  ~~~ & 153377312  &  ~~~ \\
           &               &              &  ~~~ \\
\hline \hline \vspace{1pt}
\end{tabular}
\end{table}\\
Since the 2$^+_1$ excited states in the $^{112,108}$Sn isotopes
lie close in energy, the gamma detection efficiencies are similar.
Moreover, the number of atoms per unit area of the target can be
considered to be equal in the two cases. Hence, the reduced
transition probability
B(E2;0$^+_{\text{g.s.}}$$\rightarrow$2$^+_1$) is finally
determined as:
\begin{displaymath}
\scriptsize
[B(E2;0^+_{\text{g.s.}}\rightarrow2^+_1)]_{\textbf{108}} =
[B(E2;0^+_{\text{g.s.}}\rightarrow2^+_1)]_{\textbf{112}}\times\frac{N^{\textbf{108}}_{\gamma}}{N^{\textbf{112}}_{\gamma}}
\times\frac{N^{\textbf{112}}_p}{N^{\textbf{108}}_p}\times0.88,
\end{displaymath}
yielding
\begin{displaymath}
[B(E2;0^+_{\text{g.s.}}\rightarrow2^+_1)]_{\textbf{108}} = 0.230\
(50)\ e^2b^2.
\end{displaymath}The measured B(E2)$\uparrow$ value in
$^{108}$Sn isotope is compared to the results of two independent
large scale shell model calculations. The systematics for the full
isotopic chain are discussed within a generalized seniority scheme.\\

The first set of large-scale shell model calculations was performed by the
Oslo group for all tin isotopes $^{102-130}$Sn, following the prescription outlined 
in Ref.~\cite{hjo95} and using the CD-Bonn potential for 
the bare nucleon-nucleon interaction
\cite{mac96}. Three sets of closed shell core were chosen for these calculations, 
$^{88}$Sr,
$^{100}$Sn and $^{132}$Sn. The model space for neutrons comprises in all cases
of the 1d$_{5/2}$, 0g$_{7/2}$, 1d$_{3/2}$,
2s$_{1/2}$, and 0h$_{11/2}$ neutron orbitals, and for $^{88}$Sr it includes also
protons in the $1p_{1/2}$ and $0g_{9/2}$ orbits. 
In the discussion here we focus on the results obtained with $^{100}$Sn as closed shell
core. 
A harmonic-oscillator basis was chosen for the single-particle wave
functions, with an oscillator energy $\hbar\Omega$ given by
$\hbar\Omega = 45A^{-1/3} -25A^{-2/3}$ = 8.5 MeV, A = 100 being
the mass number. The single-particle energies of the chosen model
space orbits are set, relative to the 1d$_{5/2}$ orbital
($\epsilon_{1d_{5/2}}$ = 0.0 MeV), as follows:
$\epsilon_{0g_{7/2}}$ = 0.08 MeV, $\epsilon_{1d_{3/2}}$ = 1.66
MeV, $\epsilon_{2s_{1/2}}$ = 1.55 MeV, $\epsilon_{0h_{11/2}}$ =
3.55 MeV. The results of the calculations for the energies of the
2$^+_1$ excited states and
B(E2;0$^+_{\text{g.s.}}$$\rightarrow$2$^+_1$) values for light Sn isotopes are presented
in Table~\ref{tab-2}. The neutron effective charge used in the
calculations was set to  1.0 e.\\
In the following, a different approach in the large scale shell
model calculations is described, which enables to include proton
core excitations. The starting point is the realistic interaction
(CD-Bonn potential) of the Oslo group, phenomenologically adjusted
to the spectroscopy of Sn isotopes and N = 82 isotones
\cite{gni05}. The differences consist in choosing $^{80}$Zr as a
closed shell core, and consequently a different model space for
protons, namely 0g$_{9/2}$, 0g$_{7/2}$, 1d$_{5/2}$, 1d$_{3/2}$ and
2s$_{1/2}$, and for the neutrons the same gds-shell plus the
0h$_{11/2}$ orbital. The calculations allow only up to 3p-3h core
excitations, and the effective charges are set to 1.55 $e$ and
0.72 $e$ for protons and neutrons, respectively. Since the chosen
model space is rather large, the coupled scheme code NATHAN is
used \cite{cau02}. In addition, a seniority truncation is applied.
The results with a better convergence are obtained for seniority
equal to 8, for which a B(E2)$\uparrow$ value of 0.176 e$^2$b$^2$
is deduced for $^{108}$Sn, which is in excellent agreement with
the results of the Oslo group (see Table~\ref{tab-2}). These kind
of calculations stress the importance of the core excitations for
E2 transitions as origin of the increased effective charge in pure
neutron valence
space calculations.  These conclusions are strengthened by the two additional
sets of shell-model calculations, where we use 
$^{88}$Sr and $^{132}$Sn as closed shell cores. In both cases we use the same 
nucleon-nucleon interaction as in the two previous calculations.
For the chain of Sn isotopes with $^{132}$Sn and neutron holes, the experimental 
 B(E2;0$^+_{\text{g.s.}}$$\rightarrow$2$^+_1$) transitions 
were reproduced with effective charges between $0.7-0.8 e$ for all 
isotopes from  $^{120}$Sn to $^{130}$Sn,
see Ref.~\cite{hol98} for further details.
For $^{108,112, 114}$Sn one needs however effective charge of the order
of $1.0$ e in order to reproduce the experimental values starting with 
$^{132}$Sn as closed shell core. For  $^{88}$Sr as closed shell core 
we obtain similar results as for $^{100}$Sn as closed shell core, with an 
effective charge of $1.0$ e
This indicates clearly that the effective charges  for the lighter tin isotopes show
a stronger renormalization effects, unless one allows
for core excitations in the model space.  It should also be noted that the spectra
for $^{102-114}$Sn are in better agreement with experiment with $^{100}$Sn
as closed shell core than if we use $^{132}$Sn as closed shell core.
Similarly, the excitation spectra for the heavier Sn isotopes 
$^{116-130}$Sn are better described with a $^{132}$Sn core.
This may eventually imply the need of different effective charges below and
above $^{116}$Sn.

The almost constant energy spacing between the 0$^+_{\text{g.s.}}$
and the 2$^{+}_{1}$ excited state, which is so characteristic for
the Sn isotopes, is well reproduced by the calculations. Also the
B(E2)$\uparrow$ value calculated for the $^{108}$Sn case is in
good agreement with the measured value. However, for a conclusive
interpretation of the results more information is needed. In
Fig.~\ref{fig:systematics} the systematics of the B(E2)$\uparrow$
values in the Sn isotopes ranging from mass number A = 100 to A =
132 is shown. Here the neutrons are filling the subshells between
the magic numbers 50 and 82 offering thus a unique opportunity for
examining how well the Z = 50 proton-shell closure is holding up
as valence neutrons are being added. The data used in the
systematics in Fig.~\ref{fig:systematics} are from the calculations using
$^{100}$Sn as closed shell core and effective charges of $1.0$ e. 
The theoretical results yield a parabola-like trend.
For comparison the experimental data measured recently for the
unstable heavy $^{126,128,130}$Sn isotopes \cite{rad04} and the
adopted values for the stable Sn isotopes $^{112, 114, 116, 118,
120, 124}$Sn \cite{ram01} are shown. The parabola-like trend of
the B(E2) systematics in Fig.~\ref{fig:systematics} resembles the
typical behavior of a one-body even tensor operator across a shell
in the seniority scheme, which for a seniority changing transition
($\Delta v = 2$) at first increases, then flattens out, peaking at
the midshell, and falling off thereafter. In a non-unique j shell
with many interacting orbitals this can be generalized to
B(E2,Sn$_N$;0$^{+}_{\text{g.s.}}$$\rightarrow$2$^{+}_{1}$)
$\approx$ f(1-f) with f=(N-50)/32 being the filling factor of the
shell.

If we use  $^{132}$Sn as closed the core, see for example Ref.~\cite{hol98},
two sets of effective charges may be needed, $e_{\mathrm{eff}}\approx 0.7-0.8$ e
for $^{116}$Sn to $^{130}$Sn and $e_{\mathrm{eff}}\approx 1.0$ e
for $^{108}$Sn and $^{112}$Sn, indicating in this case that
a two-parabola fit assuming a
subshell closure around N=64 might be more appropriate
A closer inspection of the experimental data
seems to suggest this as well.
This can be described
microscopically in a pairing approach \cite{bes60}. 
The overall
agreement between the experimental and theoretical B(E2) values
across the N = 50 - 82 neutron shell underlines the success of a
generalized seniority scheme in describing the properties of the
even Sn nuclei. The tendency of larger B(E2) values in the lighter
isotopes can be traced back to two effects, \textbf{(i)} enhanced
core excitation towards N=Z as supported by the shell model
calculation of ref. \cite{gni05} and/or \textbf{(ii)} the blocking
of 1p1h neutron excitations from the g, d, s, orbits to the
odd-parity h$_{11/2}$ orbit for N $\gtrsim$ 64, which can not
contribute to the E2 transition. The trend is exhibited more
clearly in the Ni isotopes one major shell lower \cite{sor02},
which show a distinct minimum in the B(E2) at N = 40. %This is
%further evidence for the quenched N = 70 (as persisting in the
%proton shell of the N = 82 isotones) respective N = 64 harmonic
%oscillator shell in the Sn isotopes.

\begin{table}[!h]
\caption{Lowest-lying excited states and E2 transitions for
$^{104-112}$Sn}\label{tab-2}
\begin{tabular}[t]{lcccc}
\hline \hline
\vspace{5pt}
Isotope~~~~~~~~~& & E$_{2^{+}_{1}}$ [keV] & ~~~~~~~& B(E2)$\uparrow$ [e$^2$b$^2$]   \\
%~~~~~~~~          &                       &\\
%\vspace{5pt}
\end{tabular}
\begin{tabular}{ccccc}
           & Exp.$^a$ & SM$^b$ & Exp. & SM$^b$   \\
\hline \hline
%           &                       &   \\
$^{104}$Sn & 1260.1~ (3) & 1343.4 & & 0.093   \\
%           &                       &   \\
$^{106}$Sn & 1207.7~ (5) & 1230.9 & & 0.137   \\
%           &                       &   \\
$^{108}$Sn & 1206.07 (10) & 1243.4 & 0.230 (50)$^c$ & 0.171   \\
%           &                       &   \\
$^{110}$Sn & 1211.89 (15) & 1252.9 & & 0.192  \\
%           &                       &   \\
$^{112}$Sn & 1256.85 (97) & 1236.1 & 0.240 (14)$^a$ & 0.203   \\
%           &                       &   \\
\hline \hline
\end{tabular}
\end{table} $^a$ from ref.\cite{ram01},$\ $
 $^b$ ref.\cite{hjo05},$\ $
 $^c$ this work
\vspace{-5pt}
\begin{figure}[hhh]\hspace{-0.5cm}
\centering\mbox{\epsfig{file=sys.eps,width=.5\textwidth}}
\caption{\small Systematics of the
B(E2;0$^{+}_{\text{g.s.}}$$\rightarrow$2$^{+}_{1}$) in the Sn
isotopes, ranging from mass number A = 100 to A = 132. The lines
through the theoretical points are only intended to guide the
eye.}\label{fig:systematics}
\end{figure}

In conclusion we have measured for the first time the
B(E2;0$^+_{\text{g.s.}}$$\rightarrow$2$^{+}_1$) value in the
unstable $^{108}$Sn isotope by intermediate-energy Coulomb
excitation, see \cite{ban05} for details. This is the highest-Z
nucleus studied by this method to date. The comparison with two
different but complementary large scale shell model calculations
shows a good agreement and proves that a combination of core
excitation and valence particles is substantial for a correct
description of the lowest 2$^+$ collective state. The successful
B(E2) measurement in $^{108}$Sn at GSI-Darmstadt opens the
research line via the experimental technique of
intermediate-energy Coulomb excitation towards perhaps the most
interesting nucleus ever experimentally synthesized, the heaviest
N=Z doubly-magic proton drip line nucleus $^{100}$Sn.

\begin{acknowledgments}
We thank the GSI accelerator staff for their efforts to produce
good intensity primary $^{124}$Xe beams, the DVEE division at GSI
for their support in data acquisition and Go4 software, and the
technical staff for experiment preparation. A.B. is greatly 
indebted to A.~Gniady for communicating the
results of their shell-model calculations prior to publication.
Discussions with H.~Grawe are gratefully acknowledged.
\end{acknowledgments}

\begin{thebibliography}{100}
\bibitem{dob94} J.~Dobaczewski et al., \Journal{\PRL}{72}{981}{1994}
\bibitem{naz95} W.~Nazarewicz et al.,\Journal{Physica Scripta T}{56}(1995)
\bibitem{ots01} T.~Otsuka et al., \Journal{\PRL}{87}{082502}{2001}
\bibitem{gra04} H.~Grawe, \Journal{Springer Lect. Notes Phys.}{LNP 651}{33}(2004)
\bibitem{lew94} M. Lewitowicz et al., \Journal{\PLB}{332}{20}{1994}
\bibitem{sch94} R Schneider et al., \Journal{\ZPA}{348}{241}{1994}
%\bibitem{lip98} M. Lipoglav\v{s}ek et al., \Journal{\PLB}{440}{246}{1998}
%\bibitem{gor98} M. G\'orska et al., \Journal{\PRC}{58}{108}{1998}
%\bibitem{gor97} M. G\'orska et al., \Journal{\PRL}{79}{2415}{1997}
%\bibitem{bla04} A.~Blazhev et al., \Journal{\PRC}{69}{64304}{2004}
\bibitem{ram01} S. Raman et al., \Journal{\ADNDT}{78}{42}{2001}
\bibitem{wol05} H.J. Wollersheim et al., \Journal{\NIMR}{537}{637}{2005}
\bibitem{loz05} R. Lozeva et al., \Journal{\Acta} {B36}{1245}{2005}
\bibitem{sim97} J. Simpson et al., \Journal{\ZPA}{358}{139}{1997}
\bibitem{win79} A. Winther and K. Alder, \Journal{\NPA}{319}{518}{1979}
\bibitem{ber03} C.A. Bertulani et al., \Journal{\CPC}{152}{317}{2003}
\bibitem{hjo95} M. Hjorth-Jensen et al., \Journal{\Prep}{261}{125}{1995}
\bibitem{mac96} R. Machleidt et al., \Journal{\PRC}{53}{1483}{1996}
\bibitem{gni05} A. Gniady, private communication
\bibitem{cau02} E. Caurier et al., \Journal{\NPA}{704}{60c}{2002}
\bibitem{hol98} A. Holt et al., \Journal{\NPA}{634}{41}{1998}
\bibitem{rad04} D.C. Radford et al., \Journal{\NPA}{746}{83c}{2004}
\bibitem{bes60} R.A. Sorensen and D.R. Bes, \Journal{\MFMDVS}{9}{32}{1960}
\bibitem{sor02} O. Sorlin et al., \Journal{\PRL}{88}{092501}{2002}
\bibitem{ban05} A. Banu, thesis, University of Mainz
\end{thebibliography}
\end{document}

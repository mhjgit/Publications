
\documentstyle[aps,preprint,psfig,epsf]{revtex}

%\documentstyle[aps,multicol,psfig,epsf]{revtex}


\draft

\begin{document}

\title{Structure of $\beta$-stable neutron star 
       matter with hyperons}

\author{I.\ Vida\~na, A.\ Polls, and A.\ Ramos}

\address{Departament d'Estructura i Constituents de la Mat\`eria,
Universitat de Barcelona, E-08028 Barcelona, Spain}

\author{L.\ Engvik, and M.\  Hjorth-Jensen}

\address{Department of Physics, University of Oslo, N-0316 Oslo, Norway}

\maketitle

\begin{abstract}

We present results from many-body calculations 
for $\beta$-stable neutron star 
matter with nucleonic and
hyperonic degrees of freedom, employing the most recent parametrizations
of the baryon-baryon interaction of the Nijmegen group. 
It is found that the only strange baryons emerging in $\beta$-stable matter 
up to total baryonic densities of 1.2 fm$^{-3}$ are $\Sigma^-$ and 
$\Lambda$.
Implications for neutron stars are discussed.    


\end{abstract}


\pacs{PACS numbers: 13.75.Ev, 21.30.-x, 21.65.+f, 26.60.+c, 97.60.Gb, 97.60.Jd}

%\begin{multicols}{2}

The physics of compact objects like neutron stars offers
an intriguing interplay between nuclear processes  and
astrophysical observables.
Neutron stars exhibit conditions far from those 
encountered on earth; typically, expected densities $\rho$ 
of a neutron star interior are of the
order of $10^3$ or more times the density  
$\rho_d\approx 4\cdot 10^{11}$ g/cm$^{3}$ at 'neutron drip',
the density at which nuclei begin to 
dissolve and merge together.
Thus, the determination of an equation of state (EoS) 
for dense matter is essential to calculations of neutron 
star properties. The EoS determines properties  such as 
the mass range, the mass-radius relationship, the crust 
thickness and the cooling rate.
The same EoS is also crucial
in calculating the energy released in a supernova explosion.

At densities near to the saturation density of nuclear 
matter, ( with number density $n_0=0.16$ fm$^{-3}$),  
we expect the matter to be composed of mainly neutrons, protons  
and electrons in $\beta$-equilibrium, since neutrinos have on average a
mean free path larger than the radius of the neutron star. The 
equilibrium conditions can then be summarized as
\begin{equation}
    \mu_n=\mu_p+\mu_e,  \hspace{1cm} n_p = n_e,
     \label{eq:npebetaequilibrium}
\end{equation}
where $\mu_i$ and $n_i$ refer to the chemical potential and number density
in fm$^{-3}$ of particle species $i$, respectively. 
At the saturation density of nuclear matter, $n_0$, 
the electron chemical potential is
of the order $\sim 100$ MeV.
Once the rest mass of the muon is exceeded, it becomes
energetically favorable for an electron at the top
of the $e^-$ Fermi surface to decay into a
$\mu^-$. We then develope a Fermi sea of degenerate negative muons,
and we have to modify the charge balance according to $n_p = n_e+n_{\mu}$,
and require that $\mu_e = \mu_{\mu}$.

As density increases, new hadronic degrees of freedom may appear in addition
to neutrons and protons. 
One such degree of freedom is the formation of hyperons, baryons with a
strangeness content. 
Contrary to terrestrial conditions where hyperons are unstable and decay 
into nucleons through the weak interaction, the equilibrium conditions
in neutron stars can make the inverse process happen, so that the 
formation of hyperons becomes energetically favorable.      
As soon as the chemical potential
of the neutron becomes sufficiently large, energetic neutrons
can decay via weak strangeness non-conserving interactions
into $\Lambda$ hyperons leading to a $\Lambda$ Fermi sea
with $\mu_{\Lambda}=\mu_n$. 
However, one expects $\Sigma^-$ to appear via
\begin{equation}
    e^-+n \rightarrow \Sigma^- +\nu_e,
\end{equation}
at lower densities than the $\Lambda$, even though $\Sigma^-$ is more
massive. The negatively charged hyperons
appear in the ground state of matter when their masses
equal $\mu_e+\mu_n$, while the neutral hyperon $\Lambda$ 
appears when $\mu_n$ equals its mass. Since the 
electron chemical potential in matter is larger than 
the mass difference $m_{\Sigma^-}-m_{\Lambda}= 81.76$ MeV,
$\Sigma^-$ will appear at lower densities than $\Lambda$.
For matter with hyperons as well 
the chemical equilibrium condition becomes,
\begin{eqnarray}
    \mu_{\Xi^-}=\mu_{\Sigma^-} = \mu_n + \mu_e, \nonumber \\ 
    \mu_{\Lambda} = \mu_{\Xi^0}=\mu_{\Sigma^0} = \mu_n , \nonumber \\
    \mu_{\Sigma^+} = \mu_p = \mu_n - \mu_e .
    \label{eq:beta_baryonicmatter}
\end{eqnarray}
We have omitted isobars $\Delta$ 
since these are not expected to appear at the densities considered 
in this work, see e.g., Refs.\ \cite{glen97,prakash97}.

Hyperonic degrees of freedom have been considered by several authors,
but mainly within the framework of relativistic 
mean field models \cite{glen97,prakash97,pke95,ms96} or parametrized
effective interactions \cite{bg97}, see also Balberg {\em et al.} \cite{blc99}
for a recent update. Realistic hyperon-nucleon interactions
were employed by Schulze {\em et al.}  recently, see Ref.\ \cite{bbs98},
in a many-body calculation in order to study 
where hyperons appear in neutron star matter.  
All these works show that hyperons appear at densities of the order of
$\sim 2n_0$.

The aim of this letter is 
however to present results from, to our knowledge,
 the first  {\em ab initio}
many-body calculation of hyperonic degrees of freedom 
for $\beta$-stable neutron star matter. 
This entails a microscopic
description of matter starting from 
realistic nucleon-nucleon, hyperon-nucleon
and hyperon-hyperon interactions. 

To achieve this goal, 
our many-body scheme starts with the most recent 
parametrization
of the free baryon-baryon potentials 
for the complete  baryon octet
as defined by Stoks and Rijken in Ref.\ 
\cite{sr99}. 
This potential model, which aims at describing all 
interaction channels
with strangeness from $S=0$ to $S=-4$, 
is based on SU(3) extensions
of the Nijmegen potential models \cite{rsy98} 
for the $S=0$ and $S=-1$ channels, which
are fitted to the available body of experimental 
data and constrain all free parameters in the model. 
In our discussion we employ 
the interaction version NSC97e of Ref.\ \cite{sr99}, although
we have also carried out calculations with version NSC97a. 
Since the results for $\beta$-stable matter are not significantly altered,
we will present results with version NSC97e only. 

The next step is to introduce effects from the nuclear medium. 
Here we will construct the so-called $G$-matrix, which
takes into account short-range correlations for all strangeness
sectors, and solve the equations for the single-particle energies 
of the various baryons self-consistently.
The $G$-matrix is formally given by 
\begin{eqnarray}
   \left\langle B_1B_2\right |G(\omega)\left | B_3B_4 \right\rangle=
   \left\langle B_1B_2\right |V\left | B_3B_4 \right\rangle+&\nonumber\\
   \sum_{B_5B_6}\left\langle B_1B_2\right |V\left | B_5B_6 \right\rangle
   \frac{1}{\omega-\varepsilon_{B_5}-\varepsilon_{B_6}+ \imath\eta}&\nonumber\\
   \times\left\langle B_5B_6\right |G(\omega)\left | B_3B_4 \right\rangle&.
   \label{eq:gmatrix}
\end{eqnarray}
Here $B_i$ represents all possible baryons $n$, $p$, $\Lambda$, $\Sigma^{-}$,
$\Sigma^0$, $\Sigma^+$, $\Xi^-$ and $\Xi^0$ and their quantum numbers
such as spin, isospin, strangeness, linear momenta and orbital momenta.
The intermediate states $B_5B_6$ are those which are allowed by
the Pauli principle and the energy variable $\omega$ is the starting energy
defined by the single-particle energies 
of the incoming external particles $B_3B_4$.
The $G$-matrix is solved using relative and centre-of-mass coordinates,
see e.g., Refs.~\cite{sl99,isaac99} for computational details.
The single-particle energies are given by
\begin{equation}
      \varepsilon_{B_i}=t_{B_i} + u_{B_i} +m_{B_i}
       \label{eq:spenergy}
\end{equation} 
where $t_{B_i}$ is the kinetic energy and $m_{B_i}$ 
the mass of baryon ${B_i}$. The
single-particle potential $u_{B_i}$ is defined by  
\begin{equation} 
       u_{B_i}=\mathrm{Re} \sum_{B_j\leq F_j}
       \left\langle B_iB_j\right |
       G(\omega=\varepsilon_{B_j}+\varepsilon_{B_i})
       \left | B_iB_j \right\rangle.
\end{equation}
The linear momentum of the intermediate 
single-particle state $B_j$ is limited by the size of the Fermi surface 
$F_j$ for particle species $B_j$. 
The last equation is displayed in terms of Goldstone diagrams 
in Fig.\ \ref{fig:upot}. Diagram (a) represents contributions
from nucleons only as hole states, while diagram (b) 
has only hyperons as holes states in case we have a finite hyperon
fraction in $\beta$-stable neutron star matter. The external legs
represent nucleons and hyperons.  

In order to satisfy the equations for $\beta$-stable matter summarized
in Eq.\ (\ref{eq:beta_baryonicmatter}), we need to solve 
Eqs.\ (\ref{eq:gmatrix}) and (\ref{eq:spenergy}) to obtain 
the single-particle energies of the
particles involved at the corresponding Fermi momenta.
Typically, for every total 
baryonic density $n=n_N+n_Y$, the density of nucleons plus hyperons,
Eqs.\ (\ref{eq:gmatrix}) and (\ref{eq:spenergy}) were solved for five nucleon
fractions and five hyperons fractions and,  for every nucleon and hyperon
fraction, we computed
three proton fractions and three fractions for the relevant hyperons.
The set of equations in Eq.\  (\ref{eq:beta_baryonicmatter}) were
then solved by interpolating between different nucleon and hyperon 
fractions.

The many-body approach outlined above is the lowest-order
Brueckner-Hartree-Fock (BHF) method extended to the hyperon sector. 
This means also that we consider only
two-body interactions. However, it is well-known from studies of nuclear
matter and neutron star matter with nucleonic degrees of freedom only
that three-body forces are important in order to reproduce the saturation 
properties of nuclear matter, see e.g., Ref.\ \cite{apr98} for the most recent
approach.  In order to include such effects, we replace the contributions
to the proton and neutron self-energies arising from intermediate 
nucleonic states only, see diagram (a) of Fig.\ \ref{fig:upot},
with those derived from Ref.\ \cite{apr98} (hereafter APR98) 
where the Argonne $V_{18}$ nucleon-nucleon interaction \cite{v18} is used with
relativistic boost corrections and a fitted three-body interaction,
model. 
The calculations of Ref.\ \cite{apr98} represent at present perhaps
the most sophisticated many-body approach to dense matter. 
In the discussions below we will thus present two sets of results for 
$\beta$-stable matter, one where the nucleonic contributions
to the self-energy of nucleons is derived from the baryon-baryon potential
model of Stoks  and Rijken \cite{sr99} and one where the nucleonic contributions
are replaced with the results from Ref.\ \cite{apr98} following the parametrization
discussed in Eq. (49) of  Ref.\ \cite{hh99}. Hyperonic contributions
will however all be calculated with the baryon-baryon interaction of 
Stoks  and Rijken \cite{sr99}.

These models for the pure nucleonic part combined with the hyperon 
contribution yield
the composition of $\beta$-stable matter, up to total baryonic number density  
$n=1.2$ fm$^{-3}$, shown in Fig.\ 
\ref{fig:fraction}. 
The corresponding energies per baryon  
are shown in Fig.\ \ref{fig:eosfig} for both pure nucleonic
(BHF and APR98 pn-matter) 
and hyperonic matter (BHF and APR98 with hyperons) in $\beta$-equilibrium
for the same baryonic densities as in Fig.\ \ref{fig:fraction}.

For both types of calculations $\Sigma^-$ appears at densities $\sim 2-3 n_0$.
Since the  EoS of APR98 for nucleonic matter yields a stiffer EoS than the
corresponding BHF calculation, $\Sigma^-$ appears at 
$n=0.27$ fm$^{-3}$ for the APR98 EoS and $n=0.35$ fm$^{-3}$ for the BHF EoS.
These results are in fair agreement with results obtained from
mean field calculations, see e.g., Refs.\ 
\cite{glen97,prakash97,pke95,ms96}. The introduction
of hyperons leads to a considerable softening of the EoS. 
Moreover, as soon as hyperons appear, the leptons tend to disappear,
totally in the APR98 case whereas in the BHF calculation only muons disappear.
This result is related to the fact that $\Lambda$ does not appear
at the densities considered here for the BHF EoS.
For the APR98 EoS, $\Lambda$ appears at a density $n=0.67$ fm$^{-3}$.
Recalling $\mu_{\Lambda} = \mu_n = \mu_p + \mu_e$ and that the APR98 EoS is stiffer
due to the inclusion of three-body forces, this 
clearly enhances the possibility of creating a $\Lambda$ with the APR98 EoS. 
However, the fact that
$\Lambda$ does not appear in the BHF calculation can also, in addition to the softer
EoS, 
be retraced to a delicate balance between
the nucleonic and hyperonic hole state contributions
(and thereby to features of the baryon-baryon interaction) 
to the self-energy of the baryons considered here, see diagrams (a) and (b) 
in Fig.\ \ref{fig:upot}. Stated differently, the contributions from $\Sigma^-$,
proton and neutron 
hole states to the $\Lambda$ chemical potential are not attractive enough 
to lower the chemical potential of the $\Lambda$ so that it equals
that of the neutron. Furthermore, the chemical potential of the neutron
does not increase enough since contributions
from $\Sigma^-$ hole states to the neutron self-energy are attractive.  

We illustrate the role played by the two different 
choices for nucleonic EoS in 
Fig.\ \ref{fig:chempots} in terms of the chemical potentials for various baryons 
for matter in $\beta$-equilibrium. 
We also note that, using the criteria in Eq.\ (\ref{eq:beta_baryonicmatter}), 
neither the $\Sigma^0$ nor $\Sigma^+$
do appear for both the BHF and the APR98 equations of state. This is due to  the 
fact that none of the $\Sigma^0$-baryon and  $\Sigma^+$-baryon interactions
are attractive enough. A similar argument  applies to $\Xi^0$ and $\Xi^-$. In the latter
case the mass of the particle is $\sim 1315$ MeV and almost $200$ MeV in attraction
is needed in order to fullfil e.g., the condition 
$\mu_{\Lambda} = \mu_{\Xi^0}= \mu_n$. 
 From the bottom panel of Fig.\ \ref{fig:chempots} we see however that $\Sigma^0$
could appear at densities close to $1.2$ fm$^{-3}$.


In summary, using the realistic EoS of Akmal {\em et al.} \cite{apr98} for the 
nucleonic sector and including hyperons through the most recent model for the 
baryon-baryon interaction of the Nijmegen group \cite{sr99}, we find 
through a many-body calculation for matter in $\beta$-equilibrium that
$\Sigma^-$ appears at a density of $n=0.27$ fm$^{-3}$ while 
$\Lambda$ appears at $n=0.67$ fm$^{-3}$. 
Due to the formation
of hyperons, the matter is deleptonized 
at a density of $n=0.85$ fm$^{-3}$. Within our many-body approach,
no other hyperons appear at densities below $n=1.2$ fm$^{-3}$.  
Although the EoS of Akmal {\em et al.} \cite{apr98} may be viewed as the currently 
most realistic approach to the nucleonic EoS, our
results have to be gauged with the uncertainty in the hyperon-hyperon and nucleon-hyperon
interactions. Especially, if the hyperon-hyperon interactions tend to be 
more attractive, this may lead to the formation of hyperons such as the $\Lambda$, 
$\Sigma^0$, $\Sigma^+$, $\Xi^-$ and $\Xi^0$ at lower densities.
The hyperon-hyperon interaction and the stiffness of the nucleonic contribution
play crucial roles in the formation of various hyperons. These results differ 
from present mean field calculations \cite{glen97,prakash97,pke95,ms96},
where all kinds of hyperons can appear at the densities considered here.
Several interesting consequences for neutron stars arise from these results. 
Compared with the EoS of Akmal {\em et al.} \cite{apr98}, the softening
of the EoS due to the presence of hyperons lowers the maximum mass of a neutron 
star from $2.1M_{\odot}$ to $1.35M_{\odot}$. Including rotational corrections to the 
total mass, see e.g., Refs.\ 
\cite{glen97,hh99,hartle67} for calculational details, 
leads to a maximum mass of  $1.48M_{\odot}$. If the large masses $M\sim 2-2.3M_{\odot}$
of neutron stars as extracted
from recent studies of quasi-periodic oscillations, see e.g., Ref.\ \cite{Lamb}
are confirmed, that would clearly exclude the softening of the EoS represented
by the introduction of hyperonic degrees of freedom.  It is also
generally accepted, see e.g., Ref.\ \cite{glen97}, that
theory must be able to account for pulsars of mass at least as large as
$1.5-1.6M_{\odot}$.
If the heavy neutron stars prove erroneous by more detailed observations
and only masses around $\sim 1.4M_\odot$ \cite{thorsett} are found, 
this implies that other degrees of freedom than pure nucleonic ones 
must occur already at a few times nuclear
saturation densities. 

Although we have only considered the formation of hyperons in neutron
stars, transitions to other degrees of freedom such as quark matter,
kaon condensation and pion condensation may or may not take place 
in neutron star matter.
We would however like to emphasize that the hyperon formation mechanisms
is perhaps the most robust one and is likely to occur in the interior 
of a neutron star, unless the hyperon self-energies are strongly repulsive due
to repulsive hyperon-nucleon and hyperon-hyperon interactions, a repulsion 
which  would contradict
present data on hypernuclei \cite{bando}. 
The EoS with hyperons yields however neutron star masses without 
rotational corrections which are even below $\sim 1.4M_\odot$.
This means that our EoS with hyperons needs to be stiffer,  
a fact which may in turn
imply that more complicated many-body terms not included in our calculations,
such as three-body forces between nucleons and hyperons and/or relativistic
effects,  are needed.

We are much indebted to H.~Heiselberg and V.~G.~J.~Stoks for many usuful comments.
This work has been supported by the DGICYT (Spain) Grant PB95-1249 and
the Program SCR98-11 from the Generalitat de Catalunya. One of the authors
(I.V.) wishes to acknowledge support from a doctoral fellowship of the
Ministerio de Educaci\'on y Cultura (Spain)."  

\begin{thebibliography}{200}


\bibitem{glen97}   N.~K.\ Glendenning, {\em Compact Stars: 
Nuclear Physics, Particle Physics and General Relativity}, 
(Springer-Verlag, New York, 1997) and references therein.
\bibitem{prakash97} M.\ Prakash, I.\ Bombaci, M.\ Prakash, 
                    P.~J.\ Ellis, J.M.\ Lattimer, and R.\ Knorren, 
                    Phys.\ Rep.\ {\bf 280}, 1 (1997). 
\bibitem{pke95} R.\ Knorren, M.\ Prakash and P.~J.\ Ellis,
Phys.\ Rev.\ C {\bf 52}, 3470 (1995).
\bibitem{ms96} J.\ Schaffner and I.\ Mishustin, Phys.\ Rev.\
C {\bf 53}, 1416 (1996). 
\bibitem{bg97} S.~Balberg and A.~ Gal, Nucl.\ Phys.\ {\bf A625}, 435 (1997).
\bibitem{blc99} S.~Balberg, I.~ Lichtenstadt, and G.~B.~Cook, preprint astro-ph/9810361.
\bibitem{bbs98} H.-J.\ Schulze, M.\ Baldo, U.\ Lombardo,
J.\ Cugnon, and A.\ Lejeune, Phys.\ Rev.\ C {\bf 57}, 704 (1998);
M.\ Baldo, G.F.\ Burgio, and H.-J.\ Schulze, Phys.\ Rev.\ C {\bf 58}, 3688 (1998).
\bibitem{sr99} V.~G.~J.\ Stoks and Th.~A.\ Rijken, preprint nucl-th/9901028.
\bibitem{rsy98} Th.~A.\ Rijken, V.~G.~J.\ Stoks, and Y.\ Yamamoto, Phys.\ Rev.\ C {\bf 59},
                21 (1998). 
\bibitem{sl99} V.~G.~J.\ Stoks and T.-S.~H.\ Lee, preprint nucl-th/9901030.
\bibitem{isaac99} I.~Vida\~na, A.\ Polls, A.\ Ramos, M.\ Hjorth-Jensen, 
                  and V.~G.~J.\ Stoks, unpublished.
\bibitem{apr98} A.\ Akmal, V.~R.\ Pandharipande, and D.~G.\ Ravenhall,
Phys.\ Rev.\ C {\bf 58}, 1804 (1998).
\bibitem{v18} R.\ B.\ Wiringa, V.~G.~J.\ Stoks, and R.\ Schiavilla, Phys.\ Rev.\ C {\bf 51}, 
38 (1995). 
\bibitem{hh99} H.\ Heiselberg and M.\ Hjorth-Jensen, Phys.\ Rep., in press 
and nucl-th/9902033.
\bibitem{hartle67} J.~B.\ Hartle, Ap.\ J.\  {\bf 150}, 1005 (1967).
\bibitem{Lamb}  M.~C.\ Miller, F.~K.\ Lamb, , and P.\  Psaltis,  Ap.\ J.\ {\bf 508}, 791 (1998).
\bibitem{thorsett} S.~E.\ Thorsett and D.\ Chakrabarty, Ap.\ J.\ {\bf 512}, 288 (1999).
\bibitem{bando} H.\ Band\={o}, T.\ Motoba, and J.\ \v{Z}ofka, Int.\ J.\ Mod.\
Phys.\ {\bf A5}, 4021 (1990).

\end{thebibliography}

%\end{multicols}

%\newpage

\begin{figure}[hbtp]
   \setlength{\unitlength}{1mm}
   \begin{picture}(100,140)
   \put(0,5){\epsfxsize=18cm \epsfbox{figone.eps}}
   \end{picture}
   \caption{Goldstone diagrams for the single-particle potential $u$.
            a) represents the contribution from nucleons only as hole
            states while b) includes only hyperons as hole states.
            The wavy line represents the $G$-matrix.}
   \label{fig:upot}
\end{figure}



\begin{figure}[hbtp]
   \setlength{\unitlength}{1mm}
   \begin{picture}(140,180)
   \put(0,-30){\epsfxsize=18cm \epsfbox{figtwo.ps}}
   \end{picture}
   \caption{Particle densities in $\beta$-stable neutron star matter
            as functions of the total baryonic density $n$.
            The upper panel represents the results obtained at the 
            Brueckner-Hartree-Fock level with the potential of Stoks and
            Rijken \protect\cite{sr99}. 
            In the  lower panel the nucleonic part of the self-energy of the nucleons
            has been replaced 
            with the EoS of Ref.\ \protect\cite{apr98}. }
   \label{fig:fraction}
\end{figure}

\begin{figure}
   \setlength{\unitlength}{1mm}
   \begin{picture}(140,180)
   \put(0,-30){\epsfxsize=18cm \epsfbox{figthree.ps}}
   \end{picture}
   \caption{Energy per baryon in $\beta$-stable neutron star matter for
            different approaches  as function of the total baryonic density $n$.
            See text for further details.}
   \label{fig:eosfig}
\end{figure}



\begin{figure}[hbtp]
   \setlength{\unitlength}{1mm}
   \begin{picture}(140,180)
   \put(0,-30){\epsfxsize=18cm \epsfbox{figfour.ps}}
   \end{picture}
   \caption{Chemical potentials in $\beta$-stable neutron star matter
            as functions of the total baryonic density $n$.
            The upper panel represents the results obtained at the 
            Brueckner-Hartree-Fock level with the potential of Stoks and
            Rijken \protect\cite{sr99}. 
            The lower panel includes results obtained
            with the EoS of Ref.\  \protect\cite{apr98}. }
   \label{fig:chempots}
\end{figure}



\end{document}


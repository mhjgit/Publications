%Hi David, 
%here's an eventual addendum to the paper, it should be less
%than a journal page. I've tried to not make it too long, but let me 
%know if you think it became too short.
%Regarding the nomenclature I always label sp orbits in the 
%shells of interest as 1s, 0d, 0f and 1p. Please correct me
%if you prefer 2s, 1d, 1f and 2p.
%Moreover, please feel free to `massage' the text the way 
%you feel for, remove whatever you may think seems 
%redundant to the discussion.
%BTW, my coordinates are
%\author{M.\ Hjorth-Jensen}
%\address{Nordita, Blegdamsvej 17, DK-2100 Copenhagen \O, Denmark}

%After 1 january 1999 it is
%\address{Department of Physics, University of Oslo, N-0316 Oslo, Norway}
%which hopefully will be my permanent address till I retire
%(another 30 years).

%All the best,
%morten

%*************   plain latex text with references at the bottom
\documentstyle{article}
\begin{document}


In order to obtain a microscopic 
 effective shell-model interaction
which spans both the $1s0d$ and the $0f1p$ shells, our many-body
scheme 
starts with a free nucleon-nucleon  interaction $V$ which is
appropriate for nuclear physics at low and intermediate energies. 
At present there are several potentials
available. The most recent versions of Machleidt and co-workers
\cite{cdbonn}, the Nimjegen group \cite{nim} and the Argonne
group \cite{v18} have a $\chi^2$ per datum close to $1$
with the respect to the Nijmegen data base \cite{nimbase}. 
The potential
model of Ref.\ \cite{cdbonn} is an extension 
of the one-boson-exchange
models of the Bonn group \cite{mac89}, where mesons like 
$\pi$, $\rho$, $\eta$, $\delta$, $\omega$ and the fictitious
$\sigma$ meson are included. In the charge-dependent version
of Ref.\ \cite{cdbonn}, the first five mesons have the same set
of parameters for all partial waves, whereas the parameters of
the $\sigma$ meson are allowed to vary. The recent Argonne potential
\cite{v18} is also a charge-dependent version of the Argonne
$V14$ \cite{v14} potential. The Argonne potential models
are local potentials in coordinate space and include
a $\pi$-exchange plus parametrizations of the short-range
and intermediate range part of the potential. The Nimjegen group   
\cite{nim} 
has contructed potentials based on meson exchange and models parametrized
in similars ways as the Argonne potentials.
Another important difference between e.g., the Bonn potentials
and the Argonne and Nimjegen potentials is the strength of the much debated
tensor force \cite{bm95}. Typically, the Bonn potentials have 
a smaller $D$-state admixture in the deuteron wave function
than the Argonne and Nimjegen potentials, as well as
other potential models. A smaller(larger) $D$-state
admixture in the ground state of the deuteron 
means that the tensor force is weaker(stronger).
The strength of the tensor force has important consequences 
in calculations of the binding energy for both
finite nuclei and infinite nuclear matter, 
see e.g., the discussion
in Ref.\ \cite{hko95}. 
A potential model  with a weak tensor force
tends to yield more attraction in a nuclear system than a 
potential with a strong tensor force. However,
all these modern nucleon-nucleon interactions yield
very similar excitation spectra. Moreover, in calculations
of Feynman-Goldstone diagrams in perturbation theory, a potential with a
weak tensor force tends to suppress certain intermediate
states of long-range character, like particle-hole excitations
\cite{sommer81}.
In this work we will thus choose to work with the charge-dependent
version of the Bonn potential models, see Ref.\ \cite{cdbonn}.

The next step 
in our many-body scheme is to handle 
the fact that the repulsive core of the nucleon-nucleon potential $V$
is unsuitable for perturbative approaches. This problem is overcome
by introducing the reaction matrix $G$ given by the solution of the
Bethe-Goldstone equation
\begin{equation}
    G=V+V\frac{Q}{\omega - H_0}G,
\end{equation}
where $\omega$ is the unperturbed energy of the interacting nucleons,
and $H_0$ is the unperturbed hamiltonian. 
The operator $Q$, commonly referred to
as the Pauli operator, is a projection operator which prevents the
interacting nucleons from scattering into states occupied by other nucleons.
In this work we solve the Bethe-Goldstone equation for several
starting
energies $\Omega$, by way of the so-called double-partitioning scheme
discussed in e.g.,  Ref.\ \cite{hko95}.  
As closed-shell core in the $G$-matrix calculation
we choose $^{40}$Ca and employ a harmonic-oscillator basis for the
single-particle
wave functions, with an oscillator energy $\hbar\Omega$ given
by
$\hbar\Omega = 45A^{-1/3} - 25A^{-2/3}=11 $ MeV,  
$A=40$ being the mass
number.

Finally, we briefly sketch how to calculate an effective 
two-body interaction for the chosen model space
in terms of the $G$-matrix.  Since the $G$-matrix represents just
the summmation to all orders of ladder diagrams with particle-particle
diagrams, there are obviously other terms which need to included
in an effective interaction. Long-range effects represented by 
core-polarizations terms are also needed.
The first step then is to define the so-called $\hat{Q}$-box given by
\begin{equation}
   P\hat{Q}P=PGP+
   P\left(G\frac{Q}{\omega-H_{0}}G\\ +G
   \frac{Q}{\omega-H_{0}}G \frac{Q}{\omega-H_{0}}G +\dots\right)P.
   \label{eq:qbox}
\end{equation}
The $\hat{Q}$-box is made up of non-folded diagrams which are irreducible
and valence linked.
We can then obtain an effective interaction
$H_{\mathrm{eff}}=\tilde{H}_0+V_{\mathrm{eff}}$ in 
terms of the $\hat{Q}$-box,
with \cite{hko95}
\begin{equation}
    V_{\mathrm{eff}}(n)=\hat{Q}+{\displaystyle\sum_{m=1}^{\infty}}
    \frac{1}{m!}\frac{d^m\hat{Q}}{d\omega^m}\left\{
    V_{\mathrm{eff}}^{(n-1)}\right\}^m,
    \label{eq:fd}
\end{equation}
where $(n)$ and $(n-1)$ refer to the effective interaction after
$n$ and $n-1$ iterations. The zeroth iteration is represented by just the 
$\hat{Q}$-box.
Observe also that the
effective interaction $V_{\mathrm{eff}}(n)$
is evaluated at a given model space energy
$\omega$, as is the case for the $G$-matrix as well. Here we choose
$\omega =-20$ MeV. The final interaction after folding
results in eigenvalues which depend rather weakly on the 
chosen starting energy, see e.g., Ref.\ \cite{converge93}
for a discussion.
All non-folded diagrams through 
second-order in the interaction $G$ are included.
For further details, see Ref.\ \cite{hko95}.
Finally, the reader should note that when one defines an
effective interaction for several shells, the effective
interaction may be strongly non-hermitian. 
This non-hermiticity should arise already at the level
of the $G$-matrix. However, since the $G$-matrix
is calculated at a fixed starting energy
for both incoming and outgoing states, it is by construction
hermitian. Since we are calculating an effective interaction at a
fixed starting energy, the individual diagrams
entering the definition of the $\hat{Q}$-box 
are thereby also made hermitian. The non-hermiticity 
which however stems from folded diagrams
is made explicitely hermitian through the 
approach of Suzuki {\em et al.\ } 
in Ref.\ \cite{kenji}.


However, microscopic effective interactions
like the above derived from available $NN$ interactions 
typically fail in reproducing properties of nuclei like 
e.g., shell-closure in $^{48}$Ca or excitation spectra
of the odd nuclei $^{47}$Ca and $^{49}$Ca \cite{hko95}.
Therefore, Zuker and co-workers \cite{zuk94,zuk98} 
have adopted a slightly
different scheme where the effective interaction is 
rewritten in terms of a multipole expansion \cite{zuk94}.
The monopole part of the interaction 
can in turn be rewritten in terms of
global, one-body and two-body contributions. 
These contributions are all proportional
to the energy centroids $\overline{V}_{ij}$
\begin{equation}
     \overline{V}_{ij}=
     \frac{\sum_J (2J+1) \left\langle ij \right | V_{\mathrm{eff}}^J 
     \left | ij \right \rangle}{\sum_J (2J+1)},
\end{equation}
where $i,j$ represent single-particle states
forming a two-body state coupled to final
angular momentum $J$.
 
In Ref.\ \cite{zuk98} it is demonstrated how
one can extract the energy centroids, which enter the 
one-body term of the monopole interaction, through the
available 
spectra of particle
and hole states of doubly magic cores. We have thus employed this
prescription to our effective interaction
This phenomenological recipe consists in replacing
our centroids $\overline{V}_{ij}$ with those 
derived by Duflo and Zuker \cite{zuk98}. 
With this prescription we are in turn able to resolve 
the spectroscopy problems we had in e.g., 
the excitation spectra of $^{47}$Ca, $^{48}$Ca and
$^{49}$Ca. Since however it is only the one-body part
of the monopole term which is corrected, the binding
energies will therefore not be properly reproduced.
But this feature does not affect the excitation spectra,
since it involves only a global term. 




% references


\begin{thebibliography}{200}
\bibitem{cdbonn} R.\ Machleidt, F.\ Sammarruca, and Y.\ Song,
Phys.\ Rev.\ C {\bf 53}, R1483  (1996).
\bibitem{nim} V.\ G.\ J.\ Stoks, R.\ A.\ M.\ Klomp, C.\ P.\ F.\ Terheggen, and J.\ J.\
de Swart, Phys.\ Rev.\ C {\bf 48}, 792 (1993).
\bibitem{v18} R.\ B.\ Wiringa, V.\ G.\ J.\ Stoks, and R.\ Schiavilla, Phys.\ Rev.\
C {\bf 51}, 38 (1995).
\bibitem{nimbase} V.\ G.\ J.\ Stoks, R.\ A.\ M.\ Klomp, C.\ P.\ F.\ Terheggen, and J.\ J.\
de Swart, Phys.\ Rev.\ C {\bf 49}, 2950 (1994).
\bibitem{mac89}  R.\ Machleidt, Adv.\ Nucl.\ Phys.\ 19 (1989)  189. 
\bibitem{v14} R.\ B.\ Wiringa, R.\ A.\ Smith, and T.\ L.\ Ainsworth,
Phys.\ Rev.\ C {\bf 29}, 1207 (1984).
\bibitem{bm95} G.\ E.\ Brown and R.\ Machleidt, 
Phys.\ Rev.\ C {\bf 50}, 1731 (1994).
\bibitem{hko95} M.\ Hjorth-Jensen, T.\ T.\ S.\ Kuo, and E.\ Osnes,
Phys.\ Rep.\ {\bf 261}, 125 (1995).
\bibitem{sommer81} H.\ M.\ Sommerman, H.\ M\"uther, K.\ C.\ Tam, and T.\ T.\ S.\ Kuo,
Phys.\ Rev.\ C {\bf 23}, 1765 (1981).
\bibitem{converge93} P.\ J.\ Ellis, T.\ Engeland, M.\ Hjorth-Jensen,
A.\ Holt, and E.\ Osnes, Nucl.\ Phys.\ {\bf A573}, 216 (1994).
\bibitem{kenji} K.\ Suzuki, R.\ Okamoto, P.\ J.\ Ellis, and
T.\ T.\ S.\ Kuo, Nucl.\ Phys.\  {\bf A567}, 565 (1994).
\bibitem{zuk94} A.\ P.\ Zuker, Nucl.\ Phys.\ {\bf A576}, 65 (1994).
\bibitem{zuk98} J.\ Duflo and A.\ P.\ Zuker, submitted to Phys.\ Rev.\ Lett.

\end{thebibliography}


\end{document}























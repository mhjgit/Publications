\documentstyle[aps,graphicx,multicol]{revtex}

\begin{document}

\begin{multicols}{2}

In order to analyze the criticality of low temperature
transitions in our nucleus, we define a function 
\begin{equation}
N(E,\beta,L) = \frac{L}{Z(\beta)}w(E)\exp\left(-\beta E\right);,
\end{equation}
where $\beta$ is the inverse temperature, $Z$ is the canonical
partition function, and $L$ represents the size of the system. 
We also define a second function, $A(E,\beta,L)=-\ln N(E,\beta,L)$.
Lee and Kosterlitz showed \cite{lk90,lk91} that the function $A(E,\beta)$ will
exhibit a characteristic double-minimum structure at energies $E_1$ and $E_2$
near a critical temperature $T_c=1/\beta_c$. At these minima, 
$A(E_1,\beta)=A(E_2,\beta)$. The bulk free-energy barrier, at energy $E_m$
between the states is then given by 
\begin{equation}
\Delta F = A(E_m,\beta)-A(E_1,\beta) \;.
\end{equation}
If $\Delta F$ increases or remains constant with increasing
system size, the transition is of first or higher order, respectively. 
We have made the assumption that $Z(\beta)$ varies slowly near the critical
value of $\beta$. Furthermore, with this assumption $A(E,\beta)$ is equal
to the free-energy $F(E,\beta)$ to within an addative constant; therefore,
in the following we refer to the free-energy. 
These ideas have recently been applied analyse phase
transitions in a schematic pairing model \cite{belic01}. 


INSERT SOMEWHERE: 
The phase transitions 
considered here are low-energy phenomena that exclude
the transition in shape of the nucleus. Many of the
systems studied here are highly deformed. While the
initial pair breaking will soften that deformation slightly,
calculations indicate that 
the transition from deformed to spherical should occur at
excitation energies between 5 and 10 MeV, somewhat higher
than the excitation energies at which the first pairs
break \cite{dean93,white00}



\begin{thebibliography}{99}
\bibitem{lk90}
J. Lee and J.M. Kosterlitz, Phys. Rev. Lett. {\bf 65}, 137 (1990).
\bibitem{lk91}
J. Lee and J.M. Kosterlitz, Phys. Rev. B {\bf 43}, 3265 (1991).
\bibitem{belic01}
A. Belic, D.J. Dean, and M. Hjorth-Jensen, preprint, 
cond-matt/0104138 (2001).
\bibitem{dean93} 
D.J. Dean, S.E. Koonin, G.H. Lang, W.E. Ormand, and B.P. Radha, 
Phys. Lett. B {\bf 317}, 275 (1993). 
\bibitem{white00}
J. White, S.E. Koonin, and D.J. Dean, Phys. Rev. C {\bf 61}, 34303 (2000).
\end{thebibliography}

\end{multicols}


\end{document}

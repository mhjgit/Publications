\documentclass{letter}

\begin{document}

\address{Magne Guttormsen\\
Department of Physics\\ 
University of Oslo\\
P.O.Box 1048 Blindern\\ 
N-0316 Oslo, NORWAY}

\signature{Magne Guttormsen}

\begin{letter}{Debbie Brodbar,\\
Assistant Editor\\                               
Physical Review C}

\opening{Re: LW8391CR\\
Free energy and criticality in the nucleon pair breaking process by\\ 
M. Guttormsen, R. Chankova, M. Hjorth-Jensen, J. Rekstad, \sl et al\rm.}

Dear Debbie Brodbar,

thank you very much for the referee reports on our manuscript LW8391CR. You 
will find our reply to the Referee below.

We would not like our draft to become a Brief Report, since it seems not 
possible to us to make all the necessary additions to the manuscript for the
benefit of the reader and the referee, and still remain below the page limit.
We would agree with possible publication in the Rapid Communication or the
Regular Article section.

We would like to point out that our manuscript was submitted in September 26. 2002, and at June 3. 2003 has been during this period "with authors" for only 22 days. Our work has not been added any new essential points, other than convincing Referee A that the inclusion of neutrons and protons in the analysis gives no changes, a fact that most nuclear physicists should know.

We were surprised that Referee A was chosen in the transfer process from Physical Review Letters to Rapid Communication. Referee A showed less competence in the very first report suggesting that the inclusion of protons and neutrons would reduce the critical temperature with a factor of two. Referee C also pointed out that referee A was wrong. We are also surprised that the Editor of Physical Review Letters found (January 9. 2003) the report from referee A persuasive, although Referee B ended his/she's report with the sentence: "These new results describing the pairing phase transition in a mesoscopic system represent an important new result which should be of general interest to the broad readership of Physical Review Letters." To us, Referee B gave the most competent report.

The revised manuscript is attached (draft.tar.gz) with the following changes:

\begin{enumerate}
\item The article is given the format of Regular Article with 5 sections and Acknowledgement. In two places 'Rapid Communication' has been replaced.
\item In the eighth paragraph, we have added 'thereby avoiding the introduction
of a caloric curve $T(E)$' at the end of the seventh sentence.
\item We have changed Fig.\ 2 and included the cases of odd-mass and odd-odd
nuclei. We have changed the ninth paragraph, the thirteenth paragraph and the
figure caption in order to reflect the changes in the figure.
\item A new Fig.\ 3 is inserted.
\item The two last paragraphs of Sect. III (in the revised manuscript) are new.
\item In the tenth paragraph, in the first sentence, we have changed 'this 
behavior' by 'the behavior'.
\item In the fourteenth paragraph, in the end of the sixth sentence, we have 
corrected 'phase transition' to 'first order phase transition'.
\item In the sixteenth paragraph, in the end of the last sentence, we have 
changed 'their even mass nuclei' to 'the even mass nuclei'.

\end{enumerate}

\closing{Yours sincerely,}

\end{letter}

\clearpage

\large Reply to the third report of Referee A\normalsize

The increase in level density should indeed be different depending on whether
only one pair (of e.g.\ neutrons) or two pairs (one pair of neutrons and one 
pair of protons) are broken. Every pair breaking gives naively an increase in 
level density (or entropy) in the order of two units of single-particle entropy
or $2s$. With $s=2k_{\mathrm{B}}$, one expects an increase in level density of
roughly a factor of $\exp(4)=55$ at the energy $E=2\Delta$. If \underline{either proton or neutron pairs} are broken at $E=2\Delta$, we will have twice the number of states, namely 110. This coincides quite well with the magnitude of the steps in Fig.\ 1 of our manuscript, showing steps in the order of a factor 100.

If \underline{both neutron and proton pairs} are broken, we enter the 4 quasiparticle regime at $E=4\Delta$. Here, the states are built of $\pi^4$, $\pi^2\nu^2$ and $\nu^4$ quasiparticles giving totally a factor of 3*55*55=9075 more states, which also coincides with the level densities found around $E=4$ MeV in Fig.\ 1. 

It is, of course, conceivable that the breaking of the first proton and the 
first neutron pair will happen at different critical temperatures. (Only the 
breaking of the first pair is relevant in the discussion). However, there are
many physics reasons why the two critical temperatures should be very similar. 

\begin{enumerate}
\item The strong interaction is approximately independent on nucleon flavor,
i.e., it is equal for protons and neutrons.
\item The gap parameter $\Delta$ is approximately equal for protons and 
neutrons and has successfully been described by $\Delta=12/\sqrt{A}$~MeV (see,
e.g., the book of Bohr and Mottelson).
\item Interactions between protons and neutrons will almost certainly wash out
most differences of the proton and neutron fluids which might arise from, e.g.,
the difference in proton and neutron number.
\item We would like to recall our previous arguments on this topic in the two 
earlier comments to the referee reports and the statement of Referee C about 
this issue.
\item The experimental data seem to favor the interpretation in the framework
of only one critical temperature for both nucleon species.
\end{enumerate}

In order to follow the requirement of the referee, we have changed Fig.\ 2 and inserted a new Fig.\ 3 with new text that explicitly discusses the role of the proton and neutron pair breaking process. We hope that the referee find our changes adequate.


\end{document}



%   this is section 1

The purpose of this work is to give  a presentation of recent progresses
in perturbative many-body theories, with applications  to
effective interactions for nuclear matter and finite nuclei.
Several of the results presented here are new
or are extended revisions of earlier compilations for nuclei in the
mass regions of oxygen, calcium and tin.



Our perturbative many-body approach
rests on the time-dependent expansion
within the framework of degenerate
or nearly degenerate Rayleigh-Schr\"{o}dinger
(RS) perturbation theory \cite{ko90,eo77,kuo81,hom92,no88,br86,fw71}.
The choice of the time-dependent
approach rather than the time-independent, exposed {\em ad extenso}
in the work by Lindgren and Morrison \cite{lm85}, is deliberate, since
it conveniently lends itself to extensions to systems
at finite temperature \cite{no88,kap88}.
Furthermore, since the nuclear systems we will cover are described
in terms of valence holes and/or particles, a convenient scheme which
microscopically accounts for the empirical shell or optical models, is
provided by a so-called valence-linked perturbative expansion
\footnote{The technicalities of the perturbative expansion are given
in section 4.}.
A key point to the present investigation, is the evaluation
of folded diagrams \cite{ko90,bran67} in connection with the
valence-linked expansion.
Folded diagrams arise
in Rayleigh-Schr\"{o}dinger (RS) perturbation theory due to the removal of
the dependence on the exact energy of the perturbation series.
A major function of the folded diagrams
is to account for the fact that the valence
particle(hole)
is not in the given single-particle(hole) state all the time. This
correction may actually be important, as its effect is to renormalize
the non-folded terms by a wound integral factor, being the
probability that the valence particle(hole) is absent from the particular 
orbit considered.




The philosophy of this work is in essence to derive properties
of selected nuclear systems from a microscopic point of view.
With microscopic we will mean a nuclear many-body approach outlined
in the following  three steps:
\begin{itemize}
\item First, one needs a nucleon-nucleon (NN) interaction $V$ which is
appropriate for nuclear physics at low and
intermediate energies.
Since we deal with strong interactions, a really
microscopic approach would be to derive nuclear expectation values
from the underlying theory of the strong interaction, namely
quantum chromodynamics (QCD). Due to the strongly
non-perturbative character of QCD in the regime of nuclear physics,
such an approach is at present not available. However, there are
both theoretical and experimental indications that interaction
models based on meson-exchange offer a viable scheme at low and
intermediate energies. With microscopic we will therefore
mean a theory which starts from selected baryons and mesons as the
nuclear constituents.
The nucleon-nucleon interaction is then described in terms of the
exchange of selected mesons,
see e.g.\ refs.\ \cite{mhe87,mac89,paris80,reid68}.
\item Secondly,
in nuclear many-body calculations, the first problem one is confronted
with is the fact
that the repulsive core of the NN potential $V$ is unsuitable for
a perturbative treatment. This problem is however overcome by
introducing the reaction matrix $G$ given by the solution
of 
\begin{equation}
G=V+V\frac{Q}{\omega - QH_0 Q}G,
\label{eq:gmat}
\end{equation}
where $\omega$ is the unperturbed energy of the interacting nucleons in
a medium, and $H_0$ is the unperturbed hamiltonian.
The operator $Q$,
commonly referred to as the Pauli operator, is a projection
operator which prevents the interacting nucleons from scattering into
states occupied by other nucleons.
\item Finally,
employing the $G$-matrix defined in the previous step and the
perturbative techniques to be discussed in this work, one
can derive expressions for effective transition operators and interactions
in terms of the
$G$-matrix. The effect of higher-order terms in the perturbative
expansion account for nuclear correlations which modify e.g.\ the
predictions made within the framework of the
independent or simple shell-model
picture. The appealing feature of perturbation theory is that it offers
a scheme by which one can interpret medium effects in terms of contributions
to the perturbative expansion. Pictorially, various terms in perturbation
theory  are conventionally exhibited by way of so-called Feynman diagrams.
The latter offer a physical and  intuitive understanding of perturbation
theory. The main emphasis of this work rests on this
diagrammatic representation of perturbative methods, though,
our
microscopic approach presumes all three steps of the algorithm we have
outlined.
\end{itemize}

This work falls in 9 sections. After the above introductory remarks, we
review the philosophy behind the $G$-matrix and Brueckner theory in section
2.
Section 3 is devoted to a discussion
of potential models for baryons interacting through meson exchange, while
in section 4 we present several perturbative many-body approaches
based on both time-dependent and time-independent perturbation
theory.
In section 5 we
detail the renormalization of the bare interaction
in a nuclear medium in terms of the $G$-matrix discussed above. An extensive
discussion of the properties of the $G$-matrix in both nuclear matter and
finite
nuclei is given. 
The subsequent section
is the first one on applications of the perturbative
methods exposed in section 4.
Here we review results from two-body effective
interactions applied to several nuclei with two or more valence nucleons.
The nuclei we investigate are in the mass regions of oxygen, calcium and tin,
with an emphasis on the evaluation
of folded diagrams in connection with the
valence-linked perturbative expansion.  Several properties and problems
related to the evaluation of the effective interaction are discussed,
such as the order-by-order convergence, the intruder state problem,
the non-hermiticity of the RS perturbation expansion and so forth.



Section 7 includes a discussion of a model-space approach to nuclear
matter, an approach which is comparable in philosophy to the
effective interaction approach for finite nuclei. Higher order
diagrams such as the all-order summation of particle-particle and
hole-hole ladder diagrams in nuclear matter are also reviewed within
the framework of this model-space approach. The systems considered
in section 6 can all be described fairly well within the framework
of non-relativistic many-body theories. However, for our nuclear
matter results of section 7, such a non-relativistic description
at high nuclear densities, is obviously inadequate. Even at the
nuclear matter saturation point, the lowest relativistic correction
to the kinetic energy per particle is of the order of $2$ MeV
\cite{kap88}, a
quantity which should be compared to the total binding energy
per particle of approximately $-16$ MeV. At densities higher than the
saturation point, relativistic effects become even more important.
A relativistic extension of our nuclear matter results of
section 7, is made  in section 8 within the framework
of the so-called Dirac-Brueckner-Hartree-Fock approach \cite{bm90,brook}.


We conclude this presentation with an
overview and a discussion of perspectives for future investigations
in section 9.

The final effective interactions we derive for e.g.\ the oxygen 
area, are rather close to the original ones of Kuo and Brown
\cite{kb66}, as also
observed by Zuker and co-workers \cite{acz91}. 
In ref.\ \cite{kb66},
a potential with a strong tensor force was employed, and only
terms through second order in Rayleigh-Schr\"{o}dinger perturbation
theory were included. There is strong
theoretical and experimental evidence that the tensor
force of the  NN potential should be weak. Most modern
NN potentials have such a weak tensor force. This, as will be
demonstrated throughout this work, introduces more attraction
in the nuclear systems considered. Though, through the
introduction of many-body effects not taken into account
by Kuo and Brown, the final renormalized effective interaction
is close to e.g.\ that  in ref.\ \cite{kb66}. The main idea
of Kuo and Brown however, i.e.\ that the effective interaction 
is obtained by the $G$-matrix renormalization plus
the core-polarization correction, which also includes
collective long-range effects, is manifest in most many-body
contributions presented in this work.



Finally, the reader should be aware of the fact
that extensive review articles
exist which cover in depth the subjects discussed in sections
3 to 4.
The derivation of potential models based on meson-exchange is
extensively detailed in the works of Machleidt and co-workers
\cite{mhe87,mac89}, whereas much of the formalism discussed in section
4 is taken from the recent monograph of Kuo and Osnes \cite{ko90},
also several tutorial text books cover in an elegant way recent advances
in many-body theories \cite{no88,br86,lm85,kap88}.
In section 8 we borrow extensively from the relativistic work
of Brockmann and Machleidt \cite{bm90} and the Brooklyn group
of Celenza and Shakin \cite{brook}.
Further, a work which follows much of
the philosophy of the present work, i.e.\
our microscopic approach, was recently
presented by Dickhoff and M\"{u}ther \cite{dm92}.
In that work, recent developments
in many-body theory applied to nuclear systems, were reviewed with a
emphasis on a self-consistent Green's function approach.
As a collation, it ought to be mentioned that in the present approach,
except for an illustratory discussion of nuclear and neutron matter in
sections 7 and 8,
we do not attempt at describing bulk properties of
nuclei, such as total binding energies or mean radii of nuclei like
$^{16}$O. A viable approach to this problem is offered by
variational methods, reviewed  by e.g.\ Pandharipande in ref.\
\cite{pand93}.


\subsection{Introduction}
There are basically two main approaches in perturbation theory used
to define an effective operator and the effective interaction,
each with its hierarchy of sub-approaches. One of these main
approaches is an energy-dependent, i.e. depending on the
exact energy $E$, approach, known as Brillouin-Wigner
perturbation theory, while the Rayleigh-Schr\"{o}dinger (RS) perturbation
expansion stands for the energy independent approach. The latter is
the most commonly used approach in the literature \cite{ko90,lm85}.

In order to derive a microscopic theory of the effective interaction
within the framework of perturbation theory, we need to introduce various
notations and definitions pertinent to the methods exposed.
This is discussed in the first subsection together with an outline of
time-independent RS perturbation theory for the effective interaction.
Throughout this section, we focus on a valence-linked effective
interaction.
In the next subsection  we review how to
calculate the effective interaction
within the framework of time-dependent degenerate
or nearly degenerate
RS perturbation theory \cite{ko90,no88}.

\subsection{Time-independent
perturbation theory}

It is common practice in perturbation theory to reduce the infinitely
many degrees of freedom of the Hilbert space to those represented
by a physically motivated subspace, the model space.
In such truncations of the Hilbert space, the notions of a projection
operator $P$ onto the model space and its complement $Q$ are
introduced. The projection operators defining the model and excluded
spaces are defined by
\begin{equation}
    P=\sum_{i=1}^{D} \ket{\psi_i}\bra{\psi_i},
\end{equation}
and
\begin{equation}
    Q=\sum_{i=D+1}^{\infty} \ket{\psi_i}\bra{\psi_i},
\end{equation}
with $D$ being the dimension of the model space, and $PQ=0$, $P^2 =P$,
$Q^2 =Q$ and $P+Q=I$. The wave functions $\ket{\psi_i}$ are eigenfunctions
of the unperturbed hamiltonian $H_0 = T+U$ (with eigenvalues
$\varepsilon_i$), where $T$ is the kinetic
energy and $U$ an appropriately chosen one-body potential, normally
that of the
harmonic oscillator (h.o.). The full hamiltonian
is then rewritten as $H=H_0 +H_1$ with $H_1=V-U$, $V$ being e.g. the
nucleon-nucleon (NN) interaction\footnote{In this section we will
assume a potential $V$ which is suitable for perturbative
methods. In section 5 we will discuss how $V$ is to be represented
in a nuclear medium, in order to apply the perturbative
techniques to be discussed here.}.

We define the projection of the exact wave function $\ket{\Psi_{\alpha}}$
of a state $\alpha$, i.e. the solution to the full
Schr\"{o}dinger equation
\begin{equation}
   H\ket{\Psi_{\alpha}} =E_{\alpha}\ket{\Psi_{\alpha}},
   \label{eq:schro}
\end{equation}
as $P\ket{\Psi_{\alpha}}=\ket{\Psi_{\alpha}^M}$ and a wave operator $\Omega$
which transforms all the model states back into the corresponding
exact states as $\ket{\Psi_{\alpha}}=\Omega\ket{\Psi_{\alpha}^M}$.  The latter
statement is however not trivial, it actually means that there is
a one-to-one correspondence between  the $d$ exact states and the
model functions. Actually, the only proof we are aware of, is given
within the framework of the complex-time approach to be discussed
in the next subsection.


We will now assume that the wave operator $\Omega$ has an inverse and
consider a similarity transformation of the hamiltonian $H$ such that
eq.\ (\ref{eq:schro}) can be rewritten as
\begin{equation}
   \Omega^{-1}H\Omega\Omega^{-1}\ket{\Psi_{\alpha}} = 
    E_{\alpha}\Omega^{-1}\ket{\Psi_{\alpha}}.
   \label{eq:htrans}
\end{equation}
Recall also that $\ket{\Psi_{\alpha}}= 
\Omega\ket{\Psi_{\alpha}^{M}}$, which means that
 $\Omega^{-1}\ket{\Psi_{\alpha}} = \ket{\Psi_{\alpha}^{M}}$ insofar as the inverse of
 $\Omega$ exists. Let us define the transformed hamiltonian ${\cal H}
 =\Omega^{-1}H\Omega$, which can be rewritten in terms of the
operators $P$ and $Q$ ($P+Q=I$) as
\begin{equation}
  {\cal H}=P{\cal H}P+P{\cal H}Q+Q{\cal H}P+Q{\cal H}Q.
\end{equation}
The eigenvalues of ${\cal H}$ are the same as those of $H$, since a
similarity transformation does not affect the eigenvalues.
If we now operate  on eq.\ (\ref{eq:htrans}), which in terms of the model
space wave function reads
\begin{equation}
  {\cal H}\ket{\Psi_{\alpha}^{M}} = E_{\alpha}\ket{\Psi_{\alpha}^{M}},
\end{equation}
with the operator $Q$, we readily see that
\begin{equation}
   Q{\cal H}P=0.\label{eq:qhp}
\end{equation}

Eq.\ (\ref{eq:qhp}) is an important relation which states that the eigenfunction
$P\ket{\Psi_{\alpha}}$ is a {\em pure model space eigenfunction}. This implies that
we can define an {\em effective model space hamiltonian}
\begin{equation}
  H_{\mathrm{eff}}=P{\cal H}P = P\Omega^{-1} H \Omega P,
\end{equation}
or equivalently
\begin{equation}
   H\Omega P=\Omega PH_{\mathrm{eff}}P,
  \label{eq:bloch}
\end{equation}
which is the Bloch \cite{bloch59} equation. This equation can
be used to determine the wave operator $\Omega$.

At this stage we should point out that there are no unique representations
for the wave operator $\Omega$. There is a considerable degree of variation
in the choice of $\Omega$, yielding different approaches to the perturbative
expansion of the effective interaction. As a consequence,
since in applications we truncate the perturbative expansion at a given
order, different choices for the wave operator $\Omega$ may then give
different results for the effective interaction. Moreover, even the same
defining equation for the wave operator can be solved by e.g.\
different iterative schemes. Let us consider some examples.
The wave operator is often expressed as
\begin{equation}
  \Omega = 1 +\chi,
\end{equation}
where $\chi$ is known as the correlation operator. The correlation
operator generates the component of the wave function in the $Q$-space
and must therefore contain at least one interaction. Observing
that $P\Omega P = P$, we see that the correlation operator $\chi$
has the properties
\begin{equation}
   P\chi P = 0, \hspace{1cm} Q\Omega P = Q\chi P =\chi P. \label{eq:chi1}
\end{equation}
Since  
$\ket{\Psi_{\alpha}}=\Omega\ket{\Psi_{\alpha}^{M}}$ determines the wave operator
only when it operates to the right on the model space, i.e. only the
$\Omega P$  part is defined, the term $\Omega Q$
never appears in the theory,
and we could therefore add the conditions $Q\chi Q =0$ and $P\chi Q =0$
to eq.\ (\ref{eq:chi1}). This leads to the following choice for $\chi$
\begin{equation}
   \chi = Q\chi P. \label{eq:chi2}
\end{equation}

The wave operator $\Omega$ can be ordered in terms of the number
of interactions with the perturbation $H_1$
\begin{equation}
\Omega = 1 +\Omega^{(1)} + \Omega^{(2)}+\dots ,
\end{equation}
where $\Omega^{(n)}$ means that we have $n$ $H_1$ terms. Explicitly,
the above
equation reads
\begin{eqnarray}\label{eq:wavef}
   \Omega\ket{\psi_{\alpha}}=&{\displaystyle\ket{\psi_{\alpha}}
   +\sum_{i}\frac{\ket{i}
   \bra{i}H_1\ket{\psi_{\alpha}}}{ \varepsilon_{\alpha}-\varepsilon_i}
   +\sum_{ij}\frac{\ket{i}
   \bra{i}H_1\ket{j}\bra{j}H_1\ket{\psi_{\alpha}}}
  {(\varepsilon_{\alpha}-\varepsilon_i)
  (\varepsilon_{\alpha}-\varepsilon_j)} }\\   \nonumber
& {\displaystyle  -\sum_{\beta i}\frac{\ket{i}
   \bra{i}H_1\ket{\psi_{\beta}}\bra{\psi_{\beta} }H_1\ket{\psi_{\alpha}}}
  {(\varepsilon_{\alpha}-\varepsilon_i)
  (\varepsilon_{\alpha}-\varepsilon_{\beta})} }
  +\dots ,
\end{eqnarray}
where $\varepsilon$ are the unperturbed energies of the $P$-space
and $Q$-space states defined by $H_0$.
The latin letters refer to
$Q$-space states, whereas greek letters refer to model-space
states.
Throughout this work we will employ such a notation. The second term
in the above equation corresponds to $\Omega^{(1)}$ while the third
and fourth define $\Omega^{(2)}$.
Note that the fourth term diverges
in case we have a degenerate or nearly degenerate model space. It is
actually divergencies like these which are to be removed by the folded
diagram procedure. We will come back to this in section 4.3. Terms
like these
arise due to the introduction of an energy independent perturbative
expansion. Conventionally, the various contributions to the
perturbative expansion are represented by Feynman-Goldstone
diagrams, or just
Feynmann-Goldstone diagrams.
In fig.\ \ref{fig:chap2effint} we display some contributions to
the wave operator eq.\ (\ref{eq:wavef})
in terms of Goldstone diagrams.
\begin{figure}[hbtp]
      \setlength{\unitlength}{1mm}
      \begin{picture}(140,80)
      \put(25,10){\epsfxsize=12cm \epsfbox{chap2effint.eps}}
      \end{picture}
\caption{Examples of diagrams which contribute to the wave operator
$\Omega^{(n)}$,
Both the railed lines, which  denote particle states,
and lines with downgoing arrows, hole states, belong to the excluded
space $Q$.}
\label{fig:chap2effint}
\end{figure}

One way of obtaining $\Omega$ is through the generalized Bloch
equation given by Lindgren and Morrison \cite{lm85}
\begin{equation}
[\Omega, H_0]P=QH_1\Omega P-\chi PH_1\Omega P,
\label{eq:lind}
\end{equation}
which offers a suitable way of generating the RS perturbation expansion.
Writing eq.\ (\ref{eq:lind}) in terms of $\Omega^{(n)}$ we have
\begin{equation}
[\Omega^{(1)}, H_0]P=QH_1P,
\end{equation}
\begin{equation}
[\Omega^{(2)}, H_0]P=QH_1\Omega^{(1)} P- \Omega^{(1)} PH_1P,
\end{equation}
and so forth,  which can be generalized to
\begin{equation}
[\Omega^{(n)}, H_0]P=QH_1\Omega^{(n-1)} P- \sum_{m=1}^{n-1}
\Omega^{(n-m)} PH_1\Omega^{(m-1)}P.
\end{equation}
The effective interaction to a given order can then be obtained from
$\Omega^{(n)}$, see \cite{lm85}.





Another possibility is to assume that we can start with a given
approximation to $\Omega$, and through an iterative scheme generate
higher order terms. Such schemes will in general differ from the
order-by-order scheme of eq.\ (\ref{eq:lind}).  Two such iterative
schemes were derived  by Lee and Suzuki \cite{ls80}. 


Having defined the wave operator $\Omega = 1 +\chi$ (note that $
\Omega^{-1}=1-\chi$) with
$\chi$ given by eq.\ (\ref{eq:chi2}) we can rewrite
eq.\ (\ref{eq:qhp}) as
\begin{equation}
     QHP-\chi HP +QH\chi - \chi H\chi = 0. \label{eq:basic}
\end{equation}
This is the basic equation to which a solution to $\chi$ is
to be sought.
Since we will work with a degenerate model space we define
\[
   PH_0 P = \omega P,
\]
where $\omega$ is the unperturbed model space eigenvalue
(or starting energy) in the degenerate case,
such that eq.\ (\ref{eq:basic}) reads in a slightly modified form
($H=H_0 + H_1$)
\[
    (\omega -QH_0 Q -QH_1 Q)\chi = QH_1 P -\chi PH_1 P -\chi PH_1 Q\chi,
\]
which yields the following equation for $\chi$
\begin{equation}
    \chi = \frac{1}{\omega - QHQ}QH_1 P -\frac{1}{\omega -QHQ}\chi\left(PH_1 P +
    PH_1 Q\chi P\right).\label{eq:chi3}
\end{equation}
Observing that  the $P$-space effective hamiltonian is given as
\[
     H_{\mathrm{eff}}= PHP+PH\chi=PH_0 P + V_{\mathrm{eff}}(\chi),
\]
with $V_{\mathrm{eff}}(\chi)= PH_1 P + PH_1Q\chi P$, eq. (\ref{eq:chi3}) becomes
\begin{equation}
     \chi = \frac{1}{\omega - QHQ}QH_1 P -\frac{1}{\omega -QHQ}
     \chi V_{\mathrm{eff}}(\chi ).
     \label{eq:chi4}
\end{equation}
Now we find it convenient to introduce the so-called $\hat{Q}$-box,
defined as
\begin{equation}
     \hat{Q}(\omega)=PH_1 P + PH_1 Q\frac{1}{\omega - QHQ}
      QH_1 P.\label{eq:qbox}
\end{equation}
The $\hat{Q}$-box is made up of non-folded diagrams which are irreducible
and valence linked. A diagram is said to be irreducible if between each pair
of vertices there is at least one hole state or a particle state outside
the model space. In a valence-linked diagram the interactions are linked
(via fermion lines) to at least one valence line. Note that a valence-linked
diagram can be either connected (consisting of a single piece) or
disconnected. In the final expansion including folded diagrams as well, the
disconnected diagrams are found to cancel out \cite{ko90}. 
This corresponds to the cancellation of unlinked diagrams
of the Goldstone expansion discussed in section 2.
We illustrate
these definitions by the diagrams shown in fig.\ 
\ref{fig:diagsexam}, where an arrow pointing upwards
(downwards) is a particle (hole) state.
\begin{figure}[hbtp]
      \setlength{\unitlength}{1mm}
      \begin{picture}(140,80)
      \put(25,10){\epsfxsize=12cm \epsfbox{diagsexam.eps}}
      \end{picture}
\caption{Different types of valence-linked diagrams. Diagram (a)
is irreducible and connected, (b) is reducible, while (c) is irreducible
and disconnected.}
\label{fig:diagsexam}
\end{figure}
Particle states outside the model space are given by railed lines. 
Diagram (a) is irreducible, valence linked and connected, 
while (b) is reducible since 
the intermediate particle states belong to the model space. 
Diagram (c) is irreducible, valence linked and disconnected. 



Multiplying both sides of eq.\ (\ref{eq:chi4}) with $PH_1$ and
adding $PH_1 P$ to both sides we get
\[
    PH_1 P + PH_1 \chi =
    PH_1 P + PH_1 Q\frac{1}{\omega - QHQ}QH_1 P -
    PH_1 \frac{1}{\omega -QHQ}\chi V_{\mathrm{eff}}(\chi ),
\]
which gives
\begin{equation}
     V_{\mathrm{eff}}(\chi )=\hat{Q}(\omega)-
     PH_1 \frac{1}{\omega -QHQ}\chi V_{\mathrm{eff}}(\chi ).
     \label{eq:veff}
\end{equation}

There are several ways to solve eq.\ (\ref{eq:veff}). The idea is
to set up an iteration scheme where we determine $\chi_n$ and
thus $V_{\mathrm{eff}}(\chi_n )$ from 
$\chi_{n-1}$ and $V_{\mathrm{eff}}(\chi_{n-1})$.
For the mere sake of simplicity we write 
$V_{\mathrm{eff}}^{(n)}=V_{\mathrm{eff}}(\chi_{n})$.

\subsubsection{The FD method}

Let us write eq.\ (\ref{eq:veff}) as
\[
   V_{\mathrm{eff}}^{(n)}=\hat{Q}(\omega)-
   PH_1 \frac{1}{\omega -QHQ}\chi_n V_{\mathrm{eff}}^{(n-1)}.
\]
The solution to this equation can be shown to be \cite{ls80}
\begin{equation}
    V_{\mathrm{eff}}^{(n)}=\hat{Q}+{\displaystyle\sum_{m=1}^{\infty}}
    \frac{1}{m!}\frac{d^m\hat{Q}}{d\omega^m}\left\{
    V_{\mathrm{eff}}^{(n-1)}\right\}^m . 
    \label{eq:fd}
\end{equation}
Observe also that the
effective interaction is $V_{\mathrm{eff}}^{(n)}$ 
is evaluated at a given model space energy
$\omega$. If
$V_{\mathrm{eff}}^{(n)}=V_{\mathrm{eff}}^{(n-1)}$, the iteration is said to
converge. In the limiting case $n\rightarrow \infty$, the
solution $V_{\mathrm{eff}}^{(\infty)}$ agrees with the formal solution of
Brandow
\cite{bran67} and Des Cloizeaux \cite{des}
\begin{equation}
    V_{\mathrm{eff}}^{(\infty)}=\sum_{m=0}^{\infty}\frac{1}{m!}
    \frac{d^{m}\hat{Q}}{d\omega^{m}}\left\{
    V_{\mathrm{eff}}^{(\infty)}\right\}^{m}.\label{eq:pert}
\end{equation}


Note that although $\hat{Q}$ and its derivatives contain disconnected
diagrams, such diagrams cancel exactly in each order \cite{ko90}, thus
yielding a fully connected expansion in eq.\ (\ref{eq:fd}).
It is further important to note both in connection with the FD expansion and in
the subsequent discussion of the LS method, that a term like
$F_1= \hat{Q}_1 \hat{Q}$ actually means $P\hat{Q}_1 P\hat{Q}P$ since
the $\hat{Q}$-box is defined in the model space only. Here we have defined
$\hat{Q}_{m}=\frac{1}{m!}\frac{d^{m}\hat{Q}}
{d\omega^{m}}$.
Due to this structure, only so-called folded diagrams
contain $P$-space intermediate states.

The perturbation expansion of eq.\ (\ref{eq:pert})
diverges if so-called intruder states are
present in the low-lying spectrum, as shown by Schucan and Weidenm\"{u}ller
\cite{sw72}. Typical examples of intruder states for nuclei with
two valence nucleons are represented by four-particle two-hole
core-deformed
states. Several techniques to handle the intruder state problem
have been discussed during the last two decades \cite{eo77,lk75,lk77,sm90}.
Here we merely observe that a regrouping of the perturbation expansion
as given by the partial summation of the folded diagrams in the FD method,
may result in a converged result even if intruder states are present.
If the FD expansion converges, the converged solutions are then those states
which have the largest overlap with the chosen model space \cite{ls80}.
If the coupling between the model space states and the intruder states
is weak, as in the weak coupling model of ref.\ \cite{ee70}, the
derived two-body effective interaction is expected to reproduce fairly well
the spectra of nuclei with few valence nucleons.  We demonstrate this 
in a simple example at the end of this subsection.

\subsubsection{The LS method}

Another iterative solution has been
presented by Lee and Suzuki \cite{ls80}.
In their approach eq.\ (\ref{eq:veff}) is rewritten as
\begin{equation}
    V_{\mathrm{eff}}^{(n)}=
    \left\{1+PH_1 \frac{1}{\omega -QHQ}\chi_{n-1}\right\}^{-1}
    \hat{Q}(\omega), \label{eq:ls}
\end{equation}
for $n\geq 1$ and with the equation for $\chi_n$ given as
\begin{equation}
       \chi_n =Q\frac{1}{\omega - QHQ}QH_1 P -Q\frac{1}{\omega -QHQ}\chi_{n-1}
       V_{\mathrm{eff}}^{(n)}.
       \label{eq:chils}
\end{equation}
Since the wave operator $\Omega$ is given as $\Omega = P+\chi$, a natural
zeroth approximation to $\chi_0$ would be to set $\chi_0 = 0$, resulting in
\[
     \chi_1 =\frac{1}{\omega - QHQ}QH_1 P,
\]
and
\[
     V_{\mathrm{eff}}^{(1)}=\hat{Q}(\omega),
\]
i.e. the bare $\hat{Q}$-box only evaluated at a given model space energy
$\omega$. No folded diagrams are included at this stage.
The next iterative step yields
\[
     \chi_2 =Q\frac{1}{\omega - QHQ}QH_1 P-Q\frac{1}{\omega -QHQ}\chi_{1}
     V_{\mathrm{eff}}^{(2)},
\]
and
\[
    V_{\mathrm{eff}}^{(2)}=\left\{1+PH_1 
    \frac{1}{\omega -QHQ}\chi_{1}\right\}^{-1}
    \hat{Q}(\omega)=\frac{1}{1-\hat{Q}_1}\hat{Q}(\omega ).
\]
The solution to this iterative scheme is given as \cite{ls80}
\begin{equation}
     V_{\mathrm{eff}}^{(n)}=\left[1-\hat{Q}_{1}-\sum_{m=2}^{n-1}\hat{Q}_{m}
     \prod_{k=n-m+1}^{n-1}V_{\mathrm{eff}}^{(k)}\right]^{-1}\hat{Q}.
\end{equation}

It should be noted that the FD and LS expansions 
may give different results for $V_{\mathrm{eff}}$ if terminated after
a given, finite number of iterations. 
The question is which method should one prefer, the FD or the LS method?
In this work we will limit the attention to the FD method.
The reason is as follows. We observe that the convergence criterion for the FD
method states that if the FD expansion converges, it converges to those
states which have the largest $P$-space overlap, while the LS method
converges to those states which have an energy closest to the chosen 
starting energy $\omega$. With the LS method we should
therefore be able to even reproduce states not accounted for
by the actual model space. However, Ellis {\em et al.} \cite{eehho94}
showed that
the criterion for the LS method is only applicable if one is able
to determine an exact $\hat{Q}$-box, else the FD and LS methods
give similar results at low orders in the
interaction, irrespectively of
the choice of starting energy. 
At low orders in the interaction we are not able to reproduce the 
$Q$-space eigenvalue  with the LS method, even for a starting energy close to this
eigenvalue.
The FD method (if it converges) always gives a
result close to the  eigenvalue with the largest $P$-space
overlap. With increasing order of the interaction, the $\hat{Q}$-box
incorporates more $Q$-space degrees of freedom and the LS and FD methods
give unstable results. 
To see this, consider the following simple example, taken from
the work of Ellis {\em et al.} \cite{eehho94}.


In order to study the importance of intruder states, let
the hamiltonian depend linearly on a strength parameter $z$
\[
       H=H_0+zH_1,
\]
with $0\leq z\leq1$, where the limits $z=0$ and $z=1$ represent the 
non-interacting (unperturbed) and fully interacting system, respectively.
The model is an eigenvalue
problem with only two available states, which we label
$P$ and $Q$. Below we will let 
state $P$ represent the model-space
eigenvalue whereas state $Q$ represents 
the eigenvalue of the excluded space.
The unperturbed solutions to this problem are
\begin{equation}
       H_0\Phi_P =\epsilon_P\Phi_P
\end{equation}
and
\begin{equation}
       H_0\Phi_Q =\epsilon_Q\Phi_Q,
\end{equation}
with $\epsilon_P < \epsilon_Q$. We label the off-diagonal
matrix elements $X$, while $X_P=\bra{\Phi_P}H_1\ket{\Phi_P}$ and
$X_Q=\bra{\Phi_Q}H_1\ket{\Phi_Q}$.
The exact eigenvalues problem
\begin{equation}
\left(\begin{array}{cc}\epsilon_P+zX_P &zX \\
zX &\epsilon_Q+zX_Q \end{array}\right)
\end{equation}
yields
\begin{eqnarray}
     \label{eq:exact}
     E(z)=&\frac{1}{2}\left\{\epsilon_P +\epsilon_Q +zX_P
     +zX_Q \pm \left(
     \epsilon_Q -\epsilon_P +zX_Q-zX_P\right) \right. \\ \nonumber
     & \left. \times\sqrt{1+\frac{4z^2X^2}{\left(
     \epsilon_Q -\epsilon_P +zX_Q-zX_P\right)^2}}
     \right\}.
\end{eqnarray}
The authors of ref.\ \cite{eo77}
demonstrated how Brillouin-Wigner and
Rayleigh-Schr\"{o}dinger (RS) perturbation theories relate
within the framework of this simple model.
An RS expansion for the lowest
eigenstate (defining states $P$ and $Q$ as the model and excluded
spaces, respectively) can be obtained by expanding the lowest
eigenvalue as
\begin{equation}
      E=\epsilon_P +zX_P+\frac{z^2X^2}{\epsilon_P -\epsilon_Q}+
      \frac{z^3X^2(X_Q-X_P)}{(\epsilon_P -\epsilon_Q)^2}+
      \frac{z^4X^2(X_Q-X_P)^2}{(\epsilon_P -\epsilon_Q)^3}
      -\frac{z^4X^4}{(\epsilon_P -\epsilon_Q)^3}+\dots,
      \label{eq:modela}
\end{equation}
which can be viewed as an effective interaction for state $P$ in which
state $Q$ is taken into account to successive orders of the perturbation.
In this work we choose the parameters $\epsilon_P=0$, $\epsilon_Q=4$,
$X_P=-X_Q=3$ and $X=0.2$. The exact solutions
given by eq.\ (\ref{eq:exact})
are shown in fig.\ \ref{fig:model} as functions of the
strength parameter $z$. Pertinent to the choice of
parameters, is that at $z\geq 2/3$,  the lowest eigenstate is
dominated by $\Phi_Q$ while the upper is $\Phi_P$. At $z=1$ the
$\Phi_P$ mixing of the lowest eigenvalue
is $1\%$ while for $z\leq 2/3$
we have a $\Phi_P$ component of more than $90\%$.
The character of the eigenvectors has therefore been interchanged
when passing $z=2/3$. The value of the parameter $X$ represents the
strength of the coupling between the model space and the excluded space.
Thus, this simple
model allows us to study how the perturbation expansion with a
model space defined to consist of state $P$ only, behaves
as the interaction strength $z$ increases. The order-by-order
convergence in eq.\ (\ref{eq:modela}) was discussed by the authors
of ref.\ \cite{eo77}. Here we will thence only repeat
their conclusion:
For small values of $z$ one obtains
good convergence to the lower eigenvalue. For larger values of $z$ 
(i.e.\ $z> 2/3$),
increasing orders in the perturbation expansion yield a divergent
perturbation series, as expected \cite{sw72}.
However, it may be possible to rewrite
the perturbative expansion in such a way that one sums
subsets of diagrams to all orders. The hope is then that the
expansion becomes convergent for appropriate infinite partial
summations.
This is actually the philosophy
behind both the FD method and the LS method. Having defined
a set of linked and irreducible valence diagrams,
the so-called $\hat{Q}$-box\footnote{
The $\hat{Q}$-box should not be confused with the exclusion
operator $Q$.}, we can define an
iterative scheme to sum the contributions
from folded diagrams.
The $\hat{Q}$-box serves therefore as the starting point
for the iterative schemes, and within the framework of the
$2\times 2$ model one can
study the FD and LS methods for various
approximations of the $\hat{Q}$-box.
As an example, to fifth order in the parameter $z$ we have
\begin{equation}
       \hat{Q}_5=zX_P+\frac{z^2X^2}{\epsilon_P -\epsilon_Q}+
       \frac{z^3X^2X_Q}{(\epsilon_P -\epsilon_Q)^2}+
       \frac{z^4X^2X_Q^2}{(\epsilon_P -\epsilon_Q)^3}+
        \frac{z^5X^2X_Q^3}{(\epsilon_P -\epsilon_Q)^4},
       \label{eq:qapprox}
\end{equation}
and it
is easy to see that a $\hat{Q}$-box of order $l+2$ can be written
as
\[
               \hat{Q}_{l+2}=
                zX_P+\frac{z^2X^2}{\epsilon_P -\epsilon_Q-zX_Q}
                \left\{1-\left(\frac{zX_Q}
                {\epsilon_P-\epsilon_Q}\right)^{l+1}\right\},
\]
which in the limit $l\rightarrow \infty$ 
gives\footnote{$l=0$ gives a second-order $\hat{Q}$-box, $l=1$ a third-order
$\hat{Q}$-box and so forth.} 
\begin{equation}
      \hat{Q}_{\mathrm{exact}}(\epsilon_P)=
      zX_P+\frac{z^2X^2}{\epsilon_P -\epsilon_Q-zX_Q},
      \label{eq:qexact}
\end{equation}
if
\begin{equation}
     \left|\frac{zX_Q}{\epsilon_P-\epsilon_Q}\right| < 1.
     \label{eq:constr}
\end{equation}
The latter equation clearly restricts the possible 
values of $\epsilon_P$ for
given $X_Q$ and $\epsilon_Q$. Actually, 
if one lets $\epsilon_P$ vary, the present values of
$X_Q$ and $\epsilon_Q$ restrict  $\epsilon_P$ to 
$\epsilon_P \leq 1$ and $\epsilon_P\geq 7$, in order 
to have a finite $\hat{Q}$-box.
 
The exact $\hat{Q}$-box
may be used to define the first iteration of
the FD expansion as a function of the starting energy 
$\omega$ as\footnote{We replace here $\epsilon_P$ with $\omega$.}
\begin{equation}
      \lambda_1= \omega+\hat{Q}_{\mathrm{exact}}(\omega)
\end{equation}
The subsequent steps are
\begin{equation}
     \lambda_2= \lambda_1+
     {\displaystyle \sum_{m=1}^{\infty}
     \frac{1}{m!}\frac{d^m \hat{Q}}{d\omega^m}(\lambda_1
     -\omega )^m },
\end{equation}
and
\begin{equation}
    \lambda_3= \lambda_1+
     {\displaystyle \sum_{m=1}^{\infty}
     \frac{1}{m!}\frac{d^m \hat{Q}}{d\omega^m}(\lambda_2-
     \omega )^m }.
\end{equation}
In general we have
\begin{equation}
     \lambda_n= \lambda_1+
     {\displaystyle \sum_{m=1}^{\infty}
     \frac{1}{m!}\frac{d^m \hat{Q}}{d\omega^m}(\lambda_{n-1}-
     \omega )^m }.
\end{equation}
If this iteration scheme converges we have, with
$\lambda=\lambda_{n}=\lambda_{n-1}$,
\begin{equation}
   \lambda=\omega +zX_P+\frac{z^2X^2}{\lambda -\epsilon_Q-zX_Q},
   \label{eq:exact2}
\end{equation}
which is just eq.\ (\ref{eq:exact}), so that one has
obtained the true eigenvalues.
In a similar way one can use the LS expansion defined
in eq.\ (\ref{eq:ls}) to sum the folded diagrams.
\begin{table}[hbtp]
\begin{center}
\caption{The exact solutions $E_P$ ( model space) and 
$E_Q$ (excluded space) of eq.\ (111) 
as functions of the strength parameter $z$.
The results obtained with the LS and FD methods with an
exact $\hat{Q}$-box as functions
of the starting energy $\omega$ are also given.
Taken from ref.\ [52].}
\begin{tabular}{rrrrrr}
&&&&&\\\hline
\multicolumn{1}{c}{$\omega$}&
\multicolumn{1}{c}{$z$}&
\multicolumn{1}{c}{$E_P$}&
\multicolumn{1}{c}{$E_Q$}&
\multicolumn{1}{c}{$FD$}&
\multicolumn{1}{c}{$LS$}\\
\hline
0.5&0.0&0.00&4.00&0.00&0.00\\
   &0.2&0.60&3.40&0.60&0.60\\
   &0.4&1.20&2.80&1.20&1.20\\
   &0.6&1.77&2.23&1.77&1.77\\
   &0.7&2.17&1.83&2.17&1.83\\
   &0.8&2.43&1.57&2.43&1.57\\
   &1.0&3.02&0.98&3.02&0.98\\
7.5&0.0&0.00&4.00&0.00&4.00\\
   &0.2&0.60&3.40&0.60&3.40\\
   &0.4&1.20&2.80&1.20&2.80\\
   &0.6&1.77&2.23&1.77&2.23\\
   &0.7&2.17&1.83&2.17&2.17\\
   &0.8&2.43&1.57&2.43&2.43\\
   &1.0&3.02&0.98&3.02&3.02\\
   \hline 
\end{tabular}
\end{center}
\end{table}
We demonstrate the properties of the LS and FD methods with an
exact $\hat{Q}$-box in table 3 for two starting energies,
$0.5$ and $7.5$ (arbitrary units). Clearly, we see that with a starting 
energy $0.5$ the LS method yields the $Q$-space eigenvalue at $z\geq 2/3$.
Below $z=2/3$, the LS method reproduces the $P$-space eigenvalue. 
With a starting energy of $7.5$, the LS method 
gives the $Q$-space eigenvalue
for $z\leq 2/3$, whereas the $P$-space eigenvalue is reproduced for
$z\geq 2/3$. Thus, this simple model demonstrates nicely the properties of
the LS scheme, i.e.\ it converges to those eigenvalues which are
closest to the chosen starting energy, irrespectively of the structure of
the wave function. From table 3 we also see that the FD
method always reproduces the $P$-space eigenvalue. 
At $z=2/3$, the FD
scheme does not stabilize, the eigenvalue fluctuates and there is
no convergence. More precisely this means that we can not
go from the lower to the upper
eigenvalue along the real axis $z$ \cite{eo77}.
The fluctuation is intimately connected to the convergence
criterion for the FD scheme. If one of the solutions contains
more than $50\%$ valence state intensity, the FD method converges
to that solution. At $z=2/3$, equal admixtures of $\Phi_P$
and $\Phi_Q$ occur in the true wave functions.


This simple example, {\em starting with the
exact $\hat{Q}$-box},  serves to demonstrate significant
differences in convergence behavior of the two methods. Of
importance is the fact that with the LS scheme
one is  able to reproduce a
$Q$-space state insofar one defines a starting energy which is
close to the actual $Q$-space state.
However, in actual nuclear structure calculations, we are not
able to define an exact $\hat{Q}$-box. The question the authors of ref.\
\cite{eehho94} addressed 
was if an
approximation of the $\hat{Q}$-box still gives the same
difference between the LS and FD methods at $z=1$.
To shed light on this, we redisplay their discussion  in 
fig.\ \ref{fig:model} for results obtained with various low-order
approximations to the $\hat{Q}$-box, recall eq.\ (\ref{eq:qapprox}).
The starting energy for the LS and 
FD calculations was set equal $0.5$, since
one is interested in seeing if the convergence criterion for the
LS method holds for an approximate $\hat{Q}$-box. 
Clearly, with a $\hat{Q}$-box of second order
in the interaction, the
difference\footnote{It ought be emphasized that all results
with either the FD or the LS method represent converged
eigenvalues, which means that contributions from folded diagrams
to high order are included.}
between the LS and the FD method is negligible. 
A second-order $\hat{Q}$-box corresponds to setting $X_Q=0$, which
means that in the range of {\em physically interesting} 
$z$ values ($0\leq z \leq 1$)
there is no intruder state, yielding an approximately straight line
for the LS method.
Up to fourth order in the interaction, both 
the LS and FD methods yield almost 
the same value. With a fifth-order
$\hat{Q}$-box the FD method becomes unstable at $z=1$.
This can 
be inferred from the structure of the FD expansion, 
which with a fifth-order $\hat{Q}$-box
reads
\[
     \lambda_n=\epsilon_P+zX_P+\frac{z^2X^2}{\lambda_{n-1} -\epsilon_Q}+
      \frac{z^3X^2X_Q}{(\lambda_{n-1} -\epsilon_Q)^2}+
      \frac{z^4X^2X_Q^2}{(\lambda_{n-1} -\epsilon_Q)^3}+
     \frac{z^5X^2X_Q^3}{(\lambda_{n-1} -\epsilon_Q)^4}.
\] 
The first iteration is just $\lambda_1=\epsilon_P+\hat{Q}$, 
which gives a result close to the model space eigenvalue
$\approx 3$ at $z=1$. Inserting $\lambda_1$ and higher iterations
in the above expansion
results in an
FD expansion which for each iteration will fluctuate
between a large value $\lambda$ and $\epsilon_P+X_P$. 
The large eigenvalue stems
from the fact that the energy denominators in the higher-order terms become
small. With $X=0.2$ this oscillating behavior sets in already at fifth 
order in the interaction, whereas 
if one chooses the coupling
between the model space and the excluded space to be $X=0.01$, 
this divergence appears first
with a $\hat{Q}$-box of tenth order. However, 
one will ultimately end up with a series which fluctuates,
except for the trivial case $X=0$. 
With the given parameters and a $\hat{Q}$-box of tenth or higher 
order in the interaction,
the FD method converges only if $z<2/3$. 
If the FD method converges, Ellis {\em et al.} \cite{eehho94}
obtained result close to the model-space eigenvalue,
and, if the coupling between the model space and 
the excluded space is weak,
low-order perturbation theory was shown to work rather well.
\begin{figure}[hbtp]
      \setlength{\unitlength}{1mm}
      \begin{picture}(140,150)
      \put(25,10){\epsfxsize=12cm \epsfbox{fig14.eps}}
      \end{picture}
       \caption{
        Exact solutions (solid line) and the results 
        obtained with  the FD and LS methods 
        with various approximations to the
        $\hat{Q}$-box. FD-2nd and LS-2nd (dashed line) 
        indicate that the FD and LS methods 
        were used with a second-order $\hat{Q}$-box, 
        while FD-4th and LS-4th are the results obtained
        with a fourth-order $\hat{Q}$-box. Similarly, LS-10th and LS-50th
        are the LS results with a tenth- and fiftieth-order
        $\hat{Q}$-box, respectively. 
        A starting energy $0.5$ was chosen in all calculations. Taken 
        from ref.\ [52].}
        \label{fig:model}
\end{figure}
For the LS method with a low-order $\hat{Q}$-box one is
not able to reproduce the lowest eigenstate, even with an adequately
chosen starting energy, as stated by the convergence criterion
of the LS method.  Ellis {\em et al.} had actually to go as far as to
a $\hat{Q}$-box of fiftieth order before they got a result close 
to the lowest eigenvalue. However, if the order of the $\hat{Q}$-box
is further increased ($\approx 150$ with the above parameters), even 
the LS method diverges.
Thus, in summary, the general convergence properties of
the LS method discussed above
are demonstrated only for an exact $\hat{Q}$-box.
If the $\hat{Q}$-box is approximated to a given order in the
interaction, the convergence criterion cannot be applied.


In realistic calculations, like those discussed by Ellis {\em et al.} 
for $^{18}$O,
one can only evaluate some low-order terms in the perturbation
expansion, 
and although the FD and LS methods
give negligible differences in the
realistic calculations,
the FD
method is the preferable one 
in nuclear structure calculations since
it converges to those eigenstates which have the largest $P$-space
overlap, a desirable property since we in general can only employ a
rather small shell-model space. We will come back to this in
section 6.









\subsection{Time-dependent perturbation theory}

In this work we will primarily be interested in valence-linked
perturbative expansions for the effective interaction. Such expansions
can be conveniently expressed within the framework of time-dependent
perturbation theory. In this section we will review some of the
properties of the time-development operator $U(t,t')$ and its
connection to the decomposition theorem, which yields a
valence-linked expression for the actual operator. However, some
of the divergencies which occur can only be handled by the
introduction of the so-called folded diagrams, to be
discussed below. The folded diagrams
arise due to the removal of the dependence on the exact
energy of the perturbative expansion. This is the price one has
to pay when introducing the Rayleigh-Schr\"{o}dinger expansion.

The time-development operator $U$ has the
properties that
\begin{equation}
     U^{\dagger}(t,t')U(t,t')=U(t,t')U^{\dagger}(t,t')=1,
\end{equation}
which implies that $U$ is unitary
\begin{equation}
     U^{\dagger}(t,t')=U^{-1}(t,t').
\end{equation}
Further,
\begin{equation}
    U(t,t')U(t't'')=U(t,t'')
\end{equation}
and
\begin{equation}
    U(t,t')U(t',t)=1,
\end{equation}
which leads to
\begin{equation}
    U(t,t')=U^{\dagger}(t',t).
\end{equation}
We can then construct the true eigenvectors $\Psi_{\alpha ,\beta}$, 
where $\alpha$ and
$\beta$ indicate the initial and final states, respectively, in
terms of the unperturbed wave functions $\psi_{\alpha,\beta}$  
through the
action of the time-development operator $U$.
In the present discussion of the time-dependent theory we will make
use of the so-called complex-time approach to describe the time
evolution operator $U$ \cite{ko90}\footnote{The time-development 
operator used here is determined by use
of the interaction picture. The suffix $I$, which is commonly
used in the literature to distinguish $U_I$ in the interaction
picture from $U$ in the
Schr\"{o}dinger or Heisenberg pictures, has been omitted.}.
This means that we
allow the time $t$ to be rotated by a small angle $\epsilon$
relative to the real time axis. The complex time $t$ is then
related to the real time $\tilde{t}$ by
\begin{equation}
t=\tilde{t}(1-i\epsilon ).
\end{equation}
Let us first study the true eigenvector $\Psi_{\alpha}$ which evolves
from the unperturbed eigenvectors $\psi_{\alpha}$ through the action of the
time development operator
\begin{eqnarray}
   U(t,t')&=\lim_{\epsilon \rightarrow 0}
   \lim_{t'\rightarrow -\infty (1-i\epsilon )}
   {\displaystyle\sum_{n=0}^{\infty}\frac{(-i)^n}{n!}
   \int_{t'}^{t}dt_1  \int_{t'}^{t}dt_2\dots  \int_{t'}^{t}dt_n}
		\\ \nonumber
	     &  \times T\left[H_1(t_1)H_1(t_2)\dots H_1(t_n)\right],
	     \label{eq:timeu}
\end{eqnarray}
where $T$ stands for the correct time-ordering \cite{no88,fw71}.

In time-dependent
perturbation theory we let $\Psi_{\alpha}$ develop from $\psi_{\alpha}$ in the
remote past to a given time $t$
\begin{equation}
    \frac{\ket{\Psi_{\alpha}}}
    {\left\langle\psi_{\alpha} | \Psi_{\alpha} \right\rangle}=
    \lim_{\epsilon \rightarrow 0}
   \lim_{t'\rightarrow -\infty (1-i\epsilon )}
   \frac{U(t,t' )\ket{\psi_{\alpha}} }
   { \bra{\psi_{\alpha}} U(t,t' )\ket{\psi_{\alpha}} },
   \label{eq:psii}
\end{equation}
and similarly, we let
$\Psi_{\beta}$ develop from $\psi_{\beta}$ in the remote future
\begin{equation}
    \frac{\bra{\Psi_{\beta}}}{\left\langle
    \psi_{\beta} | \Psi_{\beta} \right\rangle}=
    \lim_{\epsilon \rightarrow 0}
    \lim_{t'\rightarrow \infty (1-i\epsilon )}
    \frac{\bra{\psi_{\beta}}U(t' ,t) }
    { \bra{\psi_{\beta}} U(t' ,t)\ket{\psi_{\beta}} }.
    \label{eq:psif}
\end{equation}

Here we are interested in the expectation value of a given
operator ${\cal O}$ acting at a time $t=0$. This can be achieved
from eqs.\ (\ref{eq:psii}) and (\ref{eq:psif}) defining
\begin{equation}
     \ket{\Psi_{\alpha ,\beta}'}=
     \frac{\ket{\Psi_{\alpha ,\beta}}}
     {\left\langle\psi_{\alpha ,\beta} | \Psi_{\alpha ,\beta} \right\rangle}
\end{equation}
we have
\begin{equation}
   {\cal O}_{\alpha\beta}
  =\frac{N_{\beta\alpha}}{D_{\beta}D_{\alpha}},
   \label{eq:expect}
\end{equation}
where we have introduced
\begin{equation}
   N_{\beta\alpha}=
   \bra{\psi_{\beta}}U(\infty ,0){\cal O}U(0,-\infty )\ket{\psi_{\alpha}} ,
\end{equation}
and 
\begin{equation}
   D_{\alpha ,\beta}=
   \sqrt{\bra{\psi_{\alpha ,\beta}}
   U(\infty ,0)U(0,-\infty )\ket{\psi_{\alpha ,\beta}}}. 
\end{equation}
If the operator ${\cal O}$ stands for the hamiltonian $H$ we obtain
\begin{equation}
    {\displaystyle  \frac{\bra{\Psi_{\lambda}'}H\ket{\Psi_{\lambda}'} }
   { \left\langle\Psi_{\lambda}' | \Psi_{\lambda}' \right\rangle} }
   \label{eq:hexpect}.
\end{equation}

At this stage, {\em it is important to observe} that our 
expression for the expectation value of a given operator ${\cal O}$
{\em is hermitian} insofar ${\cal O}^{\dagger}={\cal O}$. This is readily 
demonstrated. Eq.\ (\ref{eq:expect}) is of the general form
\begin{equation}
U(t,t_0){\cal O}U(t_0,-t),
\end{equation}
and noting that 
\begin{equation}
   U^{\dagger}(t,t_0)=
   \left({\displaystyle e^{iH_0t}e^{-iH(t-t_0)}e^{-iH_0t}}\right)^{\dagger}
   =U(t_0,-t),
\end{equation}
since $H^{\dagger}=H$ and $H_0^{\dagger}=H_0$, we have that
\begin{equation}
    \left(U(t,t_0){\cal O}U(t_0,-t)\right)^{\dagger}
    =U(t,t_0){\cal O}U(t_0,-t).
\end{equation}
In this form, the expression for the effective interaction differs
from that given by either the folded-diagram  method of eq.\ (\ref{eq:fd}) 
or the
Lee-Suzuki method of eq.\ (\ref{eq:ls}). To define these perturbation
expansions we employed the similarity transformation given by eq.\ 
(\ref{eq:htrans})
\begin{equation}
   \Omega^{-1}H\Omega = e^{-\chi}He^{\chi},
\end{equation}
where
\begin{equation}
     \Omega =1+\chi = 1+Q\chi P = e^{\chi}.
\end{equation}
The important point to note here is that $e^{\chi}$ does not in general
fulfill the condition $\chi^{\dagger}=-\chi$, which in turn leads to
non-hermitian expansions for the effective interaction defined by 
the folded-diagram method or the Lee-Suzuki expansion. 
This non-hermiticity is however weak if we define a model space entirely
within the same harmonic oscillator shell, as we will see 
in our calculations of the effective interaction in section 6. 
If one wishes to obtain an effective
interaction which spans over several shells the non-hermiticity may no longer 
be weak. How to obtain a hermitian effective interaction in these 
cases will be discussed at the end of section 6. 
In our derivation below, we will not use eq.\ (\ref{eq:hexpect}), but act 
with the unperturbed wave function to the left of $H_1$. This will in turn
lead to the familiar Goldstone perturbation expansion or linked
valence expansion in our case. 


Note that the operator ${\cal O}$ may be of one-body, two-body or
of a more complicated structure. Similarly, the wave functions
$\psi_{\alpha}$ may represent the degrees of freedom of one, two or
more particles. As an example,
an unperturbed model-space two-body wave function can be written
as
\begin{equation}
   \ket{\psi_{\alpha}}=\sum_{\lambda\gamma=1}^{D}
    C_{\lambda\gamma}^{\alpha} a_{\lambda}^{\dagger}a_{\gamma}^{\dagger}
	       \ket{\tilde{c}},
    \label{eq:twowf}
\end{equation}
with $C_{\lambda\gamma}^{\alpha}$ 
a coupling coefficient and $\ket{\tilde{c}}$ referring
to some selected vacuum state, e.g.\ the Slater determinant of
the $^{16}$O core if we are
studying systems with mass $A=18$.
Eq.\ (\ref{eq:expect}) is our basic matrix element for an
operator ${\cal O}$, though in its present form it is not suitable
for computation, since both the numerator and denominator contain
divergencies. These divergencies have to be removed in order to
obtain a meaningful expression for ${\cal O}$. The final expression
we are seeking, will be a valence-linked expansion where
every term in the expansion for ${\cal O}$ is finite.
To see how such divergencies arise, consider the contribution
to the exact wave function shown in fig.\ \ref{fig:wavefex}.
\begin{figure}[hbtp]
      \setlength{\unitlength}{1mm}
      \begin{picture}(80,80)
      \put(30,10){\epsfxsize=8cm \epsfbox{wavefex.eps}}
      \end{picture}
\caption{A contribution to the exact wave function. The time orderings
are $0>t_1>t_2>-\infty$. The railed line belongs to the $Q$-space, while
the other represent valence space states. }
\label{fig:wavefex}
\end{figure}
The exact wave function is proportional to
\[
  U(0,-\infty )a_{\alpha}^{\dagger}a_{\beta}^{\dagger}
	       \ket{\tilde{c}},
\]
and the diagram shown in the above figure is an example
of one possible contribution. It has two interaction terms
acting successively at times $t_2$ and $t_1$. At $t_2$, the
valence particles are scattered into the valence states
$\gamma$ and $\delta$, and at $t_1$ they are scattered into
the passive particle states $i$ and $j$. It is then possible to show
that this diagram is given by \cite{ko90}
\begin{equation}
     a_{i}^{\dagger}a_{j}^{\dagger}
	       \ket{\tilde{c}}\times\frac{1}{4}\tilde{V}_{ij\gamma\delta}
               \tilde{V}_{\gamma\delta\alpha\beta}\times I,
\end{equation}
where we have used antisymmetrized interaction matrix elements
and 
\begin{equation}
  I=
  \lim_{\epsilon \rightarrow 0}
   \lim_{t'\rightarrow -\infty (1-i\epsilon )}
   {\displaystyle (-i)^2
   \int_{t'}^{0}dt_1  \int_{t'}^{t_1}dt_2}
   e^{-i(\varepsilon_{\gamma}+\varepsilon_{\delta}-\varepsilon_{i}
   -\varepsilon_{j})t_1}
      e^{-i(\varepsilon_{\alpha}+\varepsilon_{\beta}-\varepsilon_{\gamma}
   -\varepsilon_{\delta})t_2}.
\end{equation}
As we will employ a degenerate or nearly degenerate model
space\footnote{Even if we use sp energies from say a
Brueckner-Hartree-Fock calculation, the model-space sp energies may be nearly
degenerate if we have a model space which consists of the sp
orbits from one shell only.}, we have
$\varepsilon_{\alpha}+\varepsilon_{\beta}-\varepsilon_{\gamma}
-\varepsilon_{\delta}\approx 0$, and the above integral
diverges. In order to handle such divergencies, we will
in the procedure below factorize out such divergent terms in
$U(0,-\infty )a_{\alpha}^{\dagger}a_{\beta}^{\dagger}
\ket{\tilde{c}}$. We can then rewrite the diagram in
fig.\ \ref{fig:wavefex} as in fig.\ \ref{fig:wavefex2},
\begin{figure}[hbtp]
      \setlength{\unitlength}{1mm}
      \begin{picture}(140,70)
      \put(25,10){\epsfxsize=12cm \epsfbox{wavefex2.eps}}
      \end{picture}
\caption{The divergent diagram in (i) can be factorized
in two pieces, (ii) and (iii), with time orderings for (i)
$0>t_1>t_2>-\infty$, for (ii) $0>t_1>-\infty$ and $0>t_2>-\infty$ 
and for (iii) $0>t_2>t_1>-\infty$.}
\label{fig:wavefex2}
\end{figure}
where diagram (ii) represents a factorization of diagram (i)
in two independet pieces and reads
\begin{equation}
     a_{i}^{\dagger}a_{j}^{\dagger}
	       \ket{\tilde{c}}\times\frac{1}{4}\tilde{V}_{ij\gamma\delta}
               \tilde{V}_{\gamma\delta\alpha\beta}\times I(ii),
\end{equation}
with
\begin{equation}
  I(ii)=
  \lim_{\epsilon \rightarrow 0}
   \lim_{t'\rightarrow -\infty (1-i\epsilon )}
   {\displaystyle (-i)^2
   \int_{t'}^{0}dt_1  \int_{t'}^{0}dt_2}
   e^{-i(\varepsilon_{\gamma}+\varepsilon_{\delta}-\varepsilon_{i}
   -\varepsilon_{j})t_1}
      e^{-i(\varepsilon_{\alpha}+\varepsilon_{\beta}-\varepsilon_{\gamma}
   -\varepsilon_{\delta})t_2},
\end{equation}
which is also divergent.  Diagram (iii) is given as
\begin{equation}
   (iii)=(ii)-(i)=
    -a_{i}^{\dagger}a_{j}^{\dagger}
	       \ket{\tilde{c}}\times\frac{1}{4}\tilde{V}_{ij\gamma\delta}
               \tilde{V}_{\gamma\delta\alpha\beta}\times I(iii),
\end{equation}
and
\begin{equation}
  I(iii)=
  \lim_{\epsilon \rightarrow 0}
   \lim_{t'\rightarrow -\infty (1-i\epsilon )}
   {\displaystyle (-i)^2
   \int_{t'}^{0}dt_1  \int_{t_1}^{0}dt_2}
   e^{-i(\varepsilon_{\gamma}+\varepsilon_{\delta}-\varepsilon_{i}
   -\varepsilon_{j})t_1}
      e^{-i(\varepsilon_{\alpha}+\varepsilon_{\beta}-\varepsilon_{\gamma}
   -\varepsilon_{\delta})t_2}.
\end{equation}
This integral is finite even though $I(i)$ and $I(ii)$ are
infinite. To evaluate the last integral, we note first
that
\begin{equation}
   \int_{t'}^{0}dt_1  \int_{t_1}^{0}dt_2
   =   \int_{t'}^{0}dt_2  \int_{t'}^{t_2}dt_1.
\end{equation}
Substituting the above in the integral for $I(iii)$ we obtain
\begin{equation}
  I(iii)=\frac{1}{(\varepsilon_{\gamma}+\varepsilon_{\delta}-
                   \varepsilon_{i}-\varepsilon_{j})}
          \frac{1}{(\varepsilon_{\alpha}+\varepsilon_{\beta}-
                   \varepsilon_{i}-\varepsilon_{j})}.
\end{equation}
We then obtain for diagram (iii)
\begin{equation}
   (iii)=
    -a_{i}^{\dagger}a_{j}^{\dagger}
	       \ket{\tilde{c}}\times\frac{1}{4}\tilde{V}_{ij\gamma\delta}
               \tilde{V}_{\gamma\delta\alpha\beta}
             \frac{1}{(\varepsilon_{\gamma}+\varepsilon_{\delta}-
                   \varepsilon_{i}-\varepsilon_{j})}
          \frac{1}{(\varepsilon_{\alpha}+\varepsilon_{\beta}-
                   \varepsilon_{i}-\varepsilon_{j})}.
\end{equation}
This expression can be further simplified if the model space
is degenerate. If we set
$\varepsilon_{\gamma}+\varepsilon_{\delta}=
\varepsilon_{\alpha}+\varepsilon_{\beta}=\omega$, we have
\begin{equation}
   (iii)=
    -a_{i}^{\dagger}a_{j}^{\dagger}
	       \ket{\tilde{c}}\times\frac{1}{4}\tilde{V}_{ij\gamma\delta}
               \tilde{V}_{\gamma\delta\alpha\beta}
             \frac{1}{(\omega -
                   \varepsilon_{i}-\varepsilon_{j})^2},
         \label{eq:wavefex6}
\end{equation}
or
\begin{equation}
   (iii)=
    a_{i}^{\dagger}a_{j}^{\dagger}
	       \ket{\tilde{c}}\times\frac{1}{4}
               \tilde{V}_{ij\gamma\delta}
              \frac{d}{d\omega}\left(
             \frac{1}{(\omega -
                   \varepsilon_{i}-\varepsilon_{j})}\right)
               \tilde{V}_{\gamma\delta\alpha\beta} ,
\end{equation}
an observation we will utilize below in the derivation of the
effective interaction. Diagram (iii) is an example of a folded
diagram. The folded lines refer to the intermediate
particle states $\gamma$ and $\delta$, which belong to the model
space. Conventionally, they are drawn by way of downgoing
lines with a circle to distinguish them from hole lines.
The above diagrams are just mere examples of infinitely
many diagrams which arise in the evaluation of the wave function.
Further examples can be found in the work of ref.\ \cite{ko90}.
Here we will discuss how one in general can remove these
unwanted divergencies in an expedient and computationally
simple manner. Note that diagram (ii) may give rise to
a valence disconnected diagram to the
effective interaction. This can be understood if we let
the interaction $H_1$ act on the leftmost part of diagram
(ii) at $t=0$, as shown in fig.\ \ref{fig:wavefex3}.
\begin{figure}[hbtp]
      \setlength{\unitlength}{1mm}
      \begin{picture}(120,80)
      \put(25,10){\epsfxsize=12cm \epsfbox{wavefex3.eps}}
      \end{picture}
\caption{The divergent diagram in (ii) of the previous figure
may yield a valence disconnected diagram if we let the
interaction $H_1$ act at $t=0$ as shown.}
\label{fig:wavefex3}
\end{figure}
These valence disconnected diagrams are then cancelled exactly
to all orders in the folded-diagram expansion. This cancellation
in the folded-diagram series\footnote{There are also other types
of disconnected  diagrams which will be cancelled in the expansion
delineated below.}
corresponds to the familiar cancellation of unlinked diagrams
in the Goldstone expansion.
In order to discuss the removal of
these terms, we will use the
definition of the time-development operator in eq.\
(\ref{eq:timeu}), such that the numerator $N_{\alpha\beta}$ and the
denominator $D_{\alpha,\beta}$ can be expressed as collections
of time-ordered Goldstone diagrams.  The aim of the development here is
to factorize
\begin{equation}
\lim_{\epsilon \rightarrow 0}
   \lim_{t'\rightarrow -\infty (1-i\epsilon )}
\frac{U(t,t' )\ket{\psi_{\alpha}} }
{ \bra{\psi_{\alpha}} U(t,t' )\ket{\psi_{\alpha}} },
\end{equation}
as a whole. To achieve this, we will make use of the decomposition
theorem\footnote{In order to avoid unnecessary overlap
with the existing literature, only the essential results
behind the decomposition theorem will be given here. For a more detailed
account, we refer the reader to ref.\ \cite{ko90}.}
exposed in e.g.\ ref.\ \cite{ko90}. Let us illustrate the 
decomposition theorem by considering an example to the two-body
wave function of eq.\ (\ref{eq:twowf})
\begin{equation}
               U(0,-\infty )\ket{\psi_{\alpha}} =
               U(0,-\infty )a_{\lambda}^{\dagger}
               a_{\gamma}^{\dagger}
	       \ket{\tilde{c}},
\label{eq:exam1}
\end{equation}
where $\lambda$ and $\gamma$ are valence space sp states.
Examples of diagrams which contribute to eq.\ (\ref{eq:exam1}) are 
shown in fig.\ \ref{fig:waveex}.
\begin{figure}[hbtp]
      \setlength{\unitlength}{1mm}
      \begin{picture}(140,120)
      \put(25,10){\epsfxsize=12cm \epsfbox{waveex.eps}}
      \end{picture}
\caption{Diagrams which contribute to
$U(0,-\infty )\ket{\psi_{\alpha}} $. The upper part shows diagrams, (i) and (ii),
linked to at
least one valence line, while the lower part exhibits examples
of diagrams where no vertices are connected to valence lines.
Diagrams (iii) is a contribution to the true ground-state
wave function of the core system while diagram (iv) is a vacuum
fluctuation diagram.}
\label{fig:waveex}
\end{figure}
The philosophy behind the decomposition theorem is that eq.\ (\ref{eq:exam1})
can be factorized into a group of diagrams
which consists of a given number of 
interactions linked to one or both the valence lines $\lambda$ and $\gamma$ and 
a group of diagrams with no vertices linked to any of the
valence states. Examples of the latter case are shown in the lower
part of fig.\ \ref{fig:waveex}, while diagrams contributing to the 
former are shown in the upper part. Thus, 
as a first step, one can rewrite eq.\ (\ref{eq:exam1}) as
\begin{equation}
     U(0,-\infty )a_{\lambda}^{\dagger}a_{\gamma}^{\dagger}\ket{\tilde{c}}=
     U_V(0,-\infty ) a_{\lambda}^{\dagger}a_{\gamma}^{\dagger}
     \ket{\tilde{c}}\times
     U(0,-\infty )\ket{\tilde{c}}.
\label{eq:exam2}
\end{equation}
The subscript $V$ refers to the fact that we have at least one valence line
present.  Then, we can in turn factorize each term on the rhs.\ of
eq.\ (\ref{eq:exam2}). The term $U(0,-\infty )\ket{\tilde{c}}$ factorizes
as
\begin{equation}
     U(0,-\infty )\ket{\tilde{c}}=U_Q(0,-\infty )\ket{\tilde{c}}
     \bra{\tilde{c}}U(0,-\infty )\ket{\tilde{c}},
     \label{eq:corewf}
\end{equation}
where $\bra{\tilde{c}}U(0,-\infty )\ket{\tilde{c}}$ represents all vacuum
fluctuations like diagram (iv) of fig.\ \ref{fig:waveex}, whereas the first
term is the 
sum of all contributions to the true ground-state wave function of the 
core system \cite{ko90}, 
examplified by diagram (iii) of fig.\ \ref{fig:waveex}. The subscript $Q$ refers
to the fact that all states belong to the $Q$-space at $t=0$.

A similar factorization can be applied to the $U_V$ term. 
This term contains diagrams which begin and end in a model-space state
$P$ or which begin with a model-space state and end in a passive state
at $t=0$, i.e.\
\begin{equation}
    U_V(0,-\infty ) \ket{\psi_{\alpha}}=
    \ket{\chi_{\alpha}^P}+\ket{\chi_{\alpha}^Q}.
\end{equation}
The diagrams
contained in these two terms can conveniently be expressed in terms of chains
of so-called $\hat{Q}$-boxes. The $\hat{Q}$-box is defined as the sum
of all irreducible\footnote{At least one passive state between two interactions
$H_1$.} and 
valence-linked diagrams with at least one $H_1$ vertex. This \qbox is defined
for incoming and outgoing model-space states. Examples of diagrams
which contribute are given in fig.\ \ref{fig:pqbox}. 
\begin{figure}[hbtp]
      \setlength{\unitlength}{1mm}
      \begin{picture}(140,80)
      \put(25,10){\epsfxsize=12cm \epsfbox{pqbox.eps}}
      \end{picture}
\caption{Diagrams which contribute to the \qbox with only model-space
states as final and initial states.}
\label{fig:pqbox}
\end{figure}
However, in order
to obtain the $\ket{\chi_i^Q}$ term, we need a \qbox defined for
incoming model-space states and outgoing $Q$-space states. Typical
examples of topologies which arise in the determination of the
latter are shown in fig.\ \ref{fig:qqbox}.
\begin{figure}[hbtp]
      \setlength{\unitlength}{1mm}
      \begin{picture}(140,90)
      \put(25,10){\epsfxsize=12cm \epsfbox{qqbox.eps}}
      \end{picture}
\caption{Diagrams which contribute to the \qbox with initial model-space
states and final $Q$-space  states. }
\label{fig:qqbox}
\end{figure}
The structures of $\ket{\chi_{\alpha}^P}$ and $\ket{\chi_{\alpha}^Q}$ 
is shown in the 
upper and lower parts of fig.\ \ref{fig:chipq}, respectively.
\begin{figure}[hbtp]
      \setlength{\unitlength}{1mm}
      \begin{picture}(140,140)
      \put(25,10){\epsfxsize=12cm \epsfbox{chipq.eps}}
      \end{picture}
\caption{The upper part shows a schematic representation
of $\ket{\chi_{\alpha}^P}$, where the time $t=0$ is set at the top of the diagram.
For contributions to an operator, there will always be an interaction
or operator at $t=0$. The lower part of this figure is a sketch of the
structure of $\ket{\chi_{\alpha}^Q}$. The circles are shorthands for the \qbox .}
\label{fig:chipq}
\end{figure}
As discussed in the previous subsection in connection with eq.\ (\ref{eq:wavef}),
certain terms of $\ket{\chi_{\alpha}^Q}$ give rise to divergencies. These
divergencies can however be factorized out through the folding operation
\cite{ko90}, such that a term proportional to $\ket{\chi_{\alpha}^P}$ can be 
extracted from $\ket{\chi_{\alpha}^Q}$. The final expression for $U_V$ then reads 
\begin{equation}
     U_V(0,-\infty ) \ket{\psi_{\alpha}} ={\displaystyle \sum_{\beta=1}^{D}
     U_{VQ}(0,-\infty )\ket{\psi_{\beta}}\bra{\psi_{\beta}}
     U_V(0,-\infty )\ket{\psi_{\alpha}}}.
     \label{eq:uv}
\end{equation}
The form of $U_{VQ}$ is shown in fig.\ \ref{fig:uvq}, where the integration
sign represents the folding operation. The first term vanishes 
if $\beta\neq\alpha$
while the remaining terms end in passive states at $t=0$.
\begin{figure}[hbtp]
      \setlength{\unitlength}{1mm}
      \begin{picture}(140,70)
      \put(25,10){\epsfxsize=12cm \epsfbox{uvq.eps}}
      \end{picture}
\caption{Schematic structure of $U_{VQ}$. Note, that, except for the
first term, all other contributions end in a $Q$-space state.}
\label{fig:uvq}
\end{figure}
Correspondingly, the structure of
$\bra{\psi_{\beta}}U_V(0,-\infty )\ket{\psi_{\alpha}}$ is displayed in fig.\ 
\ref{fig:uvv}.
\begin{figure}[hbtp]
      \setlength{\unitlength}{1mm}
      \begin{picture}(140,90)
      \put(25,10){\epsfxsize=12cm \epsfbox{uvv.eps}}
      \end{picture}
\caption{Schematic structure of
$\bra{\psi_{\beta}}U_V(0,-\infty )\ket{\psi_{\alpha}}$.}
\label{fig:uvv}
\end{figure}

Thus, collecting eqs.\ (\ref{eq:corewf}) and (\ref{eq:uv}) we arrive at
the following form of eq.\ (\ref{eq:exam1})
\begin{equation}
     U(0,-\infty )\ket{\psi_{\alpha}} =
     U_Q(0,-\infty )\ket{\tilde{c}}\bra{\tilde{c}}
     U(0,-\infty )\ket{\tilde{c}}
     {\displaystyle \sum_{\beta=1}^{D}
     U_{VQ}(0,-\infty )\ket{\psi_{\beta}}\bra{\psi_{\beta}}
     U_V(0,-\infty )\ket{\psi_{\alpha}}}
\label{eq:decomp}
\end{equation}
The decomposition theorem applies equally well to eq.\ (\ref{eq:psif}).


Starting with the decomposition theorem as given by 
eq.\ (\ref{eq:decomp}), we will here
detail the derivation of a valence-linked expression
for the effective interaction appropriate for finite nuclei. 



Let now the operator ${\cal O}$ correspond to the
familiar two-body hamiltonian $H=H_0 +H_1$. The unperturbed
part $H_0$ is
\begin{equation}
       H_0(t) ={\displaystyle \sum_{\alpha\beta}\bra{\alpha}(t+u)\ket{\beta}
		       a_{\alpha}^{\dagger}(t)a_{\beta}(t)},
    \label{eq:unpert}
\end{equation}
with $t$ and $u$ the sp kinetic energy and auxiliary potential, 
respectively. The eigenfunctions of $H_0$ are the unperturbed
wave 
functions given by eq.\ (\ref{eq:twowf}).
The eigenvalues
correspond to the sum of the unperturbed sp energies. The interaction
$H_1$ is given by a two-body term
\begin{equation}
H_1(t) = {\displaystyle \frac{1}{2}
	   \sum_{\alpha\beta\gamma\delta}\bra{\alpha\beta}H_1
	   \ket{\gamma\delta}a_{\alpha}^{\dagger}(t)
	   a_{\beta}^{\dagger}(t)
	   a_{\delta}(t)a_{\gamma}(t)}.
     \label{eq:inter}
\end{equation}
Here we let the indices $\alpha\beta\gamma\delta$ run over both valence-
and $Q$-space sp states.

The effective interaction we will derive should have the following
properties:
\begin{itemize}
\item The original eigenvalue problem given by eq.\ (\ref{eq:schro})
 reduces to a model-space eigenvalue problem
\begin{equation}
    PH_{\mathrm{eff}}P\Psi_{\alpha} = E_{\alpha} P\Psi_{\alpha}.
    \label{eq:mspacee1}
\end{equation}
 The main purpose is to derive an effective interaction from the 
  original hamiltonian. Moreover, we wish our final model-space
  eigenvalue problem to reproduce the empirical shell-model
  secular equation. More explicitly, we expect that eq.\ (\ref{eq:mspacee1})
  can be separated into a valence-space part and a core contribution
\begin{equation}
    PH_{\mathrm{eff}}'P\Psi_{\alpha} = 
    \left( E_{\alpha} -E_C\right) P\Psi_{\alpha},
    \label{eq:mspacee}
\end{equation}
  where $E_C$ is the true energy of the core. If one considers a nucleus
 like $^{18}$O, $E_C$ corresponds then to the energy of $^{16}$O.
 In the empirical shell model we have
\begin{equation}
  PH_{\mathrm{eff}}'P = PH_0'P + PH_1'P,
\end{equation}
 where the sp energies of $H_0'$ in the last equation are taken
 to be thedifference in binding energy between a state in the appropriate nucleus
 with one nucleon in addition to closed shells and the ground state of
the corresponding closed-shell nucleus. Following this prescription,
 $PH_1'P$ should contain two-body interactions only. Note well that 
 $H_1'$ is different from $H_1$ since the latter includes both one-
and two-body terms. Similarly, $H_0'$ is normally approximated with the
experimental sp energies.
\item
  The model-space eq.\ (\ref{eq:mspacee}) is supposed to give
  $D$ solutions, $D$ being the dimension of the model space. We need therefore
  a scheme which gives us a one-to-one correspondence between $D$ parent
states $\ket{\rho_{\lambda}}$ in the model space and the true eigenstates 
$\ket{\Psi_{\lambda}}$. 
We define the parent states as the projections of the true eigenvectors
onto the model space. Further, we assume that they are linearly independent
and expand them in terms of the model-space basis states $\ket{\psi_{\alpha}}$
\begin{equation}
    \ket{\rho_{\lambda}}=\sum_{\alpha=1}^{D}
    C_{\alpha}^{(\lambda)}\ket{\psi_{\alpha}},
\label{eq:parent}
\end{equation}
where 
\begin{equation}
{\left\langle\rho_{\lambda} | P\Psi_{\mu} \right\rangle}=0,
\end{equation}
for $\lambda\neq\mu =1,2,\dots ,D$. Recall that the unperturbed
eigenfunctions $\ket{\psi_{\alpha}}$ are the eigenfunctions of $H_0$, with
corresponding unperturbed eigenvalues $\varepsilon_{\alpha}$.
The latter equality holds because we
have assumed that the parent states are linearly independent.
The parent states should only be regarded as a mathematical tool in order
to obtain the effective interaction, since the construction of the parent
states depends on the projection of the true eigenfunctions onto the model
space. This projection is not available until one knows the effective
interaction. The final expression for eq. (\ref{eq:mspacee}) should
therefore not depend on the knowledge of $\ket{\rho_{\lambda}}$, as will
be demonstrated below.
Thus we wish to have a one-to-one correspondence between between $D$ parent
states $\ket{\rho_{\lambda}}$ in the model space and the true eigenstates
\begin{equation}
   \lim_{\epsilon \rightarrow 0}
   \lim_{t'\rightarrow -\infty (1-i\epsilon )}
   \frac{U(0,t' )\ket{\rho_{\lambda}} }
   { \bra{\rho_{\lambda}} U(0,t' )\ket{\rho_{\lambda}} }=
   \frac{\ket{\Psi_{\lambda}}}
   {\left\langle\rho_{\lambda} | \Psi_{\lambda} \right\rangle},
\end{equation}
such that 
\begin{equation}
    H\frac{U(0,-\infty )\ket{\rho_{\lambda}} }
    { \bra{\rho_{\lambda}} U(0,-\infty )\ket{\rho_{\lambda}} }=
    E_{\lambda}\frac{U(0,-\infty )\ket{\rho_{\lambda}} }
    { \bra{\rho_{\lambda}} U(0,-\infty )\ket{\rho_{\lambda}} }.
    \label{eq:truee}
\end{equation}
This equation states that the parent states $\rho_{\lambda}$
should give the true eigenvalues $E_{\lambda}$. In eq.\ (\ref{eq:truee})
we have suppressed the complex time limit. Such a one-to-one correspondence
can only be proven in the complex time approach, as demonstrated in ref.\
\cite{ko90}\footnote{We will not repeat the proof leading to the above statement.
For the details, we refer the reader to ref.\ \cite{ko90}, p.\ 35f.}. 
According to the complex-time approach, the eigenvalues
reproduced by eq.\ (\ref{eq:mspacee})
are the lowest $D$ eigenvalues with eigenvectors $\ket{\Psi_{\lambda}}$,
$\lambda =1,2,\dots ,D$, with non-zero projection $P\ket{\Psi_{\lambda}}$ onto
the model space. In actual calculations however, we may not obtain the 
lowest $D$ eigenstates of $H$ with non-zero model-space overlaps since we are
are not able to compute the effective interaction exactly. In the present
work we will approximate the effective hamiltonian with certain classes of 
diagrams, thus only qualitative arguments about e.g.\ the convergence
property of the effective interaction can be made. 
\end{itemize}

We wish now to derive an expression for the model-space effective 
interaction which has the structure of eq.\ (\ref{eq:mspacee}). 
Employing the definition of the parent state in eq.\ (\ref{eq:parent}),
we can rewrite eq.\ (\ref{eq:truee}) as
\begin{equation}
    {\displaystyle \sum_{\alpha =1}^{D}C_{\alpha}^{(\lambda )}
    H\frac{U(0,-\infty )\ket{\psi_{\alpha}} }
    { \bra{\rho_{\lambda}} U(0,-\infty )\ket{\rho_{\lambda}} }=
    \sum_{\beta =1}^{D}C_{\beta}^{(\lambda )}
    E_{\lambda}\frac{U(0,-\infty )\ket{\psi_{\beta}} }
    { \bra{\rho_{\lambda}} U(0,-\infty )\ket{\rho_{\lambda}} }},
    \label{eq:truee1}
\end{equation}
which, through use of the decomposition theorem in eq.\ (\ref{eq:decomp}) 
can be written as
\begin{equation}
{\displaystyle \sum_{\gamma=1}^{D}b_{\gamma}^{(\lambda )}
HU_Q(0,-\infty )\ket{\tilde{c}}U_{VQ}(0,-\infty )\ket{\psi_{\gamma}}=
\sum_{\sigma =1}^{D}b_{\sigma}^{(\lambda )}E_{\lambda}
U_Q(0,-\infty )\ket{\tilde{c}}U_{VQ}(0,-\infty )\ket{\psi_{\sigma}}},
\label{eq:truee2}
\end{equation}
where we have defined 
\begin{equation}
b_{\sigma}^{(\lambda )}=
{\displaystyle \sum_{\alpha =1}^{D}C_{\alpha}^{(\lambda )}
\frac{\bra{\psi_{\sigma}}U_V(0,-\infty )\ket{\psi_{\alpha}}
\bra{\tilde{c}}U(0,-\infty )\ket{\tilde{c}}}
{ \bra{\rho_{\lambda}} U(0,-\infty )\ket{\rho_{\lambda}} }}.
\label{eq:bk}
\end{equation}
Note that both the terms 
$\bra{\psi_{\sigma}}U_V(0,-\infty )\ket{\psi_{\alpha}} $ and
$\bra{\tilde{c}}U(0,-\infty )\ket{\tilde{c}}$ contain
divergencies, these are however cancelled by corresponding terms in the
denominator of eq.\ (\ref{eq:bk}). The quantity $b_{\sigma}$ 
is in turn proportional
to the projection of the true eigenstate onto the model space, a property
which follows by multiplying 
\begin{equation}
    \frac{\ket{\Psi_{\lambda}}}
    {\left\langle\rho_{\lambda} | \Psi_{\lambda} \right\rangle},
\end{equation}
with the model-space basis state $\bra{\psi_{\sigma}}$
\begin{equation}
    \frac{\left\langle\psi_{\sigma} | \Psi_{\lambda} \right\rangle }
    {\left\langle\rho_{\lambda} | \Psi_{\lambda} \right\rangle}=
    {\displaystyle \sum_{\alpha =1}^{D}C_{\alpha}^{(\lambda )}
    \frac{\bra{\psi_{\sigma}}U_V(0,-\infty )\ket{\psi_{\alpha}} 
    \bra{\tilde{c}}U(0,-\infty )\ket{\tilde{c}}}
    { \bra{\rho_{\lambda}} U(0,-\infty )\ket{\rho_{\lambda}} }}=
    b_{\sigma}^{(\lambda )}.
    \label{eq:bk1}
\end{equation}
Thus, the only dependence of the model-space eigenvalue problem
on the parent state $\ket{\rho_{\lambda}}$ is through the
coefficient $b_{\sigma}^{(\lambda )}$, and, as we will 
demonstrate below, we may solve the model-space eigenvalue
problem directly for $b_{\sigma}^{(\lambda )}$. 
For a known $b_{\sigma}^{(\lambda )}$, 
eq.\ (\ref{eq:bk1}) serves to establish that the parent state 
really can be constructed as the projection of the true eigenstate. We can
then construct a model-space eigenstate $\ket{b_{\lambda}}$ as
\begin{equation}
     \ket{b_{\lambda}}=\sum_{\alpha =1}^{D}
     b_{\alpha}^{(\lambda )}\ket{\psi_{\alpha}}=
     \frac{P\ket{\Psi_{\lambda}}}
     {\left\langle\rho_{\lambda} | \Psi_{\lambda} \right\rangle}.
     \label{eq:wfb}
\end{equation}
Before we construct the final expression for $H_{\mathrm{eff}}$ we observe that in general
\begin{equation}
    {\left\langle b_{\lambda} | b_{\lambda} \right\rangle}\neq 1,
\end{equation}
since the projections of the true orthogonal eigenvectors onto the 
model space, do not in general preserve orthogonality. This deficiency, which
may lead to a non-hermitian $H_{\mathrm{eff}}$ can be overcome by introducing the 
biorthogonal wave function
\begin{equation}
     \ket{\overline{b}_{\lambda}}=\sum_{\alpha=1}^{D}
      \overline{b}_{\alpha}^{(\lambda )}
     \ket{\overline{\psi}_{\alpha}},
     \label{wfbi}
\end{equation}
such that 
\begin{equation}
     {\left\langle \overline{b}_{\lambda} | b_{\mu} \right\rangle}=
     \delta_{\lambda\mu}.
\end{equation}

With these preliminaries we are now able to write a model-space equation
for the effective interaction of the form given by eq.\
(\ref{eq:expect}). 
Let us first introduce the shorthand notation
\begin{equation}
    U_L(t,t')\ket{\psi_{\alpha}}=U_{VQ}(t,t' )U_Q(t,t' )\ket{\psi_{\alpha}}
\end{equation}
which can be written as, employing the folding operation \cite{ko90},
\begin{equation}
    U_L(t,t')\ket{\psi_{\alpha}}=\ket{\psi_{\alpha}}+
     \left( {\displaystyle W-W\int\hat{Q} 
    +W\int\hat{Q}\int\hat{Q} -\dots}\right)\ket{\psi_{\alpha}}
    =\ket{\psi_{\alpha}} +\tilde{W}(t,t')\ket{\psi_{\alpha}},
\label{eq:wavefold}
\end{equation}
where the $\int$ signs represent the folding operation and the structure
of the wave function operator $\tilde{W}$ is shown in fig.\ \ref{fig:uvq}.
Observe also that 
\begin{equation}
U_L^{\dagger}(0,-\infty)=U_L(\infty ,0).
\end{equation}
The first term in eq.\ (\ref{eq:wavefold}) is just the free propagation 
of a state. The remaining terms include at least one nuclear interaction
and always end in a passive state. The folding operation $\int$ 
means that the states
connecting the \qbox and the wave operator $W$  must be active states.

Multiplying eq.\ (\ref{eq:truee2}) from the left with 
$\bra{\psi_{\sigma}}$ we obtain
\begin{equation}
     {\displaystyle
     \sum_{\gamma =1}^{D}b_{\gamma}^{(\lambda )}\bra{\psi_{\sigma}}
     HU_L(0,-\infty )\ket{\psi_{\gamma}}  } =
     E_{\lambda}b_{\sigma}^{(\lambda )},
     \label{eq:truee3}
\end{equation}
or
\begin{equation}
     PH_{\mathrm{eff}}\ket{P\Psi_{\lambda}}=
     E_{\lambda}\ket{P\Psi_{\lambda}},
     \label{eq:truee4}
\end{equation}
where
\begin{equation}
     H_{\mathrm{eff}}=
     \bra{\psi_{\sigma}}
     HU_L(0,-\infty )\ket{\psi_{\gamma}}  
\label{eq:heff1}
\end{equation}
Eqs.\ (\ref{eq:truee3}) and (\ref{eq:truee4}) have indeed the form of a
model space effective interaction equation. 
The reader should now note an important point. To obtain
eq.\ (\ref{eq:truee3}) we multiply from the left with the
unperturbed wave function $\bra{\psi_{\sigma}}$. This operation gives rise
to a non-hermitian effective interaction if the expansion
is truncated at a given order in the interaction, though  for model spaces
defined within a single shell, this  non-hermiticity is weak. At the 
end of section 6 we will discuss how to obtain a hermitian effective
interaction which is hermitian to all orders. If one is able to evaluate
the effective interaction to all orders, then all perturbative schemes
should in principle result in the same and hermitian effective interaction.
Our presentation here follows the traditional Goldstone expansion for
valence spaces.

The hamiltonian $H$ is written as the sum of an unperturbed part $H_0$ and an
interaction term $H_1$.
In order to understand the structure of eq.\
(\ref{eq:heff1}) and obtain a secular model-space equation which resembles
that of the empirical shell model, we study first the contributions from
$H_0$ to eq.\ (\ref{eq:heff1}). This contribution can be written as
\begin{equation}
    \bra{\psi_{\sigma}}
    H_0U_L(0,-\infty )\ket{\psi_{\gamma}}  =
    \bra{\psi_{\sigma}}H_0\ket{\psi_{\gamma}}
\end{equation}
The term on the rhs.\ of the
latter equation comes through due to the fact that the only $P$-space
component in the wave operator $U_L$ is the unperturbed
wave function $\psi$. This term represents therefore nothing but the
unperturbed energies and is conventionally \cite{ko90}
split into a valence
component and a core contribution
\begin{equation}
     \bra{\psi_{\sigma}}H_0\ket{\psi_{\gamma}}=\delta_{\sigma\gamma}\left(
     \varepsilon_V + \varepsilon_C \right),
\end{equation}
where $\varepsilon_V$ is the eigenvalue of the valence part while 
$\varepsilon_C$ is the core contribution. In our example,
$^{18}$O, $\varepsilon_V$ is the unperturbed energy of the valence
particles while $\varepsilon_C$ is the unperturbed energy of the
$^{16}$O core.  Below we will show that the effective interaction can 
conveniently be expressed in terms of eq.\ (\ref{eq:fd}).
To obtain this, we decompose the interaction term $H_1$
into a valence part $H_1(V)$ and 
into a part $H_1(C)$ which at the time $t=0$ is not linked to any valence
line. The latter contributions give rise to contributions to the true core
energy $E_C$. Examples of such terms are exhibited in the upper part
of fig.\ \ref{fig:heff12}. Typical contributions to $H_1(V)$ are shown
in the lower part of fig.\ \ref{fig:heff12}. Thus, summarizing, we may write
\begin{figure}[hbtp]
      \setlength{\unitlength}{1mm}
      \begin{picture}(90,90)
      \put(30,10){\epsfxsize=9cm \epsfbox{heff12.eps}}
      \end{picture}
\caption{Examples of terms which contribute to $H_1(C)$,
upper part, and $H_1(V)$, lower part.}
\label{fig:heff12}
\end{figure}
\begin{equation}
    \bra{\psi_{\sigma}}
    HU_L(0,-\infty )\ket{\psi_{\gamma}}  =
    \delta_{\sigma\gamma}\left(
    \varepsilon_V + \varepsilon_C \right)+
    \bra{\psi_{\sigma}}
    \left(H_1(V)+H_1(C)\right) U_L(0,-\infty )
    \ket{\psi_{\gamma}} ,
\end{equation}
which finally yields 
\begin{equation}
     {\displaystyle
     \sum_{\gamma =1}^{D}b_{\gamma}^{(\lambda )}\bra{\psi_{\sigma}}
     \left(H_0(V)+H_1(V)\right) U_L(0,-\infty )
     \ket{\psi_{\sigma}}  } =
     \left(E_{\lambda}-E_C\right) b_{\sigma}^{(\lambda )},
     \label{eq:truee5}
\end{equation}
with $E_C=\varepsilon_C +H_1(C)$. Clearly, eq.\ (\ref{eq:truee5}) has the 
form of the empirical shell-model secular eq.\ (\ref{eq:mspacee}). 
Due to the inclusion of folded diagrams, eq.\ (\ref{eq:truee3}) is 
in general non-hermitian. 
A standard approach to obtain a hermitian effective
interaction, is 
\begin{equation}
      \bra{\psi_{\sigma}}H_{\mathrm{eff}}^{(\mathrm{her})}
      \ket{\psi_{\gamma}}\approx
      \frac{1}{2}\left( \bra{\psi_{\sigma}}H_{\mathrm{eff}}^{(\mathrm{nher})}
      \ket{\psi_{\gamma}}+
      \bra{\psi_{\gamma}}H_{\mathrm{eff}}^{(\mathrm{nher})}
      \ket{\psi_{\sigma}}\right),
      \label{eq:bruteforce}
\end{equation}
where $(\mathrm{her})$ and $(\mathrm{nher})$
stand for hermitian and non-hermitian,
respectively.
The non-hermiticity is not unphysical, since, if we were able
to sum all possible terms, this non-hermiticity will not arise, though
it is possible to obtain an effective interaction which is
hermitian for every order in the interaction, as demonstrated 
in section 6.6.

The expression we have obtained from time-dependent perturbation
theory is not directly applicable for computations. A compact 
expression like eq.\ (\ref{eq:fd}) is desirable. 

When we include all terms from 
the wave operator $U_L$ in eq.\ (\ref{eq:wavefold}), we obtain,
by acting with $H$ on $U_L$,
\begin{equation}
   \bra{\psi_{\beta}}H_{\mathrm{eff}}\ket{\psi_{\alpha}}=
   \bra{\psi_{\beta}}HU_L(t,t')\ket{\psi_{\alpha}},
\end{equation} 
which we illustrate in fig.\ \ref{fig:foldedexp}.
\begin{figure}[hbtp]
      \setlength{\unitlength}{1mm}
      \begin{picture}(140,60)
      \put(25,10){\epsfxsize=12cm \epsfbox{foldedexp.eps}}
      \end{picture}
      \caption{The structure of the folded-diagram expansion.}
       \label{fig:foldedexp}
\end{figure}
The various $\hat{Q}$-boxes have different meanings. Both
$\hat{Q}$-boxes are composed of irreducible diagrams in which
the vertices are linked to at least one valence line and $\hat{Q}'$
is at least second order in the interaction, while $\hat{Q}$ 
starts with first order terms\footnote{With the inclusion 
of the nuclear reaction matrix $G$, the difference between 
the two $\hat{Q}$-boxes vanish. Further, we have dropped any
reference to so-called last-moment core-insertions \cite{ko90},
since this diagrams also vanish when we include the nuclear reaction matrix
$G$.}. The structure of the effective interaction can then be
written as
\begin{equation}
   H_{\mathrm{eff}}=
     \hat{Q}   -\hat{Q}'\int\hat{Q} 
    +\hat{Q}'\int\hat{Q}\int\hat{Q} -\dots
\end{equation} 
A contribution to the second term in this equation is shown
in fig.\ \ref{fig:foldedex5}
\begin{figure}[hbtp]
      \setlength{\unitlength}{1mm}
      \begin{picture}(140,70)
      \put(25,10){\epsfxsize=12cm \epsfbox{foldedex5.eps}}
      \end{picture}
      \caption{(i) shows an example of  a contribution to $\hat{Q}'\int\hat{Q}$,
       where (ii) belongs to $\hat{Q}'$ and (iii) belongs to $\hat{Q}$.}
       \label{fig:foldedex5}
\end{figure}
where the time $t_2$ is between $0$ and $t_1$. With a degenerate model space
we obtain the expression for (i)
\begin{equation}
              \frac{1}{8}\tilde{V}_{\alpha\beta ij}
              \tilde{V}_{ij \gamma\delta}
               \tilde{V}_{\gamma\delta\mu\nu}
             \frac{1}{(\omega -
                   \varepsilon_{i}-\varepsilon_{j})^2},
\end{equation}
where we have used the expression for the wave function contribution
in eq.\ (\ref{eq:wavefex6}). In fact, what we have done is to act on the
expression in eq.\ (\ref{eq:wavefex6}) with $H_1$. The expression
for diagram (i) can in turn be written as
\begin{equation}
              -\frac{1}{8}\frac{d}{d\omega}
              \left(\tilde{V}_{\alpha\beta ij}
              \tilde{V}_{ij \gamma\delta}
              \frac{1}{\omega -
              \varepsilon_{i}-\varepsilon_{j}}\right)
               \tilde{V}_{\gamma\delta\mu\nu}.
\end{equation}
This is a rather general result which can be extended to all possible
folded diagrams. Using the fact that the \qbox can be written as
\begin{equation}
  \hat{Q}_{\alpha\beta}=\tilde{V}_{\alpha\beta}+
          \sum_i\frac{\tilde{V}_{\alpha i}\tilde{V}_{i\beta}}
                     {\omega -\varepsilon_i}+
           \sum_{ij}\frac{\tilde{V}_{\alpha i}\tilde{V}_{ij}\tilde{V}_{j\beta}}
                     {(\omega -\varepsilon_i)(\omega -\varepsilon_j)}+\dots ,
\end{equation}
where $ij$ are intermediate states not belonging to the model space, it 
is easy to show that a term like $-\hat{Q}'\int\hat{Q}$ is equal
to
\begin{equation}
   -\hat{Q}'\int\hat{Q}=-\frac{d\hat{Q}'(\omega)}{d\omega}\hat{Q}(\omega),
\end{equation}
and we have assumed that $0>t_2>t_1$ and used generalized time ordering
so that all diagrams with $0>t_2>t_1$ are included.
A contribution to the folded diagram expansion like
$\hat{Q}'\int\hat{Q}\int\hat{Q}$ are of the form as shown
in fig.\ \ref{fig:foldedex6}.
\begin{figure}[hbtp]
      \setlength{\unitlength}{1mm}
      \begin{picture}(140,75)
      \put(25,10){\epsfxsize=12cm \epsfbox{foldedex6.eps}}
      \end{picture}
       \caption{Examples of contributions to
       $\hat{Q}'\int\hat{Q}\int\hat{Q}$. (i) has the time
       ordering $0>t_4>t_2>t_1$, $t_2>t_3$ and $t_4>t_5$. For (ii)
       we have $0>t_2>t_1$, $t_2>t_4>t_3$ and $t_4>t_5$.}
       \label{fig:foldedex6}
\end{figure}
Diagram (i) in fig.\ \ref{fig:foldedex6} reads
\begin{equation}
  (i)=\frac{1}{2!}\frac{d^2\hat{Q}'(\omega)}{d\omega^2}
       \hat{Q}(\omega)\hat{Q}(\omega),
\end{equation}
and diagram (ii) in the same figure is
\begin{equation}
  (ii)=\frac{d\hat{Q}'(\omega)}{d\omega}\frac{d\hat{Q}(\omega)}{d\omega}
       \hat{Q}(\omega).
\end{equation}
The total contribution from diagrams (i) and (ii) has the following
time constraints 
\begin{equation}
  0>t_2>t_1, \hspace{5mm} 0>t_4>t_3,\hspace{5mm} t_4>t_5,
  \hspace{5mm} t_2>t_3.
\end{equation}

These results can be extended to the general case with $n$ folds. In
an $n$-folded \qbox there are of course $n+1$ $\hat{Q}$-boxes. The 
general expression for an $n$-folded \qbox is then
\begin{equation}
        \hat{Q}   -\hat{Q}'\int\hat{Q} 
    +\hat{Q}'\int\hat{Q}\int\hat{Q} -\dots=
    {\displaystyle\sum_{m_1m_2\dots m_n}}
    \frac{1}{m_1!}\frac{d^{m_1}\hat{Q}'}{d\omega^{m_1}}P
    \frac{1}{m_2!}\frac{d^{m_2}\hat{Q}}{d\omega^{m_2}}P
    \dots
    \frac{1}{m_n!}\frac{d^{m_n}\hat{Q}}{d\omega^{m_n}}P\hat{Q},
\label{eq:fdfinal}
\end{equation}
where we have the constraints
\[
  m_1+m_2+\dots m_n=n,
\]
\[
m_1\geq 1,
\]
\[
m_2, m_3, \dots m_n \geq 0,
\]
and
\[
m_k \leq n-k+1.
\]
The last restriction follows from the fact that there are only
$n-k+1$ $\hat{Q}$-boxes to the right of $k^{\mathrm{th}}$ \qbox.
Thus, it can at most be differentiated $n-k+1$ times. 
We have inserted the model-space projection operator in the above
expression, in order to emphasize that folded diagrams have as intermediate
states between successive $\hat{Q}$-boxes 
only model-space states. Therefore, the sum in eq.\ (\ref{eq:fdfinal})
includes a sum over all model-space states with the same quantum 
numbers such as isospin and total angular momentum. It is understood
that the \qbox  and its derivatives are evaluated at the
same starting energy, which should correspond to the unperturbed 
energy of the model-space state.     
It is then straightforward to show that eq.\ (\ref{eq:fdfinal}) 
can be recast into the form (we exclude $H_0$)
\begin{equation}
    V_{\mathrm{eff}}^{(n)}=\hat{Q}+{\displaystyle\sum_{m=1}^{\infty}}
    \frac{1}{m!}\frac{d^m\hat{Q}'}{d\omega^m}\left\{
    V_{\mathrm{eff}}^{(n-1)}\right\}^m,
\end{equation}
where we have chosen $V_{\mathrm{eff}}^{(0)}=\hat{Q}$. 

This expansion will be studied for finite nuclei in section 6. In the next
section we discuss the evaluation of the $G$-matrix for both finite
nuclei and nuclear matter.

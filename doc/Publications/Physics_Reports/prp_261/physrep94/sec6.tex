
\subsection{Introduction}

In this section  we calculate effective interactions
appropriate for nuclei in the mass regions of oxygen, calcium and
tin, using the formalism presented in section 4   and
$G$-matrices obtained with the Bonn A, B and C potentials
from table A.1 of \cite{mac89}. However, before
presenting these results, we need to deal with technicalities and problems
which arise in the evaluation of the effective interaction.
Typical problems which arise are the convergence of the
order-by-order perturbative expansion, the summation over
intermediate states which arise in the various Feynman-Goldstone
diagrams and so forth. In the next subsection we discuss
some of these problems, and methods to overcome the difficulties
which arise. In the subsequent
subsections, we present our results
for nuclei in the above mass regions. Finally, in subsection 6.6 we outline
how to calculate a hermitian effective interaction, appropriate
for multi-shell effective interactions.

Concluding remarks and open
problems resulting from our calculations
will be presented in section 9. 


\subsection{Problems in perturbation theory}

\subsubsection{Intruder states}

It is a well-known fact that the presence of so-called
intruder states \cite{ko90,sw72}
may lead to the divergence of the order-by-order
perturbative expansion
for $H_{\mathrm{eff}}$.
To understand the intruder-state problem, let
us illustrate their meaning and importance by
considering the spectra for the
three lowest-lying $0^+$ states in $^{18}$O.
These are displayed in fig.\ \ref{fig:intruder}.
\begin{figure}[hbtp]
    \setlength{\unitlength}{1mm}
    \begin{picture}(100,70)
      \put(25,10){\epsfxsize=10cm \epsfbox{intruder.eps}}
     \end{picture}
  \caption{Intruder states in $^{18}$O.}
\label{fig:intruder}
\end{figure}
In calculations of effective interactions for $^{18}$O,
one normally chooses a model space
consisting of a closed $^{16}$O core with two nucleons
in the sp states of the $sd$-shell.
This model space consists of two-particle states only,
and for $J=0$ we have
three $0^+$ states.
However, from model calculations of energies and
electromagnetic properties and experimental
data, the belief is that the $0_2^+$ state is predominantly of a
four-particle-two-hole nature, while the ground
state and the $0_3^+$ state are classified as two-particle
states. In the left part of fig.\ \ref{fig:intruder} we show, in a
schematic way, the unperturbed
eigenstates with a degenerate model space,
arising from the unperturbed hamiltonian
$H_0$. The model space states are separated from those
of the excluded space by an amount $\Delta$.
We let the hamiltonian of the system be described as $H_0+zH_1$,
where $H_1$ is the interaction term and $0\leq z\leq 1$ is just
a strength parameter with $z=1$ resulting in the fully interacting
system.
As we switch
on the interaction term $zH_1$, the $Q$-space
terms come down into the $P$-space region,
and we have a situation where one
$Q$-space eigenvalue becomes the $0_2^+$ state.
Thus, as inferred by
Schuchan and Weidenm\"{u}ller \cite{sw72},
the order-by-order perturbative expansion should not
converge if we restrict a model space to
consist of two-particle states only.


One way to handle intruder states is that of introducing an
enlarged
model space which includes intruder states.  Such calculations
do however become prohibitively time-consuming for nuclei heavier
than $sd$-shell nuclei.
There are also other approaches which aim at overcoming the
divergence of the order-by-order expansion of $H_{\mathrm{eff}}$. One may
e.g.\
regroup the perturbative expansion and perform infinite partial
summations to
obtain convergence. The folded-diagram approach  
we will apply in our calculations,
is an example of such a method.

An important point in connection with the intruder state discussion,
is that the derived two-body (or three-body) effective interaction,
is in conventional shell-model calculations used to compute 
the spectra for nuclei with more than two valence nucleons.
For such systems, the degrees of freedom represented by 
intruder states, may not be relevant for say $^{28}$Si. From 
this argument, although the effective $sd$-shell interaction
is not fully appropriate for $^{18}$O, it may very well
be a good approximation to the low-lying spectra 
of nuclei with more than two valence nucleons. 
Finally, even if the effective interaction expansion for $^{18}$O should
diverge at higher order, most effective interaction calculations 
show that (see e.g.\ \cite{eehho94}) the admixture of 
$Q$-space states is weak at low orders in the interaction, and
gives a fairly good reproduction of the lowest lying states,
though, this effective interaction, 
as we will discuss in the next subsections,
is not able to reproduce e.g.\ the second $0^+$ state in $^{18}$O.

\subsubsection{Evaluation of the \qbox}

There are also other problems which arise in the
evaluation of the effective interactions.
In section 5 we studied the stability of the
$G$-matrix
as a function of $n_3$, where $n_3$ defines the Pauli operator
for finite nuclei, see
fig.\ \ref{fig:qoperat}.
Here we extend this discussion to higher-order
diagrams as well.
Other questions we deal with is how to approximate the $\hat{Q}$-box,
and how to restrict the summation over intermediate states
in each \qbox diagram in  a reliable way. The \qbox is the starting
point in our calculations of the folded-diagram (FD) effective
interaction.
Thus, the reader should note that when we discuss the convergence
of the effective interaction, we will not only deal with
the order-by-order convergence of the effective interaction, but
extend our discussion on convergence to include the convergence
of the sum over intermediate states in each diagram, and the convergence
in terms of $n_3$.
We limit the discussions  to the $sd$-shell,
although the qualitative behaviors presented apply equally well to the
other mass areas.

To obtain the effective interaction we first need to
calculate the Brueckner $G$-matrix by solving the Bethe-Goldstone
equation. The latter
has been
solved employing the Bonn A, B and C potentials discussed
in section 5  for various starting energies $\omega$, with
$\omega=-140,-90,-50,-20,-5$ MeV. The Pauli-exclusion operator
is defined such as to prevent scattering into
intermediate
particle states with a nucleon in $0s$ or $0p$ states or both
nucleons in the $1s-0d$ or $1p-0f$ shell, or in terms of the
boundaries $n_1, n_2$ and $n_3$ $(n_1=3, n_2=10, n_3=\infty )$. Below we will
see that $n_3=66$ is a fairly good approximation.
We use an oscillator parameter $b$ defined through the
relation
$\hbar\omega = 45A^{-1/3} - 25A^{-2/3}= 14$ MeV, which gives $b=1.72$ fm.
The ladder terms with two particles in the $1p0f$
shell are taken into account explicitly when we evaluate the
diagrams which
enter the $\hat{Q}$-box.

The open question is how to approximate the $\hat{Q}$-box. Until
recently, a $\hat{Q}$-box including all linked and irreducible diagrams
through second order in the $G$-matrix has been the common choice.
Examples of such calculations with a second-order
$\hat{Q}$-box can be found in ref.\ \cite{skd83} for the $sd$-shell
and the recent work of Polls {\em et al.}
\cite{prm90} for the $pf$-shell.
The constituent topologies of a second-order $\hat{Q}$-box are listed
in appendix A for the one-body
and two-body diagrams respectively, i.e.\ diagrams (1-1) to (1-4)
in fig.\ \ref{fig:onebody} and (2-1) to (2-4) in fig.\ \ref{fig:twobody}.
Conventionally, since the single-particle energies are poorly reproduced, one
approximates the one-body part with the experimental
single-particle energies.
Having accounted for the one-body terms in an effective way,
one could represent the
$\hat{Q}$-box by only including the bare $G$-matrix and the
core-polarization term,
diagram (2-2) in fig.\ \ref{fig:twobody}.
This was the original approach presented by Kuo and Brown
\cite{kb66,kuo68} in order to
determine an effective interaction for mass-18 nuclei.
The remaining second-order
two-body diagrams of fig.\ \ref{fig:twobody} were
also included. Moreover,
the area of investigation was extended to heavier nuclei,
such as those in the calcium region \cite{kb68} and
to the lead region in the work
by Kuo and Herling \cite{hk72}.

The success of these early calculations is demonstrated,
in spite of their age,  by the frequent
use of the obtained interactions in various calculations
of nuclear properties, either
in their original form or in a somewhat
modified form \cite{brown88,ryd90,wb91}.
Several improvements to the original Kuo-Brown
approach can be found in the literature
\cite{eo77}, using either the partial summations
of the folded diagrams as discussed
above, or by including renormalized particle-hole
interactions as advocated by the
random-phase approximation or studying the perturbation expansion
order-by-order
in the interaction. Following the latter philosophy,
Barrett and Kirson \cite{bk70}
showed that third-order contributions played a sizable role, questioning
thereby many of the conclusions reached in previous works.
Typical examples of third-order topologies
are shown in appendix A, figs.\ \ref{fig:onebody} and \ref{fig:twobody}.
To evaluate the $\hat{Q}$-box 
we will expand the denominator of eq.\ (\ref{eq:qbox}) such that
we obtain a perturbative expansion for the $\hat{Q}$-box which reads
\begin{equation}
P\hat{Q}P=PH_1P+
P\left(H_1 \frac{Q}{\omega-H_{0}}H_1+H_1
\frac{Q}{\omega-H_{0}}H_1 \frac{Q}{\omega-H_{0}}H_1 +\dots\right)P,
\label{eq:newqbox}
\end{equation}
In actual calculations, the interaction $H_1$ is replaced by $G-U$.

Before we study various properties of the \qbox,
let us first discuss the physical meaning of many of the diagrams
shown in appendix A.
The second-order
two-particle ladder diagram has already been discussed in section 5.
Diagram 2-4 of fig.\ \ref{fig:twobody} in appendix A,
represents the
admixture of four-particle-two-hole states, configurations
which are important for nuclei near a closed core nucleus like
$^{16}$O. In this sense, diagram 2-4 is supposed to represent
degrees of freedom (such as 4p-2h intruder states)
not included in the definition of the model
space. One might then argue that in calculations of the $sd$-shell
effective interactions one should not include diagrams with
4p-2h intermediate states in order to avoid  double counting.
However, the majority of 4p-2h states are not intruder states, so one
would probably make less error by including diagrams with 4p-2h
intermediate states.
Also, to low orders in the interaction, the
$Q$-space component is weak \cite{eehho94}. Diagram 2-2 in the
same figure is the well-know core-polarization term, and its
physical interpretation has been discussed by several authors,
see e.g.\ refs.\ \cite{eo77,kb66}.
Here we restate the arguments of ref.\ \cite{eo77}. A decomposition
of the core-polarization diagram into $J$ and $T$ of the intermediate
states, shows that the dominant contribution comes from $J=2$ and $T=0$,
which corresponds to a quadropole-quadropole interaction. This
quadropole component gives rise to a long-range component
in the effective interaction, a long-range component which 
can be understood from the following pictorial configuration 
space representation of the core polarization process in fig.\ 
\ref{fig:pictorialcore}. 
\begin{figure}
    \setlength{\unitlength}{1mm}
    \begin{picture}(140,100)
     \put(25,10){\epsfxsize=12cm \epsfbox{snapshot.eps}}
     \end{picture}
\caption{Snapshots of the core-polarization process.}
\label{fig:pictorialcore}
\end{figure}
In order for two nucleons to interact via the $G$-matrix, they must be
fairly close to each other. Thus two valence nucleons on opposite sites
of the core, see (a) in the above figure, cannot interact directly
via $G$. However, they may interact via the exchange of a particle-hole
excitation as shown by the series of snapshots of the process depicted
in (b)-(d). First, the valence particle $\alpha$ interacts, thus
being left in a state $\gamma$, with a nearby core particle, creating
a particle-hole pair (b). This particle-hole pair 
propagates over to the valence particle $\beta$ (c) in a time
interval $\Delta T \approx \hbar/\Delta E$, where $\Delta E$ is the
excitation energy. When the particle-hole pair is close enough 
to $\beta$ (d), it is then annihilated, leaving $\beta$ in a state
$\delta$ (e). Thus a long-range effective interaction is generated
between a pair of valence particles $\alpha$ and $\beta$, leaving them
in states $\gamma$ and $\delta$. 

For several years, the belief was after the work of
Kuo and Brown \cite{kb66}, that
the effective interaction could very well be approximated by
the $G$-matrix, which accounts for the short-range part
of the interaction, and the core-polarization diagram to second-order.
However, although the results of Kuo and Brown resulted
in a good agreement with the data, these findings were questioned
by Barrett and Kirson \cite{bk70}
who found many third-order
$JT=01$ matrix elements to be as large as the second-order contributions,
and of opposite sign.
To demonstrate the importance of third-order diagrams, we present in
table \ref{tab:sdmtxel} various contributions to a third-order
\qbox for selected model-space
configurations\footnote{Note well that we omit higher-order
diagrams with Hartree-Fock insertions. We will tacitly assume that
a h.o.\ basis reproduces the spectra fairly well, although
for the less bound states in e.g.\ $^{18}$O, this may not be too
realistic a choice, see the discussion of ref.\ \cite{eo77}.}
in the $sd$-shell. The value of $n_3=66$ was used,
an approximation we found
to be rather good in section 5 and in the discussion below, see
table \ref{tab:n32ndconv}.
\begin{table}[hbtp]
\caption{The $sd$-shell two-body contributions to the $\hat{Q}$-box
for selected configurations
obtained with the Bonn A potential.
$G$ denotes the bare $G$-matrix, $CP$
is the second-order core-polarization diagram
whereas $2nd$ means all two-body second-order diagrams.
$3rd$ means all third-order diagrams without folded diagrams.
All entries in MeV.}
\begin{center}
\begin{tabular}{lllllrrrr}
\hline
$JT$&$j_{a}$&$j_{b}$&$j_{c}$&$j_{d}$&
\multicolumn{1}{c}{$G$}&
\multicolumn{1}{c}{$CP$}&
\multicolumn{1}{c}{$2nd$}&
\multicolumn{1}{c}{$3rd$}\\ \hline
01&$d_{5/2}$&$d_{5/2}$&$d_{5/2}$&$d_{5/2}$&-1.712&-1.086&-1.763&0.471\\
10&&&&&-0.314&-0.353&-1.126&-0.179\\
01&$d_{5/2}$&$d_{5/2}$&$d_{3/2}$&$d_{3/2}$&-3.462&-0.843&-1.193&1.247\\
10&&&&&2.668&-0.721&-0.692&-0.044\\
01&$d_{5/2}$&$d_{5/2}$&$s_{1/2}$&$s_{1/2}$&-0.810&-0.409&-0.588&-0.135\\
10&&&&&-0.493&-0.204&-0.548&-0.263\\
01&$d_{3/2}$&$d_{3/2}$&$d_{3/2}$&$d_{3/2}$&-0.323&-0.522&-1.057&0.061\\
10&&&&&-0.113&-0.108&-0.428&-1.006\\
01&$d_{3/2}$&$d_{3/2}$&$s_{1/2}$&$s_{1/2}$&-0.662&-0.253&-0.399&-0.007\\
10&&&&&-0.395&0.253&0.310&0.225\\
01&$s_{1/2}$&$s_{1/2}$&$s_{1/2}$&$s_{1/2}$&-2.417&-0.054&-0.136&-0.301\\
10&&&&&-3.509&0.176&-0.422&-0.665\\ \hline
\end{tabular}
\end{center}
\label{tab:sdmtxel}
\end{table}
We see from table \ref{tab:sdmtxel} that, although many third-order
contributions are small compared with the second-order ones, some of
them are rather large, of the size of, or even larger than
the $G$-matrix. Thus, we can
not neglect third-order terms completely and will retain
them in our calculation of the \qbox. Moreover, in this table
we have neglected third-order folded diagrams,
which may be rather large and repulsive.
Similar conclusions apply to matrix
elements in the $pf$- and $sdg$-shells as well. The reader should
note that the above $JT$ combinations have been chosen since they
give the largest differences between second and third order.
For the other $JT$ configurations such differences are smaller.
The $JT=01$ and $JT=10$ configurations yield the largest third-order
terms
in the tin and calcium regions as well.

The third-order contributions can however be reduced if one uses
a Hartree-Fock basis, as shown in ref.\ \cite{homs90},
though the
agreement with the experimental data for $^{18}$O was not as good
as that obtained with a h.o.\ basis. Another mechanism which
reduces the size of a higher-order diagram, is to take into account
the
dependence of the $G$-matrix of the starting energy, see
appendix A for relevant examples. Since the $G$-matrix is not
on-shell for the intermediate states, a proper evaluation
of the starting energy dependence reduces the absolute value of
the $G$-matrix (see the discussion in section 5), and thereby
the absolute value of the higher-order diagram.
For starting energies close to the chosen unperturbed
model-space sp energies, the variations in $G$ are rather small,
though for a diagram like the 2p-1h diagram discussed in appendix
A, the starting energy of the intermediate $G$-matrices is
$\omega \approx \varepsilon_h -\varepsilon_p$, which is of the order
$-30$ MeV or more. This number should be compared to the
starting energy appropriate for $sd$-shell nuclei, approximately
$-10$ MeV. The relevant matrix elements are on the average
reduced with $5\%$ when decreasing the starting energy from
$-10$ to $-30$ MeV. This effect comes in squared in the 2p-1h
diagram, since we have two interactions which are reduced.
For higher-order diagrams this effect is even stronger, and the
total effect on third-order diagrams with a proper evaluation
of the starting energy, is a reduction of their absolute value
of the order of $10-20\%$. The latter effect is properly taken 
into account into the results shown in the above
table, and in all our calculations of the effective interaction.

Although there are several third-order contributions, most of them
can be given a physical interpretation. Diagram 2-14 in fig.\
\ref{fig:twobody} in appendix A, is the third-order core-polarization
diagram, and can be interpreted as a renormalization of the particle-hole
propagator
of the second-order core-polarization diagram.
In a similar way, diagrams
2-15(A) and 2-15(B), which  are the third-order RPA diagrams,
and diagrams 2-18 and
2-19 in fig.\ \ref{fig:twobody} are also examples of corrections to the
second-order core-polarization diagram, since they modify the
propagator
which couples the particle-hole pair to the valence particles.

Another class of third-order diagrams which have a neat physical
interpretation are diagrams 2-5 to 2-10 in the appendix. We redisplay
diagrams 2-5 and 2-6 plus a third-order folded diagram
not included in the \qbox 
in fig.\ \ref{fig:nocosets}. It was suggested by Brandow
\cite{bran67} 
\begin{figure}
    \setlength{\unitlength}{1mm}
    \begin{picture}(140,80)
    \put(25,10){\epsfxsize=12cm \epsfbox{nocons.eps}}
     \end{picture}
\caption{The set of number-conserving diagrams discussed
in text.}
\label{fig:nocosets}
\end{figure}
that these diagrams should cancel approximately
according to the principle of number conservation.
If the middle vertex is replaced by the number operator, these
diagrams should cancel exactly. In physical terms, these diagrams
may be thought of as a correction to the bare $G$-matrix due to the creation
of a particle-hole pair. In 2-5, the particle in the particle-hole
pair interacts with the other valence particle. The folded diagram
is a correction of the same form as 2-5 to the $G$-matrix, but it
accounts for the fact that the valence particle may not be
in the same orbit all the time. This diagram has in general an opposite
sign to diagram 2-5. Diagram 2-6  represents the interaction of the
hole (core) with the other valence particle and balances
diagram 2-5.
Most NN potentials reproduce only approximately this number conserving
feature
\cite{eko85,ho90} in the $T=1$ channel, while in the $T=0$ channel,
the cancellation is rather poor. The qualitative conclusions reached
in refs.\ \cite{eko85,ho90} apply to the present potentials as well.
The second set of number-conserving diagrams exhibits a similar
behavior, here diagrams 2-7, 2-8 and 2-10(B) in fig.\ \ref{fig:twobody}
in the appendix
should cancel if the
middle interaction vertex is replaced by the number operator.
There are  also two other diagrams belonging
to this number-conserving set,
but they do not contribute if we restrict the
attention to excitations of $2\hbar\omega$.

We have seen that third-order contributions may be large, and
can not be neglected. In the evaluation of the \qbox, a third-order
\qbox contains some hundred diagrams if we include Hartree-Fock
contributions, and it is rather straightforward, but tedious, to
evaluate such a \qbox.
To fourth order in the interaction there are several hundreds of
diagrams which contribute to the effective interaction, and it is
still an open question how important such contributions will be, although
an estimation by Goode and Koltun \cite{gk75} by way of a diagrammatic
technique where averages of valence-particle diagrams in terms
of diagrams of closed-shell form are studied,
shows that fourth-order contributions
may be large. We can therefore not say anything about the
order-by-order convergence of the effective interaction.
However, as we discussed in section 4, a reordering of the perturbation
expansion can in principle yield a convergent perturbation
expansion. This is the philosophy behind the \qbox approach and the
summation of the folded diagrams, employed in this work. As we will show
in this section, the differences between $\hat{Q}$-boxes to various
orders in perturbation theory, become small when we sum folded diagrams
to all orders.


In the calculation of the \qbox and the
effective interaction for $sd$-nuclei
we will only allow intermediate
energy state excitations
of $2\hbar\omega -4\hbar\omega$ in oscillator energy when we evaluate
the various diagrams.
It was demonstrated by Sommerman {\em et al.} \cite{sommer}
that when one employs potentials with a weak tensor force, the
core-polarization diagram, (diagram 2-2 in
appendix A), the
intermediate state contributions could be approximated by
$2\hbar\omega$ in oscillator energy. The potentials
we will employ all exhibit a weak tensor force.
We will not include high-lying sp states in
our analysis of various \qbox diagrams, since an appropriate inclusion
of high-lying intermediate particle states would require a 
description in terms of plane waves. A $G$-matrix where
one or two of the particle states are in a plane wave basis
and the other two sp states are described in terms of 
h.o.\ states, can in principle be calculated  using the
techniques outlined at the end of section 5 (see eqs.\
(\ref{eq:kho}) and (\ref{eq:kk}). Though, the 
numerical expenditure would grow tremendously.
Here we will assume the analysis 
of ref.\ \cite{sommer} to be approximately valid for the
other diagrams as well, and carry
out our calculations restricting the summation over intermediate
states to $2\hbar\omega -4\hbar\omega$ in oscillator energy.
Actually, for third-order \qbox diagrams, the contributions from
high-lying states is indeed rather weak, since the energy
denominator appears as $(n\hbar\omega)^2$ ($n$ being the
number of oscillator shells). Thus such contributions are largely
quenched in higher-order diagrams.

We conclude this subsection by justifying our choice
of $n_3$ (which defines the Pauli operator
for finite nuclei in fig.\ \ref{fig:qoperat})
in table \ref{tab:n32ndconv}, where we show
\begin{table}[hbtp]
\caption{Dependence of the second-order diagrams
on the choice of $n_3$ for
matrix elements in the $sd$-shell for $J=0,1$ and $T=0,1$.
The results are for the $\hat{Q}$-box without the auxiliary
potential $U$.
Note that the \qbox includes both one-body and
two-body diagrams. No folded diagrams are included, and the results
were obtained with the Bonn A potential at a starting energy
$\omega =-10$ MeV.}
\begin{center}
\begin{tabular}{lllllrrrrr}
\\\hline
$JT$&$j_{a}$&$j_{b}$&$j_{c}$&$j_{d}$&
\multicolumn{1}{c}{(3,10,45)}&
\multicolumn{1}{c}{(3,10,66)}&
\multicolumn{1}{c}{(3,10,78)}&
\multicolumn{1}{c}{(3,10,91)}&
\multicolumn{1}{c}{(3,10,120)}
\\\hline
01&$d_{5/2}$&$d_{5/2}$&$d_{5/2}$&$d_{5/2}$&-60.33&-60.13&-60.10&-60.07&-60.05\\
10&&&& &-58.56&-58.97&-58.33&-58.29&-58.28\\
01&$d_{5/2}$&$d_{5/2}$&$d_{3/2}$&$d_{3/2}$&-4.59&-4.59&-4.60&-4.60&-4.60\\
10&&&& &2.12&2.11&2.11&2.11&2.11\\
01&$d_{5/2}$&$d_{5/2}$&$s_{1/2}$&$s_{1/2}$&-1.39&-1.40&-1.40&-1.40&-1.40\\
10&&&& &-1.12&-1.12&-1.12&-1.12&-1.12\\
01&$d_{3/2}$&$d_{3/2}$&$d_{3/2}$&$d_{3/2}$&-47.19&-47.05&-47.02&-47.00&-46.98\\
10&&&& &-46.59&-46.45&-46.42&-46.40&-46.40\\
01&$d_{3/2}$&$d_{3/2}$&$s_{1/2}$&$s_{1/2}$&-1.06&-1.06&-1.06&-1.06&-1.06\\
10&&&& &-0.02&-0.03&-0.03&-0.03&-0.03\\
01&$s_{1/2}$&$s_{1/2}$&$s_{1/2}$&$s_{1/2}$&-60.54&-60.09&-59.99&-59.91&-59.85\\
10&&&& &-62.13&-61.67&-61.56&-61.48&-61.41 \\\hline
\end{tabular}
\end{center}
\label{tab:n32ndconv}
\end{table}
the dependence of the \qbox on the choice of $n_3$ in the
calculation of the $G$-matrix.
The results list all diagrams of appendix A
to second order in $G$ (third-order diagrams show a neglible
dependence upon $n_3$), except for the auxiliary potential $U$,
diagram 1-2 in fig.\ \ref{fig:onebody}.
The \qbox consists of both one-body and two-body diagrams, and
as can be deduced from this table, the dependence on $n_3$
is rather weak, and the largest discrepancies occur for
the diagonal contributions. This is ascribed to
the first-order Hartree-Fock diagram, which does not stabilize
properly, as can be seen from table \ref{tab:un3dep}. However,
we subtract this diagram from our effective interaction, and our
approximation of $n_3=66$ introduces on the average an error of the order
$0.1\%$ or even less, a negligible effect.
\begin{table}[hbtp]
\caption{Dependence of
the Hartree-Fock sp potential, diagram 1-1 in fig.\ A.7,
on the choice of $n_3$ for the Bonn A
potential for a starting energy
$-10$ MeV for various sp orbits.}
\begin{center}
\begin{tabular}{lrrrrr}
\\ \hline
&
\multicolumn{1}{c}{(3,10,45)}&
\multicolumn{1}{c}{(3,10,66)}&
\multicolumn{1}{c}{(3,10,78)}&
\multicolumn{1}{c}{(3,10,91)}&
\multicolumn{1}{c}{(3,10,120)}
\\  \hline
$\bra{(1s_{1/2})}U\ket{(1s_{1/2})}$&-25.728&-25.539&-25.500&-25.464&-25.436\\
$\bra{(0d_{3/2})}U\ket{(0d_{3/2})}$&-21.031&-20.957&-20.945&-20.931&-20.922\\
$\bra{(0d_{5/2})}U\ket{(0d_{5/2})}$&-26.517&-26.405&-26.384&-26.364&-26.350\\
\\ \hline
\end{tabular}
\end{center}
\label{tab:un3dep}
\end{table}





\subsection{Results for nuclei in the mass region of $^{16}$O}

We present here
the results from effective interaction calculations for several
nuclei in the $sd$-shell.
First we will discuss the convergence of the order-by-order
perturbative expansion for the effective interaction, by considering
the results through third order in conventional Rayleigh-Schr\"{o}dinger (RS)
perturbation theory. In connection with the convergence of RS perturbation
theory,
we focus also on various approximations to the \qbox and study
the summation of the folded diagrams. Thereafter, we discuss the
convergence of the folded-diagram method itself. To study these
topics we will use nuclei in the $sd$-shell with two valence
nucleons only, in our case $^{18}$O and $^{18}$F. The effective
interactions we derive will in turn be used to calculate the
spectra for nuclei in the $sd$-shell with more than two
valence nucleons, here the nuclei $^{19}$F,
$^{20}$O, $^{20}$Ne,
and $^{22}$Ne. The effective matrix elements
are tabulated in appendix C.
The starting point for our calculations is the $G$-matrix, which
was defined in section 5 with
$(n_1=3, n_2=10, n_3 =66)$ and an oscillator parameter
$b=1.72$ fm. The three Bonn potentials, A, B and C will be used,
in order to study the role played by different tensor forces.


\subsubsection{Order-by-order perturbation theory and folded diagrams}

Here we present the results for various approximations to perturbation
theory, including the converged results from the folded-diagram
method with a \qbox of second and third order. These results are listed
in figs.\ \ref{fig:rs18oa}-\ref{fig:rs18fc} for the nuclei $^{18}$O and
$^{18}$F. Observe that, throughout this work we will only plot
the lowest lying states for each value of $J$. Thereby we omit
any discussion on intruder states and focus on those states which
from experimental data are expected to have a large overlap
with the  chosen model spaces.
We have employed all three Bonn potentials, i.e.\
versions A, B and C of table A.1 of ref.\ \cite{mac89}.

The label $\hat{Q}^{(1)}$ in these six figures stands for perturbation
theory to first order in the $G$-matrix. $\hat{Q}^{(2)}$ is
second-order perturbation theory, while $\hat{Q}^{(3)}$
is the third-order \qbox, i.e.\ no third-order folded diagrams
are included. Third-order folded diagrams are included in
$H_{\mathrm{eff}}^{(3)}$, which is RS perturbation theory
to third order. The converged results (for a discussion
on how to obtain these results, see below)
of a folded diagrams calculation
are shown in columns five and six, under the labels
$\hat{Q}^{(2)}+fold$ and $\hat{Q}^{(3)}+fold$, obtained with a
\qbox of second order and third order in $G$, respectively.
\begin{figure}[hbtp]
\begin{center}
\setlength{\unitlength}{1.0mm}
\begin{picture}(150,120)(0,-80)
\thicklines
\put(5,-65){\line(0,1){85}}
\put(5,20){\line(1,0){145}}
\put(5,-65){\line(1,0){145}}
\put(150,-65){\line(0,1){85}}
\multiput(5,-60)(0,10){8}{\line(1,0){2}}
\thinlines
\put(0,-61){-6}
\put(0,-51){-5}
\put(0,-41){-4}
\put(0,-31){-3}
\put(0,-21){-2}
\put(0,-11){-1}
\put(1,-1){0}
\put(1,9){1}
\multiput(5,-55)(0,10){8}{\line(1,0){1}}
\put(10,-30.3){\nl{0}}
\put(10,-19.9){\nl{2}}
\put(10,4.9){\nl{3}}
\put(10,-8.7){\nl{4}}
\put(30,-55.9){\nl{0}}
\put(30,-26.5){\nl{2}}
\put(30,11.6){\nl{3}}
\put(30,-6.1){\nl{4}}
\put(50,-47.4){\nl{0}}
\put(50,-27.1){\nl{2}}
\put(50,13.3){\nl{3}}
\put(50,-7.7){\nl{4}}
\put(70,-37.5){\nl{0}}
\put(70,-21.8){\nl{2}}
\put(70,13.9){\nl{3}}
\put(70,-5.5){\nl{4}}
\put(90,-38.1){\nl{0}}
\put(90,-18.8){\nl{2}}
\put(90,10.3){\nl{3}}
\put(90,-4.1){\nl{4}}
\put(110,-37.4){\nl{0}}
\put(110,-19.95){\nl{2}}
\put(110,11.46){\nl{3}}
\put(110,-5.3){\nl{4}}
\put(130,-39.0){\nl{0}}
\put(130,14.8){\nl{3}}
\put(130,-19.2){\nl{2}}
\put(130,-3.5){\nl{4}}
\put(10,-75){$\hat{Q}^{(1)}$}
\put(30,-75){$\hat{Q}^{(2)}$}
\put(50,-75){$\hat{Q}^{(3)}$}
\put(70,-75){$H_{\mathrm{eff}}^{(3)}$}
\put(90,-75){$\hat{Q}^{(2)}+$fold}
\put(110,-75){$\hat{Q}^{(3)}+$fold}
\put(132,-75){Expt}
\end{picture}
\end{center}
\caption{The low-lying spectra for $^{18}$O with the Bonn A potential.}
\label{fig:rs18oa}
\end{figure}
\begin{figure}[hbtp]
\begin{center}
\setlength{\unitlength}{1.0mm}
\begin{picture}(150,120)(0,-80)
\thicklines
\put(5,-65){\line(0,1){85}}
\put(5,20){\line(1,0){145}}
\put(5,-65){\line(1,0){145}}
\put(150,-65){\line(0,1){85}}
\multiput(5,-60)(0,10){8}{\line(1,0){2}}
\thinlines
\put(0,-61){-6}
\put(0,-51){-5}
\put(0,-41){-4}
\put(0,-31){-3}
\put(0,-21){-2}
\put(0,-11){-1}
\put(1,-1){0}
\put(1,9){1}
\multiput(5,-55)(0,10){8}{\line(1,0){1}}
\put(10,-30){\nl{0}}
\put(10,-19.9){\nl{2}}
\put(10,4.9){\nl{3}}
\put(10,-8.7){\nl{4}}
\put(30,-54.4){\nl{0}}
\put(30,-26.2){\nl{2}}
\put(30,11.5){\nl{3}}
\put(30,-6.1){\nl{4}}
\put(50,-46.7){\nl{0}}
\put(50,-26.7){\nl{2}}
\put(50,13.0){\nl{3}}
\put(50,-7.7){\nl{4}}
\put(70,-37.2){\nl{0}}
\put(70,-21.6){\nl{2}}
\put(70,13.5){\nl{3}}
\put(70,-5.5){\nl{4}}
\put(90,-36.8){\nl{0}}
\put(90,-18.4){\nl{2}}
\put(90,10.2){\nl{3}}
\put(90,-4.1){\nl{4}}
\put(110,-36.1){\nl{0}}
\put(110,-19.5){\nl{2}}
\put(110,11.2){\nl{3}}
\put(110,-5.2){\nl{4}}
\put(130,-39.0){\nl{0}}
\put(130,14.8){\nl{3}}
\put(130,-19.2){\nl{2}}
\put(130,-3.5){\nl{4}}
\put(10,-75){$\hat{Q}^{(1)}$}
\put(30,-75){$\hat{Q}^{(2)}$}
\put(50,-75){$\hat{Q}^{(3)}$}
\put(70,-75){$H_{\mathrm{eff}}^{(3)}$}
\put(90,-75){$\hat{Q}^{(2)}+$fold}
\put(110,-75){$\hat{Q}^{(3)}+$fold}
\put(132,-75){Expt}
\end{picture}
\end{center}
\caption{The low-lying spectra for $^{18}$O with the Bonn B potential.}
\label{fig:rs18ob}
\end{figure}
\begin{figure}[hbtp]
\begin{center}
\setlength{\unitlength}{1.0mm}
\begin{picture}(150,120)(0,-80)
\thicklines
\put(5,-65){\line(0,1){85}}
\put(5,20){\line(1,0){145}}
\put(5,-65){\line(1,0){145}}
\put(150,-65){\line(0,1){85}}
\multiput(5,-60)(0,10){8}{\line(1,0){2}}
\thinlines
\put(0,-61){-6}
\put(0,-51){-5}
\put(0,-41){-4}
\put(0,-31){-3}
\put(0,-21){-2}
\put(0,-11){-1}
\put(1,-1){0}
\put(1,9){1}
\multiput(5,-55)(0,10){8}{\line(1,0){1}}
\put(10,-30.7){\nl{0}}
\put(10,-20.0){\nl{2}}
\put(10,4.9){\nl{3}}
\put(10,-8.7){\nl{4}}
\put(30,-54.2){\nl{0}}
\put(30,-26.0){\nl{2}}
\put(30,11.2){\nl{3}}
\put(30,-6.1){\nl{4}}
\put(50,-47.4){\nl{0}}
\put(50,-26.6){\nl{2}}
\put(50,13.0){\nl{3}}
\put(50,-7.6){\nl{4}}
\put(70,-38.0){\nl{0}}
\put(70,-21.7){\nl{2}}
\put(70,13.0){\nl{3}}
\put(70,-5.5){\nl{4}}
\put(90,-36.4){\nl{0}}
\put(90,-18.1){\nl{2}}
\put(90,10.0){\nl{3}}
\put(90,-4.1){\nl{4}}
\put(110,-36.0){\nl{0}}
\put(110,-19.2){\nl{2}}
\put(110,10.9){\nl{3}}
\put(110,-5.1){\nl{4}}
\put(130,-39.0){\nl{0}}
\put(130,14.8){\nl{3}}
\put(130,-19.2){\nl{2}}
\put(130,-3.5){\nl{4}}
\put(10,-75){$\hat{Q}^{(1)}$}
\put(30,-75){$\hat{Q}^{(2)}$}
\put(50,-75){$\hat{Q}^{(3)}$}
\put(70,-75){$H_{\mathrm{eff}}^{(3)}$}
\put(90,-75){$\hat{Q}^{(2)}+$fold}
\put(110,-75){$\hat{Q}^{(3)}+$fold}
\put(132,-75){Expt}
\end{picture}
\end{center}
\caption{The low-lying spectra for $^{18}$O with the Bonn C potential.}
\label{fig:rs18oc}
\end{figure}
\begin{figure}[hbtp]
\begin{center}
\setlength{\unitlength}{1.0mm}
\begin{picture}(150,150)(0,-110)
\thicklines
\put(5,-95){\line(0,1){125}}
\put(5,30){\line(1,0){145}}
\put(5,-95){\line(1,0){145}}
\put(150,-95){\line(0,1){125}}
\multiput(5,-80)(0,10){11}{\line(1,0){2}}
\thinlines
\put(0,-81){-8}
\put(0,-71){-7}
\put(0,-61){-6}
\put(0,-51){-5}
\put(0,-41){-4}
\put(0,-31){-3}
\put(0,-21){-2}
\put(0,-11){-1}
\put(1,-1){0}
\put(1,9){1}
\put(1,19){2}
\multiput(5,-75)(0,10){10}{\line(1,0){1}}
\put(10,-53.1){\nl{1}}
\put(10,-26.7){\nl{2}}
\put(10,-39.9){\line(1,0){10}\raisebox{-1.0ex}{\hspace{1mm}3}}
\put(10,4.3){\nl{4}}
\put(10,-36.5){\nl{5}}
\put(30,-82){\nl{1}}
\put(30,-34){\nl{2}}
\put(30,-51.5){\nl{3}}
\put(30,1.5){\nl{4}}
\put(30,-39.8){\nl{5}}
\put(50,-89.1){\nl{1}}
\put(50,-34.1){\nl{2}}
\put(50,-58){\nl{3}}
\put(50,-2.6){\nl{4}}
\put(50,-46.0){\nl{5}}
\put(70,-68.7){\nl{1}}
\put(70,-23.6){\nl{2}}
\put(70,-49){\nl{3}}
\put(70,3.7){\nl{4}}
\put(70,-41){\nl{5}}
\put(90,-52.6){\nl{1}}
\put(90,-18.9){\nl{2}}
\put(90,-36.5){\line(1,0){10}\raisebox{-1.0ex}{\hspace{1mm}3}}
\put(90,4.3){\nl{4}}
\put(90,-31.8){\nl{5}}
\put(110,-57.8){\nl{1}}
\put(110,-21.0){\nl{2}}
\put(110,-40.7){\line(1,0){10}\raisebox{-1.0ex}{\hspace{1mm}3}}
\put(110,6.5){\nl{4}}
\put(110,-36.9){\nl{5}}
\put(130,-50){\nl{1}}
\put(130,-40.7){\line(1,0){10}\raisebox{-1.5ex}{\hspace{1mm}3}}
\put(130,-38.8){\nl{5}}
\put(130,-24.8){\nl{2}}
\put(130,3.5){\nl{4}}
\put(10,-105){$\hat{Q}^{(1)}$}
\put(30,-105){$\hat{Q}^{(2)}$}
\put(50,-105){$\hat{Q}^{(3)}$}
\put(70,-105){$H_{\mathrm{eff}}^{(3)}$}
\put(90,-105){$\hat{Q}^{(2)}+$fold}
\put(110,-105){$\hat{Q}^{(3)}+$fold}
\put(132,-105){Expt}
\end{picture}
\end{center}
\caption{The low-lying spectra for $^{18}$F with the Bonn A potential.}
\label{fig:rs18fa}
\end{figure}
\begin{figure}[hbtp]
\begin{center}
\setlength{\unitlength}{1.0mm}
\begin{picture}(150,150)(0,-110)
\thicklines
\put(5,-95){\line(0,1){125}}
\put(5,30){\line(1,0){145}}
\put(5,-95){\line(1,0){145}}
\put(150,-95){\line(0,1){125}}
\multiput(5,-80)(0,10){11}{\line(1,0){2}}
\thinlines
\put(0,-81){-8}
\put(0,-71){-7}
\put(0,-61){-6}
\put(0,-51){-5}
\put(0,-41){-4}
\put(0,-31){-3}
\put(0,-21){-2}
\put(0,-11){-1}
\put(1,-1){0}
\put(1,9){1}
\put(1,19){2}
\multiput(5,-75)(0,10){10}{\line(1,0){1}}
\put(10,-50.4){\nl{1}}
\put(10,-25.3){\nl{2}}
\put(10,-38.6){\line(1,0){10}\raisebox{-1.0ex}{\hspace{1mm}3}}
\put(10,5.4){\nl{4}}
\put(10,-35.6){\nl{5}}
\put(30,-78.2){\nl{1}}
\put(30,-32.3){\nl{2}}
\put(30,-49.9){\nl{3}}
\put(30,2.7){\nl{4}}
\put(30,-38.8){\nl{5}}
\put(50,-84.2){\nl{1}}
\put(50,-31.7){\nl{2}}
\put(50,-55.7){\nl{3}}
\put(50,-1){\nl{4}}
\put(50,-44.6){\nl{5}}
\put(70,-65.1){\nl{1}}
\put(70,-21.8){\nl{2}}
\put(70,-46.8){\nl{3}}
\put(70,5.0){\nl{4}}
\put(70,-39.8){\nl{5}}
\put(90,-48.8){\nl{1}}
\put(90,-17.1){\nl{2}}
\put(90,-34.9){\line(1,0){10}\raisebox{-1.0ex}{\hspace{1mm}3}}
\put(90,1.3){\nl{4}}
\put(90,-30.7){\nl{5}}
\put(110,-52.2){\nl{1}}
\put(110,-18.8){\nl{2}}
\put(110,-38.6){\line(1,0){10}\raisebox{-1.0ex}{\hspace{1mm}3}}
\put(110,8.3){\nl{4}}
\put(110,-35.3){\nl{5}}
\put(130,-50){\nl{1}}
\put(130,-40.7){\line(1,0){10}\raisebox{-1.5ex}{\hspace{1mm}3}}
\put(130,-38.8){\nl{5}}
\put(130,-24.8){\nl{2}}
\put(130,3.5){\nl{4}}
\put(10,-105){$\hat{Q}^{(1)}$}
\put(30,-105){$\hat{Q}^{(2)}$}
\put(50,-105){$\hat{Q}^{(3)}$}
\put(70,-105){$H_{\mathrm{eff}}^{(3)}$}
\put(90,-105){$\hat{Q}^{(2)}+$fold}
\put(110,-105){$\hat{Q}^{(3)}+$fold}
\put(132,-105){Expt}
\end{picture}
\end{center}
\caption{The low-lying spectra for $^{18}$F with the Bonn B potential.}
\label{fig:rs18fb}
\end{figure}
\begin{figure}[hbtp]
\begin{center}
\setlength{\unitlength}{1.0mm}
\begin{picture}(150,150)(0,-110)
\thicklines
\put(5,-95){\line(0,1){125}}
\put(5,30){\line(1,0){145}}
\put(5,-95){\line(1,0){145}}
\put(150,-95){\line(0,1){125}}
\multiput(5,-80)(0,10){11}{\line(1,0){2}}
\thinlines
\put(0,-81){-8}
\put(0,-71){-7}
\put(0,-61){-6}
\put(0,-51){-5}
\put(0,-41){-4}
\put(0,-31){-3}
\put(0,-21){-2}
\put(0,-11){-1}
\put(1,-1){0}
\put(1,9){1}
\put(1,19){2}
\multiput(5,-75)(0,10){10}{\line(1,0){1}}
\put(10,-48.0){\nl{1}}
\put(10,-24.1){\nl{2}}
\put(10,-37.3){\line(1,0){10}\raisebox{-1.0ex}{\hspace{1mm}3}}
\put(10,6.6){\nl{4}}
\put(10,-34.4){\nl{5}}
\put(30,-75.0){\nl{1}}
\put(30,-31.0){\nl{2}}
\put(30,-48.3){\nl{3}}
\put(30,3.9){\nl{4}}
\put(30,-37.6){\nl{5}}
\put(50,-80.1){\nl{1}}
\put(50,-30.1){\nl{2}}
\put(50,-53.5){\nl{3}}
\put(50,0.5){\nl{4}}
\put(50,-43.1){\nl{5}}
\put(70,-62.3){\nl{1}}
\put(70,-20.9){\nl{2}}
\put(70,-45.2){\nl{3}}
\put(70,6.2){\nl{4}}
\put(70,-38.6){\nl{5}}
\put(90,-45.3){\nl{1}}
\put(90,-15.6){\nl{2}}
\put(90,-33.2){\line(1,0){10}\raisebox{-1.0ex}{\hspace{1mm}3}}
\put(90,14.0){\nl{4}}
\put(90,-29.4){\nl{5}}
\put(110,-49.1){\nl{1}}
\put(110,-17){\nl{2}}
\put(110,-36.5){\line(1,0){10}\raisebox{-1.0ex}{\hspace{1mm}3}}
\put(110,10.1){\nl{4}}
\put(110,-33.5){\nl{5}}
\put(130,-50){\nl{1}}
\put(130,-40.7){\line(1,0){10}\raisebox{-1.5ex}{\hspace{1mm}3}}
\put(130,-38.8){\nl{5}}
\put(130,-24.8){\nl{2}}
\put(130,3.5){\nl{4}}
\put(10,-105){$\hat{Q}^{(1)}$}
\put(30,-105){$\hat{Q}^{(2)}$}
\put(50,-105){$\hat{Q}^{(3)}$}
\put(70,-105){$H_{\mathrm{eff}}^{(3)}$}
\put(90,-105){$\hat{Q}^{(2)}+$fold}
\put(110,-105){$\hat{Q}^{(3)}+$fold}
\put(132,-105){Expt}
\end{picture}
\end{center}
\caption{The low-lying spectra for $^{18}$F with the Bonn C potential.}
\label{fig:rs18fc}
\end{figure}



The following is to be noted from these figures:
\begin{itemize}
\item For all potentials and both nuclei, order-by-order
RS perturbation theory does not seem to converge. This conclusion
was first pointed out by Barrett and Kirson \cite{bk70},
and later several workers in the field have obtained the same
conclusion, but with different potentials. Though, the difference
between the results with a second-order and a third-order
\qbox is not as large as the difference between pertubation
theory to first- and second-order in $G$. Thus, one is led
to conclude that the order-by-order convergence, if we
omit any discussion on intruder states, is slow in terms of the
$G$-matrix.
\item For $^{18}$O we observe however that there are small
differences between the results obtained with third-order
perturbation theory, and folded diagrams with either a
second- or a third-order \qbox. This is a very promising
conclusion, and may indicate that the most important higher-order
terms come from folded diagrams. For $^{18}$O we note that
the probably largest contribution to the effective interaction,
stems from folded diagrams of third order in $G$, since the
difference between
$H_{\mathrm{eff}}^{(3)}$ and $\hat{Q}^{(3)}+fold$, is negligible.
\item For $^{18}$F, the contribution from folded diagrams
of order higher than third order, is important, since we find a difference
of approximately 1 MeV between
$H_{\mathrm{eff}}^{(3)}$ and $\hat{Q}^{(3)}+fold$. However, as
was the case with $^{18}$O, the difference between
$\hat{Q}^{(2)}+fold$ and $\hat{Q}^{(3)}+fold$ is small, lending
support to our conclusion that the contributions from folded diagrams
which are important. The fact that high-order folded diagrams are
more important in the $T=0$ than in the $T=1$ can simply be
ascribed to the role of the tensor force.
\item In connection with the tensor force, we note that the potential
with the weakest tensor force, introduces too much binding in the
$T=0$ channel. This has in turn consequences for the
excited spectra as well, yielding a larger separation
between the ground state and the excited states.
Our $T=1$ results show small differences for the various potentials,
and the agreement with the data is rather good. Compared with the
experimental value, we obtain less binding, though this is
desirable since the inclusion of configurations representing
intruder states are expected to bring in an additional binding
of $0.1-0.5$ MeV \cite{ee70}. For $T=0$, the Bonn B and Bonn C
potentials seem to the preferable ones. From our results,
one could therefore feel tempted to infer that the tensor
force of the Bonn A potential is too weak.
However,
the reader should observe that there are several
many-body effects we have not
considered, such as the replacement of a h.o.\ basis with a
Brueckner-Hartree-Fock basis. As demonstrated by Goodin {\em et al.}
\cite{geg77}, the average effect of a self-consistent Hartree-Fock
basis is to reduce the effective $sd$-shell matrix elements by a
factor $0.7$. Then, the Bonn A potential would be the most likely
candidate. It is important to note that
all the Bonn potentials we consider have a weak tensor
force compared with earlier potential models.
\end{itemize}



In connection with the folded-diagram expansion for
$H_{\mathrm{eff}}$, the most important question is
actually whether they converge or not and which factors influence
the convergence. As can be seen from the equation for the folded-diagram
expansion of eq.\ (\ref{eq:fd}), which we rewrite here
\[
    V_{\mathrm{eff}}^{(n)}=\hat{Q} +{\displaystyle\sum_{m=1}^{\infty}}
    \frac{1}{m!}\frac{d^m\hat{Q}}{d\omega^m}\left\{
    V_{\mathrm{eff}}^{(n-1)}\right\}^m,
\]
the
energy dependence of the $\hat{Q}$-box is of central importance.
As discussed in section 4, we approximate
\[
V_{\mathrm{eff}}^{(0)}=\hat{Q}.
\]
Since the $\hat{Q}$-box employed in this work is the sum of all
non-folded linked-valence diagrams through third order, one ought
to find the energy dependence of the $\hat{Q}$-box to be more
pronounced
the higher the order of the diagrams included. This can be seen
from table \ref{tab:qboxder1}, where we list the values of some selected
diagrams through third order.
\begin{table}
\caption{Dependence on the starting energy $\omega$ for various
included in the
$\hat{Q}$-box. We have chosen diagrams with a similar magnitude
at $\omega =-2$ MeV.
All results have been obtained with the Bonn A potential.
The numbering follows figure A.7 in appendix A. Diagram 2-1 is then
the bare $G$-matrix and diagram 2-2 is the core-polarization
diagram, while 2-5 is the third-order diagram.
All entries in MeV.}
\begin{center}
\begin{tabular}{llllllrrrrr}\\\hline
\multicolumn{1}{c}{$JT$}&\multicolumn{1}{c}{$j_{a}$}&
\multicolumn{1}{c}{$j_{b}$}&\multicolumn{1}{c}{$j_{c}$}&
\multicolumn{1}
{c}{$j_{d}$}&\multicolumn{1}{c}{}&\multicolumn{1}{c}
{$\omega =-18$}&\multicolumn{1}{c}{$\omega =-14$}
&\multicolumn{1}{c}{$\omega =-10$}&
\multicolumn{1}{c}{$\omega =-6$}
&\multicolumn{1}{c}{$\omega =-2$}\\ \hline
01&$d_{5/2}$&$d_{5/2}$&$d_{5/2}$&$d_{5/2}$&2-1&-1.687&1.693
&-1.712&-1.726&-1.739\\
&&&&&2-2&-0.821&-0.936&-1.086&-1.286&-1.569\\
&&&&&2-5&-0.470&-0.606&-0.807&-1.122&-1.653\\
\hline
\end{tabular}
\end{center}
\label{tab:qboxder1}
\end{table}
The diagrams listed in table \ref{tab:qboxder1}
are all of comparable size in
absolute value at $\omega =-2$ MeV,
and it is clearly seen that the third-order
diagram (diagram 2-5)  exhibits 
the strongest dependence on energy when one
compares the relative increment for each diagram.
A similar conclusion was reached by Shurpin
{\em et al.} \cite{skd83}. They calculated the folded diagrams using
four approximations to the $\hat{Q}$-box, in the first 
approximation they used with 
all diagrams up to second order in $G$, while the most extensive
approach was to include 
all diagrams to second order in $G$ plus hole-hole and
particle-hole phonons. The latter calculation, involving
{\em selected} higher-order effects, resulted in a much more
pronounced dependence on the energy as compared to the
first approach to the $\hat{Q}$-box.
It was then shown by Shurpin {\em et al.} \cite{skd83} that larger
$\hat{Q}$-box derivatives led to a slower converging FD
expansion. This can easily be understood from the structure of
eq.\ (\ref{eq:fd}).


Therefore, since some individual third-order diagrams exhibit a
stronger dependence
upon the starting energy $\omega$, we may expect larger
$\hat{Q}$-box derivatives, which may suggest that the effects of
folded
diagrams should be larger if third-order diagrams are considered
in the $\hat{Q}$-box.
This conclusion, however, seems to be premature. One must keep
in mind that there are many
(see appendix A) third-order topologies which
contribute to the $\hat{Q}$-box. Many of these third-order
topologies have been found to be quite large in absolute value
\cite{hom92,bk70}. Therefore one should not extrapolate
from
the energy
dependence of some individual diagrams to that of all terms.
Actually, in ref.\ \cite{hom92}
we found several of the third-order
$\hat{Q}$-box derivatives to be smaller than those obtained with
a second-order $\hat{Q}$-box. Especially, the derivatives of the
diagonal matrix elements tend to be smaller with a third-order
$\hat{Q}$-box. This is mainly due to the contribution
from the one-body-diagrams with  a passive valence line.
As discussed in ref.\ \cite{hom92},
some {\em total} third-order contributions to the $\hat Q$-box are
repulsive. Furthermore, they tend to increase with energy, and hence
result in a derivative of the third-order contribution which is positive.
The derivatives of the $\hat{Q}$-boxes including all diagrams through
third order tend therefore
to be smaller than the corresponding derivatives of
the $\hat{Q}$-boxes
to second order in the interaction. This
should improve
the convergence of the folded diagram expansion if all terms
through third order
are included in the $\hat Q$-box as compared to a calculation
which only
accounts for terms through second order.
Another interesting feature observed in ref.\ \cite{hom92}, is that the
higher derivatives become small, this means in turn that, if the \qbox is
small (of the order some few MeV), the series in eq.\ (\ref{eq:fd}) can
be terminated after the inclusion of some few higher-order terms, since
the derivatives have to be divided by their respective factorials.
These observations apply to this work as well.
In most calculations, we find the first derivative to be in the range
$0.01-0.3$ MeV. This is the case in  the other mass areas studied
in this work. The fourth derivative multiplied with its factorial
is on the average of  the order $10^{-7}-10^{-6}$, and we can in most cases
terminate our expansion in eq.\ (\ref{eq:fd}) after  $m\approx 4-5$.
The effective interaction
$V_{\mathrm{eff}}^{(1)}$
can then in turn be used to calculate $V_{\mathrm{eff}}^{(2)}$ and so forth.
In all our calculations we truncate the sum over $m$ at $m=10$, which means
that we need to evaluate the \qbox at 11 starting energies in
order to calculate
derivatives up till the tenth order derivative.
In this sense we differ from ref.\ \cite{hom92}. There only four
iterations were considered, and the FD method did not  converge properly.
Thus, it was important to have a \qbox which yielded a weaker
dependence on the starting energy $\omega$. In our calculations
however, the difference in $\omega$ dependence of a \qbox
of second or third order in $G$, is not so important, since the
results stabilize with $n \approx 6-10$. 
We show in fig.\ \ref{fig:fdconv}
the reliability of our approximation. We see that after four or
five  iterations
the eigenvalues for the low-lying states in $^{18}$O are close
to the result after ten iterations in  eq.\ (\ref{eq:fd}). The
lowest $JT=01$ state
and the lowest $JT=10$ state (not shown in this figure) require
more iterations than the other states in order to arrive at a converged
result for the effective matrix elements with an accuracy
of $0.0001$ MeV. However, the convergence is in general rather
rapid for all effective interactions investigated in this work.
\begin{figure}[hbtp]
\begin{center}
\setlength{\unitlength}{1.0mm}
\begin{picture}(150,120)(0,-80)
\thicklines
\put(5,-65){\line(0,1){85}}
\put(5,20){\line(1,0){145}}
\put(5,-65){\line(1,0){145}}
\put(150,-65){\line(0,1){85}}
\multiput(5,-60)(0,10){8}{\line(1,0){2}}
\thinlines
\put(0,-61){-6}
\put(0,-51){-5}
\put(0,-41){-4}
\put(0,-31){-3}
\put(0,-21){-2}
\put(0,-11){-1}
\put(1,-1){0}
\put(1,9){1}
\multiput(5,-55)(0,10){8}{\line(1,0){1}}
\put(10,-47.4){\nl{0}}
\put(10,-27.1){\nl{2}}
\put(10,13.3){\nl{3}}
\put(10,-7.7){\nl{4}}
\put(30,-35.45){\nl{0}}
\put(30,-18.67){\nl{2}}
\put(30,11.50){\nl{3}}
\put(30,-5.0){\nl{4}}
\put(50,-37.63){\nl{0}}
\put(50,-20.55){\nl{2}}
\put(50,12.04){\nl{3}}
\put(50,-5.42){\nl{4}}
\put(70,-37.56){\nl{0}}
\put(70,-20.20){\nl{2}}
\put(70,11.93){\nl{3}}
\put(70,-5.36){\nl{4}}
\put(90,-37.58){\nl{0}}
\put(90,-20.18){\nl{2}}
\put(90,11.92){\nl{3}}
\put(90,-5.36){\nl{4}}
\put(110,-37.58){\nl{0}}
\put(110,-20.18){\nl{2}}
\put(110,11.92){\nl{3}}
\put(110,-5.36){\nl{4}}
\put(130,-37.58){\nl{0}}
\put(130,-20.18){\nl{2}}
\put(130,11.92){\nl{3}}
\put(130,-5.36){\nl{4}}
\put(10,-75){$n=0$}
\put(30,-75){$n=1$}
\put(50,-75){$n=4$}
\put(70,-75){$n=5$}
\put(90,-75){$n=8$}
\put(110,-75){$n=9$}
\put(130,-75){$n=10$}
\end{picture}
\end{center}
\caption{The low-lying spectra for $^{18}$O with the Bonn A potential as
function of the number of iterations $n$ needed in the FD
expansion. The results have been obtained with a \qbox of third order
in the interaction $G$. The label $n=0$ represents 
the results with the \qbox
only. All entries in MeV.}
\label{fig:fdconv}
\end{figure}

It also worth noting that for values
of $\omega$ close to the chosen starting energy, the \qbox
shows only a weak dependence upon the starting energy.
This behavior of
the \qbox is important, since we will obtain an effective interaction
and energy eigenvalues which depend only weakly on the chosen
$\omega$, see e.g.\ figs.\ 2 and 3 in ref.\ \cite{eehho94}.


In summary, we have seen that the potential models
of the Bonn group are successful in describing the property
of $^{18}$O ans $^{18}$F. Although the results of an FD
calculation with either a second- or third-order \qbox are
rather similar, we will henceforth employ the third-order
results in the calculations of nuclei with more than two
valence nucleons. This is done in order to be consistent with
the fact that third-order terms in the \qbox are not negligible.
Finally,
before
we can conclude on which potential model one should prefer,
we need to test our effective interactions by applying
them in the calculation of properties of nuclei with more than
two valence nucleons and to other mass areas as well.



\subsubsection{Results for nuclei with more than two valence nucleons}

We end this subsection by computing the spectra for nuclei with more than
two valence nucleons. The FD results with a \qbox of third order in the
$G$-matrix
has been used as the effective interaction. We display the results
for $^{19}$F, $^{20}$O, $^{20}$Ne and $^{22}$Ne as functions
of the Bonn A, B and C potentials in figs.\
\ref{fig:19f}-\ref{fig:ne22}. We only include the lowest-lying states
for each angular momentum, and avoid thereby any discussion
on possible intruder configurations. For a review of the latter,
see e.g.\ the text of Lawson \cite{law80}.
%    19f
\begin{figure}[hbtp]
\setlength{\unitlength}{1.0mm}
\begin{center}
\begin{picture}(140,110)(0,-30)
\thicklines
\put(5,-10){\line(0,1){85}}
\put(5,75){\line(1,0){115}}
\put(5,-10){\line(1,0){115}}
\put(120,-10){\line(0,1){85}}
\multiput(5,0)(0,10){8}{\line(1,0){2}}
\thinlines
\put(1,0){0}
\put(1,9){1}
\put(1,19){2}
\put(1,29){3}
\put(1,39){4}
\put(1,49){5}
\put(1,59){6}
\put(1,69){7}
\multiput(5,5)(0,10){8}{\line(1,0){1}}
\put(10,0){\nl{1}}
\put(10,2.5){\line(1,0){10}\raisebox{0.2ex}{\hspace{1mm}5}}
\put(10,8.5){\nl{3}}
\put(10,29.9){\nl{9}}
\put(10,50.4){\line(1,0){10}\raisebox{-2.2ex}{\hspace{1mm}7}}
\put(10,50.8){\nl{13}}
\put(10,51.7){\line(1,0){10}\raisebox{1.0ex}{\hspace{1mm}5}}
\put(40,0){\nl{1}}
\put(40,2.8){\line(1,0){10}\raisebox{0.2ex}{\hspace{1mm}5}}
\put(40,9.3){\nl{3}}
\put(40,27.8){\nl{9}}
\put(40,48.5){\line(1,0){10}\raisebox{-2.2ex}{\hspace{1mm}13}}
\put(40,48.7){\nl{7}}
\put(40,49.7){\line(1,0){10}\raisebox{1.0ex}{\hspace{1mm}5}}
\put(70,0){\nl{1}}
\put(70,2.7){\line(1,0){10}\raisebox{0.2ex}{\hspace{1mm}5}}
\put(70,9.8){\nl{3}}
\put(70,26.6){\nl{9}}
\put(70,47){\line(1,0){10}\raisebox{-2.0ex}{\hspace{1mm}13}}
\put(70,47.6){\nl{7}}
\put(70,48){\line(1,0){10}\raisebox{1.0ex}{\hspace{1mm}5}}
\put(100,0){\nl{1}}
\put(100,2.0){\line(1,0){10}\raisebox{0.2ex}{\hspace{1mm}5}}
\put(100,15.5){\nl{3}}
\put(100,27.8){\nl{9}}
\put(100,46.5){\line(1,0){10}\raisebox{0.8ex}{\hspace{1mm}13}}
\put(100,43.8){\line(1,0){10}\raisebox{-1.5ex}{\hspace{1mm}7}}
\put(100,45.5){\nl{5}}
\put(9,-7){\small{(--24.44)}}
\put(39,-7){\small{(--23.87)}}
\put(69,-7){\small{(--23.40)}}
\put(99,-7){\small{(--23.7)}}
\put(15,-20){A}
\put(45,-20){B}
\put(75,-20){C}
\put(101,-20){Expt}
\end{picture}
\end{center}
\caption{The low-lying  $^{19}$F spectra relative to $^{16}$O.
The labels A, B, and C refer to the  various versions of the Bonn potential
discussed in the text. Twice the total
angular momentum is used.
All energies in MeV.}
\label{fig:19f}
\end{figure}
%    ox-20
\begin{figure}[hbtp]
\setlength{\unitlength}{1.0mm}
\begin{center}
\begin{picture}(140,80)(0,-30)
\thicklines
\put(5,-10){\line(0,1){60}}
\put(5,50){\line(1,0){115}}
\put(5,-10){\line(1,0){115}}
\put(120,-10){\line(0,1){60}}
\multiput(5,0)(0,10){4}{\line(1,0){2}}
\thinlines
\put(1,0){0}
\put(1,9){1}
\put(1,19){2}
\put(1,29){3}
\multiput(5,5)(0,10){4}{\line(1,0){1}}
\put(10,0){\nl{0}}
\put(10,16){\nl{2}}
\put(10,40.1){\nl{4}}
\put(10,32.5){\nl{2}}
\put(10,35.6){\nl{0}}
\put(40,0){\nl{0}}
\put(40,15.3){\nl{2}}
\put(40,38.44){\nl{4}}
\put(40,31.4){\nl{2}}
\put(40,35){\nl{0}}
\put(70,0){\nl{0}}
\put(70,15.3){\nl{2}}
\put(70,37.7){\nl{4}}
\put(70,30.9){\nl{2}}
\put(70,34.5){\nl{0}}
\put(100,0){\nl{0}}
\put(100,16.7){\nl{2}}
\put(100,35.7){\nl{4}}
\put(100,40.7){\nl{2}}
\put(100,44.5){\nl{0}}
\put(9,-7){\small{(--24.40)}}
\put(39,-7){\small{(--24.22)}}
\put(69,-7){\small{(--24.22)}}
\put(99,-7){\small{(--23.75)}}
\put(15,-20){A}
\put(45,-20){B}
\put(75,-20){C}
\put(101,-20){Expt}
\end{picture}
\end{center}
\caption{The low-lying $^{20}$O spectra relative to $^{16}$O.}
\label{fig:ox20}
\end{figure}
%    ne-20
\begin{figure}[hbtp]
\setlength{\unitlength}{1.0mm}
\begin{center}
\begin{picture}(140,135)(0,-30)
\thicklines
\put(5,-10){\line(0,1){110}}
\put(5,100){\line(1,0){115}}
\put(5,-10){\line(1,0){115}}
\put(120,-10){\line(0,1){110}}
\multiput(5,0)(0,10){10}{\line(1,0){2}}
\thinlines
\put(1,0){0}
\put(1,9){1}
\put(1,19){2}
\put(1,29){3}
\put(1,39){4}
\put(1,49){5}
\put(1,59){6}
\put(1,69){7}
\put(1,79){8}
\put(1,89){9}
\multiput(5,5)(0,10){10}{\line(1,0){1}}
\put(10,0){\nl{0}}
\put(10,17.1){\nl{2}}
\put(10,39.5){\nl{4}}
\put(10,85.2){\nl{6}}
\put(40,0){\nl{0}}
\put(40,15.6){\nl{2}}
\put(40,37.3){\nl{4}}
\put(40,80.4){\nl{6}}
\put(70,0){\nl{0}}
\put(70,14.8){\nl{2}}
\put(70,36.1){\nl{4}}
\put(70,77.5){\nl{6}}
\put(100,0){\nl{0}}
\put(100,16.3){\nl{2}}
\put(100,42.5){\nl{4}}
\put(100,87.8){\nl{6}}
\put(9,-7){\small{(--41.81)}}
\put(39,-7){\small{(--40.62)}}
\put(69,-7){\small{(--39.65)}}
\put(99,-7){\small{(--40.70)}}
\put(15,-20){A}
\put(45,-20){B}
\put(75,-20){C}
\put(101,-20){Expt}
\end{picture}
\end{center}
\caption{The low-lying  $^{20}$Ne spectra relative to $^{16}$O.}
\label{fig:ne20}
\end{figure}
%    ne-22
\begin{figure}[hbtp]
\setlength{\unitlength}{1.0mm}
\begin{center}
\begin{picture}(140,100)(0,-30)
\thicklines
\put(5,-10){\line(0,1){80}}
\put(5,70){\line(1,0){115}}
\put(5,-10){\line(1,0){115}}
\put(120,-10){\line(0,1){80}}
\multiput(5,0)(0,10){7}{\line(1,0){2}}
\thinlines
\put(1,0){0}
\put(1,9){1}
\put(1,19){2}
\put(1,29){3}
\put(1,39){4}
\put(1,49){5}
\put(1,59){6}
\multiput(5,5)(0,10){6}{\line(1,0){1}}
\put(10,0){\nl{0}}
\put(10,12.7){\nl{2}}
\put(10,32.3){\nl{4}}
\put(10,64){\nl{6}}
\put(40,0){\nl{0}}
\put(40,11.9){\nl{2}}
\put(40,31.0){\nl{4}}
\put(40,62.1){\nl{6}}
\put(70,0){\nl{0}}
\put(70,11.5){\nl{2}}
\put(70,30.5){\nl{4}}
\put(70,61){\nl{6}}
\put(100,0){\nl{0}}
\put(100,12.7){\nl{2}}
\put(100,33.6){\nl{4}}
\put(100,63.1){\nl{6}}
\put(9,-7){\small{(--60.46)}}
\put(39,-7){\small{(--58.96)}}
\put(69,-7){\small{(--57.80)}}
\put(99,-7){\small{(--57.75)}}
\put(15,-20){A}
\put(45,-20){B}
\put(75,-20){C}
\put(101,-20){Expt}
\end{picture}
\end{center}
\caption{The low-lying  $^{22}$Ne spectra relative to $^{16}$O.}
\label{fig:ne22}
\end{figure}

For $^{19}$F we see that all potentials give, except for the
first $\frac{3}{2}^+$ state, a very  good agreement with the data.
Also, as compared with earlier calculations such as those
of Shurpin {\em et al.} \cite{skd83}, 
all potentials result in a binding
energy for the ground state which is close to the experimental
value. We note however that both the Bonn A and Bonn B potentials
introduce too much binding, a fact which can be ascribed to the
strong attraction in the $T=0$ matrix elements. However, here one has
to consider effective three-body forces as well. As shown by Polls
{\em et al.} \cite{pmfko82}, the effect of three-body diagrams
is small for systems with three valence nucleons only, though
the net result would be to bring our results with the Bonn A and
B potentials closer to the experimental binding energy.
For the two Ne isotopes in figs.\ \ref{fig:ne20} and \ref{fig:ne22},
the agreement with the excited spectra is very good with the Bonn A
potential, although this potential yields too much binding. 
Again, however,
we would expect effective three-body forces to bring the ground state
closer to the experimental value. For the Bonn B and C potentials, which
have stronger tensor forces, we obtain less attraction and a more
compressed spectrum. We will observe this feature in the other
mass areas as well, i.e.\ the potential which has the
weakest tensor force, yields a more bound ground state and a less
compressed excited spectrum compared with potentials with a
stronger tensor force.

Another point which should be noted from figs.\ \ref{fig:ne20}
and \ref{fig:ne22}, is that the $4^+$ and $6^+$ states
in $^{20}$Ne are slightly below the experimental value for all potentials,
whereas in $^{22}$Ne, the agreement with the data is excellent.

Our effective interactions have also been applied to other $sd$-shell
nuclei,
with rather good results, except for odd-odd nuclei like $^{22}$Na, where
all realistic effective interactions
fail to reproduce
the order between the $1^+$ and $3^+$ states. In our calculations
however, the $1^+$ state is only some few keV below the $3^+$ state.

In summary, we have seen that our effective interactions do
fairly well in reproducing the properties of several nuclei in the
$sd$-shell. Although the Bonn A potential introduces more attraction
than the Bonn B and C potentials, all these potentials yield a much
better reproduction of both the binding energy and the excited
spectra than previous calculations with potentials with strong
tensor forces, see e.g.\ \cite{skd83}. Also, there is room for
both introduction of three-body forces and intruder state
configurations. We have however not tested the quality of our
wave functions in the evaluation of transition operators.
As mentioned in the introductory section, we will defer from this
here.

The effective interactions are listed in table C.1, together with
the phenomenological 
interaction of Brown and Wildenthal \cite{brown88}.

\subsection{Results for nuclei in the mass region of
$^{40}$Ca}

Recently, several systematic shell-model calculations
have been presented for nuclei in the $pf$-shell, to mention
a few, we note  the work of Richter {\em et al.} \cite{richt91}, the
compilations of the Strasbourg-Madrid groups \cite{pz80,czpm94}
or refs.\ \cite{heho94,heho92}.
The model space has usually been chosen to be represented
by the degrees of freedom of the  $pf$-shell. In ref.\
\cite{richt91}, a phenomenological effective interaction is
constructed through a fit to the available data, whereas in
the work of Zuker and Poves and co-workers \cite{pz80,czpm94},
the philosophy has been to adjust realistic effective interactions
so that the modified interaction reproduces the spectra
of different nuclei. Especially, in refs.\ \cite{pz80,czpm94},
monopole terms of the Kuo-Brown interaction in the
$pf$-shell were modified by adding constant shifts to make
the interaction spectroscopically viable.

Our philosophy is however to start with the free NN potential,
which is determined by the various meson parameters, and
use this interaction in a perturbative scheme. If the final
effective interaction does not reproduce various
properties of selected nuclei, this deficiency may be ascribed
to the potential itself, or the chosen many-body approach,
or the chosen model space. From our calculations in the
$sd$-shell we feel that both the potential and the many-body
formalism is reliable. If we then observe discrepancies
between the calculated spectra and the data, a possible
improvement is to extend the model space.
As we will show in this subsection,
a model
space given by the $pf$-shell
is sufficient to reproduce the data, although our agreement
for the $A=42$ nuclei is poor.

The single-particle energies of $^{41}$Ca are \cite{kb68}
$\varepsilon_{f_{5/2}}-\varepsilon_{f_{7/2}}=6.5$ MeV,
$\varepsilon_{p_{3/2}}-\varepsilon_{f_{7/2}}=2.1$ MeV and
$\varepsilon_{p_{1/2}}-\varepsilon_{f_{7/2}}=3.9$ MeV.
The unperturbed energy at which
we evaluate the $\hat{Q}$-box was set to $-16$ MeV,
approximately twice the energy
of the $f_{7/2}$ state. The $G$-matrix was defined  with
$n_1=6, n_2=15, n_3 =66$ and an oscillator parameter
$b=2.02$ fm.
The Bonn
potentials
we use here are the same as those employed for $sd$-shell nuclei.

Again we note
that the difference between results obtained with the folded
diagram summations with either a second- or third-order \qbox,
are rather similar, see the results
labelled $\hat{Q}^{(2)}+fold$ and $\hat{Q}^{(3)}+fold$ in the preceeding
subsection.
For the $T=1$ channel there is barely any difference. We note also,
as was the case in the $sd$-shell, that the difference between
the results obtained with a \qbox of second order in $G$ and the
third-order results without folded diagrams, is not so large.
Actually, the average third-order contributions
in the $pf$-shell are slightly smaller than
those in the $sd$-shell.
This is an interesting property, since it is not a priori clear
whether third-order contributions may be large or small in the
$pf$-shell. For the $pf$-shell effective interaction, there are
two mechanisms which may increase the contribution from a given
third-order diagram. These are the smaller energy denominators
due to a smaller oscillator energy and the increased
number of intermediate states which contribute. The mechanism
which counterbalances these effects is the decrease in absolute
value of the $G$-matrix, since the oscillator energy is smaller
here, i.e.\ $11$ MeV in the $pf$-shell and $14$ MeV in the
$sd$-shell. These conclusions apply to the nuclei in the
tin mass areas as well. However, third-order
terms are still not negligible, and in order to be
consistent we will, as in the $sd$-shell, use effective
interactions obtained with a \qbox through third order in the
interaction.
We show our results obtained with a third-order \qbox and summing
the folded diagrams to arbitrary order in table \ref{tab:casc}.
\begin{table}[hbtp]
\caption{Eigenvalues for the $T=1$ and $T=0$ $(pf)^2$ systems
obtained with
the Bonn A, B and C potentials using a third-order \qbox
and the FD method.
All entries in MeV.}
\begin{center}
\begin{tabular}{crrrr}
\\\hline
$JT$&
\multicolumn{1}{c}{A}&
\multicolumn{1}{c}{B}&
\multicolumn{1}{c}{C}&
\multicolumn{1}{c}{Expt}
\\\hline
$10$&-3.44&-2.99&-2.60&-2.57\\
$30$&-1.53&-1.36&-1.20&-1.69\\
$50$&-1.43&-1.32&-1.21&-1.67\\
$70$&-2.47&-2.36&-2.24&-2.56\\
$01$&-2.69&-2.58&-2.56&-3.12\\
$21$&-1.17&-1.13&-1.12&-1.60\\
$41$&-0.28&-0.28&-0.28&-0.37\\
$61$&0.07&0.06&0.04 &0.07\\
\hline
\end{tabular}
\end{center}
\label{tab:casc}
\end{table}
The agreement with the data is rather poor. For all three 
potentials, the
$JT=10$ state is below the $JT=01$ state. The latter state
differs $400-600$ keV from the experimental value, however, we
would expect 
that the inclusion of intruder state configurations
will bring the calculated ground state closer to the experimental
value and account for a correct
ordering between the lowest $JT=01$ and $JT=10$ states, since the
corrections from intruder state configurations are stronger
in the 
$T=1$ channel than in the $T=0$ channel.
This conclusion was also made in refs.\
\cite{heho94,heho92}. 
Since intruder state configurations are important near the 
closed shell nucleus, we will apply
the effective interactions to calculate the spectra of 
nuclei with more than two valence nucleons, here
$^{44}$Ca, $^{44}$Sc, $^{44}$Ti, $^{45}$Ca, $^{46}$Ca and $^{48}$Ca.
These are shown in figs.\
\ref{fig:ca44}-\ref{fig:ca48}.
A comparison with the experimental data from \cite{endt90}
including the Coulomb corrections of Richter {\em et al.}
\cite{richt91} is also added.
%    ca-44
\begin{figure}[hbtp]
\setlength{\unitlength}{1.0mm}
\begin{center}
\begin{picture}(140,70)(0,-30)
\thicklines
\put(5,-10){\line(0,1){50}}
\put(5,40){\line(1,0){115}}
\put(5,-10){\line(1,0){115}}
\put(120,-10){\line(0,1){50}}
\multiput(5,0)(0,10){4}{\line(1,0){2}}
\thinlines
\put(1,0){0}
\put(1,9){1}
\put(1,19){2}
\put(1,29){3}
\multiput(5,5)(0,10){4}{\line(1,0){1}}
\put(10,0){\nl{0}}
\put(10,13.2){\nl{2}}
\put(10,22){\nl{4}}
\put(10,30.8){\nl{6}}
\put(40,0){\nl{0}}
\put(40,12.7){\nl{2}}
\put(40,21.0){\nl{4}}
\put(40,29){\nl{6}}
\put(70,0){\nl{0}}
\put(70,12.7){\nl{2}}
\put(70,20.9){\nl{4}}
\put(70,28.2){\nl{6}}
\put(100,0){\nl{0}}
\put(100,11.57){\nl{2}}
\put(100,22.83){\nl{4}}
\put(100,32.85){\nl{6}}
\put(9,-7){\small{(--38.44)}}
\put(39,-7){\small{(--38.29)}}
\put(69,-7){\small{(--38.27)}}
\put(99,-7){\small{(--38.91)}}
\put(15,-20){A}
\put(45,-20){B}
\put(75,-20){C}
\put(101,-20){Expt}
\end{picture}
\end{center}
\caption{The low-lying $^{44}$Ca spectra relative to $^{40}$Ca.}
\label{fig:ca44}
\end{figure}
%    sc-44
\begin{figure}[hbtp]
\setlength{\unitlength}{1.0mm}
\begin{center}
\begin{picture}(140,95)(0,-30)
\thicklines
\put(5,-10){\line(0,1){75}}
\put(5,65){\line(1,0){115}}
\put(5,-10){\line(1,0){115}}
\put(120,-10){\line(0,1){75}}
\multiput(5,0)(0,10){6}{\line(1,0){2}}
\thinlines
\put(1,0){0}
\put(1,9){1}
\put(1,19){2}
\put(1,29){3}
\put(1,39){4}
\put(1,49){5}
\multiput(5,5)(0,10){6}{\line(1,0){1}}
\put(10,0){\line(1,0){10}\raisebox{-2.0ex}{\hspace{1mm}2}}
\put(10,7.2){\line(1,0){10}}
\put(10,6.0){\nl{4,6}}
\put(10,3.7){\line(1,0){10}\raisebox{-1.0ex}{\hspace{1mm}1}}
\put(10,8.6){\line(1,0){10}\raisebox{0.2ex}{\hspace{1mm}3}}
\put(10,16.2){\nl{7}}
\put(10,13){\nl{5}}
\put(10,34.3){\nl{9}}
\put(10,44.5){\nl{11}}
\put(10,51.0){\nl{10}}
\put(40,0){\nl{2}}
\put(40,6.2){\line(1,0){10}}
\put(40,5.4){\line(1,0){10}\raisebox{-1.2ex}{\hspace{1mm}1,4,6}}
\put(40,4.3){\line(1,0){10}}
\put(40,8){\nl{3}}
\put(40,14.4){\line(1,0){10}\raisebox{0.2ex}{\hspace{1mm}7}}
\put(40,11.99){\nl{5}}
\put(40,31.1){\nl{9}}
\put(40,40.1){\nl{11}}
\put(40,47){\nl{10}}
\put(70,0){\nl{2}}
\put(70,5.4){\line(1,0){10}}
\put(70,4.9){\nl{4,1,6}}
\put(70,7.6){\line(1,0){10}\raisebox{-0.4ex}{\hspace{1mm}3}}
\put(70,13){\line(1,0){10}\raisebox{0.2ex}{\hspace{1mm}7}}
\put(70,11.2){\nl{5}}
\put(70,29){\nl{9}}
\put(70,38.1){\nl{11}}
\put(70,44.4){\nl{10}}
\put(100,0){\nl{2}}
\put(100,2.7){\line(1,0){10}}
\put(100,3.5){\nl{6,4}}
\put(100,6.7){\line(1,0){10}}
\put(100,7.6){\nl{1,3}}
\put(100,9.5){\line(1,0){10}}
\put(100,10.5){\line(1,0){10}\raisebox{0.2ex}{\hspace{1mm}7,5}}
\put(100,26.72){\nl{9}}
\put(100,35.67){\nl{11}}
\put(100,41.1){\nl{10}}
\put(9,-7){\small{(--42.39)}}
\put(39,-7){\small{(--41.85)}}
\put(69,-7){\small{(--41.42)}}
\put(99,-7){\small{(--41.70)}}
\put(15,-20){A}
\put(45,-20){B}
\put(75,-20){C}
\put(101,-20){Expt}
\end{picture}
\end{center}
\caption{The low-lying  $^{44}$Sc spectra relative to $^{40}$Ca.
For the cases with more than one angular momentum value per level,
the first angular momentum value, corresponds to the
lowest lying level.}
\label{fig:sc44}
\end{figure}
%    ti-44
\begin{figure}[hbtp]
\setlength{\unitlength}{1.0mm}
\begin{center}
\begin{picture}(140,115)(0,-30)
\thicklines
\put(5,-10){\line(0,1){95}}
\put(5,85){\line(1,0){115}}
\put(5,-10){\line(1,0){115}}
\put(120,-10){\line(0,1){95}}
\multiput(5,0)(0,10){9}{\line(1,0){2}}
\thinlines
\put(1,0){0}
\put(1,9){1}
\put(1,19){2}
\put(1,29){3}
\put(1,39){4}
\put(1,49){5}
\put(1,59){6}
\put(1,69){7}
\put(1,79){8}
\multiput(5,5)(0,10){8}{\line(1,0){1}}
\put(10,0){\nl{0}}
\put(10,13.7){\nl{2}}
\put(10,24.8){\nl{4}}
\put(10,34){\nl{6}}
\put(10,65.2){\nl{8}}
\put(10,81.3){\nl{10}}
\put(40,0){\nl{0}}
\put(40,12.5){\nl{2}}
\put(40,23.0){\nl{4}}
\put(40,32){\nl{6}}
\put(40,60.3){\nl{8}}
\put(40,75.2){\nl{10}}
\put(70,0){\nl{0}}
\put(70,11.8){\nl{2}}
\put(70,21.94){\nl{4}}
\put(70,30.78){\nl{6}}
\put(70,57.1){\nl{8}}
\put(70,71.2){\nl{10}}
\put(100,0){\nl{0}}
\put(100,10.8){\nl{2}}
\put(100,24.54){\nl{4}}
\put(100,40.15){\nl{6}}
\put(100,65.09){\nl{8}}
\put(100,76.71){\nl{10}}
\put(9,-7){\small{(--48.76)}}
\put(39,-7){\small{(--47.79)}}
\put(69,-7){\small{(--47.03)}}
\put(99,-7){\small{(--48.24)}}
\put(15,-20){A}
\put(45,-20){B}
\put(75,-20){C}
\put(101,-20){Expt}
\end{picture}
\end{center}
\caption{The low-lying  $^{44}$Ti spectra relative to $^{40}$Ca.}
\label{fig:ti44}
\end{figure}
%    ca-45
\begin{figure}[hbtp]
\setlength{\unitlength}{1.0mm}
\begin{center}
\begin{picture}(140,55)(0,-30)
\thicklines
\put(5,-10){\line(0,1){35}}
\put(5,25){\line(1,0){115}}
\put(5,-10){\line(1,0){115}}
\put(120,-10){\line(0,1){35}}
\multiput(5,0)(0,10){3}{\line(1,0){2}}
\thinlines
\put(1,0){0}
\put(1,9){1}
\put(1,19){2}
\multiput(5,5)(0,10){4}{\line(1,0){1}}
\put(10,0){\line(1,0){10}\raisebox{-1.5ex}{\hspace{1mm}7}}
\put(10,1.4){\line(1,0){10}\raisebox{1.0ex}{\hspace{1mm}5}}
\put(10,14.4){\line(1,0){10}\raisebox{-2.0ex}{\hspace{1mm}3}}
\put(10,15.3){\nl{11}}
\put(10,16){\line(1,0){10}\raisebox{1.0ex}{\hspace{1mm}9}}
\put(40,0){\nl{7}}
\put(40,1.5){\line(1,0){10}\raisebox{0.8ex}{\hspace{1mm}5}}
\put(40,14.0){\line(1,0){10}\raisebox{-2.0ex}{\hspace{1mm}3}}
\put(40,14.6){\nl{11}}
\put(40,15.4){\line(1,0){10}\raisebox{1.2ex}{\hspace{1mm}9}}
\put(70,0){\nl{7}}
\put(70,1.9){\line(1,0){10}\raisebox{1.0ex}{\hspace{1mm}5}}
\put(70,13.9){\line(1,0){10}\raisebox{-2.0ex}{\hspace{1mm}3}}
\put(70,14.3){\nl{11}}
\put(70,15.34){\line(1,0){10}\raisebox{1.2ex}{\hspace{1mm}9}}
\put(100,0){\nl{7}}
\put(100,1.7){\line(1,0){10}\raisebox{0.8ex}{\hspace{1mm}5}}
\put(100,14.4){\line(1,0){10}\raisebox{-2.0ex}{\hspace{1mm}3}}
\put(100,15.6){\nl{11}}
\put(100,19){\line(1,0){10}\raisebox{0.5ex}{\hspace{1mm}9}}
\put(9,-7){\small{(--46.03)}}
\put(39,-7){\small{(--46.00)}}
\put(69,-7){\small{(--46.06)}}
\put(99,-7){\small{(--46.32)}}
\put(15,-20){A}
\put(45,-20){B}
\put(75,-20){C}
\put(101,-20){Expt}
\end{picture}
\end{center}
\caption{The low-lying  $^{45}$Ca spectra relative to $^{40}$Ca. Twice
the angular momentum is used. }
\label{fig:ca45}
\end{figure}
%    ca-46
\begin{figure}[hbtp]
\setlength{\unitlength}{1.0mm}
\begin{center}
\begin{picture}(140,70)(0,-30)
\thicklines
\put(5,-10){\line(0,1){50}}
\put(5,40){\line(1,0){115}}
\put(5,-10){\line(1,0){115}}
\put(120,-10){\line(0,1){50}}
\multiput(5,0)(0,10){4}{\line(1,0){2}}
\thinlines
\put(1,0){0}
\put(1,9){1}
\put(1,19){2}
\put(1,29){3}
\multiput(5,5)(0,10){4}{\line(1,0){1}}
\put(10,0){\nl{0}}
\put(10,12.8){\nl{2}}
\put(10,27.8){\nl{4}}
\put(10,32.9){\nl{6}}
\put(40,0){\nl{0}}
\put(40,12.3){\nl{2}}
\put(40,26.4){\nl{4}}
\put(40,30.7){\nl{6}}
\put(70,0){\nl{0}}
\put(70,12.4){\nl{2}}
\put(70,25.76){\nl{4}}
\put(70,29.6){\nl{6}}
\put(100,0){\nl{0}}
\put(100,13.46){\nl{2}}
\put(100,25.74){\nl{4}}
\put(100,29.73){\nl{6}}
\put(9,-7){\small{(--56.45)}}
\put(39,-7){\small{(--56.38)}}
\put(69,-7){\small{(--56.47)}}
\put(99,-7){\small{(--56.72)}}
\put(15,-20){A}
\put(45,-20){B}
\put(75,-20){C}
\put(101,-20){Expt}
\end{picture}
\end{center}
\caption{The low-lying $^{46}$Ca spectra relative to $^{40}$Ca.}
\label{fig:ca46}
\end{figure}
%    ca-48
\begin{figure}[hbtp]
\setlength{\unitlength}{1.0mm}
\begin{center}
\begin{picture}(140,80)(0,-30)
\thicklines
\put(5,-10){\line(0,1){60}}
\put(5,50){\line(1,0){115}}
\put(5,-10){\line(1,0){115}}
\put(120,-10){\line(0,1){60}}
\multiput(5,0)(0,10){5}{\line(1,0){2}}
\thinlines
\put(1,0){0}
\put(1,9){1}
\put(1,19){2}
\put(1,29){3}
\put(1,39){4}
\multiput(5,5)(0,10){5}{\line(1,0){1}}
\put(10,0){\nl{0}}
\put(10,10.8){\line(1,0){10}\raisebox{-1.5ex}{\hspace{1mm}0}}
\put(10,12.6){\nl{2}}
\put(10,18.5){\nl{2}}
\put(10,24.5){\line(1,0){10}\raisebox{0.2ex}{\hspace{1mm}4}}
\put(40,0){\nl{0}}
\put(40,10.4){\line(1,0){10}\raisebox{-1.0ex}{\hspace{1mm}0}}
\put(40,13.5){\nl{2}}
\put(40,18.9){\nl{2}}
\put(40,22.9){\line(1,0){10}\raisebox{0.2ex}{\hspace{1mm}4}}
\put(70,0){\nl{0}}
\put(70,11.2){\line(1,0){10}\raisebox{-1.0ex}{\hspace{1mm}0}}
\put(70,14.6){\nl{2}}
\put(70,19.9){\nl{2}}
\put(70,22.7){\line(1,0){10}\raisebox{0.2ex}{\hspace{1mm}4}}
\put(100,0){\nl{0}}
\put(100,38.3){\nl{2}}
\put(100,45){\nl{4}}
\put(9,-7){\small{(--73.26)}}
\put(39,-7){\small{(--73.28)}}
\put(69,-7){\small{(--73.57)}}
\put(99,-7){\small{(--73.94)}}
\put(15,-20){A}
\put(45,-20){B}
\put(75,-20){C}
\put(101,-20){Expt}
\end{picture}
\end{center}
\caption{The low-lying $^{48}$Ca spectra relative to $^{40}$Ca.}
\label{fig:ca48}
\end{figure}


Although our results for the $A=42$ system did not reproduce
properly the experimental levels, we see that the
correspondence between levels in all the nuclei
shown in figs.\ \ref{fig:ca44}-\ref{fig:ca48} is
in general rather good, except for the excited states
in $^{48}$Ca. The binding energies are within some few keV close to
the experimental values, and our results with a two-body
effective interaction in the $pf$-shell
allow both for intruder state corrections\footnote{Again
we omit any reference to possible intruder states in our spectra.}
and the inclusion
of effective three-body forces, probably without loosing
the degree of accuracy presented in these figures. Actually, the
present results are close to those of Richter {\em et al.}
\cite{richt91} where a phenomenological effective interaction
was constructed through a fit to the available body of data.
Though, our effective interactions are not directly
comparable to those of Richter {\em et al.}, since these authors
try to construct an effective interaction meant to reproduce
the observed levels,
with the least possible deviation. In
such a fit, intruder state admixtures may be present
in the wave functions of the various states, and
accounting for such admixtures could then very well spoil their
agreement with the data. 
The matrix elements of Richter {\em et al.} \cite{richt91} are
shown in table C.2 together with our effective interactions. The differences
are discussed below in connection with the results for $^{48}$Ca.

The remarkable feature of the effective interactions
obtained with the Bonn potentials is that their monopole
terms are realistic enough to reproduce the known
binding energies of most of the nuclei we have studied.
This should be contrasted to the Kuo-Brown interaction
in the $pf$-shell \cite{kb68}, which produced too little
binding for the ground states in general. Extensive
discussions of the effective interactions from ref.\ \cite{kb68}
can be found in refs.\ \cite{richt91} and
\cite{pz80}. In the latter work, the Kuo-Brown
matrix elements were adjusted in order to make
them spectroscopically viable.
These findings
apply to other potentials which have a tensor force
stronger than those of the Bonn potentials, like
the Reid potential \cite{reid68} or the Paris potential
\cite{paris80}. Results obtained with the latter potential
are given in ref.\
\cite{heho92}. There the binding energy of e.g.\
$^{44}$Sc was 2 MeV lower than the experimental value.
For this nucleus, shown in fig.\ \ref{fig:sc44},
although our effective interactions
do not
reproduce precisely the correct ordering, the deviations
are not very large, the calculated ground states are close
to the experimental value. The high-spin states
$9^+$, $10^+$ and $11^+$, which are expected to be dominated by the
$(f_ {7/2})^4$ configuration, are reproduced
at about the right energies for the Bonn C potential,
while the Bonn A and B potentials, which introduce too much attraction,
give high-spin states in less agreement with
the experimental values. This
conforms with what we saw in the $sd$-shell, the potentials
with a stronger tensor force result in spectra which are
more compressed.
In general, our excited spectra are within some $100-300$ keV
of the experimental values.   One of the exceptions are the low-lying
states of $^{48}$Ca. Although the binding energy is correctly
reproduced, we are not able (even if we include the $g_{9/2}$
orbit) to reproduce the typical shell-closure feature of
$^{48}$Ca. It seems that the pairing structure
seen in the lighter calcium isotopes like
$^{42}$Ca, $^{44}$Ca and $^{46}$Ca prevails
in $^{48}$Ca as well, with an almost constant
spacing between the $0_1^+$ and $2_1^+$
states.
This observation applies to the Kuo-Brown
interaction as well \cite{richt91}, or to the effective
interaction derived from the Paris potential in ref.\
\cite{heho92}.
The question we then pose is whether our results for $^{48}$Ca could
be retraced to certain effective matrix elements. In $^{48}$Ca the
$f_{7/2}$ orbit is filled up, and the $p_{3/2}$ orbit is the next close
one. We would therefore expect matrix elements
which involve these two orbits to be the most important ones in the
description of the low-lying spectra of $^{48}$Ca. Moreover, since the
matrix elements of Richter {\em et al.} \cite{richt91}
reproduce nicely the experimental
$0_1^+$ and $2_1^+$ states for this nucleus, it is of interest to compare
our matrix elements with theirs. In table \ref{tab:fpveff} we list
our matrix elements and those of Richter {\em et al.} For the Bonn C potential
there is a rather close agreement for almost all matrix elements\footnote{The average
difference is 125 KeV for both $T=0$ and $T=1$ matrix elements.}
with the empirical ones. The only exceptions are
our $\bra{f_{7/2}p_{3/2}}V_{\mathrm{eff}}\ket{f_{7/2}p_{3/2}}$
matrix elements for $T=1$ and $J=3,4,5$, which
are more attractive  than those in ref.\ \cite{richt91}. Our energy centroid
$\overline{V}$ for
this matrix element, i.e.\
\[
\bra{f_{7/2}p_{3/2}}\overline{V}\ket{f_{7/2}p_{3/2}}_{T=1}
=\frac{\sum_{J}\hat{J}^2
 \bra{f_{7/2}p_{3/2}J}V_{\mathrm{eff}}\ket{f_{7/2}p_{3/2}J}_{T=1}}
  {\sum_{J}\hat{J}^2},
\]
is $-0.1303$ MeV, whereas with the empirical interaction we get
$+0.1631$ MeV,
a repulsitivity which is expected from phenomenological studies. The latter
number is almost equal to the one extracted by Cole \cite{cole91}
from an analysis of sp spectra in the mass regions of $A=40-48$, see table
1 of ref.\ \cite{cole91}.

The question then is which mechanisms may affect these matrix elements.
One possibililty is to abandon the h.o. basis and use a
self-consistent Brueckner-Hartree-Fock (BHF) basis. The latter yields in
general less attractive matrix elements, though the
final result for the spectra
of all the studied calcium isotopes may not turn out as good as it is
with a h.o. basis. Results from BHF calculations in the $sd$-shell
seem to suggest this \cite{homs90}.
Another possibility is to look at the nucleon-nucleon
interaction.
There one could
consider the relevant medium modifications of the
parameters which define the NN potential, such as the meson and nucleon
masses. However, this is a fairly involved scheme  which implies
that one has to solve self-consistently the Dyson equations
for the various particles, in order to obtain their effective masses.
The outcome of this scheme is also not clear.
Moreover, the reader has probably noticed that we have omitted
effective three-body forces like those displayed in fig.\
\ref{fig:threebody}.
\begin{figure}[hbtp]
    \setlength{\unitlength}{1mm}
    \begin{picture}(140,80)
      \put(25,10){\epsfxsize=12cm \epsfbox{threebody.eps}}
     \end{picture}
  \caption{Examples of effective three-body forces not included
  in the definition of the \qbox.}
\label{fig:threebody}
\end{figure}
Three-body forces are in general repulsive \cite{pmfko82},
and we would therefore expect that their inclusion  may improve the agreement
for $^{48}$Ca. Calculations with three-body forces are under study
\cite{three94}.
In summary, however, before any conclusion can be drawn, we need
to further investigate some of the possible mechanisms
discussed above.






\subsection{Nuclei with two or more
valence nucleons in the mass region of $^{100}$Sn}

Considerable attention is at present being devoted to the
experimental
study of light Sn isotopes \cite{schu92,schu91,grawe92,ryk92,joh93}.
During
the last few years,
a rich
variety of data has 
become available for nuclei far from the stability line.
Recently, substantial progress has been made in the spectroscopic approach
to the neutron deficient doubly magic $^{100}$Sn core. Experimental
spectroscopic data are presently available down to $^{104}$Sn
\cite{schu92,schu91,grawe92,ryk92}. A recent work by Schneider
{\em et al.} \cite{sn100} reports on the production and
identification of $^{100}$Sn. This nucleus is the heaviest
doubly-magic nucleus close to the proton drip line, and has
been studied theoretically, using
approaches inspired by the relativistic Serot-Walecka \cite{sw86}
model, such as the calculations of Hirata {\em et al.} \cite{hir91}
and Nikolaus {\em et al.} \cite{hnm92}, or non-relativistic
models used by Leander {\em et al.} \cite{lean84}.  The theoretical
studies of refs.\ \cite{hir91,hnm92} indicate 
that $^{100}$Sn is a reasonably stable closed shell model
core. 

To evaluate physical quantities in light tin isotopes, we will
proceed in the standard fashion as done in the previous
subsections. But this requires the knowledge of the
experimental sp energies of $^{101}$Sn, which are not known.
One can however guess on an optimal set of sp energies by
extrapolating them from the data of tin isotopes with more
valence nucleons, as done here, in refs.\ \cite{ehho93,holt94,physcripta94}
and in the work of
Sandulescu {\em et al.} \cite{nicu93,nicu94}.
In ref.\
\cite{nicu94},
the sp energies were derived from a fit to the levels in
$^{111}$Sn with a one quasiparticle content. These authors
use the BCS method, and the spuriosities that appear as a
result of the breaking of the particle number conservation symmetry,
were removed using the quasiparticle multistep shell-model method
(QMSM) of ref.\ \cite{pomar92}. The QMSM method offers a direct
and nice interpretation of the spectra of
tin isotopes in terms of quasiparticle excitations. Recently, 
the authors of ref.\ \cite{nicuoslo94}
have performed a quantitative comparison with the
QMSM and the shell-model approach discussed here, using the effective
interactions of ref.\ \cite{ehho93}. 
The agreement between the two methods was striking
\cite{nicuoslo94}, both for odd and even tin isotopes.
Since shell-model calculations like the ones reported here, are
rather intensive from a computational point of view,
methods like the QMSM may provide information on the sp energies
from nuclei like $^{111}$Sn, were full shell-model
calculations are extremely time consuming. That the dimensionality
of the systems to consider is large, can be seen from table
\ref{tab:tindim}.
\begin{table}
\label{tab:tindim}
\begin{center}
\caption{Number of basic states for the shell-model calculation
for selected tin isotopes with the 
$1d_{5/2}$, $0g_{7/2}$, $1d_{3/2}$, $2s_{1/2}$
and the $0h_{11/2}$ sp orbits defining the model space.}
\begin{tabular}{lr lr}\\ \hline
System & Dimension & System & Dimension\\ \hline
$^{102}$Sn  & 26&  $^{110}$Sn  & 1,853,256\\
$^{103}$Sn  & 245&  $^{111}$Sn  & 3,608,550\\
$^{104}$Sn  & 1504&  $^{112}$Sn  & 6,210,638\\
$^{105}$Sn  & 7,451&  $^{113}$Sn  & 9,397,335\\
$^{106}$Sn  & 31,124&  $^{114}$Sn  & 12,655,280\\
$^{107}$Sn  & 108,297&  $^{115}$Sn  & 15,064,787\\
$^{108}$Sn  & 323,682&  $^{116}$Sn  & 16,010,204 \\
$^{109}$Sn  & 828,422&              &           \\ \hline
\end{tabular}
\end{center}
\end{table}

In this subsection we present results for
the even  $^{102, 104, 106, 108, 110}$Sn isotopes\footnote{The results
for the odd tin isotopes will be presented
in refs.\ \cite{physcripta94,nicuoslo94}.} using the doubly magic
nucleus $^{100}$Sn as
core and distributing the valence neutrons over
the single-particle orbits in
the $N = 4$ oscillator shell
($1d_{5/2}$, $0g_{7/2}$, $1d_{3/2}$, $2s_{1/2}$)
and the orbital $0h_{11/2}$ from the $N = 5$ oscillator shell.
We include the  $0h_{11/2}$ orbital, which is known to
be important for the structure of the heavier Sn isotopes \cite{bon84}
particularly in order to produce the pairing correlation found in these
nuclei. This is also supported by the conclusions of refs.\
\cite{ehho93,holt94,physcripta94,nicu94,nicuoslo94}.


Since
no experimental information is available for the $^{101}$Sn
one-neutron system
to establish the sp energies,
we adopt here the sp energies from \cite{nicuoslo94},
which reproduce the lowest lying
$h_{11/2}$ energy level n $^{107,109}$Sn, and the $g_{7/2}$
and the $d_{5/2}$ energy levels in the same nuclei. This means
that our sp energies are
$\varepsilon_{0g_{7/2}}-\varepsilon_{1d_{5/2}} =0.20$ MeV,
$\varepsilon_{1d_{3/2}}-\varepsilon_{1d_{5/2}} =1.50$ MeV,
$\varepsilon_{2s_{1/2}}-\varepsilon_{1d_{5/2}} =2.80$ MeV and
$\varepsilon_{0h_{11/2}}-\varepsilon_{1d_{5/2}}$ set to 3.0 MeV.
At present we have not yet carried out full shell-model calculations
of $^{111}$Sn, thus our sp energies for the $1d_{3/2}$ and the
$2s_{1/2}$ orbits may not be fully adequate.
The QMSM calculation of ref.\ \cite{nicuoslo94}
using the effective interactions of ref.\ \cite{ehho93}, 
suggest that the sp energies for the
latter two orbits should be changed to roughly $2.5$ MeV.
However, for all other
tin isotopes, the agreement between the data  and our shell-model
results from realistic effective
interactions is rather satisfactory, though we
are not able to reproduce well the only known E2 transitions in
$^{104,106}$Sn (see also the discussion below).
The Pauli operator used in the evaluation of the $G$-matrix
was defined in chapter 5, with $n_1=11, n_2=21, n_3=66$, so as to
prevent scattering into intermediate states with a single
nucleon in any of the
states defined by the orbitals from the $0s$ state to the $0g_{9/2}$ state
or with two nucleons in the $sdg$ ($0g_{9/2}$ excluded) or the $pfh$
shells. A harmonic-oscillator basis was chosen for the
single-particle wave functions, with an oscillator energy $\hbar\omega$,
where $\omega$  is determined through the relation
$\hbar\omega = 45A^{-1/3} - 25A^{-2/3}= 8.5$ MeV,  $A$ being the mass
number. Only effective interactions obtained
from the FD method with a third-order
\qbox are used  in the calculation of the spectra.
\begin{table}[hbpt]
\caption{Results for $^{102}$Sn for the Bonn A, B and C
potentials.}\label{tab:sn102}
\begin{center}
\begin{tabular}{rcccc}
\\\hline
$J$& A& B
& C & Expt \\[1ex]
\hline
0 & 0.000   & 0.000   & 0.000&\\
 2 &1.485  &1.452  &1.481 &1.3\\
 4 &1.864  &1.715  &1.752 &\\
 6 &1.968  &1.814  &1.851 &\\
 8 &8.150  &7.974  &8.000 &\\
 10&8.271  &8.085  &8.116 &\\
\hline
\end{tabular}
\end{center}
\end{table}
\begin{table}[hbpt]
\caption{Results for $^{104}$Sn for the Bonn A, B and C
potentials.}\label{tab:sn104}
\begin{center}
\begin{tabular}{rcccc}
\\\hline
$J$& A& B
& C & Expt \\[1ex]
\hline
0 & 0.000   & 0.000   & 0.000&0.000\\
 2 & 1.265 & 1.211 &1.230 &1.259\\
 4 & 1.702 & 1.621 &1.641 &1.942\\
 6 & 2.137 & 1.962 &1.999 &2.257\\
 8 & 3.262 & 3.005 &3.073 &3.440\\
 10& 3.808 & 3.495 &3.561 &3.980\\
\hline
\end{tabular}
\end{center}
\end{table}
\begin{table}[hbpt]
\caption{Results for $^{106}$Sn for the Bonn A, B and C
potentials.}\label{tab:sn106}
\begin{center}
\begin{tabular}{rcccc}
\\\hline
$J$& A& B
& C & Expt \\[1ex]
\hline
0 & 0.000   & 0.000   & 0.000&0.000\\
 2 & 1.186 &1.150  &1.164 &1.206\\
 4 & 1.835 &1.745  &1.756 &2.017\\
 6 & 2.255 &2.066  &2.099 &2.321\\
 8 & 3.313 &3.079  &3.119 &3.476\\
 10& 4.169 &3.796  &3.864 &4.128\\
\hline
\end{tabular}
\end{center}
\end{table}
The results of the calculations for $^{102, 104, 106}$Sn, are shown
in tables \ref{tab:sn102}-\ref{tab:sn106}
for the three
potentials used in the previous sections.
The $2^+$ state
in $^{102}$Sn is tentative \cite{janb93}. The other experimental
data are from refs.\ \cite{schu92,schu91,grawe92}.
Since the experimental binding energies relative to $^{100}$Sn are
unknown, these are not
displayed in these tables, though, as was the case in the
$sd$- and $pf$-shells, it is the Bonn A potential which gives
the most strongly bound ground state. Similarly, as observed in previous
sections, this yields excited spectra for high-spin states
which are close to the experimental values. For the Bonn B and
C potentials, we obtain spectra which are more compressed, in
agreement with our findings in the $sd$- and $pf$-shells.
We note also that with an increasing number of valence nucleons,
the agreement for the theoretical high-spin states with the
experimental values, becomes better.


Engeland {\em et al.} \cite{ehho93} have evaluated
the only known E2 transitions in $^{104,106}$Sn, namely
$6_1^+ \rightarrow 4_1^+$. In that work, one was  not able
to reproduce the experimental values for these transitions,
and in $^{104}$Sn the computed $6_3^+ \rightarrow 4_2^+$
was closest to the experimental value of $4$ W.u. These results
were however premature, since the main decay trend follows the
Yrast states \cite{janb93}. The theoretical  $6_1^+ \rightarrow 4_1^+$
E2 transitions for $^{104,106}$Sn with the interaction in ref.\
\cite{ehho93} and the present work, disagree with
the experimental values \cite{holtp94}. One possible explanation
is that the choice of a h.o.\ basis for the wave functions
is not too realistic in the E2 calculations.
%    tin isotopes
\begin{figure}[hbtp]
\setlength{\unitlength}{1.0mm}
\begin{center}
\begin{picture}(140,95)(0,-30)
\thicklines
\put(5,-10){\line(0,1){70}}
\put(5,60){\line(1,0){115}}
\put(5,-10){\line(1,0){115}}
\put(120,-10){\line(0,1){70}}
\multiput(5,0)(0,10){6}{\line(1,0){2}}
\thinlines
\put(1,0){0}
\put(1,9){1}
\put(1,19){2}
\put(1,29){3}
\put(1,39){4}
\put(1,49){5}
\multiput(5,5)(0,10){6}{\line(1,0){1}}
\put(10,0){\nl{0}}
\put(10,12.65){\nl{2}}
\put(10,12.59){\ndot}
\put(10,17.02){\line(1,0){10}\raisebox{-0.1ex}{\hspace{1mm}4}}
\put(10,19.42){\ndot}
\put(10,21.37){\line(1,0){10}\raisebox{-0.1ex}{\hspace{1mm}6}}
\put(10,22.57){\ndot}
\put(10,32.62){\line(1,0){10}\raisebox{-0.1ex}{\hspace{1mm}8}}
\put(10,34.40){\ndot}
\put(10,38.08){\line(1,0){10}\raisebox{-0.1ex}{\hspace{1mm}10}}
\put(10,39.80){\ndot}
\put(40,0){\nl{0}}
\put(40,11.86){\nl{2}}
\put(40,12.06){\ndot}
\put(40,18.35){\line(1,0){10}\raisebox{-0.1ex}{\hspace{1mm}4}}
\put(40,20.17){\ndot}
\put(40,22.55){\line(1,0){10}\raisebox{-0.1ex}{\hspace{1mm}6}}
\put(40,23.21){\ndot}
\put(40,33.13){\nl{8}}
\put(40,34.76){\ndot}
\put(40,41.69){\nl{10}}
\put(40,41.28){\ndot}
\put(70,0){\nl{0}}
%    gs = -7.2629 MeV
\put(70,12.9){\nl{2}}
\put(70,12.06){\ndot}
\put(70,20.6){\line(1,0){10}\raisebox{-1.0ex}{\hspace{1mm}4}}
\put(70,21.12){\ndot}
\put(70,23.5){\nl{6}}
\put(70,23.65){\ndot}
\put(70,37.08){\line(1,0){10}\raisebox{-0.9ex}{\hspace{1mm}8}}
\put(70,35.66){\ndot}
\put(70,43.52){\nl{10}}
\put(70,42.52){\ndot}
%    gs = -8.3228 MeV  110 sn
\put(100,0){\nl{0}}
\put(100,13.0){\nl{2}}
\put(100,21.00){\line(1,0){10}\raisebox{-1.4ex}{\hspace{1mm}4}}
\put(100,23){\nl{6}}
\put(100,36){\nl{8}}
\put(100,42){\nl{10}}
\put(100,12.59){\ndot}
\put(100,19.42){\ndot}
\put(100,22.57){\ndot}
\put(100,34.40){\ndot}
\put(100,39.80){\ndot}
\put(15,-20){$^{104}$Sn}
\put(45,-20){$^{106}$Sn}
\put(75,-20){$^{108}$Sn}
\put(101,-20){$^{110}$Sn}
\end{picture}
\end{center}
\caption{The low-lying spectra for $^{104, 106, 108, 110}$Sn
obtained with the folded-diagram method and the Bonn A potential.
Solid lines are the theoretical values while a dashed line
represents the experimental values.}
\label{fig:tin100}
\end{figure}


Since it is the Bonn A potential which gives the best agreement
with the data, we choose to display the final results
for all the nuclei $^{104, 106, 108, 110}$Sn for this
potential only in fig.\ \ref{fig:tin100}. The experimental
values are those with the dashed lines. Clearly, as can be seen
from this figure, the agreement with the data is indeed rather
good.
The corresponding effective interaction is listed in appendix C.

For  the heavier tin isotopes, only recently the sp energies
for $^{131}$Sn were reasonably established \cite{fb84}.
For these isotopes, one can then get a much clearer picture of whether
our effective interaction is suited or not for calculating properties
of nuclei like $^{126}$Sn. 
The results for these nuclei will be presented elsewehere.

\subsection{Hermitian effective interactions}

Typical examples where the non-hermiticity is weak, can be found
in the calculations presented hitherto.
\begin{figure}[hbtp]
      \setlength{\unitlength}{1mm}
      \begin{picture}(120,100)
      \put(25,10){\epsfxsize=12cm \epsfbox{hermitic.eps}}
      \end{picture}
      \caption{Illustration of the non-hermiticity
               for the core-polarization diagram.}
      \label{fig:corepol}
\end{figure}
Consider again the core-polarization diagram shown in (a) and (b)
of fig.\ \ref{fig:corepol}.
Diagram (a) is proportional to
\begin{equation}
  (a) \propto \frac{1}{\varepsilon_{\alpha}+\varepsilon_h
      -\varepsilon_p-\varepsilon_{\gamma}},
\end{equation}
while (b) is proportional to
\begin{equation}
   (b) \propto \frac{1}{\varepsilon_{\gamma}+\varepsilon_h
       -\varepsilon_p-\varepsilon_{\alpha}},
\end{equation}
where $\varepsilon$ is the unperturbed single-particle energy.
If the model space is degenerate, $(a)-(b)=0$, the contribution to the
effective interaction from the core-polarization diagram is hermitian.
In general
however, the model space is not degenerate,
although if we define a model-space
to consist of single-particle orbitals  from the same shell,
we will
have $(a)-(b)\approx 0$. As discussed in the
preceeding subsections, such a model
space is believed to be  appropriate for
calculations of e.g.\ nuclei like $^{18}$O or $^{42}$Ca.

For model spaces which include orbits from several shells
the non-hermiticity may not be
weak, and the above brute force
scheme questionable. As an example, consider
a model space consisting of both the $sd$- and $pf$-shells.
Such a multi-shell model space may be appropriate in the
description of closed shell nuclei like $^{40}$Ca.
We let the labels $\gamma$ and $\delta$ 
in fig.\ \ref{fig:corepol} be single-particle
orbits in the $pf$ shell, while $\alpha$ and $\beta$ 
are $sd$-shell single-particle
orbits. If we restrict the attention to intermediate state excitations
of $2\hbar\omega$ in oscillator energy, then diagram (a) in fig.\
\ref{fig:corepol} gives
\begin{equation}
	  \frac{1}{\varepsilon_{sd}+\varepsilon_h
	  -\varepsilon_p-\varepsilon_{pf}}=
	  -\frac{1}{2\hbar\omega},
\end{equation}
when we employ harmonic oscillator single-particle energies and the hole
state $h$ is in the $0p$ shell and the particle state $p$ is in the
$sd$-shell. Diagram (b)
gives, with the same single-particle orbits,
\begin{equation}
	 \frac{1}{\varepsilon_{pf}+\varepsilon_h
	 -\varepsilon_p-\varepsilon_{sd}}=
	 -\frac{1}{0}.
\end{equation}
Even with the insertion of self-consistent Brueckner-Hartree-Fock or
experimental single-particle energies, diagrams (a)
and (b) will be rather different, yielding
a possibly large non-hermitian contribution to $H_{\mathrm{eff}}$.

In summary, this example shows that with a single-shell model space, the
non-hermiticity may be weak, whereas multi-shell model spaces may result in
large non-hermitian contributions to the effective interaction. Thus,
multi-shell effective interactions call for a hermitian description.
Even the $G$-matrix is non-hermitian if we employ
different starting energies.
Several procedures exist to obtain a hermitian effective interaction. One
is to transform the non-hermitian matrix through bi-unitary
transformations to a hermitian matrix. Furthermore,
recently, several authors have presented a hermitian formulation for the
effective interaction $H_{\mathrm{eff}}$ \cite{lindgren91,suzuki93,kuo93},
obtained within the framework of time-independent perturbation theory
\cite{eo77,lm85,bran67}.
These authors impose subsidiary conditions
in order to obtain a hermitian effective interaction.
Though, the reader should note that there is nothing more fundamental
in the hermitian schemes compared to the traditional RS schemes. To all
orders, whatever formulation of perturbation theory we employ, we
should obtain the same results.
However, in actual calculations we are only able to evaluate
a limited number of terms in the perturbation expansion. If we truncate
the various expansions at a given order, the results may differ.


The question now is
whether we may obtain a hermitian secular equation for the effective interaction
which has the structure of eq.\ (\ref{eq:mspacee})
and is of the form given by eq.\ (\ref{eq:hexpect}).
In deriving such an equation, we will follow Ellis, Kuo, Suzuki and co-workers
\cite{suzuki93,kuo93}.


Let us first recall some of the results and definitions of section 4.
In that section we defined a model-space eigenstate
$\ket{b_{\lambda}}$ in eq.\ (\ref{eq:wfb}) as
\[
     \ket{b_{\lambda}}=\sum_{\alpha =1}^{D}
     b_{\alpha}^{(\lambda )}\ket{\psi_{\alpha}}
\]
and the biorthogonal wave function
\[
     \ket{\overline{b}_{\lambda}}=\sum_{\alpha=1}^{D}
      \overline{b}_{\alpha}^{(\lambda )}
     \ket{\overline{\psi}_{\alpha}},
\]
such that
\[
     {\left\langle \overline{b}_{\lambda} | b_{\mu} \right\rangle}=
     \delta_{\lambda\mu}.
\]
Finally, in
eq.\ (\ref{eq:truee3}) we defined a model-space eigenvalue equation
in terms of the time-development operator and the unperturbed
wave functions
\[
     {\displaystyle
     \sum_{\gamma =1}^{D}b_l^{(\lambda )}\bra{\psi_{\sigma}}
     HU_L(0,-\infty )\ket{\psi_{\gamma}}\ket{\Psi_{\tilde{c}}} } =
     E_{\lambda}b_{\sigma}^{(\lambda )},
\]
and  an effective interaction as
\[
     H_{\mathrm{eff}}=
     \bra{\psi_{\sigma}}
     HU_L(0,-\infty )\ket{\psi_{\gamma}}\ket{\Psi_{\tilde{c}}} .
\]
The exact wave function was expressed in terms of the correlation 
operator
\begin{equation}
      \ket{\Psi_{\lambda}}=({\bf 1}+\chi)\ket{\psi_{\lambda}}.
\end{equation}
The part $\chi\ket{\psi_{\lambda}}$ can be expressed in terms
of the time-development operator
\begin{equation}
  \chi\ket{\psi_{\lambda}}\propto   
  \tilde{W}(0,-\infty )\ket{\psi_{\lambda}}\ket{\Psi_{\tilde{c}}} ,
\end{equation}
or using the time-independent formalism
\begin{equation}
  \chi\ket{\psi_{\lambda}}=
  \frac{Q}{E_{\lambda}-QHQ}QVP\ket{\psi_{\lambda}},
  \label{eq:newchi}
\end{equation}
where $Q$ is the exclusion operator. Note that this equation is given
in terms of the Brillouin-Wigner perturbation expansion, since
we have the exact energy $E_{\lambda}$ in the denominator.
As in section 4, we will use the folded-diagram technique to remove
this energy dependence.

Using the normalization condition for the true wave function
we obtain
\begin{equation}
  \left\langle \Psi_{\gamma} |\Psi_{\lambda}\right\rangle
  N_{\lambda}\delta_{\lambda\gamma}=
  \bra{\psi_{\gamma}}({\bf 1} +\chi^{\dagger}\chi\ket{\psi_{\lambda}},
  \label{eq:newnorm}
\end{equation}
where we have used the fact that
$\bra{\psi_{\gamma}}\chi\ket{\psi_{\lambda}}=0$. Recalling that the
time-development operator is hermitian we have that $\chi^{\dagger}\chi$ is
also hermitian. We can then define an orthogonal basis $d$ whose eigenvalue
relation is
\begin{equation}
   \sum_{\alpha}\bra{\psi_{\beta}}\chi^{\dagger}
   \chi\ket{\psi_{\alpha}}d_{\alpha}^{\lambda}=\mu^{2}_{\lambda}
   d_{\beta}^{\lambda},
   \label{eq:newbasis1}
\end{equation}
with eigenvalues greater than $0$.
Using the definition in eq.\ (\ref{eq:newchi}), we note
that the diagonal element of
\begin{equation}
   \bra{\psi_{\lambda}}\chi^{\dagger}
   \chi\ket{\psi_{\lambda}}=
   \bra{\psi_{\lambda}}PVQ\frac{1}{(E_{\lambda}-QHQ)^2}QVP
   \ket{\psi_{\lambda}},
\end{equation}
which is nothing but the derivative of the \qbox, with an additional
minus sign. Thus, noting that if $\gamma\neq\lambda$
\begin{equation}
  \left\langle \Psi_{\gamma} |\Psi_{\lambda}\right\rangle=0=
    \left\langle \psi_{\gamma} |\psi_{\lambda}\right\rangle
   +\bra{\psi_{\gamma}}\chi^{\dagger}\chi\ket{\psi_{\lambda}},
\end{equation}  
we can write $\chi^{\dagger}\chi$
in operator form as
\begin{equation}
   \chi^{\dagger}\chi =-\sum_{\alpha}\ket{\overline{\psi}_{\alpha}}
   \bra{\psi_{\alpha}}\hat{Q}_1(E_{\alpha})\ket{\psi_{\alpha}}
   \bra{\overline{\psi}_{\alpha}}
   -\sum_{\alpha\neq\beta}\ket{\overline{\psi}_{\alpha}}
    \left\langle \psi_{\alpha} |\psi_{\beta}\right\rangle
   \ket{\overline{\psi}_{\beta}}.
\end{equation}
Using the new basis in eq.\ (\ref{eq:newbasis1}), we see that
eq.\ (\ref{eq:newnorm}) allows us to define another
orthogonal basis $h$
\begin{equation}
  h_{\alpha}^{\lambda}=\sqrt{\mu_{\alpha}^2+1}\sum_{\beta}
  d_{\beta}^{\alpha}b_{\beta}^{\lambda}\frac{1}{\sqrt{N_{\lambda}}}
  =\frac{1}{\sqrt{\mu_{\alpha}^2+1}}\sum_{\beta}
  d_{\beta}^{\alpha}\overline{b}_{\beta}^{\lambda}\sqrt{N_{\lambda}},
\end{equation}
where we have used the orthogonality properties of the vectors
involved. The vector $h$ was used by the authors of ref.\ \cite{kuo93}
to obtain a hermitian effective interaction as
\begin{equation}
   \bra{\psi_{\alpha}}V_{\mathrm{eff}}^{\mathrm{(her)}}\ket{\psi_{\beta}}=
   \frac{\sqrt{\mu_{\alpha}^2+1}\bra{\psi_{\alpha}}
   V_{\mathrm{eff}}^{\mathrm{(nher)}}\ket{\psi_{\beta}}
   +\sqrt{\mu_{\beta}^2+1}\bra{\psi_{\alpha}}
   V_{\mathrm{eff}}^{\dagger\mathrm{(nher)}}\ket{\psi_{\beta}}   }
   {\sqrt{\mu_{\alpha}^2+1}+\sqrt{\mu_{\beta}^2+1} },
\end{equation}
where (her) and (nher) stand for hermitian and non-hermitian
respectively. This equation is rather simple to compute, since
we can use either the Lee-Suzuki or the folded-diagram method
to obtain the non-hermitian part.
To obtain the total hermitian effective interaction, we have to
add the $H_0$ term. The above equation is manifestly hermitian and
was used in ref.\ \cite{kuo93} to compute effective
interactions with the Bonn A potential of table A.1 in ref.\
\cite{mac89}, using the same parameters for $sd$-shell nuclei as in this
work.
The Lee-Suzuki method was used to obtain
the non-hermitian effective interactions
$V_{\mathrm{eff}}^{\mathrm{(nher)}}$, see section 4. The results
for both a hermitian and a non-hermitian effective interactions
from ref.\ \cite{kuo93} are shown in table \ref{tab:hermiteveff}.
Clearly, the non-hermiticity is rather weak, 
and the hermitian  $sd$-shell
effective is in general close to
\begin{equation}
   \bra{\psi_{\alpha}}V_{\mathrm{eff}}^{\mathrm{(her)}}\ket{\psi_{\beta}}
   \approx\frac{1}{2}\left(\bra{\psi_{\alpha}}
    V_{\mathrm{eff}}^{\mathrm{(nher)}}
    \ket{\psi_{\beta}}
   +\bra{\psi_{\alpha}}V_{\mathrm{eff}}^{\dagger\mathrm{(nher)}}
   \ket{\psi_{\beta}}\right).
\end{equation}
Similar results were also obtained for the $pf$-shell.
\begin{table}[hbtp]
\caption{Hermitian and non-hermitian effective
interactions derived from a second-order \qbox using the
Bonn A potential. The results are taken from ref.\ [113].}
\begin{center}
\begin{tabular}{lllllrrr}
\\\hline
$JT$&$j_{a}$&$j_{b}$&$j_{c}$&$j_{d}$&
\multicolumn{1}{c}{$V_{\mathrm{eff}}^{\mathrm{(her)}}$}&
\multicolumn{1}{c}{$V_{\mathrm{eff}}^{\mathrm{(nher)}}$}&
\multicolumn{1}{c}{$V_{\mathrm{eff}}^{\dagger\mathrm{(nher)}}$}
\\\hline
01&$d_{5/2}$&$d_{5/2}$&$d_{5/2}$&$d_{5/2}$&-2.676&&-2.676\\
01&$d_{5/2}$&$d_{5/2}$&$d_{3/2}$&$d_{3/2}$&-3.546&-3.589&-3.567\\
01&$d_{5/2}$&$d_{5/2}$&$s_{1/2}$&$s_{1/2}$&-1.080&-1.032&-1.056\\
01&$d_{3/2}$&$d_{3/2}$&$d_{3/2}$&$d_{3/2}$&-1.059&&-1.059\\
01&$d_{3/2}$&$d_{3/2}$&$s_{1/2}$&$s_{1/2}$&-0.826&-0.774&-0.800\\
01&$s_{1/2}$&$s_{1/2}$&$s_{1/2}$&$s_{1/2}$&-2.011&&-2.013\\
\hline
\end{tabular}
\end{center}
\label{tab:hermiteveff}
\end{table}
However, if one takes into account a model space which spans over more
than one shell, the eigenvalues of eq.\ (\ref{eq:newbasis1}) may become
rather large, and the approximation used in this work may no longer
be valid. For such extended model spaces, the hermitian formulation
of ref.\ \cite{kuo93} may be appropriate, though one of the
problems with this expansion, is that eq.\ (\ref{eq:newbasis1})
depends on the exact energy $E$. This is an undesirable property
of the above expansion, since we would like to use this effective
interaction for nuclei other than those with two valence nucleons
only. The expansion presented by Lindgren \cite{lindgren91} seems
at present to be free of such problems, though it would be desirable
to present an effective hermitian interaction based on time-dependent
perturbation theory. Such an approach will be presented elsewhere
\cite{mhj94}. Finally, we mention that an expression for a folded-diagram
expansion with a non-degenerate model space has been presented by
Suzuki {\em et al.} \cite{suz94}.








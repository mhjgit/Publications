
Throughout this work we have put emphasis on the connections between the
free nucleon-nucleon (NN) interaction and the many-body theory
appropriate for selected nuclear systems.
Of special interest was the
connection between the tensor force of the nucleon-nucleon interaction
and properties of nuclear systems, like the binding energy per particle
in nuclear matter and
the reproduction of
nuclear spectra. In these systems, it was seen that potentials
with a weak tensor force, i.e.\ strong cancellation between the
contributions from $\pi$ and $\rho$ mesons, introduced more attraction
in both nuclear matter and finite nuclei.
Further, in addition to the study of the tensor force component in a nuclear
medium, relativistic effects were reviewed in the nuclear matter and neutron
matter analyses. At densities greater than the saturation density of
nuclear matter, relativistic effects begin to dominate.
Actually, these effects were seen to
be of importance in order to reproduce the empirical nuclear matter data,
a feature which is normally not accounted for by non-relativistic approaches.



With the above comments  in mind, following the many-body approach
discussed in this work, we wish to draw the attention to some of our
results in sections 6-8 and discuss
selected perspectives
for future works:

\begin{itemize}
\item


In section 6 we calculated effective interactions for
nuclei in the mass regions of oxygen, calcium  and tin.
The effective
two-body interaction was in turn used to calculate spectra of
nuclei with more than two valence nucleons. The agreement with the
experimental data was in many cases rather good.
Especially, the reproduction
of binding energies for several of the nuclei studied
is an improvement compared to earlier calculations with
older and more phenomenological
models for the nucleon-nucleon (NN)
interaction\footnote{A word on semantics. We ought to stress
that our approach is also a phenomenological one, in the sense that
we employ an effective model for the NN potential, where the
particles which enter our formalism are not the
fundamental ones. Though, the parameters which
enter the theory are physically motivated ones (e.g.\ meson masses
and coupling constants), and many
of these parameters are measurable quantities.}. Our results support
the view that the tensor force of the NN interaction should
be weak. Also, the excited spectra for many of the
nuclei studied were in satisfactory agreement with the
data. Similarly, in sections 7-8, possible improvements
in nuclear matter calculations were reviewed.

However, although it is {\em satisfactory} to have
a microscopic theory which yields a nice reproduction
of the data, {\em it is our firm belief
that one learns more from the disagreements
than from the agreements with the data}.
Such disagreements with the data were discussed by us in
connection with the results for the calcium isotopes in section
6.4.
It is always possible
to construct say effective interactions which yield
a good agreement with the spectra for
calcium isotopes. A typical example is given in the
work of Richter {\em et al.} \cite{richt91}, where
an effective $pf$-shell interaction is constructed. For nuclei
like $^{42}$Ca with two valence nucleons outside the closed
$^{40}$Ca core, there is both theoretical and experimental
evidence that configurations outside the chosen model space,
the $pf$-shell, play an important role. Therefore,
an effective interaction defined within the $pf$-shell
only should not reproduce the data for nuclei like $^{42}$Ca.
This was indeed the case in our calculations in section 6.4.
There, the binding energies of both $^{42}$Ca and $^{42}$Sc
were in a poor agreement with the data. We argued then that
the inclusion of
so-called intruder state configurations which were excluded
from our model space, would improve the agreement with the data.
For isotopes like $^{44}$Ca and $^{46}$Ca the agreement with
the data was much better, whereas for $^{48}$Ca the
theoretical spectra
were too compressed compared to the data. Most microscopic
effective interactions, from those of Kuo and Brown \cite{kb68}
to the present, are not able to reproduce the typical shell
closure (the $f_{7/2}$ orbit) in $^{48}$Ca, even if we extend
the model space so as to include the $g_{9/2}$ orbit.
This is still a puzzle to us\footnote{A similar behavior has been
observed in tin isotopes for $^{116}$Sn, where the $1d_{5/2}$
and $0g_{7/2}$ orbits are filled up \cite{eng94}.}.

One should then be able to find the
reasons for the disagreements by re-examining the various
approximations made in the many-body approach. One possible
reason is the omission of certain configurations in the
definition of the model space. Another reason is whether neglected
many-body effects like effective three-body forces
would play an important role, or if the parameters which
enter the NN potential, which is defined through scattering
data for free particles, should reflect the medium dependence.
There one has to take into account medium modifications
of the
meson masses and the nucleon mass in the NN potential, following
e.g.\ the scaling of masses proposed by Brown and Rho \cite{br91}.
In order for such a scheme to be a consistent one, one would
have to solve self-consistently the Dyson equation for both
the nucleon and the mesons involved, in order to obtain
the self-energy and the effective masses of these hadrons in
a medium. Even the calculation of the self-energy for the nucleon (see the
discussion below in connection with the
self-energy of baryons) is a highly non-trivial pursuit.

One
could also ascribe the discrepancies between the data and our
calculations to the use
of a harmonic oscillator basis. It is well
known that a harmonic oscillator basis gives in general
matrix elements which are more attractive compared with those
obtained through a self-consistent Brueckner-Hartree-Fock (BHF)
calculation. Whether such a BHF basis really will improve
our results, is not clear to us. Results from BHF calculations
in the $sd$-shell \cite{homs90} yield a rather poor agreement for
nuclei in the mass region of oxygen.

Another possible extension of our many-body approach is to account
for relativistic effects. For finite nuclei such calculations
have recently been pursued by Brockmann, Machleidt and
M\"{u}ther and
co-workers \cite{mmb90,mbm88,bt93,fmm93,fm94}, although at the level
of the $G$-matrix. Thus, from our experience in non-relativistic
calculations, we would expect that
there is still room for important
many-body correlations.
In this work,
relativistic effects were studied only for nuclear and neutron matter
in sections 7-8,
since the nuclei studied in section 6 could be described by the
degrees of freedom of selected valence nucleons. The density of these
valence nucleons is rather small, and relativistic effects negligible,
a fact confirmed by the analysis of M\"{u}ther {\em et al.} \cite{mbm88}.
However, for hole states, the density is larger, and the above relativistic
effects become important, as demonstrated in
the recent work of Zheng {\em et al.} \cite{zzm92}.

Finally, the reader should note that in the microscopic
approaches applied in sections 6-8,
we have put emphasis on certain properties
of finite nuclei and nuclear matter, in order to study how the various
components of the NN potential behave in a nuclear medium.
It ought to be stressed that properties like the tensor
force may behave differently for other observables,
as discussed by Zamick and Zheng \cite{zz94} for the
case of particle-hole states in closed-shell nuclei,
such as the $0^{-}$ state in $^{16}$O. Thus, a careful
study of several nuclear observables is needed in order
to shed more light on the connections between the many-body
approaches and the relevant free NN interaction.

Last but not least, this application of the NN potential
in calculations of various medium properties,
implies also the development
of the many-body approaches.  It should be clear that the approach
outlined for finite nuclei in section 4 has its limitations. On one side,
there are problems like the dimensionality of the model space used
in the shell model calculations\footnote{The problems we think of here
arise both in the construction of the effective two-body interaction
and in the calculation of spectra for nuclei with several valence nucleons,
like $^{110}$Sn.}.
These are to some extent technical problems, which with modern
computers can be handled to a certain level of complexity.
On the other side, the approach
in section 4 is taylored to solve problems within the framework of
the nuclear shell model.
Recent experimental results from electron scattering
facilities have provided both
confirmation of shell-model related concepts (such as the
sp wave function for the description of nucleons in nuclei)
as well as detailed informations on its limitations, see e.g.\ refs.\
\cite{ms91,frois,exp} for both theoretical and experimental surveys.
Clearly, if one wishes to evaluate the self-energy of the nucleon in
a finite nuclues (this self-energy could in turn be used
in a medium dependent NN potential), one has to account for high-lying
states in energy ranges beyond the
validity of the shell model. Also, a description of such high-lying states,
means that one has to consider representations of the wave
functions where particles may be bound or in the continuum.
Thus we believe, that the many-body approach outlined in section 4, needs to be
extended in order to handle the new results from e.g.\ electron scattering
experiments. At present, a viable approach seems to be 
offered by the so-called self-consistent Green's function method, see e.g.\
refs.\ \cite{dm92,rpd89,dickhoff94}.
The above comments lead us to the next point, namely, how to evaluate the 
self-energy of baryons in a nuclear medium (we do not mention the self-energy
of mesons, although this is also an obvious area of research).

\item
In this work we have limited the attention to effective
interactions,  though the extension to the calculation of effective
operators is straightforward.
An effective operator which is of interest is the self-energy
of e.g.\ a baryon in a nuclear medium.
The phenomenological
optical model has widely been used to analyze the data
of elastic  nucleon-nucleus scattering experiments. This large
amount of empirical information alone \cite{ms91}, would give
a sufficient justification for the different attempts of theoretical
nuclear physics to derive the potential of the optical model
within a microscopic many-body theory, starting with a realistic
interaction between the relevant baryons.
Such theoretical investigations are useful in
obtaining  a microscopic justification  for the real and imaginary
part of the optical potential. 
The starting point for these investigations is the 
determination of the {\em self-energy} 
of a given baryon
as a function of the two independent variables energy and 
momentum.
The self-energy can be
used as an important ingredient for an evaluation of the structure
function or response function of nuclei, which are used to
analyze the excitation modes of nuclei as they are observed
e.g.\ in (e,e') experiments \cite{frois}
From the self-energy one can furthermore derive the single-particle
Green's function through the solution of the Dyson equation and
the spectral function depending on momentum
and energy. The integration of the spectral function with respect
to the energy yields the occupation
probabilities, which provide important information on nuclear
correlations. Several experimental studies have been made
to obtain reliable information on these quantities \cite{exp}

The above remarks serve to demonstrate that the self-energy of
a baryon can be considered as a key point for many
investigations of nuclear structure. Thus, quite a lot of effort
has been made in order to evaluate the self-energy \cite{ms91},
though most attempts to determine this quantity are based on studies
in nuclear matter \cite{rpd89}, where the single-particle
states can be described in terms of plane waves. 

One of the main obstacles for theoretical investigations
in finite
nuclei is related to the fact that one has to consider
a representation for 
single-particle
states in terms of both bound and scattering states. 
To overcome this problem, the authors of refs.\
\cite{bon89,bbmp92,hbmp93} devoloped a method which allows a
direct calculation of the self-energy for the nucleon in terms
of matrix elements derived from a realistic interaction for the
baryons of interest. 
The novelty of this method, resides in
the fact that one can calculate the self-energy from a
microscopic approach, where scattering states are described in terms
of plane waves while bound states are described by the wave functions
of the harmonic oscillator.
This is easily achieved using eqs.\ (\ref{eq:kk}) and (\ref{eq:kho}).
The self-energy one  obtains from these
calculations, is then
used to derive an optical potential which is
parametrized in terms of a local and energy dependent Wood-Saxon potential.
The formalism of refs.\ \cite{bon89,bbmp92,hbmp93}, can be applied
to the study of the isobar $\Delta$ self-energy, as done
in ref.\ \cite{hmp93}.
Moreover, recently, much attention has been devoted
to the study of baryons with a strangeness content,
such as the
$\Lambda$ and $\Sigma$, and their behavior in a nuclear
medium
Of importance here is the evaluation of the self-energy for these
baryons, since the self-energy is intimately related to the
decay width \cite{ossigma}.
This is especially relevant for the $\Sigma$, since the existence
of relatively long-lived $\Sigma$-bound states has to be yet
understood.



\item
Another effective operator
of particular interest
is the study of the nuclear renormalization of the axial coupling constants.
The renormalizations of the isovector axial
charge and current coupling constants form an interesting and nontrivial
topic. On the one
hand second order configuration mixing of the nucleon states and
exchange current effects are known to cause a quenching of the axial
current coupling constant, $g_A^{GT}$, the magnitude of which is about
30\%, in $sd$ shell nuclei \cite{del81}. On the other hand the value of the
axial charge $g_A^C$, which equals $g_A^{GT}$ for free nucleons, is
strongly enhanced by up to 100\% in heavy nuclei \cite{warb91a,warb91b}
by exchange current
effects \cite{town92,krt92}. The main difference between the
exchange current
contributions to the effective isoscalar and isovector axial charge
coupling
constants is the absence of any long range pion exchange current
contribution to the former.
In the nuclear matter calculations of ref.\
\cite{mkrt92},  it was shown that the charge and current
coupling constants of the
effective isoscalar axial current density of a nucleon in a nucleus differ
appreciably from the corresponding free nucleon values.
The exchange current
correction implied by the short range components of the NN
interaction leads to a large enhancement of 40-50\% of the
isoscalar axial charge coupling constant at nuclear matter densities,
the precise value depending on the
potential model employed. An obvious extension of the above results,
would be to
calculate the corresponding effective operator for various finite nuclei,
in order to test the dependence on the different nuclear media.





\end{itemize}





We would like to thank Gerry Brown for the many stimulating 
discussions and initiatives in the field of effective interactions
and nuclear many-body physics in general.
One of us (MHJ) is much indebted to Torgeir Engeland
and Anne Holt for numerous discussions on the shell-model
code developed at Oslo and to Lars Engvik for the many days
spent on the nuclear matter calculations. We are also 
indebted to Ruprecht Machleidt for providing us with
figs.\ 7-9 prior to publication and for several discussions
on the Dirac-Brueckner-Hartree-Fock approach.
Moreover, throughout the course of this work we have largely
benefitted from interactions with many colleagues, amongst these:
Gang Bao, Jan Blomqvist, Paul Ellis, Marios Kagarlis,
Herbert M\"{u}ther, Chris Pethick, Arturo Polls,
Dan Olof Riska,  Nicolae Sandulescu, Dan Strottman, Kazuo Tsushima
and Erlend \O stgaard.


All calculations behind the results
reported here, except those which are
explicitly referenced,
have been performed at the IBM cluster at the
University of Oslo. Support for this by the Norwegian Research
Council (NFR) is greatly acknowledged. Further, T.T.S.\ Kuo
thanks the Nordic Institute for Theoretical Physics (Nordita)
for travel support. M.\ Hjorth-Jensen thanks the Research 
council of Norway (NFR) and the ECT* for financial support.














\appendix


\section{Evaluation of Feynman-Goldstone diagrams }


In this appendix we review first the rules for calculating Feynman-Goldstone
diagrams in an uncoupled
representation. Thereafter, we show by way of selected examples how to
calculate these diagrams in a coupled representation,
e.g.\ two particles coupled to a final angular momentum $J$.
Finally,
we list the one-body and two-body diagrams
included in our evaluation of the \qbox.



\subsection{Diagram rules}

A matrix element of an operator ${\cal O}$ 
in time-dependent perturbation theory may have the
form
\begin{equation}
{\cal O}_{fi} =\bra{C}A_f {\cal O}(0) U(0,-\infty)A_i^{\dagger}\ket{C},
\end{equation}
where the time evolution operator $U$ discussed in section 4 is
\begin{eqnarray}
   U(t,t')&=\lim_{\epsilon \rightarrow 0}
   \lim_{t'\rightarrow -\infty (1-i\epsilon )}
   {\displaystyle\sum_{n=0}^{\infty}\frac{(-i)^n}{n!}
   \int_{t'}^{t}dt_1  \int_{t'}^{t}dt_2\dots  \int_{t'}^{t}dt_n}
		\\ \nonumber
	     &  \times T\left[H_1(t_1)H_1(t_2)\dots H_1(t_n)\right],
\end{eqnarray}
In this work we have mainly dealt with a valence-linked expansion, which
means that an $n$-body valence space wave function is defined through 
\begin{equation}
A^{\dagger}\ket{C}=a_1^{\dagger}a_2^{\dagger}\dots a_n^{\dagger}\ket{C},
\end{equation}
where $\ket{C}$ refers to e.g.\ a closed $^{16}$O core wave function.


The diagrams rules in the uncoupled scheme are rather simple to obtain through
use of Wicks theorem and the symmetry properties of the interaction
$V$ or $G$. These rules are rather well-known, see \cite{ko90}, and we will
simply state them in the following.
\begin{itemize}
\item Draw all topological distinct diagrams and for
each interaction there is an antisymmetrized and normalized matrix element
\[
\bra{ab}V\ket{cd}_{AS}=\bra{ab}V\ket{cd}-\bra{ab}V\ket{dc},
\]
or in terms of the $G$-matrix
\[
\bra{ab}G\ket{cd}_{AS}=\bra{ab}G\ket{cd}-\bra{ab}G\ket{dc}.
\]
If not stated differently, througout this work we will always employ
anti-symmetrized matrix elements, and thence skip the label $AS$.

\item 
From the time integrations in the time evolution operator we 
obtain  an energy denominator
\begin{equation}
\frac{1}{\omega -(\sum\varepsilon_p -\sum\varepsilon_h )},
\end{equation}
between successive interactions. Here $\omega$ is the unperturbed energy
of the valence particles, or
the starting energy, and $\sum\varepsilon_p$
and $\sum\varepsilon_h$ represent the sum over intermediate particle
and hole sp unperturbed energies between two interactions.
\item
By applying Wicks theorem we obtain for each diagram an overall factor
\begin{equation}
\frac{\left(-1\right)^{n_h+n_l+n_c+n_{exh}}}{2^{n_{ep}}},
\label{eq:ovfactor}
\end{equation}
where $n_h$ is the number of hole lines in a diagram, $n_l$ is the number 
of closed loops, $n_c$ is the numbers of crossings of different
external lines as they trace through the diagram and $n_{exh}$ is the
number of external hole lines which continuously trace through a diagram.
Finally, $n_{ep}$ is the number of pairs of lines which start at the same
interaction and end at the same interaction.
\end{itemize}
As an example we list the expressions for the diagrams shown in fig.\
\ref{fig:uncoupex}.
\begin{figure}[hbtp]
   \setlength{\unitlength}{1mm}
   \begin{picture}(140,90)
    \put(25,10){\epsfxsize=12cm \epsfbox{a1fig.eps}}
   \end{picture}
\caption{Examples of diagrams which contribute to the
effective interaction.}
\label{fig:uncoupex}
\end{figure}
Diagram (a), which is the 2p1h diagram, reads
\begin{equation}
{\displaystyle
\frac{1}{2}\sum_{p_1p_2h}\frac{\bra{\alpha h}G
\ket{p_1p_2}\bra{p_1p_2}G\ket{\beta h}}
{\varepsilon_{\beta}+\varepsilon_h -\varepsilon_{p_1}-\varepsilon_{p_2}}},
\label{eq:un2p1h}
\end{equation}
where we have $n_h=1$, $n_l=1$, $n_{exh}=0$ and $n_c=0$ giving rise to a factor
$+1$. Finally, we have two intermediate particle lines which start and end at the
same interaction, yielding $n_{ep}=1$.

Similarly, diagram (b), the 3p2h diagram, reads
\begin{equation}
{\displaystyle
-\frac{1}{2}\sum_{h_1h_2p}\frac{\bra{\alpha p}G
\ket{h_1h_2}\bra{h_1h_2}G\ket{\beta p}}
{\varepsilon_{h_1} +\varepsilon_{h_2}-\varepsilon_{\alpha}-\varepsilon_{p}}},
\label{eq:un3p2h}
\end{equation}
where we have $n_h=2$, $n_l=1$, $n_{exh}=0$ and $n_c=0$ giving rise to a factor
$-1$. Moreover, we have two intermediate hole lines which start and end at the
same interaction, yielding $n_{ep}=1$.

For both expressions, we sum freely over the quantum numbers of the intermediate
states, expressed through the labels $p_{1,2}h$ and $h_{1,2}p$.

For diagram (c), which is the third-order TDA diagram, we have
\begin{equation}
{\displaystyle\sum_{p_{1}p_{2}}\sum_{h_{1}h_{2}}}
\frac{\bra{\alpha p_2}G\ket{\gamma h_2}\bra{h_2p_1}G\ket{p_2h_1}
\bra{h_1\beta}G\ket{p_1\delta}}
{(\varepsilon_{\gamma}+\varepsilon_{h_{2}}-\varepsilon_{p_{2}}-\varepsilon_{\alpha})
(\varepsilon_{\gamma}+\varepsilon_{h_{1}}-
  \varepsilon_{p_{1}}-\varepsilon_{\alpha})},
\end{equation}
where we have $n_h=2$, $n_l=2$ and $n_{ep}=n_{exh}=n_c=0$.

Diagram (d) is a third-order contribution to the hole-hole or
particle-hole effective interaction and has an external hole which 
traces through the diagram, i.e.\ $n_{exh}=1$. Moreover, we have
$n_h=3$, $n_l=2$, $n_c=0$ and $n_{ep}=0$. The final expression in an
uncoupled scheme is then
\begin{equation}
{\displaystyle\sum_{hh_{1}h_{2}}}
{\displaystyle\sum_{p_{1}p_{2}}}
\frac{\bra{hh_2}G\ket{\beta p_2}\bra{p_2h_1}G\ket{h_2p_1}
\bra{\alpha p_1}G\ket{hh_2}}
{(\varepsilon_{h}+\varepsilon_{h_{1}}-
\varepsilon_{p_{1}}-\varepsilon_{\alpha})(\varepsilon_{h}+
\varepsilon_{h_{2}}-\varepsilon_{\alpha}-
\varepsilon_{p_{2}})}.
\end{equation}



The analytical expressions in an angular momentum representation
for many of the diagrams employed in our definition
of the \qbox and the effective operator can be
found in the works of Barrett and Kirson \cite{bk70} and
Kassis \cite{kassis}, 
though, here we have used a method
described by Kuo {\em et al.} \cite{kstop81}. By the use of this method,
the analytical expressions are in general simpler compared to
those obtained in the $m-$scheme, and moreover, they are less
time-consuming from a computational point of view. 
Since this method is extensively detailed in \cite{kstop81},
we will not give all details
in connection with the calculations of the various diagrams of fig.\
\ref{fig:uncoupex}, though certain basic steps will be presented.

Before we proceed we need to define so-called 
cross-coupled matrix elements (not normalized) which 
are defined as\newline
\begin{eqnarray}
\bra{(\alpha\rightarrow \gamma)JT}G
\ket{(\beta\rightarrow \delta)JT}=&(-)^{j_{\beta}+j_{\delta}+J+T}
\sqrt{(1+\delta_{j_{\alpha}j_{\beta}})(1+\delta_{j_{\gamma}j_{\delta}})}
\hat{J}\hat{T}\nonumber\\&
\times
{\displaystyle \sum_{J',T'}}
\left\{
\begin{array}{ccc}
       j_{\alpha}&j_{\gamma}&J\\j_{\delta}&j_{\beta}&J'
\end{array}
\right\}
\left\{
\begin{array}{ccc}
       \frac{1}{2}&\frac{1}{2}&T\\ \frac{1}{2}&\frac{1}{2}&T'
\end{array}
\right\}\nonumber\\& \times
(2J'+1)(2T'+1)
\bra{(\alpha\beta)J'T'}G\ket{(\delta\gamma)J'T'}
\label{eq:cross}
\end{eqnarray}

The normalized two-body wave function $\ket{(\alpha\beta)J}$ (omitting isospin)
in the above equation is given as
\begin{equation}
  \ket{(\alpha\beta)J}=\ket{(\alpha\beta)JM}=
  \frac{1}{\sqrt{1+\delta_{j_{\alpha}j_{\beta}}}}
  {\displaystyle \sum_{m_{\alpha} m_{\beta}}
  {\cal C}_{m_{\alpha}m_{\beta}M}^{j_{\alpha} j_{\beta} J}
  \ket{(j_{\alpha}m_{\alpha})(j_{\beta}m_{\beta})}},
  \label{eq:twobodywf}
\end{equation}
where $j_{\alpha ,\beta}$ and $m_{\alpha ,\beta}$
are the quantum numbers defining  a sp state and
${\cal C}$
is a standard Clebsch-Gordan coefficient defined according to the
Condon-Shortley phase convention. Both the two-body wave function and the
sp wave function transform under rotations as a spherical tensor.
In fig.\ \ref{fig:crosscouple} we show in terms of diagrams the meaning of
the coupling orders
$\bra{(\alpha\rightarrow \gamma)JT}G\ket{(\beta\rightarrow \delta)JT}$ and
$\bra{(\alpha\beta )JT}G\ket{(\gamma\delta )JT}$.
Eq.\ (\ref{eq:cross}) is just an example of one possible recoupling, another
possible recoupling
is
\newline
\begin{eqnarray}
\bra{(\alpha\rightarrow \delta)JT}G
\ket{(\beta\rightarrow \gamma)JT}=&(-)^{j_{\beta}+j_{\delta}+J+T}
\sqrt{(1+\delta_{j_{\alpha}j_{\beta}})(1+\delta_{j_{\gamma}j_{\delta}})}
\hat{J}\hat{T}\nonumber\\&
\times
{\displaystyle \sum_{J',T'}}
\left\{
\begin{array}{ccc}
       j_{\alpha}&j_{\delta}&J\\j_{\gamma}&j_{\beta}&J'
\end{array}
\right\}
\left\{
\begin{array}{ccc}
       \frac{1}{2}&\frac{1}{2}&T\\ \frac{1}{2}&\frac{1}{2}&T'
\end{array}
\right\}\nonumber\\& \times
(2J'+1)(2T'+1)(-)^{J'+T'}
\bra{(\alpha\beta)J'T'}G\ket{(\delta\gamma)J'T'}
\label{eq:cross2}
\end{eqnarray}

\begin{figure}[hbtp]
  \setlength{\unitlength}{1mm}
  \begin{picture}(80,80)
   \put(25,10){\epsfxsize=8cm \epsfbox{recouple.eps}}
   \end{picture}
\caption{The two-body matrix element coupled
according to $\bra{(\alpha\rightarrow \gamma )JT}G
\ket{(\beta\rightarrow \delta )JT}$
is shown in (a) while (b)
represents the matrix element
$\bra{(\alpha\beta )JT}G\ket{(\gamma\delta )JT}$.}
\label{fig:crosscouple}
\end{figure}
To illustrate the advantages of this method we present the
analytical expressions for the diagrams of fig.\ \ref{fig:uncoupex}.

In brief, the essence of the method of ref.\ 
\cite{kstop81},
for evaluating Goldstone diagrams 
in a coupled basis, is to rewrite  a diagram (or parts of a
diagram) into ladder forms, by opening selected internal lines. An internal line
consists of two pieces, an incoming and an outgoing one, which are coupled to 
a scalar. This coupling can be broken by a suitable recoupling transformation, and
the technique is to cut the ``correct'' internal line in order to factorize the
diagram into the form of a ladder diagram. To illustrate this, consider first the 
2p1h diagram of fig.\ \ref{fig:uncoupex}, diagram (a). 
First we couple the external lines
to a scalar, as shown in step (a) of fig.\ \ref{fig:2p1hex}. Note that the external
lines are coupled to zero total momentum. 
\begin{figure}[hbtp]
    \setlength{\unitlength}{1mm}
    \begin{picture}(140,90)
     \put(25,10){\epsfxsize=12cm \epsfbox{a3fig.eps}}
    \end{picture}
\caption{Various recouplings for the 2p1h diagram.}
\label{fig:2p1hex}
\end{figure}
This gives rise to a factor
$\frac{1}{\hat{j}_{\alpha}}$ for angular momentum 
and $\frac{1}{\sqrt{2}}$ for isospin. In our angular momentum
conventions, see eq.\  (\ref{eq:cross}) and fig.\ \ref{fig:crosscouple}, we let 
angular momentum couplings run from an outgoing line to an ingoing line, opposite to what
is done in ref.\ \cite{kstop81}. Next we cut open the internal hole line labeled $h$, as shown
in (b) of fig.\ \ref{fig:2p1hex}. For the internal lines we obtain a factor
$\hat{j}_h$ for angular momentum 
and $\sqrt{2}$ for isospin, see eq.\ (45) of ref.\  
\cite{kstop81}. Finally, we recouple this diagram as shown in
(c) of fig.\ \ref{fig:2p1hex}. The latter is clearly in a ladder form and its 
angular momentum part reads (omitting isospin) 
\begin{equation}
(a)= \frac{1}{\hat{j}_{\alpha}}\hat{j}_h\sum_{J}\hat{J}^2
\left\{
\begin{array}{ccc}
      j_{\alpha}&j_{\alpha}&0\\j_h&j_h&0\\J&J&0
\end{array}
\right\}
\bra{(\alpha h)J}G
\ket{(p_1p_2)J}\bra{(p_1p_2)J}G\ket{(\beta h)J},
\end{equation}
which reduces to,
using the properties of the Wigner $9j$ symbols,
\begin{equation}
(a)= \frac{1}{\hat{j}_{\alpha}^2}\sum_{J}\hat{J}^2
\bra{(\alpha h)J}G
\ket{(p_1p_2)J}\bra{(p_1p_2)J}G\ket{(\beta h)J}.
\end{equation}
We can then write out the final expression for the 
2p1h diagram (including isospin) as
\begin{equation}
     (a)= \frac{1}{4\hat{j}_{\alpha}^2}\sum_{JT}\hat{J}^2\hat{T}^2
     \frac{\bra{(\alpha h)JT}G(\omega_1)
     \ket{(p_1p_2)JT}\bra{(p_1p_2)JT}G(\omega_2)\ket{(\beta h)JT}}
     {\varepsilon_{\beta} +\varepsilon_h -\varepsilon_{p_1}-\varepsilon_{p_2}}.
\end{equation}
In this equation we have also introduced the energy dependent $G$-matrix
(recall the discussion in section 5), through the variables $\omega_1$ 
and $\omega_2$. This is due to the fact that the various interactions
have starting energies which need not coincide with those of the model
space states. In the above, $\omega_1= \varepsilon_{\beta}+\varepsilon_h$
and $\omega_2= \varepsilon_{\beta}+\varepsilon_h$, which, unless
$\varepsilon_h=0$ differs from the model space energies
$\varepsilon_{\alpha ,\beta}$. As the $G$-matrix becomes weaker for more
negative starting energies, the energy dependence of the $G$-matrix serves
to reduce the magnitude of higher order contributions.

The expression for the 3p2h diagram in fig.\ \ref{fig:uncoupex} can be obtained
in a similar manner as the 2p1h diagram. Here we cut open the internal particle
line and obtain
\begin{equation}
    (b)= -\frac{1}{4\hat{j}_{\alpha}^2}\sum_{JT}\hat{J}^2\hat{T}^2
    \frac{\bra{(\alpha p)JT}G(\omega_1)
    \ket{(h_1h_2)JT}\bra{(h_1h_2)JT}G(\omega_2)\ket{(\beta p)JT}}
    {\varepsilon_{h_1} +\varepsilon_{h_2} -\varepsilon_p-\varepsilon_{\alpha}}.
\end{equation}
Here we have $\omega_1= \varepsilon_{h_1}+\varepsilon_{h_2}$
and $\omega_2= \varepsilon_{h_1}+\varepsilon_{h_2}
+\varepsilon_{\beta}-\varepsilon_{\alpha}$.



Diagram (c) of fig.\ \ref{fig:uncoupex}, the third-order TDA diagram, 
can be recoupled
to give the diagram (b) in fig.\ \ref{fig:tdadiag}. 
\begin{figure}[hbtp]
    \setlength{\unitlength}{1mm}
     \begin{picture}(140,90)
      \put(25,10){\epsfxsize=12cm \epsfbox{tda.eps}}
     \end{picture}
\caption{The TDA diagram (a) can be recoupled  as shown in (b).}
\label{fig:tdadiag}
\end{figure}
This recoupling gives rise to the following angular momentum relation
(omitting isospin) \newline
\begin{equation}
\bra{(\alpha\beta)J}TDA\ket{(\gamma\delta)J}=
(2J+1)\sum_{J'}(2J'+1)
\left\{
\begin{array}{ccc}
      j_{\alpha}&j_{\beta}&J\\j_{\gamma}&j_{\delta}&J\\J'&J'&0
\end{array}
\right\}
\bra{(\alpha\rightarrow \gamma)J'}TDA
\ket{(\beta\rightarrow \delta)J'}.
\end{equation}
TDA in the above expressions stands for all three interactions.
\begin{figure}[hbtp]
    \setlength{\unitlength}{1mm}
    \begin{picture}(140,120)
     \put(25,10){\epsfxsize=12cm \epsfbox{a5fig.eps}}
     \end{picture}
\caption{Various recouplings for the third-order TDA diagram. See text for further
details.}
\label{fig:tdaex}
\end{figure}
The next step is to cut open the internal particle-hole pairs, as shown
in (a) of fig.\ \ref{fig:tdaex}. This gives an additional factor $(-)^{2j_h}$ for every 
internal particle-hole pair which is cut open \cite{kstop81}, though this factor
reduces to $+1$ when we introduce isospin. We are then left with three
crosscoupled matrix elements defined according to eq.\ (\ref{eq:cross}) and shown 
in (b) of fig.\ \ref{fig:tdaex}.
The final expression for the TDA diagram (including isospin)
is then given by
\begin{eqnarray}
&{\displaystyle \frac{1}
{\sqrt{(1+\delta_{j_{\alpha}j_{\beta}})(1+\delta_{j_{\gamma}j_{\delta}})}}}
{\displaystyle\sum_{p_{1},p_{2},h_{1},h_{2}}^{}}
{\displaystyle\sum_{J',T'}^{}}
\left\{
\begin{array}{ccc}
      j_{\alpha}&j_{\beta}&J\\j_{\delta}&j_{\gamma}&J'
\end{array}
\right\}
\left\{
\begin{array}{ccc}
\frac{1}{2}&\frac{1}{2}&T\\\frac{1}{2}&\frac{1}{2}&T'
\end{array}
\right\}\nonumber\\
&\nonumber\\&\times{\displaystyle
\frac{(-)^{j_{\beta}+j_{\gamma}+1+J+J'+T+T'}}
{\hat{J}'\hat{T}'}}
\left\{(\omega+\varepsilon_{h_{2}}-\varepsilon_{p_{2}}-\varepsilon_{\delta})
(\omega+\varepsilon_{h_{1}}-
  \varepsilon_{p_{1}}-\varepsilon_{\delta})\right\}^{-1}
\nonumber\\&\nonumber
\\
&\times\bra{(\alpha\rightarrow \gamma)J'T'}G(\omega_{1})
 \ket{(p_2\rightarrow h_2)J'T'}
 \bra{(h_2\leftarrow p_2)J'T'}G(\omega_{2})
 \ket{(p_1\rightarrow h_1)J'T'} \nonumber\\&\nonumber \\
&\times\bra{(h_1\leftarrow p_1)J'T'}G(\omega_{3})
\ket{(\beta\rightarrow \delta)J'T'}
\label{eq:tda}
\end{eqnarray}
where $\omega=\varepsilon_{\gamma}+\varepsilon_{\delta}$
and the energy variables are
$\omega_{1}=\omega+\varepsilon_{h_{2}}-\varepsilon_{\delta}$,
 $\omega_{2}=\omega+\varepsilon_{h_{1}}+
\varepsilon_{h_{2}}-\varepsilon_{\alpha}-\varepsilon_{\delta}$ and
$\omega_{3}=\omega+\varepsilon_{h_{1}}-\varepsilon_{\delta}$.
We finally note that, compared to the
expression presented by Barrett and Kirson
\cite{bk70}, diagram 15 in their appendix, the number of
summations is reduced by a factor of two, and the
number of $6j$ coefficients are reduced from eight to
two in eq.\ (\ref{eq:tda}).
Furthermore, the cross-coupled matrix elements can be calculated
once and for all.

To calculate diagram (d) of fig.\ \ref{fig:uncoupex}, we proceed in 
a similar way as for the 2p1h diagram. 
We first recouple the external lines 
to a scalar, thereafter we recouple the
external lines to the internal hole line labelled $h$. The next step
is to cut open the internal particle-hole pairs, and finally 
we are left with three cross-coupled matrix elements of the type defined in 
eq.\ (\ref{eq:cross}). 
The final expression for diagram (d) is then
\begin{equation}
\begin{array}{ll}
(a)=&{\displaystyle\sum_{JT}}
{\displaystyle\sum_{hh_{1}h_{2}}}
{\displaystyle\sum_{p_{1}p_{2}}}
\frac{1}{\hat{J}\hat{T}2\hat{j}_{\beta}^2}
\left((\varepsilon_{h}+\varepsilon_{h_{1}}-
\varepsilon_{p_{1}}-\varepsilon_{\alpha})(\varepsilon_{h}+
\varepsilon_{h_{2}}-\varepsilon_{\alpha}-
\varepsilon_{p_{2}})\right)^{-1}\\
&\\
&\times\bra{(\alpha\rightarrow h)JT}G(\omega_{1})
       \ket{(p_1\rightarrow h_1)JT}
       \bra{(p_{2}\rightarrow h_{2})JT}G(\omega_{2})
       \ket{(h_{1}\leftarrow p_{1})JT} \\
&\\
&\times
       \bra{(h\leftarrow \beta)JT}G(\omega_{3})
       \ket{(h_2\leftarrow p_{2})JT},\\
&\\
\end{array}
\end{equation}
with the starting energies given by
\[
\begin{array}{ll}
\omega_{1}=&\varepsilon_{h_{1}}+\varepsilon_{h}\\
\omega_{2}=&\varepsilon_{h_{1}}+\varepsilon_{h}+
\varepsilon_{h_{2}}-\varepsilon_{\alpha}\\
\omega_{3}=&\varepsilon_{h_{2}}+\varepsilon_{h}+
\varepsilon_{\beta}-\varepsilon_{\alpha},
\end{array}
\]




\subsection{List of Goldstone diagrams}

The one-body and two-body linked-valence diagrams through third
order included in the evaluation of
the $\hat{Q}$-box are shown in
figs.\ \ref{fig:onebody} and \ref{fig:twobody}, respectively.
Each topology is a sum of one to four diagrams, the one given in
figs.\ \ref{fig:onebody}  and  \ref{fig:twobody}
plus all those that can be obtained through the
exchange of the external pairs of lines. 
We omit
diagrams involving Hartree-Fock contributions.

Many of the third-order two-body diagrams give rise to
non-hermitian contributions
to the effective interaction. However, these diagrams are
generally followed by similar diagrams, and the sum of the first
diagram and its "companion" results in a hermitian contribution.
The first of these diagrams is then labeled with an (A) and the
second with a (B) in figure \ref{fig:twobody}.

\begin{figure}[hbtp]
   \setlength{\unitlength}{1mm}
    \begin{picture}(140,150)
      \put(25,10){\epsfxsize=12cm \epsfbox{onebd1.eps}}
    \end{picture}
\caption{One-body Goldstone diagrams without a spectator valence line
included in the evaluation of the \qbox .}
\label{fig:onebody}
\end{figure}

\begin{figure}[hbtp]
    \setlength{\unitlength}{1mm}
    \begin{picture}(140,150)
      \put(25,10){\epsfxsize=12cm \epsfbox{onebd2.eps}}
     \end{picture}
\begin{center}{Fig.\ \ref{fig:onebody} - continued}\end{center}
\end{figure}


\begin{figure}[hbtp]
    \setlength{\unitlength}{1mm}
    \begin{picture}(140,150)
      \put(25,10){\epsfxsize=12cm \epsfbox{onebd3.eps}}
     \end{picture}
\begin{center}{Fig.\ \ref{fig:onebody} - continued}\end{center}
\end{figure}



\begin{figure}[hbtp]
   \setlength{\unitlength}{1mm}
   \begin{picture}(140,150)
      \put(25,10){\epsfxsize=12cm \epsfbox{twobd1.eps}}
   \end{picture}
\caption{Two-body Goldstone diagrams
included in the evaluation of the \qbox .}
\label{fig:twobody}
\end{figure}

\begin{figure}[hbtp]
   \setlength{\unitlength}{1mm}
   \begin{picture}(140,150)
      \put(25,10){\epsfxsize=12cm \epsfbox{twobd2.eps}}
   \end{picture}
\begin{center}{Fig.\ \ref{fig:twobody} - continued}\end{center}
\end{figure}

\begin{figure}[hbtp]
   \setlength{\unitlength}{1mm}
   \begin{picture}(140,150)
      \put(25,10){\epsfxsize=12cm \epsfbox{twobd3.eps}}
   \end{picture}
\begin{center}{Fig.\ \ref{fig:twobody} - continued}\end{center}
\end{figure}

\begin{figure}[hbtp]
   \setlength{\unitlength}{1mm}
   \begin{picture}(140,150)
      \put(25,10){\epsfxsize=12cm \epsfbox{twobd4.eps}}
   \end{picture}
\begin{center}{Fig.\ \ref{fig:twobody} - continued}\end{center}
\end{figure}

\begin{figure}[hbtp]
   \setlength{\unitlength}{1mm}
   \begin{picture}(140,150)
      \put(25,10){\epsfxsize=12cm \epsfbox{twobd5.eps}}
   \end{picture}
\begin{center}{Fig.\ \ref{fig:twobody} - continued}\end{center}
\end{figure}

\begin{figure}[hbtp]
   \setlength{\unitlength}{1mm}
   \begin{picture}(140,100)
      \put(25,10){\epsfxsize=12cm \epsfbox{twobd6.eps}}
   \end{picture}
\begin{center}{Fig.\ \ref{fig:twobody} - continued}\end{center}
\end{figure}


\clearpage


\section{Numerical evaluation of the $T$-matrix and the $G$-matrix}

In this section we outline how to calculate the $G$-matrix for nuclear
matter, the $G_F$-matrix used in the evaluation of the $G$-matrix
for finite nuclei and the $T$-matrix through matrix inversion.

Let us first consider uncoupled channels, e.g.\ the $^1S_0$ partial wave.
The one-dimensional equation for either the $G$-matrix or the $T$-matrix
can be written as 
\begin{equation}
     G(kk'K) = V(kk'K)- \int_0 ^{\infty} F(kqK)G(qk'K)dq,
\end{equation}
where
\begin{equation}
    F(kqK) =\frac{2}{\pi} q^2\frac{V(kqK)Q(qK)}{q^2 - \omega -K^2}.
\end{equation}
For nuclear matter we have that the Pauli operator $Q$ is given by the
angle-average BHF operator discussed in subsection 5.3. For finite nuclei
we set $Q=1$, since we are only interested in the so-called free $G$-matrix
$G_F$. The equation for $G_F$ differs however from that of the $T$-matrix,
which also employs $Q=1$, in the choice of starting energy. For $G_F$ the
starting energy is negative, whereas for $T$ it is positive. The above is 
an integral equation in the relative momentum $k$, and the $G$- or $T$-matrices
are defined for a given center-of-mass momentum $K$. Below we will skip 
this variable. We assume also that $G$, $V$ and $Q$ are continuous in $k$
and that their derivatives are also continuous. Further, we expect high-lying
momenta to play a negligible role due to the denominator in $F$, and 
replace then the infinity limit with a finite value of $q$. In this way we
can solve our problem numerically and rewrite the equation for $G$ as  
\begin{equation} 
\label{eq:numintegral}
   G(k_i,k_j)  =  V(k_i,k_j)- \sum _{n=1} ^N F(k_i,k_n)G(k_n,k_j)w_n.
\end{equation}  
The numerical integration is done by using Gaussian points $k_i$ and weights
$w_i$ 
\begin{eqnarray}  
\label{eq:gaussavb}
k_i & = C \tan(\frac{(1+x_i)\pi }{4}) \\
w_i & = C v_i\frac{\pi}{4} \cos^{-2} (\frac{(1+x_i)\pi }{4}),  \nonumber
\end{eqnarray}                                          
where $-1\leq x \leq 1$ and $C$ is a constant chosen so as to optimize
the numerical grid. It is convenient to define $G$, $F$ and $V$ as $N\times N$
matrices in $k$-space ($N$ is the number of mesh points)
$F_{ij}=F(k_i,k_j)w_j$, $V_{ij}=V(k_i,k_j)$ and
$G_{ij}=G(k_i,k_j)$. This allows us to rewrite the equation for $G$ as a
matrix equation 
\begin{eqnarray}
\left( \begin{array}{cccc}
G_{11} & G_{12} &  \cdots &  G_{1N} \\
G_{21} & G_{22} &  \cdots &  G_{2N} \\
\vdots & \vdots &         & \vdots  \\
G_{N1} & G_{N2} &  \cdots &   G_{NN}   
\end{array} \right )
& = &  
\left( \begin{array}{cccc}
V_{11} & V_{12} & \cdots  &  V_{1N} \\
V_{21} & V_{22} & \cdots   &  V_{2N} \\
\vdots & \vdots &         & \vdots  \\
V_{N1} & V_{N2} & \cdots  &  V_{NN}   
\end{array} \right ) \nonumber  
\\ 
& - &
\left( \begin{array}{cccc}
F_{11} & F_{12} & \cdots  & F_{1N} \\
F_{21} & F_{22} & \cdots  & F_{2N} \\
\vdots & \vdots &         & \vdots \\
F_{N1} & F_{N2} & \cdots  & F_{NN}   
\end{array} \right )
\left( \begin{array}{cccc}
G_{11} & G_{12} &  \cdots &  G_{1N} \\
G_{21} & G_{22} &  \cdots &  G_{2N} \\
\vdots & \vdots &         & \vdots  \\
G_{N1} & G_{N2} &  \cdots &   G_{NN}   
\end{array} \right ).
\end{eqnarray}
or in a more compact form
\begin{eqnarray}
 {\bf G}  & = & {\bf V} - {\bf F}{\bf G}  \nonumber \\
({\bf \hat{1}} +{\bf F}) {\bf G}& = & {\bf V} \nonumber \\
{\bf G} & = & ({\bf \hat{1} +F})^{-1} {\bf V }.
\end{eqnarray}                                                                        
The $G$-matrix can now be obtained for a given starting energy
and relative momenta by inverting a matrix 
$({\bf \hat{1} +F})$ with the following matrix elements
\[
\delta _{ij} + \frac{2}{\pi}V_{ij}\frac{Q_j}{ k_j ^2-\omega }  k_j ^2 w_j.
\] 
Hitherto we have only considered uncoupled partial waves, though the extension
to coupled channels is easily accomplished by defining a $2N\times 2N$ matrix
$G_{ll'}$ (this matrix represents   
$G^{\alpha}_{ll'}(k_ik_jK\omega)$) 
The couplings between various angular momenta are then accounted through the
solution of 
\begin{equation}
\label{eq:matrlign}           
\left( \begin{array}{lr}
G_{ll}  & G_{ll'} \\
G_{l'l} & G_{l'l'} 
\end{array} \right)
=
\left( \begin{array}{lr}
V_{ll}   & V_{ll'} \\
V_{l'l}  & V_{l'l'} 
\end{array} \right)
-
\left( \begin{array}{lr}
F_{ll}  & F_{ll'} \\
F_{l'l} & F_{l'l'} 
\end{array} \right)
\left( \begin{array}{lr}
G_{ll}  & G_{ll'} \\
G_{l'l} & G_{l'l'} 
\end{array} \right),
\end{equation}
which means that we need to invert a $(2N\times 2N)$-matrix.
As an example, consider the $^3S_1$-$^3D_1$ channel. There we have four
different $G$-matrices, given by the partial wave combinations
$^3S_1$-$^3S_1$, $^3S_1$-$^3D_1$, $^3D_1$-$^3S_1$ and $^3D_1$-$^3D_1$.

A numerical stable and accurate (to 5--6 leading digits)
result for both $G$ in nuclear matter and $G_F$ for finite
nuclei is obtained using $40-50$ mesh points.
The above method is rather  simple and efficient to implement 
numerically, and is used in our computation of $G$.

\clearpage

\section{List of two-body matrix elements}
We list here the matrix elements for the various mass areas
discussed in the text. These matrix elements have been obtained
with the Bonn A, B and C potentials of table A.1 of ref.\
\cite{mac89}. The folded-diagram method with a 
third-order \qbox has been used. For the tin isotopes we list only the $T=1$
results obtained with the Bonn A potential.
In tables \ref{tab:sdveff} and \ref{tab:fpveff} we list also the
empirically derived matrix elements of Brown and co-workers
\cite{brown88,richt91}. Although our matrix elements and those
of refs.\ \cite{brown88,richt91} are not directly 
comparable (this is due to the fact that the empirical 
matrix elements may contain intruder state components since 
a fit was made to the available data), we see from these tables that the
differences are rather small, especially in the
$fp$-shell where they are on the order of some few keV only. For the
$fp$-shell, we find, without accounting for differences
in sp energies, an average difference between the matrix elements
with the Bonn C potential and the matrix elements of Richter
{\em et al.} of $125$ keV.
For the $fp$-shell, we choose to compare with the
interaction labelled $FPMI3$.

The reader should note that the authors of refs.\ \cite{brown88,richt91}
use slightly different sp energies. These are
$\varepsilon_{0d_{5/2}}=-3.9478$ MeV,
$\varepsilon_{1s_{1/2}}=-3.1635$ MeV,
$\varepsilon_{0d_{3/2}}=1.6466$ MeV for the $sd$-shell \cite{brown88},
and 
$\varepsilon_{0f_{7/2}}=-8.3637$ MeV,
$\varepsilon_{0f_{5/2}}=-2.1539$ MeV,
$\varepsilon_{1p_{3/2}}=-6.3325$ MeV,
$\varepsilon_{1p_{1/2}}=-4.4068$ MeV for the
$fp$-shell \cite{richt91}.
These should be compared to those used in this work 
$\varepsilon_{0d_{5/2}}=-4.14$ MeV,
$\varepsilon_{1s_{1/2}}=-3.27$ MeV,
$\varepsilon_{0d_{3/2}}=0.94$ MeV for the $sd$-shell,
and 
$\varepsilon_{0f_{7/2}}=-8.36$ MeV,
$\varepsilon_{0f_{5/2}}=-1.86$ MeV,
$\varepsilon_{1p_{3/2}}=-6.26$ MeV,
$\varepsilon_{1p_{1/2}}=-4.46$ MeV for the
$fp$-shell. 

Finally, in table \ref{tab:tins}, we list the effective $T=1$
interactions
for the light tin isotopes discussed in section 6.5\footnote{For
all mass areas except the $sd$-shell, there are more than
hundred matrix elements. A more convenient listing of the various
effective interactions can be obtained through electronic
post to one of us (MHJ) at mhjensen@fys.uio.no (internet).}.

\begin{table}[hbtp]
\caption{The $sd$-shell two-body effective interactions with $^{16}$O as core.
The labels A, B and C refer to the results with the Bonn A, B and
C potentials. Emp are the matrix elements of Brown {\em et al.},
see ref.\ [73]. 
All entries in MeV.}
\small
\begin{center}
\begin{tabular}{llllllrrrr}
\hline
$2j_{a}$&$2j_{b}$&$2j_{c}$&$2j_{d}$&$2T$&$2J$&
\multicolumn{1}{c}{A}&
\multicolumn{1}{c}{B}&
\multicolumn{1}{c}{C}&
\multicolumn{1}{c}{Emp}\\\hline
  5& 5& 5& 5& 0& 2& -1.4315& -1.1773& -1.0001&-1.6321\\
  5& 5& 5& 3& 0& 2&  3.1790&  3.0167&  2.8543&2.5435\\
  5& 5& 3& 3& 0& 2&  1.7666&  1.6503&  1.5547&.7221\\
  5& 5& 3& 1& 0& 2&   .3628&   .3187&   .2922&1.1026\\
  5& 5& 1& 1& 0& 2&  -.8749&  -.8103&  -.7593&-1.1756\\
  5& 3& 5& 3& 0& 2& -6.5104& -6.1933& -5.8803&-6.5058\\
  5& 3& 3& 3& 0& 2&  -.0200&   .0811&   .1523&.5647\\
  5& 3& 3& 1& 0& 2&  1.7250&  1.7993&  1.8374&1.7080\\
  5& 3& 1& 1& 0& 2&  1.8887&  1.8113&  1.7418&2.1042\\
  3& 3& 3& 3& 0& 2& -1.3404& -1.1359&  -.9805&-1.4151\\
  3& 3& 3& 1& 0& 2&  -.8402&  -.8319&  -.8360&-.3983\\
  3& 3& 1& 1& 0& 2&   .0405&   .0048&  -.0171&.0275\\
  3& 1& 3& 1& 0& 2& -3.3056& -3.2149& -3.1355&-4.2930\\
  3& 1& 1& 1& 0& 2&  -.2441&  -.3513&  -.4181&-1.2501\\
  1& 1& 1& 1& 0& 2& -3.3313& -3.1451& -2.9803&-3.2628\\
  5& 3& 5& 3& 0& 4& -4.5004& -4.3628& -4.2156&-3.8253\\
  5& 3& 5& 1& 0& 4& -1.2555& -1.1941& -1.1508&-.0968\\
  5& 3& 3& 1& 0& 4& -1.4793& -1.4065& -1.3542&-.2832\\
  5& 1& 5& 1& 0& 4&  -.4109&  -.3142&  -.2632&-1.4474\\
  5& 1& 3& 1& 0& 4& -2.7050& -2.6508& -2.5870&-2.0664\\
  3& 1& 3& 1& 0& 4& -1.3883& -1.2959& -1.2283&-1.8194\\
  5& 5& 5& 5& 0& 6&  -.8478&  -.7239&  -.6200&-1.5012\\
  5& 5& 5& 3& 0& 6&  2.1769&  2.0897&  1.9916&2.2216\\
  5& 5& 5& 1& 0& 6& -1.4992& -1.4574& -1.4132&-1.2420\\
  5& 5& 3& 3& 0& 6&   .8466&   .7837&   .7291&1.8949\\
\hline
\end{tabular}
\end{center}
\label{tab:sdveff}
\end{table}
\clearpage
\small
\begin{center}
\begin{tabular}{llllllrrrr}
\hline
$2j_{a}$&$2j_{b}$&$2j_{c}$&$2j_{d}$&$2T$&$2J$&
\multicolumn{1}{c}{A}&
\multicolumn{1}{c}{B}&
\multicolumn{1}{c}{C}&
\multicolumn{1}{c}{Emp}\\\hline
  5& 3& 5& 3& 0& 6& -1.0712&  -.9881&  -.9042&-.5377\\
  5& 3& 5& 1& 0& 6&  1.0367&  1.0201&   .9982&1.2032\\
  5& 3& 3& 3& 0& 6&  2.1625&  2.1363&  2.0980&2.0337\\
  5& 1& 5& 1& 0& 6& -3.6000& -3.4532& -3.2943&-3.8598\\
  5& 1& 3& 3& 0& 6&   .1668&   .1476&   .1331&.1887\\
  3& 3& 3& 3& 0& 6& -2.9026& -2.7918& -2.6684&-2.8842\\
  5& 3& 5& 3& 0& 8& -4.4330& -4.2504& -4.0678&-4.5062\\
  5& 5& 5& 5& 0&10& -3.6858& -3.5261& -3.3532&-4.2256\\
  5& 5& 5& 5& 2& 0& -2.5418& -2.4640& -2.4660&-2.8197\\
  5& 5& 3& 3& 2& 0& -2.9807& -2.9531& -2.9448&-3.1856\\
  5& 5& 1& 1& 2& 0& -1.0885& -1.0542& -1.0407&-1.3247\\
  3& 3& 3& 3& 2& 0& -1.1624& -1.1046& -1.1095&-2.1845\\
  3& 3& 1& 1& 2& 0&  -.7911&  -.7734&  -.7684&-1.0835\\
  1& 1& 1& 1& 2& 0& -2.0617& -1.9866& -1.9592&-2.1246\\
  5& 3& 5& 3& 2& 2&  -.4249&  -.3983&  -.4008&1.0334\\
  5& 3& 3& 1& 2& 2&  -.0304&  -.0148&  -.0079&-.1874\\
  3& 1& 3& 1& 2& 2&   .3994&   .3767&   .3343&.6066\\
  5& 5& 5& 5& 2& 4&  -.9932&  -.9857&  -.9906&-1.0020\\
  5& 5& 5& 3& 2& 4&  -.1394&  -.1530&  -.1706&-.2828\\
  5& 5& 5& 1& 2& 4&  -.7957&  -.7790&  -.7635&-.8616\\
  5& 5& 3& 3& 2& 4&  -.9399&  -.9271&  -.9122&-1.6221\\
  5& 5& 3& 1& 2& 4&   .8477&   .8240&   .8025&.6198\\
  5& 3& 5& 3& 2& 4&  -.4043&  -.3996&  -.4092&-.3248\\
  5& 3& 5& 1& 2& 4&  -.2469&  -.2297&  -.2263&-.4770\\
  5& 3& 3& 3& 2& 4&  -.9871&  -.9573&  -.9317&-.6149\\
  5& 3& 3& 1& 2& 4&   .6449&   .6358&   .6235&.5247\\
\hline
\end{tabular}
\end{center}
\begin{center}{Table \ref{tab:sdveff} - continued}\end{center}
\clearpage
\small
\begin{center}
\begin{tabular}{llllllrrrr}
\hline
$2j_{a}$&$2j_{b}$&$2j_{c}$&$2j_{d}$&$2T$&$2J$&
\multicolumn{1}{c}{A}&
\multicolumn{1}{c}{B}&
\multicolumn{1}{c}{C}&
\multicolumn{1}{c}{Emp}\\\hline
  5& 1& 5& 1& 2& 4& -1.2335& -1.2159& -1.2084&-.8183\\
  5& 1& 3& 3& 2& 4&  -.6317&  -.6087&  -.5940&-.4041\\
  5& 1& 3& 1& 2& 4&  1.4633&  1.4327&  1.3981&1.9410\\
  3& 3& 3& 3& 2& 4&   .1427&   .1041&   .0663&-0.0665\\
  3& 3& 3& 1& 2& 4&   .1787&   .1753&   .1761&.5154\\
  3& 1& 3& 1& 2& 4&  -.2767&  -.2791&  -.2955&-.4064\\
  5& 3& 5& 3& 2& 6&   .5050&   .4364&   .3750&.5898\\
  5& 3& 5& 1& 2& 6&  -.1021&  -.1046&  -.0974&-.6741\\
  5& 1& 5& 1& 2& 6&   .2781&   .2565&   .2223&.7626\\
  5& 5& 5& 5& 2& 8&   .0356&   .0026&  -.0290&-.1641\\
  5& 5& 5& 3& 2& 8& -1.4942& -1.4410& -1.3867&-1.2363\\
  5& 3& 5& 3& 2& 8& -1.6941& -1.6455& -1.6074&-1.4497\\
\hline
\end{tabular}
\end{center}
\begin{center}{Table \ref{tab:sdveff} - continued}\end{center}

\clearpage



\begin{table}[hbtp]
\caption{The $fp$-shell two-body effective interactions 
with $^{40}$Ca as core.
The labels A, B and C refer to the results with the Bonn A, B and
C potentials. Emp are the matrix elements of Richter {\em et al.},
see ref.\ [84].
All entries in MeV.}
\small
\begin{center}
\begin{tabular}{llllllrrrr}
\hline
$2j_{a}$&$2j_{b}$&$2j_{c}$&$2j_{d}$&$2T$&$2J$&
\multicolumn{1}{c}{A}&
\multicolumn{1}{c}{B}&
\multicolumn{1}{c}{C}&
\multicolumn{1}{c}{Emp}
\\\hline
  7& 7& 7& 7& 0& 2& -1.3648& -1.1529&  -.9966&  -.7676\\
  7& 7& 7& 5& 0& 2&  2.0982&  1.9577&  1.8225&  1.8940\\
  7& 7& 5& 5& 0& 2&  1.0393&   .9626&   .9023&  1.0710\\
  7& 7& 5& 3& 0& 2&   .1409&   .1143&   .1006&   .2270\\
  7& 7& 3& 3& 0& 2&  -.6701&  -.6089&  -.5563&  -.4268\\
  7& 7& 3& 1& 0& 2&   .5128&   .4685&   .4280&   .3920\\
  7& 7& 1& 1& 0& 2&   .1302&   .1042&   .0877&   .1840\\
  7& 5& 7& 5& 0& 2& -5.3926& -5.0793& -4.7837& -3.7062\\
  7& 5& 5& 5& 0& 2&  -.5502&  -.4528&  -.3718&  -.4160\\
  7& 5& 5& 3& 0& 2&  1.3780&  1.4309&  1.4515&   .9530\\
  7& 5& 3& 3& 0& 2&   .9914&   .9209&   .8550&   .8710\\
  7& 5& 3& 1& 0& 2& -1.8391& -1.7577& -1.6710& -1.4490\\
  7& 5& 1& 1& 0& 2&   .1946&   .2316&   .2549&   .1680\\
  5& 5& 5& 5& 0& 2& -1.0273&  -.8585&  -.7368&  -.0120\\
  5& 5& 5& 3& 0& 2&  -.3557&  -.3481&  -.3505&  -.2920\\
  5& 5& 3& 3& 0& 2&   .0568&   .0287&   .0101&   .0400\\
  5& 5& 3& 1& 0& 2&  -.1239&  -.0882&  -.0614&  -.1590\\
  5& 5& 1& 1& 0& 2&  -.2665&  -.2411&  -.2266&  -.0840\\
  5& 3& 5& 3& 0& 2& -2.6432& -2.5716& -2.5208& -2.1727\\
  5& 3& 3& 3& 0& 2&  -.1029&  -.1470&  -.1642&  -.0730\\
  5& 3& 3& 1& 0& 2&   .8006&   .8691&   .9089&   .6200\\
  5& 3& 1& 1& 0& 2&  -.5586&  -.5602&  -.5748&  -.3930\\
  3& 3& 3& 3& 0& 2&  -.8391&  -.6673&  -.5677&  -.6655\\
  3& 3& 3& 1& 0& 2&  1.9727&  1.8707&  1.7732&  1.5540\\
  3& 3& 1& 1& 0& 2&  1.0701&   .9961&   .9285&   .7090\\
\hline
\end{tabular}
\end{center}
\label{tab:fpveff}
\end{table}
\clearpage
\small
\begin{center}
\begin{tabular}{llllllrrrr}
\hline
$2j_{a}$&$2j_{b}$&$2j_{c}$&$2j_{d}$&$2T$&$2J$&
\multicolumn{1}{c}{A}&
\multicolumn{1}{c}{B}&
\multicolumn{1}{c}{C}&
\multicolumn{1}{c}{Emp}
\\\hline
  3& 1& 3& 1& 0& 2& -2.8963& -2.7302& -2.5939& -2.2910\\
  3& 1& 1& 1& 0& 2&   .7883&   .8340&   .8519&   .6860\\
  1& 1& 1& 1& 0& 2& -1.4512& -1.3347& -1.2501& -1.0740\\
  7& 5& 7& 5& 0& 4& -3.4362& -3.3226& -3.2065& -2.7595\\
  7& 5& 7& 3& 0& 4&  -.7384&  -.7051&  -.6834&  -.7350\\
  7& 5& 5& 3& 0& 4&  -.7482&  -.6958&  -.6583&  -.7050\\
  7& 5& 5& 1& 0& 4&   .5841&   .5677&   .5547&   .4270\\
  7& 5& 3& 1& 0& 4&  -.7963&  -.7604&  -.7291&  -.6590\\
  7& 3& 7& 3& 0& 4&  -.6715&  -.5777&  -.5189&  -.3564\\
  7& 3& 5& 3& 0& 4& -1.3050& -1.2687& -1.2335& -1.1230\\
  7& 3& 5& 1& 0& 4&  1.0777&  1.0424&  1.0051&   .8380\\
  7& 3& 3& 1& 0& 4&  -.6269&  -.6127&  -.6049&  -.4810\\
  5& 3& 5& 3& 0& 4& -1.4871& -1.4338& -1.3844& -1.2448\\
  5& 3& 5& 1& 0& 4&   .4170&   .3824&   .3627&   .2550\\
  5& 3& 3& 1& 0& 4&  -.3428&  -.3114&  -.2995&  -.0710\\
  5& 1& 5& 1& 0& 4&  -.4739&  -.4059&  -.3552&  -.0810\\
  5& 1& 3& 1& 0& 4&   .5322&   .5406&   .5453&   .2410\\
  3& 1& 3& 1& 0& 4& -2.4943& -2.3287& -2.1921& -1.8970\\
  7& 7& 7& 7& 0& 6&  -.3909&  -.3069&  -.2404&  -.4498\\
  7& 7& 7& 5& 0& 6&  1.2049&  1.1408&  1.0768&  1.0050\\
  7& 7& 7& 3& 0& 6&  -.6178&  -.5939&  -.5680&  -.8315\\
  7& 7& 7& 1& 0& 6&   .5610&   .5471&   .5322&   .8315\\
  7& 7& 5& 5& 0& 6&   .7076&   .6631&   .6218&   .5170\\
  7& 7& 5& 3& 0& 6&  -.0494&  -.0549&  -.0578&  -.0870\\
  7& 7& 5& 1& 0& 6&   .0047&  -.0003&  -.0020&  -.0330\\
\hline
\end{tabular}
\end{center}
\begin{center}{Table \ref{tab:fpveff} - continued}\end{center}
\clearpage
\small
\begin{center}
\begin{tabular}{llllllrrrr}
\hline
$2j_{a}$&$2j_{b}$&$2j_{c}$&$2j_{d}$&$2T$&$2J$&
\multicolumn{1}{c}{A}&
\multicolumn{1}{c}{B}&
\multicolumn{1}{c}{C}&
\multicolumn{1}{c}{Emp}\\\hline
  7& 7& 3& 3& 0& 6&  -.3727&  -.3490&  -.3276&  -.5049\\
  7& 5& 7& 5& 0& 6& -1.3805& -1.2872& -1.1969& -1.0140\\
  7& 5& 7& 3& 0& 6&   .3140&   .3109&   .3033&   .3280\\
  7& 5& 7& 1& 0& 6&  -.6050&  -.5806&  -.5587&  -.4400\\
  7& 5& 5& 5& 0& 6&   .6526&   .6508&   .6431&   .6750\\
  7& 5& 5& 3& 0& 6&   .7027&   .7018&   .6936&   .6280\\
  7& 5& 5& 1& 0& 6&   .5797&   .5793&   .5767&   .5380\\
  7& 5& 3& 3& 0& 6&   .5515&   .5327&   .5128&   .5380\\
  7& 3& 7& 3& 0& 6&  -.6121&  -.5555&  -.5070&  -.7433\\
  7& 3& 7& 1& 0& 6&  1.4540&  1.3932&  1.3339&  1.2950\\
  7& 3& 5& 5& 0& 6&   .1533&   .1462&   .1415&   .1580\\
  7& 3& 5& 3& 0& 6&  -.7063&  -.6706&  -.6356&  -.4080\\
  7& 3& 5& 1& 0& 6&   .7707&   .7277&   .6871&   .5180\\
  7& 3& 3& 3& 0& 6&  -.6286&  -.6185&  -.6076&  -.6812\\
  7& 1& 7& 1& 0& 6& -1.7374& -1.6619& -1.5826& -1.4840\\
  7& 1& 5& 5& 0& 6&  -.1966&  -.1955&  -.1931&  -.2700\\
  7& 1& 5& 3& 0& 6&   .1269&   .1203&   .1188&   .1400\\
  7& 1& 5& 1& 0& 6&   .4884&   .4802&   .4721&   .4000\\
  7& 1& 3& 3& 0& 6&   .6161&   .5940&   .5733&   .4610\\
  5& 5& 5& 5& 0& 6&  -.5724&  -.5141&  -.4557&  -.1960\\
  5& 5& 5& 3& 0& 6&  -.2678&  -.2759&  -.2827&  -.3270\\
  5& 5& 5& 1& 0& 6&  -.5582&  -.5422&  -.5325&  -.3910\\
  5& 5& 3& 3& 0& 6&  -.0842&  -.0946&  -.1011&  -.1820\\
  5& 3& 5& 3& 0& 6&  -.5947&  -.5617&  -.5283&  -.4640\\
  5& 3& 5& 1& 0& 6& -1.1232& -1.0922& -1.0644&  -.9880\\
\hline
\end{tabular}
\end{center}
\begin{center}{Table \ref{tab:fpveff} - continued}\end{center}
\clearpage
\small
\begin{center}
\begin{tabular}{llllllrrrr}
\hline
$2j_{a}$&$2j_{b}$&$2j_{c}$&$2j_{d}$&$2T$&$2J$&
\multicolumn{1}{c}{A}&
\multicolumn{1}{c}{B}&
\multicolumn{1}{c}{C}&
\multicolumn{1}{c}{Emp}\\
\hline
  5& 3& 3& 3& 0& 6&  -.3664&  -.3860&  -.3955&  -.3400\\
  5& 1& 5& 1& 0& 6& -1.5476& -1.4911& -1.4342& -1.2340\\
  5& 1& 3& 3& 0& 6&  -.0443&  -.0539&  -.0576&  -.0840\\
  3& 3& 3& 3& 0& 6& -2.2762& -2.1266& -1.9908& -1.8885\\
  7& 5& 7& 5& 0& 8& -1.9959& -1.9249& -1.8418& -1.9080\\
  7& 5& 7& 3& 0& 8&  -.0697&  -.0514&  -.0400&  -.0390\\
  7& 5& 7& 1& 0& 8&  -.6415&  -.6270&  -.6184&  -.6740\\
  7& 5& 5& 3& 0& 8&  -.6705&  -.6396&  -.6162&  -.5440\\
  7& 3& 7& 3& 0& 8&  -.2790&  -.2378&  -.1975&  -.1742\\
  7& 3& 7& 1& 0& 8&   .1239&   .1267&   .1177&   .1080\\
  7& 3& 5& 3& 0& 8&  -.8228&  -.8018&  -.7770&  -.6140\\
  7& 1& 7& 1& 0& 8&  -.9010&  -.8288&  -.7664&  -.7460\\
  7& 1& 5& 3& 0& 8& -1.5802& -1.5329& -1.4818& -1.1880\\
  5& 3& 5& 3& 0& 8&  -.9084&  -.8557&  -.8091&  -.7336\\
  7& 7& 7& 7& 0&10&  -.5074&  -.4574&  -.4065&  -.4830\\
  7& 7& 7& 5& 0&10&  1.0521&  1.0092&   .9614&   .9010\\
  7& 7& 7& 3& 0&10&  -.8523&  -.8286&  -.8043& -1.0085\\
  7& 7& 5& 5& 0&10&   .3078&   .2754&   .2483&   .1700\\
  7& 5& 7& 5& 0&10&  -.0647&  -.0448&  -.0211&  -.1265\\
  7& 5& 7& 3& 0&10&   .4811&   .4732&   .4624&   .4050\\
  7& 5& 5& 5& 0&10&  1.2317&  1.1961&  1.1617&  1.1380\\
  7& 3& 7& 3& 0&10& -2.4203& -2.3146& -2.2033& -2.3544\\
  7& 3& 5& 5& 0&10&   .0869&   .0821&   .0784&   .0480\\
  5& 5& 5& 5& 0&10& -2.0419& -1.9610& -1.8756& -1.6899\\
  7& 5& 7& 5& 0&12& -2.6701& -2.5567& -2.4425& -2.2650\\
\hline
\end{tabular}
\end{center}
\begin{center}{Table \ref{tab:fpveff} - continued}\end{center}
\clearpage
\small
\begin{center}
\begin{tabular}{llllllrrrr}
\hline
$2j_{a}$&$2j_{b}$&$2j_{c}$&$2j_{d}$&$2T$&$2J$&
\multicolumn{1}{c}{A}&
\multicolumn{1}{c}{B}&
\multicolumn{1}{c}{C}&
\multicolumn{1}{c}{Emp}
\\\hline
  7& 7& 7& 7& 0&14& -2.4664& -2.3571& -2.2416& -2.6034\\
  7& 7& 7& 7& 2& 0& -2.1644& -2.0713& -2.0452& -2.1900\\
  7& 7& 5& 5& 2& 0& -1.9841& -1.9764& -1.9875& -2.7800\\
  7& 7& 3& 3& 2& 0&  -.9018&  -.8712&  -.8588& -1.1133\\
  7& 7& 1& 1& 2& 0&  -.5294&  -.5245&  -.5292&  -.7140\\
  5& 5& 5& 5& 2& 0& -1.4539& -1.3709& -1.3481&  -.8837\\
  5& 5& 3& 3& 2& 0&  -.5436&  -.5382&  -.5457&  -.7770\\
  5& 5& 1& 1& 2& 0&  -.4792&  -.4568&  -.4499&  -.3920\\
  3& 3& 3& 3& 2& 0& -1.3933& -1.3007& -1.2769& -1.2613\\
  3& 3& 1& 1& 2& 0& -1.4725& -1.4278& -1.3929& -1.4650\\
  1& 1& 1& 1& 2& 0&  -.3584&  -.3086&  -.3083&  -.2490\\
  7& 5& 7& 5& 2& 2&  -.6502&  -.6031&  -.5827&  -.2890\\
  7& 5& 5& 3& 2& 2&   .0264&   .0368&   .0407&   .0350\\
  7& 5& 3& 1& 2& 2&  -.2387&  -.2209&  -.2060&  -.3420\\
  5& 3& 5& 3& 2& 2&  -.0898&  -.0838&  -.0947&  -.0279\\
  5& 3& 3& 1& 2& 2&  -.0143&  -.0037&   .0008&   .0580\\
  3& 1& 3& 1& 2& 2&   .1363&   .1458&   .1269&   .1520\\
  7& 7& 7& 7& 2& 4&  -.8700&  -.8481&  -.8419&  -.8690\\
  7& 7& 7& 5& 2& 4&   .2250&   .2016&   .1760&   .0000\\
  7& 7& 7& 3& 2& 4&  -.5229&  -.5112&  -.4999&  -.8826\\
  7& 7& 5& 5& 2& 4&  -.6214&  -.6181&  -.6151&  -.6380\\
  7& 7& 5& 3& 2& 4&   .3338&   .3224&   .3109&   .2850\\
  7& 7& 5& 1& 2& 4&  -.3824&  -.3791&  -.3756&  -.4700\\
  7& 7& 3& 3& 2& 4&  -.3616&  -.3476&  -.3351&  -.4233\\
  7& 7& 3& 1& 2& 4&  -.1950&  -.1940&  -.1950&  -.2500\\
\hline
\end{tabular}
\end{center}
\begin{center}{Table \ref{tab:fpveff} - continued}\end{center}
\clearpage
\small
\begin{center}
\begin{tabular}{llllllrrrr}
\hline
$2j_{a}$&$2j_{b}$&$2j_{c}$&$2j_{d}$&$2T$&$2J$&
\multicolumn{1}{c}{A}&
\multicolumn{1}{c}{B}&
\multicolumn{1}{c}{C}&
\multicolumn{1}{c}{Emp}
\\\hline
  7& 5& 7& 5& 2& 4&  -.4688&  -.4423&  -.4301&  -.1061\\
  7& 5& 7& 3& 2& 4&  -.0548&  -.0447&  -.0438&  -.0140\\
  7& 5& 5& 5& 2& 4&  -.6304&  -.6057&  -.5869&  -.6310\\
  7& 5& 5& 3& 2& 4&   .3208&   .3172&   .3094&   .2950\\
  7& 5& 5& 1& 2& 4&  -.1404&  -.1473&  -.1517&  -.3420\\
  7& 5& 3& 3& 2& 4&   .0910&   .0840&   .0751&   .0180\\
  7& 5& 3& 1& 2& 4&  -.2218&  -.2113&  -.2044&  -.0790\\
  7& 3& 7& 3& 2& 4& -1.0055&  -.9716&  -.9546&  -.9036\\
  7& 3& 5& 5& 2& 4&  -.3035&  -.2963&  -.2959&  -.4460\\
  7& 3& 5& 3& 2& 4&   .2962&   .3001&   .3011&   .4020\\
  7& 3& 5& 1& 2& 4& -1.0051&  -.9810&  -.9572&  -.9660\\
  7& 3& 3& 3& 2& 4&  -.3675&  -.3610&  -.3518&  -.4503\\
  7& 3& 3& 1& 2& 4&  -.4814&  -.4649&  -.4506&  -.3190\\
  5& 5& 5& 5& 2& 4&  -.2880&  -.2943&  -.3060&  -.2180\\
  5& 5& 5& 3& 2& 4&  -.0404&  -.0369&  -.0326&   .0230\\
  5& 5& 5& 1& 2& 4&  -.3153&  -.3011&  -.2921&  -.1860\\
  5& 5& 3& 3& 2& 4&  -.0758&  -.0787&  -.0821&  -.1280\\
  5& 5& 3& 1& 2& 4&  -.2607&  -.2527&  -.2479&  -.1590\\
  5& 3& 5& 3& 2& 4&   .2122&   .1970&   .1754&   .2156\\
  5& 3& 5& 1& 2& 4&   .5653&   .5287&   .5045&   .3890\\
  5& 3& 3& 3& 2& 4&   .0774&   .0732&   .0715&  -.0680\\
  5& 3& 3& 1& 2& 4&   .2180&   .2165&   .2122&   .1590\\
  5& 1& 5& 1& 2& 4&  -.1450&  -.1568&  -.1738&  -.1061\\
  5& 1& 3& 3& 2& 4&  -.1504&  -.1409&  -.1362&  -.0470\\
  5& 1& 3& 1& 2& 4&  -.2866&  -.2817&  -.2758&  -.2410\\
  3& 3& 3& 3& 2& 4&  -.4912&  -.4587&  -.4498&  -.3755\\
\hline
\end{tabular}
\end{center}
\begin{center}{Table \ref{tab:fpveff} - continued}\end{center}
\clearpage
\small
\begin{center}
\begin{tabular}{llllllrrrr}
\hline
$2j_{a}$&$2j_{b}$&$2j_{c}$&$2j_{d}$&$2T$&$2J$&
\multicolumn{1}{c}{A}&
\multicolumn{1}{c}{B}&
\multicolumn{1}{c}{C}&
\multicolumn{1}{c}{Emp}
\\\hline
  3& 3& 3& 1& 2& 4&  -.7773&  -.7506&  -.7270&  -.6010\\
  3& 1& 3& 1& 2& 4&  -.9044&  -.8531&  -.8291&  -.6880\\
  7& 5& 7& 5& 2& 6&   .2248&   .1871&   .1527&   .0191\\
  7& 5& 7& 3& 2& 6&  -.1372&  -.1349&  -.1298&  -.1180\\
  7& 5& 7& 1& 2& 6&  -.0318&  -.0259&  -.0259&  -.0120\\
  7& 5& 5& 3& 2& 6&  -.1862&  -.1724&  -.1612&  -.1320\\
  7& 5& 5& 1& 2& 6&   .0889&   .0890&   .0881&   .0800\\
  7& 3& 7& 3& 2& 6&  -.0411&  -.0506&  -.0612&   .2396\\
  7& 3& 7& 1& 2& 6&   .1984&   .1770&   .1638&   .0780\\
  7& 3& 5& 3& 2& 6&   .1004&   .0949&   .0899&   .0260\\
  7& 3& 5& 1& 2& 6&  -.0674&  -.0738&  -.0755&  -.0590\\
  7& 1& 7& 1& 2& 6&   .0094&   .0020&  -.0099&   .0290\\
  7& 1& 5& 3& 2& 6&   .2649&   .2499&   .2358&   .1400\\
  7& 1& 5& 1& 2& 6&  -.1171&  -.1090&  -.1026&  -.0780\\
  5& 3& 5& 3& 2& 6&   .2897&   .2502&   .2176&   .1290\\
  5& 3& 5& 1& 2& 6&  -.0107&  -.0033&  -.0020&   .0070\\
  5& 1& 5& 1& 2& 6&   .2517&   .2294&   .2066&   .1290\\
  7& 7& 7& 7& 2& 8&  -.0540&  -.0745&  -.0946&  -.0868\\
  7& 7& 7& 5& 2& 8&  -.3530&  -.3462&  -.3399&  -.3860\\
  7& 7& 7& 3& 2& 8&  -.4584&  -.4397&  -.4203&  -.4527\\
  7& 7& 7& 1& 2& 8&  -.2219&  -.2209&  -.2189&  -.2930\\
  7& 7& 5& 5& 2& 8&  -.4473&  -.4395&  -.4285&  -.4000\\
  7& 7& 5& 3& 2& 8&   .3754&   .3629&   .3502&   .3050\\
  7& 5& 7& 5& 2& 8&   .0459&   .0268&   .0068&   .0085\\
  7& 5& 7& 3& 2& 8&  -.0365&  -.0359&  -.0386&  -.1140\\
\hline
\end{tabular}
\end{center}
\begin{center}{Table \ref{tab:fpveff} - continued}\end{center}
\clearpage
\small
\begin{center}
\begin{tabular}{llllllrrrr}
\hline
$2j_{a}$&$2j_{b}$&$2j_{c}$&$2j_{d}$&$2T$&$2J$&
\multicolumn{1}{c}{A}&
\multicolumn{1}{c}{B}&
\multicolumn{1}{c}{C}&
\multicolumn{1}{c}{Emp}
\\\hline
  7& 5& 7& 1& 2& 8&  -.1887&  -.1740&  -.1669&  -.0930\\
  7& 5& 5& 5& 2& 8&  -.4483&  -.4321&  -.4186&  -.4680\\
  7& 5& 5& 3& 2& 8&   .5002&   .4886&   .4753&   .4930\\
  7& 3& 7& 3& 2& 8&  -.0421&  -.0596&  -.0727&   .1891\\
  7& 3& 7& 1& 2& 8&  -.6690&  -.6352&  -.6073&  -.5020\\
  7& 3& 5& 5& 2& 8&  -.1546&  -.1484&  -.1446&  -.1560\\
  7& 3& 5& 3& 2& 8&   .7554&   .7268&   .6989&   .6190\\
  7& 1& 7& 1& 2& 8&  -.3554&  -.3529&  -.3523&  -.2740\\
  7& 1& 5& 5& 2& 8&  -.2396&  -.2305&  -.2238&  -.2520\\
  7& 1& 5& 3& 2& 8&   .7068&   .6794&   .6519&   .5310\\
  5& 5& 5& 5& 2& 8&   .2506&   .2158&   .1920&   .2949\\
  5& 5& 5& 3& 2& 8&   .1398&   .1356&   .1328&   .1020\\
  5& 3& 5& 3& 2& 8&  -.3310&  -.3263&  -.3260&  -.2570\\
  7& 5& 7& 5& 2&10&   .4665&   .4069&   .3584&   .1523\\
  7& 5& 7& 3& 2&10&  -.0614&  -.0592&  -.0530&  -.0100\\
  7& 3& 7& 3& 2&10&   .1900&   .1748&   .1533&   .5781\\
  7& 7& 7& 7& 2&12&   .2374&   .2044&   .1773&   .1519\\
  7& 7& 7& 5& 2&12&  -.9213&  -.8796&  -.8398&  -.7160\\
  7& 5& 7& 5& 2&12& -1.2375& -1.1886& -1.1484&  -.9971\\
\hline
\end{tabular}
\end{center}
\begin{center}{Table \ref{tab:fpveff} - continued}\end{center}
\clearpage

\begin{table}[hbtp]
\caption{The $T=1$ two-body effective interactions for light 
tin isotopes with $^{100}$Sn as core.
The labels A, B and C refer to the results with the Bonn A, B and
C potentials.
All entries in MeV.}
\small
\begin{center}
\begin{tabular}{llllllrrr}
\hline
$2j_{a}$&$2j_{b}$&$2j_{c}$&$2j_{d}$&$2T$&$2J$&
\multicolumn{1}{c}{A}&
\multicolumn{1}{c}{B}&
\multicolumn{1}{c}{C}
\\\hline
  7& 7& 7& 7& 2& 0& -1.5157& -1.3143& -1.3745\\
  7& 7& 5& 5& 2& 0&  -.5107&  -.5101&  -.5067\\
  7& 7& 3& 3& 2& 0&  -.5502&  -.5024&  -.5184\\
  7& 7& 1& 1& 2& 0&  -.2501&  -.2411&  -.2429\\
  7& 7&11&11& 2& 0&  1.5688&  1.5586&  1.5382\\
  5& 5& 5& 5& 2& 0&  -.8643&  -.8032&  -.8167\\
  5& 5& 3& 3& 2& 0&  -.8803&  -.8802&  -.8585\\
  5& 5& 1& 1& 2& 0&  -.4022&  -.3788&  -.3838\\
  5& 5&11&11& 2& 0&   .4465&   .4308&   .4397\\
  3& 3& 3& 3& 2& 0&  -.4309&  -.3842&  -.4016\\
  3& 3& 1& 1& 2& 0&  -.3027&  -.2892&  -.2924\\
  3& 3&11&11& 2& 0&   .4964&   .5023&   .4955\\
  1& 1& 1& 1& 2& 0&  -.7807&  -.7451&  -.7325\\
  1& 1&11&11& 2& 0&   .2210&   .2197&   .2204\\
 11&11&11&11& 2& 0& -1.0621&  -.9339&  -.9662\\
  7& 5& 7& 5& 2& 2&  -.2045&  -.1889&  -.1887\\
  7& 5& 5& 3& 2& 2&  -.0350&  -.0329&  -.0253\\
  7& 5& 3& 1& 2& 2&  -.0849&  -.0747&  -.0787\\
  5& 3& 5& 3& 2& 2&  -.0708&  -.0588&  -.0572\\
  5& 3& 3& 1& 2& 2&  -.0287&  -.0259&  -.0206\\
  3& 1& 3& 1& 2& 2&   .0038&   .0061&   .0111\\
  7&11& 7&11& 2& 4&  -.9097&  -.8081&  -.8181\\
  7& 7& 7& 7& 2& 4&  -.4814&  -.4462&  -.4551\\
  7& 7& 7& 5& 2& 4&  -.0763&  -.0648&  -.0669\\
\hline
\end{tabular}
\end{center}
\label{tab:tins}
\end{table}
\clearpage
\small
\begin{center}
\begin{tabular}{llllllrrr}
\hline
$2j_{a}$&$2j_{b}$&$2j_{c}$&$2j_{d}$&$2T$&$2J$&
\multicolumn{1}{c}{A}&
\multicolumn{1}{c}{B}&
\multicolumn{1}{c}{C}
\\\hline
  7& 7& 7& 3& 2& 4&  -.2844&  -.2560&  -.2640\\
  7& 7& 5& 5& 2& 4&  -.0628&  -.0678&  -.0677\\
  7& 7& 5& 3& 2& 4&  -.1541&  -.1483&  -.1462\\
  7& 7& 5& 1& 2& 4&  -.1525&  -.1523&  -.1482\\
  7& 7& 3& 3& 2& 4&  -.1930&  -.1745&  -.1786\\
  7& 7& 3& 1& 2& 4&   .0908&   .0850&   .0869\\
  7& 7&11&11& 2& 4&   .3476&   .3388&   .3425\\
  7& 5& 7& 5& 2& 4&  -.0114&  -.0031&  -.0128\\
  7& 5& 7& 3& 2& 4&   .2714&   .2451&   .2476\\
  7& 5& 5& 5& 2& 4&   .0681&   .0623&   .0637\\
  7& 5& 5& 3& 2& 4&   .0670&   .0695&   .0704\\
  7& 5& 5& 1& 2& 4&   .0629&   .0675&   .0634\\
  7& 5& 3& 3& 2& 4&   .0167&   .0183&   .0161\\
  7& 5& 3& 1& 2& 4&  -.0725&  -.0626&  -.0681\\
  7& 5&11&11& 2& 4&  -.2141&  -.2073&  -.2034\\
  7& 3& 7& 3& 2& 4&  -.3595&  -.3414&  -.3463\\
  7& 3& 5& 5& 2& 4&  -.1562&  -.1470&  -.1468\\
  7& 3& 5& 3& 2& 4&  -.1249&  -.1257&  -.1221\\
  7& 3& 5& 1& 2& 4&  -.1817&  -.1773&  -.1774\\
  7& 3& 3& 3& 2& 4&  -.1362&  -.1337&  -.1333\\
  7& 3& 3& 1& 2& 4&   .1908&   .1823&   .1831\\
  7& 3&11&11& 2& 4&   .3925&   .3874&   .3826\\
  5& 5& 5& 5& 2& 4&  -.3147&  -.3001&  -.2968\\
  5& 5& 5& 3& 2& 4&  -.1201&  -.1209&  -.1200\\
  5& 5& 5& 1& 2& 4&  -.2726&  -.2628&  -.2611\\
\hline
\end{tabular}
\end{center}
\begin{center}{Table \ref{tab:tins} - continued}\end{center}
\clearpage
\small
\begin{center}
\begin{tabular}{llllllrrr}
\hline
$2j_{a}$&$2j_{b}$&$2j_{c}$&$2j_{d}$&$2T$&$2J$&
\multicolumn{1}{c}{A}&
\multicolumn{1}{c}{B}&
\multicolumn{1}{c}{C}
\\\hline
  5& 5& 3& 3& 2& 4&  -.2255&  -.2199&  -.2202\\
  5& 5& 3& 1& 2& 4&   .2314&   .2209&   .2199\\
  5& 5&11&11& 2& 4&   .2164&   .2090&   .2077\\
  5& 3& 5& 3& 2& 4&  -.1240&  -.1113&  -.1128\\
  5& 3& 5& 1& 2& 4&  -.1318&  -.1205&  -.1219\\
  5& 3& 3& 3& 2& 4&  -.2367&  -.2300&  -.2245\\
  5& 3& 3& 1& 2& 4&   .1911&   .1924&   .1868\\
  5& 3&11&11& 2& 4&   .0337&   .0347&   .0362\\
  5& 1& 5& 1& 2& 4&  -.4939&  -.4651&  -.4647\\
  5& 1& 3& 3& 2& 4&  -.2093&  -.2047&  -.1980\\
  5& 1& 3& 1& 2& 4&   .4537&   .4454&   .4366\\
  5& 1&11&11& 2& 4&   .1270&   .1245&   .1258\\
  3& 3& 3& 3& 2& 4&  -.0623&  -.0596&  -.0613\\
  3& 3& 3& 1& 2& 4&   .1031&   .0960&   .0985\\
  3& 3&11&11& 2& 4&   .0912&   .0919&   .0918\\
  3& 1& 3& 1& 2& 4&  -.2515&  -.2308&  -.2313\\
  3& 1&11&11& 2& 4&  -.1577&  -.1564&  -.1547\\
 11&11&11&11& 2& 4&  -.8137&  -.7522&  -.7551\\
  7&11& 7&11& 2& 6&  -.4237&  -.3774&  -.3864\\
  7&11& 5&11& 2& 6&   .1761&   .1570&   .1605\\
  5&11& 5&11& 2& 6&  -.5454&  -.5284&  -.5306\\
  7& 5& 7& 5& 2& 6&   .1055&   .0740&   .0813\\
  7& 5& 7& 3& 2& 6&   .1423&   .1256&   .1291\\
  7& 5& 7& 1& 2& 6&  -.1235&  -.1100&  -.1105\\
  7& 5& 5& 3& 2& 6&  -.0406&  -.0363&  -.0361\\
\hline
\end{tabular}
\end{center}
\begin{center}{Table \ref{tab:tins} - continued}\end{center}
\clearpage
\small
\begin{center}
\begin{tabular}{llllllrrr}
\hline
$2j_{a}$&$2j_{b}$&$2j_{c}$&$2j_{d}$&$2T$&$2J$&
\multicolumn{1}{c}{A}&
\multicolumn{1}{c}{B}&
\multicolumn{1}{c}{C}
\\\hline
  7& 5& 5& 1& 2& 6&  -.0186&  -.0183&  -.0191\\
  7& 3& 7& 3& 2& 6&   .0937&   .0670&   .0717\\
  7& 3& 7& 1& 2& 6&  -.1289&  -.1096&  -.1137\\
  7& 3& 5& 3& 2& 6&   .0109&   .0066&   .0098\\
  7& 3& 5& 1& 2& 6&   .0138&   .0075&   .0098\\
  7& 1& 7& 1& 2& 6&   .1639&   .1366&   .1411\\
  7& 1& 5& 3& 2& 6&  -.0169&  -.0158&  -.0141\\
  7& 1& 5& 1& 2& 6&   .0244&   .0242&   .0205\\
  5& 3& 5& 3& 2& 6&   .0507&   .0375&   .0459\\
  5& 3& 5& 1& 2& 6&  -.0257&  -.0270&  -.0247\\
  5& 1& 5& 1& 2& 6&  -.0307&  -.0282&  -.0218\\
  7&11& 7&11& 2& 8&  -.1305&  -.1305&  -.1326\\
  7&11& 5&11& 2& 8&   .0604&   .0584&   .0572\\
  7&11& 3&11& 2& 8&  -.1903&  -.1656&  -.1699\\
  5&11& 5&11& 2& 8&  -.0600&  -.0720&  -.0711\\
  5&11& 3&11& 2& 8&  -.1409&  -.1244&  -.1264\\
  3&11& 3&11& 2& 8&  -.1439&  -.1390&  -.1430\\
  7& 7& 7& 7& 2& 8&  -.0053&  -.0315&  -.0327\\
  7& 7& 7& 5& 2& 8&   .0443&   .0434&   .0432\\
  7& 7& 7& 3& 2& 8&  -.2020&  -.1771&  -.1845\\
  7& 7& 7& 1& 2& 8&   .0650&   .0578&   .0601\\
  7& 7& 5& 5& 2& 8&  -.0473&  -.0475&  -.0479\\
  7& 7& 5& 3& 2& 8&  -.1529&  -.1463&  -.1445\\
  7& 7&11&11& 2& 8&   .2351&   .2260&   .2279\\
  7& 5& 7& 5& 2& 8&   .1179&   .0928&   .0937\\
\hline
\end{tabular}
\end{center}
\begin{center}{Table \ref{tab:tins} - continued}\end{center}
\clearpage
\small
\begin{center}
\begin{tabular}{llllllrrr}
\hline
$2j_{a}$&$2j_{b}$&$2j_{c}$&$2j_{d}$&$2T$&$2J$&
\multicolumn{1}{c}{A}&
\multicolumn{1}{c}{B}&
\multicolumn{1}{c}{C}
\\\hline
  7& 5& 7& 3& 2& 8&   .1932&   .1818&   .1800\\
  7& 5& 7& 1& 2& 8&  -.2456&  -.2218&  -.2231\\
  7& 5& 5& 5& 2& 8&   .0565&   .0515&   .0519\\
  7& 5& 5& 3& 2& 8&   .1381&   .1319&   .1337\\
  7& 5&11&11& 2& 8&  -.1480&  -.1441&  -.1427\\
  7& 3& 7& 3& 2& 8&   .0758&   .0520&   .0526\\
  7& 3& 7& 1& 2& 8&   .1564&   .1444&   .1457\\
  7& 3& 5& 5& 2& 8&  -.0494&  -.0460&  -.0458\\
  7& 3& 5& 3& 2& 8&  -.1335&  -.1261&  -.1265\\
  7& 3&11&11& 2& 8&   .1492&   .1468&   .1462\\
  7& 1& 7& 1& 2& 8&  -.1091&  -.1147&  -.1138\\
  7& 1& 5& 5& 2& 8&   .1037&   .0973&   .0964\\
  7& 1& 5& 3& 2& 8&   .1740&   .1653&   .1644\\
  7& 1&11&11& 2& 8&  -.1516&  -.1474&  -.1466\\
  5& 5& 5& 5& 2& 8&  -.0813&  -.0796&  -.0773\\
  5& 5& 5& 3& 2& 8&  -.3620&  -.3477&  -.3435\\
  5& 5&11&11& 2& 8&   .1027&   .0987&   .0984\\
  5& 3& 5& 3& 2& 8&  -.5678&  -.5470&  -.5350\\
  5& 3&11&11& 2& 8&   .1447&   .1427&   .1419\\
 11&11&11&11& 2& 8&  -.3057&  -.2881&  -.2899\\
  7&11& 7&11& 2&10&  -.0921&  -.0994&  -.1007\\
  7&11& 5&11& 2&10&   .0471&   .0499&   .0480\\
  7&11& 3&11& 2&10&  -.1986&  -.1745&  -.1795\\
  7&11& 1&11& 2&10&   .0913&   .0823&   .0840\\
  5&11& 5&11& 2&10&  -.0276&  -.0498&  -.0475\\
\hline
\end{tabular}
\end{center}
\begin{center}{Table \ref{tab:tins} - continued}\end{center}
\clearpage
\small
\begin{center}
\begin{tabular}{llllllrrr}
\hline
$2j_{a}$&$2j_{b}$&$2j_{c}$&$2j_{d}$&$2T$&$2J$&
\multicolumn{1}{c}{A}&
\multicolumn{1}{c}{B}&
\multicolumn{1}{c}{C}
\\\hline
  5&11& 3&11& 2&10&   .1481&   .1444&   .1431\\
  5&11& 1&11& 2&10&  -.2678&  -.2512&  -.2517\\
  3&11& 3&11& 2&10&   .0597&   .0354&   .0358\\
  3&11& 1&11& 2&10&   .2156&   .2020&   .2023\\
  1&11& 1&11& 2&10&  -.1313&  -.1463&  -.1444\\
  7& 5& 7& 5& 2&10&   .1193&   .0839&   .0912\\
  7& 5& 7& 3& 2&10&   .0281&   .0238&   .0277\\
  7& 3& 7& 3& 2&10&   .1535&   .1277&   .1327\\
  7&11& 7&11& 2&12&   .0258&  -.0006&   .0005\\
  7&11& 5&11& 2&12&   .0122&   .0136&   .0123\\
  7&11& 3&11& 2&12&  -.1375&  -.1217&  -.1245\\
  7&11& 1&11& 2&12&  -.0338&  -.0286&  -.0301\\
  5&11& 5&11& 2&12&   .1528&   .1207&   .1226\\
  5&11& 3&11& 2&12&   .0257&   .0240&   .0239\\
  5&11& 1&11& 2&12&  -.1828&  -.1624&  -.1662\\
  3&11& 3&11& 2&12&   .1290&   .0996&   .1004\\
  3&11& 1&11& 2&12&  -.1196&  -.1057&  -.1084\\
  1&11& 1&11& 2&12&   .1336&   .1081&   .1083\\
  7& 7& 7& 7& 2&12&   .1973&   .1517&   .1513\\
  7& 7& 7& 5& 2&12&   .1173&   .1026&   .1073\\
  7& 7&11&11& 2&12&   .1635&   .1561&   .1567\\
  7& 5& 7& 5& 2&12&  -.3418&  -.3297&  -.3240\\
  7& 5&11&11& 2&12&  -.2128&  -.2047&  -.2038\\
 11&11&11&11& 2&12&  -.0873&  -.0913&  -.0924\\
  7&11& 7&11& 2&14&  -.1154&  -.1382&  -.1331\\
\hline
\end{tabular}
\end{center}
\begin{center}{Table \ref{tab:tins} - continued}\end{center}
\clearpage
\small
\begin{center}
\begin{tabular}{llllllrrr}
\hline
$2j_{a}$&$2j_{b}$&$2j_{c}$&$2j_{d}$&$2T$&$2J$&
\multicolumn{1}{c}{A}&
\multicolumn{1}{c}{B}&
\multicolumn{1}{c}{C}
\\\hline
  7&11& 5&11& 2&14&   .0718&   .0753&   .0712\\
  7&11& 3&11& 2&14&  -.2879&  -.2662&  -.2663\\
  5&11& 5&11& 2&14&   .1101&   .0805&   .0827\\
  5&11& 3&11& 2&14&   .2972&   .2802&   .2784\\
  3&11& 3&11& 2&14&  -.2772&  -.2853&  -.2803\\
  7&11& 7&11& 2&16&   .0852&   .0508&   .0532\\
  7&11& 5&11& 2&16&  -.0570&  -.0508&  -.0526\\
  5&11& 5&11& 2&16&   .0712&   .0520&   .0509\\
 11&11&11&11& 2&16&   .0139&   .0004&  -.0004\\
  7&11& 7&11& 2&18&  -.2321&  -.1901&  -.1923\\
 11&11&11&11& 2&20&   .1343&   .1171&   .1161\\
\hline
\end{tabular}
\end{center}
\begin{center}{Table \ref{tab:tins} - continued}\end{center}






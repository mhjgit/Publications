%  this file, sec2.tex, contains section 2 of physrep.tex

\section{Brueckner theory and the $G$-matrix}
The Brueckner $G$-matrix is probably the most important ingredient
in many-body calculations of nuclear systems. In this section, we will
briefly survey the philosophy behind the $G$-matrix.

Historically, the $G$-matrix was developed in microscopic nuclear
matter calculations using realistic nucleon-nucleon (NN) interactions.
It is an ingenuous as well as an interesting method to overcome the
difficulties caused by the strong, short-range repulsive core contained
in all modern models for the NN interaction. The $G$-matrix method was
originally developed by Brueckner \cite{bruck57}, and further
developed by Goldstone \cite{gold58} and Bethe, Brandow and Petschek
\cite{bbp63}. In the literature it is generally referred to as the
Brueckner theory or the Brueckner-Bethe-Goldstone theory.

Suppose we want to calculate the nuclear matter ground-state
energy $E_0$ using the non-relativistic Schr\"{o}dinger equation
\begin{equation}
      H\Psi_0(A)=E_0(A)\Psi_0(A),
\end{equation}
with $H=T+V$ where $A$ denotes the number of particles, $T$
is the kinetic energy and $V$ is
the nucleon-nucleon
(NN)  potential. Models for the NN potential will be discussed
in section 3. The corresponding unperturbed
problem is
\begin{equation}
      H_0\psi_0(A)=W_0(A)\psi_0(A).
\end{equation}
Here $H_0$ is just kinetic energy $T$ and $\psi_0$ is a Slater
determinant representing the Fermi sea, where all orbits through the
Fermi momentum $k_F$ are filled. We write
\begin{equation}
      E_0=W_0+\Delta E_0,
\end{equation}
where $\Delta E_0$ is the ground-state energy shift.
If we know how to calculate $\Delta E_0$, then we know $E_0$, since
$W_0$ is easily obtained. In the limit $A\rightarrow \infty$,
the quantities $E_0$ and $\Delta E_0$ themselves are not well
defined, but the ratios $E_0/A$ and $\Delta E_0/A$ are. The
nuclear-matter binding energy per nucleon is commonly denoted
by $BE/A$, which is just $-E_0/A$. In passing, we note that
the empirical value for symmetric nuclear matter (proton number
$Z$=neutron number $N$) is $\approx 16$ MeV.
There exists a formal theory for the calculation of $\Delta E_0$.
According to the well-known Goldstone linked-diagram theory
\cite{gold58}, the energy shift $\Delta E_0$ is given exactly by the
diagrammatic expansion shown in fig.\ \ref{fig:goldstone}. This theory,
is a linked-cluster perturbation expansion for the ground state
energy of a many-body system, and applies equally well to both
nuclear matter and closed-shell nuclei such as the doubly magic
nucleus $^{40}$Ca. 
We will not discuss the Goldstone expansion, but rather discuss
briefly how it is used in calculations.
For systems with particles outside closed shells,
a similar theory exists, namely the so-called folded-diagram theory,
which we will review in section 4.
\begin{figure}
      \setlength{\unitlength}{1mm}
      \begin{picture}(140,100)
      \put(25,10){\epsfxsize=12cm \epsfbox{goldstone.eps}}
      \end{picture}
      \caption{Diagrams which enter the definition of the ground-state
      shift energy $\Delta E_0$. Diagram (i) is first order in the 
      interaction $V$, while diagrams (ii) and (iii) are examples of
       contributions to second and third order, respectively.}
      \label{fig:goldstone}
\end{figure}
Using the standard diagram rules listed  in appendix A, the various
diagrams contained in the above figure can be readily calculated, viz.\
(in an uncoupled scheme) we
have
\begin{equation}
   (i)=\frac{(-)^{n_h+n_l}}{2^{n_{ep}}}\sum_{k_i,k_j\leq k_F}
       \bra{k_ik_j}\tilde{V}\ket{k_ik_j}_{AS},
\end{equation}
with $n_h=n_l=2$ and $n_{ep}=1$. As discussed  in appendix A, $n_h$
denotes the number of hole lines, $n_l$ the number of closed
fermion loops and $n_{ep}$ is the number of so-called
equivalent pairs.
The factor $1/2^{n_{ep}}$ is needed since we want to count a pair 
of particles only once. We will carry this factor $1/2$ with us
in the equations below. 
The subscript
$AS$ denotes the antisymmetrized and normalized matrix element
\begin{equation}
     \bra{k_ik_j}\tilde{V}\ket{k_ik_j}_{AS}=\bra{k_ik_j}V\ket{k_ik_j}-
     \bra{k_jk_i}V\ket{k_ik_j}.
\end{equation}
Similarly, diagrams (ii) and (iii) read
\begin{equation}
   (ii)=\frac{(-)^{2+2}}{2^2}\sum_{k_i,k_j\leq k_F}\sum_{k_mk_n>k_F}
   \frac{\bra{k_ik_j}\tilde{V}\ket{k_mk_n}_{AS}
   \bra{k_mk_n}\tilde{V}\ket{k_ik_j}_{AS}}
   {\varepsilon_i+\varepsilon_j-\varepsilon_m-\varepsilon_n},
\end{equation}
and
\begin{equation}
   (iii)=\frac{(-)^{2+2}}{2^3}\sum_{k_i,k_j\leq k_F}\sum_{k_mk_n>k_F}
   \sum_{k_pk_q>k_F}
   \frac{\bra{k_ik_j}\tilde{V}\ket{k_mk_n}_{AS}
   \bra{k_mk_n}\tilde{V}\ket{k_pk_q}_{AS}
   \bra{k_pk_q}\tilde{V}\ket{k_ik_j}_{AS}}
   {(\varepsilon_i+\varepsilon_j-\varepsilon_m-\varepsilon_n)
   (\varepsilon_i+\varepsilon_j-\varepsilon_p-\varepsilon_q)}.
\end{equation}
In the above, $\varepsilon$ denotes the sp energies defined by
$H_0$.
The steps leading to the above expressions for the various
diagrams are rather straightforward. Though, if we wish to compute the
matrix elements for the potential $V$, a serious problem
arises. Typically, the matrix elements will contain a term
(see the next section for the formal details) $V(|{\bf r}|)$, which
represents the interaction potential $V$ between two nucleons, where
${\bf r}$ is the internucleon distance.
As we shall further detail in the next section, all modern models
for $V$ have a strong short-range repulsive core. Hence,
matrix elements involving $V(|{\bf r}|)$, will result in large
(or infinitely large for a potential with a hard core)
and repulsive contributions to the ground-state energy. Thus, the
diagrammatic expansion for the ground-state energy in terms of the
potential $V(|{\bf r}|)$ becomes meaningless.

The resolution of this problem is provided by the well-known
Brueckner theory or the Brueckner $G$-matrix, or just the
$G$-matrix. In fact, the $G$-matrix is an indispensable
tool in almost every microscopic nuclear structure
calculation. Its main idea may be paraphrased as follows.
Suppose we want to calculate the function $f(x)=x/(1+x)$. If
$x$ is small, we may expand the function $f(x)$ as a power series
$x+x^2+x^3+\dots$ and it may be adequate to just calculate the first
few terms. In other words, $f(x)$ may be calculated using a low-order
perturbation method. But if $x$ is large
(or infinitely large), the above
power series is obviously meaningless.
However, the exact function
$x/(1+x)$ is still well defined in the limit
of $x$ becoming very large.

These arguments suggest that one should sum up the diagrams
(i), (ii), (iii) in fig.\ \ref{fig:goldstone} and the similar ones
to all orders, instead of computing them one by one. Denoting this
all-order sum as $1/2\tilde{G}_{ijij}$, where we have
introduced the shorthand notation
$\tilde{G}_{ijij}=\bra{k_ik_j}\tilde{G}\ket{k_ik_j}_{AS}$
(and similarly for $\tilde{V}$),
we have that
\begin{eqnarray}
      \frac{1}{2}\tilde{G}_{ijij}=&{\displaystyle
      \frac{1}{2}\tilde{V}_{ijij}
      +\sum_{mn>k_F}\frac{1}{2}\tilde{V}_{ijmn}\frac{1}
      {\varepsilon_i+\varepsilon_j-\varepsilon_m-\varepsilon_n}}
      \nonumber \\
      &{\displaystyle \times\left[\frac{1}{2}\tilde{V}_{mnij}+\sum_{pq>k_F}
      \frac{1}{2}\tilde{V}_{mnpq}\frac{1}
      {\varepsilon_i+\varepsilon_j-\varepsilon_p-\varepsilon_q}
      \frac{1}{2}V_{pqij}+\dots  \right] }.
\end{eqnarray}
The factor $1/2$ is the same as that discussed above, namely we want 
to count a pair of particles only once.
The quantity inside the brackets is just
$1/2\tilde{G}_{mnij}$ and the above equation can be
rewritten as an integral equation
\begin{equation}
      \tilde{G}_{ijij}=\tilde{V}_{ijij}
      +\sum_{mn>k_F}\frac{1}{2}\tilde{V}_{ijmn}\frac{1}
      {\varepsilon_i+\varepsilon_j-\varepsilon_m-\varepsilon_n}
      \tilde{G}_{mnij}.
\end{equation}
Note that $\tilde{G}$ is the antisymmetrized $G$-matrix since
the potential $\tilde{V}$ is also antisymmetrized. This means that
$\tilde{G}$ obeys
\begin{equation}
  \tilde{G}_{ijij}=-\tilde{G}_{jiij}=-\tilde{G}_{ijji}.
\end{equation}
The $\tilde{G}$-matrix  is defined as
\begin{equation}
    \tilde{G}_{ijij}=G_{ijij}-G_{jiij},
\end{equation}
and the equation for $G$ is
\begin{equation}
      G_{ijij}=V_{ijij}
      +\sum_{mn>k_F}V_{ijmn}\frac{1}
      {\varepsilon_i+\varepsilon_j-\varepsilon_m-\varepsilon_n}
      G_{mnij},
      \label{eq:ggeneral}
\end{equation}
which is the familiar $G$-matrix equation. The above
matrix is specifically designed to treat a class of diagrams
contained in $\Delta E_0$, of which typical contributions
were shown in fig.\ \ref{fig:goldstone}. In fact the sum of the diagrams
in fig.\ \ref{fig:goldstone} is equal to $1/2(G_{ijij}-G_{jiij})$.

Let us now define a more general $G$-matrix as
\begin{equation}
      G_{ijij}=V_{ijij}
      +\sum_{mn>0}V_{ijmn}\frac{Q(mn)}
      {\omega -\varepsilon_m-\varepsilon_n}
      G_{mnij},
      \label{eq:gwithq}
\end{equation}
which is an extension of eq.\ (\ref{eq:ggeneral}). Note that 
eq.\ (\ref{eq:ggeneral}) has
$\varepsilon_i+\varepsilon_j$ in the energy denominator, whereas
in the latter equation we have a general energy variable $\omega$
in the denominator. Furthermore, in eq.\ (\ref{eq:ggeneral})
we have a restricted
sum over $mn$, while in eq.\ (\ref{eq:gwithq})
we sum over all $mn$ and we have
introduced a weighting factor $Q(mn)$. In eq.\ (\ref{eq:gwithq}) $Q(mn)$
corresponds to the choice
\begin{equation}
   Q(k_m , k_n ) =
    \left\{\begin{array}{cc}1,&min(k_m ,k_n ) > k_F\\
    0,&\mathrm{else}.\end{array}\right. ,
\end{equation}
where $Q(mn)$ is usually referred to as the $G$-matrix Pauli
exclusion operator. The role of $Q$ is to enforce a selection
of the intermediate states allowed in the $G$-matrix equation. The above
$Q$ requires that the momenta of the intermediate particles $m$ and $n$
must be both above the Fermi surface defined by $k_F$. We may enforce
a different requirement by using a summation over intermediate states
different from that in eq.\ (\ref{eq:gwithq}).
An example is the Pauli operator
for the model-space Brueckner-Hartree-Fock method to be discussed
in section 7.


Before ending this section, let us rewrite the $G$-matrix equation
in a more compact form.
The sp energies $\varepsilon$ and wave functions are defined
by the unperturbed hamiltonian $H_0$ as
\begin{equation}
   H_0\ket{\psi_m}=\varepsilon_m\ket{\psi_m},
\end{equation}
and similarly
\begin{equation}
   H_0\ket{\psi_m\psi_n}=(\varepsilon_m+\varepsilon_n)
   \ket{\psi_m\psi_n}.
\end{equation}
The $G$-matrix equation can then be rewritten in the following
compact form
\begin{equation}
   G(\omega )=V+V\frac{Q}{\omega -H_0}G(\omega ),
\end{equation}
with
${\displaystyle Q=\sum_{mn}\ket{\psi_m\psi_n}Q(mn)\bra{\psi_m\psi_n}}$.
In terms of diagrams, $G$ corresponds to an all-order sum of the
"ladder-type" interactions between two particles with the
intermediate states restricted by $Q$, as schematically
shown in fig.\ \ref{fig:gschem}.
\begin{figure}
    \setlength{\unitlength}{1mm}
    \begin{picture}(140,80)
      \put(25,10){\epsfxsize=12cm \epsfbox{fig2.eps}}
    \end{picture}
    \caption{Schematic representation of the ladder renormalization
    of the free NN potential. The wavy line may represent either the
    scattering matrix $T$ or the medium dependent reaction matrix $G$.
    The latter
    differs from the scattering matrix since it contains the Pauli
    exclusion operator $Q$ and medium dependent single-particle energies.
    The physical interpretation of either $T$ or $G$ is that the particles
    must interact virtually with each other an arbitrary
    number of times in order
    to produce a finite interaction matrix element.}
\label{fig:gschem}
\end{figure}

The $G$-matrix equation has a very simple form. But its
calculation is rather complicated, particularly for finite
nuclear systems such as the nucleus $^{18}$O. There are a
number of complexities. To mention a few, the Pauli operator
$Q$ may not commute with the unperturbed hamiltonian
$H_0$ and we have to make the replacement
\[
\frac{Q}{\omega -H_0}\rightarrow Q\frac{1}{\omega -QH_0Q}Q.
\]
The determination of the starting energy $\omega$ is also another
problem. These and other problems will be dealt with in
section 5.












\subsection{Introduction}
As we discussed in sections 2 and 3,
in nuclear structure and nuclear matter calculations one has to
face the problem that any realistic nucleon-nucleon (NN) potential $V$
exhibits a strong short-range repulsion, which in turn makes
a perturbative treatment of the nuclear many-body problem
prohibitive. If the potential has a so-called hard core,
the matrix elements of such a potential $\bra{\psi}V\ket{\psi}$
evaluated
for an uncorrelated two-body wave function $\psi (r)$ diverge,
since the uncorrelated wave function is different from zero also for
relative distances $r$ smaller than the hard-core radius. Similarly,
even if one uses potentials with softer cores, the matrix elements of the
potential become very large at short distances.
The above problem was however overcome by
introducing the reaction matrix $G$ (displayed by  the summation
of ladder type of diagrams in fig.\ \ref{fig:gschem})
which accounts for the
effects of two-nucleon correlations.

The matrix elements of the
potential $V$ then become
\begin{equation}
\bra{\psi}G\ket{\psi} =\bra{\psi}V\ket{\Psi}
\end{equation}
where $\Psi$ is now the correlated wave function. By accounting for the
correlations in the two-body wave functon $\Psi$, the matrix elements of
the potential become finite, even for a hard-core potential $V$. Moreover,
as will be discussed below, compared with the uncorrelated
wave function, the correlated wave function enhances the
matrix elements of $V$ at distances for which the interaction is
attractive.
The $G$-matrix applies to bound states, while  for scattering of free particles, 
the 
$G$-matrix is replaced by the $T$-matrix. The difference between the matrices
resides in the medium dependence provided by the Pauli operator and 
the energy $\omega$ of the interacting particles.
Recalling the discussion at the end of section 2, we defined the
$G$-matrix by either
\begin{equation}
   G(\omega )=V+V\frac{Q}{\omega -H_0}G(\omega ),
   \label{eq:g1}
\end{equation}
or
\begin{equation}
   G(\omega )=V+VQ\frac{1}{\omega -QH_0Q}QG(\omega ).
      \label{eq:g2}
\end{equation}
The former equation applies if the Pauli operator $Q$  commutes
with the unperturbed hamiltonian $H_0$, whereas the latter is
needed if $[H_0,Q]\neq 0$.
Similarly, the correlated wave function $\Psi$
is given as
\begin{equation}
    \ket{\Psi}=\ket{\psi}+\frac{Q}{\omega - H_0}G\ket{\psi},
    \label{eq:wave}
\end{equation}
or
\begin{equation}
   \ket{\Psi}=\ket{\psi}+Q\frac{1}{\omega - QH_0Q}QG\ket{\psi}.
\end{equation}

These equations look rather simple, but their actual calculation
is in fact rather complicated and difficult. Approximate
methods were devised in the "early days", amongst these methods were
the famous separation and reference spectrum methods. These
methods present a physically intuitive picture of the
$G$-matrix, and it is one of the scopes of this 
section to briefly review those methods  here. This is done
in subsection 5.2. Further, we will
also discuss how to calculate the $G$-matrix for both nuclear
matter and finite nuclei in subsecs.\ 5.3 and 5.4, respectively.
 
\subsection{The separation and reference-spectrum methods for
calculating the $G$-matrix}

In section 3 we discussed the fact that the NN potential
diverges at small internucleon distances. Fig.\ \ref{fig:bonna1s0}
displays
\begin{figure}
      \setlength{\unitlength}{1mm}
      \begin{picture}(120,100)
      \put(25,10){\epsfxsize=12cm \epsfbox{ones0.eps}}
      \end{picture}
\caption{Schematic plot of the nucleon-nucleon
potential for the $^1S_0$ partial wave. See text
for further details.}
\label{fig:bonna1s0}
\end{figure}
a schematic plot of the nucleon-nucleon potential
for the $^1S_0$
partial wave. The internucleon distance is $r$. As indicated
in the figure, we divide the potential into two parts, the
short-range part $V_S$ and the long range part $V_L$. Formally,
this may written as
\begin{equation}
  V(r)\approx V_S(r)\Theta (d-r)+V_L(r)\Theta (r-d),
  \label{eq:vsep}
\end{equation}
where $\Theta$ is the familiar step function, and $d$ is
known as the separation distance. The separation distance
is chosen so that the attractive part of
$V_S$ balances that of the repulsive core.
A remarkable result
is then to a good approximation
\begin{equation}
   G \propto V_L(r),
\end{equation}
based on the separation method.
This is quite an appealing result. The whole potential $V(r)$
is "singular" in the sense that it has a very strong repulsive
core. In contrast, $V_L(r)$ is obviously no longer singular and
hence the $G$-matrix is also no longer singular.
We have not yet shown how to obtain this result, nor have we
discussed what $d$ is and how to determine $d$\footnote{$d$ is
typically about 1 fm.}.

We will now derive the above result and also try to motivate
for the physics behind the separation and reference-spectrum 
methods.
First we define a useful identity following Bethe, Brandow and
Petschek \cite{bbp63}. Suppose we have two
different $G$-matrices, defined by
\begin{equation}
    G_1=V_1+V_1\frac{Q_1}{e_1}G_1,
\end{equation}
and
\begin{equation}
    G_2=V_2+V_2\frac{Q_2}{e_2}G_2,
\end{equation}
where $Q_1/e_1$ and $Q_2/e_2$ are the propagators of
either eq.\ (\ref{eq:g1}) or eq.\ (\ref{eq:g2}). $G_1$ and $G_2$
are two different $G$-matrices having two different interactions
and/or different propagators. We aim at an identity
which will enable us to calculate $G_1$ in terms of $G_2$,
or vice versa.
Defining the wave operators
\begin{equation}
    \Omega_1=1+\frac{Q_1}{e_1}G_1,
\end{equation}
and
\begin{equation}
    \Omega_2=1+\frac{Q_2}{e_2}G_2,
\end{equation}
we can rewrite the above $G$-matrices as
\begin{equation}
    G_1=V_1\Omega_1,
    \label{eq:omega1}
\end{equation}
and
\begin{equation}
    G_2=V_2\Omega_2.
    \label{eq:omega2}
\end{equation}
Using these relations, we rewrite $G_1$ as
\begin{eqnarray}
   G_1=&G_1 -{\displaystyle 
         G_2^{\dagger}\left(\Omega_1-1-\frac{Q_1}{e_1}G_1\right)
        +\left(\Omega_2^{\dagger}-1-G_2^{\dagger}\frac{Q_2}{e_2}\right)G_1} 
         \nonumber \\
       =&{\displaystyle G_2^{\dagger} +G_2^{\dagger}\left(\frac{Q_1}{e_1}-
        \frac{Q_2}{e_2}\right)G_1
        +\Omega_2^{\dagger}G_1 -G_2^{\dagger}\Omega_1},
\end{eqnarray}
and using eqs.\ (\ref{eq:omega1}) and (\ref{eq:omega2}) we obtain
the identity
\begin{equation}
        G_1=G_2^{\dagger} +G_2^{\dagger}
        \left(\frac{Q_1}{e_1}-\frac{Q_2}{e_2}\right)G_1
        +\Omega_2^{\dagger}(V_1-V_2)\Omega_1.
        \label{eq:gidentity}
\end{equation}
The second term on the rhs.\ is called the propagator-correction term;
it vanishes if $G_1$ and $G_2$ have the same propagators. The third term
is often referred to as the potential-correction term, and it disappears
if $G_1$ and $G_2$  have the same potentials.

The reader may now ask what is the advantage of the above identity. If
we assume that by some physical reasoning we are able to calculate
$G_2$ (we will see examples of this below both for nuclear matter and for
finite nuclei), and that the expression for $G_2$ can be calculated
easily, and further that $G_2$ is a good approximation
to the original $G$-matrix, then we can use the above identity to
perform a perturbative calculation of $G_1$ in terms of $G_2$.
Below we will use eq.\ (\ref{eq:gidentity}) to discuss
the separation and
the reference-spectrum methods.

\subsubsection{The separation method}

This method was first introduced by Moszkowski and Scott for
evaluating the $G$-matrix in nuclear matter \cite{ms61,ms62}.
In eq.\ (\ref{eq:vsep}) we separated the NN potential into two parts,
$V_S$ and $V_L$. Let us define the $G$-matrices
\begin{equation}
    G_1=G=V+V\frac{Q}{e}G,
\end{equation}
and
\begin{equation}
    G_2=G_S=V_S+V_S\frac{Q_2}{e_2}G_S,
    \label{eq:gs}
\end{equation}
where $V$ is the NN potential. Eq.\ (\ref{eq:gidentity}) gives
\begin{equation}
        G=G_S^{\dagger} +G_S^{\dagger}
        \left(\frac{Q}{e}-\frac{Q_2}{e_2}\right)G
        +\Omega_S^{\dagger}V_L\Omega.
        \label{eq:gidsep}
\end{equation}
The idea is that we can make $G_S\approx 0$, which in turn yields the
advertised relation
\[
   G\approx V_L.
\]
We emphasize that the above result is an approximation, but the hope
is that this approximation is a physically reasonable one.
To proceed, let us define the correlated wave function $\Psi_a$  
\begin{equation}
   G\ket{\psi_a}=V\ket{\Psi_a},
\end{equation}
where $\psi_a$ is the 
unperturbed wave function. 
Using the definition of the correlated wave function in eq.\
(\ref{eq:wave}) we have
\begin{equation}
    \ket{\Psi_a}=\ket{\psi_a}+\frac{Q}{\omega - H_0}G\ket{\psi_a}=
    \ket{\psi_a}+\frac{Q}{\omega - H_0}V\ket{\Psi_a}.
    \label{eq:wavesep}
\end{equation}
Note that we have assumed that the Pauli operator $Q$ and the unperturbed
hamiltonian $H_0$ commute. If we are able to obtain the correlated 
wave function, we get the $G$-matrix by
\begin{equation}
   \bra{\psi_a}G\ket{\psi_b}=\bra{\psi_a}V\ket{\Psi_b}.
\end{equation}
We will try to obtain $G_S$ through this scheme. $G_S$ was defined
according to eq.\ (\ref{eq:gs}) with the propagator $Q_2/e_2$.
We have not yet specified how to obtain $Q_2$ and $e_2$, though
the identity
in eq.\ (\ref{eq:gidsep}) is valid  for any choice of $Q_2$ and $e_2$. 
The potential which defines $G_S$ is $V_S$, which is a short-range 
repulsive potential. Using the uncertainty principle, the intermediate
states included by $V_S$ should be predominantly those at high momenta.
For such scattering states there is hardly any Pauli blocking, which 
means that the probability of finding other nucleons at such momenta
is practically equal to zero. This physically intuitive reasoning
leads us to choose $Q_2=1$. With this choice we obtain
\begin{equation}
    G_S=V_S+V_S\frac{1}{\omega -H_0}G_S,
\end{equation}
and the corresponding correlated wave function
\begin{equation}
    \ket{\Psi_a^S}=
    \ket{\psi_a}+\frac{1}{\omega - H_0}V_S\ket{\Psi_a^S}.
\end{equation}
As mentioned earlier we are free to make whatever choice for $e_2$
in the operator $\omega -H_0$ of $G_S$. Let us employ a specific
on-shell choice, i.e.\ $\omega =\varepsilon_a$, and $H_0\ket{\psi_a}
=\varepsilon_a\ket{\psi_a}$. In this case we have
$(\omega -H_0)\ket{\psi_a}=0$. Using the above definition
for the correlated wave function we obtain for the diagonal
matrix element
\begin{equation}
     \bra{\psi_a}G_S\ket{\psi_a}=\bra{\psi_a}V_S\ket{\Psi_a^S}
     =\bra{\psi_a}(\varepsilon_a -H_0)\ket{\Psi_a^S},
     \label{eq:eq22}
\end{equation}
with $H_0$ acting on $\Psi_a^S$.
We will also assume that $\psi_a$ is a plane-wave two-particle
state which can be written as
\begin{equation}
    \psi_a=\frac{1}{\Omega}e^{i{\bf kr}}e^{i{\bf KR}}
    = \frac{1}{\sqrt{\Omega}}\psi ({\bf r})e^{i{\bf KR}},
\end{equation}
where ${\bf k}$ and ${\bf K}$ denote respectively the relative and
center-of-mass
momenta. The term $\Omega$ in the denominators
serves to normalize the wave function and is the volume of the system.
We approximate $H_0$ with the two-particle kinetic energy, denoted
by the operators $T_r$ and $T_R$ for the relative and center-of-mass
systems, respectively. The energy $\varepsilon_a$ is then just
$k^2/M_N + K^2/4M_N$, with $M_N$ being the mass of the nucleon.
Since the potential $V_S$ is a function of $r$ only, it has no effect
on the center-of-mass wave function. Another possibility is to use
harmonic oscillator (h.o.) wave functions, as detailed by Kuo and
Brown \cite{kb66}.
The correlated wave function takes
the form
\begin{equation}
    \Psi_a^S
    = \frac{1}{\sqrt{\Omega}}\Psi_a^S (r)e^{i{\bf KR}}.
\end{equation}
Since $V_S(r)=0$ for $r>d$, the matrix of eq.\ (\ref{eq:eq22}) is
given by, using Greens theorem,
\begin{equation}
  \bra{\psi}G_S\ket{\psi}=\frac{1}{\Omega M_N}\oint_{r=d}
   \left[\Psi^S{\bf \nabla}\psi^*-\psi^*{\bf \nabla}\Psi^S\right] dS,
\end{equation}
where the integral is a surface integral over a sphere with
radius equal $d$.

The separation distance is still not yet specified. The main
idea of the separation
method is to choose $d$ such that the above surface
integral vanishes. This is possible for the relative
$l=0$ partial wave. In this case we write the integrand as
\[
   \Psi^S{\bf \nabla}\psi^*-\psi^*{\bf \nabla}\Psi^S
   =\Psi^S\frac{\partial \psi^*}{\partial r}-
   \psi^*\frac{\partial \Psi^S}{\partial r}
\]
and the above integral vanishes if the wave function obeys
the logarithmic boundary condition
\begin{equation}
      \frac{1}{\psi}\frac{\partial \psi}{\partial r}
      = \frac{1}{\Psi^S}\frac{\partial \Psi^S}{\partial r}
      \hspace{0.5cm} \mathrm{at}\hspace{0.5cm} r=d.
\end{equation}
Here we assume that the potential has a hard core and hence the correlated
wave function is identically equal to zero inside the core radius $r_c$. 
Note that the unperturbed wave function $\psi$ is not vanishing
for $r<r_c$. The correlated wave function $\Psi_S$ is pushed out by the
core, but once outside $r_c$ it is pulled in by the exterior attraction.
The separation distance is chosen such that there is just enough 
exterior attraction to balance the core repulsion so that the correlated 
wave function heals at $r=d$. By healing we mean the logarithmic 
condition in the above equation. This condition means that $V_S$
results in a zero phase shift. Hence an equivalent definition of
$d$ is that the short-range potential $V_S$ produces zero phase shifts.
Note that this is not always possible, as we will discuss in the next
subsection. In this case the separation distance does not exist.
Hitherto we have only exposed the basic ideas on which the
separation method is founded. We refer the reader to the literature
for numerical results, see e.g.\ \cite{kb66}.

Our original purpose is the calculation of the full $G$-matrix, not only
that of $G_S$. In fact, $G_S$ is merely an auxiliary potential,
introduced as an intermediate step in the calculation of $G$. Recall also
that we only calculated the diagonal matrix element of $G_S$, and that
these vanish. The hope is however that the full $G_S$ is small enough so
that $G$ may be calculated by a low-order perturbation expansion
in terms of $G_S$.
Using eq.\ (\ref{eq:gidsep}) we may write $G$ as
\begin{equation}
        G\approx G_S^{\dagger} +
        G_S^{\dagger}\left(\frac{Q}{e}-\frac{Q_2}{e_2}\right)G_S
        +\left(1+G_S^{\dagger}\frac{1}{e_2}\right)V_L
        \left(1+\frac{1}{e_2}G_S\right).
        \label{eq:gseploworder}
\end{equation}
The second term on the rhs.\ may be written as
\[
    G_S^{\dagger}\frac{Q-1}{e}G_S+
    G_S^{\dagger}\left(\frac{1}{e}-\frac{1}{e_2}\right)G_S,
\]
where the first term has commonly been denoted the Pauli correction term,
while the second is referred to as the dispersion correction term.

Here we note that one of the appealing features of the
separation method is the physically intuitive picture provided by it,
namely that the long-range part $V_L$ is the leading term of $G$.
Nevertheless, there is the lingering question about the accuracy
of this method. How accurate is $G$ given by, for example, the low-order
expansion of eq.\ (\ref{eq:gseploworder})?
Higher-order correction terms are progressively
more difficult to calculate, and in most actual cases, the numerics
sets a limit to how far we can go 
in this direction. It will indeed be very
desirable if one could formulate a different approach, where
one has the possibility of calculating $G$ almost exactly. We will see
how to obtain such a scheme in our calculations of the
$G$-matrix for finite nuclei and nuclear matter. Below, we will however
briefly revisit the so-called reference-spectrum method introduced
by Bethe, Brandow and Petschek \cite{bbp63}, an excellent
presentation of this method
can be found in the review article of Day \cite{day67}, which in spite of
its age still reads well.


\subsubsection{The reference-spectrum method}

This method is also based on the identity of eq.\ (\ref{eq:gidentity}).
Here we choose an auxiliary $G$-matrix $G_R$, the reference-spectrum
$G$-matrix, defined by
\begin{equation}
     G_R=V+V\frac{1}{e_R}G_R.
     \label{eq:grefs}
\end{equation}
Using eq.\ (\ref{eq:gidentity}), the $G$-matrix can be written as
\begin{equation}
  G=G_R^{\dagger}+G_R^{\dagger}\left(\frac{Q}{e}-\frac{1}{e_R}\right)G.
  \label{eq:grefsp}
\end{equation}
Unlike the separation method, we note that we have the whole potential
$V$ in the definition of $G_R$. Moreover, for some partial-wave channels,
the potential is repulsive for all $r$ and we have no exterior attraction. 
In this case the separation method we discussed in the previous subsection,
can not be used, as there is no exterior attraction to heal the wave 
function. It is mainly this difficulty which motivated the introduction
of the reference-spectrum method. In this method, one first evaluates the
defect wave function $\chi$ and thereafter one calculates $G_R$ in terms
of $\chi$.
The correlated wave function $\Psi^R$ is defined by
\begin{equation}
     G_R\ket{\psi_a}=V\ket{\Psi_a^R}
\end{equation}
where $\psi_a$ is the unperturbed wave function. From eq.\ (\ref{eq:grefs}),
we have
\begin{equation}
      \ket{\Psi_a^R}=\ket{\psi_a}+\frac{1}{e_R}V\ket{\Psi_a^R}.
      \label{eq:correlrefs}
\end{equation}
The defect wave function is defined as the difference
\begin{equation}
     \chi_a=\psi_a-\Psi_a^R.
\end{equation}
In other words, the correlated wave function is equal to the
unperturbed wave function minus the defect wave function (we will see 
examples of various defect wave functions in the next subsection, where
we discuss nuclear matter results).
Eq.\ (\ref{eq:correlrefs}) then leads to the result
\begin{equation}
   (e_R-V)\ket{\chi}=-V\ket{\psi},
   \label{eq:refspeq}
\end{equation}
where we for convenience have dropped the subscript $a$. The $G_R$-matrix
is just an auxiliary $G$-matrix, which we need in our determination of the
full $G$-matrix defined in eq.\ (\ref{eq:grefsp}). We need however to
define the energy denominator $e_R$. In principle $G$ is
independent of $e_R$, though in practice we hope that $G_R$ alone
should provide a good approximation to $G$. Suppose we define
\begin{equation}
   e_R=\omega -T_r -T_R,
\end{equation}
where $T_r$ and $T_R$ denote respectively the relative and center-of-mass kinetic
energies of the two interacting particles. Since $V$ depends on the
internucleon distance $r$ only, we can write the wave functions
\begin{equation}
    \psi ({\bf r}_1{\bf r}_2)=
    \frac{1}{\sqrt{\Omega}}\psi ({\bf r})e^{i{\bf KR}},
\end{equation}
and
\begin{equation}
    \chi ({\bf r}_1{\bf r}_2)=
    \frac{1}{\sqrt{\Omega}}\chi ({\bf r})e^{i{\bf KR}},
\end{equation}
as was done in connection with the separation method. Eq.\
(\ref{eq:refspeq}) takes then the form (in partial waves)
\begin{equation}
  \left(\omega -\frac{K^2}{4M_N}+\frac{1}{M_N}\frac{d^2}{dr^2}
  -\frac{l(l+1)}{r^2}-V\right)\chi_l(r)=-V\psi_l(r).
    \label{eq:resppartial}
\end{equation}
The asymptotic behaviors of the defect wave function $\chi$ can now
be readily seen. We let $V$ be a hard-core potential with core
radius $r_c$. For $r=r_c$, $V$ is infinite and we must therefore have
$\chi_l(r)=\psi_l(r)$ inside $r_c$. Now comes an important point.
If $(\omega -K^2/4M_N)$ is negative, we can define
\begin{equation}
     -\frac{\gamma^2}{M_N}=\omega -\frac{K^2}{4M_N},
\end{equation}
with $\gamma > 0$.
Eq.\ (\ref{eq:resppartial}) gives then
\begin{equation}
    \chi_l(r)\propto e^{-\gamma r}
    \hspace{0.5cm} \mathrm{as} \hspace{0.5cm}
     r\rightarrow \infty
\end{equation}
Since $\Psi^R=\psi -\chi$, we see that the correlated wave function $\Psi$
heals to $\psi$ at large $r$ values. A typical behavior of the defect
wave function for the $l=0$ channel is illustrated in fig.\
\ref{fig:correlwave}.
\begin{figure}
      \setlength{\unitlength}{1mm}
      \begin{picture}(100,80)
      \put(25,10){\epsfxsize=10cm \epsfbox{defex.eps}}
      \end{picture}
\caption{A typical defect wave function $\chi$ for a hard-core potential
with core radius $r_c$.}
\label{fig:correlwave}
\end{figure}
We see that the correlated wave function $\Psi^R$ is totally pushed out
from the hard core and is zero inside $r_c$. At large $r$-values it is bound to
heal to $\psi$. At some indermediate values of $r$, $\Psi^R$ overshoots
$\psi$ in order to accomodate the wave function displaced by the core.
With $\chi (r)$ obtained by solving the differential equation
indicated by eq.\ (\ref{eq:resppartial}), we can calculate the
matrix element
\begin{equation}
    \bra{\psi_a}G_R\ket{\psi_b}=\bra{\psi_a}V\ket{\psi_b-\chi_b},
\end{equation}
where the radial integral is now well-defined, even for a hard-core
potential. For a hard-core potential $\bra{\psi}V\ket{\psi}$ is not
well-defined, but $\bra{\psi}V\ket{\psi -\chi}$ is since $\psi -\chi=0$
inside the hard core.

A computational advantage of the reference-spectrum method is that it
provides a convenient scheme, as outlined above, to obtain the
reference $G$-matrix $G_R$. However, our goal is not only to obtain
$G_R$ but the full $G$-matrix $G$. The hope is that $G_R$ is already
a good approximation to $G$, and that $G$ may be accurately calculated
by way of a low-order perturbation expansion in terms of $G_R$. For example,
we may approximate eq.\ (\ref{eq:grefsp}) by
\begin{eqnarray}
  G&\approx {\displaystyle G_R^{\dagger}+
           G_R^{\dagger}\left(\frac{Q}{e}-\frac{1}{e_R}\right)
           G_R^{\dagger}}\nonumber \\
    &={\displaystyle G_R^{\dagger}+G_R^{\dagger}
           \left(\frac{Q}{e}-\frac{1}{e})G_R^{\dagger}+
           G_R^{\dagger}(\frac{1}{e}-\frac{1}{e_R}\right)G_R^{\dagger}}.
\end{eqnarray}
Here the second and third terms are usually referred to as the Pauli- and
the dispersion-correction terms, respectively.
However, in spite of the nice features of the reference-spectrum method,
we are still left with the problem of how to obtain a reliable
estimate of higher-order terms of the full $G$-matrix in terms
of $G_R$.
In the next subsection we demonstrate how one can obtain
(within the framework of the angle-average approximation) an ``exact''
solution of the $G$-matrix equation for nuclear matter.


\subsection{The G-matrix for nuclear matter}

In
a medium such as nuclear 
matter\footnote{We will postpone a thorougher discussion of
nuclear matter to sections 7 and 8. This subsection serves to
illustrate how one can calculate the $G$-matrix
"exactly" through matrix inversion techniques.} ,
we must account
for the fact that certain states are not available as intermediate
states in the calculation of the $G$-matrix.
Following the discussion
in the previous two subsections,
this is achieved by introducing the medium
dependent Pauli operator $Q$ in
eq.\ (\ref{eq:bspartial}). Further, the
energy $\omega$ of the incoming particles, given by a pure kinetic
term in eq.\ (\ref{eq:bspartial}), must be modified so as to allow
for medium corrections.
How to evaluate the Pauli operator for
e.g.\ nuclear matter is, however, not straightforward.
Before discussing how to evaluate the Pauli operator for nuclear matter,
we note that the $G$-matrix
is conventionally given in terms of partial waves and
the coordinates of the relative and center-of-mass motion, as
in eq.\ (\ref{eq:bspartial}).
If we assume that the $G$-matrix is diagonal in $\alpha$ ($\alpha$ is a shorthand
notation for $J$, $S$, $L$ and $T$),
eq.\ (\ref{eq:gmat}) can then be recast in the form of a
coupled-channels equation in the relative and center-of-mass system
\cite{ht70}
\begin{equation}
   G_{ll'}^{\alpha}(kk'K\omega )=V_{ll'}^{\alpha}(kk')
   +\sum_{l''}\int \frac{d^3 q}{(2\pi )^3}V_{ll''}^{\alpha}(kq)
   \frac{Q(q,K)}{\omega -H_0}
   G_{l''l'}^{\alpha}(qk'K\omega).
   \label{eq:gnonrel}
\end{equation}
This equation is similar in structure to the scattering
equations discussed in section 3, except that we now have
introduced the Pauli operator $Q$ and a medium dependent two-particle
energy $\omega$. The notations in this equation follow those of
subsection 3.3, where we discuss the solution of the scattering
matrix $T$.
The numerical details on how to solve the above $G$-matrix
equation through matrix inversion techniques are discussed in appendix B.
Note however that the $G$-matrix may not be diagonal in $\alpha$.
This is due to the fact that the
Pauli operator $Q$ is not diagonal
in the above representation in the relative and center-of-mass
system. The Pauli operator depends on the
angle between the relative momentum and the center of mass momentum.
This angle dependence causes $Q$ to couple states with different
relative angular
momentum ${\cal J}$. A partial wave decomposition of eq.\ (\ref{eq:gmat})
becomes therefore rather difficult.
The angle dependence of the Pauli operator
can be eliminated by introducing the angle-average
Pauli operator, where one replaces the exact Pauli operator $Q$
by its average $\bar{Q}$ over all angles for fixed relative and center-of-mass
momenta.
The choice of Pauli operator is decisive to the determination of the
sp
spectrum. Basically, to first order in the reaction matrix $G$,
there are three commonly used sp spectra, all
defined by the solution of the following equations
\begin{equation}
   \varepsilon_{m} = \varepsilon (k_{m})
   = t_{m} + u_{m}=\frac{k_{m}^2}{2M_N}
   +u_{m},
   \label{eq:spnrel}
\end{equation}
and
\begin{equation}
   \begin{array}{ccc}\\
   u_{m} =& {\displaystyle \sum_{h \leq k_F}}
   \left\langle m h \right| G(\omega = \varepsilon_{m} + \varepsilon_h )
   \left| m h \right\rangle_{AS} & \hspace{3mm}k_m \leq k_M,  \\ \\
   u_m=&0,& k_m > k_M. \end{array}
   \label{eq:selfcon}
\end{equation}
For notational economy, we set $|{\bf k}_m|=k_m$.
Here we employ antisymmetrized matrix elements (AS), and $k_M$ is a cutoff
on the momentum. Further, $t_m$ is the sp kinetic
energy and similarly $u_m$
is the
sp potential.
The choice of cutoff $k_M$ is actually what determines the three
commonly used sp spectra.
In the conventional BHF approach one employs $k_M = k_F$,
which leads
to a Pauli operator $Q_{\mathrm{BHF}}$ (in the laboratory system) given by
\begin{equation}
   Q_{\mathrm{BHF}}(k_m , k_n ) =
    \left\{\begin{array}{cc}1,&min(k_m ,k_n ) > k_F\\
    0,&\mathrm{else}.\end{array}\right.
    \label{eq:bhf},
\end{equation}
or, since we will define an
angle-average Pauli operator in the relative and center-of-mass
system, we have
\begin{equation}
     \bar{Q}_{\mathrm{BHF}}(k,K)=\left\{\begin{array}{cc}
         0,&k\leq \sqrt{k_{F}^{2}-K^2/4}\\
         1,&k\geq k_F + K/2\\
	\frac{K^2/4+k^2 -k_{F}^2}{kK}&\mathrm{else},\end{array}\right.
    \label{eq:qbhf}
\end{equation}
with $k_F$ the momentum at the Fermi surface.
The difference between the Pauli operators in the lab and the
relative and center-of-mass systems
is shown in fig.\ \ref{fig:pauliops} (a) and (b),
respectively. 
See e.g.\  \cite{ht70} for further details.
\begin{figure}[hbtp]
      \setlength{\unitlength}{1mm}
      \begin{picture}(140,90)
      \put(25,10){\epsfxsize=12cm \epsfbox{pauli.eps}}
      \end{picture}
\caption{In (a) we show the Pauli operator for standard BHF theory
in the lab frame, while (b) is the corresponding diagrammatic
representation in the relative and center-of-mass system.
Note that ${\bf K}={\bf k}_n+{\bf k_m}$ and ${\bf k}=
\frac{1}{2}({\bf k}_n-{\bf k_m})$.}
\label{fig:pauliops}
\end{figure}
The BHF choice sets $u_k = 0$ for $k > k_F$, which leads
to an unphysical, large gap at the Fermi surface, typically
of the order of $50-60$ MeV. 
To overcome the gap
problem, Mahaux and collaborators \cite{mah85}
introduced a continuous sp spectrum
for all values of $k$. The divergencies
which then may occur in eq.\ (\ref{eq:gnonrel}) are taken care of by
introducing
a principal value integration in eq.\ (\ref{eq:gnonrel}),
to retain only the
real part contribution to the $G$-matrix.

Finally,
the model-space BHF approach which we also will
employ (to be discussed in section 7), adopts a cutoff $k_M =  ak_F$,
where $a$ is a constant. Thus, the cutoff $k_M$ is given
as a multiple
of the Fermi momentum. A frequently used value \cite{km83}
is $a= 2$.
This means that we extend
the BHF spectrum
to go beyond $k_F$. Here we
limit the attention to the standard BHF calculations. A more detailed
comparison with the BHF and the MBHF methods will be given in
section 7.
To define the energy denominators we will also make use of the
angle-average approximation.
The angle dependence is handled by the
so-called effective mass approximation. The single-particle energies
in nuclear matter are assumed to have the simple quadratic form
\begin{equation}
   \begin{array}{ccc}
   \varepsilon (k_m)=&
   {\displaystyle\frac{\hbar^{2}k_m^2}
   {2M_{N}^{*}}}+\Delta ,&\hspace{3mm}k_m\leq k_F\\
   &&\\
   =&{\displaystyle\frac{\hbar^{2}
   k_m^2}{2M_{N}}},&\hspace{3mm}k_m> k_F ,\\
   \end{array}
   \label{eq:spen}
\end{equation}
where $M_{N}^{*}$ is the effective mass of the nucleon and $M_{N}$ is the
bare nucleon mass. For particle states above the Fermi sea we choose
a pure kinetic energy term, whereas for hole states,
the terms $M_{N}^{*}$ and $\Delta$, the latter being 
an effective single-particle
potential related to the $G$-matrix, are obtained through the
self-consistent Brueckner-Hartree-Fock procedure.
The sp potential is obtained through the same angle-average approximation
\cite{ht70}
\begin{eqnarray}
  \label{eq:Uav}
   U(k_m) & = & \sum_{l\alpha} (2T+1)(2J+1)
   \left \{ \frac{8}{\pi}\int_{0}^{(k_F-k_m)/2}
   k^2dk G_{ll}^{\alpha}(k,\bar{K}_1) \right.  \\
   &  &  \left.
    + \frac{1}{\pi k_m}\int_{(k_F-k_m)/2}^{(k_F+k_m)/2}
   kdk (k_F ^2-(k_m-2k)^2)
   G_{ll}^{\alpha}(k,\bar{K}_2)  \right \}  \nonumber,
\end{eqnarray}
where we have defined
\begin{equation}
    \bar{K}_1^2=4(k_m^2+k^2),
\end{equation}
and
\begin{equation}
    \bar{K}_2^2=4(k_m^2+k^2)-(2k+k_m-k_F)(2k+k_1+k_F).
\end{equation}
This
self-consistency scheme consists in choosing adequate initial values of the
effective mass and $\Delta$. The obtained $G$-matrix is in turn used to
obtain new values for $M_{N}^{*}$ and $\Delta$. This procedure
continues until these parameters vary little.


Having discussed how to evaluate the $G$-matrix for nuclear matter,
we will in the remainder of this subsection study how the much debated
tensor
force of the NN potential influences observables like the binding
energy per particle in nuclear matter. To examplify the role of the
tensor force, we will show results for the potential energy per particle
 and the defect wave function for the $^3S_1$  partial wave in nuclear
matter. This discussion is closely
connected to that in the next subsection, the $G$-matrix for
finite nuclei, and the discussion on the $T$-matrix in subsection
3.3.

It has been argued that the increased attraction provided
by modern meson-exchange potentials like those of the Bonn group
\cite{mhe87,mac89}, could be ascribed to the fairly weak tensor force
exhibited by these interactions. In order to quantitatively
explain this increased binding we show in this subsection examples
using the Bonn potentials defined
in table A.2 of ref.\ \cite{mac89}, where version A exhibits the weakest
tensor force.
The nuclear matter results for the
average potential energy for the partial wave $^{3}S_1$
obtained with potentials A, B and C are displayed
in fig.\ \ref{fig:bhfnm}
for the BHF calculation discussed above considering various values of
the Fermi
momentum $k_F$ (improvements of these results are given
to section 7).
\begin{figure}[hbtp]
      \setlength{\unitlength}{1mm}
      \begin{picture}(140,150)
      \put(25,10){\epsfxsize=12cm \epsfbox{fig24.eps}}
      \end{picture}
\caption{Contributions to the binding energy per particle
from the $^3S_1$ partial wave from the Bonn A, B and C
potentials discussed in the text. Taken from ref.\ [59].}
\label{fig:bhfnm}
\end{figure}

The bulk of the $G$-matrix in the $^3S_1$ wave 
behaves similarly to the scattering matrix $T$, i.e.\,
\begin{equation}
    G\approx V_C + V_T \frac{\tilde{Q}}
    {\omega - \tilde{Q}T\tilde{Q}}V_T.\label{eq:gappro}
\end{equation}
The latter equation differs however from the equation for the
$T$-matrix (see subsection 3.3) due to the
Pauli operator and the fact that the starting energy is given for particles
in a medium. The second term in eq.\ (\ref{eq:gappro}) is then quenched
by the combined effect of the Pauli operator and the attractive energy
denominator. Thus, since all potentials are fit to the same set of data,
a potential with a weak (strong) tensor force needs a larger (weaker)
central force to arrive at the same on-shell scattering matrix. In the
nuclear medium, the quenching of the second term will then be the more
important the larger the tensor force. This is clearly reflected in fig.
\ref{fig:bhfnm}. At small values for the relative monentum $k$, the
potentials differ negligibly, which reflects the fact that all potentials
yield the same on-shell  $T$-matrix. At higher densities, the
quenching mechanisms due to the Pauli operator and the energy denominator
account for the differences in binding exhibited by the three potentials,
with the potential exhibiting the weakest tensor force (A) being the most
attractive.

In connection with the tensor force,
it is also instructive to consider the behavior of different potentials
in a nuclear medium, by analyzing the
renormalized wave functions arising from these potentials.
We see from the eq.\ (\ref{eq:wave})
that the second term contains the Pauli
operator $Q$, such that the difference between the uncorrelated
and the correlated wave functions, 
the defect wave function $\chi$ discussed in the two previous
subsections, contains only contributions compatible
with the Pauli principle.
In configuration space, for a given value of $k$, we can write the
wave function $\Psi$ as
\begin{equation}
   \Psi_{ll'}^{\alpha}(r,k)=j_l(kr)\delta_{ll'}+
   \frac{2}{\pi}\int_{0}^{\infty}\frac{dqq^2 j_{l'}(qr)
   Q(q,K)G_{ll'}^{\alpha}(qkK\omega)}
   {\omega -H_{0}(q,K)},
\label{eq:waver}
\end{equation}
with the starting energy $\omega$ defined above.
To calculate the wave function $\Psi$ we employ here the Bonn A potential defined
by the parameters of table A.2 in ref.\ \cite{mac89}. Further, rather than using 
the Bonn B and C potentials, we plot results for the Paris
potential \cite{paris80} and the Reid-soft-core potential \cite{reid68}, potentials
which have been widely used in both nuclear matter and finite nuclei calculations.
The tensor force of the Paris potential is similar to that of the
Bonn C potential.
The BHF self-consistent results
with $k_F =1.4$ fm$^{-1}$ have been used.
\begin{figure}[hbtp]
      \setlength{\unitlength}{1mm}
      \begin{picture}(140,150)
      \put(25,10){\epsfxsize=12cm \epsfbox{fig25.eps}}
      \end{picture}
\caption{Contributions to the defect wave function
for the $^3S_1$ partial wave from the Bonn A, Paris and Reid-soft-core
potentials discussed in the text for $k=0.8$ fm$^{-1}$.}
\label{fig:chap3def}
\end{figure}
In fig.\ \ref{fig:chap3def} we show the defect wave function, i.e.~
the second term on the right hand side of eq.\ (\ref{eq:waver}), in coordinate
space for the above three potentials
for the $^{3}S_1$ channel. We also include the defect wave function
obtained with
the short-range parametrization
$j_0 (kr)\left(1-j_0 (qr)\right)$ with
$q=800$ MeV (roughly the mass of the heaviest mesons
in OBE models for the NN potential). The latter parametrization has been
a much favored choice in the literature \cite{town87,town92},
and it is amusing to
see how well these Bessel functions agree with the defect function of 
the Reid potential.
However, it is important to note that 
fig.\ \ref{fig:chap3def} reflects the fact that potentials with
a weaker tensor force exhibit the smallest defect function at
short distances and at distances were the potential is attractive. This
leads in turn to larger $G$-matrix elements and explains why potentials
like the Bonn A result in more binding energy for nuclear matter
and finite nuclei.
Finally, it is also worth noting that all potentials
result in similar defect wave functions at larger $r$ values, where
one-pion exchange dominates.




\subsection{Double-partitioned scheme for calculations of the $G$-matrix
for finite nuclei}

Before we proceed in detailing the calculation of the $G$-matrix
appropriate for finite nuclei, certain approximations need be explained.

As discussed in section 4, the philosophy behind perturbation theory is
to reduce the intractable full Hilbert space problem to one which
can be solved within a physically motivated model space, defined by the
operator $P$. The excluded degrees of freedom are represented by the
projection operator $Q$. The definition of these operators is connected
to the nuclear system and to the perturbative expansions defined
in section 4. Consider the evaluation of the effective interaction
needed in calculations of the low-lying states of $^{18}$O. From
experimental data and theoretical calculations the belief is that
several properties of this nucleus can be described by a model
space consisting of a closed $^{16}$O core (consisting of the filled
$0s$- and $0p$-shells) and two valence neutrons
in the $1s0d$-shell. In fig.\ \ref{fig:orbits} we exhibit this division
in terms of h.o.~sp orbits.
\begin{figure}[hbtp]
    \setlength{\unitlength}{1mm}
    \begin{picture}(140,130)
      \put(25,10){\epsfxsize=14cm \epsfbox{ho.eps}}
    \end{picture}
\caption{Classification of harmonic oscillator single-particle
orbits.}
\label{fig:orbits}
\end{figure}
The active sp states in the $1s0d$-shell are then given by the  $0d_{5/2}$, 
$0d_{3/2}$ and $1s_{1/2}$ orbits, labels $4-6$ in fig.\ \ref{fig:orbits}.
The remaining states enter the definition of
$Q$. Once we have defined $P$ and $Q$ we proceed in constructing the $G$-matrix
and the corresponding perturbative expansion in terms of the $G$-matrix. 
There are however several ways of choosing $Q$. A common procedure is to
specify the boundaries of $Q$ by three numbers, $n_1$, $n_2$ and $n_3$, explained
in fig.\ \ref{fig:qoperat}.
\begin{figure}[hbtp]
      \setlength{\unitlength}{1mm}
      \begin{picture}(110,110)
      \put(25,10){\epsfxsize=11cm \epsfbox{paulifin.eps}}
      \end{picture}
\caption{Definition of the $P$ (shaded area) and $Q$ operators
appropriate for the definition of the $G$-matrix and the effective
interaction. Outside the shaded area limited by the boundaries $n_1$,
$n_2$ and $n_3$ $P=0$ and $Q=1$.}
\label{fig:qoperat}
\end{figure}
For $^{18}$O we would choose $(n_1=3,n_2=6,n_3=\infty)$. 
Our choice of 
$P$-space implies that the passive single-particle states start from the
$1p0f$-shell (numbers 7--10 in fig.\ \ref{fig:orbits}), and orbits 1, 2
and 3 are passive hole states. Stated differently, this means that $Q$
is constructed so as to prevent scattering into intermediate 
two-particle states 
with one particle in the $0s$- or $0p$-shells or both particles
in the $1s0d$-shell. This definition of the $Q$-space influences the determination
of the \qbox discussed in section 4. Consider the diagrams displayed
in fig.\ \ref{fig:qboxexam1}.
\begin{figure}[hbtp]
      \setlength{\unitlength}{1mm}
      \begin{picture}(140,80)
      \put(25,10){\epsfxsize=12cm \epsfbox{qdiags.eps}}
      \end{picture}
\caption{Examples of diagrams which may define the \qbox. Greek
letters are active sp states while orbits which define $Q$ are given by
latin letters, with $p_i$ and $h_i$ representing particle and hole
state $i$, respectively.
An upgoing arrow represents a particle state (both
active and passive) while a downgoing arrow is a hole state.}
\label{fig:qboxexam1}
\end{figure}
Diagram (a) of this figure is just the $G$-matrix and is allowed in the definition
of the \qbox. With our choice $(n_1=3,n_2=6,n_3=\infty)$, diagram (b) is not
allowed since the intermediate state consists of passive particle
states  and is already included in the evaluation of the $G$-matrix. Similarly,
diagram (c) is also not allowed whereas diagram (d) is allowed. Now an important
subtlety arises. If we evaluate the $G$-matrix with the boundaries
$(n_1=3,n_2=10,n_3=\infty)$, and define the $P$-space of {\em 
the effective interaction}
by including orbits 4 to 6 only, then diagrams (b) and (c)
are allowed if $7\leq p_1 , p_2 \leq 10$\footnote{A word on notations:
In referring to model-space diagrams, greek letters will always
represent the sp orbits which define the model space, whereas latin letters
represent the remaining orbits. Notice however that, particle or hole 
labels may run over model-space orbits as well. In this case, the total
two-body wave function does not belong to the chosen model space.}.
In this way we allow for 
intermediate two-particle states as well with orbits outside the 
model-space of the effective interaction. The reader should notice the above
differences, i.e.\ that the $Q$-space defining the $G$-matrix and 
$H_{\mathrm{eff}}$
may differ. Throughout the rest of this work we will apply such  
different $Q$-spaces, and label the $Q$-operator of the $G$-matrix
by $\tilde{Q}$. Several arguments 
for choosing $(n_1=3,n_2=10,n_3=\infty)$ in calculations
of the $G$-matrix appropriate for $^{18}$O can be found in refs.\
\cite{tk72,kkko76}. Here we limit the attention to the removal
of the dependence on the auxiliary potential $U$ in the definition of 
$G$. Since the interaction term $H_1=V-U$ is used to define the \qbox, there
certainly are diagrams with $U$ vertices in the \qbox . Certain of these
diagrams can be removed by adding ladder type diagrams with $U$ insertions
to the intermediate states of the $G$-matrix. Such terms are shown in fig.\
\ref{fig:uinsert}. 
\begin{figure}[hbtp]
      \setlength{\unitlength}{1mm}
      \begin{picture}(140,80)
      \put(25,10){\epsfxsize=12cm \epsfbox{bhfladder.eps}}
      \end{picture}
\caption{
Examples of ladder diagrams with $U$ and BHF insertions to the
intermediate particle states.}
\label{fig:uinsert}
\end{figure}
By adding such diagrams ( this amounts to summing a geometric series in $U$)
we obtain a new $G$-matrix $\tilde{G}$
\begin{equation}
  \tilde{G}=V+V\frac{\tilde{Q}}{\omega -\tilde{Q}T\tilde{Q}}\tilde{G},
\label{eq:gfinite}
\end{equation}
defined as the sum of all ladder diagrams with $U$ insertions to the intermediate
states as well. The question now arises, which $G$-matrix should one choose,
$G$ or $\tilde{G}$? Here we recast some of the arguments given by the authors of
ref.\ \cite{kkko76}.
First, since the hamiltonian $H$ does not depend on $U$ one should seek to minimize
the dependence of $G$ on $U$. This is actually accounted for in our definition
of $\tilde{G}$. Secondly, the BHF self-energy insertions for high-lying states
in diagram (b) of fig.\ \ref{fig:uinsert} may not cancel the corresponding
$U$ insertion diagram in (a), since the BHF insertions to a particle line
are off the energy shell. For low-lying particle states the off-shell character
of the BHF insertions is in general rather weak \cite{kkko76}, and the cancellation
between diagrams (a) and (b) is approximately exact. For high-lying states
it is not clear whether this cancellation may take place. Thus a 
$G$-matrix defined in eq.\ (\ref{eq:gfinite}) is to be preferred since
the $G$-matrix of eq.\ (\ref{eq:gmat}) presupposes that $U$ and BHF 
insertions to intermediate particle states cancel exactly. 
There are also considerations which favor $\tilde{G}$ from studies of three-body
clusters in nuclear matter \cite{kkko76}. In those studies, certain
three-body clusters nearly cancel diagrams with BHF insertions to the
intermediate particle states, leaving only terms with $U$ insertions
included in $\tilde{G}$. 

The above remarks made the authors of ref.\ \cite{kkko76} suggest a 
two-step approach to the calculation of the $G$-matrix for finite nuclei.
The intermediate states are divided into two parts, high-lying and low-lying
ones.
For the low-lying states one ignores all self-energy insertions, due to the
above nuclear matter arguments. Thus one should use $\tilde{G}$ for
high-lying intermediate states only. For the low-lying states, one assumes
that the $U$ and BHF insertions cancel approximately, leading to the
$G$-matrix of eq.\ (\ref{eq:gmat}) with $V$ replaced by $\tilde{G}$ and
$\tilde{Q}$ restricted to low-lying intermediate particle states. 
This explains also our choice of boundaries for $\tilde{P}$ and
$\tilde{Q}$, given by $(n_1=3,n_2=10,n_3=\infty)$ for $^{18}$O.

Let us be more specific and detail this double-partitioned procedure.
We have defined $\tilde{G}$ in terms of plane wave intermediate states,
while $G$ has harmonic oscillator intermediate states (this is one
possible choice for $U$). We divide the exclusion operator
into two parts, one which represents the low-lying states $Q_l$ and
one which accounts for high-lying states $Q_h$, viz.\
\[
    Q=Q_l+Q_h=Q_l+\tilde{Q}.
\]
If we consider $^{18}$O as our pilot nucleus, we may define $Q_l$ to consist
of the sp orbits of the $1p0f$-shell, orbits $7-10$ in fig.\ \ref{fig:orbits},
described by h.o.\ states. $Q_h$ represents then the remaining orthogonalized
intermediate states.
Using the identity of Bethe, Brandow and Petschek \cite{bbp63} of
eq.\ (\ref{eq:gidentity}) we can express $G$ in terms of $\tilde{G}$ as
\begin{equation}
        G=\tilde{G} +\tilde{G}
        \left(\frac{Q_l}{\omega -H_0}\right)G,
        \label{eq:gidfinite}
\end{equation}
where
\[
  \tilde{G}=V+V\frac{Q_h}{\omega -T}\tilde{G},
\]
and we have assumed that $\tilde{G}$ is hermitian and that $[Q_l,H_0]=0$.
We see again that the philosophy behind both the separation and
reference-spectrum methods is employed here as well. We first calculate
a ``reference'' $G$-matrix ($\tilde{G}$ in our case), and then insert this
in the expression for the full $G$-matrix. The novelty here is that
we are able to calculate $\tilde{G}$ exactly through operator relations
to be discussed below. In passing we note that $G$ depends significantly 
on the choice of $H_0$, though the low-lying intermediate states
are believed to be fairly well represented by h.o.\ states. 
Also, the authors of ref.\ \cite{kkko76} demonstrate that low-lying
intermediate states are not so important in $G$-matrix calculations,
being consistent with the short-range nature of the NN potential.
Since we let $Q_l$ to be defined by the orbits of the $1p0f$-shell,
and the energy difference between two particles in the 
$sd$-shell and $pf$ shell is of the order $-14$ MeV, we can treat 
$G$ as a perturbation expansion in $\tilde{G}$.
Eq.\ (\ref{eq:gidfinite}) can then be written as 
\begin{equation}
        G=\tilde{G} +\tilde{G}
        \left(\frac{Q_l}{\omega -H_0}\right)\tilde{G}
        +\tilde{G}
        \left(\frac{Q_l}{\omega -H_0}\right)\tilde{G}
        \left(\frac{Q_l}{\omega -H_0}\right)\tilde{G} +\dots
\end{equation}
The only intermediate states are those defined by the $1p0f$-shell.
The second term on the rhs.\ is nothing but the second-order 
particle-particle ladder. The third term is then the third-order ladder
diagram in terms of
$\tilde{G}$. As shown by the authors of ref.\ \cite{kkko76}, the inclusion
of the second-order particle-particle diagram in the evaluation
of the \qbox, represents a good approximation (in our work we will
include all diagrams through third order).
However,
the still unsettled problem is how to define
the boundary between 
$Q_l$ and $Q_h$. 



Now we will discuss how to compute $\tilde{G}$. The quenching
of the tensor force  
due to the dependence of the starting
energy and the Pauli operator are also discussed. Finally, the reader
has probably noted that in the above examples we have set 
$n_3=\infty$. This is however not feasible in actual calculations, which
in turn dictates an approximation to $n_3$. We discuss such approximations as
well.

In this work we solve the equation for the $G$-matrix
for finite nuclei by employing
a formally
exact technique for handling $\tilde{Q}$, originally presented by
Tsai and Kuo \cite{tk72} and discussed in ref.\ \cite{kkko76}.
Tsai and Kuo employed the matrix identity
\begin{equation}
  \tilde{Q}\frac{1}{\tilde{Q}A\tilde{Q}}
  \tilde{Q}=\frac{1}{A}-
   \frac{1}{A}\tilde{P}\frac{1}{\tilde{P}A^{-1}\tilde{P}}\tilde{P}\frac{1}{A},
\end{equation}
with $A=\omega -T$, to rewrite eq.\ (\ref{eq:gfinite}) as\footnote{We will omit the
label $\tilde{G}$ for the $G$-matrix for finite nuclei, however it is
understood that the $G$-matrix for finite nuclei is calculated according
to eq.\ (\ref{eq:gfinite}) This means that we have to 
include the particle-particle ladder diagrams in the 
\qbox. The labels $\tilde{Q}$ and $\tilde{P}$
remain in order to distinguish these from those employed in the
definition of the effective interaction.}
\begin{equation}
   G = G_{F} +\Delta G,\label{eq:gmod}
\end{equation}
where $G_{F}$ is the free $G$-matrix defined as
\begin{equation}
   G_{F}=V+V\frac{1}{\omega - T}G_{F}. \label{eq:freeg}
\end{equation}
The term $\Delta G$ is a correction term defined entirely within the
model space $\tilde{P}$ and given by
\begin{equation}
   \Delta G =-V\frac{1}{A}\tilde{P}
   \frac{1}{\tilde{P}A^{-1}\tilde{P}}\tilde{P}\frac{1}{A}V.
\end{equation}
Employing the definition for the free $G$-matrix of eq.\ (\ref{eq:freeg}),
one can rewrite the latter equation as
\begin{equation}
  \Delta G =-G_{F}\frac{1}{e}\tilde{P}
  \frac{1}{\tilde{P}(e^{-1}+e^{-1}G_{F}e^{-1})
  \tilde{P}}\tilde{P}\frac{1}{e}G_F,
\end{equation}
with $e=\omega -T$.
We see then that the $G$-matrix for finite nuclei
is expressed as the sum of two
terms; the first term is the free $G$-matrix with no Pauli corrections
included, while the second term accounts for medium modifications
due to the Pauli principle. The second term can easily
be obtained by some simple matrix operations involving
the model-space matrix $\tilde{P}$ only.

To calculate $G_F$ one needs only to solve eq.\ (\ref{eq:freeg})
via e.g.\ momentum space inversion techniques \cite{ht70}.
The equation for the free matrix $G_F$ is solved in momentum space and we
obtain
\begin{equation}
    \bra{kKlL{\cal J}ST}G_F\ket{k'Kl'L{\cal J}S'T}.\label{eq:freeg2}
\end{equation}
Transformations from the relative and center-of-mass motion
system to the lab system will be discussed
below.

To obtain a $G$-matrix in a h.o.~basis, eq.\ (\ref{eq:freeg2})
is transformed into
\begin{equation}
     \bra{nNlL{\cal J}ST}G_F\ket{n'N'l'L'{\cal J}S'T},
\end{equation}
with $n$ and $N$ the principal quantum numbers of the relative and
center-of-mass motion, respectively.
Moreover, the wave function $\ket{nlNL{\cal J}ST}$, with $nNlL$ the
oscillator quantum numbers of the relative motion and center-of-mass,
is related
to $\ket{klKL{\cal J}ST}$ by \cite{bon89}
\begin{equation}
   \ket{nlNL{\cal J}ST}= \int k^{2}K^{2}dkdKR_{nl}(\sqrt{2}\alpha k)
R_{NL}(\sqrt{1/2}\alpha K)
\ket{klKL{\cal J}ST},
\end{equation}
with $\alpha$ being the oscillator length and $R_{nl}$ and $R_{NL}$ the HO
functions in momentum space.

Depending on the choice of single-particle basis, the $G$-matrix elements
needed in the evaluation of the above expressions take different forms.
The most commonly employed sp basis is the harmonic oscillator, which
in turn means that
a two-particle wave function with total angular momentum $J$
and isospin $T$
can be expressed as \cite{law80}
\begin{equation}
\begin{array}{ll}
\ket{(n_{a}l_{a}j_{a})(n_{b}l_{b}j_{b})JT}=&
{\displaystyle
\frac{1}{\sqrt{(1+\delta_{12})}}
\sum_{\lambda S{\cal J}}\sum_{nNlL}}
F\times \langle ab|\lambda SJ \rangle\\
&\times (-1)^{\lambda +{\cal J}-L-S}\hat{\lambda}
\left\{\begin{array}{ccc}L&l&\lambda\\S&J&{\cal J}
\end{array}\right\}\\
&\times \left\langle nlNL| n_al_an_bl_b\right\rangle
\ket{nlNL{\cal J}ST},\end{array}\label{eq:hoho}
\end{equation}
where the term
$\left\langle nlNL| n_al_an_bl_b\right\rangle$
is the familiar Moshinsky bracket, see e.g. ref.\ \cite{law80}.
The term $\langle ab|LSJ \rangle $ is a shorthand
for the $LS-jj$ transformation coefficient,
\begin{equation}
     \langle ab|\lambda SJ \rangle = \hat{j_{a}}\hat{j_{b}}
     \hat{\lambda}\hat{S}
     \left\{
    \begin{array}{ccc}
       l_{a}&s_a&j_{a}\\
       l_{b}&s_b&j_{b}\\
       \lambda    &S          &J
    \end{array}
    \right\}.\label{eq:lstrans}
\end{equation}
Here
we use $\hat{x} = \sqrt{2x +1}$.
The factor $F$ is defined as $F=\frac{1-(-1)^{l+S+T}}{\sqrt{2}}$ if
$s_a = s_b$.

The $G$-matrix in terms of harmonic oscillator wave functions reads
\begin{equation}
   \begin{array}{ll}
  \bra{(ab)JT}G\ket{(cd)JT}=&
  {\displaystyle \sum_{\lambda \lambda ' SS' {\cal J}}\sum_{nln'l'NL}
  \frac{1}{\sqrt{(1+\delta_{ab})
  (1+\delta_{cd})}}\left(1-(-1)^{l+S+T}\right)}
  \\
  &\times\langle ab|\lambda SJ\rangle \langle cd|\lambda 'S'J\rangle
  \left\langle nlNL| n_{a}l_{a}n_{b}l_{b}\lambda\right\rangle
  \left\langle n'l'NL| n_{c}l_{c}n_{d}l_{d}\lambda ' \right\rangle
  \\
  &\times \hat{{\cal J}}(-1)^{\lambda + \lambda ' +l +l'}
  \left\{\begin{array}{ccc}L&l&\lambda\\S&J&{\cal J}
  \end{array}\right\}
  \left\{\begin{array}{ccc}L&l'&\lambda '\\S&J&{\cal J}
  \end{array}\right\}
  \\
  &\times\bra{nNlL{\cal J}ST}G\ket{n'N'l'L'{\cal J}S'T},
  \end{array} \label{eq:gmat2}
\end{equation}
where $G$ is the given by the sum $G = G_{F} +\Delta G$.
The label $a$ represents here all the single particle quantum numbers
$n_{a}l_{a}j_{a}$.

The only approximation needed in the calculation of the $G$-matrix 
is the choice of the boundary variable $n_3$. In table \ref{tab:n3converge}
we display matrix elements of the $sd$-shell for $J=0,1$ and $T=0,1$ using
different approximations to $n_3$\footnote{Hereafter we abbreviate
$(n_1=3,n_2=10,n_3=x)$ to $(3,10,x)$.}. The results have been obtained
with the Bonn A potential of table A.1 of ref.\ \cite{mac89}.
An oscillator energy of $14$ MeV was used. 
The starting energy is $-10$ MeV.
The most significant contribution stems from values of $n_3\leq 21$, whereas
greater $n_3$ values give only small contributions. For the
above matrix elements this can be understood from simple considerations
on conservation of the center-of-mass momentum. The matrix elements are
given for valence space states, and the valence particles have momentum 
around $k_F$. This means that an intermediate particle state in the 
calculation of the $G$-matrix can at most achieve a momemtum  $\sim 3k_F$,
when the boundaries are defined as $(3,10,x)$, since the hole momentum
can at most be equal $k_F$.  
\begin{table}[hbtp]
\caption{Dependence of the $G$-matrix on the choice of $n_3$ for
matrix elements in the $sd$-shell for $J=0,1$ and $T=0,1$.}
\begin{center}
\begin{tabular}{lllllrrrrr}
\\\hline
$JT$&$j_{a}$&$j_{b}$&$j_{c}$&$j_{d}$&
\multicolumn{1}{c}{(3,10,45)}&
\multicolumn{1}{c}{(3,10,66)}&
\multicolumn{1}{c}{(3,10,78)}&
\multicolumn{1}{c}{(3,10,91)}&
\multicolumn{1}{c}{(3,10,120)}
\\\hline
01&$d_{5/2}$&$d_{5/2}$&$d_{5/2}$&$d_{5/2}$
&-1.713&-1.712&-1.712&-1.712&-1.712\\
10&&&&&-0.552&-0.549&-0.548&-0.548&-0.548\\
01&$d_{3/2}$&$d_{3/2}$&$d_{3/2}$&$d_{3/2}$
&-0.324&-0.323&-0.323&-0.323&-0.323\\
10&&&&&-0.339&-0.336&-0.335&-0.335&-0.335\\
01&$s_{1/2}$&$s_{1/2}$&$s_{1/2}$&$s_{1/2}$
&-2.418&-2.417&-2.417&-2.417&-2.417\\
10&&&&&-3.762&-3.718&-3.700&-3.691&-3.690
\\ \hline
\end{tabular}
\end{center}
\label{tab:n3converge}
\end{table}
It is worth noting however that the $T=0$ matrix elements show a much
weaker convergence compared with the $T=1$ elements. The latter can
be approximated with the choice $(3,10,45)$, whereas the $T=0$
matrix elements show differences of the order $0.1\%$-$0.7\%$ 
when going from
$n_3=66$ to $n_3=91$. For $n_3=91$ and higher values 
the difference is negligible\footnote{For $n_3=66$, this means that the
we include sp orbits up to the last orbit in the $5s4d3g2i1k0m$-shell,
$n_3=78$ means the last orbit in the $5p4f3h2j1l0n$-shell and so
forth.}.
Could this poorer convergence in the $T=0$ channel be retraced to the
tensor force? To better understand this let us single out the
matrix elements
\begin{equation}
     \bra{(0s_{1/2})^{2}JT=10}G\ket{(0s_{1/2})^{2}JT=10},
\end{equation}
and
\begin{equation}
    \bra{(0s_{1/2})^{2}JT=01}G\ket{(0s_{1/2})^{2}JT=01}.
\end{equation}
The first matrix element receives contributions from 
the $^{3}S_1$ partial wave only. The $^{3}D_1$ partial wave comes
in as an intermediate state contribution in the evaluation of the
$G$-matrix and serves to quench the second- and higher-order
contributions. Thus, the first matrix element allows us to study
directly the dependence on the tensor force. Also, the partial
waves $^{3}S_1$ and $^{3}D_1$ are the most important ones in
the analysis of the tensor force (recall the nuclear and neutron matter
discussion). The second matrix element receives contributions from
the $^{1}S_0$ partial wave only.

In order to test the dependence upon the tensor force we display
in table \ref{tab:tensorss} the results for the above two 
matrix elements obtained with both the Bonn A and C potentials of
table A.1 of ref.\ \cite{mac89}, with
the C potential exhibiting the stronger tensor force.
\begin{table}[hbtp]
\caption{Dependence of
the $\bra{(0s_{1/2})^{2}}G\ket{(0s_{1/2})^{2}}$
matrix element
on the choice of $n_3$ for the Bonn A (first row in each entry)
and the Bonn C potential (second row) for a starting energy
$-5$ MeV.}
\begin{center}
\begin{tabular}{lllllrrrrr}
\\ \hline
$JT$&\multicolumn{4}{c}{$\bra{(0s_{1/2})^{2}}G\ket{(0s_{1/2})^{2}}$} &
\multicolumn{1}{c}{(3,10,45)}&
\multicolumn{1}{c}{(3,10,66)}&
\multicolumn{1}{c}{(3,10,78)}&
\multicolumn{1}{c}{(3,10,91)}&
\multicolumn{1}{c}{(3,10,120)}
\\  \hline
10&&&&&-9.341&-9.331&-9.328&-9.328&-9.327\\
  &&&&&-9.100&-9.088&-9.084&-9.083&-9.082\\
01&&&&&-6.783&-6.783&-6.783&-6.783&-6.783\\
  &&&&&-6.793&-6.793&-6.793&-6.793&-6.793
\\ \hline
\end{tabular}
\end{center}
\label{tab:tensorss}
\end{table}
We see from this table that the $JT=01$ matrix elements converge
rapidly, irrespective of potential. The differences between the two potentials
is also neglible, reflecting the fact that the $^{1}S_0$ partial wave
does not receive contributions from the tensor force.
The
$JT=10$ matrix elements stabilize first around $n_3 \sim 66-91$. However,
we see that the convergence of the Bonn A potential is slightly faster than
the Bonn C potential.
The difference between (3,10,45) and (3,10,78) is $0.14\%$ for the 
Bonn A potential whereas
for the Bonn C potential we have $0.18\%$. Although these differences
are small, they reflect that potentials with a weak tensor force
experience a larger quenching of $G_F$ 
in the $T=0$ channel. 
\begin{figure}[hbtp]
      \setlength{\unitlength}{1mm}
      \begin{picture}(140,150)
      \put(25,10){\epsfxsize=12cm \epsfbox{fig31.eps}}
      \end{picture}
\caption{The free $G$-matrix ($G_F$) and the $G$-matrix ($G$) for
$\bra{(0s_{1/2})^{2}JT=01}G\ket{(0s_{1/2})^{2}JT=01}$ obtained
with the Bonn A (solid lines) and C
(dashed lines) potentials described in the text as a function
of the starting energy.}
\label{fig:quench1s0}
\end{figure}
We demonstrate  this in figs.\ \ref{fig:quench1s0} and 
\ref{fig:quench3s1} where we plot
both the total $G$-matrix and $G_F$ as functions of the starting
energy for both potentials with a Pauli operator
defined as (3,10,66). Fig. \ref{fig:quench1s0} shows the results
for the $JT=01$ channel, whereas fig.\ \ref{fig:quench3s1} shows the
$JT=10$ matrix elements.

The $JT=01$ matrix elements obtained with the Bonn A and C potentials
show only negligible differences, both for $G_F$ and $G$. This explains 
why the convergence in the above table is rather
similar for the two potentials. Moreover, the difference between 
$G_F$ and $G$ is not so large in the $JT=01$ channel.
For the $JT=10$ matrix elements at small starting energies, the
difference between $G_F$ and $G$ is much more dramatic, as shown in fig.\
\ref{fig:quench3s1}.
\begin{figure}[hbtp]
      \setlength{\unitlength}{1mm}
      \begin{picture}(140,150)
      \put(25,10){\epsfxsize=12cm \epsfbox{fig30.eps}}
      \end{picture}
      \vspace{1cm}
\caption{The free $G$-matrix ($G_F$) and the $G$-matrix ($G$) for
$\bra{(0s_{1/2})^{2}JT=10}G\ket{(0s_{1/2})^{2}JT=10}$ obtained
with the Bonn A (solid lines) and C (dashed lines)
potentials described in the text as a function
of the
starting energy.}
\label{fig:quench3s1}
\end{figure}
The $G_F$-matrix 
for the two potentials differs already at small energies, by 
roughly $2$ MeV. This may explain why potential A
converges faster in table \ref{tab:tensorss}.
The quenching of $G_F$ as a function of the starting
energy for the two potentials is not
so different however, though the quenching of $G_F$ for the Bonn C
potential is slightly larger.
$G_F$ for the Bonn A potential is quenched $63\%$ when
going from a starting energy of $-5$ MeV to $-140$ MeV. The corresponding
number for the Bonn C potential is $67\%$. For
the final $G$-matrix, which now takes into account {\em both the Pauli
operator and the starting energy} through $\Delta G$,
the quenching is  much stronger than that for $G_F$. The $G$-matrix 
of the Bonn A potential
is quenched $17\%$ when going from $-5$ to $-140$ MeV in starting energy,
whereas the corresponding number for the C potential is $26\%$.
It is also worth noting that the effect of the Pauli operator is more
important in the $T=0$ channel at small starting energies.

The quenching due to the Pauli operator and the starting energy, 
explains why a potential
with a weak tensor force results in $G$-matrix
elements which in general are more attractive
compared to a potential with a stronger tensor force.
We will see examples of this in sections  6-8. For the 
applications of the $G$-matrix to calculations of the effective 
interaction in chapter four, we will approximate $n_3=66$. For $sd$-shell
nuclei we then have the boundaries (3,10,66). For nuclei in the mass
region of calcium, the sp states in the $1p0f$ shell are used to determine
the model space appropriate for the effective interaction. The boundaries
$n_1$, $n_2$ and $n_3$ become then (6,15,66), see fig.\ \ref{fig:orbits}
for the number assignements. For nuclei in the mass region of tin, we 
choose the $2s1d0g$ shell (except the $0g_{9/2}$ orbit) as our model
space and the $G$-matrix is calculated with the boundaries (11,21,66).

Finally, instead of a h.o.\ basis we could employ a
representation for the two-particle wave function where both particles
are represented by plane waves. In this case we have
\cite{bon89,kkr79,wc72}
\begin{equation}
\begin{array}{ll}
\ket{(k_{a}l_{a}j_{a})(k_{b}l_{b}j_{b})JT}=&
{\displaystyle \sum_{lL\lambda S{\cal J}}\int k^{2}dk\int K^{2}dK}
F\times\langle ab|\lambda SJ \rangle
\\
&\times (-1)^{\lambda +{\cal J}-L-S}
\hat{\lambda}
\left\{\begin{array}{ccc}L&l&\lambda\\S&J&{\cal J}
\end{array}\right\}
\\
&\times \left\langle klKL| k_{a}l_{a}k_{b}l_{b}\right\rangle
\ket{klKL{\cal J}ST},\end{array}\label{eq:kk}
\end{equation}
where the term $\left\langle klKL| k_{a}l_{a}k_{b}l_{b}\right\rangle$
is the vector bracket defined in refs.\ \cite{kkr79,wc72}. We could
correspondingly define a $G$-matrix in terms of plane waves only, or 
in a mixed representation by noting that
the two-particle wave functions with
$\ket{(n_{a}l_{a}j_{a})(k_{b}l_{b}j_{b})JT}$,
can be obtained from the latter equation through
\begin{equation}
\ket{(n_{a}l_{a}j_{a})(k_{b}l_{b}j_{b})JT}=
\int k_{a}^{2}dk_{a}R_{n_{a}l_{a}}(\alpha k_{a})
\ket{(k_{a}l_{a}j_{a})(k_{b}l_{b}j_{b})JT}.    \label{eq:kho}
\end{equation}


\subsection{Introduction}

In this section and section 8 we will focus on nuclear matter calculations.
Central
to the discussion is the Brueckner-Hartree-Fock (BHF) approach, where
variational methods are used to obtain self-consistent
sp energies and wave functions.

Nuclear matter is an idealized infinite medium with an equal number
of protons and neutrons and the Coulomb interaction between the protons
being artificially removed. These restrictions imply that the phenomenological
Weizs\"{a}cker mass formula, extrapolated to $A\rightarrow \infty$, gives
a binding energy per particle ${\cal E}/A=16\pm 1$ MeV. A main purpose 
of nuclear matter theories is to derive this binding energy
${\cal E}/A$ from first principle,
i.e.\ from the underlying NN interaction. Following this microscopic
line of approach, one can in turn determine the equilibrium nuclear matter
density $\rho_0$ and the incompressibility coefficient $\kappa$.
From electron-nucleus scattering experiments one has estimated 
$\rho_0 \sim 0.17 $ fm$^{-3}$, and from the analysis of nuclear
breathing modes $\kappa$ is of the order $\sim 200$ MeV.
Finally, calculations are simplified in 
nuclear matter, since the solutions to either the Dirac or the 
sp Schr\"{o}dinger equations are given in terms of plane waves.

Until recently, most microscopic calculations of the energy per particle
for nuclear or neutron matter have been carried out within a
non-relativistic framework \cite{mah85,km83,wff88},
where the non-relativistic Schr\"{o}dinger equation is used to describe the
sp motion in the nuclear medium. Various degrees of
sophistication are accounted for in the literature \cite{mac89},
ranging from first order calculations in the reaction matrix $G$ to the
inclusion of two- and three-body higher order
effects \cite{dm92,mah85,km83,wff88}.
A common problem to non-relativistic nuclear matter calculations, is
however the
simultaneous reproduction of both the binding energy per nucleon
(${\cal E}/A=16\pm 1 $MeV) and the saturation density, with fermi momentum
$k_F=1.35\pm 0.05 fm^{-1}$.
Results obtained with a variety of methods and nucleon-nucleon (NN)
interactions, are located along a band,
denoted the ``Coester band'', which
does not meet the empirical data for nuclear matter.
Albeit these deficiencies, much progress has been achieved recently
in the description of the saturation properties of nuclear matter.
Of special relevance is the replacement of the non-relativistic
Schr\"{o}dinger equation with the Dirac equation to describe the
sp motion, referred to as the Dirac-Brueckner (DB)
approach.
This is motivated by the success of the
phenomenological Dirac approach in nucleon-nucleus scattering
\cite{ray91} and in the description of properties of finite nuclei
\cite{hnm92}, such as e.g. the spin-orbit splitting
in finite nuclei \cite{br78}. Moreover, rather promising results
within the framework of the DB approach, have been obtained by Machleidt,
Brockmann and M\"{u}ther \cite{bm90,mmb90,lmb92}, employing the
OBE models of the Bonn group. Actually, the empirical properties
of nuclear matter were quantitatively reproduced by Brockmann
and Machleidt \cite{bm90}. However, although interesting results
have been obtained within the DB scheme, these findings need further
refinements through e.g.\ the introduction of more
complicated many-body corrections.


\subsection{Non-relativistic approaches}

In section 5.3 we discussed the $G$-matrix for nuclear matter,
using the Brueckner-Hartree-Fock (BHF) approch. Here we will extend 
those calculations by considering the so-called model-space
Breuckner-Hartree-Fock (MBHF) method. This method follows much of
the philosophy behind the effective model-space calculations
discussed in section 6.

In the MBHF method, the cutoff $k_M$ is given
as a multiple
of the Fermi momentum. A frequently used value \cite{km83}
is $k_M= 2k_F$,
a choice also used here. This means that we extend
the BHF spectrum
beyond $k_F$. The resulting Pauli operator $Q_{\mathrm{MBHF}}$
(in the laboratory system)
for the model-space BHF method is
\begin{equation}
   Q_{\mathrm{MBHF}}(k_m , k_n ) =
     \left\{\begin{array}{cccc}1,
     &min(k_m ,k_n ) > k_F&\mathrm{and}&max(k_m ,k_n ) > k_M \\
     0,&&\mathrm{else}.&\end{array}\right.
    \label{eq:qmbhf}
\end{equation}
If we set $k_M = k_F$ , we obtain the traditional BHF choice. Note that
at least one particle must be outside the model space, which means
that we must have at least $k_m$ or $k_n$ to be greater than $k_M$.
In actual calculations, it will again be convenient to express the
MBHF Pauli operator in the coordinates of the relative and
center-of-mass system. We replace the operator
$Q_{\mathrm{MBHF}}$ with its
angle average $\overline{Q}_{\mathrm{MBHF}}$ given by
\begin{equation}
     \overline{Q}_{\mathrm{MBHF}}(k,K)=\left\{\begin{array}{cr}
         1&\mathrm{in} \hspace{3mm}
         \mathrm{regions}\hspace{3mm} \mathrm{a, b}\\
         0&\mathrm{c}\\
	\left(K^2/4+k^2 -k_{F}^2\right)/kK&\mathrm{d}\\
	\left((K/2+k)^2 -k_{M}^2\right)/kK&\mathrm{e}\\
	\left(K^2/2+2k^2 -k_F^2-k_M^2\right)/kK&\mathrm{f},
         \end{array}\right.
    \label{eq:mbhf}
\end{equation}
with the various regions and values of $k$ and $K$ defined in
fig.\ \ref{fig:mbhfpop}. We have assumed that all relative directions
between ${\bf k}$ and ${\bf K}$ are equally likely and thereby we
can average
over these directions.
\begin{figure}[hbtp]
      \setlength{\unitlength}{1mm}
      \begin{picture}(140,90)
      \put(25,10){\epsfxsize=12cm \epsfbox{mbhfpauli.eps}}
      \end{picture}
\caption{In (a) we show the Pauli operator for the MBHF choice
in the lab frame, while (b) is the corresponding diagrammatic
representation in the relative and center-of-mass system. We have defined
$k_a$ by $k_a^2=\frac{1}{2}(k_F^2+k_M^2)$.
Note that ${\bf K}={\bf k}_n+{\bf k}_m$ and ${\bf k}=
\frac{1}{2}({\bf k}_n-{\bf k}_m)$.}
\label{fig:mbhfpop}
\end{figure}
To better understand the analytical expressions for
$\overline{Q}_{\mathrm{MBHF}}$ in regions
$d$, $e$ and $f$ it is instructive to consider $\overline{Q}_{\mathrm{MBHF}}$
as composed of one part which has one particle with momentum between
$k_F$ and $k_M$ and the other particle with momentum above $k_M$, and
another piece which has two particles above $k_M$. The latter corresponds
to region $d$, and here we can use the angle-average $(k^2+K^2/4-k_F^2)/kK$
from the BHF Pauli operator. In region $e$ we have that the standard BHF
has $\overline{Q}_{BHF}=1$, but we must take the angle-average for momenta below
$k_M$, i.e.\
\[
     1+\frac{k^2+K^2/4-k_M^2}{kK}=\frac{(k+K/2)^2-k_M^2}{kK}.
\]
In region $f$ we need to satisfy
\begin{eqnarray}
     k^2+K^2/4+2kKcos\theta_1 &\geq k_M^2 \nonumber\\
     k^2+K^2/4+2kKcos\theta_1 &\geq k_F^2 \nonumber\\
     k^2+K^2/4-2kKcos\theta_2 &\geq k_M^2 \nonumber\\
     k^2+K^2/4-2kKcos\theta_2 &\geq k_F^2 ,
\end{eqnarray}
and taking the angle average results in the expression for region $f$.
Thus, the MBHF $G$-matrix takes into account intermediate energy states
where we may have a state with one particle outside the model space or
two particles outside the model space.
The MBHF Pauli operator and the energy denominators can then be
expressed as
\begin{equation}
     \frac{\overline{Q}_{\mathrm{MBHF}}}{\omega -H_0}=
     \frac{Q_{1p}}{\omega - E_{1p}}
     +\frac{Q_{2p}}{\omega - E_{2p}},
\end{equation}
where the labels $1p$ and $2p$ refer to one or two particles outside the
model space, respectively. 
To define the energy denominators we will also make use of the
angle-average approximation.
The angle dependence is again handled by the
so-called effective mass approximation.
The self-consistency scheme consists in
choosing adequate initial values of the
effective mass and $\Delta$. The obtained $G$-matrix is in turn used to
determine  new values for $M_{N}^{*}$ and $\Delta$. This procedure
continues until these parameters vary little.
The energy variables $E_{1p}$ and $E_{2p}$ are then determined by
\begin{equation}
    E_{1p}= \frac{k^2}{M_N}+\frac{K^2}{4M_N}+\frac{1}{4M_N}
            \left(\frac{M_N}{M_N^*}
            -1\right) (k_F^2+k_M^2) +\Delta
\end{equation}
and 
\begin{equation}
     E_{2p}=\frac{k^2}{M_N}+\frac{K^2}{4M_N}.
\end{equation}
Similarly, for two particles below $k_M$ we will have
\begin{equation}
    E_{0p}= \frac{k^2}{M_N^*}+\frac{K^2}{4M_N^*}+2\Delta,
\end{equation}
and this equation defines the starting energy $\omega$ in the
MBHF scheme.

The $G$-matrix within the MBHF scheme can then be written as
\begin{equation}
  G_{ijkl}^M(\omega)=V_{ijkl}+\sum_{mn}V_{ijmn}
  \frac{\overline{Q}_{\mathrm{MBHF}}(mn)}{\omega -\varepsilon_m-\varepsilon_n}
  G_{mnkl}^M(\omega),
  \label{eq:gmbhf}
\end{equation}
From $G^M$ we obtain the antisymmetrized model-space BHF matrix
$\tilde{G}^M$, defined as
\begin{equation}
    \tilde{G}_{ijkl}^M =\frac{1}{2}\left(G_{ijkl}^M-G_{ijlk}^M\right),
\end{equation}
which we will employ in our evaluation
of the energy shift. For further notations,
recall the discussion in section 2, eqs.\ (11) to (15).
The structure of eq.\ (\ref{eq:gmbhf}) is shown in fig.\
\ref{fig:structgmbhf}.
\begin{figure}[hbtp]
      \setlength{\unitlength}{1mm}
      \begin{picture}(140,80)
      \put(25,10){\epsfxsize=12cm \epsfbox{gmmat.eps}}
      \end{picture}
\caption{The structure of $G^M$.}
\label{fig:structgmbhf}
\end{figure}
The sp states are divided in two groups in $G^M$, those with momenta
$k>k_M$ and those with momenta $k\leq k_M$. The particles belonging
to the former are depicted with railed lines in the above figure. The
intermediate states of $G^M$, must have both particles
with $k>k_F$ and at least one of them must be outside the
model space.
If we relate the model-space BHF
approach to the conventional BHF sp spectrum
and the continuous sp spectrum, one may say that the model-space BHF is
an intermediate
scheme,
in the sense that we introduce an extended Pauli operator in eq.\
(\ref{eq:mbhf}), such that we have a continuous sp spectrum for $k < k_M$,
while for $k>k_M$ the spectrum is that of a free particle.
Moreover, the
model-space BHF definition of the Pauli operator,
gives a $G$-matrix which does
not account for scattering into intermediate states if both particles
have momenta between $k_F$ and $k_M$. This is however a welcome feature
of the model-space BHF
method as it allows one to consider collective excitations
like the summation of particle-particle hole-hole (pp-hh) terms
without double counting problems. These terms will be discussed
below in connection with the higher-order corrections.
Finally, the non-relativistic
energy per particle ${\cal E}/A$ is formally given as
\begin{equation}
   {\cal E}/A =
   \frac{1}{A}\sum_{h\leq k_F}
   \frac{k_h^2}{2M_N}+
   \frac{1}{2A}\sum_{hh'\leq k_F}
   \bra{hh'}G(\omega=\varepsilon_h +\varepsilon_{h'})\ket{hh'}_{AS},
   \label{eq:enrel}
\end{equation}
where the $G$-matrix may be defined with the BHF or the MBHF
Pauli operators.



We present the results of non-relativistic calculations in
fig.\ \ref{fig:chap3beanr} obtained with the Bonn potentials
of table A.1 of ref.\ \cite{mac89}.
There we show the saturation curves employing both the BHF
and the MBHF choices for the sp spectrum. As can be observed from
this figure, both the BHF (full line) and the MBHF (dotted line)
choices for
all potentials miss the empirical area, a feature
common to all non-relativistic calculations of nuclear
matter.  Moreover, the potential
with the weakest tensor force, Bonn A, gives the largest
binding energy per
nucleon.
The latter can be understood from the
qualitative arguments already presented in section 5.3.
Since all potentials
give essentially the same scattering matrix, they are quenched differently
in a medium. At small densities
where both the Pauli operator and the starting energy are small,
the potentials
yield similar results, irrespective of whether we use the BHF
or the MBHF choice
for the sp spectrum. However, as the density increases, the quenching
of
\begin{equation}
     V_T Q\frac{1}{\omega - QTQ}QV_T
\end{equation}
becomes significant
for potentials with a
strong tensor force. Fig.\ \ref{fig:chap3beanr}
illustrates this point clearly.
\begin{figure}[hbtp]
      \setlength{\unitlength}{1mm}
      \begin{picture}(140,150)
      \put(25,10){\epsfxsize=12cm \epsfbox{mbhf.eps}}
      \end{picture}
\caption{The nuclear matter saturation curves as functions of
the Fermi momentum $k_F$ for the Bonn A, B and C potentials
discussed in the text. The full lines are the results obtained with
the conventional non-relativistic BHF method,
while the dashed line represents
the results with the MBHF method. The rectangular box describes the
empirical data.}
\label{fig:chap3beanr}
\end{figure}
In connection with the non-relativistic calculations,
it is worth noting
that the saturation density is left almost unchanged
irrespective
of sp spectrum approach, i.e.~with the same potential employing
the BHF and MBHF methods. The binding energy at
saturation per nucleon changes
however with approximately
$-3$ MeV for each potential when we replace  the BHF
choice with the MBHF choice.
This is solely due to the introduction of additional
short-range effects through
the enlargement of the model space. The increased importance of
short-range
correlations in the MBHF method is also reflected in the change of the
$^{3}S_1$ $-$  $^{3}D_1 $ partial wave contribution.
With $k_F = 1.5$ fm$^{-1}$, this
partial wave contribution to the potential energy per particle
for the Bonn A potential changes from $-20.22$ to $-21.57$
MeV for the BHF
and the MBHF choices, respectively. For the remaining
partial waves
the changes are less significant or negligible. A thorougher discussion
of different partial waves will be given below where we also discuss
the contributions from ring diagrams as presented by
Song {\em et al.\ } \cite{shk87}.
The interesting feature of this summation of ring diagrams
is that it takes hole-hole propagations in intermediate states
into account. These are long-range correlations which are
ignored in the standard BHF theory. Hole-hole propagation
was found to provide an effect which is repulsive and
strongly density dependent \cite{rpd89}. This effect serves
to improve the predictions from the conventional BHF results
discussed above. Short-range correlations through the summation
of particle-particle ladders are also taken into account, and
introduce more binding. Thus, the essential effect of these
particle-particle and hole-hole (pp-hh) correlations is
to increase the binding energy and decrease the saturation
density. Note that hitherto we have restricted the attention
to non-relativistic results.




\subsection{Higher-order terms}

Here we will study the contribution from higher order
terms represented by the all-order summation of the
so-called particle-particle
and hole-hole (pp-hh) ring-diagrams. In so-doing we will make
contact with the the theory for Green's functions. Moreover, we will
extend the formalism discussed in section 2, where only we sum over
intermediate particle states in the calculation of the energy shift.
A similar approach was presented by Ramos {\em et al.} \cite{rpd89},
though these authors start with the Green's function approach, as
discussed in e.g.\ Fetter and Walecka \cite{fw71}.

Before we present the final equation for the energy shift, it is instructive
to reconsider the steps of section 2 which led to the definition of
the $G$-matrix. In fig.\ \ref{fig:goldstone} we showed
contributions to the energy shift $\Delta E$ to third order in the
interaction.
The contributions to the energy shift from the three diagrams
of fig.\ \ref{fig:goldstone} can be rewritten as
\begin{equation}
   (i)=\lim_{\eta \rightarrow 0}
       -\frac{1}{2\pi i}\int_{-\infty}^{\infty}
       d\omega e^{i\omega\eta}F_{ij}(\omega)\tilde{V}_{ijij},
\end{equation}
where we defined in section 2
\begin{equation}
     \tilde{V}_{ijij}=\frac{1}{2}
     (V_{ijij}-V_{ijij}).
\end{equation}
Similarly, diagrams (ii) and (iii) read
\begin{equation}
   (ii)=\lim_{\eta \rightarrow 0}
       -\frac{1}{2\pi i}\int_{-\infty}^{\infty}
       d\omega e^{i\omega\eta}\frac{1}{2}F_{ij}(\omega)
       \tilde{V}_{ijmn}F_{mn}(\omega)\tilde{V}_{mnij},
\end{equation}
and
\begin{equation}
   (iii)=\lim_{\eta \rightarrow 0}
       -\frac{1}{2\pi i}\int_{-\infty}^{\infty}
       d\omega e^{i\omega\eta}\frac{1}{3}F_{ij}(\omega)
      \tilde{V}_{ijmn}F_{mn}(\omega)
     \tilde{V}_{mnpq}F_{pq}(\omega )\tilde{V}_{pqij},
   \label{eq:thirdeo}
\end{equation}
where repeated indices are summed over all sp states and $F_{ij}$
is the unperturbed pair propagator
\begin{equation}
    F_{ij}(\omega)=\frac{\overline{n}_i\overline{n}_j}
    {\omega -\varepsilon_i -\varepsilon_j +i\eta}
    -\frac{n_in_j}
    {\omega -\varepsilon_i -\varepsilon_j -i\eta}.
\end{equation}
Here we set $n_i=1$ if $k_i\leq k_F$ and $0$ if $k_i > k_F$, and
$\overline{n}_i=1-n_i$. The diagrams of
fig.\ \ref{fig:goldstone}, i.e.\ diagrams (i), (ii) and (iii),
can then be summed to all orders in much a similar way as we did in the
construction of the $G$-matrix in section 2, yielding
\begin{equation}
   \Delta E^{pp-hh}=\lim_{\eta \rightarrow 0}
       -\frac{1}{2\pi i}\int_{-\infty}^{\infty}
       d\omega e^{i\omega\eta}Tr\left( F\tilde{V}+
        \frac{1}{2}(F\tilde{V})^2+\frac{1}{3}(F\tilde{V})^3
        +\dots\right).
\end{equation}
For computational purposes it is however convenient to introduce
the strength parameter $z$ discussed in section 4.2, so that we
can rewrite the preceeding equation as
\begin{equation}
   \Delta E^{pp-hh}=\lim_{\eta \rightarrow 0}
       -\frac{1}{2\pi i}\int_0^1\frac{dz}{z}\int_{-\infty}^{\infty}
       d\omega e^{i\omega\eta}Tr\left( zF\tilde{V}+
        \frac{1}{2}(zF\tilde{V})^2+\frac{1}{3}(zF\tilde{V})^3
        +\dots\right).
        \label{eq:ppdeltae}
\end{equation}
The strength parameter $z$ makes the trace a simple geometric
series, and $z$ tells us how the many-body system evolves from the
non-interacting one with $z=0$ to the fully interacting one with $z=1$.
At this stage, let us introduce the $z$-dependent particle-particle
Green's function $G_{ijkl}^{pp}(\omega ,z)$ defined as
\begin{equation}
    G_{ijkl}^{pp}(\omega ,z)=F_{ij}(\omega )\delta_{ij,kl}
    +zF_{ij}(\omega ) \tilde{V}_{ijmn}G_{mnkl}^{pp}(\omega ,z).
\end{equation}
The first term on the rhs.\ is simply the free propagation of a pair.
Eq.\ (\ref{eq:ppdeltae}) can be rewritten as
\begin{equation}
   \Delta E^{pp-hh}=\lim_{\eta \rightarrow 0}
       -\frac{1}{2\pi i}\int_0^1\frac{dz}{2\pi i}\int_{-\infty}^{\infty}
       d\omega e^{i\omega\eta}Tr\left(
       z\tilde{V}G^{pp}(\omega ,z)
       \right).
        \label{eq:ppdeltae1}
\end{equation}
Alternatively, we could define a generalized $G$-matrix $K$ as
\begin{equation}
    K_{ijkl}^{pp}(\omega ,z)=z\tilde{V}_{ijkl}
    +zF_{ij}(\omega ) \tilde{V}_{ijmn}K_{mnkl}^{pp}(\omega ,z),
    \label{eq:generalG}
\end{equation}
and eq.\ (\ref{eq:ppdeltae}) reads
\begin{equation}
   \Delta E^{pp-hh}=\lim_{\eta \rightarrow 0}
       -\frac{1}{2\pi i}\int_0^1\frac{dz}{z}\int_{-\infty}^{\infty}
       d\omega e^{i\omega\eta}Tr\left( F(\omega )K^{pp}(\omega ,z)
        \right).
        \label{eq:ppdeltae2}
\end{equation}
The reader should notice the distinction between the generalized
reaction matrix $K$ and the $G$-matrix discussed in section 2.
In section 2, we restricted the attention to repetead interactions
between pairs of particle lines. Such interactions form the
starting point for nuclear matter calculations within the
conventional Brueckner-Hartree-Fock (BHF) scheme discussed in section
5.3. The generalized reaction matrix $K$ of eq.\ (\ref{eq:generalG})
depends however on the pair propagator $F$. This means that eq.\
(\ref{eq:ppdeltae2}), or equally well eq.\ (\ref{eq:ppdeltae1}),
also include the contributions from repeated interactions
between two hole lines, i.e.\ hole-hole correlations.
Thus, in addition to including all particle-particle
interactions through the definition of the $G$-matrix (standard BHF),
we have hole-hole correlations and ground-state correlations
connecting two particle lines to two hole lines to all
orders.  These differences are illustrated in fig.\ \ref{fig:pphh-bhf}.
\begin{figure}
      \setlength{\unitlength}{1mm}
      \begin{picture}(140,100)
      \put(25,10){\epsfxsize=12cm \epsfbox{ring1.eps}}
      \end{picture}
      \caption{Diagram (i) is a general ring diagram, where the indices
      $i$ and $j$ can be either particles or holes. A typical diagram
      which enters the usual BHF nuclear matter calculation is
      shown in (ii). All particles are determined by $k>k_F$.}
      \label{fig:pphh-bhf}
\end{figure}

We wish  now to make contact with the model-space Brueckner-Hartree-Fock
approach discussed above, in order to study one possible approximation
to the generalized reaction matrix $K$. We rewrite the equation
for the model-space BHF $G$-matrix $G^M$ as
\begin{equation}
  G_{ijkl}^M(\omega)=V_{ijkl}+\sum_{mn}V_{ijmn}
  \frac{\overline{Q}_{\mathrm{MBHF}}(mn)}
  {\omega -\varepsilon_m-\varepsilon_n+i\eta}
  G_{mnkl}^M(\omega),
\end{equation}
with the Pauli operator already defined in eq.\ (\ref{eq:qmbhf}). For
further details, see the discussions in section 7.2.
The factor $i\eta$
ensures that it is only the retarded part of the pair propagator
$F$ (i.e.\ only two lines propagating upward) that can take
part in the intermediate states of $G^M$, or stated
differently, the poles of $G^M$ are only in the lower half of the
complex $\omega$ plane. This is important since
we do not want to double count contributions from intermediate
particle states already included in the definition of $G^M$ in the
sum of particle-particle and hole-hole ladder diagrams.
To better understand this, consider the ring diagrams of fig.\
\ref{fig:ringdiagrams}, where we now use the antisymmetrized
$G^M$-matrix $\tilde{G}^M$.
\begin{figure}
      \setlength{\unitlength}{1mm}
      \begin{picture}(140,100)
      \put(25,10){\epsfxsize=12cm \epsfbox{ring2.eps}}
      \end{picture}
      \caption{Examples of ring diagrams with $\tilde{G}^M$
      interactions discussed in the text.}
      \label{fig:ringdiagrams}
\end{figure}
Diagram (i) is a pp-hh ring diagram with propagators
for the pairs $(ij)$, $(kl)$ and $(mn)$ all defined within the model
space $P$, i.e.\ they are composed of both
retarded and advanced parts.
For diagram (ii), the propagators for the same pairs
are also all within the model space. However, the propagators for
the pairs $(uv)$, $(pq)$ and $(rs)$ are only retarded ones, whereas
the pairs $(ij)$, $(kl)$ and $(mn)$ have propagators with both
advanced and retarded parts in $F$. Also, at least one of the
intermediate states in the pairs $(uv)$, $(pq)$ and $(rs)$ must
be outside the model space in order to avoid the double
counting problem when we substitute the interaction $\tilde{V}$
with $\tilde{G}^M$. The authors of ref.\ \cite{shk87} show 
then  that e.g.\ eq.\ (\ref{eq:thirdeo}) can 
be rewritten as
\begin{equation}
   (iii)=\lim_{\eta \rightarrow 0}
       -\frac{1}{2\pi i}\int_{-\infty}^{\infty}
       d\omega e^{i\omega\eta}\sum_{ijmnpq}^{P}\frac{1}{3}
       F_{ij}(\omega)
      \tilde{G}_{ijmn}^MF_{mn}(\omega)
      \tilde{G}_{mnpq}^MF_{pq}(\omega )
      \tilde{G}_{pqij}^M,
\end{equation}
where the sum is now restricted to sp states within the model space
only. With the replacement
\[
   \tilde{V}\rightarrow \tilde{G}^M,
\]
we can now repeat the steps leading to e.g.\ eq.\ (\ref{eq:ppdeltae1})
and obtain
\begin{equation}
   \Delta E^{pp-hh}=\lim_{\eta \rightarrow 0}
       -\frac{1}{2\pi i}\int_0^1\frac{dz}{z}\int_{-\infty}^{\infty}
       d\omega e^{i\omega\eta}Tr_P\left(
       z\tilde{G}^M(\omega )G^{pp}(\omega ,z)
       \right),
\end{equation}
where the trace is taken over model space states only. Alternatively,
we could also rewrite eq.\ (\ref{eq:ppdeltae1}) in terms of $\tilde{G}^M$.
However,
the authors of ref.\ \cite{shk87} found it more convenient from a
computational point of view to evaluate eq.\ (\ref{eq:ppdeltae1}) rather
than eq.\ (\ref{eq:ppdeltae2}). A useful prescription is then
to write the Green's function $G_{ijkl}^{pp}(\omega ,z)$ in its
Lehmann  representation
\begin{equation}
G_{ijkl}^{pp}(\omega ,z)=\sum_n\frac{X_n(ij,z)X_n^*(kl,z)}
                         {\omega -\omega_n^+(z)+i\eta}
                        -\sum_m\frac{Y_m(ij,z)Y_m^*(kl,z)}
                         {\omega -\omega_m^-(z)-i\eta},
\end{equation}
where we have defined
\begin{equation}
     \omega_n^+(z)=E_n^{A+2}(z)-E_0^A(z),
\end{equation}
\begin{equation}
     \omega_m^-(z)=E_0^A(z)-E_m^{A-2}(z),
\end{equation}
\begin{equation}
     X_n(ij,z)=\bra{\Psi_0^A(z)}a_ja_i\ket{\Psi_n^{A+2}(z)},
\end{equation}
and
\begin{equation}
     Y_m(ij,z)=\bra{\Psi_m^{A-2}(z)}a_ja_i\ket{\Psi_0^{A}(z)}.
\end{equation}
We have defined the eigenfunctions and eigenvalues through
\[
     (H_0+zH_1)\Psi_0^A(z)=E_0^A(z)\Psi_0^A(z),
\]
\[
     (H_0+zH_1)\Psi_n^{A+2}(z)=E_n^{A+2}(z)\Psi_n^{A+2}(z),
\]
and similarly for $\Psi_m^{A-2}$ and $E_m^{A-2}$. Substituting
the above equations into eq.\ (\ref{eq:ppdeltae1}) yields
\begin{equation}
   \Delta E^{pp-hh}=
       \int_0^1\frac{dz}{z}
        \sum_{m(A-2)}\sum_{ijkl}Y_m(ij,z)Y_m^*(kl,z)z\tilde{V}_{ijkl}.
        \label{eq:ppdeltae3}
\end{equation}
The
convergence factor $e^{i\omega\eta}$ (connected with the
limit $\eta\rightarrow 0$) in eq.\
(\ref{eq:ppdeltae1}) enables us to
convert the integral over $\omega$ in eq.\ (\ref{eq:ppdeltae1})
into a contour integral closed in the upper $\omega$ plane.
Using $\tilde{G}^M$ we obtain the final result of ref.\
\cite{shk87}
\begin{equation}
       \Delta E^{pp-hh}=
       \int_0^1\frac{dz}{z}
       \sum_{m(A-2)}\sum_{ijkl}^{P}Y_m(ij,z)
       Y_m^*(kl,z)\tilde{G}_{ijkl}^M(\omega_m^-(z))z,
       \label{eq:ppdeltae4}
\end{equation}
where the sum is restricted to model space states only.
The amplitudes $Y_m$ are determined through the RPA-type
equation of
\begin{equation}
   \sum_{ef}\left\{(\varepsilon_i+\varepsilon_j)\delta_{ij,ef}
   +(\overline{n}_i\overline{n}_j-n_in_j)z\tilde{G}_{ijef}^M\right\}
   Y_m(ef,z)=\omega_m^-(z)Y_m(ij,z).
\end{equation}
Note that the indices $ijklef$ run over all
possible sp states within the model space. At the present
stage only two-body terms are included in the above expression
through the approximation $\tilde{G}^M$. As discussed by the authors
of ref.\ \cite{shk87}, one-body terms are also important, and
can be taken into account when sp self-energy insertions are included.
The normalization condition for the $Y$-amplitudes is \cite{shk87}
\begin{equation}
  \sum_{p_1>p_2>k_F}^{P}\left| Y_m(p_1p_2,z)\right|^2
 -\sum_{h_1<h_2<k_F}^{P}\left| Y_m(h_1h_2,z)\right|^2
  =-\left(1-\frac{\partial \omega_m^-(z)}{\partial\omega}\right)^{-1},
\end{equation}
where $p_1,p_2,h_1,h_2$ are sp states within the chosen model space
and the rhs.\ of this equation  expresses the model-space
overlap. Computational details for solving
the above set of equations can be found in ref.\ \cite{shk87}.

How does this ring-diagram approach compare with the usual
BHF method in section 5.3 and the MBHF method
of section 7.2
for nuclear matter? To understand
these differences, consider the three diagrams of fig.\
\ref{fig:diff-mbhfring}.
\begin{figure}
      \setlength{\unitlength}{1mm}
      \begin{picture}(140,100)
      \put(25,10){\epsfxsize=12cm \epsfbox{ring3.eps}}
      \end{picture}
      \caption{Diagram (a) is included in the conventional
      BHF procedure, while (b) is included in the MBHF method.
      Diagram (c) represents a ring-diagram. The label B is the
      BHF reaction matrix, while label M refers to the model-space
      BHF reaction matrix. }
      \label{fig:diff-mbhfring}
\end{figure}
Diagram (a) is the typical two-hole line diagram included in the
conventional BHF scheme. The label B represents the BHF $G$-matrix
using a discontinuous sp spectrum at $k_F$. Diagram (b) is
a diagram included in the MBHF scheme, and the label M refers to
the MBHF $G$-matrix with the particle labels $p_1$, $p_2$, $p_3$ and $p_4$
with momenta between $k_F$ and $k_M$. The main difference between the BHF 
and the MBHF schemes,
as discussed in section 7.2, is the use of different sp spectra
between $k_F$ and $k_M$. For the BHF approach, a free particle
sp spectrum is used for momenta larger than $k_F$, whereas for the
MBHF scheme, a free particle sp spectrum is employed for momenta
larger than $k_M$. In the ring-diagram approach, we sum
diagrams of type (c) to all orders, but in this case, the
intermediate states from $i$ to $s$ are all within the model
space, and they can be both particle pairs and hole pairs.
Thus, we sum a larger class of diagrams in this approach
than in either the BHF or the MBHF approaches.

In summary, the calculational steps in the ring-diagram method
proceed as follows: first we evaluate the MBHF $G$-matrix
and obtain a sp spectrum
following the steps in section 7.2. Then we solve the RPA-type of 
equations
to obtain the $Y$-amplitudes. These are in turn used to evaluate
the potential energy for nuclear matter by the integral
of eq.\ (\ref{eq:ppdeltae4}) involving the product
$YY^{\dagger}G^M$.
We conclude then this section by showing the results from ring-diagram
calculations using the Bonn potentials discussed in
sections 3, 5, 6 and  7.2. These results are compared
with those from the conventional BHF and the MBHF approaches of section
7.2.

In fig.\ \ref{fig:benrring} we show the non-relativistic
ring-diagram results, compared with those from the
BHF calculation of the previous section. As clearly can be seen from
this figure,
\begin{figure}[hbtp]
      \setlength{\unitlength}{1mm}
      \begin{picture}(140,150)
      \put(25,10){\epsfxsize=12cm \epsfbox{ring.eps}}
      \end{picture}
\caption{The nuclear matter saturation curves as functions of
the Fermi momentum $k_F$ for the Bonn A, B and C potentials
discussed in the text. The dashed lines are the results obtained with
ring-diagram scheme,
while the full lines represents
the results with the BHF method. The rectangular box describes the
empirical data. The ring-diagram results are taken from ref.\
[123].}
\label{fig:benrring}
\end{figure}
we see a rather substantial difference between the BHF
scheme and the ring-diagram scheme. More attraction is provided
by the ring-diagram approach, in addition to an improved
saturation density. Moreover, the differences in saturation
energies and density for the various potentials is smaller
within the framework of the ring-diagram approach. This means
that the uncertainty due to the choice of potential
model is reduced in nuclear matter calculation. Typically, for
the ring-diagram calculations, the difference between the binding
energies and the saturation densities for the Bonn A and Bonn C
potentials are $2.4$ MeV and $0.11$ fm$^{-1}$, respectively. For
the BHF calculation the corresponding differences are
$4$ MeV and $0.2$ fm$^{-1}$, respectively.

It is also instructive to consider the contributions from the
different partial waves. This is shown in table \ref{tab:rings1}.
\begin{table}[hbtp]
\caption{Partial wave contribution to the ring-diagram
energy shift $\Delta E_0^{pp-hh}/A$ at $k_F=1.4$ fm$^{-1}$
for the three Bonn potentials. All entries in MeV. Taken from
ref.\ [123].}
\begin{center}
\begin{tabular}{lrrr}
\\\hline
&
\multicolumn{1}{c}{Bonn A}&
\multicolumn{1}{c}{Bonn B}&
\multicolumn{1}{c}{Bonn C}
\\ \hline
$^{1}S_0$&-17.65&-17.61&-17.57\\
$^{3}S_1$-$^{3}D_1$&-22.21&-21.31&-20.58\\
$^{1}P_1$&5.10&5.17&4.87\\
$^{3}P_0$&-3.95&-3.85&-3.75\\
$^{3}P_1$&11.30&11.19&10.76\\
$^{3}P_2$-$^{3}F_2$&-9.19&-9.24&-9.13\\
$^{1}D_2$&-2.78&-2.74&-2.67\\
$^{3}D_2$&-4.67&-4.57&-4.52\\
$^{3}D_3$-$^{3}G_3$&0.49&0.49&0.49\\
$^{1}F_3$&1.00&0.98&0.96\\
$^{3}F_3$&1.86&1.81&1.77\\
$^{3}F_4$-$^{3}H_4$&-0.49&-0.49&-0.47\\
$^{1}G_4$&-0.51&-0.50&-0.49\\
$^{3}G_4$&-0.87&-0.85&-0.83\\
\hline
\end{tabular}
\end{center}
\label{tab:rings1}
\end{table}
Clearly, the largest difference arises  for the 
$^3S_1$-$^3D_1$ contribution.
\begin{table}[hbtp]
\caption{The $z$ dependence of the potential energy
contribution
$\Delta E_0^{pp-hh}/A$ in the $^{3}S_1$-$^{3}D_1$ channel
for the Bonn A potential. All energies in MeV. Taken from
ref.\ [123].}
\begin{center}
\begin{tabular}{lrrr}
\\\hline
\multicolumn{1}{c}{$k_F$ fm$^{-1}$}&
\multicolumn{1}{c}{$z=0.113$}&
\multicolumn{1}{c}{$z=0.500$}&
\multicolumn{1}{c}{$z=0.887$}
\\ \hline
1.2&-12.66&-18.37&-28.19\\
1.3&-14.74&-20.21&-27.19\\
1.4&-16.83&-21.96&-27.98\\
1.5&-18.91&-23.57&-28.78\\
1.6&-21.06&-25.12&-29.54\\
1.7&-22.99&-26.37&-29.98\\
\hline
\end{tabular}
\end{center}
\label{tab:rings2}
\end{table}
As mentioned previously, the ring-diagram effect is strongest
in this channel. Therefore, it may be of some interest to to look into
the relevance of the tensor force for the ring-diagram effect. We rewrite
the energy-shift as
\begin{equation}
  \Delta E_0^{pp-hh}=\int_0^1dz\Delta E_0^{pp-hh}(z).
\end{equation}
The dependence of $\Delta E_0^{pp-hh}(z)$ on $z$ is a direct measure
for the importance of the higher-order ring diagrams. In tables
\ref{tab:rings2} and \ref{tab:rings3}, we plot the dependence on $z$ for the
Bonn A and the Bonn C potentials, respectively. We see that the difference
between $\Delta E_0^{pp-hh}(z=0.887)$ and $\Delta E_0^{pp-hh}(z=0.113)$ is 
of the order $-10$ MeV, which implies that the higher-order ring diagrams
are important and not negligible. We also see that the difference is larger
with the Bonn C potential, thus the larger the tensor force, the larger the
effect of higher-order ring diagrams. These tables also tell us that the
difference decreases with density. Note that the energy in the 
$^3S_1$-$^3D_1$ channel is attractive; therefore such a decrease implies
an increase of the repulsive effect with density. This effect has the
important consequence that the saturation density will be lowered.
\begin{table}[hbtp]
\caption{The $z$ dependence of the potential energy
contribution
$\Delta E_0^{pp-hh}/A$ in the $^{3}S_1$-$^{3}D_1$ channel
for the Bonn C potential. All energies in MeV. Taken from
ref.\ [123].}
\begin{center}
\begin{tabular}{lrrr}
\\\hline
\multicolumn{1}{c}{$k_F$ fm$^{-1}$}&
\multicolumn{1}{c}{$z=0.113$}&
\multicolumn{1}{c}{$z=0.500$}&
\multicolumn{1}{c}{$z=0.887$}
\\ \hline
1.2&-10.29&-17.72&-28.19\\
1.3&-11.69&-19.18&-28.15\\
1.4&-12.94&-20.37&-28.55\\
1.5&-14.04&-21.24&-28.72\\
1.6&-15.13&-21.87&-28.63\\
1.7&-15.67&-21.81&-27.80\\
\hline
\end{tabular}
\end{center}
\label{tab:rings3}
\end{table}





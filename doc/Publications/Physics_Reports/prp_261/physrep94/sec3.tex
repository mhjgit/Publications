
\subsection{Introduction}

Since quantum chromodynamics (QCD) is commonly
accepted as the theory of the strong interaction, the
NN interaction $V$ is completely determined by the underlying
quark-quark dynamics in QCD. However, due to the non--perturbative
character of QCD at low energies, one is still far from a
quantitative understanding of the NN interaction from the QCD point of
view. This problem is circumvented by introducing
models containing some of the properties of QCD, such as
confinement and chiral symmetry breaking. One of the most used models
is the so-called bag model, where a crucial question is the
size of the radius ($R$) of the confining bag. If the size of the bag radius
is chosen as in the "Little bag" ($R\leq 0.5$ fm) \cite{br79}, then
low-energy nuclear physics phenomena can be fairly well described
in terms of hadrons like nucleons, isobars and various mesons,
which are to be understood as effective descriptions of complicated
multiquark interactions.
However, other models which
seek to approximate QCD also indicate that an effective theory in terms
of hadronic degrees of freedom may very well be the most appropriate
picture for low energy nuclear physics.
Although there is no unique prescription for how to construct
a free NN interaction, a description
of the NN interaction in terms of various meson exchanges is presently
the most quantitative representation of the NN interaction
\cite{mac89,mes79}
in the energy regime of nuclear physics.
We will assume that meson-exchange is an appropriate picture at low
and intermediate energies. Further, we will restrict the attention
to one-boson-exchange (OBE) models only.

In the next subsection we discuss the three-dimensional
reduction of the Bethe-Salpeter equation and
the phenomenological lagrangians
which define the interactions between the various baryons and mesons.
Thereafter, we briefly review the general properties of the nucleon-nucleon
(NN) potential.
To avoid unnecessary overlaps with already existing review articles and
books
on the subject \cite{mhe87,mac89,erk74,bj76}, only a brief
exposition of the underlying theory is given. The reader is referred
to the above extensive works for the more sophisticated details.


\subsection{The one-boson-exchange interaction and the Bethe-Salpeter
equation}

To describe the interaction between the various baryons and mesons of table
\ref{tab:chap3tab1}, we choose the following phenomenological
lagrangians\footnote{We will also employ a pseudovector
lagrangian for the pseudoscalar mesons.
In this case we have
   ${\cal L}_{pv} =g^{pv}\overline{\Psi}\gamma^{5}
   \gamma^{\mu}\Psi\partial_{\mu}\phi^{(ps)}$, where $g^{pv}$ is the
   pseudovector coupling constant.
The relativistic notations follow
Itzykson and Zuber \cite{iz80}. Throughout this work we will also
set $c=\hbar =G=1$, where $G$ is the gravitational constant.}
for spin $1/2$ baryons
\begin{equation}
   {\cal L}_{ps} =g^{ps}\overline{\Psi}\gamma^{5}
   \Psi\phi^{(ps)},
   \label{eq:pseudo}
\end{equation}
\begin{equation}
   {\cal L}_{s} =g^{s}\overline{\Psi}\Psi\phi^{(s)},
   \label{eq:scalar}
\end{equation}
and
\begin{equation}
   {\cal L}_{v} =g^{v}\overline{\Psi}\gamma_{\mu}\Psi\phi_{\mu}^{(v)}
   +g^{t}\overline{\Psi}\sigma^{\mu\nu}\Psi\left
   (\partial_{\mu}\phi_{\nu}^{(v)}
   -\partial_{\nu}\phi_{\mu}^{(v)}\right),
   \label{eq:vector}
\end{equation}
for pseudoscalar (ps), scalar (s) and vector (v) coupling, respectively.
The factors $g^{v}$ and $g^{t}$ are the vector
and tensor coupling constants, respectively.
Similarly the factor $g^s$ is the
phenomenological coupling coefficient for scalar mesons while
$g^{ps}$ is the corresponding coupling constant for
pseudoscalar meson
exchanges. These coupling constants may be
constrained by e.g. the nucleon-nucleon scattering data.
In the above equations, we have defined $\Psi$ to be the baryon field for
spin $1/2$ baryons, while $\phi^{(ps)}$, $\phi^{(s)}$ and $\phi^{(v)}$
are the corresponding meson fields for pseudoscalar,
scalar and vector mesons, respectively.
\begin{table}[hbtp]
\caption{List over selected baryons and mesons.
We also include strange hadrons since the non-relativistic
transition potential we will present applies to these hadrons as well.}
\begin{center}
\begin{tabular}{llll}
\\\hline
\multicolumn{1}{c}{Baryons}&
\multicolumn{1}{c}{Mass (MeV)}&
\multicolumn{1}{c}{Mesons}&
\multicolumn{1}{c}{Mass (MeV)}
\\ \hline
$p,n$&938.926&$\pi$&138.03\\
$\Lambda$&1116.0&$\eta$&548.8\\
$\Sigma$&1197.3&$\sigma$&$\approx 550.0$\\
$\Delta$&1232.0&$\rho$&770\\
$\Sigma^{\star}$&1385.0&$\omega$&782.6\\
&&$\delta$&983.0\\
&&$K$&495.8\\
&&$K^{\star}$&895.0\\ \hline
\end{tabular}
\end{center}
\label{tab:chap3tab1}
\end{table}
In table \ref{tab:chap3tab1}\footnote{Note that we keep
the name $\delta$ for the scalar isovector meson with mass $980$ MeV,
in order to abide to the usage in previous works \cite{mac89}. The recent
label is $a_0(980)$.} we list selected baryons with masses below $1400$ MeV,
though in our calculations we will only deal with those mesons which define
the NN potential.
Note that the above equations are for isoscalar mesons, however,
for isovector mesons, the fields $\phi$ trivially modify to
$\mbox{\boldmath $\tau$} \phi$ with
$\mbox{\boldmath $\tau$}$ the familar isospinor Pauli matrices.
For spin $1/2$ baryons, the fields $\Psi$ are expanded
in terms of the Dirac spinors (positive energy
solution shown here with $\overline{u}u=1$)
\begin{equation}
   u(k\sigma)=\sqrt{\frac{E(k)+m}{2m}}
	  \left(\begin{array}{c} \chi\\ \\
	  \frac{\mbox{\boldmath $\sigma$}{\bf k}}{E(k)+m}\chi
	  \end{array}\right), 
   \label{eq:freespinor}
\end{equation}
with $\chi$ the familiar Pauli spinor and $E(k) =\sqrt{m^2 +|{\bf k}|^2}$. 
The positive energy part of the field $\Psi$ reads
\begin{equation}
\Psi (x)={\displaystyle \frac{1}{(2\pi )^{3/2}}
        \sum_{{\bf k}\sigma}u(k\sigma)e^{-ikx}a_{{\bf k}\sigma}},
\end{equation}
with $a$ a fermion annihilation operator.
Similarly, the spin $3/2$ field operator $\Psi_{\mu}$ 
are represented in terms of the Rarita-Schwinger
spinors $u_{\mu}$,
a spin $1/2$ spinor coupled to a four-vector, which in the rest-frame reads
\begin{equation}
   u_{\mu}({\bf 0},j)=
   \sum_{\sigma_z,r}{\cal C}_{mrj}^{\frac{1}{2}1\frac{3}{2}}\epsilon_{\mu}(r)
   u({\bf 0},\sigma_z),\hspace{0.5cm}
          j=-\frac{3}{2},-\frac{1}{2},\frac{1}{2},\frac{3}{2},
   \label{eq:spin32}
\end{equation}
where ${\cal C}$ is a Clebsch-Gordan coefficient,
$\epsilon_{\mu}$ a spin-1 spinor and
$u({\bf 0}, \sigma_z)$ the free Dirac spinor in
the rest frame with spin projections
$\sigma_z=\pm \frac{1}{2}$. The $u_{\mu}({\bf 0},j)$ can be boosted to an
arbitrary frame such that the Rarita-Schwinger field is
\begin{equation}
\Psi_{\mu}(x)={\displaystyle \frac{1}{(2\pi )^{3/2}}
	\sum_{{\bf k}s}u_{\mu}(ks)e^{-ikx}a_{{\bf k}s}}.
\end{equation}
The lagrangians which describe e.g. the interaction
between a spin $1/2$ and a spin $3/2$ baryon are
\begin{equation}
   {\cal L}_{ps} =g^{ps'} \overline{\Psi}\Psi_{\mu}\partial^{\mu}
		 \phi^{(ps)},
   \label{eq:pseudo13}
\end{equation}
for the exchange of a pseudoscalar meson
and
\begin{equation}
   {\cal L}_{v} =ig^{v'} \overline{\Psi}\gamma_{5}\gamma_{\mu}
    \Psi^{\nu}\left(\partial_{\mu}\psi_{\nu}^{(v)}
   -\partial_{\nu}\psi_{\mu}^{(v)}\right)+h.c. ,
   \label{eq:vector13}
\end{equation}
for the exchange of a vector meson.
As mentioned above, we limit the attention in this work to nucleons only
and thereby spin $1/2$ baryons, 
and the corresponding mesons.
Employing the above lagrangians,
it is then possible to construct a one-boson-exchange
potential model. Typically, a contribution $V_s^{OBE}$
arising from the exchange of a scalar meson
between two spin $1/2$ baryons with equal masses is given by
\begin{equation}
   \bra{p_1'p_2'}V_s^{OBE}\ket{p_1p_2}=
   g_s^2\frac{\overline{u}(p_1')\overline{u}(p_2')u(p_1)u(p_2)}
   {\left(p_1'+p_2'-p_1-p_2\right)^2-m_s^2},
\end{equation}
where $m_s$ is the mass of the exchanged scalar  
meson\footnote{Detailed expressions for the 
one-boson-exchange contributions can be 
found in e.g. refs.\ \cite{mhe87,mac89}.}. The diagrammatic structure
of this equation is shown in fig.\ \ref{fig:scalarobe}.
\begin{figure}[hbtp]
\setlength{\unitlength}{1mm}
      \begin{picture}(140,70)
      \put(25,10){\epsfxsize=12cm \epsfbox{fig4.eps}}
\end{picture}
\caption{
Diagrammatic representation of the exchange of a scalar boson between
two equal mass and spin baryons. The leftmost picture is a shorthand
for the two time-orderings to the right.}
\label{fig:scalarobe}
\end{figure}

In order to obtain the parameters which define an
NN potential derived from OBE models,
the Bethe-Salpeter equation is used as 
the starting point for most calculations.
This equation serves to define a
two-particle interaction
${\cal T}$, meant to reproduce 
properties like low-energy scattering data.
A self-consistent comparison with the data, may in turn define
the parameters
of the NN interaction, such as meson cutoff parameters, or coupling
constants not constrained by the data.  The fully covariant
Bethe-Salpeter equation
reads (suppressing spin and isospin) in an arbitrary frame
\begin{eqnarray}
   \label{eq:bs4dim}
   \bra{p_1'p_2'}{\cal T}\ket{p_1p_2}&=
   \bra{p_1'p_2'}{\cal V}\ket{p_1p_2}
   {\displaystyle +\frac{i}{(2\pi )^4} \int d^4 k}
   \bra{p_1'p_2'} {\cal V} \ket{ P+k,P-k}\\ \nonumber
   & \times  S_{(1)}(P+k)S_{(2)}(P-k)
   \bra{P+k,P-k}{\cal T}\ket{p_1p_2},
\end{eqnarray}
where we have defined $P$ to be {\em half} the total four-momentum,
i.e. $P=\frac{1}{2}(p_1 +p_2)$, and
$k$ to be the relative four-momentum. The term $S_{(i)}$
is the fermion propagator, which for e.g.\ positive energy spin $1/2$ baryons
reads
\begin{equation}
  S_{(i)}(p) = \left(\not p_i -m_i +i\epsilon\right)^{-1},
\end{equation}
with the subscript $i$
referring to baryon $i$.

In principle ${\cal V}$ is supposed to represent all kinds of
irreducible two-particle interactions, though it is commonly
approximated by the lowest order two-particle diagram. With this
prescription we obtain the familiar ladder approach to the Bethe-Salpeter
equation,
similar to the approach discussed in connection with the $G$-matrix. The
schematic structure of the ladder equation, representative for both
the scattering matrix and the reaction matrix $G$, is shown in fig.\
\ref{fig:gschem}.
Eq.\ (\ref{eq:bs4dim}) is a four-dimensional integral equation, which is
rather tedious to solve numerically.
It is therefore commonly replaced by a three-dimensional
quasi-potential equation, where the time components of the four-momenta
of the incoming and outgoing particles have been fixed by some adequate
choice. Eq.\ (\ref{eq:bs4dim}) becomes, still in an arbitrary frame,
\begin{eqnarray}
   \label{eq:bs3dim}
   \bra{{\bf p}_1'{\bf p}_2'}T\ket{{\bf p}_1{\bf p}_2}&=
   \bra{{\bf p}_1'{\bf p}_2'}V\ket{{\bf p}_1{\bf p}_2}
   +{\displaystyle \frac{1}{(2\pi )^3}\int d^3k}
   \bra{{\bf p}_1'{\bf p}_2'}V
   \ket{{\bf P}+{\bf k},{\bf P}-{\bf k}} \\ \nonumber
   & g({\bf k},s)
   \bra{{\bf P}+{\bf k},{\bf P}-{\bf k}}
   T\ket{{\bf p}_1{\bf p}_2}.
\end{eqnarray}
$V$ is now the so-called ``quasi-potential'', with fixed time components
of the in- and outgoing particle momenta. As such, it is no longer
an independent quantity.
A much favored choice for the time components of the four-momenta is to fix
$p_1^0=p_2^0=\frac{1}{2}\sqrt{s}$, $s=(p_1+p_2)^2$ being one of 
the familiar Mandelstam invariants. This prescription puts the two
particles symmetrically off-shell, and is used in connection with the
Blankenbecler-Sugar \cite{bbs68} and the
Thompson \cite{thomp70} equations. The
latter are two possible of several 
three-dimensional reductions of the 
Bethe-Salpeter equation. The quantity $g$ is 
related to the baryon propagators
$S_{(1)}S_{(2)}$ through
\begin{equation}
     g({\bf k},s)=-i\int dk_0 S_{(1)}(k) S_{(2)}(-k).
\end{equation}
The Blankenbecler-Sugar choice for $g$ is 
(assuming $m_1=m_2$ and spin $1/2$ fermions)
\begin{equation}
   g({\bf k},s)=\frac{m^2}{E_k}
                \frac{\Lambda_{(1)}^+ ({\bf k})\Lambda_{(2)}^+ ({\bf -k})}
                {\frac{1}{4}s -E_k^2 +i\epsilon},
   \label{bbs}
\end{equation}
and the Thompson choice
\begin{equation}
   g({\bf k},s)=\frac{m^2}{2E_k^2}
                \frac{\Lambda_{(1)}^+ ({\bf k})\Lambda_{(2)}^+ ({\bf -k})}
                {\frac{1}{2}\sqrt{s} -E_k +i\epsilon}.
   \label{gthomp}
\end{equation}
Note that the above equations are in the two-baryon center-of-mass frame with
${\bf P}=0$.
Here  $\Lambda_i^+ ({\bf k})$ is the projection operator for positive 
energy states with spin $1/2$
\begin{equation}
     \Lambda^+({\bf k})=
     \sum_{\sigma}\ket{u({\bf k},\sigma)}\bra{u({\bf k},\sigma)}=
     \frac{\gamma\cdot k +m}{2m}.
\end{equation}
With a
pseudovector coupling
for the pion one obtains also a strong reduction of negative energy states compared
with the pseudoscalar choice \cite{zt81}.
By construction, the operator $g$ has the same
imaginary cut as the propagators in the full four-dimensional
Bethe-Salpeter equation and preserves therefore the unitarity relations
satisfied by ${\cal T}$.
The mere fact that we have fixed the time component of the four-momenta of
the incoming baryons to be equally off-shell, implies that
the exchanged bosons transfer three-momentum only. The meson propagators
then reduce  to
\begin{equation}
  \frac{i}{-{\bf k}^2-m_{\alpha}^2},
  \label{eq:mesprop}
\end{equation}
with $m_{\alpha}$ the mass of the exchanged boson.
This prescription for the meson propagators
yields a so-called static interaction.
The one-boson interaction then
takes the form (omitting spin and isospin
assignements)
\begin{equation}
   V=\bar u({\bf p}_1{'}) \bar u({\bf p}_2{'})
   \left(\sum_{j=1}^{5} v_j({\bf k})F_j\right) u({\bf p}_1)u({\bf p}_2).
   \label{eq:vobe}
\end{equation}
Here $F_j$ ($j=1,...,5$) denote the Fermi invariants
defined as:
\begin{equation}
S=1^{(1)}1^{(2)}, \quad V=\gamma_\mu^{(1)} \gamma_\mu^{(2)},\quad T={1\over2}
\sigma_{\mu \nu}^{(1)} \sigma_{\mu \nu}^{(2)},
\end{equation}
\begin{equation}
A=i\gamma_5^{(1)} \gamma_\mu^{(1)} i\gamma_5^{(2)} \gamma_\mu^{(2)},
\quad P=\gamma_5^{(1)} \gamma_5^{(2)},
\end{equation}
or if we wish to have a pseudovector coupling for the pseudoscalar mesons
one has to replace $P$ with the on-shell equivalent pseudovector
invariant
\begin{equation}
  P'\propto (\gamma_{\mu}^{(1)}\cdot \partial^{\mu})\gamma_5^{(1)}
(\gamma_{\nu}^{(2)}\cdot \partial^{\nu})\gamma_5^{(2)},
\end{equation}
with the labels $(1,2)$ meaning
the interacting baryons $1$ and $2$. The $v_j$ term
refers to the appropriate meson
propagator defined in eq.\ (\ref{eq:mesprop}), including the appropriate
coupling constants. In general, the coefficients $v_j({\bf k})$ are functions
of the three Mandelstam invariants $s$, $t$ and $u$,
though we have chosen the interaction
to depend on the momentum transfer.
It ought also to be stressed that neither the Blankenbecler-Sugar nor the
Thompson approximations are unique approaches. However, the
extensive compilations of Tjon and co-workers \cite{zt81} indicate that
the three-dimensional Blankenbecler-Sugar 
reduction of the Bethe-Salpeter equation gives
only small differences compared 
with the results obtained by solving the full four-dimensional
eq.\ (\ref{eq:bs4dim}). This conclusion applies to the Thompson choice as well
\cite{mac89,bm90}.

Expanding the free Dirac spinors
in terms of $1/m$ ($m$ is here the mass of the relevant baryon) 
results, to lowest order, in the familiar non-relativistic
expressions for baryon-baryon potentials. Explicitly, in momentum space, 
the non-relativistic version of eq.\
(\ref{eq:vobe}) reads
\begin{equation}
   V(k)=
   \sum_{j=1}^{5} \tilde{v}_j(k)\tilde{F}_j.
   \label{eq:vobek}
\end{equation}
Here the operators $\tilde{F}_j$ are given by the familar expressions 
\begin{equation}
\tilde{F}_1=1, \quad \tilde{F}_2={1\over2}i(\mbox{\boldmath $\sigma$}_1
+\mbox{\boldmath $\sigma$}_2)\cdot {\bf k}\times {\bf P},
\quad \tilde{F}_3=\mbox{\boldmath $\sigma$}_1\cdot
\mbox{\boldmath $\sigma$}_2k^2
-3\mbox{\boldmath $\sigma$}_1\cdot {\bf k}
\mbox{\boldmath $\sigma$}_2\cdot {\bf k},
\end{equation}
\begin{equation}
\tilde{F}_4=\mbox{\boldmath $\sigma$}_1\mbox{\boldmath $\sigma$}_2
\quad \tilde{F}_5=\mbox{\boldmath $\sigma$}_1\cdot {\bf k}\times {\bf P}
\mbox{\boldmath $\sigma$}_2\cdot {\bf k}\times {\bf P}.
\end{equation}
In certain models also operators like ${\bf L}^2$ or an antisymmetric
spin-orbit term
appear \cite{mac89,lomon92}, but these can be expressed as linear combinations
of the above set on the energy shell \cite{br92}. Velocity dependent terms are also
omitted and $\mbox{\boldmath $\sigma$}$ is the spin operator
for spin $1/2$ particles.
We end this subsubsection by displaying the expression for a baryon-baryon
interaction in configuration space.
The configuration space version of the interaction can be obtained through Hankel
transformations of the above
momentum-space components \cite{br92}.  The following
expressions listed below are for a general baryon-baryon interaction for
spin $1/2$ fermions where
the baryons at each vertex may have different masses. Hereafter, baryon
masses are always represented with a capital $M$, with the nucleon mass
given by the average of the proton and neutron masses $M_N$. Following
Greenberg and Lomon \cite{lomon92} we have the
time-honored expression (omitting isospin)
\begin{eqnarray}
V({\bf r})&= \left\{ C^0_C + C^1_C + C_\sigma 
\mbox{\boldmath $\sigma$}_1\cdot\mbox{\boldmath $\sigma$}_2
 + C_T \left( 1 + {3\over m_\alpha r} + {3\over
\left(m_\alpha r\right)^2}
\right) S_{12} (\hat r)\right. \\ \nonumber 
&\left. + C_{SL} \left( {1\over m_\alpha r} + {1\over \left( m_\alpha r\right)^2}
\right) {\bf L}\cdot {\bf S}
\right\} \frac{e^{-m_\alpha r}}{m_\alpha r},
\end{eqnarray}
where $m_{\alpha}$ is the mass of the relevant meson and
$S_{12}$ is the familiar tensor term.
The coefficients are as
follows.  For a scalar meson exchange ($\sigma$ or $\delta$):
\begin{equation}
C^0_C = - {g_1 g_2\over 4\pi} m_\alpha,
\end{equation}
and
\begin{equation}
4C^1_C = - C_{SL} = {g_1 g_2\over 4\pi}\ {m^3_\alpha\over 4}
\left( {1\over
M'_1M_1} + {1\over M'_2M_2}\right) ,
\end{equation}
\begin{equation}
C_\sigma = C_T = 0.
\end{equation}

For a pseudoscalar meson exchange $(\pi,\eta ,K)$:
\begin{equation}
C_\sigma =C_T = {g_1g_2\over 4\pi}\ {m^3_\alpha\over 48} \left(
{1\over M'_1} + {1\over M_1}\right) \left( {1\over M'_2}+{1\over M_2}\right),
\end{equation}
\begin{equation}
C^0_C = C^1_C = C_{SL} = 0.
\end{equation}

For a vector meson exchange $(\rho,\omega ,K^*)$:
\begin{equation}
C^0_C = {m_\alpha\over 4\pi} \left[ G_1 G_2 - {G_1 g^t_2\over 2M}
\left( M'_2 + M_2\right) - {G_2 g^t_1\over 2M} \left( M'_1 + M_1\right) +
{g^t_1g^t_2\over 4M^2}
\left( M'_1+M_1\right) \left( M'_2 + M_2\right)\right],
\end{equation}
and
\begin{eqnarray}
C^1_C &= {m^3_\alpha\over 4\pi} \Biggl[ {G_1G_2\over 16} \left\{ \left(
{1\over M_1} - {1\over M'_1}\right) \left( {1\over M_2}-{1\over M'_2}\right) -
{1\over M'_1M_1}-{1\over M'_2M_2}\right\} \\ \nonumber
&- {G_1 g^t_2\over 16M} \left\{ {\left( M'_2+M_2\right)\over 2} \left( {1\over
M'_2 M_2} - {1\over M'_1 M_1}\right) + \left( {1\over M'_2}+{1\over M_2}
\right) \right\} \\ \nonumber
&- {G_2 g^t_1\over 16M} \left\{ {\left( M'_1 + M_1\right)\over 2} \left(
{1\over M'_1M_1} - {1\over M'_2 M_2}\right) + \left( {1\over M'_1} + {1\over
M_1}\right) \right\} \\ \nonumber
&+ {g^t_1g^t_2\over 32M^2} \left\{ {\left( M'_1 + M_1\right)\over 2}
\left(M'_2 + M_2\right) \left(
{1\over M'_1M_1} + {1\over M'_2 M_2}\right) \right. \\ \nonumber
& + \left. \left( M'_1 + M_1\right) \left( {1\over M'_2} + {1\over
M_2}\right) + \left( M'_2 + M_2\right) \left( {1\over M'_1} + {1\over
M_1}\right) \right\} \Biggr],
\end{eqnarray}
and
\begin{equation}
C_\sigma = - 2C_T = {m^3_\alpha\over 4\pi} \  {G_1G_2\over 24} \left( {1\over
M'_1} + {1\over M_1}\right) \left( {1\over M'_2} + {1\over M_2}\right),
\end{equation}
and
\begin{eqnarray}
C_{SL} &= - {m^3_\alpha\over 32\pi} \Biggl[ 2G_1 G_2 \left( {1\over M'_1M_1}
+{1\over M'_2 M_2} + {2\over M'_2 M_1} + {2\over M'_1M_2}\right) \\ \nonumber
&- {G_1 g^t_2\over M} \left( {M'_2 + M_2\over M'_1 M_1} + {2\over M_1} +
{2\over M'_1} - {1\over M_2} - {1\over M'_2}\right) \\ \nonumber
&- {G_2 g^t_1\over M} \left( {M'_1+ M_1\over M'_2 M_2} + {2\over M_2} +
{2\over M'_2} - {1\over M_1} - {1\over M'_1}\right) \\ \nonumber
&- {g^t_1 g^t_2\over 2M^2} \left( M'_1 + M_1\right) \left( M'_2 + M_2\right)
\left( {1\over M'_1M_1} + {1\over M'_2 M_2}\right) \Biggr]
\end{eqnarray}


The $g_i$ are the standard scalar and pseudovector
coupling constants of the meson-baryon-baryon vertex, 
the $g^t_i$ are the vector meson exchange tensor
coupling constants, and
\begin{equation}
    G_i = g^v_i + {M'_i + M_i\over 2M_N} g^t_i 
\end{equation}
where the $g^v_i$ are the vector meson exchange vector coupling constants. 

As an example, consider the NN potential. Here we may have e.g., the
exchange of $\pi , \eta , \rho , \omega , \sigma$ and $\delta$ mesons.
Then we have $M_1'=M_1=M_2'=M_2=M_N$ and the coefficients for the
exchange of a $\pi$ meson become
\begin{equation}
   C_\sigma =C_T = {g_{NN\pi}^2\over 4\pi}\frac{m^3_{\pi}}{12M_N^2}, 
\end{equation}
and
\begin{equation}
     C^0_C = C^1_C = C_{SL} = 0.
\end{equation}
For the exchange of a $\delta$ meson we have
\begin{equation}
     C^0_C = - {g_{NN\delta}^2\over 4\pi} m_\delta,
\end{equation}
and
\begin{equation}
    4C^1_C = - C_{SL} = {g_{NN\delta}^2\over 4\pi}\ {m^3_\delta\over 2M_N},
\end{equation}
and
\begin{equation}
     C_\sigma = C_T = 0.
\end{equation}
Further, for the vector mesons we will throughout this work use the
relations
\begin{equation}
     \frac{g_{\rho NN}^t}{g_{\rho NN}^v} =6.1,
\end{equation}
and
\begin{equation}
    \frac{g_{\omega NN}^t}{g_{\omega NN}^v} =0.0.
\end{equation}
In so doing we follow reference \cite{mac89}. The fact that we choose such
a strong factor for the relation between $g^t$ and $g^v$ for the
$\rho$ meson, leads to a tensor force for the NN interaction which is weak.
Recall that the other important contribution to the tensor force
stems from the exchange of the $\pi$ meson. Although the coupling constants
for the various mesons in different potential models vary ( the variation
is in general rather small), the 
$g_{\rho NN}^t/g_{\rho NN}^v$ relation of e.g.\ the Bonn potentials
\cite{mac89} employed here remains fixed. As we will discuss in the
subsequent section, the strength of the tensor force is of central
inportance in explaining how a free NN potential is modified in the
nuclear medium.

In conclusion, although we list an analytical expression for 
a non-relativistic potential in $r$-space, we will work 
in momentum space, since both the equations for the scattering
matrix and the $G$-matrix are given in terms of integral equations.
In coordinate space, the corresponding equations are given by differential
equations, which are numerically more involved than integral equations.
In terms of the relative and center-of-mass momenta ${\bf k}$ and
${\bf K}$, the potential in momentum space is related to the nonlocal operator
$V({\bf r},{\bf r}')$ by
\begin{equation}
      \bra{{\bf k'K'}}V \ket{{\bf kK}} =
       \int d {\bf r}d {\bf r'}
        e^{-\imath {\bf k'r'}}V({\bf r'},{\bf r}) e^{\imath {\bf kr}}
       \delta({\bf K},{\bf K'}).
\end{equation}
We will assume that the interaction is spherically symmetric and use
the partial wave expansion of the plane waves in
terms of spherical harmonics.
This means that we can separate the radial part of the wave function from its
angular dependence. The wave function of the relative motion is described
in terms of plane waves as
\begin{equation}
       e^{\imath {\bf kr}}  =
       \bra{\bf r}{\bf k}\rangle =  4\pi \sum_{lm} \imath ^{l}
        j_{l} (kr) Y_{lm}^{*}({\bf \hat{k}}) Y_{lm}({\bf \hat{r}}),
\end{equation}
where $j_l$ is a spherical Bessel function and $Y_{lm}$ the
spherical harmonic.
This partial wave basis is useful for defining the operator for
the nucleon-nucleon interaction, which
is symmetric with respect to rotations, parity and
isospin transformations. These symmetries imply that the interaction is
diagonal with respect to the quantum numbers of total angular
momentum $J$, spin $S$ and isospin $T$. Using the above plane wave expansion,
and coupling to final $J$, $S$ and $T$ we get
\begin{equation}
      \bra{{\bf k'}} V \ket{{\bf k}}
       = (4\pi)^2 \sum_{STll'mJ}
      \imath ^{l+l'} Y_{lm}^{*}({\bf \hat{k}}) Y_{l'm'}({\bf \hat{k}'})
      {\cal C}_{m'M_SM}^{l'SJ}{\cal C}_{mM_SM}^{lSJ}
      \bra{k'l'STJM}V \ket{klSTJM},
\label{eq:vpartial}
\end{equation}
where we have defined
\begin{equation}
    \bra{k'l'STJM}V \ket{klSTJM}=
    \int   j_{l'}(k'r')\bra{l'STJM}V(r',r)\ket{lSTJM}j_l(kr) {r'}^2 dr' r^2 dr.
\end{equation}
We have omitted the momentum of the center-of-mass motion ${\bf K}$ and the 
corresponding orbital momentum $L$, since the interaction is diagonal
in these variables. The potentials we will employ in this work, like
those of the Bonn group, are all non-local potentials defined in 
momentum space, and we will therefore not need the last equation.




\subsection{The nucleon-nucleon potential}\label{subsec:sec312}

One of the key topics of this work
is the behavior of the NN interaction in a nuclear medium. We therefore
restrict the attention here to those details of the NN interaction which
we feel are essential to our purposes. For details about the theoretical
foundation of the one-boson-exchange (OBE) picture, we refer the reader to the
literature, see e.g. refs. \cite{erk74,bj76,mac93}. Here we will however
briefly repeat some of the arguments presented by the Bonn group
\cite{mhe87,mac89} to justify
the OBE picture. 
During the last three decades, several NN potential models have been presented,
however, the reason for focusing on the Bonn potential is the
fact that, this model offers perhaps the most consistent
approach to the nuclear many-body problem at low energies.

As our line of approach is based on one-boson-exchange models, we need to
present arguments which justify this effective picture. The full Bonn
potential model includes
higher-order meson-exchange processes. Typical examples
of such higher-order contributions are shown in fig.\ \ref{fig:chap32pi}.
\begin{figure}[hbtp]
      \setlength{\unitlength}{1mm}
      \begin{picture}(140,100)
      \put(25,10){\epsfxsize=12cm \epsfbox{fig5.eps}}
      \end{picture}
\caption{Selected contributions to $2\pi$ exchange processes. The single
line represents a nucleon, the double line is an isobar $\Delta$, while
the dashed line is the $\pi$-meson. The circles represent
$\pi\pi$ correlations.}
\label{fig:chap32pi}
\end{figure}
The diagrams with circles
in fig.\ \ref{fig:chap32pi} represent  contributions from $\pi\pi$ correlations,
while
the remaining diagrams represent
exchanges of two pions, giving
rise to the contributions of the longest range.
While the two-pion correlations
in the $P$-wave form a resonance (the $\rho$ meson), there
is no proper resonance in the $S$-wave. However, there are strong correlations
in the $S$-wave at low energies, and Durso {\em et al.\ } \cite{dur77} have
shown that these correlations can be described in terms of a broad mass
distribution of about $600\pm 260$ MeV. This can then in turn be
approximated by a zero-width scalar-isoscalar meson of mass $\approx 550$ MeV,
justifying the introduction of the
fictitious $\sigma$ meson in one-boson-exchange models. For peripheral
partial waves, the low-energy scattering data can be accounted for by the
construction of a potential which includes $(\pi +2\pi )$-meson
and $\omega$-meson
exchanges. At shorter internucleonic distances, this model gives however
too much attraction. The necessary repulsion at the shorter 
ranges, and correspondingly lower partial waves, 
is provided by the inclusion
of the $\pi\rho$ diagrams shown in fig.\ \ref{fig:chap3rho}.
\begin{figure}[hbtp]
\setlength{\unitlength}{1mm}
      \begin{picture}(140,80)
      \put(25,10){\epsfxsize=12cm \epsfbox{fig6.eps}}
\end{picture}
\caption{Selected $\pi\rho$ contributions to the NN potential.}
\label{fig:chap3rho}
\end{figure}

The full Bonn potential model can be approximated by a one-boson-exchange
model, where selected mesons (with masses below 1000 MeV) are included. 
Thus, as in the full Bonn model, the only parameters which enter the nuclear
many-body approach exposed in this work are the coupling constants and cutoff
parameters which serve to regularize the potential at short distances. 
As an example we redisplay in table \ref{tab:chap3tab2} the mesons and their
related parameters which define the Bonn potential
models employed in this work. These potential models
employ the Blankenbecler and Sugar (defined in table A.1 of ref.\
\cite{mac89}) and
Thompson (defined in table A.2 of ref.\ \cite{mac89})
reductions of the four-dimensional Bethe-Salpeter
equation.
In total there are 13 parameters which
need to be determined through a fit to the low-energy scattering data.
\begin{table}[hbtp]
\caption{The meson parameters which define the Bonn A, B and C
potentials of tables A.1 and A.2 from ref.\ [12]. The first row
in each entry gives
the parameters from table A.1, while the second line represents
those of table A.2. The numbers in parentheses refer to the $T=0$ channel
for the potentials defined in table A.1. Potential A uses a mass $710$ MeV,
while B and C employ $720$ MeV for the $\sigma$-meson.}
\begin{center}
\begin{tabular}{cccccccc}
\\ \hline
&&
\multicolumn{2}{c}{A}&
\multicolumn{2}{c}{B}&
\multicolumn{2}{c}{C}\\
\multicolumn{1}{c}{Meson}&
\multicolumn{1}{c}{mass(MeV)}&
\multicolumn{1}{c}{$g^{2}/4\pi$}&
\multicolumn{1}{c}{$\Lambda$(GeV)}&
\multicolumn{1}{c}{$g^{2}/4\pi$}&
\multicolumn{1}{c}{$\Lambda$(GeV)}&
\multicolumn{1}{c}{$g^{2}/4\pi$}&
\multicolumn{1}{c}{$\Lambda$(GeV)}
\\\hline
$\pi$&138.03&14.7&1.3&14.4&1.7&14.2&3.0\\
&138.03&14.9&1.05&14.6&1.2&14.6&1.3\\
$\eta$&548.8&4&1.5&3&1.5&0&-\\
&548.8&7&1.5&5&1.5&3&1.5\\
$\rho$&769&0.86&1.95&0.9&1.85&1.0&1.7\\
&769&0.99&1.3&0.95&1.3&0.95&1.3\\
$\omega$&782.6&25&1.35 &24.5&1.85&24&1.4\\
&782.6&20&1.5 &20&1.5&20&1.5\\
$\sigma$&550&8.8&2.0&8.9437&1.9&8.6289&1.7\\
&(710-720)&(17.194)&(2.0)&(18.3773)&(2.0)&(17.5667)&(2.0)\\
&550&8.3141&2.0&8.0769&2.0&8.0279&1.8\\
$\delta$&983&1.3&2.0&2.488&2.0&4.722&2.0\\
&983&0.7707&2.0&3.1155&1.5&5.0742&1.5\\
\hline
\end{tabular}
\end{center} \label{tab:chap3tab2}
\end{table}
Although we will concentrate on the Bonn 
potential models, other potential models
will also be discussed, such as the Paris 
\cite{paris80} or the Reid-soft-core potentials \cite{reid68}. In principle, all
NN potentials
reproduce essentially the same set of low-energy NN scattering data
($E_{lab} \leq 300$ MeV) and properties of the deuteron.
These are referred to as the
``on-shell'' properties of an NN potential, since all potential
models result in a roughly
similar on-shell scattering matrix $T$.
As discussed by Machleidt and Li \cite{mac93}, the on-shell properties of
various potentials yield only rather small differences in nuclear structure
observables. The crucial point is
then the differing off-shell behavior of the
NN potentials.
Since the body of scattering data is conventionally
given in terms of partial waves, the Blankenbecler-Sugar
approach to the four-dimensional
Bethe-Salpeter equation reads in the center-of-mass  system (omitting
angular momentum, isospin, spin etc.\ assignments)
\begin{equation}
   T({\bf k}, {\bf k}')=V({\bf k}, {\bf k}')
   +\int_{0}^{\infty} \frac{d^3 q}{(2\pi )^3}
     V({\bf k}, {\bf q})\frac{M_N^2}{E_q}
     \frac{\Lambda_{(1)}^+ ({\bf q})\Lambda_{(2)}^+ ({\bf -q})}
     {{\bf k}^2-{\bf q}^2 +i\epsilon}T({\bf q}, {\bf k}'),
\end{equation}
where $E_q=\sqrt{M_N^2+{\bf q}^2}$.
Since we are only interested in the matrix elements for positive-energy
spinors, we obtain
\begin{equation}
   T({\bf k}, {\bf k}')=V({\bf k}, {\bf k}')
   +\int_{0}^{\infty} \frac{d^3 q}{(2\pi )^3}
     V({\bf k}, {\bf q})\frac{M_N^2}{E_q}
     \frac{1}
     {{\bf k}^2-{\bf q}^2 +i\epsilon}T({\bf q}, {\bf k}').
    \label{eq:bbs3}
\end{equation}
If we define
\begin{equation}
   \hat{T}({\bf k}, {\bf k}')=
   \sqrt{\frac{M_N}{E_{k'}}}T({\bf k}, {\bf k}')\sqrt{\frac{M_N}{E_{k}}},
\end{equation}
and
\begin{equation}
    \hat{V}({\bf k}, {\bf k}')=
    \sqrt{\frac{M_N}{E_{k'}}}V({\bf k}, {\bf k}')\sqrt{\frac{M_N}{E_{k}}},
\end{equation}
we can rewrite eq.\ (\ref{eq:bbs3}) as
\begin{equation}
  \hat{ T}({\bf k}, {\bf k}')=\hat{V}({\bf k}, {\bf k}')
   +\int_{0}^{\infty} \frac{d^3 q}{(2\pi )^3}
     \hat{V}({\bf k}, {\bf q})
     \frac{1}
     {{\bf k}^2-{\bf q}^2 +i\epsilon}\hat{T}({\bf q}, {\bf k}'),
   \label{eq:bbs2}
\end{equation}
which has the same form as the non-relativistic Lippmann-Schwinger
equation, and serves therefore as a starting point for non-relativistic
nuclear structure calculations. In our calculations of properties
of finite nuclei in section 6 (where relativistic effects
are negligible), we will therefore employ the Bonn potential
as it is defined in table A.1 of ref.\ \cite{mac89},
with the Blankenbecler-Sugar
reduction of the four-dimensional Bethe-Salpeter equation.

With the Thompson choice, we obtain
\begin{equation}
   T({\bf k}, {\bf k}')=V({\bf k}, {\bf k}')
   +\int_{0}^{\infty} \frac{d^3 q}{(2\pi )^3}
     V({\bf k}, {\bf q})\frac{M_N^2}{2E_q^2}
     \frac{1}
     {E_k-E_q +i\epsilon}T({\bf q}, {\bf k}'),
\label{eq:thomp2}
\end{equation}
and defining
\begin{equation}
\tilde{T}({\bf k}, {\bf k}')=
\frac{M_N}{E_{k'}}T({\bf k}, {\bf k}')\frac{M_N}{E_{k}},
\end{equation}
and
\begin{equation}
\tilde{V}({\bf k}, {\bf k}')=
\frac{M_N}{E_{k'}}V({\bf k}, {\bf k}')\frac{M_N}{E_{k}},
\end{equation}
we can rewrite eq.\ (\ref{eq:thomp2}) as
\begin{equation}
  \tilde{ T}({\bf k}, {\bf k}')=\tilde{V}({\bf k}, {\bf k}')
   +\int_{0}^{\infty} \frac{d^3 q}{(2\pi )^3}
     \tilde{V}({\bf k}, {\bf q})
     \frac{1}
     {E_k-E_q +i\epsilon}\tilde{T}({\bf q}, {\bf k}'),
   \label{eq:thomp3}
\end{equation}
which has the form of the Lippmann-Schwinger equation, but with
relativistic energies. We will use the potentials defined through the
Thompson choice in our nuclear matter and neutron matter calculations, where
relativistic effects become of importance at large momenta.

Let us now assign the relevant quantum numbers to the $T$-matrix. Recalling
the structure of eq.\ (\ref{eq:vpartial}),
the general structure of the $T$-matrix is
\begin{equation}
   T_{ll'}^{\alpha}(kk'K\omega)=V_{ll'}^{\alpha}(kk')
   +{\displaystyle \frac{2}{\pi}\sum_{l''mM_S}\int_{0}^{\infty} d {\bf q}
   ({\cal C}_{mM_SM}^{l''S{\cal J}})^2
   \frac{Y_{l''m}^*(\hat{{\bf q}})
   Y_{l''m}(\hat{{\bf q}}) V_{ll''}^{\alpha}(kq)
   T_{l''l'}^{\alpha}(qk'K\omega)}
   {\omega -H_0}},
   \label{eq:bspartial}
\end{equation}
where we let the denominator $\omega -H_0$ be an abbreviation
for one of the denominators in
eqs.\ (\ref{eq:bbs2}) and (\ref{eq:thomp3}).
Further, the  shorthand notation
\begin{equation}
    T_{ll'}^{\alpha}(kk'K\omega)=
   \bra{kKlL{\cal J}ST}T(\omega)\ket{k'Kl'L{\cal J}ST},
\end{equation}
is introduced to denote the $T$-matrix
with momenta $k$ and $k'$ and orbital momenta $l$ and $l'$
of the relative motion, and
$K$ is the corresponding momentum of
the center-of-mass motion. Further, $L$, ${\cal J}$, $S$ and $T$
are the orbital momentum of the center-of-mass motion, the
total angular momentum,
spin and isospin, respectively. They are all represented by the
label $\alpha$. This notation applies to the potential $V$ and the
$G$-matrix as well.

Eq.\ (\ref{eq:bspartial}) can be further simplified
by using the orthogonality
properties of the Clebsch-Gordan coefficients and the spherical harmonics.
Eq.\ (\ref{eq:bspartial}) then reduces to the well-known
one-dimensional angle independent
integral equation
\begin{equation}
   T_{ll'}^{\alpha}(kk'K\omega)=V_{ll'}^{\alpha}(kk')
   +\frac{2}{\pi}\sum_{l''}\int_{0}^{\infty} dqq^2
   \frac{V_{ll''}^{\alpha}(kq)
   T_{l''l'}^{\alpha}(qk'K\omega)}
   {\omega -H_0}.
\end{equation}
Inserting the denominators for the Blankenbecler-Sugar and Thompson
reductions of the full Bethe-Salpeter equation, eqs.\
(\ref{eq:bbs2}) and (\ref{eq:thomp3}) can be rewritten as
\begin{equation}
   \hat{T}_{ll'}^{\alpha}(kk'K)=\hat{V}_{ll'}^{\alpha}(kk')
   +\frac{2}{\pi}\sum_{l''}\int_{0}^{\infty} dqq^2
   \hat{V}_{ll''}^{\alpha}(kq)
   \frac{1}{k^2-q^2 +i\epsilon}
   \hat{T}_{l''l'}^{\alpha}(qk'K),
\end{equation}
for eq.\ (\ref{eq:bbs2}) and
\begin{equation}
   \hat{T}_{ll'}^{\alpha}(kk'K)=\hat{V}_{ll'}^{\alpha}(kk')
   +\frac{2}{\pi}\sum_{l''}\int_{0}^{\infty} dqq^2
   \hat{V}_{ll''}^{\alpha}(kq)
    \frac{1}
     {E_k-E_q +i\epsilon}\hat{T}_{l''l'}^{\alpha}(qk'K)
\end{equation}
for eq.\ (\ref{eq:thomp3}).
Techniques to solve these equations will be discussed in
section 5. The reader should also observe that these equations are similar
to that for the $G$-matrix, the only difference being the omission of the
Pauli operator $Q$ and the medium dependence of the sp energies.
For intermediate states in the two latter equations with $k\neq q$, energy
is not conserved and the nucleons are off their energy shell. It is this
aspect of various potentials which becomes of importance in a nuclear medium.
Consider the contribution to the above scattering matrices for the $^3S_1$
partial wave, exhibited in
fig.\ \ref{fig:chap33s1} to second order
in the ladder approximation. The second-order terms contain
contributions from intermediate states in the $^3D_1$ partial wave
which connects only through the tensor force of the NN interaction. The
$^3S_1$-$^3D_1$ matrix elements account for the largest
second-order contribution.
Since various potentials yield essentially the same on-shell $T$-matrix,
contributions to the $^3S_1$ partial wave may be obtained from
rather different first and second- or higher-order terms to the $T$-matrix.
As an example, the tensor force contribution from the $^3D_1$ intermediate
state shown in fig.\ \ref{fig:chap33s1} may be rather weak as compared
to the first contribution arising from the central part of the potential.
The converse may also be the case, but the outcome may still yield
the same on-shell $T$-matrix.
\begin{figure}[hbtp]
      \setlength{\unitlength}{1mm}
      \begin{picture}(140,80)
      \put(25,10){\epsfxsize=12cm \epsfbox{fig8.eps}}
\end{picture}
\caption{Contributions through second order in the interaction $V$
to the $T$-matrix for the $^3S_1$ partial wave. }
\label{fig:chap33s1}
\end{figure}
In other words, the bulk of the $T$-matrix can be expressed as
\begin{equation}
T\approx V_C + V_T \frac{1}{\omega -H_0}V_T,
\end{equation}
 where $V_C$ is the central part of the NN interaction while $V_T$ is the
tensor force. Moreover, $\omega$ is the energy of the incoming
nucleons and $H_0$ is the unperturbed hamiltonian representing
the intermediate scattering states.
Thus, if the tensor force is weak (strong), a stronger
(weaker) central force is needed to arrive at the same on-shell
$T$-matrix. A similar mechanism is present when we evaluate
the $G$-matrix for either nuclear matter or finite  nuclei,
though,
in these cases we must also account for medium effects
such as the modification of the energy denominator and the
inclusion of the Pauli principle.



These remarks about the off-shell properties of the NN potential play
in turn, as we will discuss below, an important role in explaining
properties of nuclear systems, such as the binding energy of nuclear
matter. Actually,
among the above potential models, the recent versions
of the Bonn group \cite{mac89} have been rather successful in
reproducing properties of nuclear many-body systems.
The success of these potential models have by several investigators
been ascribed
to the strength of the much debated tensor force, examplified
in fig.\ \ref{fig:chap33s1} by the $^3S_1$-$^3D_1$ contribution to the
$^3S_1$ partial wave. Akin to the Bonn
potentials is a rather weak tensor force compared to older and commonly
employed potential models like the semi-phenomenological
Paris potential \cite{paris80} or the purely
phenomenological Reid-soft-core \cite{reid68} potential.

The strength of the tensor force in meson-exchange models
is mainly dominated by the partial cancellation of the tensor components
from the pseudoscalar $\pi$-meson and the vector $\rho$-meson. The larger
(smaller) the tensor-coupling constant of the $\rho$-meson, the weaker
(stronger) the tensor force of the NN potential.
A measure of the strength of the tensor force is given by
the $D$-state probability
of the deuteron. Typical values for the Bonn potentials of table A.2 range from
$P_{D}=4.47$ for the Bonn A potential, $P_{D}=5.10$ for the Bonn B potential and
to $P_{D}=5.53$ for the Bonn C potential \cite{mac89}, whereas the
Paris and the Reid potential cited above have $P_{D}=5.8$ and
$P_{D}=6.5$, respectively. 
In conclusion, one should note that although
the potentials discussed hitherto exhibit different tensor forces,
all potentials are fit essentially to the same set of scattering data,
which in turn implies that a potential with a weak (strong) tensor force
needs a stronger (weaker) central and/or spin-orbit part to arrive at
the same on-shell scattering matrix for free nucleons \cite{mac89}.
The behaviors of these potentials in the nuclear medium will be discussed
in sections 5-8.

The reader should be aware of the fact that the tensor force
is not a directly measurable quantity. Information on the
strength of the tensor force can be obtained through
the determination of  the mixing parameter $\epsilon_1$. The recent reanalysis
of the NN scattering data below $160$ MeV by Henneck \cite{hen93} seems
to favor a strong tensor force in the NN interaction. This conclusion,
as will
be shown throughout this work, is at askance with
most present compilations of
nuclear matter and finite nuclei. Also, a recent comment by Brown
and Machleidt \cite{bm94} shows that the ratio of tensor to vector coupling
constant
\[
     \frac{g_{\rho NN}^t}{g_{\rho NN}^v},
\]
has to be strong, in order to reproduce the phase shifts at higher
scattering energies as well. This is shown in fig.\ \ref{fig:mach1}
for the $^3S_1$-$^3D_1$ mixing parameter $\epsilon_1$. A strong
tensor to vector coupling constant can also be inferred from the
$\pi\pi -N\bar{N}$ analysis of H\"{o}hler and Pietarinen \cite{hp75}.
\begin{figure}[hbtp]
   \setlength{\unitlength}{1mm}
   \begin{picture}(140,180)
   \end{picture}
   \caption{The $\epsilon_1$ mixing parameter at low and intermediate
energies. Taken from ref.\ [37]. Predictions by the Bonn B potential of table A.2
(solid line), the Paris [13] (solid line) and the Reid potential [14] (dashed line)
are shown. The phase shift analysis by Henneck [36] is represented by the
diamonds, while those of Arndt [39] (solid triangles), Bugg and Bryan [40]
(solid dots), and the Nijmegen group [41] (solid squares) are also shown.}
\label{fig:mach1}
\end{figure}
Similarly, the Bonn B potential of table A.2 gives a very good reproduction
of the $^3P_0$ and $^3P_2$ phase shifts for proton-proton scattering, see
figs.\ \ref{fig:mach2} and \ref{fig:mach3}.
\begin{figure}[hbtp]
   \setlength{\unitlength}{1mm}
   \begin{picture}(140,180)
   \end{picture}
   \caption{The  $^3P_0$ phase shifts
    for proton-proton scattering.
    The solid line gives the prediction by a meson model that includes a
    weak tensor force (strong $\rho$), while the
    dashed line employs a strong tensor force. The solid dots are the
    Nijmegen [41] proton-proton phase shift analysis. Taken from ref.\ [37].}
\label{fig:mach2}
\end{figure}

\begin{figure}[hbtp]
   \setlength{\unitlength}{1mm}
   \begin{picture}(140,180)
   \end{picture}
   \caption{The $^3P_2$ phase shifts
    for proton-proton scattering.
    The solid line gives the prediction by a meson model that includes a
    weak tensor force (strong $\rho$), while the
    dashed line employs a strong tensor force. The solid dots are the
    Nijmegen [41] proton-proton phase shift analysis. Taken from ref.\ [37].}
\label{fig:mach3}
\end{figure}

Finally, the many few-body calculations with the various Bonn potentials,
see e.g.\ Song and Machleidt \cite{sm94} and Sammarucca, Xu  and Machleidt
\cite{sxm92},
lend strong support to a potential which has a weak tensor force.
The fact that the tensor force should be weak, is
also confirmed in the present work for heavier
nuclei, see the discussions in sections 5-8.



%    this is section 8

\subsection{Introduction}

The properties of compact
objects like neutron stars depend on the equation of state
at densities up to an order of magnitude higher than those observed
in ordinary nuclei. At such densities, relativistic effects
certainly prevail.

Among relativistic approaches to the nuclear
many-body problem, the so-called Dirac-Hartree and Dirac-Hartree-Fock
approaches have received much interest.
One of the early successes of these
approaches was the quantitative reproduction of spin observables,
which are only poorly described by the non-relativistic theory.
Pertinent to these methods,
was the introduction of a strongly attractive scalar component
and a repulsive vector component \cite{sw86}
to describe the nucleon self-energy.
The sp motion is described by the Dirac equation which includes
this self-energy. Due to the scalar field, the nucleon mass is reduced
enhancing thereby the ratio between the small and large
components of the Dirac spinors. Moreover, the scalar field decouples
causing a
strongly density dependent repulsive effect. In a non-relativistic
language, the reduction of the nucleon mass means that e.g.\
the repulsive spin-orbit term in the
NN potential is enhanced
at large densities.

Inspired
by the successes of the Dirac-Hartree-Fock method, a relativistic
extension of Brueckner theory was proposed by the Brooklyn group
\cite{brook}, known as the Dirac-Brueckner theory.
The Dirac-Brueckner approach differs from the Dirac-Hartree-Fock one,
in the sense that in the former one starts from the free NN potential
which is only constrained by a fit to the world NN data, whereas the
Dirac-Hartree-Fock method pursues a line where the parameters of the
theory are determined so as to reproduce the bulk properties
of nuclear matter. It ought however to be stressed that the Dirac-Brueckner
approach \cite{bm90,brook,hm87},
which starts from NN potentials based on meson-exchange,
is a non-renormalizable theory, where the short-range part
of the potential depends on additional parameters like
vertex cutoffs, clearly minimizing the
sensitivity of calculated results to short-distance inputs. This
should be contrasted to the Dirac-Hartree-Fock
method pioneered by Walecka and Serot \cite{sw86,ser92}.
Nonetheless, one of the appealing features of the Dirac-Brueckner approach
is the self-consistent determination of the relativistic sp
energies and wave functions.
The description presented here for the Dirac-Brueckner approach, follows
closely that of Brockmann and Machleidt \cite{bm90}. We will thus
start from the meson-exchange models of the Bonn group, defined
in table A.2 of ref.\ \cite{mac89}, and discussed in section 3.
Thence, including the necessary medium effects like the
Pauli operators and the starting
energy, we will rewrite eq.\ (\ref{eq:gnonrel}) using 
the Thompson equation. Then, in a self-consistent way, we
determine the above mentioned scalar and vector components  which
define the nucleon self-energy. In this sense we also differ
from the non-relativistic approach discussed above, where the
parameters which are varied at each iterative step are the
non-relativistic effective mass and the effective sp potential
$\Delta$.

\subsection{Symmetric nuclear matter}

To account for medium modifications to the free Dirac equation,
we introduce the notion of the self-energy $\Sigma (p)$.
As we assume parity to be a good quantum, and assume the self-energy
to be hermitian, the self-energy of a
nucleon can be formally written as
\begin{equation}
       \Sigma(p) =
       \Sigma_S(p) -\gamma_0 \Sigma^0(p)
       +\mbox{\boldmath $\gamma$}{\bf p}\Sigma^V(p).
\end{equation}
The momentum dependence of $\Sigma^0$ and $\Sigma_S$ is
rather weak \cite{sw86}.
Moreover, $\Sigma^V << 1$, such
that the features of the Dirac-Brueckner-Hartree-Fock
procedure can be discussed within the framework of the phenomenological
ansatz
\begin{equation}
\Sigma \approx \Sigma_S -\gamma_0 \Sigma^0 = U_S + U_V
\end{equation}
where $U_S$ is an attractive scalar field and $U_V$ the timelike component
of a repulsive vector field.
The finite self-energy modifies the
free Dirac spinors of eq.\ (\ref{eq:freespinor}) as
\begin{equation}
   \tilde{u}(p,s)=\sqrt{\frac{\tilde{E}(p)+\tilde{M}_N}{2\tilde{M}_N}}
	  \left(\begin{array}{c} \chi_s\\ \\
	  \frac{\mbox{\boldmath $\sigma$}{\bf p}}{\tilde{E}(p)+\tilde{M}_N}\chi_s
	  \end{array}\right),
\end{equation}
where we let the terms with tilde represent the medium modified quantities.
Here we have defined \cite{bm90,sw86}
\begin{equation}
   \tilde{M}_N=M_N+U_S
\end{equation}
and
\begin{equation}
      \tilde{E}_{\alpha}=
      \tilde{E}(p_{\alpha})=\sqrt{\tilde{M}_N^2+{\bf p}_{\alpha}^2}
\end{equation}
The relativistic analog of eq.\ (\ref{eq:spnrel}) is \cite{bm90}
\begin{equation}
  \begin{array}{ccc}
   \tilde{\varepsilon}_{\alpha}=&\tilde{E}_{\alpha} +U_V,
   &|{\bf p}_{\alpha}|\leq k_M\\
   &&\\
  =&\tilde{E}_{\alpha}, &|{\bf p}_{\alpha}|> k_M \\
  \end{array}
\end{equation}
where $k_M=k_F$ gives the traditional BHF sp spectrum while $k_M=ak_F$
with $a>1$ results in the MBHF approach.
The sp potential is given as
\begin{equation}
   u_{\alpha} =\sum_{h\leq k_F} \frac{\tilde{M}_N^2}{\tilde{E}_h
	       \tilde{E}_{\alpha}}
	\bra{\alpha h}G(\tilde{E}=\tilde{\varepsilon}_{\alpha}
	+\tilde{\varepsilon}_q)\ket{\alpha h}_{AS},
	\label{eq:urel}
\end{equation}
or if we wish to express it in terms of the constants $U_S$ and
$U_V$,
we have
\begin{equation}
   u_{\alpha} = \frac{\tilde{M}_N}{\tilde{E}_{\alpha}}U_S +U_V.
   \label{eq:sppotrel}
\end{equation}
In eq.\ (\ref{eq:urel}), we have introduced the relativistic $G$-matrix,
which in a partial wave representation is given by
\begin{equation}
   G_{ll'}^{\alpha}(kk'K\tilde{E})=\tilde{V}_{ll'}^{\alpha}(kk')
   +\sum_{l''}\int \frac{d^3 q}{(2\pi )^3}\tilde{V}_{ll''}^{\alpha}(kq)
   \frac{\tilde{M}_N^2}{\tilde{E}_{\frac{1}{2}K+q}^2}
   \frac{Q(q,K)}{\tilde{E}-H_0}
   G_{l''l'}^{\alpha}(qk'K\tilde{E}),
   \label{eq:grel}
\end{equation}
where the relativistic starting energy is
$\tilde{E}=2\tilde{E}_{\frac{1}{2}K+k}$.


Equations (\ref{eq:urel})-(\ref{eq:grel}) are solved self-consistently
in the same fashion as in the non-relativistic case, starting
with adequate values for the scalar and vector components
$U_S$ and $U_V$. This iterative scheme is continued until these
parameters vary little. The calculation is carried out in the
nuclear matter rest frame, avoiding thereby a cumbersome
transformation between the two-nucleon center of mass system
and the nuclear matter rest frame. The additional factors
$\tilde{M}_N/\tilde{E}$ in the above equations
arise due to the normalization of the
nuclear matter spinors $\tilde{w}$, i.e.
$\tilde{w}^{\dagger}\tilde{w}=1$ \cite{bm90}.


Finally, the relativistic version of eq.\ (\ref{eq:enrel}) reads
\begin{equation}
   {\cal E}/A =
   \frac{1}{A}\sum_{h\leq k_F}
   \frac{\tilde{M}_N M_N+{\bf p}_h^2}{\tilde{E}_h}+
   \frac{1}{2A}\sum_{hh' \leq k_F}\frac{\tilde{M}_N^2}{\tilde{E}_h\tilde{E}_{h'}}
   \bra{hh'}G(\tilde{E}=
   \tilde{\varepsilon}_h +\tilde{\varepsilon}_{h'})\ket{hh'}_{AS} -M_N.
   \label{eq:erel}
\end{equation}

The importance of relativistic effects are clearly demonstrated in fig.\
\ref{fig:chap3bearel}. There we show results obtained with the
Bonn A potential of table A.2 in \cite{mac89}.
At small densities the relativistic and 
non-relativistic results almost coincide, whereas at higher densities
the relativistic calculations almost reproduce the empirical points.
\begin{figure}[hbtp]
\setlength{\unitlength}{1cm}
      \setlength{\unitlength}{1mm}
      \begin{picture}(140,150)
      \put(25,10){\epsfxsize=12cm \epsfbox{dbhf.eps}}
      \end{picture}
      \caption{Comparisons of the total binding energy for nuclear
       matter obtained with the BHF method with both a relativistic 
      (dashed line) and
      a non-relativistic (solid line) approach. 
      The continuous sp choice has been
      employed for the dashed and the solid line.
      The results have been
      obtained with the Bonn A potential and are taken from
      ref.\ [15]. We also include the relativistic
      results from a ring-diagram calculation (dotted line) of ref.\ [128].}
      \label{fig:chap3bearel}
\end{figure}
The differences between the relativistic and non-relativistic results
can be understood from the following two arguments, see e.g.\ 
ref.\ \cite{bwbs87}.
Firstly, relativistic 
effects introduce a strongly density dependent repulsive term in the
energy per particle, of the order $(n/n_0)^{8/3}$ where $n_0$ is the 
nuclear matter saturation density in fm$^{-3}$. 
This contribution\footnote{The relativistic 
effects can also be understood as a
special class of many-body forces.}
is important in order to saturate nuclear matter, and is interpreted by
the authors of ref.\ \cite{bwbs87}
as a density dependent correction 
to the mass of the scalar boson $\sigma$, which
is responsible for the scalar term $U_S$ in the relativistic nucleon mass. 
In the 
vacuum, the $\sigma$-meson has self-energy contributions due to its
coupling to virtual nucleon-antinucleon pairs. In nuclear matter,
scattering into states with $k<k_F$ are Pauli blocked, giving in turn
a repulsive contribution to the energy per particle.  This quenching
of the $\sigma$ contribution can also be understood from the 
expression for the one-boson-exchange (OBE) diagram, see eq.\ (28). There,
due to the normalization of the Dirac spinors in nuclear matter,
an additional factor $(\tilde{M}_N/\tilde{E})^2$ has to be 
incorporated in the OBE contribution. This factor decreases
with increasing density, and yields a repulsive 
relativistic effect, in addition to the kinetic energy.
Secondly, the nucleon-nucleon spin-orbit interaction from the 
$\omega$-meson in $P$-waves is enhanced
(the spin-orbit force is repulsive), since the 
relativistic effective mass is changed due to the scalar fields which
couple to negative energy states. 

In the above figure, we show BHF calculations with the continuous choice 
for the sp spectrum taken from ref.\ \cite{bm90}, and
relativistic ring-diagram calculations from ref.\ \cite{jkm93}. The
latter are obtained from the formalism in section 7.2, the differences
between the non-relativistic and the relativistic ring-diagram
scheme being the use of the relativistic $G$-matrix and relativistic
sp energies. 
We note that the differences between 
the relativistic BHF and ring-diagram calculations are small, though the
effect of the infinite summation of particle-particle and hole-hole
ring diagrams is to move the minimum of the curve towards smaller densities.
Moreover, the curvature of the relativistic ring-diagram calculation
is smaller compared with the DBHF calculation of Brockmann and Machleidt
\cite{bm90}, yielding a softer equation of state.
Again, as discussed in 7.2, ring diagrams bring in an additional 
density-dependent repulsive effect. This feature can be studied in
table \ref{tab:rings4}, where we display the $z$ dependence of
the energy shift for the $^3S_1$-$^3D_1$ channel. It is seen that the
difference in energy shift between $z=0.887$ and $z=0.113$ 
becomes less attractive with increasing density, which means that we
have a repulsive effect that increases with density, and therefore shifts
the saturation point.
\begin{table}[hbtp]
\caption{The $z$ dependence of the potential energy
contribution
$\Delta E_0^{pp-hh}/A$ in the $^{3}S_1$-$^{3}D_1$ channel
for a relativistic calculation. All energies in MeV. Taken from
ref.\ [128].}
\begin{center}
\begin{tabular}{rrr}
\\\hline
\multicolumn{1}{c}{$k_F$ fm$^{-1}$}&
\multicolumn{1}{c}{$z=0.113$}&
\multicolumn{1}{c}{$z=0.887$}
\\ \hline
1.2&-11.30&-20.87\\
1.3&-12.52&-21.32\\
1.4&-13.41&-21.41\\
1.5&-13.94&-21.13\\
1.6&-14.07&-20.20\\
1.7&-13.66&-18.68\\
\hline
\end{tabular}
\end{center}
\label{tab:rings4}
\end{table}

Compared with the non-relativistic ring-diagram calculations in 7.2,
we observe that the difference between a calculation which accounts
for ring diagrams and the DBHF results, is much smaller
than in the non-relativistic comparison between BHF calculations
and ring diagrams. This is in turn connected with the fact that
in the non-relativistic calculations, the sp energies depend
quadratically on the momentum, and thereby exhibit an increased
momentum dependence with increasing densities. In the relativistic
case, the dependence on $k$ is only linear.

Thus, in summary, for a relativistic calculation, the chief mechanism
behind the near reproduction of the empirical points for nuclear 
matter, is mainly due to the quenching of the relativistic 
nucleon self-energy. Higher-order effects like ring diagrams, are
much smaller in the relativistic case, as compared with the 
non-relativistic approach. However, needless to say, relativistic
calculations are still beset with uncertainties, and, although
a calculation of first order in the $G$-matrix almost 
reproduces the empirical nuclear matter data, 
there are still many-body effects 
which need to be understood. It is rather questionable if a
calculation to first order in the interaction should reproduce
properties of strongly interacting matter. Our conclusions
in section 6, clearly indicate that more complicated 
many-body effects need to be accounted for. Another effect
which also is an open question is the spin-orbit splitting
in e.g.\ finite nuclei. In the framework of the Walecka model
or the DBHF approach, one is able to reproduce fairly well
the spin-orbit splitting between the 
$0p_{1/2}$ and the $0p_{3/2}$ sp states in $^{16}$O. These are hole
states and should therefore depend more on the density than
the $0d_{3/2}$ and the $0d_{5/2}$ states in $^{17}$O. The latter
are valence nucleons, and most non-relativistic many-body
calculations reproduce failry well the spin-orbit splitting
$\varepsilon_{0d_{3/2}}-\varepsilon_{0d_{5/2}}=5.08$ MeV, whereas
for $\varepsilon_{0p_{3/2}}-\varepsilon_{0p_{1/2}}=6.18$ MeV, most
non-relativistic calculations yield $4\pm 1$ MeV. Relativistic
calculations improve this spin-orbit splitting, as demonstrated
by Brockmann \cite{br78}. An alternative approch is advocated
by Pandharipande and co-workers \cite{pand93}. There, non-relativistic
Monte-Carlo calculations with three-body forces yield a
similar spin-orbit splitting as the relativistic calculations.  
Therefore, although the relativistic 
nuclear matter results are appealing,
we believe that there still are many effects which call for further
studies. The reader should also note that in our calculation
of the relativistic $G$-matrix, a constant effective mass is used 
in the integral over all momenta. This approach is obviously
questionable. Rather, a density dependent relativistic
effective mass should have been used. Whether the saturation
mechanisms seen in the relativistic calculations persist
with a density dependent effective mass, 
remains to see. Moreover, one should not only solve self-consistently
the Dirac equation, but also the Klein-Gordon equation for the various
mesons, in order to obtain effective masses for these particles as 
well. The obtained effective masses, should then be introduced in e.g.\
the medium dependent nucleon-nucleon interaction. 



\subsection{Neutron stars and asymmetric nuclear matter}

We end this section with applications of the Dirac-Brueckner-Hartree-Fock
theory to the calculation of the equation of state for neutron
stars.
The physics of compact objects like neutron stars
offers
an intriguing interplay between nuclear processes  and
astrophysical observables \cite{chris92a,wg91}.
Neutron stars exhibit conditions far from those encountered on earth;
typically, expected densities of a neutron star interior are of the
order $10^3$ or more than the density at neutron drip, $10^{11}$ gcm$^{-3}$.
Thus, central to calculations of neutron star properties, is the
determination of an equation of state for dense matter. This determines
the mass range as well as the mass-radius relantionship for these stars.
It is also an important ingredient to the determination of the
composition of dense matter and to how
thick the crust of a neutron star is.
The latter influences neutrino generating processes and the cooling
of neutron stars \cite{chris92b}.
The properties of neutron stars depend on the equation of state (EOS)
at densities up to an order of magnitude higher than those observed
in ordinary nuclei.
Data on the EOS can be obtained from many sources, such as studies of
the monopole resonance in finite nuclei, high-energy
nuclear collisions, supernovae and neutron  stars.
Supernova simulations seem to require  an EOS which is too soft to
support some observed masses of neutrons stars, whereas analyses of
high-energy nuclear collisions indicate a rather stiff EOS, predicting
in turn neutron star masses which are too large.
Thus, no definite statements can be made
about the equation of state EOS at high densities, except that the
equation of state should probably be moderately stiff in order to
support maximum neutron star masses in the range
from approximately $1.4 M_{\odot}$ to $1.9 M_{\odot}$ \cite{thorsett93}.

However, although quantitative calculations of the EOS for dense
nuclear matter are still beset with many problems, there have recently
been important changes in the qualitative picture. Several mechanisms
have been studied, mechanisms which
result in an EOS which is soft enough to support
neutron stars with maximum masses in the range of the observed data.
Amongst such processes
we have exotic states of nuclear matter, such as
kaon 
\cite{kn87,brown93} or pion condesation.
Another scenario which gives neutron star masses 
within the experimental 
values, has been presented by Pethick and co-workers
\cite{lrp93,hps93}. These authors study
properties of various phase transitions
from spherical nuclei to uniform nuclear matter, and the coexistence
of quark matter and nuclear matter over a finite fraction of the
neutron star volume. 


Here  we derive the EOS
for pure neutron matter
and asymmetric nuclear
matter, using the relativistic DBHF approach discussed above.
For the NN potential, we choose again the Bonn A potential
in table A.2 of \cite{mac89}.
For these calculations we restrict the attention to
the continuous choice for the sp spectra, but we will treat the 
Pauli operator for various proton asymmetries ``exactly'', following 
the expressions given by the authors of ref.\ \cite{swk92}.
Asymmetric nuclear matter is important in e.g.\ studies of neutron
star cooling, as demonstrated recently by Lattimer {\em et al.}
\cite{lpph91} who showed that ordinary nuclear matter
with a small asymmetry parameter can cool by the so-called
direct URCA process, even more rapidly than matter in an exotic state.
In addition, at high densities, degrees of freedom represented by
isobars and hyperons may also result in a rapid cooling of neutron stars
\cite{pplp92}.


With these preliminaries, we show here the results of Engvik {\em et al.} 
from ref.\ 
\cite{lhobo94} for the
energy per particle with various proton fractions 
in fig.\ \ref{fig:eos}.
\begin{figure}
      \setlength{\unitlength}{1mm}
      \begin{picture}(140,160)
      \put(25,10){\epsfxsize=12cm \epsfbox{eos.eps}}
      \end{picture}
\caption{The energy per particle for asymmetric nuclear matter
as function of the particle density with
different proton fractions. Taken from ref.\ [140].}
\label{fig:eos}
\end{figure}
From this figure we note that all proton fractions yield
an energy per
particle which exhibits much the same curvature as the curve for pure neutron 
matter at high densities, although the energy per particle is less repulsive
at high densities. 
At low densities, the situation is rather 
different.
This is due to the contributions from various isospin $T=0$ partial waves,
especially the contribution from the $^3S_1$-$^3D_1$ channel, where the 
tensor force component of the nucleon-nucleon potential provides
additional binding. With a proton fraction of $0.15$, the energy per particle
starts to become attractive at low densities (in the region $0.07$ fm$^{-3}$
to $0.3$ fm$^{-3}$),  and with increasing proton fractions, additional
attraction in the energy 
per particle is introduced.



The next step in the calculations is thus to evaluate the EOS and the total mass
and radius of a neutron star from the above energies per particle with
the proton fractions shown in fig.\ \ref{fig:eos}, in order
to see how different proton fractions influence the mass and radius of 
neutron stars.
Here we
assume that the neutron stars we study exhibit an isotropic
mass distribution. Hence, from the general theory of relativity,
the structure of a neutron star is determined through the
Tolman-Oppenheimer-Volkov equation, i.e.\
\begin{equation}
   \frac{dP}{dr}=
    - \frac{\left\{\rho (r)+P(r) \right\}
    \left\{M(r)+4\pi r^3 P(r)\right\}}{r^2- 2rM(r)},
   \label{eq:tov}
\end{equation}
where $P(r)$ is the pressure and $M(r)$ is
the gravitional mass inside a radius $r$.
To obtain 
observables like the mass and radius of a neutron star, Engvik {\em et al.}
\cite{lhobo94}
combine
eq.\ (\ref{eq:tov}) with the equation of state (EOS), which is defined as
$P(n)=n^2 \left(\partial \epsilon/\partial n\right)$,
where $\epsilon ={\cal E}/A$ is
the energy per particle and $n$ is the particle
density.
The EOS is valid in a limited density range from $0.1$ fm$^{-3}$
to $0.8$ fm$^{-3}$. It is therefore coupled to other equations
of state at higher and lower densities as outlined in ref.\ 
\cite{behoo94}. Total masses and radii are calculated and parametrized
as functions of the central density $n_c$. The results from
ref.\ \cite{lhobo94} are redisplayed  in
figs.\ \ref{fig:mass} and \ref{fig:radius}.
\begin{figure}
      \setlength{\unitlength}{1mm}
      \begin{picture}(140,150)
      \put(25,10){\epsfxsize=12cm \epsfbox{nmass.eps}}
      \end{picture}
\caption{$M/M_{\odot}$ for various proton fractions
as function of the central density $n_c$. Taken from ref.\ [140].}
\label{fig:mass}
\end{figure}
\begin{figure}
      \setlength{\unitlength}{1mm}
      \begin{picture}(140,150)
      \put(25,10){\epsfxsize=12cm \epsfbox{nrad.eps}}
      \end{picture}
\caption{The total radius  $R$ for different
proton fractions as function of the central density $n_c$.
Taken from ref.\ [140].}
\label{fig:radius}
\end{figure}
The neutron star equation of state should probably be only 
moderately stiff to support maximum neutron star masses of
only $1.9 M_{\odot}$ \cite{thorsett93}. From figs.\ \ref{fig:mass}
and \ref{fig:radius} we see that the relativistic EOS for pure neutron
matter seems to be too stiff, since it gives a predicted 
maximum mass of $M_{\mathrm{max}}\approx 2.4 M_{\odot}$ with
a corresponding radius of $R=12$ km. However, the EOS for neutron star
matter could be softened considerably due to pion or kaon
condensation because of the lower symmetry energy of nuclear matter,
and maximum masses are then reduced correspondingly
from the cases with no condensates. We note from figs.\ \ref{fig:mass}
and \ref{fig:radius} that for the EOS the calculated maximum mass
can be reduced to $M_{\mathrm{max}}\approx 2.0 M_{\odot}$
with a corresponding radius of $R\approx 10$ km. Pions may be likely
to condense in neutron star matter because neutrons at the top of
the Fermi sea could decay to protons plus electrons,
and kaon condensation is also believed to be a possible mechanism
which could be energetically favorable in the interior of neutron
stars \cite{brown93,thorson93,brown93b}. Both pion and kaon
condensation would then increase the proton abundance in the matter,
possibly up to more than $40\%$ protons, i.e.\ close to symmetric
nuclear matter, and produce a softer EOS and smaller maximum mass.
The results from Engvik {\em et al.}
\cite{lhobo94} discussed here show this, if one assumes that e.g.\ 
kaon condensation is
a likely mechanism, an increased proton fraction results in smaller
masses. The EOS, even with a proton fraction close to
symmetric nuclear matter, results in maximum masses which are slightly
above the experimental values \cite{thorsett93}. However, following the
the relativistic ring-diagram results above, 
we would
expect that higher order many-body contributions to
soften the EOS further.  Moreover, we have not discussed the energy gain
due to kaon condesation, an energy gain which will soften further
the above EOS at high densities, as demonstrated in ref.\ \cite{thorson93}.


















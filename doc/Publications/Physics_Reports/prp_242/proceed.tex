\documentstyle{article}


%               definition of various commands

\setlength{\hoffset}{-0.5in}
\setlength{\textwidth}{6in}
\setlength{\voffset}{-0.5in}
\setlength{\textheight}{8.5in}
%\renewcommand{\baselinestretch}{1,5}

\newcommand{\be}{\begin{equation}}
\newcommand{\ee}{\end{equation}}
\newcommand{\bra}[1]{\left\langle #1 \right|}
\newcommand{\ket}[1]{\left| #1 \right\rangle}
\newcommand{\nl}[1]{\line(1,0){10}\raisebox{-0.6ex}{ #1}}

%               end of definitions and begin of document

\begin{document}
\pagestyle{plain}
\title{Perturbative many-body approaches to finite nuclei\thanks{To appear in Physics
Reports.} }
\author{M. Hjorth-Jensen,
T. Engeland, A. Holt and E. Osnes\\ Department of Physics,
University of Oslo, \\ Box 1048 Blindern, N-0316 Oslo, Norway}
\maketitle
\clearpage
\begin{abstract}
In this work we discuss
various approaches to the effective interaction appropriate
for finite nuclei. The methods we review
are the folded-diagram
method of Kuo and co--workers and the summation of folded diagrams
as advocated by Lee and Suzuki. Examples of applications to $sd$--shell nuclei
from previous works are discussed together with hitherto unpublished results
for nuclei in the $pf$--shell.
Since we find the method of Lee and Suzuki to yield the best converged results,
we apply this method to calculate
the effective interaction
for nuclei in the $pf$--shell.
To calculate this effective interaction we have used three recent versions
of the Bonn meson-exchange potential model. These versions are fitted to the
same set of data and differ only in the strength of the tensor force. The
importance of the latter for finite nuclei is also discussed.
\end{abstract}
\clearpage
\section{Introduction}
The evaluation of the shell--model effective interaction between nucleons
in  nuclei is one of the most fundamental problems in nuclear many--body
theory. Several theoretical studies have been presented throughout the
last three decades,
and among these studies, perturbative approaches
have been much favored \cite{eo77,kuo81,ko90}, though due to the exceedingly complicated character
of the nuclear many--body problem, there are still many problems which
need further investigation.

Starting from the pioneering work of Kuo and Brown \cite{kb66}, we
attempt in this contribution to review the
perturbative approaches developed by Kuo and co--workers, see e.g.\
ref.\ \cite{ko90} for a tutorial review,
and to discuss their qualities and merits in applications to nuclear physics.
Pertinent to these approaches is the partial summation of a subset
of terms which appear in the perturbative expansion of the effective interaction.
The terms we refer to are the so--called folded diagrams, originally
introduced by Brandow \cite{bran67}, which
arise due to the removal of the dependence of the perturbation expansion
on the exact energy of the nuclear hamiltonian.

This work is organized as follows:
In the next section we discuss the nucleon--nucleon (NN) interaction
in the light of modern meson--exchange potential models, and techniques
to calculate the nuclear reaction matrix $G$ for both nuclear matter
and finite nuclei. The $G$--matrix is introduced to render the intractable
repulsivity of the NN interaction at short distances suitable
for the above mentioned perturbative techniques. Moreover, we will also
in this section discuss how properties of the bare NN interaction,
such as the much debated strength of the tensor force, are
reflected in calculations of nuclear expectation values.
Section three is fully devoted to a discussion of perturbation
theory and the summation of the so--called folded diagrams.
As examples of applications to nuclear structure problems, we
review the results for $sd$--shell nuclei presented in ref.\ \cite{hom92}.
In section four we present hitherto
unpublished results for nuclei with two and more valence nucleons
in the $pf$-shell,
employing the perturbative methods to be discussed
in section three.
In this section we employ three recent versions of the Bonn
meson--exchange potentials described in table A.2 of ref.\ \cite{mac89}
which are fit to the same set of scattering data,
and as such these results differ from previously reported results by us
\cite{heho92}
for $pf$--shell nuclei.
Finally, in section five, some conclusions are drawn.



\section{The Nucleon-Nucleon interaction and the $G$--matrix}
In nuclear many--body calculations, the first problem one is confronted
with is the fact
that the repulsive core of the NN potential $V$ is unsuitable for
perturbative approaches.


This problem is however overcome by
introducing the reaction matrix $G$, displayed by  the summation
of ladder type of diagrams in fig.\ \ref{fig:gmat}, which accounts for the
effects of two--nucleon correlations. The $G$--matrix is
given by the solution
of the Bethe--Goldstone equation
\be
G=V+V\frac{\tilde{Q}}{\omega - \tilde{Q}T\tilde{Q}}G,
\label{eq:gmat}
\ee
where $\omega$ is the energy of the interacting nucleons in
a medium and $V$ is the free NN potential.
Here we have assumed that the energy of the intermediate
states can be replaced by the free kinetic spectrum $T$, since
these states are predominantly of high excitation energy.
The operator $\tilde{Q}$,
commonly referred to as the Pauli operator, is a projection
operator which prevents the interacting nucleons scattering into
states occupied by other nucleons.
In this section we discuss briefly the NN potential $V$
in terms of meson exchange. Furthermore, solutions to eq.\ (\ref{eq:gmat}) for
both nuclear matter and finite nuclei are discussed by way of selected
examples.

\subsection{The one--boson--exchange picture}
Since quantum chromodynamics (QCD) is commonly
accepted as the theory of the strong interaction, the
NN interaction $V$ is completely determined by the underlying
quark--quark dynamics in QCD. However, due to the non--perturbative
character of QCD at low energies, one is still far from a
quantitative understanding of the NN interaction from the QCD point of
view. This problem is circumvented by introducing
models containing some of the properties of QCD, such as
confinement and chiral symmetry breaking. One of the most used models
is the so--called bag model, where a crucial question is the
size of the radius ($R$) of the confining bag. If the size of the bag radius
is chosen as in the "Little bag" ($R\leq 0.5$ fm) \cite{br79}, then
low--energy nuclear physics phenomena can be fairly well described
in terms of hadrons like nucleons, isobars and various mesons,
which are to be understood as effective descriptions of complicated
multiquark interactions.
However, other models which
seek to approximate QCD also indicate that an effective theory in terms
of hadronic degrees of freedom may very well be the most appropriate
picture for low energy nuclear physics \cite{zahed88}.
Although there is no unique prescription for how to construct
a free NN interaction, a description
of the NN interaction in terms of various meson exchanges is presently
the most quantitative representation of the NN interaction
\cite{mac89,mes79}
in the energy regime of nuclear physics.

Motivated by these arguments we employ in this work a model for the
NN interaction based on the exchange of six non--strange mesons, including
the pseudo--scalar $\pi$ and $\eta$ mesons, the vector $\rho$ and $\omega$
mesons and the scalar $\delta$ and $\sigma$ mesons. The latter is
a fictitious meson used in one--boson--exchange (OBE) models to provide in an
effective way the intermediate
range attraction of the potential. This attraction
originates from the $2\pi$ exchange processes.
Realistic calculations \cite{mac89,mhe87}
which include $2\pi$ exchange differ only negligibly from calculations
with OBE models, justifying thereby the introduction of the effective
$\sigma$ meson.

The NN potential V is then a superposition of OBE interactions $V_i$
and is represented as
\[
V=\sum_{i=\pi\eta\rho\omega\delta\sigma}V_i.
\]
The OBE interaction $V_i$, is then constructed using
standard Lagrangian expressions \cite{mhe87,iz80,bj76}.


To evaluate the OBE amplitudes we use in this work the
Bonn potential as it is defined by the meson parameters in table
A.2 of ref.\ \cite{mac89}. For further discussion of this topic, see
e.g. Machleidt's contribution in these proceedings.
There are three sets of meson parameters which then define three potentials,
referred to as the Bonn A, B and C potentials. These potentials differ
in the strength of the tensor force, which is reflected in the
probability of the $D$--state of the deuteron. The significance of the
tensor force for both nuclear matter and finite nuclei will be discussed
in subsection 2.3.
The coupling
constants, cutoffs and masses  of the various mesons of table A.2 of \cite{mac89} are
redisplayed in table  \ref{tab:mespar}. These meson
parameters are obtained through a solution of the scattering equation
for free nucleons. The solution of the scattering equation is given by the
${\cal T}$--matrix, and since all three potentials are fit to the same set of
scattering data, they yield the same on--shell ${\cal T}$--matrix.


The ${\cal T}$--matrix which defines the parameters of table
\ref{tab:mespar} are obtained
by solving the Thompson equation \cite{thomp70}, which in 
operator form reads
\be
{\cal T}=V+V\frac{1}{\hat{e}}{\cal T},
\label{eq:thomp}
\ee
where $\hat{e}$ is the energy denominator for free nucleons.


The Thompson equation is one among many three--dimensional reductions
of the Bethe--Salpeter equation \cite{bd64}.
A discussion of the Bethe--Salpeter
equation and several possible three--dimensional reductions can be found
in refs.\ \cite{bj76,wj73}. Moreover, Tjon and co--workers \cite{tjon81}
have compared results obtained by solving the full four--dimensional
Bethe--Salpeter equation with all relevant OBE diagrams with those
obtained from the Blankenbecler--Sugar (Bbs) \cite{bbs68} and other three
dimensional reductions. For the Bbs equation, which differs little from the
Thompson equation, only small differences were found compared with the
full Bethe--Salpeter equation.

Finally, it is worth noting that the bulk of the ${\cal T}$--matrix is given by
\be
{\cal T}\approx V_C + V_T \frac{1}{\hat{e}}V_T.\label{eq:tappro}
\ee
$V_C$ is the central part of the NN interaction while $V_T$ is the
tensor force. Thus, if the tensor force is weak (strong), a stronger
(weaker) central force is needed to arrive at the same on--shell
${\cal T}$--matrix. A similar mechanism is present when we evaluate
the $G$--matrix for either nuclear matter or finite  nuclei
as well, though, anticipating the discussion in the next subsection,
in these cases we must also account for medium effects
such as the modification of the energy denominator $e$ and the
inclusion of the Pauli principle.


\subsection{Construction of the $G$--matrix}
In this subsection we briefly discuss how to construct the $G$--matrix
for both nuclear matter and finite nuclei.
Several methods to solve the Bethe--Goldstone equation
can be found in the literature, for
a review see ref.\ \cite{ms92}. The basic difficulty resides in the fact
that the Pauli operator $\tilde{Q}$ is diagonal in the laboratory
representation of the two--particle states, while it is non--diagonal in
the relative and center of mass representation.
It has then been common
to approximate the Pauli operator $\tilde{Q}$
with an operator which is diagonal in both
the center of mass and the laboratory systems.

For nuclear matter, the $G$--matrix $G_{NM}$
is conventionally given in terms of partial waves and
the coordinates of the relative and center of mass motion
\be
\bra{kKlL{\cal J}ST}G_{NM}\ket{k'Kl'L{\cal J}ST},\label{eq:nucmat}
\ee
with $k$ and $l$ the momentum and orbital momentum of the relative motion,
respecively, while $K$ and $L$ are the corresponding quantum numbers of
the center of mass motion. Further, ${\cal J}$, $S$ and $T$
are the total angular momentum,
spin and isospin, respectively.
The angle--average Pauli operator used to define the above equation is
\be
\tilde{Q}(k,K)_{NM}=\left\{\begin{array}{cc}
0&k\leq \sqrt{k_{F}^{2}-K^2/4}\\
1&k\geq k_F + k/2\\
\frac{K^2/4+k^2 -k_{F}^2}{kK}&else\end{array}\right.
\label{eq:qnm}
\ee
with $k_F$ the momentum at the Fermi surface. The Pauli
operator for nuclear matter is clearly density dependent.

For finite nuclei we will not use the angle--average option, but rather
a formally
exact technique for handling $\tilde{Q}$, originally presented by
Tsai and Kuo \cite{tk72} and discussed in ref.\ \cite{kkko76}.
Tsai and Kuo employed the matrix identity
\[
\tilde{Q}\frac{1}{\tilde{Q}A\tilde{Q}}
\tilde{Q}=\frac{1}{A}-
\frac{1}{A}\tilde{P}\frac{1}{\tilde{P}A^{-1}\tilde{P}}\tilde{P}\frac{1}{A},
\]
with $A=\omega -T-V$, to rewrite eq.\ (\ref{eq:gmat}) as
\be
G = G_{F} +\Delta G,\label{eq:gmod}
\ee
where $G_{F}$ is the free $G$--matrix defined as
\be
G_{F}=V+V\frac{1}{\omega - T}G_{F}. \label{eq:freeg}
\ee
The term $\Delta G$ is a correction term defined entirely within the
model space $\tilde{P}$ and given by
\[
\Delta G =-V\frac{1}{A}\tilde{P}\frac{1}{\tilde{P}A^{-1}\tilde{P}}\tilde{P}\frac{1}{A}V.
\]
Employing the definition for the free $G$--matrix of eq.\ (\ref{eq:freeg}),
one can rewrite the latter equation as
\[
\Delta G =-G_{F}\frac{1}{e}\tilde{P}
\frac{1}{\tilde{P}(e^{-1}+e^{-1}G_{F}e^{-1})\tilde{P}}\tilde{P}\frac{1}{e}G_F,
\]
with $e=\omega -T$.

We see then that the $G$--matrix for finite nuclei
is expressed as the sum of two
terms; the first term is the free $G$--matrix with no Pauli corrections
included, while the second term accounts for medium modifications
due to the Pauli principle. The second term can easily
be obtained by some simple matrix operations involving
the model--space matrix $\tilde{P}$ only.
Since a common choice for the single--particle wave function in finite
nuclei is the harmonic oscillator (HO), there is not
an explicit density dependence, except for the oscillator parameter,
in the above Pauli correcting term,
contrary the nuclear matter Pauli operator of eq.\ (\ref{eq:qnm})
Both terms in eq.\ (\ref{eq:gmod}) do however depend explicitely on the
starting energy $\omega$, which is medium dependent.
To calculate $G_F$ one needs only to solve eq.\ (\ref{eq:freeg})
via e.g.\ momentum space inversion techniques \cite{ms92}.
The free matrix $G_F$ can then be written in the same manner
as the $G$--matrix for nuclear matter, i.e.,
\be
\bra{kKlL{\cal J}ST}G_F\ket{k'Kl'L{\cal J}ST}.\label{eq:freeg2}
\ee

To obtain a $G$--matrix in a HO basis, eq.\ (\ref{eq:freeg2})
can be transformed into
\[
\bra{nNlL{\cal J}ST}G_F\ket{n'N'l'L'{\cal J}ST},
\]
with $n$ and $N$ the principal quantum numbers of the relative and
center of mass motion, respectively.
Thus, to
represent the $G$--matrix in the laboratory frame of reference
in terms of a normalized and antisymmetrized
HO two--body state, coupled to total angular momentum $J$
and isospin $T$ ( the $jj$-coupling scheme), one needs the transformation
\be
\begin{array}{ll}
&\\
\bra{(ab)JT}G\ket{(cd)JT}=&
{\displaystyle \sum_{\lambda \lambda ' S{\cal J}}\sum_{nln'l'NL}
\frac{1}{\sqrt{(1+\delta_{ab})(1+\delta_{cd})}}\left(1-(-1)^{l+S+T}\right)}
\\&\\
&\times\langle ab|\lambda SJ\rangle \langle cd|\lambda 'SJ\rangle
\left\langle nlNL| n_{a}l_{a}n_{b}l_{b}\lambda\right\rangle
\left\langle n'l'NL| n_{c}l_{c}n_{d}l_{d}\lambda ' \right\rangle
\\&\\
&\times \hat{{\cal J}}(-1)^{\lambda + \lambda ' +l +l'}
\left\{\begin{array}{ccc}L&l&\lambda\\S&J&{\cal J}
\end{array}\right\}
\left\{\begin{array}{ccc}L&l'&\lambda '\\S&J&{\cal J}
\end{array}\right\}
\\&\\
&\times\bra{nNlL{\cal J}ST}G\ket{n'N'l'L'{\cal J}ST},
\end{array} \label{eq:gmat2}
\ee
where $G$ is the given by the sum $G = G_{F} +\Delta G$.
The term
$\left\langle nlNL| n_{a}l_{a}n_{b}l_{b}\right\rangle$
is the familiar Moshinsky bracket, and
$\langle ab|LSJ \rangle $ is a shorthand
for the $LS-jj$ transformation coefficient,
\begin{equation}
\langle ab|\lambda SJ \rangle = \hat{j_{a}}\hat{j_{b}}
\hat{\lambda}\hat{S}
\left\{
\begin{array}{ccc}
       l_{a}&\frac{1}{2}&j_{a}\\
       l_{b}&\frac{1}{2}&j_{b}\\
       \lambda    &S          &J
\end{array}
\right\},\label{eq:lstrans}
\end{equation}
see e.g. ref.\ \cite{law80}. Here
we use $\hat{x} = \sqrt{2x +1}$.
The label $a$ represents the single particle quantum numbers
$n_{a}l_{a}j_{a}$.



\subsection{Strength of the tensor force}
It has been argued that the increased attraction provided
by modern meson--exchange potentials like those of the Bonn group
\cite{mac89,mhe87}, could be ascribed to the fairly weak tensor force
exhibited by these interactions. In order to quantitatively
explain this increased binding we show in this subsection examples
from nuclear matter and for finite nuclei with the potentials A, B and C defined
in table A.2 of ref.\ \cite{mac89}. For nuclear matter, the present
discussion has already been presented by Machleidt in ref.\  \cite{mac89},
whereas for finite nuclei a similar explanation has been
outlined by M\"{u}ther and Sauer \cite{ms92}. For the mere sake of
completeness we restate here some of these arguments. These arguments
will also serve as a link to the discussion on
the spin--tensor decomposition of
the effective interaction performed in section four.


The nuclear matter results for the average potential energy for the partial wave $^{3}S_1$
obtained with potentials A, B and C are displayed
in fig.\ \ref{fig:bhfnm} for a BHF calculation considering various values of
the Fermi
momentum $k_F$. The traditional choice for the single--particle
spectrum above the Fermi momentum has been used in this calculation,
i.e. the particle energies for $k > k_F$ are given by a pure kinetic
term,
yielding an undesirable gap at $k=k_F$. Several improvements to this choice
can be found in the literature,
see e.g. refs.\ \cite{mb90,ms89,km83}. However, since we here are
merely interested in assessing  the properties of the $G_{NM}$--matrix
at various densities, a more
profound discussion of these many--body effects is beyond the scope
of the discussion in this subsection.

The bulk of the $G_{NM}$--matrix ( and the $G$--matrix for
finite nuclei) behaves similarly to the scattering matrix ${\cal T}$--matrix
in eq.\ (\ref{eq:tappro}), i.e.,
\be
G_{NM}\approx V_C + V_T \frac{\tilde{Q}}
{\omega - \tilde{Q}T\tilde{Q}}V_T.\label{eq:gappro}
\ee

The latter equation differs however from eq.\ (\ref{eq:tappro}) due to the
Pauli operator and the fact that the starting energy is given for particles
in a medium. The second term in eq.\ (\ref{eq:gappro}) is then quenched
by the combined effect of the Pauli operator and the attractive energy
denominator. Thus, since all potentials are fit to the same set of data,
a potential with a weak (strong) tensor force needs a larger (weaker)
central force to arrive at the same on--shell scattering matrix. In the
nuclear medium, the quenching of the second term will then be the more
important the larger the tensor force. This is clearly reflected in fig.
\ref{fig:bhfnm}. At small values for the relative monentum $k$, the
potentials differ negligibly, which reflects the fact that all potentials
yield the same on--shell  ${\cal T}$--matrix. At higher densities, the
quenching mechanisms due to the Pauli operator and the energy denominator
account for the differences in binding exhibited by the three potentials,
with the potential exhibiting the weakest tensor force (A) being the most
attractive.

For finite nuclei the situation is not as transparent as in nuclear
matter. The density dependence of the Pauli operator is couched by the
fact that one integrates over all relative momenta $k$ and $k'$ to
obtain a $G$--matrix in an HO basis. Moreover, the summmation
over partial waves in eq.\ (\ref{eq:gmat2}) may include several
channels representing contributions from central, spin--orbit and tensor
contributions. However, the quenching of the tensor force due to the
energy denominator can easily be studied.

In this work we have calculated the $G$--matrix for finite nuclei
in the $pf$--shell with the
above--mentioned three potentials. The scheme described in the previous
subsection and in ref.\ \cite{kkko76} has been used to calculate $G$.
The Pauli operator for the $pf$--shell $\tilde{Q}$ is
defined so as to prevent scattering into intermediate states
with one nucleon in states up to the $1s0d$--shell or two nucleons
in the $1p0f$-- or the $2s1d0g$--shells. The oscillator energy
for $^{40}Ca$ was set to $10$ MeV.

In order to study the behavior of the $G$--matrix for the Bonn A, B and C
potentials as functions of the starting energy $\omega$, we display
in fig.\ \ref{fig:0s0s} the matrix element
\[
\bra{(0s_{1/2})^{2}JT=10}G\ket{(0s_{1/2})^{2}JT=10}.
\]
As in the nuclear matter example discussed above, 
this matrix element is defined by the $^{3}S_1$
partial wave only.
From fig.\ \ref{fig:0s0s} one observes
that at small negative starting energies, the difference between the various
potentials is not so large ( see also the discussion above for nuclear
matter), whereas for more negative energies the differences become important.
Potential A, which has the smallest tensor force, is quenched the least,
roughly a reduction of $18.5\%$ in absolute value when going from $-5$ to $-140$ MeV
in starting energy. The corresponding numbers for potentials B and C are
$23.6\%$ and $27.9\%$,
respectively.

The quenching mechanisms discussed hitherto, explain why a potential
with a weak tensor force results in $G$--matrix
elements which are more attractive
compared to a potential with a stronger tensor force.
It ought
however to be remarked that the modern potentials of the Bonn group, like
those discussed here, exhibit a weak tensor force compared to old
phenomenological or semi--phenomenological potential models.

As a final remark on the properties of the NN interaction, we note
that several investigators \cite{sw86,br89,bmp90,ban92} have argued that
the masses of the mesons and nucleons should be modified in a medium as
compared to the free values given in e.g. table \ref{tab:mespar}.
As shown by Celenza and Shakin \cite{cs86},
the Dirac spinors change in the nuclear
medium, a change which may be described by a decrease of the effective
mass of the nucleons. This change of the nucleon mass increases the
spin--orbit force. On the other hand, a reduction of the meson masses,
except for the pion, weakens the tensor force. The combined action
of these two effects was recently examined by Zheng {\em et al.}
\cite{zzm92} in carbon isotopes, showing that an increase
of the spin--orbit force and a decrease of the tensor force tended
towards a better description of the data. In this work we will
not consider such effects.



\section{Perturbative methods}
One of the motivations behind the use of perturbative methods
in nuclear many--body problems is the possibility of reducing the
Schr\"{o}dinger equation for an $A$--nucleon system given by
\be
H\ket{\Psi_i} = E_i\ket{\Psi_i},\label{eq:schr}
\ee
to a secular equation acting solely within a
physically selected subspace of the
full Hilbert space, denoted as the model space. The hamiltonian $H$ is
given as $H=T+V$, and $\ket{\Psi_i}$ and $E_i$ are the corresponding exact
wave function and eigenvalue of a state $i$.
The Hilbert space is commonly divided into a model space and
an excluded space, defined by operators $P$ and $Q$ which project the
exact wave function into a model part and an excluded part, respectively.
The Schr\"{o}dinger equation is then rewritten as
\be
PH_{eff}P\ket{\Psi_i} = E_i P\ket{\Psi_i},
\ee
where $H_{eff}$ is the model space effective hamiltonian to be defined
below.

There are basically two main approaches in perturbation theory used
to define the effective hamiltonian,
each with its hierarchy of sub--approaches. One of these main
approaches is an energy--dependent, i.e. depending on the
exact energy $E_i$, approach, known as Brillouin--Wigner
perturbation theory, while the Rayleigh--Schr\"{o}dinger (RS) perturbation
expansion stands for the energy independent approach. The latter is
the most commonly used approach in the literature \cite{ko90,lm85}.
Thus, in this work as well, we will only review methods within the framework
of time--independent RS perturbation theory, and within this
theory we will deal with the open--shell many--body
problem for a degenerate model space.
Before we proceed in deriving a general
formula for RS theory, it will be useful to introduce the notations
to be used.
In perturbation theory it is customary to introduce
an auxiliary single--particle potential $U$ such that $H$ can be rewritten
as
\[
H=H_0 + H_I, \hspace{1cm} H_0 = T+U, \hspace{1cm} H_I = V-U.
\]
In nuclear many--body theory, $V$ is replaced by the $G$--matrix
defined in the previous section. If $U$ is chosen such that
$H_I$ becomes small, then $H_I$ can be treated as a perturbation.
$H_0$ defines the unperturbed hamiltonian, with eigenvalues
$\varepsilon_i$ and eigenfunction  and $\ket{\psi_i}$.
The operators $P$ and $Q$ are then defined as
\[
P=\sum_{i=1}^{d}\ket{\psi_i}\bra{\psi_i},
\]
and
\[
Q=\sum_{i=d+1}^{\infty}\ket{\psi_i}\bra{\psi_i},
\]
with $d$ the dimension of the model space. Note that $P^2 =P$, $Q^2 =Q$
and $PQ=QP=0$.
We define the projection of the exact wave function $\ket{\Psi_i}$
as $P\ket{\Psi_i}=\ket{\Psi_{i}^M}$ and a wave operator $\Omega$
which transforms all the model states back into the corresponding
exact states as $\ket{\Psi_i}=\Omega\ket{\Psi_{i}^M}$.  The latter
statement is however not trivial, it actually means that there is
a one--to--one correspondence between  the $d$ exact states and the
model functions.
For a proof of this statement see e.g. ref.\ \cite{ko90}.

We will now assume that the wave operator $\Omega$ has an inverse and
consider a similarity transformation of the hamiltonian $H$ such that
eq.\ (\ref{eq:schr}) can be rewritten as
\be
\Omega^{-1}H\Omega\Omega^{-1}\ket{\Psi_i} = E_i\Omega^{-1}\ket{\Psi_i}.
\label{eq:htrans}
\ee
Recall also that $\ket{\Psi_i}= \Omega\ket{\Psi_i^{M}}$, which means that
 $\Omega^{-1}\ket{\Psi_i} = \ket{\Psi_i^{M}}$ insofar as the inverse of
 $\Omega$ exists. Let us define the transformed hamiltonian ${\cal H}
 =\Omega^{-1}H\Omega$, which can be rewritten in terms of the
operators $P$ and $Q$ ($P+Q=I$) as
\[
{\cal H}=P{\cal H}P+P{\cal H}Q+Q{\cal H}P+Q{\cal H}Q.
\]
The eigenvalues of ${\cal H}$ are the same as those of $H$, since a
similarity transformation does not affect the eigenvalues.
If we now operate  on eq.\ (\ref{eq:htrans}), which in terms of the model
space wave function reads
\[
{\cal H}\ket{\Psi_i^{M}} = E_i\ket{\Psi_i^{M}},
\]
with the operator $Q$, we readily see that
\be
Q{\cal H}P=0.\label{eq:qhp}
\ee

Eq.\ (\ref{eq:qhp}) is an important relation which states that the eigenfunction
$P\ket{\Psi_i}$ is a {\em pure model space eigenfunction}. This implies that
we can then define an {\em effective model space hamiltonian} 
\[
H_{eff}=P{\cal H}P = P\Omega^{-1} H \Omega P,
\]
or equivalently
\be
H\Omega P=\Omega PH_{eff}P,
\ee
which is the Bloch \cite{bloch59} equation. This equation can 
be used to determine the wave operator $\Omega$.

At this stage we should point out that there are no unique representations
for the wave operator $\Omega$. There is a considerable degree of variation
in choice of $\Omega$, yielding different approaches to the perturbative
expansion of the effective interaction. As a consequence,
since in applications we truncate the perturbative expansion at a given
order, different choices for the wave operator $\Omega$ may then give
different results for the effective interaction.
There are however two facets to the above statement.
{\em Firstly}, we write the wave operator $\Omega$ as
\[
\Omega = 1 +\chi,
\]
where $\chi$ is known as the correlation operator. Observing
that $P\Omega P = P$, we see that the correlation operator $\chi$
has the properties
\be
P\chi P = 0, \hspace{1cm} Q\Omega P = Q\chi P =\chi P. \label{eq:chi1}
\ee
Since  $\ket{\Psi_i}=\Omega\ket{\Psi_i^{M}}$ determines the wave operator
only when it operates to the right on the model space, i.e. only the
$\Omega P$  part is defined, the term $\Omega Q$
never appears in the theory,
and we could therefore add the conditions $Q\chi Q =0$ and $P\chi Q =0$
to eq.\ (\ref{eq:chi1}) This leads to the following choice for $\chi$
\be
\chi = Q\chi P. \label{eq:chi2}
\ee
This has been the traditonal choice in perturbation theory.
A more general ansatz is represented by the coupled--cluster method
(CCM) of K\"{u}mmel, Coester and co--workers \cite{klz76}. In the CCM
the wave operator is also defined in the $Q$ space. For a
review of the CCM, see e.g. \cite{lm85} or Lindgren's contribution
in these proceedings.

{\em Secondly}, it is worth noting that there exist different methods for
solving  eq.\
(\ref{eq:chi2}) for the correlation operator $\chi$, and these may lead to
different eigenvalues $E_i$ of eq.\ (12).
It is the scope
of the next subsection to review two such methods, one of these
is the so--called folded--diagram theory of Kuo and co--workers, whereas
the other method was presented by Lee and Suzuki \cite{ls80}.
Hereon we will refer to the method of Lee and Suzuki as the LS method,
whereas the abbreviation FD is reserved for the folded--diagram
method exposed e.g. in ref.\ \cite{ko90}.


\subsection{Summation of folded diagrams}
Having defined the wave operator $\Omega = 1 +\chi$ (note that $
\Omega^{-1}=1-\chi$) with
$\chi$ given by eq.\ (\ref{eq:chi2}) we can rewrite
eq.\ (\ref{eq:qhp}) as
\be
QHP-\chi HP +QH\chi - \chi H\chi = 0. \label{eq:basic}
\ee
This is the basic equation to which a solution to $\chi$ is
to be sought.
Since we will work with a degenerate model spave we define
\[
PH_0 P = E_M P,
\]
where $E_M$ is the model space eigenvalue in the degenerate case
, such that eq.\ (\ref{eq:basic}) reads in a slightly modified form
($H=H_0 + H_I$)
\[
(E_M -QH_0 Q -QH_I Q)\chi = QH_I P -\chi PH_I P -\chi PH_I Q\chi,
\]
which yields the following equation for $\chi$
\be
\chi = \frac{1}{E_M - QHQ}QH_I P -\frac{1}{E_M -QHQ}\chi\left(PH_I P +
PH_I Q\chi P\right).\label{eq:chi3}
\ee
Observing that  the $P$--space effective hamiltonian is given as
\[
H_{eff}= PHP+PH\chi=PH_0 P + V_{eff}(\chi),
\]
with $V_{eff}(\chi)= PH_I P + PH_I\chi$, eq. (\ref{eq:chi3}) becomes
\be
\chi = \frac{1}{E_M - QHQ}QH_I P -\frac{1}{E_M -QHQ}\chi V_{eff}(\chi ).
\label{eq:chi4}
\ee
Now we find it convenient to introduce the so-called $\hat{Q}$-box,
defined as
\be
\hat{Q}(\omega)=PH_I P + PH_I Q\frac{1}{\omega - QHQ}
QH_I P.\label{eq:qbox}
\ee
The $\hat{Q}$--box is made up of non--folded diagrams which are irreducible
and valence linked. A diagram is said to be irreducible if between each pair
of vertices there is at least one hole state or a particle state outside 
the model space. In a valence--linked diagram the interactions are linked
(via fermion lines) to at least one valence line. Note that a valence--linked
diagram can be either connected (consisting of a single piece) or
disconnected. In the final expansion including folded diagrams as well, the
disconnected diagrams are found to cancel out \cite{ko90}. We illustrate
these definitions by the diagrams shown in fig.\ 4, where an arrow pointing upwards
(downwards) is a particle (hole) state. Particle states outside the model space are given by railed lines. Diagram (a) is irreducible, valence linked and connected, 
while (b) is reducible since the intermediate particle states belong to the model space. Diagram (c) is irreducible, valence linked and disconnected. 



Multiplying both sides of eq.\ (\ref{eq:chi4}) with $PH_I$ and
adding $PH_I P$ to both sides we get
\[
PH_I P + PH_I \chi =
PH_I P + PH_I Q\frac{1}{E_M - QHQ}QH_I P -
PH_I \frac{1}{E_M -QHQ}\chi V_{eff}(\chi ),
\]
which gives
\be
V_{eff}(\chi )=\hat{Q}(E_M)-
PH_I \frac{1}{E_M -QHQ}\chi V_{eff}(\chi ).
\label{eq:veff}
\ee

There are several ways to solve eq.\ (\ref{eq:veff}). The idea is
to set up an iteration scheme where we determine $\chi_n$ and
thus $V_{eff}(\chi_n )$ from $\chi_{n-1}$ and $V_{eff}(\chi_{n-1})$.
For the mere sake of simplicity we write $V_{eff}^{(n)}=V_{eff}(\chi_{n})$.

\subsubsection{The FD method}
Let us write eq.\ (\ref{eq:veff}) as
\[
V_{eff}^{(n)}=\hat{Q}(E_{M})-
PH_I \frac{1}{E_M -QHQ}\chi_n V_{eff}^{(n-1)}.
\]
The solution to this equation can be shown to be \cite{ls80}
\be
V_{eff}^{(n)}={\displaystyle\sum_{m=0}^{\infty}}
\frac{1}{m!}\frac{d^{m}\hat{Q}(E_{M)}}{d\omega^{m}}\left\{
V_{eff}^{(n-1)}\right\}^{m}. \label{eq:fd}
\ee
Observe also that the
effective interaction is $V_{eff}^{(n)}$ is evaluated at a given model space energy
$E_M$. If
$V_{eff}^{(n)}=V_{eff}^{(n-1)}$, the iteration is said to
converge. In the limiting case $n\rightarrow \infty$, the
solution $V_{eff}^{(\infty)}$ agrees with the formal solution of
Brandow
\cite{bran67} and Des Cloizeaux \cite{des}
\be
V_{eff}^{(\infty)}=\sum_{m=0}^{\infty}\frac{1}{m!}
\frac{d^{m}\hat{Q}(E_{M})}{d\omega^{m}}\left\{
V_{eff}^{(\infty)}\right\}^{m}.\label{eq:pert}
\ee
The expansion in eq.\ (\ref{eq:fd}) is customarily rewritten as
\be
V_{eff}=\sum_{i=0}^{\infty}F_{i}
\ee
where $F_{i}$ is
\be
\begin{array}{ll}
F_{0}=&\hat{Q}\nonumber\\
F_{1}=&\hat{Q}_{1}\hat{Q}\nonumber\\
F_{2}=&\hat{Q}_{2}\hat{Q}\hat{Q}+\hat{Q}_{1}\hat{Q}_{1}\hat{Q}
\nonumber\\
F_{3}=&\hat{Q}_{3}\hat{Q}\hat{Q}\hat{Q}+\hat{Q}_{2}\hat{Q}_{1}
\hat{Q}\hat{Q}+
\hat{Q}_{2}\hat{Q}\hat{Q}_{1}\hat{Q}+\hat{Q}_{1}\hat{Q}_{2}
\hat{Q}\hat{Q}+\hat{Q}_{1}\hat{Q}_{1}\hat{Q}_{1}\hat{Q}
\end{array}
\ee
and so forth, where we have defined
$\hat{Q}_{m}=\frac{1}{m!}\frac{d^{m}\hat{Q}}
{d\omega^{m}}|_{\omega=E_{M}}$.
Note that although $\hat{Q}$ and its derivatives contain disconnected
diagrams, such diagrams cancel exactly in each order \cite{ko90}, thus
yielding a fully connected expansion in eq.\ (26).
It is further important to note both in connection with the FD expansion and in
the subsequent discussion of the LS method, that a term like
$F_1= \hat{Q}_1 \hat{Q}$ actually means $F_1= P\hat{Q}_1 P\hat{Q}P$ since
the $\hat{Q}$--box is defined in the model space only. Due to this structure
of the terms $F_i$ with $i\geq 1$, only so--called folded diagrams 
contain $P$--space intermediate states.

The perturbation expansion of eq.\ (\ref{eq:pert})
diverges if so--called intruder states are
present in the low--lying spectrum, as shown by Schucan and Weidenm\"{u}ller
\cite{sw72}. Typical examples of intruder states for nuclei with
two valence nucleons are represented by four--particle two--hole
core--deformed
states. Several techniques to handle the intruder state problem
have been discussed during the last two decades \cite{eo77,lk75,lk77,sm90}.
Here we merely observe that a regrouping of the perturbation expansion
as given by the partial summation of the folded diagrams in the FD method,
may result in a converged result even if intruder states are present.
If the FD expansion converges, the converged solutions are then those states
which have the largest overlap with the chosen model space \cite{ls80}.
If the coupling between the model space states and the intruder states
is weak, as in the weak coupling model of ref.\ \cite{ee70}, the
derived two--body effective interaction is expected to reproduce fairly well
the spectra of nuclei with few valence nucleons.
\subsubsection{The LS method}
Another iterative solution has been
presented by Lee and Suzuki \cite{ls80}.
In their approach eq.\ (\ref{eq:veff}) is rewritten as
\be
V_{eff}^{(n)}=\left\{1+PH_I \frac{1}{E_M -QHQ}\chi_{n-1}\right\}^{-1}
\hat{Q}(E_M), \label{eq:ls}
\ee
for $n\geq 1$ and with the equation for $\chi_n$ given as
\be
\chi_n =Q\frac{1}{E_M - QHQ}QH_I P -Q\frac{1}{E_M -QHQ}\chi_{n-1}
V_{eff}^{(n)}.
\label{eq:chils}
\ee
Since the wave operator $\Omega$ is given as $\Omega = P+\chi$, a natural
zeroth approximation to $\chi_0$ would be to set $\chi_0 = 0$, resulting in
\[
\chi_1 =\frac{1}{E_M - QHQ}QH_I P,
\]
and
\[
V_{eff}^{(1)}=\hat{Q}(E_M),
\]
i.e. the bare $\hat{Q}$--box only evaluated at a given model space energy
$E_M$. No folded diagrams are included at this stage.
The next iterative step yields
\[
\chi_2 =Q\frac{1}{E_M - QHQ}QH_I P-Q\frac{1}{E_M -QHQ}\chi_{1}
V_{eff}^{(2)},
\]
and
\[
V_{eff}^{(2)}=\left\{1+PH_I \frac{1}{E_M -QHQ}\chi_{1}\right\}^{-1}
\hat{Q}(E_M)=\frac{1}{1-\hat{Q}_1}\hat{Q}(E_M ).
\]
The solution to this iterative scheme is given as \cite{ls80}
\be
V_{eff}^{(n)}=\left[1-\hat{Q}_{1}-\sum_{m=2}^{n}\hat{Q}_{m}
\prod_{k=n-m+1}^{n-1}V_{eff}^{(k)}\right]^{-1}\hat{Q}.
\ee

It should be noted that the FD and LS expansions of eqs.\ (25) and (30)
may give different results for $V_{eff}$ if terminated after
a given, finite number of iterations. For example,
if we expand the LS expression for $V_{eff}^{(2)}$, 
we obtain a geometric series in $\hat{Q}_1$
\[
V_{eff}^{(2)}=\frac{1}{1-\hat{Q}_1} \hat{Q}=1+\hat{Q}_1 \hat{Q}+
(\hat{Q}_1 )^2 \hat{Q} + \dots ,
\]
which includes the term $F_1=\hat{Q}_1 \hat{Q}$ of the FD expansion.
We see thus that the LS method contains terms which appear in higher--order
iterations of the FD method.
Below we will demonstrate the importance of including
the $\hat{Q}_1$ terms to all orders in the expression for $V_{eff}^{(2)}$,
as given by the LS method. Actual calculations in both the $sd$-- and
$pf$--shells show that the LS method gives after a given number of iterations
a result which is similar to that obtained from $V_{eff}^{(2)}$,
demonstrating thereby the importance of the $\hat{Q}_1$ term compared
to higher--order derivatives. This is not surprising, as each differentiation
of the $\hat{Q}$--box with respect to the starting energy gives an extra
energy denominator.

The LS expansion has also been shown to converge even when intruder
states are present \cite{ls80}. The converged result gives the energy
of those states which have their energy closest to the unperturbed energy $E_M$.
Compared with the FD expansion, the convergence of the LS procedure is
independent of the actual structure of the wave functions of the
eigenstates of concern.

\subsection{Choice of $\hat{Q}$--box}
The first step in the evaluation of the FD or the LS expansions, is the
calculation of the $\hat{Q}$--box and its derivatives, all evaluated
at a given energy
$E_M$, recalling that we have a degenerate model space.
The open question is then how to calculate the $\hat{Q}$--box. Until
recently, a $\hat{Q}$--box including all linked and irreducible diagrams
through second order in the $G$--matrix has been the common choice.
Examples of such calculations with a second--order
$\hat{Q}$--box can be found in ref.\ \cite{skd83} for the $sd$-shell
and the recent work of Polls {\em et al.} \cite{prm90} for the $pf$-shell.
The constituent topologies of a second--order $\hat{Q}$--box are given
in figs.\ \ref{fig:onebody} and \ref{fig:twobody} for the one--body
and two--body diagrams respectively, neglecting second--order diagrams
with Hartree--Fock insertions.



If we choose a second--order $\hat{Q}$--box as our starting point for either the 
LS or the FD expansions, the terms 
$F_0$ and $V_{eff}^{(1)}$ for the LS method then reproduce perturbation theory to second order
in the interaction, depicted by the diagrams in the two previous figures. Since many--body
calculations of the single--particle energies for the valence nucleons do not
properly reproduce the experimental values, a pragmatic approach in perturbation theory has 
been to replace the one--body terms by the experimental single--particle energies. With this
approach, the one--body terms must not be included.

Having accounted for the one--body terms in an effective way, one could represent the
$\hat{Q}$--box by only including the bare $G$--matrix and the core--polarization term
in fig.\ \ref{fig:twobody}. This was the original approach presented by Kuo and Brown
\cite{kb66} in order to determine an effective interaction for mass-18 nuclei.
In subsequent works the remaining two--body diagrams of fig.\ \ref{fig:twobody} were
also included \cite{kuo68}. Moreover, the area of investigation was extended to heavier nuclei,
such as those in the calcium region \cite{kb68} and to the lead region in the work
by Kuo and Herling \cite{hk72}.

The success of these early calculations is demonstrated, in spite of their age,  by the frequent
use of the obtained interactions in various calculations of nuclear properties, either
in their original form or in a somewhat modified form \cite{brown88,ryd90}. A recent
paper by Warburton and Brown even has the title ``Appraisal of the Kuo--Herling interaction
for nuclei with mass numbers 210-212'' \cite{wb91}.

Several improvements to the original Kuo--Brown approach can be found in the literature
\cite{os87}, using either the partial summations of the folded diagrams as discussed
above, or by including renormalized particle--hole interactions as advocated by the
random--phase approximation or studying the perturbation expansion order--by--order
in the interaction. Following the latter philosophy, Barrett and Kirson \cite{bk70}
showed that third--order contributions played a sizeable role.
Typical examples of third--order topologies which play an important
role are shown in fig.\ \ref{fig:third}.


Diagram (a) in this figure is the third--order TDA diagram, while diagrams
(b), (c) and (d)
are examples of third--order diagrams involving other particle--hole
excitations.

Recently we have defined a $\hat{Q}$--box which consists of all relevant topologies
through third order in the interaction \cite{hom92,heho92}, for
a full listing of topologies included we refer the reader to the
appendix of ref.\ \cite{hom92}.
It is then the scope of the next
subsection to discuss results obtained for nuclei in the $sd$--shell using the perturbative
methods described by us, and discuss the importance of third--order topologies.
In section four we will extend the area of investigation to $pf$--shell nuclei.



\subsection{Results for nuclei in the $sd$--shell}
The purpose here is to review results already presented by us for nuclei
in the $sd$--shell in refs.\ \cite{hom92,homs90,homsk91}.
\subsubsection{Details of the calculations for $sd$--shell nuclei}
In the following we find it convenient to use the
following shorthand notations for the LS effective interaction
\[
V_{eff}^{(n+1)} = R_n \hspace{1cm} n=0,1,2, \dots ,
\]
and for the folded--diagram effective interaction FD
\[
V_{eff}^{(n+1)}= \sum_{i=0}^{n}F_i \hspace{1cm} n=0, 1, 2 \dots .
\]
The total effective hamiltonian is then given by either
$H=H_0 + R_n$ or $H=H_0 +\sum_{i=0}^{n}F_i$. Observe that the effective
interaction defined here contains both one--body and two--body terms.
We mentioned above that the one--body diagrams were replaced by the empirical
single--particle energies, since the sum of the one--body diagrams is actually
what determines the single--particle energies of e.g. $^{17}O$.
The effective interactions for either the LS or FD methods should then only include
two--body diagrams. For the LS and FD iterations this is easily accomplished
for the first terms, either $R_0$ or $F_0$. We simply do not include the one--body
diagrams in our $\hat{Q}$--box. If we were to approximate the $\hat{Q}$--box
with the two--body diagrams of fig.\ \ref{fig:twobody}, we would obtain RS
perturbation theory to second order, which was the original approach by
Kuo and Brown \cite{kb66}. Approximating the $\hat{Q}$--box
with all non--folded diagrams through third order 
would however not give RS perturbation
theory through third order since
folded diagrams of third order in the interaction
would be missing.

For terms in the LS or FD expansion beyond the first term,
one has to be aware of the fact that the experimental single--particle energies
already include folded--diagram contributions. Since we use the
experimental single--particle energies we have to subtract from the effective
interaction all terms in which one--body diagrams with a spectator valence line
are folded upon themselves.
This can be done by introducing the
so--called $\hat{S}$--box which consists of all one--body diagrams with a
spectator valence line. The FD effective interaction is then defined as
\be
\tilde{F}_n = F_n (\hat{Q}) - F_n (\hat{S}),
\label{eq:fdtilde}
\ee
to retain only two--body connected diagrams. Similarly, for the LS
expansion we have
\be
\tilde{R}_n = R_n (\hat{Q}) - R_n (\hat{S}),
\label{eq:lstilde}
\ee
with
\be
R_{n-1}(\hat{S})=V_{eff}^{(n)}(\hat{S})=
\left[1-\hat{S}_{1}-\sum_{m=2}^{n}\hat{S}_{m}
\prod_{k=n-m+1}^{n-1}V_{eff}^{(k)}(\hat{S})\right]^{-1}\hat{S}.
\ee

For the $sd$--shell we solved the Bethe--Goldstone equation by defining
a Pauli operator which prevents scattering into intermediate particle states
with a nucleon in the $0s$ or $0p$ states or both nucleons in the $1s0d$--
or $1p0f$--shells. The $G$--matrix was defined for the same 
five starting energies
$\omega$ used in the $pf$--shell.

The single--particle energies are taken form the experimental values of
$^{17}O$, i.e. $\varepsilon_{s_{1/2}}-\varepsilon_{d_{5/2}}=0.87$ MeV
and $\varepsilon_{d_{3/2}}-\varepsilon_{d_{5/2}}=5.08$ MeV. The oscillator
energy was set to $\hbar\omega = 13.3$ MeV. Finally, the auxiliary potential
$U$ used in either the FD or the LS expansions was set to be equal the sum
of all one--body diagrams through third--order in the interaction, as this choice
was found to yield the best convergence in ref.\ \cite{hom92}.


Before we discuss the results obtained from the FD and the LS methods
for the $sd$--shell, we show in fig.\ \ref{fig:hobhf} the behavior
of the effective interaction through third order in
Rayleigh--Schr\"{o}dinger perturbation theory. This means that only folded
diagrams of third order in the $G$--matrix are evaluated.


In this figure we have used both a harmonic--oscillator (HO)
and a Brueckner--Hartree--Fock (BHF) basis for the
single--particle wave functions in order to study the behavior
of the RS perturbation theory at low orders. What can be seen from this
figure is that the BHF basis yields a smaller overlap between
states in the excluded space and the model space, reflected in the small
change when going from second order to third order in the perturbation
expansion. The BHF spectra are however too compressed and in poor
agreement with experiment. This is probably related to the fact that
the radii obtained for the self--consistent single--particle wave functions
are much smaller than the empirical ones \cite{homs90}.

For the HO choice, one observes a clearly non--convergent behavior
when going from second to third order in the perturbation expansion for the
lowest $J=0$ and $J=2$ states.
Actually, as can be seen from the above figure and  as we will show
in section four, third--order contributions are sizable,  a
change of $1.8$ MeV for the lowest $J=0$ state, and care should therefore
be exercised when approximating either the perturbation expansion or the
$\hat{Q}$--box to a given order.


We next consider the results obtained from the FD and LS methods
considering a third--order $\hat{Q}$--box. To demonstrate the convergence
of these iteration schemes we single out the matrix elements coupled
to $JT=01$, which together with the $JT=10$ state are the $JT$ combinations
which show the poorest convergence after four iterations. In table
2  we list the FD matrix elements where the terms
$\tilde{F_i}$ represent
the contributions from fold $i$ of eq.\ (\ref{eq:fdtilde}). The rightmost
column, labeled $V_{eff}$ is the final FD effective interaction including
terms up to $\tilde{F_4}$. As can be deduced from this table, some
of the $\tilde{F_3}$ and $\tilde{F_4}$ terms are quite large, and one can not
decide whether the expansion in terms of the number of foldings $\tilde{F_i}$
is converged after four foldings  only.

The corresponding LS results are shown in table \ref{tab:lssd}.
The convergence of this method is much better
than
that of the FD method, and we find that the  $\tilde{R_1}$ term of the LS scheme gives a result
which is very similar to the one after four iterations, namely $\tilde{R_4}$.
Recalling the discussion in subsection 3.1, this
means that the terms which include the $\hat{Q}$--box and
its first derivative are the most important. Moreover,
the importance of these two terms is reflected in
the differences between the LS and the FD methods. The success of the LS method as compared
to the FD method, indicates that each successive term in the FD expansion is so large
that the series must summed to all orders as done in the LS method.


In fig.\ \ref{fig:lsconv}
we show the low--lying spectra  for $^{18}O$ obtained with the LS method after
four iterations.

It is interesting to note that a potential like the Bonn A potential
defined in table A.1 of ref.\ \cite{mac89} provides enough attraction to generate
a realistic interaction for the valence nucleons of the $sd$--shell. This 
increased attraction is ascribed to the weaker tensor force exhibited by
this potential, compared to potentials like the phenomenological Reid potential
\cite{reid68} or the semi--phenomenological Paris potential \cite{paris80}.
A measure for the strength of the tensor force
is the predicted $D$--state probability
$P_D$ of the deuteron. For the Bonn A potential of table A.1 of ref.\ \cite{mac89},
one has $P_D = 4.4$ whereas the Paris potential has $P_D =5.8$.  The analysis
by Clajus {\em et al.} \cite{clajus90} of the NN tensor force in the
three--nucleon system and the empirical deduction of matrix elements by
Daehnick \cite{daen83} favor
a potential with a weak tensor force. In addition
to the fact that a potential with a weak tensor force introduces more binding,
the summation over intermediate states in the evaluation of the constituent
$\hat{Q}$--box diagrams can very well be approximated by including excitations
of $2\hbar\omega$ in oscillator energy only \cite{ms92,sommer}.
As a final remark, we ought to point out that a potential with a weak tensor force
like Bonn A, gives too much attraction in the $T=0$ channel and especially for the
lowest--lying $J=1$ state. This can be seen from fig.\ \ref{fig:f18} where we display
the LS results for $^{18}F$.

In the next section, we show that a potential with too weak a tensor force
introduces too much attraction in the $JT=10$ for nuclei in the $pf$--shell
as well.

\section{Results for nuclei in the $pf$-shell}
Since we found the LS method to yield the best converged results for the
effective $sd$--shell interaction, we calculate the two--body effective interaction
for the $pf$--shell with this method only. Similar conclusions regarding
the properties of the LS and FD methods have been reached by other investigators
\cite{skd83,mpk85,hok92}.

For nuclei with two or more valence nucleons in the $pf$--shell we take the
single--particle energies of $^{41}Ca$ to be
$\varepsilon_{f_{5/2}}-\varepsilon_{f_{7/2}}=6.5$ MeV,
$\varepsilon_{p_{3/2}}-\varepsilon_{f_{7/2}}=2.1$ MeV and
$\varepsilon_{p_{1/2}}-\varepsilon_{f_{7/2}}=3.9$ MeV. The auxiliary potential $U$ which
enters in the definition of $H_I = G-U$ is taken to be the sum of all one--body diagrams
through third order, as in the $sd$--shell. The unperturbed energy at which
we evaluate the $\hat{Q}$--box was set to $E_M = -16$ MeV, approximately twice the energy
of the $f_{7/2}$ state. The $G$--matrix was defined in section two. The Bonn 
potentials 
we use here are  different from the one employed by us in ref.\ \cite{heho92}, where
potential A defined in table A.1 of ref.\ \cite{mac89} was used.
Whereas the potential A (and B and C as well)
defined in table A.1 
is obtained through a non--relativistic approximation of the Blanckenbecler--Sugar
equation,  no such approximation has been done when solving the Thompson equation
for the potentials in table A.2, or in the construction of the $G$--matrix. 
The main differences between the 
spectra presented here for the $pf$--shell 
and the results of
ref.\ \cite{heho92}, are seen in the total binding energy, with the potentials
defined in table A.2 yielding the least bound spectra.

\subsection{Nuclei with two valence nucleons}
In this subsection we first discuss the results obtained through second order in RS
perturbation theory, which corresponds to setting 
the terms $\tilde{F_0}$ and $\tilde{R_0}$
equal to the second--order two--body diagrams displayed in fig.\ \ref{fig:twobody}.
No folded diagrams are included at this level. Next, we discuss the differences
between a second-- and a third--order $\hat{Q}$--box. Furthermore, the third--order
$\hat{Q}$--box is used to evaluate the LS effective interaction $\tilde{R_i}$.

\subsubsection{The effective interaction through second order}
In figs.\ \ref{fig:secondca} and \ref{fig:secondsc} we show the low--lying
spectra for $^{42}Ca$ and $^{42}Sc$ obtained from the Bonn A, B and C
potentials, by approximating the perturbation expansion
to second order in the interaction. The second--order results have been
obtained by including summations up to $6\hbar\omega$ in oscillator quanta
in the evaluation of the diagrams in fig.\ \ref{fig:twobody}. 
The notation
in this figure is the same as in fig.\ \ref{fig:hobhf}.
It should be noted that
our definition of the $G$--matrix in section 2 for the $pf$--shell does not allow for
excitations of e.g. $6\hbar\omega$ from intermediate particle states in 
the $hfp$--shell or higher shells. Thus, our procedure is not fully consistent,
however, a diagram like the core--polarization diagram can fairly well be approximated
by including excitations of $2\hbar\omega$ only if one employs a potential
with a weak tensor force. 

For $^{42}Ca$ we observe from fig.\ \ref{fig:secondca} that all potentials
result in spectra which are too compressed if we include the bare $G$--matrix
only. It is also worth  noting that at this level of approximation, the
differences in binding energies are negligible, though the tensor force
of the various interactions is rather different. This observation
is related to the fact that the strength of the tensor force is less
inportant in the $T=1$ channel, where the $^{3}P_1 - ^{3}F_1$ partial
wave plays the dominant role.
If we incorporate the
second--order diagrams from fig.\ \ref{fig:twobody}, diagrams (b), (c) and (d),
the renormalized effective interaction $H_{eff}^{(2)}$ introduces
enough attraction to give a better reproduction of the data. In the $T=1$
channel, the core--polarization diagram, (b) of fig.\
\ref{fig:twobody}, gives the largest contributions ( see table \ref{tab:diags}
below) for certain diagonal matrix elements.
Furthermore, this diagram gets contributions from both $T=0$ and $T=1$ $G$--matrix
elements
, which explains why potential A gives the strongest binding 
in the ground state.


In this, and the subsequent figure, we have omitted states
which represent possible intruder states in the low--lying spectra.
Inclusion of such states may bring in an additional attraction
of $0.1-0.5$ MeV \cite{ee70}, thereby degrading the agreement with
the data. However, as we will show below, third--order contributions
may be large, and their sign is not {\em a priori} known.


The spectra for $^{42}Sc$ show that the different  strengths in the tensor
force are already reflected at the level of the bare $G$--matrix, with
potential A being the most attractive. The qualitative pattern for the
first--order effective interaction pertains to second order as well.


Finally, in connection with fig.\ \ref{fig:secondsc}, we see that
the potential with the weakest tensor force, Bonn A,
results in too much binding
energy for the ground state, in line with the  observation we made
for $^{18}F$. As shown below, this finding is almost independent of the order
of the perturbative expansion, rather it seems to be a property akin
to the potential itself, if we apply the harmonic oscillator
choice for the single--particle wave functions. 
A similar observation was made by us in ref.\
\cite{heho92}. This statement is our major
conclusion in this work. Its implications will be discussed below.




\subsubsection{Inclusion of third--  and higher--order contributions}
In order to understand the importance of the third--order contributions,
we list in table
\ref{tab:diags}  the contributions from the two--body
diagrams up to third order in the interaction for the Bonn B
potential\footnote{The Bonn A and C potentials exhibit a similar pattern, and are
hence omitted.}.
No folded diagrams
are included in our third--order diagrams. 
For the $JT=01$ configurations only certain non--diagonal third--order 
contributions are sizable, and one could therefore be tempted to approximate
the $\hat{Q}$--box with second--order diagrams only. Inspection of 
table \ref{tab:diags} tells us however that in the $JT=10$ channel, third--order
contributions may be large, actually of the size of the bare $G$--matrix or the
total second--order contributions. 

Table 4 tells us then that care should be exercised when approximating the
$\hat{Q}$--box to a given order. {\em Actually, we will claim, that in order to be
consistent, one has to include at least all diagrams through third order
in the interaction.} Moreover, another remark regarding our procedure is that several
large and important third--order diagrams which are included in the 
present {\em brute force} scheme are not included in most partial summation schemes
like e.g. the random--phase approximation. Finally, our procedure gives the
{\em correct} off-shell treatment of the various $G$--matrix interactions in a given
diagram.  

In order to discuss the differences between the various order of the $\hat{Q}$--box
and the properties of the potentials under consideration, we display
in table \ref{tab:casc} the eigenvalues of the low--lying
states for both isospin channels, obtained by applying the LS method
with both a second-- and a third--order $\hat{Q}$--box.  

In our calculations of the $\hat{Q}$--boxes we have included 
excitations up to $6\hbar\omega$ in oscillator energy. This means that 
we also receive contributions from $0s$ hole states. In ref.\ \cite{heho92} 
we showed that it was necessary to include these excitations in order
to get a reasonable convergence in terms of the summation over intermediate
states in each diagram. (It should be borne in mind that this convergence
property is not the same as the order--by--order convergence of the effective
interaction). In the $sd$-shell calculations discussed in the previous section, we 
could approximate the summation over intermediate states to excitations of 
$2\hbar\omega$ only, whereas in the $pf$--shell we need excitations up to
$4-6\hbar\omega$. The latter is a medium effect, reflecting the fact that 
there are more intermediate states to sum over. 

From table \ref{tab:casc} we see that the Bonn A potential is the one which
gives the strongest binding in the ground state, irrespective of the
approximation used for the $\hat{Q}$--box.
An LS calculation with a second--order $\hat{Q}$--box 
results in the best reproduction of the data, when employing the Bonn A
potential. The problem
of the $JT=10$ state we discussed
in connection with second--order perturbation theory has vanished. However, in order
to be consistent, third--order contributions have to be included. In this case, 
potential A introduces too much binding for the $JT=10$ state, in line with our 
previous comments on result for the $sd$--shell. 
If one were to account for so--called intruder states, potential B is seemingly the 
most appropriate candidate for nuclear structure calculations. {\em The latter 
statement should however be accompanied with the important side remark that
these results apply to the harmonic--oscillator  choice for the
single--particle wave functions.} With a BHF basis, potential A might turn out
to be the most likely candidate anyway, in line with observations
made in nuclear matter \cite{mb90}. 
Nonetheless, all potentials discussed hitherto exhibit a 
weak tensor force compared to other widely used potential models such as the
Paris potential \cite{paris80}. Our results, see refs.\ \cite{heho92,hok92} as well,
so far indicate that the Bonn potential models, are at present, 
the best candidates for a consistent microscopic description of low--energy
spectra.


\subsection{Spin--tensor decomposition of the effective interaction}
In section two we discussed a mechanism which qualitatively
explained why a potential with a weak tensor force introduced more
attraction in the calculation of energy spectra of nuclei.
The results we have presented for the $pf$--shell in the preceeding
subsections, clearly indicate that potentials with a weak tensor
force tend to introduce more binding, though a weak tensor
force, like potential A resulted in too highly bound spectra in
the $T=0$ channel.
To gain a more quantitative insight into the nature of the effective
interaction, we perform a spin--tensor decomposition
of some of the effective interactions we have discussed. To be more
specific; we decompose the bare $G$--matrix for all three potentials
in the $pf$--shell, since the $G$--matrix depends only on one
isospin value, contrary to the effective interaction to second or higher
order, where in general $G$--matrix elements of different isospin 
contribute to a given diagram.
Secondly, we will decompose the
effective interactions obtained with the Lee--Suzuki method employing
a third--order $\hat{Q}$--box and including excitations of $6\hbar\omega$ in
oscillator energy.
Such spin--tensor decompositions of the effective interaction
have mainly been carried out for $sd$--shell nuclei
\cite{skd83,brown88,os92}, though a
recent work by Richter {\em et al.} \cite{richt91} reports a
decomposition of empirically derived effective matrix elements
for the $pf$--shell.

Another reason for carrying out a spin--tensor decomposition
of the effective interaction,
is to get an assessment of the features which have to be incorporated
into effective interactions for heavier nuclei. These effective interactions
 are not easily
calculable from many--body theory due to the many single--particle
degrees of freedom.


The effective interaction discussed in the previous subsections is a
scalar two--body operator. A general scalar two--body operator
${\cal V}$ can be written as
\begin{equation}
{\cal V} = \sum_{k} {\cal V}_{k} = \sum_{k} \mbox{\boldmath $C^{(k)}\cdot
Q^{(k)}$},
\end{equation}
where the operators \mbox{\boldmath $C^{(k)}$} and
\mbox{\boldmath $Q^{(k)}$} are irreducible spherical tensor
operators of rank $k$, acting in spin and coordinate space,
respectively. The value of $k$ is limited to $k\le 2$,
with $k=0$ referring to the central component of the two--body
operator.
The values of $k=1$ and $k=2$ are called
the spin--orbit and the tensor components, respectively.



In this work we interpret the
two--body matrix elements of ${\cal V}_k$
in the representation of the $LS$--coupling scheme as done by Wildenthal,
Brown and co--workers \cite{brown88,richt91} and Hosaka and Toki
\cite{ht91}. This representation allows for a more direct comparison
with the NN interaction, which is expressed in terms of partial
waves.
The $LS$--coupled
matrix element of a given component $k$
$\bra{(\alpha\beta )LSJ'T}{\cal V}_{k}\ket{(\gamma\delta )LSJ'T} $,
with $\alpha = n_{\alpha}l_{\alpha}\frac{1}{2}$,
is related to the corresponding matrix elements of the total interaction
in the $jj$--scheme
by
\be
\begin{array}{ll}
&\\
\bra{(\alpha\beta )LSJ'T}{\cal V}_{k}\ket{(\gamma\delta )L'S'J'T}&=
{\displaystyle\frac{1}{\sqrt{(1+\delta_{\alpha\beta})(1+\delta_{\gamma\delta})}}}
(-1)^{J'}\hat{k}^{2}\left\{\begin{array}{ccc}L&S&J'\\S'&L'&k
\end{array}\right\}
\\&\\
&\times {\displaystyle\sum_{J}}(-1)^{J}\hat{J}^{2}\left\{\begin{array}{ccc}L&S&J\\S'&L'&k
\end{array}\right\}
{\displaystyle \sum_{j_a j_b j_c j_d}}
\langle ab|LSJ\rangle
\langle cd|L'S'J\rangle
\\&\\
&\times\sqrt{(1+\delta_{ab})(1+\delta_{cd})}\bra{(ab)JT}{\cal V}\ket{(cd)JT}
.\end{array}\label{eq:lsdeco}
\ee
The quantum numbers of the $LS$--coupled
matrix elements derived from eq.\ (\ref{eq:ls})
and their respective numbering are displayed in the appendix for isospin $T=0$.

The reason we omit the $T=1$ channel is the fact that the
differences between the tensor and central components of the Bonn A,
B and C potentials are negligible, irrespective of whether we use
the bare $G$--matrix or the renormalized LS interaction.

We present our results for the central component obtained with the decomposition
of eq.\ (\ref{eq:lsdeco}) for $T=0$ in fig.\ \ref{fig:cent}. The upper parts of this
figure shows the decomposition obtained with the bare $G$--matrix only, whereas the lower
part dipslays the LS results.

The results for the tensor component are presented in a similar 
way in fig.\ \ref{fig:tens}.

eon interaction,
From these figures we see that, although generally small, the differences between
the three potentials mainly show up in the central component (note the smaller
scale in fig.\ \ref{fig:cent}). There are indeed negligible differences among the
tensor components of the three potentials. This conclusion is rather dissapointing
if we expected the slightly different tensor strengths of the three NN potentials
to give markedly different tensor strengths in a nuclear medium. This is not
surprising however, in view of the quenching of the tensor force taking place 
in a nuclear medium, as discussed in section 2. 
On the other hand, the fact that the three potentials after all give slightly
different central components, does imply that it is useful to analyse an
effective interaction in terms of its central, spin--orbit and tensor 
components. Thus, it is hoped that such analyses of realistic effective interactions
would allow construction of phenomenological effective interactions
for heavier nuclei, where present many--body techniques cannot easily be applied.
A thorough spin--tensor analysis of realistic effective interactions in the $sd$--
and $pf$--shells is in progress
\cite{hhos92}.



\subsection{Calcium isotopes with more than two valence nucleons}
We conclude our discussion on the derived effective interactions by
employing the two--body matrix elements in the calculation of the eigenvalues
for nuclei with more than two valence nucleons. Here we limit our
attention to two isotopes, i.e. $^{44}Ca$
and $^{44}Sc$. The effective
interactions we use are those obtained by using the LS method with a
third--order $\hat{Q}$--box and by including  excitations up to
$6\hbar\omega$ in oscillator energy in the evaluation of the adhering
diagrams.

The reason we include $^{44}Sc$ in our discussion, is the fact that
for this nucleus with $T=1$, the $JT=10$ two--body matrix elements, discussed above
in connection with the increased binding provided by a potential
with a weak tensor force, come into play. For $^{44}Ca$, which has
isospin $T=2$, these matrix elements do not occur.
It is therefore instructive to compare the spectra obtained with
all three Bonn potentials for these two nuclei.

A comparison with the experimental data from \cite{endt90}
including the Coulomb corrections of Richter {\em et al.}
\cite{richt91} is also added.



Our results for the low--lying $^{44}Ca$ and $^{44}Sc$ spectra are shown in
figs.\ \ref{fig:ca44} and \ref{fig:sc44}, respectively. In these
figures we also include some of the second--excited states for a given
$JT$ combination, in order to understand whether some of these states
represent configurations not included in our model space. As is well known,
near closed shells the interplay between spherical and deformed
configurations is important. To obtain a complete
description of the low--lying spectra, states representing 
so--called intruder states should be included in the model space. Though,
it is a problem of its own to assess how good an approximation a spherical
basis is.

In fig.\ \ref{fig:ca44}
we note that all potentials yield results with only minor differences, as already
observed for $^{42}Ca$, with potential A yielding the largest binding energy.
We also
notice that none of the potentials reproduce properly
the second $0^+$, $2^+$ and $4^+$ states, indicating that these states
are candidates for intruder states. Actually, though the agreement with the
data for the spherical states is rather good, there is still enough room
for corrections for intruder states.


For $^{44}Sc$, we observe from fig.\ \ref{fig:sc44} that though the Bonn A
potential gives a result which is close to the experimental value, the
lowest--lying $J=1$ is too strongly bound. The problem we observed in the
discussion of the $JT=10$--channel for $^{42}Sc$ is also reflected
in the spectra of $^{44}Sc$. The correct ordering of the states is however
restored if we apply either the Bonn B or C potentials, lending support
to our suggestion that a potential with too weak a tensor force gives
$JT=10$ matrix elements which are too attractive.
For this nucleus as well we see that the second $3^+$ state is not reproduced
by the $pf$--model space. This state may be  an intruder state.
Again it is worth noting that the agreement with the data for the spherical
configurations is good, and for the above two nuclei, the agreement is better
than that reached by Richter {\em et al.} \cite{richt91}
through empirically derived
matrix elements. As mentioned previously, the latter authors ignored
admixtures from intruder states in their fitting procedure. Accounting
for such admixtures could then very well spoil their agreement
with the data.



\section{Conclusions}
In this work we have reviewed several perturbative 
many--body techniques which are appropriate
to reproduce an effective interaction in low energy nuclear physics. Moreover,
to calculate the reaction matrix $G$, three recent versions of the Bonn potentials
defined in ref.\ \cite{mac89}, have been used. We list below our main conclusions.
\begin{itemize}
\item  Third--order contributions to the effective interaction
 are important and can not be neglected.
\item The partial summation scheme of the folded diagrams introduced by
Lee and Suzuki \cite{ls80} yields a rather rapid convergence of the 
effective interaction.
\item The Bonn A potential, with the weakest tensor force, introduces too much
binding in the $JT=10$ channel, whereas for the remaining eigenstates, a good
reproduction of the data can be achieved insofar as one accounts for so--called
intruder states. It must however be remarked that the results we have 
presented have been obtained using a harmonic oscillator basis for the 
single--particle wave functions. With a BHF basis, potential A might turn out
to be the best candidate. 
\item Throughout the last three years, we have, together with collegues
at  T\"{u}bingen, Stony Brook, Barcelona  and Los Alamos, carried
out several investigations of the nuclear many--body problem, employing
conventional techniques, in order to assess the various features of the NN 
potential in the medium. At present we would like to conclude that the Bonn
meson--exchange potential models offer to date the most consistent starting
point for a nuclear structure calculation. There are however still many open
problems, such as the reproduction of the single--particle energies, or the 
simultaneous description of binding energy and charge radius for the 
ground state of light nuclei. Another problem which has been much discussed
recently is whether the masses of the nucleons and the mesons should be modified
in the medium, as advocated by the authors in ref.\ \cite{bmp90}. 
We feel tempted here to state that these effects are not yet 
fully understood, and that one should, as far as possible, 
investigate the conventional many--body
approach, before introducing new effects. 
\end{itemize}
\subsection*{Acknowledgments}
We are indebted to Lars Engvik for preparing the results displayed in fig.\ 2.

Stimulating
discussions with Profs.\ Paul Ellis,
Herbert M\"{u}ther, Ruprecht Machleidt and Ingvar Lindgren are greatly
acknowledged.

\clearpage
\section*{Appendix}
In table \ref{tab:lsnum} we list the $LS$--scheme
quantum numbers for isospin $T=0$ used in figs.\ \ref{fig:cent} and
\ref{fig:tens} in the evaluation of eq.\ (\ref{eq:lsdeco}).





%        bibliography

\clearpage
\footnotesize{\begin{thebibliography}{99}
\bibitem{eo77} P. J. Ellis and E. Osnes, Rev. Mod. Phys. {\bf 49} (1977) 777
\bibitem{kuo81} T. T. S. Kuo, Lecture Notes in
Physics; Topics in Nuclear Physics, eds. T. T. S. Kuo and S. S.
M. Wong  (Springer, Berlin, 1981) Vol. 144, p. 248
\bibitem{ko90} T. T. S. Kuo and E. Osnes, Folded-Diagram Theory
of the Effective Interaction in Atomic Nuclei, Springer Lecture
Notes in Physics (Springer, Berlin, 1990) Vol. 364
\bibitem{kb66} T. T. S. Kuo and G. E. Brown, Nucl. Phys. {\bf A85}
(1966) 40
\bibitem{bran67} B. H. Brandow, Rev. Mod. Phys. {\bf 39} (1967) 771
\bibitem{hom92} M. Hjorth-Jensen, E. Osnes and H. M\"{u}ther,
Ann. of Phys.  {\bf 213} (1992) 102
\bibitem{mac89} R. Machleidt, Adv. Nucl. Phys. {\bf 19} (1989) 189
\bibitem{heho92} M. Hjorth-Jensen, T. Engeland, A. Holt and
E. Osnes,
Nucl. Phys. {\bf A541} (1992) 105
\bibitem{br79} G. E Brown and M. Rho, Phys. Lett. {\bf B82} (1979) 177
\bibitem{zahed88} U.--G. Meissner and I. Zahed,
Adv. Nucl. Phys. {\bf 17} (1987) 143
\bibitem{mes79}  see e.g. Mesons in Nuclei, eds.\ M. Rho and D. H. Wilkinson
(North--Holland, Amsterdam, 1979)
\bibitem{mhe87} R. Machleidt, K. Holinde and C. Elster, Phys. Rep.
{\bf 149} (1987) 1
\bibitem{iz80} C. Itzykson and J.--B Zuber, Quantum Field Theory
(McGraw--Hill, New--York, 1980)
\bibitem{bj76}  G. E. Brown and A. D. Jackson, The Nucleon--Nucleon Interaction
(North--Holland, Amsterdam, 1976)
\bibitem{thomp70} R. H. Thompson, Phys. Rev. {\bf D1} (1970) 110
\bibitem{bd64} E. E. Salpeter and H. A. Bethe, Phys. Rev. {\bf 84} (1951) 1232
\bibitem{wj73}  R. Woloshyn and A. D. Jackson, Nucl. Phys. {\bf B64} (1973) 269
\bibitem{tjon81} M. J. Zuilhof and J. A. Tjon, Phys. Rev. {\bf C24} (1981) 736;
J. Fleischer and J. A. Tjon, Phys. Rev. {\bf C21} (1980) 87
\bibitem{bbs68} R. Blanckenbecler and R. Sugar, Phys. Rev. {\bf 142} (1966) 1051
\bibitem{ms92} H. M\"{u}ther and P. U. Sauer, preprint Univ. T\"{u}bingen
1991 and to
appear in Springer Lecture notes in Physics
\bibitem{tk72} S. F. Tsai and T. T. S. Kuo, Phys. Lett. {\bf B39} (1972) 427
\bibitem{kkko76} E. M. Krenciglowa, C. L. Kung, T. T. S. Kuo and
E. Osnes, Ann. of Phys. (NY) {\bf 101} (1976) 154
\bibitem{law80} R. D. Lawson, Theory of the Nuclear Shell Model
(Clarendon Press, Oxford, 1980) p. 208
\bibitem{mb90} R. Brockmann and R. Machleidt, Phys. Rev. {\bf C42} (1990) 1965
\bibitem{ms89} C. Mahaux and R. Sartor, Phys. Rev. {\bf C40} (1989) 1833
\bibitem{km83} T. T. S. Kuo and Z. Y. Ma, Phys. Lett. {\bf B127} (1983) 137
\bibitem{sw86} S. D. Serot and J. D. Walecka,
Adv. Nucl. Phys. {\bf 16} (1986) 1
\bibitem{br89} G. E. Brown and M. Rho, Phys. Lett. {\bf B222} (1989) 324
\bibitem{bmp90}G. E. Brown, H. M\"{u}ther and M. Prakash, Nucl.
Phys. {\bf A506} (1990) 565
\bibitem{ban92} M. K. Banerjee, Phys. Rev. {\bf C45} (1992) 1359
\bibitem{cs86} L. S. Celenza and C. Shakin, Relativistic Nuclear Physics
(World Scientific, Singapore, 1986) 
\bibitem{zzm92} D. C. Zheng, L. Zamick and H. M\"{u}ther,
Phys. Rev. {\bf C45} (1992) 275
\bibitem{lm85} I. Lindgren and J. Morrison, Atomic Many--Body Theory
(Springer, Berlin, 1985)
\bibitem{bloch59} C. Bloch, Nucl. Phys. {\bf 6} (1958) 329;
Nucl. Phys. {\bf 7} (1958) 451
\bibitem{klz76} H. K\"{u}mmel, K. H. L\"{u}hrman and J. G.
Zabolitzky, Phys. Rep. {\bf 36C} (1978) 1
\bibitem{ls80} S. Y. Lee and K. Suzuki, Prog. Theor. Phys.
{\bf 64} (1980) 2091
\bibitem{des} J. Des Cloizeaux, Nucl. Phys. {\bf 20} (1960) 321
\bibitem{sw72} T. H. Schucan and H. A. Weidenm\"{u}ller, Ann. of Phys. 
{\bf 73} (1972) 108; {\bf 76} (1973) 483
\bibitem{lk75} J. M. Leinaas and T. T. S. Kuo, Ann. of Phys.
{\bf 98} (1975) 177
\bibitem{lk77} J. M. Leinaas and T. T. S. Kuo, Ann. of Phys.
{\bf 111} (1977) 19
\bibitem{sm90} L. D. Skouras and H. M\"{u}ther, Nucl. Phys. {\bf A515}
(1990) 93
\bibitem{ee70} P. J. Ellis and T. Engeland, Nucl. Phys. {\bf A144} (1970) 161
\bibitem{skd83} J. Shurpin, T. T. S. Kuo and D. Strottman,
Nucl. Phys. {\bf A408} (1983) 310
\bibitem{prm90} A. Polls, A. Ramos and H. M\"{u}ther, Nucl. Phys. {\bf
A518} (1990) 421
\bibitem{kuo68} T. T. S. Kuo, Nucl. Phys. {\bf A90} (1967) 199
\bibitem{kb68} T. T. S. Kuo and G. E. Brown, Nucl. Phys. {\bf A114} (1968)
241
\bibitem{hk72} G. H. Herling and T. T. S. Kuo, US Naval Research Laboratory
report No. 2258, 1971 (unpublished)
\bibitem{brown88}B. A. Brown, W. A. Richter, R. E. Julies
and B. H. Wildenthal, Ann. of Phys. {\bf 182} (1988) 191
\bibitem{ryd90} L. Rydstr\"{o}m, J. Blomquist, R. J. Liotta and C. Pomar,
Nucl. Phys. {\bf A512} (1990) 217
\bibitem{wb91} E. K. Warburton and B. A. Brown,
Phys. Rev. {\bf C43} (1991) 602
\bibitem{os87}  E. Osnes, in Windsurfing the Fermi Sea Vol II, eds. T. T. S. Kuo
and J. Speth  (Elsevier Science Publishers, B. V. , 1987) p. 226
\bibitem{bk70} B. R. Barrett and M. W. Kirson, Nucl. Phys.
{\bf A148} (1970) 145
\bibitem{homs90} M. Hjorth-Jensen, E. Osnes, H. M\"{u}ther
and K. W. Schmid, Phys. Lett. {\bf B248} (1990) 281
\bibitem{homsk91}M. Hjorth-Jensen, E. Osnes, H. M\"{u}ther,
K. W. Schmid and T. T. S. Kuo, in  Understanding the Variety
of Nuclear Excitations, ed. A. Covello  (World Scientific, Singapore, 1991)
p. 87
\bibitem{reid68} R. V. Reid, Ann. of Phys. {\bf 50} (1968) 411
\bibitem{paris80} M. Lacombe {\em et al.} , Phys. Rev. {\bf C21} (1980)
861
\bibitem{clajus90} M. Clajus {\em et al.} , Phys. Lett. {\bf B245} (1990) 333
\bibitem{daen83} W. W. Daehnick, Phys. Rep. {\bf 96} (1983) 317
\bibitem{sommer} H. M. Sommerman, H. M\"{u}ther, K. C. Tam, T. T. S. Kuo
and A. Faessler, Phys. Rev. {\bf C23} (1981) 1765
\bibitem{mpk85} H. M\"{u}ther, A. Polls and T. T. S. Kuo, Nucl. Phys.
{\bf A435} (1985) 548
\bibitem{hok92} M. Hjorth--Jensen, E. Osnes and T. T. S. Kuo,
Nucl. Phys. {\bf A540} (1992) 145
\bibitem{os92} E. Osnes and D. Strottman, Phys. Rev. {\bf C45} (1992) 662
\bibitem{richt91} W. A. Richter, M. G. Van der Merwe, R. E. Julies
and B. A. Brown, Nucl. Phys. {\bf A523} (1991) 325
\bibitem{ht91} A. Hosaka and H. Toki, Nucl. Phys. {\bf A529} (1991) 429
\bibitem{hhos92} A. Holt, M. Hjorth--Jensen, E. Osnes and D. Strottman,
in preparation
\bibitem{endt90} P. M. Endt, Nucl. Phys. {\bf A521} (1990) 1

\end{thebibliography}}

\clearpage
\begin{table}[hbtp]
\caption{{\em The meson parameters which define the Bonn A, B and C
potentials of table A.2 from ref.\ [7].}}
\begin{center}
\begin{tabular}{cccccccc}
&&&&&&&\\
&&&&&&&\\ \hline
&&&&&&&\\
&&
\multicolumn{2}{c}{A}&
\multicolumn{2}{c}{B}&
\multicolumn{2}{c}{C}\\
&&&&&&&\\
\multicolumn{1}{c}{Meson}&
\multicolumn{1}{c}{mass(MeV)}&
\multicolumn{1}{c}{$g^{2}/4\pi$}&
\multicolumn{1}{c}{$\Lambda$(GeV)}&
\multicolumn{1}{c}{$g^{2}/4\pi$}&
\multicolumn{1}{c}{$\Lambda$(GeV)}&
\multicolumn{1}{c}{$g^{2}/4\pi$}&
\multicolumn{1}{c}{$\Lambda$(GeV)}
\\&&&&&&&\\  \hline&&&&&&&\\
$\pi$&138.03&14.9&1.05&14.6&1.2&14.6&1.3\\
$\eta$&548.8&7&1.5&5&1.5&3&1.5\\
$\rho$&769&0.99&1.3&0.95&1.3&0.95&1.3\\
$\omega$&782.6&20&1.5 &20&1.5&20&1.5\\
$\sigma$&550&8.3141&2.0&8.0769&2.0&8.0279&1.8\\
$\delta$&983&0.7707&2.0&3.1155&1.5&5.0742&1.5\\
 &&&&&&&\\ \hline &&&&&&&\\
\end{tabular}
\end{center} \label{tab:mespar}
\end{table}
\clearpage

\begin{table}[hbtp]
\begin{center}
\caption{{\em $JT=01$ FD $sd$-shell matrix elements with a third-order
$\hat{Q}$-box.
The terms $\tilde{F}_{i}$ represent the
contributions from fold $i$. All entries in MeV. Taken from ref.\ [6]. }}
\begin{tabular}{lllllrrrrrr}
&&&&&&&&&&\\
&&&&&&&&&&\\ \hline
&&&&&&&&&&\\
$JT$&$j_{a}$&$j_{b}$&$j_{c}$&$j_{d}$&\multicolumn{1}{c}
{$\tilde{F}_{0}$}&\multicolumn{1}{c}{$\tilde{F}_{1}$}&
\multicolumn{1}{c}{$\tilde{F}_{2}$}&\multicolumn{1}{c}
{$\tilde{F}_{3}$}&\multicolumn{1}{c}{$\tilde{F}_{4}$}&
\multicolumn{1}{c}{$V_{eff}$}\\
&&&&&&&&&&\\  \hline&&&&&&&&&&\\
01&$d_{5/2}$&$d_{5/2}$&$d_{5/2}$&$d_{5/2}$&-3.010&0.427&-0.095
&0.025&-0.013&-2.666\\
&$d_{5/2}$&$d_{5/2}$&$d_{3/2}$&$d_{3/2}$&-3.265&0.542&-0.110
&0.024&-0.010&-2.820\\
&$d_{5/2}$&$d_{5/2}$&$s_{1/2}$&$s_{1/2}$&-1.403&0.290&-0.092
&0.035&-0.019&-1.189\\
&$d_{3/2}$&$d_{3/2}$&$d_{5/2}$&$d_{5/2}$&-3.265&0.321&0.026
&-0.057&0.038&-2.937\\
&$d_{3/2}$&$d_{3/2}$&$d_{3/2}$&$d_{3/2}$&-1.309&0.047&0.052
&-0.048&0.030&-1.229\\
&$d_{3/2}$&$d_{3/2}$&$s_{1/2}$&$s_{1/2}$&-0.982&0.112&0.010
&-0.030&0.025&-0.865\\
&$s_{1/2}$&$s_{1/2}$&$d_{5/2}$&$d_{5/2}$&-1.403&0.646&-0.423
&0.303&-0.225&-1.103\\
&$s_{1/2}$&$s_{1/2}$&$d_{3/2}$&$d_{3/2}$&-0.982&0.472&-0.318
&0.230&-0.172&-0.769\\
&$s_{1/2}$&$s_{1/2}$&$s_{1/2}$&$s_{1/2}$&-2.910&1.108&-0.564
&0.343&-0.232&-2.255\\
&&&&&&&&&&\\
&&&&&&&&&&\\ \hline
\end{tabular}
\end{center} \label{tab:fdsd}
\end{table}
\clearpage

\begin{table}[hbtp]
\caption{{\em $JT=01$ Lee-Suzuki $sd$-shell matrix elements.
The terms $\tilde{R}_{i}$ represent the effective
interaction after $i$ iterations. All entries in
MeV. Taken from ref.\ [6]. }}
\begin{center}
\begin{tabular}{lllllrrrrr}
&&&&&&&&&\\
&&&&&&&&&\\ \hline
&&&&&&&&&\\
$JT$&$j_{a}$&$j_{b}$&$j_{c}$&$j_{d}$&\multicolumn{1}{c}
{$\tilde{R}_{0}$}&\multicolumn{1}{c}{$\tilde{R}_{1}$}&
\multicolumn{1}{c}{$\tilde{R}_{2}$}&\multicolumn{1}{c}
{$\tilde{R}_{3}$}&\multicolumn{1}{c}{$\tilde{R}_{4}$}\\
&&&&&&&&&\\
\hline&&&&&&&&&\\
01&$d_{5/2}$&$d_{5/2}$&$d_{5/2}$&$d_{5/2}$&
-3.010&-2.647&-2.649&-2.663&-2.659\\
&$d_{5/2}$&$d_{5/2}$&$d_{3/2}$&$d_{3/2}$&
-3.265&-2.810&-2.807&-2.818&-2.814\\
&$d_{5/2}$&$d_{5/2}$&$s_{1/2}$&$s_{1/2}$&
-1.403&-1.173&-1.176&-1.183&-1.181\\
&$d_{3/2}$&$d_{3/2}$&$d_{5/2}$&$d_{5/2}$&
-3.265&-2.974&-2.932&-2.955&-2.949\\
&$d_{3/2}$&$d_{3/2}$&$d_{3/2}$&$d_{3/2}$&
-1.309&-1.255&-1.225&-1.243&-1.238\\
&$d_{3/2}$&$d_{3/2}$&$s_{1/2}$&$s_{1/2}$&
-0.982&-0.885&-0.865&-0.878&-0.874\\
&$s_{1/2}$&$s_{1/2}$&$d_{5/2}$&$d_{5/2}$&
-1.403&-0.940&-1.018&-1.004&-1.006\\
&$s_{1/2}$&$s_{1/2}$&$d_{3/2}$&$d_{3/2}$&
-0.982&-0.645&-0.704&-0.693&-0.695\\
&$s_{1/2}$&$s_{1/2}$&$s_{1/2}$&$s_{1/2}$&
-2.919&-2.102&-2.166&-2.157&-2.159\\
&&&&&&&&&\\
&&&&&&&&&\\ \hline
\end{tabular}
\end{center} \label{tab:lssd}
\end{table}

\clearpage
\begin{table}[hbtp]
\caption{{\em $pf$-shell two-body contributions to the $\hat{Q}$-box
for selected configurations
obtained with the Bonn B potential.
$G$ denotes the bare $G$-matrix, $CP$
is the second-order core-polarization diagram
whereas $2nd$ means all two-body second-order diagrams.
$3rd$ means all third-order diagrams whereas $Total$ is the
sum of all two-body topologies
through third-order in $G$. All entries in MeV.}}
\begin{center}
\begin{tabular}{lllllrrrrr}
&&&&&&&&&\\
&&&&&&&&&\\ \hline
&&&&&&&&&\\
$JT$&$j_{a}$&$j_{b}$&$j_{c}$&$j_{d}$&
\multicolumn{1}{c}{$G$}&
\multicolumn{1}{c}{$CP$}&
\multicolumn{1}{c}{$2nd$}&
\multicolumn{1}{c}{$3rd$}&
\multicolumn{1}{c}{Total}
\\&&&&&&&&&\\  \hline&&&&&&&&&\\
01&$f_{7/2}$&$f_{7/2}$&$f_{7/2}$&$f_{7/2}$
&-1.005&-0.821&-1.396&0.008&-2.393\\
10&&&&&-0.334&-0.431&-0.772&-0.245&-1.351\\
01&$f_{7/2}$&$f_{7/2}$&$f_{5/2}$&$f_{5/2}$
&-2.264&-0.295&-0.620&0.717&-2.167\\
10&&&&&1.407&-0.610&-0.524&-0.020&0.863\\
01&$f_{7/2}$&$f_{7/2}$&$p_{3/2}$&$p_{3/2}$
&-0.510&-0.301&-0.534&-0.023&-1.067\\
10&&&&&-0.095&-0.267&-0.410&-0.168&-0.074\\
01&$f_{7/2}$&$f_{7/2}$&$p_{1/2}$&$p_{1/2}$
&-0.530&-0.153&-0.278&0.165&-0.643\\
10&&&&&0.119&0.069&0.058&-0.056&0.120\\
01&$f_{5/2}$&$f_{5/2}$&$f_{5/2}$&$f_{5/2}$
&-0.351&-0.576&-1.057&-0.141&-1.548\\
10&&&&&-0.135&-0.233&-0.396&-0.435&-0.965\\
01&$f_{5/2}$&$f_{5/2}$&$p_{3/2}$&$p_{3/2}$
&-0.603&-0.120&-0.284&0.211&-0.676\\
10&&&&&0.011&-0.090&0.022&0.018&0.051\\
01&$f_{5/2}$&$f_{5/2}$&$p_{1/2}$&$p_{1/2}$
&-0.231&-0.171&-0.333&0.035&-0.530\\
10&&&&&-0.045&-0.115&-0.162&-0.064&-0.271\\
01&$p_{3/2}$&$p_{3/2}$&$p_{3/2}$&$p_{3/2}$
&-1.027&-0.159&-0.335&-0.370&-1.732\\
10&&&&&-0.575&-0.076&-0.265&-0.244&-1.084\\
01&$p_{3/2}$&$p_{3/2}$&$p_{1/2}$&$p_{1/2}$
&-1.310&-0.139&-0.245&-0.130&-1.684\\
10&&&&&0.670&0.009&0.015&0.318&1.030\\
01&$p_{1/2}$&$p_{1/2}$&$p_{1/2}$&$p_{1/2}$
&-0.100&-0.073&-0.174&-0.193&-0.468\\
10&&&&&-0.823&-0.140&-0.380&-0.545&-1.747\\
&&&&&&&&&\\
&&&&&&&&&\\ \hline
\end{tabular}
\end{center}
\label{tab:diags}
\end{table}

\clearpage
\begin{table}[hbtp]
\caption{{\em Eigenvalues for the $T=1$ and $T=0$ systems
obtained with the
Lee-Suzuki method after four iterations.
The $\hat{Q}$-box has been
approximated with the inclusion of all non-folded
linked-valence diagrams through second and third order in the interaction.
The eigenvalues are listed for excitations up to $6\hbar\omega$
in oscillator energy. All entries
in MeV.}}
\begin{center}
\begin{tabular}{crrrrrrr}
\hline
&&&&&&&\\
&\multicolumn{3}{c}{2nd-order}&
\multicolumn{3}{c}{3rd-order}&
\multicolumn{1}{c}{Expt}\\
$JT$&
\multicolumn{1}{c}{Bonn A}&
\multicolumn{1}{c}{Bonn B}&
\multicolumn{1}{c}{Bonn C}&
\multicolumn{1}{c}{Bonn A}&
\multicolumn{1}{c}{Bonn B}&
\multicolumn{1}{c}{Bonn C}&
\\&&&&&&&\\  \hline&&&&&&&\\
$01$&-2.42&-2.35&-2.26&-2.49&-2.43&-2.39&-3.12\\
$10$&-2.13&-1.77&-1.47&-2.65&-2.15&-1.75&-2.57\\
$21$&-0.99&-0.95&-0.92&-1.06&-1.02&-0.99&-1.60\\
$30$&-1.05&-0.90&-0.77&-1.32&-1.10&-0.94&-1.69\\
$41$&-0.26&-0.26&-0.26&-0.21&-0.22&-0.23&-0.37\\
$50$&-1.08&-0.98&-0.90&-1.29&-1.15&-1.05&-1.67\\
$61$&0.08&0.06&0.05&0.09&0.07&0.05&0.07\\
$70$&-1.97&-1.87&-1.80&-2.37&-2.21&-2.10&-2.56\\
&&&&&&&\\
&&&&&&&\\ \hline
\end{tabular}
\end{center}
\label{tab:casc}
\end{table}

\clearpage




\begin{table}[hbtp]
\begin{center}
\caption{{\em The quantum numbers of the matrix elements in the $LS$--scheme
for isospin $T = 0$.
The numbers in the
two rightmost columns refer to the numbering in figs.\ 13 and 14
for the central ($N_C$) and tensor ($N_T$) components.}}
\begin{tabular}{ccclrcclrlrl}
&&&&&&&&&&&\\
\hline
&&&&&&&&&&&\\
 $l_{\alpha}$ & $l_{\beta}$ & $l_{\gamma}$ & $l_{\delta}$\hspace{.5cm} &
\hspace{.5cm}$L$ & $L'$ & $S$ & $S'$\hspace{.5cm} &
\hspace{.5cm}$J'$ & $T$\hspace{.5cm} & \hspace{.5cm}$N_C$ &
$N_T$\hspace{.5cm} \\
&&&&&&&&&&&\\
\hline
&&&&&&&&&&&\\
 3 & 3 & 3 & 3 & 1 & 1 & 0 & 0 & 1 & 0 & 1 &  \\
 3 & 3 & 3 & 3 & 3 & 3 & 0 & 0 & 3 & 0 & 2 &  \\
 3 & 3 & 3 & 3 & 5 & 5 & 0 & 0 & 5 & 0 & 3 &  \\
 3 & 3 & 3 & 1 & 3 & 3 & 0 & 0 & 3 & 0 & 4 &  \\
 3 & 3 & 1 & 1 & 1 & 1 & 0 & 0 & 1 & 0 & 5 &  \\
 3 & 1 & 3 & 1 & 2 & 2 & 0 & 0 & 2 & 0 & 6 &  \\
 3 & 1 & 3 & 1 & 3 & 3 & 0 & 0 & 3 & 0 & 7 &  \\
 3 & 1 & 3 & 1 & 4 & 4 & 0 & 0 & 4 & 0 & 8 &  \\
 1 & 1 & 1 & 1 & 1 & 1 & 0 & 0 & 1 & 0 & 9 &  \\
 3 & 3 & 3 & 3 & 0 & 0 & 1 & 1 & 1 & 0 & 10 &  \\
 3 & 3 & 3 & 3 & 2 & 0 & 1 & 1 & 1 & 0 & & 1 \\
 3 & 3 & 3 & 3 & 2 & 2 & 1 & 1 & 3 & 0 & 11 & 2 \\
 3 & 3 & 3 & 3 & 4 & 2 & 1 & 1 & 3 & 0 & & 3 \\
 3 & 3 & 3 & 3 & 4 & 4 & 1 & 1 & 5 & 0 & 12 & 4 \\
 3 & 3 & 3 & 3 & 6 & 4 & 1 & 1 & 5 & 0 & & 5 \\
 3 & 3 & 3 & 3 & 6 & 6 & 1 & 1 & 7 & 0 & 13 & 6 \\
 3 & 3 & 3 & 1 & 0 & 2 & 1 & 1 & 1 & 0 & & 7 \\
 3 & 3 & 3 & 1 & 2 & 2 & 1 & 1 & 3 & 0 & 14 & 8 \\
 3 & 3 & 3 & 1 & 2 & 3 & 1 & 1 & 3 & 0 & & 9 \\
&&&&&&&&&&&\\
\hline
\end{tabular}
\end{center}
\label{tab:lsnum}
\end{table}
\clearpage
\begin{table}
\begin{center}
\begin{tabular}{cccl rccl rlrl}
\hline
&&&&&&&&&&&\\
$l_{\alpha}$ & $l_{\beta}$ & $l_{\gamma}$ & $l_{\delta}$\hspace{.5cm} &
\hspace{.5cm}$L$ & $L'$ & $S$ & $S'$\hspace{.5cm} &
\hspace{.5cm}$J'$ & $T$\hspace{.5cm} & \hspace{.5cm}$N_C$ &
$N_T$\hspace{.5cm} \\
&&&&&&&&&&&\\
\hline
&&&&&&&&&&&\\
 3 & 3 & 3 & 1 & 2 & 4 & 1 & 1 & 3 & 0 & & 10 \\
 3 & 3 & 3 & 1 & 4 & 2 & 1 & 1 & 3 & 0 & & 11 \\
 3 & 3 & 3 & 1 & 4 & 3 & 1 & 1 & 4 & 0 & & 12 \\
 3 & 3 & 3 & 1 & 4 & 4 & 1 & 1 & 5 & 0 & 15 & 13 \\
 3 & 3 & 3 & 1 & 6 & 4 & 1 & 1 & 5 & 0 & & 14 \\
 3 & 3 & 1 & 1 & 0 & 0 & 1 & 1 & 1 & 0 & 16 &  \\
 3 & 3 & 1 & 1 & 0 & 2 & 1 & 1 & 1 & 0 & & 15 \\
 3 & 3 & 1 & 1 & 2 & 0 & 1 & 1 & 1 & 0 & & 16 \\
 3 & 3 & 1 & 1 & 2 & 2 & 1 & 1 & 3 & 0 & 17 & 17 \\
 3 & 3 & 1 & 1 & 4 & 2 & 1 & 1 & 3 & 0 & & 18 \\
 3 & 1 & 3 & 1 & 2 & 2 & 1 & 1 & 3 & 0 & 18 & 19 \\
 3 & 1 & 3 & 1 & 3 & 2 & 1 & 1 & 3 & 0 & & 20 \\
 3 & 1 & 3 & 1 & 3 & 3 & 1 & 1 & 4 & 0 & 19 & 21 \\
 3 & 1 & 3 & 1 & 4 & 2 & 1 & 1 & 3 & 0 & & 22 \\
 3 & 1 & 3 & 1 & 4 & 3 & 1 & 1 & 4 & 0 & & 23 \\
 3 & 1 & 3 & 1 & 4 & 4 & 1 & 1 & 5 & 0 & 20 & 24 \\
 3 & 1 & 1 & 1 & 2 & 0 & 1 & 1 & 1 & 0 & & 25 \\
 3 & 1 & 1 & 1 & 2 & 2 & 1 & 1 & 3 & 0 & 21 & 26 \\
 3 & 1 & 1 & 1 & 3 & 2 & 1 & 1 & 3 & 0 & & 27 \\
 3 & 1 & 1 & 1 & 4 & 2 & 1 & 1 & 3 & 0 & & 28 \\
 1 & 1 & 1 & 1 & 0 & 0 & 1 & 1 & 1 & 0 & 22 &  \\
 1 & 1 & 1 & 1 & 2 & 0 & 1 & 1 & 1 & 0 & & 29 \\
 1 & 1 & 1 & 1 & 2 & 2 & 1 & 1 & 3 & 0 & 23 & 30 \\
&&&&&&&&&&&\\
\hline
&&&&&&&&&&&\\
\multicolumn{12}{c}{Table 6 --continued}\\
\end{tabular}
\end{center}
\end{table}



\clearpage
\begin{figure}[hbtp]
\vspace{15cm}
\caption{{\em Diagrammatical representation of the nuclear reaction
matrix $G$. Due to the non--perturbative character of the bare interaction $V$,
the particles must interact virtually with each other an arbitrary number of times
in order to obtain a finite reaction matrix $G$ which is suitable for
a perturbative treatment.}}\label{fig:gmat}
\end{figure}
\clearpage
\begin{figure}[hbpt]
\vspace{15cm}
\caption{{\em Nuclear matter average potential energy for the partial wave
$^{3}S_1$ obtained with the Bonn A, B and C potentials as functions of
the Fermi momentum $k_F$.}}
\label{fig:bhfnm}
\end{figure}
\clearpage
\begin{figure}[hbtp]
\vspace{15cm}
\caption{{\em The $G$--matrix for
$\bra{(0s_{1/2})^{2}JT=10}G\ket{(0s_{1/2})^{2}JT=10}$ obtained
with the Bonn A, B and C potentials described in the text as a function
of the
starting energy $\omega$. The quenching of the matrix elements
is then only due to the dependence
of the starting energy, so--called dispersive effects.}}
\label{fig:0s0s}
\end{figure}
\clearpage
\begin{figure}[hbtp]
\vspace{15cm}
\caption{{\em Different types of valence--linked diagrams. Diagram (a)
is irreducible and connected, (b) is reducible, while (c) is irreducible
and disconnected.}}
\label{fig:diags}
\end{figure}
\clearpage
\begin{figure}[hbtp]
\vspace{15cm}
\caption{{\em The one--body diagrams through second--order
without a spectator valence line.
Diagram (a) is  the Hartree--Fock
term. Diagram (b) is the auxiliary potential $U$,
while (c) and (d) are the 2p1h and 3p2h
diagrams, respectively.}}
\label{fig:onebody}
\end{figure}
\clearpage
\begin{figure}[hbtp]
\vspace{15cm}
\caption{{\em The corresponding two--body diagrams.
Diagram (a) is the $G$--matrix. Diagram (b) is the core--polarization
diagram
while (c) and (d) are the 2p ladder and the
4p2h diagrams, respectively.}}
\label{fig:twobody}
\end{figure}
\clearpage

\begin{figure}[hbtp]
\vspace{15cm}
\caption{{\em Significant third--order contributions to the effective interaction.}}
\label{fig:third}
\end{figure}
\clearpage
\begin{figure}[hbtp]
\vspace{15cm}
\caption{{\em Theoretical and experimental low--lying spectrum for $^{18}O$
obtained with the Bonn A potential defined in table A.1 of ref. [7], using
both a HO basis and a BHF basis. The terms $H_{eff}^{(1)}$, $H_{eff}^{(2)}$
and $H_{eff}^{(3)}$ denote the effective interaction through first, second
and third order in Rayleigh--Schr\"{o}dinger (RS) perturbation theory. All
energies in MeV. Taken from ref.\ [53]. }}
\label{fig:hobhf}
\end{figure}
\clearpage
\begin{figure}[hbtp]
\vspace{15cm}
\caption{{\em The $^{18}O$ spectra obtained with a third--order $\hat{Q}$--box
using the LS method. All energies
in MeV. Taken from ref.\ [6]. }}
\label{fig:lsconv}
\end{figure}
\clearpage
\begin{figure}[hbtp]
\vspace{15cm}
\caption{{\em The $^{18}F$ spectra obtained with a third--order $\hat{Q}$--box
using the LS method. All energies
in MeV. Taken from ref.\ [6]. }}
\label{fig:f18}
\end{figure}

\clearpage
\begin{figure}[hbtp]
\setlength{\unitlength}{1.0mm}
\begin{picture}(130,55)(0,-50)
\thicklines
\put(5,-50){\line(0,1){55}}
\multiput(5,-30)(0,10){3}{\line(1,0){2}}
\thinlines
\put(0,-31){-3}
\put(0,-21){-2}
\put(0,-11){-1}
\put(1,-1){0}
\multiput(5,-35)(0,10){4}{\line(1,0){1}}
\put(10,-15){\nl{0}}
\put(10,-8.5){\nl{2}}
\put(10,-4.3){\line(1,0){10}\raisebox{-1.2ex}{\hspace{1mm}4}}
\put(10,-2.3){\nl{6}}
\put(30,-34.1){\nl{0}}
\put(30,-13.8){\nl{2}}
\put(30,-3.9){\nl{4}}
\put(30,0.5){\nl{6}}
\put(50,-15){\nl{0}}
\put(50,-8.5){\nl{2}}
\put(50,-4.3){\line(1,0){10}\raisebox{-1.2ex}{\hspace{1mm}4}}
\put(50,-2.3){\nl{6}}
\put(70,-33.4){\nl{0}}
\put(70,-13.4){\nl{2}}
\put(70,-3.9){\nl{4}}
\put(70,0.05){\nl{6}}
\put(90,-15){\nl{0}}
\put(90,-8.5){\nl{2}}
\put(90,-4.3){\line(1,0){10}\raisebox{-1.2ex}{\hspace{1mm}4}}
\put(90,-2.3){\nl{6}}
\put(110,-32.2){\nl{0}}
\put(110,-13){\nl{2}}
\put(110,-3.8){\nl{4}}
\put(110,0.03){\nl{6}}
\put(130,-31.2){\nl{0}}
\put(130,-16){\nl{2}}
\put(130,-3.7){\nl{4}}
\put(130,0.7){\nl{6}}
\put(12,-42){$H_{eff}^{(1)}$}
\put(32,-42){$H_{eff}^{(2)}$}
\put(52,-42){$H_{eff}^{(1)}$}
\put(72,-42){$H_{eff}^{(2)}$}
\put(92,-42){$H_{eff}^{(1)}$}
\put(112,-42){$H_{eff}^{(2)}$}
\put(130,-42){Expt}
\put(12.5,-45){\line(1,0){28}}
\put(52.5,-45){\line(1,0){28}}
\put(92.5,-45){\line(1,0){28}}
\put(20,-50){Bonn A}
\put(60,-50){Bonn B}
\put(100,-50){Bonn C}
\end{picture}
\caption{{\em The spectra for $^{42}Ca$ obtained with the bare $G$--matrix
($H_{eff}^{(1)}$) and
Rayleigh--Schr\"{o}dinger perturbation theory through second order in the $G$--matrix
($H_{eff}^{(2)}$). All three potentials have been employed. Energies in MeV.}}
\label{fig:secondca}
\end{figure}

\clearpage
\begin{figure}[hbtp]
\setlength{\unitlength}{1.0mm}
\begin{picture}(130,50)(0,-50)
\thicklines
\put(5,-50){\line(0,1){45}}
\multiput(5,-30)(0,10){3}{\line(1,0){2}}
\thinlines
\put(0,-31){-3}
\put(0,-21){-2}
\put(0,-11){-1}
\multiput(5,-35)(0,10){3}{\line(1,0){1}}
\put(10,-16.5){\line(1,0){10}\raisebox{-1.5ex}{\hspace{1mm}1}}
\put(10,-10.5){\nl{3}}
\put(10,-13.1){\line(1,0){10}\raisebox{-1.5ex}{\hspace{1mm}5}}
\put(10,-22.9){\nl{7}}
\put(30,-35){\nl{1}}
\put(30,-18){\line(1,0){10}\raisebox{-1.5ex}{\hspace{1mm}3}}
\put(30,-16.2){\nl{5}}
\put(30,-24.5){\nl{7}}
\put(50,-13.4){\line(1,0){10}\raisebox{-1.5ex}{\hspace{1mm}1}}
\put(50,-9.0){\nl{3}}
\put(50,-12){\nl{5}}
\put(50,-21.8){\nl{7}}
\put(70,-30.3){\nl{1}}
\put(70,-15.9){\line(1,0){10}\raisebox{-2.0ex}{\hspace{1mm}3}}
\put(70,-15.1){\nl{5}}
\put(70,-23.5){\nl{7}}
\put(90,-10.7){\line(1,0){10}\raisebox{-0.1ex}{\hspace{1mm}1}}
\put(90,-7.8){\line(1,0){10}\raisebox{-0.3ex}{\hspace{1mm}3}}
\put(90,-11.1){\line(1,0){10}\raisebox{-1.5ex}{\hspace{1mm}5}}
\put(90,-21){\nl{7}}
\put(110,-26){\nl{1}}
\put(110,-14.2){\line(1,0){10}\raisebox{0.4ex}{\hspace{1mm}3}}
\put(110,-14.6){\line(1,0){10}\raisebox{-1.5ex}{\hspace{1mm}5}}
\put(110,-22.7){\nl{7}}
\put(130,-25.7){\line(1,0){10}\raisebox{-1.8ex}{\hspace{1mm}1}}
\put(130,-16.9){\line(1,0){10}\raisebox{-1.8ex}{\hspace{1mm}3}}
\put(130,-16.7){\line(1,0){10}\raisebox{0.3ex}{\hspace{1mm}5}}
\put(130,-25.6){\line(1,0){10}\raisebox{0.3ex}{\hspace{1mm}7}}
\put(12,-42){$H_{eff}^{(1)}$}
\put(32,-42){$H_{eff}^{(2)}$}
\put(52,-42){$H_{eff}^{(1)}$}
\put(72,-42){$H_{eff}^{(2)}$}
\put(92,-42){$H_{eff}^{(1)}$}
\put(112,-42){$H_{eff}^{(2)}$}
\put(130,-42){Expt}
\put(12.5,-45){\line(1,0){28}}
\put(52.5,-45){\line(1,0){28}}
\put(92.5,-45){\line(1,0){28}}
\put(20,-50){Bonn A}
\put(60,-50){Bonn B}
\put(100,-50){Bonn C}
\end{picture}
\caption{{\em The spectra for $^{42}Sc$. Notations as in the previous figure.}}
\label{fig:secondsc}
\end{figure}

\clearpage
\begin{figure}[hbtp]
\vspace{15cm}
\caption{{\em Spin--tensor decomposition of the effective interaction for the central
component. The upper part shows the results obtained with the bare $G$--matrix for
the Bonn A, B and C potentials. The lower figure exhibits the results derived from
the LS effective interaction discussed in the text. The numbering of the matrix
elements follows the table in the appendix.}}
\label{fig:cent}
\end{figure}
\clearpage

\begin{figure}[hbtp]
\vspace{15cm}
\caption{{\em Spin--tensor decomposition of the effective interaction for the tensor
component. Notations as in the previous figure.}}
\label{fig:tens}
\end{figure}
\clearpage


%               44-ca       scale by 20
\begin{figure}[hbtp]
\setlength{\unitlength}{1.0mm}
\begin{center}
\begin{picture}(140,90)(0,-20)
\thicklines
\put(5,-20){\line(0,1){90}}
\multiput(5,0)(0,20){4}{\line(1,0){2}}
\thinlines
\put(1,0){0}
\put(1,19){1}
\put(1,39){2}
\put(1,59){3}
\multiput(5,10)(0,20){3}{\line(1,0){1}}
\put(10,0){\nl{0}}
\put(10,25.2){\nl{2}}
\put(10,41.6){\nl{4}}
\put(10,52.2){\nl{4}}
\put(10,57.2){\line(1,0){10}\raisebox{-1.0ex}{\hspace{1mm}6}}
\put(10,59.6){\nl{2}}
\put(10,62.4){\nl{2}}
\put(40,0){\nl{0}}
\put(40,25.2){\nl{2}}
\put(40,41){\nl{4}}
\put(40,49.8){\nl{4}}
\put(40,54.4){\nl{6}}
\put(40,58.2){\nl{2}}
\put(40,62){\nl{2}}
\put(70,0){\nl{0}}
\put(70,24.2){\nl{2}}
\put(70,39.4){\nl{4}}
\put(70,47){\nl{4}}
\put(70,51.2){\nl{6}}
\put(70,55.6){\nl{2}}
\put(70,61){\nl{2}}
\put(100,0){\nl{0}}
\put(100,22.4){\nl{2}}
\put(100,37.6){\nl{0}}
\put(100,45.6){\nl{4}}
\put(100,50.2){\nl{2}}
\put(100,60){\nl{4}}
\put(100,65.8){\line(1,0){10}\raisebox{-1.8ex}{\hspace{1mm}6}}
\put(100,66){\line(1,0){10}\raisebox{0.5ex}{\hspace{1mm}2}}
\put(9,-7){\small{(--38.03)}}
\put(39,-7){\small{(--37.94)}}
\put(69,-7){\small{(--37.82)}}
\put(99,-7){\small{(--38.91)}}
\put(15,-15){A}
\put(45,-15){B}
\put(75,-15){C}
\put(101,-15){Expt}
\end{picture}
\end{center}
\caption{{\em The low-lying  $^{44}Ca$ spectra relative to $^{40}Ca$.
The labels A, B, and C refer to the  various versions of the Bonn potential
discussed in the text. See text for further explanation.
All energies in MeV.}}
\label{fig:ca44}
\end{figure}
\clearpage


%               44-sc   scale by 40

\begin{figure}[hbtp]
\setlength{\unitlength}{1.0mm}
\begin{center}
\begin{picture}(140,100)(0,-20)
\thicklines
\put(5,-20){\line(0,1){100}}
\multiput(5,0)(0,40){2}{\line(1,0){2}}
\thinlines
\put(1,0){0}
\put(1,39){1}
\multiput(5,20)(0,40){2}{\line(1,0){1}}
\put(10,0){\nl{2}}
\put(10,17.2){\line(1,0){10}\raisebox{-1.5ex}{\hspace{1mm}1}}
\put(10,19.6){\nl{4}}
\put(10,21.2){\line(1,0){10}\raisebox{0.4ex}{\hspace{1mm}6}}
\put(10,28.4){\nl{3}}
\put(10,50.4){\line(1,0){10}\raisebox{-1.5ex}{\hspace{1mm}3}}
\put(10,43.2){\nl{5}}
\put(10,51.6){\line(1,0){10}\raisebox{0.5ex}{\hspace{1mm}7}}
\put(10,68.8){\nl{5}}
\put(40,0){\nl{2}}
\put(40,20.8){\nl{1}}
\put(40,16.8){\nl{4}}
\put(40,16){\line(1,0){10}\raisebox{-2.0ex}{\hspace{1mm}6}}
\put(40,27.2){\nl{3}}
\put(40,49.2){\nl{3}}
\put(40,39.6){\nl{5}}
\put(40,43.2){\nl{7}}
\put(40,63.2){\nl{5}}
\put(70,0){\nl{2}}
\put(70,23.2){\line(1,0){10}\raisebox{-1.0ex}{\hspace{1mm}1}}
\put(70,14.4){\line(1,0){10}\raisebox{0.3ex}{\hspace{1mm}4}}
\put(70,12.4){\line(1,0){10}\raisebox{-1.0ex}{\hspace{1mm}6}}
\put(70,25.2){\line(1,0){10}\raisebox{0.3ex}{\hspace{1mm}3}}
\put(70,47.6){\nl{3}}
\put(70,36){\line(1,0){10}\raisebox{-1.0ex}{\hspace{1mm}5}}
\put(70,36.8){\line(1,0){10}\raisebox{0.3ex}{\hspace{1mm}7}}
\put(70,58.8){\nl{5}}
\put(100,0){\nl{2}}
\put(100,10.8){\nl{6}}
\put(100,14){\nl{4}}
\put(100,26.8){\nl{1}}
\put(100,30.4){\nl{3}}
\put(100,38.8){\line(1,0){10}\raisebox{-1.5ex}{\hspace{1mm}7}}
\put(100,39.6){\line(1,0){10}\raisebox{-0.1ex}{\hspace{1mm}3}}
\put(100,42){\line(1,0){10}\raisebox{0.2ex}{\hspace{1mm}5}}
\put(100,47.6){\nl{3}}
\put(100,55.6){\nl{3}}
\put(100,61.2){\nl{5}}
\put(9,-7){\small{(--41.50)}}
\put(39,-7){\small{(--40.87)}}
\put(69,-7){\small{(--40.38)}}
\put(99,-7){\small{(--41.70)}}
\put(15,-15){A}
\put(45,-15){B}
\put(75,-15){C}
\put(101,-15){Expt}
\end{picture}
\end{center}
\caption{{\em The low-lying  $^{44}Sc$ spectra relative to $^{40}Ca$.
Legend as in the previous figure.}}
\label{fig:sc44}
\end{figure}



\end{document}






\documentstyle[a4wide,12pt,epsf]{article}

\newcommand{\bra}[1]{\left\langle #1 \right|}
\newcommand{\ket}[1]{\left| #1 \right\rangle}

\begin{document}

\pagestyle{empty}


\section{Introduction}
\clearpage
\section{Perturbative many-body methods}
{\Large
\[
        P=\sum_{i=1}^{D} \left|\Phi_i\right\rangle
        \left\langle\Phi_i\right |
\]\vspace{1cm}

\[
        Q=\sum_{i=D+1}^{\infty} \left|\Phi_i\right\rangle
        \left\langle\Phi_i\right |
\]\vspace{1cm}
\[
                H\left|\Psi_{\alpha}\right\rangle= 
                E_{\alpha}\left|\Psi_{\alpha}\right\rangle
\]\vspace{1cm}
\[
               PH_{\mathrm{eff}}P\left|\Psi_{\alpha}\right\rangle=
               E_{\alpha}P\left|\Psi_{\alpha}\right\rangle=
              E_{\alpha}\left|\Phi_{\alpha}\right\rangle
\]\vspace{1cm}
\[
               PH_{\mathrm{eff}}P=PH_1P +PH_1\frac{Q}{e}H_1 P+
               PH_1\frac{Q}{e}H_1 \frac{Q}{e}H_1 P+\dots
               \label{eq:effint}
\]\vspace{1cm}
\[
                \left|\Psi_{\alpha}\right\rangle=
                \left|\Phi_{\alpha}\right\rangle+
                \frac{Q}{e}H_1\left|\Phi_{\alpha}\right\rangle
                +\frac{Q}{e}H_1\frac{Q}{e}H_1\left|\Phi_{\alpha}\right\rangle
                +\dots
                \label{eq:wavef}
\]
\clearpage
\[
              \Omega = 1 +\chi,
\]\vspace{1cm}
\[
               P\chi P = 0, \hspace{1cm} Q\Omega P = 
              Q\chi P =\chi P. \label{eq:chi1}
\]\vspace{1cm}
\[
                   \chi = Q\chi P. \label{eq:chi2}
\]\vspace{1cm}

\[
              \Omega = 1 +\Omega^{(1)} + \Omega^{(2)}+\dots 
\]\vspace{1cm}
\begin{eqnarray}
         \Omega\left|\Phi_i\right\rangle=
         &{\displaystyle\left|\Phi_i\right\rangle
         +\sum_{\alpha}\frac{\left|\alpha\right\rangle
         \left\langle\alpha\right|
          V\left|\Phi_i\right\rangle}{\varepsilon_i -\varepsilon_{\alpha}}
         +\sum_{\alpha\beta}\frac{\left|\alpha\right\rangle
        \left\langle\alpha\right| V
         \left|\beta\right\rangle\left\langle\beta\right| V
         \left|\Phi_i\right\rangle }
         {(\varepsilon_i-\varepsilon_{\alpha})
       (\varepsilon_i-\varepsilon_{\beta})} }\nonumber\\
&       {\displaystyle  -\sum_{\alpha j}\frac{\left|\alpha\right\rangle
       \left\langle\alpha\right|
         V\left|\Phi_j\right\rangle
        \left\langle\Phi_j\right| V\left|\Phi_i\right\rangle}
       {(\varepsilon_i-\varepsilon_{\alpha})
      (\varepsilon_i-\varepsilon_{j})} }
       +\dots \nonumber
\end{eqnarray}
\begin{figure}[hbtp]
      \setlength{\unitlength}{1mm}
      \begin{picture}(100,100)
      \put(25,0){\epsfxsize=14cm \epsfbox{vertex.eps}}
      \end{picture}
      \caption{The various vertices to first order in the interaction
               $V$ which contribute to the wave operator
               $\Omega =1+\chi$. Hartree-Fock
               terms are not included. Possible hermitian conjugate 
                diagrams are also not shown. Indicated are also possible
               angular momentum coupling orders.}
\end{figure}
\clearpage

\begin{figure}[hbtp]
      \setlength{\unitlength}{1mm}
      \begin{picture}(100,100)
      \put(25,0){\epsfxsize=10cm \epsfbox{recouple.eps}}
      \end{picture}
      \caption{Coupling order for the $[12]$ (a), $[13]$ (b) and
               $[14]$ (c) channels.}
      \label{fig:channelsdef}
\end{figure}
\[
       V_{1234J}^{[12]}
       =\left\langle (12)J\right | V
       \left | (34)J\right\rangle
       \label{eq:12channel}
\] 

\[
      V_{1234J}^{[13]}=
      {\displaystyle \sum_{J'}}(-)^{j_1+j_4+J+J'}\hat{J'}^2
      \left\{
      \begin{array}{ccc}
       j_3&j_1&J\\j_2&j_4&J'
      \end{array}
       \right\}V_{1234J}^{[12]}
       \label{eq:13channel}
\]

\[
       V_{1234J}^{[14]}=
      {\displaystyle \sum_{J'}}(-)^{j_1+j_4+J+2j_3}\hat{J'}^2
      \left\{
      \begin{array}{ccc}
       j_4&j_1&J\\j_2&j_3&J'
      \end{array}
       \right\}
       V_{1234J}^{[12]}
       \label{eq:14channel}
\]
\[
       V_{1234J}^{[14]}=
      {\displaystyle \sum_{J'}}(-)^{2j_1+2j_2+2j_3}\hat{J'}^2
      \left\{
      \begin{array}{ccc}
       j_4&j_1&J\\j_3&j_2&J'
      \end{array}
       \right\}
       V_{1234J}^{[13]}
       \label{eq:1413channel}
\]
\clearpage

\section{Summation of diagrams in the $[12]$ channel}




\[
     \Gamma^{[12]}_{1234J}=\Gamma^{[12]}_{2143J}=-\Gamma^{[12]}_{2134J}=
     \Gamma^{[12]}_{1243J}
     \label{eq:symproperties}
\]\vspace{1cm}
\[
    s=\varepsilon_1+\varepsilon_2=\varepsilon_3+\varepsilon_4
    \label{eq:energy12}
\]\vspace{1cm}
\[
     \Gamma^{[12]}=V^{[12]}+V^{[12]}(gg)\Gamma^{[12]}
     \label{eq:schematic12}
\]\vspace{1cm}



\[
    \hat{{\cal G}}^{[12]}=
    \frac{Q^{[12]}_{\mathrm{pp}}}{s-\varepsilon_5-\varepsilon_6+\imath \eta}-
    \frac{Q^{[12]}_{\mathrm{hh}}}{s-\varepsilon_5-\varepsilon_6-\imath \eta}
    \label{eq:paulioperator12}
\]
\clearpage
\[
      \Gamma^{[12]}_{1234J}(s) = 
      V^{[12]}_{1234J}+\frac{1}{2}
      \sum_{56}
      V^{[12]}_{1256J}\hat{{\cal G}}^{[12]}
      \Gamma^{[12]}_{5634J}(s)
      \label{eq:first12}
\]
\begin{figure}[hbtp]
      \setlength{\unitlength}{1mm}
      \begin{picture}(100,100)
      \put(25,0){\epsfxsize=10cm \epsfbox{sigmagamma.eps}}
      \end{picture}
      \caption{(a) represents the two-body vertex $\Gamma$ function while (b)  
               represents the self-energy $\Sigma$.}
      \label{fig:selfcons12}
\end{figure}


\clearpage

\begin{figure}[hbtp]
      \setlength{\unitlength}{1mm}
      \begin{picture}(100,100)
      \put(25,0){\epsfxsize=10cm \epsfbox{channel12.eps}}
      \end{picture}
      \caption{Diagrams (a)-(d) give examples of 
               diagrams which are summed up by  
               pp-hh equations.
               Diagrams (e) and (f) are examples of core-polarization
               terms which are not generated by the $[12]$ channel.}
      \label{fig:gamma12}
\end{figure}
\[
      (a)=\frac{1}{2}\sum_{pq}V^{[12]}_{12pq J}
      \frac{1}{s-\varepsilon_p-
                \varepsilon_q} V^{[12]}_{pq34 J}
      \label{eq:secondg}
\]

\[
      (b)=\frac{1}{2}\sum_{\alpha\beta}V^{[12]}_{12\alpha\beta J}
      \frac{1}{-s+\varepsilon_{\alpha}+
                \varepsilon_{\beta}} V^{[12]}_{\alpha\beta 34 J}
\]
\[
      (c)=\frac{1}{4}\sum_{pqrw}V^{[12]}_{12pq J}
      \frac{1}{s-\varepsilon_p-
                \varepsilon_q} 
      V^{[12]}_{pqrw J}
      \frac{1}{s-\varepsilon_r-
                \varepsilon_w} 
       V^{[12]}_{rw34 J}
      \label{eq:thirdg}
\]
\[
      (d)=\frac{1}{4}\sum_{\alpha\beta pq}V^{[12]}_{12pq J}
      \frac{1}{s-\varepsilon_p-
                \varepsilon_q+\varepsilon_{\alpha}+
                \varepsilon_{\beta}} 
      V^{[12]}_{pq\alpha\beta J}
      \frac{1}{s-\varepsilon_p-
                \varepsilon_q} 
       V^{[12]}_{\alpha\beta 34 J}
      \label{eq:thirdg2h}
\]
\clearpage
\section{Screening corrections and vertex renormalization, the equations
for the $[13]$ and  $[14]$ channels}
\label{sec:sec4}


\[
     \Gamma^{[13]}=V^{[13]}+V^{[13]}(gg)\Gamma^{[13]}
\]\vspace{1cm}
\[
     \Gamma^{[14]}=V^{[14]}+V^{[14]}(gg)\Gamma^{[14]}
\]\vspace{1cm}

\[ 
     t=\varepsilon_3-\varepsilon_1=\varepsilon_2-\varepsilon_4
\]  \vspace{1cm}  
\[
     u=\varepsilon_1-\varepsilon_4=\varepsilon_3-\varepsilon_2
\]\vspace{1cm}

\[
    \hat{{\cal G}}^{[13]}=
    \frac{Q^{[13]}_{\mathrm{ph}}}{t-\varepsilon_p+\varepsilon_h+\imath \eta}
    -\frac{Q^{[13]}_{\mathrm{hp}}}{t+\varepsilon_p-\varepsilon_h-\imath \eta}
    \label{eq:paulioperator13}
\]\vspace{1cm}

\[
    \hat{{\cal G}}^{[14]}=
    \frac{Q^{[14]}_{\mathrm{ph}}}{u-\varepsilon_p+\varepsilon_h+\imath \eta}
    -\frac{Q^{[14]}_{\mathrm{hp}}}{u+\varepsilon_p-\varepsilon_h-\imath \eta}
    \label{eq:paulioperator14}
\]

\clearpage
\[
      \Gamma^{[13]}_{1234J}(t) = 
      V^{[13]}_{1234J}+
      \sum_{ph}
      V^{[13]}_{12phJ}\hat{{\cal G}}^{[13]}
      \Gamma^{[13]}_{ph34J}(t)
      \label{eq:first13}
\]

\[
      \Gamma^{[14]}_{1234J}(u) = 
      V^{[14]}_{1234J}-
      \sum_{ph}
      V^{[14]}_{12phJ}\hat{{\cal G}}^{[14]}
      \Gamma^{[14]}_{ph34J}(u)
      \label{eq:first14}
\]

\begin{figure}[hbtp]
      \setlength{\unitlength}{1mm}
      \begin{picture}(100,100)
      \put(25,0){\epsfxsize=10cm \epsfbox{ch1314.eps}}
      \end{picture}
      \caption{(a) shows the structure of the integral equation for 
               the interaction vertex in the $[13]$ channel. (b) represents
               the integral channel for the $[14]$ channel.
               The coupling order is displayed as well.}
      \label{fig:figs1314}
\end{figure}

\clearpage

\begin{figure}[hbtp]
      \setlength{\unitlength}{1mm}
      \begin{picture}(100,100)
      \put(25,0){\epsfxsize=14cm \epsfbox{secph.eps}}
      \end{picture}
      \caption{Second-order perturbation theory corrections to the ph
               interaction vertex.}
      \label{fig:phvertex}
\end{figure}
\begin{figure}[hbtp]
      \setlength{\unitlength}{1mm}
      \begin{picture}(100,100)
      \put(25,0){\epsfxsize=14cm \epsfbox{thirdph.eps}}
      \end{picture}
      \caption{Corrections beyond second order in the interaction $V$ 
               to the ph
               interaction vertex. (a) is in the $[14]$ channel and (b) is in
               $[13]$ channel.}
      \label{fig:phhigher}
\end{figure}
\begin{figure}[hbtp]
      \setlength{\unitlength}{1mm}
      \begin{picture}(100,100)
      \put(25,0){\epsfxsize=14cm \epsfbox{ph22.eps}}
      \end{picture}
       \caption{Corrections to second order in $V$ of the 2p2h
               vertex.}
       \label{fig:pphhvertex}
\end{figure}
\begin{figure}[hbtp]
      \setlength{\unitlength}{1mm}
      \begin{picture}(100,100)
      \put(25,0){\epsfxsize=14cm \epsfbox{ph21.eps}}
      \end{picture}
       \caption{The corrections to second order in $V$ of the 2p1h
               vertex.}
       \label{fig:2p1hvertex}
\end{figure}
\begin{figure}[hbtp]
      \setlength{\unitlength}{1mm}
      \begin{picture}(100,100)
      \put(25,0){\epsfxsize=14cm \epsfbox{kirson.eps}}
      \end{picture}
       \caption{Examples of diagrams which can arise from the [13] and [14]
                self-consistent equations.}
       \label{fig:kirsoniterate}
\end{figure}

\clearpage 
\section{Effective interactions for finite nuclei 
         from Parquet diagrams}
\label{sec:sec5} 



\[ 
      \Gamma= \Gamma^{[ij]}+\Gamma^{[ij]}\hat{{\cal G}}^{[ij]}\Gamma,
      \label{eq:generalchannel}
\]\vspace{1cm}

\[
    L=\Gamma^{[12]}-G_F
\]\vspace{1cm}

\[
    L=\Delta G =G-G_F
\]\vspace{1cm}

\[
    L=G_F\hat{{\cal G}}^{[12]}G_F+G_F\hat{{\cal G}}^{[12]}L
    \label{eq:ladder}
\]\vspace{1cm}

\[
    R^{[13]}=\Gamma^{[13]}-G_F
\]\vspace{1cm}
and 
\[
    R^{[14]}=\Gamma^{[14]}-G_F
\]\vspace{1cm}

\[
    R^{[ij]}=G_F\hat{{\cal G}}^{[ij]}G_F+G_F\hat{{\cal G}}^{[ij]}R^{[ij]}
    \label{eq:ring}
\]\vspace{1cm}

\[
    \Gamma=G_F+L+R^{[13]}+R^{[14]}
     \label{eq:gammap}
\]\vspace{1cm}

\[
    L=\left(G_F+R^{[13]}+R^{[14]}\right)\hat{{\cal G}}^{[12]}\left(G_F+R^{[13]}+R^{[14]}\right)\]\vspace{1cm}
\[
      +\left(G_F+R^{[13]}+R^{[14]}\right)\hat{{\cal G}}^{[12]}L
    \label{eq:laddernext}
\]\vspace{1cm}

\[
    R^{[13]}=\left(G_F+L+R^{[14]}\right)\hat{{\cal G}}^{[13]}
             \left(G_F+L+R^{[14]}\right) + \]\vspace{1cm}
\[
             \left(G_F+L+R^{[14]}\right)\hat{{\cal G}}^{[13]}R^{[14]}
    \label{eq:ring13next}
\]\vspace{1cm}
and
\[
    R^{[14]}=\left(G_F+L+R^{[13]}\right)\hat{{\cal G}}^{[14]}
             \left(G_F+L+R^{[13]}\right) + \]\vspace{1cm}
\[
             \left(G_F+L+R^{[13]}\right)\hat{{\cal G}}^{[14]}R^{[14]}
    \label{eq:ring14next}
\]\vspace{1cm}

}




\end{document}











Caro Umberto eccoti il mio riassunto.
Ho anche allegato un riassunto da parte di Oeystein Elgaroy, che
e' un studente al dottorato qui in Oslo, ma che incomincera' come
post-doc a Nordita l'anno prossimo. E' un ragazzo bravissimo,
e per il dottorato che sta per finalizzare ha gia' 12 articoli,
3 PRL, 1 PRA, 1 PRD, 4 PRC,  2 NPA ed 1 proceeding,
in due anni solo!

Il mio collega Torgeir Engeland, con cui ho' fatto calcoli
shell model, dai quali ricaviamo il pairing gap da per esempio
gli isotopi dello stagno, ti spedira' un contributo la prossima settimana
sinche' e' via per un convegno in questa settimana.

ciao e cari saluti,
Martino


-------------------------------------------------
 Morten Hjorth-Jensen
 Department of Physics, University of Oslo
 P.O.B. 1048 Blindern, N-0316 OSLO, Norway
 tlf. +47-22856432 or +47-22856428 (secretariat)
 fax  +47-22856422, e-mail: m.h.jensen@fys.uio.no
 Office: FV123
 http://www.uio.no/~mhjensen
-------------------------------------------------
 
*******   il mio riassunto (abstract)

\documentstyle[aps, preprint]{revtex}

\begin{document}

\title{Pairing and self-consistent effective interactions}

\author{M.\ Hjorth-Jensen}

\address{Department of Physics, University of Oslo, N-0316 Oslo, Norway\\
         e-mail: mhjensen@fys.uio.no}

\maketitle

\abstract{
Various perturbative and non-perturbative many-body techniques
are discussed. Especially, we will focus on the summation
of so-called Parquet diagrams with emphasis on applications 
for finite nuclei and infinite nuclear matter.
Here, the subset of two-body Parquet equations will be discussed. 
These equations allow for a self-consistent evaluation of both
ladder diagrams and polarization effects.
A 
practical implementation of the corresponding equations
for studies of effective interactions is outlined with applications
to pairing problems in infinite matter.  
Summation of three-body diagrams will  also discussed.}




\end{document}




***** il riassunto di Elgaroy


\documentstyle[aps, preprint]{revtex}

\begin{document}

\title{Microscopic structure of a vortex line in superfluid neutron star matter }

\author{\O.\ Elgar\o y}

\address{Department of Physics, University of Oslo, N-0316 Oslo, Norway\\
         e-mail: oelgaroy@fys.uio.no}

\maketitle

\abstract{
The microscopic structure of an isolated vortex line in superfluid neutron star matter is studied by solving the
       Bogoliubov-de Gennes equations. Our calculation, which is the starting point for a microscopic calculation of pinning
       forces in neutron stars, shows that the size of the vortex core varies differently with density, and is in general
       smaller than assumed in some earlier calculations of vortex pinning in neutron star crusts. The implications of this
       result are discussed 
}



\end{document}




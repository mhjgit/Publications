
In this section we briefly sketch how to evaluate an effective
interaction appropriate for the 
low-lying states for nuclei in the Sn region, within the framework
of degenerate Rayleigh-Schr\"{o}dinger perturbation theory. Central
here is the summation to all orders of the so-called folded diagrams,
which arise due to the removal of the dependence of the exact energy 
in perturbation theory. A major function of the folded diagrams
is to account for the fact that a valence particle, is not
in the single-particle state $i$ all the time.

As discussed in the previous section,
one is only interested in solving eq.\ (1)
for certain
low-lying states. We then divided the Hilbert space
in a model space defined by the operator $P$
and $Q$ in eq.\ (2).
The assumption then is that the components of these low-lying
states can be fairly well reproduced by configurations consisting
of few particles and holes occupying physically selected orbits.
These selected orbits then define the model space.
Eq.\ (1)  was then be rewritten as the secular equation given in eq.\ (3),
where $H_{eff}$  now is an effective hamiltonian acting solely
within the chosen model space.
Our scheme to obtain an effective interaction appropriate for e.g.\
nuclei in the mass region of tin  can be divided into three steps,
outlined below\footnote{A  more detailed
exposition can be found in refs.\ \cite{ko90,hko94}.}. 

\subsection{The nucleon-nucleon potential}

First, one needs a free NN interaction $V$ which is
appropriate for nuclear physics at low and intermediate energies. 
Since quantum chromodynamics (QCD) is commonly
accepted as the theory of the strong interaction, the
NN interaction $V$ should be  completely determined by the underlying
quark-quark dynamics in QCD. However, due to the nonperturbative
character of QCD at low energies, one is still far from a
quantitative understanding of the NN interaction from the QCD point of
view. This problem is circumvented by introducing
models containing some of the properties of QCD, such as
confinement and chiral symmetry breaking. One of the most used models
is the so-called bag model, where a crucial question is the
size of the radius ($R$) of the confining bag. If the size of the bag radius
is chosen as in the "Little bag" ($R\leq 0.5$ fm) \cite{br79}, then
low-energy nuclear physics phenomena can be fairly well described
in terms of hadrons like nucleons, isobars and various mesons,
which are to be understood as effective descriptions of complicated
multiquark interactions.
However, other models which
seek to approximate QCD also indicate that an effective theory in terms
of hadronic degrees of freedom may very well be the most appropriate
picture for low energy nuclear physics.
Although there is no unique prescription for how to construct
a free NN interaction, a description
of the NN interaction in terms of various meson exchanges is presently
the most quantitative representation of the NN interaction
\cite{mac89}
in the energy regime of nuclear physics.

Motivated by these arguments we employ in this work a model for the
NN interaction based on the exchange of six non-strange mesons, including
the pseudo-scalar $\pi$ and $\eta$ mesons, the vector $\rho$ and $\omega$
mesons and the scalar $\delta$ and $\sigma$ mesons. The latter is
a fictitious meson used in one-boson-exchange (OBE) models to provide in an
effective way the intermediate
range attraction of the potential. This attraction
originates from the $2\pi$ exchange processes.
Realistic calculations \cite{mac89}
which include $2\pi$ exchange differ only negligibly from calculations
with OBE models, justifying thereby the introduction of the effective
$\sigma$ meson.
The NN potential V is then a superposition of OBE interactions $V_i$
and is represented as
\[
V=\sum_{i=\pi\eta\rho\omega\delta\sigma}V_i.
\]
The OBE interaction $V_i$, is then constructed using
standard Lagrangian expressions.
To evaluate the OBE amplitudes we use in this work the
Bonn potential as it is defined by the meson parameters of the Bonn A potential
in table
A.2 of ref.\ \cite{mac89}. The reason why we favor the Bonn meson-exchange 
potential models, is that  they  all exhibit a weak tensor force,
as compared to previous models for the NN potential.
A weak tensor force introduces more binding in systems like nuclear matter
and finite nuclei, yielding a better agreement with the 
data compared with earlier calculations. 
Actually, results obtained with the Bonn potentials
for several nuclear systems \cite{mac89,hko94} may indicate that these potential
models allow for a more consistent description of nuclear structure than more
traditional nuclear-force models. The Bonn A potential is chosen in this
work, since it is the potential model which gives the best agreement 
with the data \cite{hko94}.


\subsection{The nuclear reaction matrix $G$}

Secondly, in nuclear many-body calculations, the first problem one is
confronted with is the fact that the repulsive core of the NN potential $V$
is unsuitable for perturbative approaches. This problem is overcome
by introducing the reaction matrix $G$ given by the solution of the
Bethe-Goldstone equation
\begin{equation}
    G=V+V\frac{Q}{\omega - H_0}G,
\end{equation}
where $\omega$ is the unperturbed energy of the interacting nucleons, 
and $H_0$ is the unperturbed hamiltonian. 
The operator $Q$, 
is again the projection operator which prevents the
interacting nucleons from scattering into states occupied by other nucleons.
Note that the definition of $Q$ in eq.\ (2) may differ from that 
used in the calculations of the $G$-matrix. If this is the case, then
certain diagrams have to be included or excluded from the calculation
of the effective interaction in eq.\ (3). For further details,
see e.g.\ ref.\ \cite{hko94}.

The equation for the $G$-matrix is solved in this work using the
double-partitioning scheme discussed in ref.\ \cite{kkko76}. 

\subsection{Perturbative many-body approaches}

Finally, we briefly sketch how to calculate an effective interaction
appropriate for nuclei like $^{102}$Sn in terms of the $G$-matrix. 
In this work we define the model space to consist of
the orbitals in fig.\ 1.
The first step here is to define the so-called $\hat{Q}$-box given by
\begin{equation}
   P\hat{Q}P=PH_1P+
   P\left(H_1 \frac{Q}{\omega-H_{0}}H_1+H_1
   \frac{Q}{\omega-H_{0}}H_1 \frac{Q}{\omega-H_{0}}H_1 +\dots\right)P,
   \label{eq:qbox}
\end{equation}
where we will replace $H_1$ with $G-U$ ($G$ replaces the free NN interaction
$V$).
The $\hat{Q}$-box\footnote{The $\hat{Q}$-box should not be confused
with the exclusion operator $Q$.} 
is made up of non-folded diagrams which are irreducible
and valence linked. A diagram is said to be irreducible if between each pair
of vertices there is at least one hole state or a particle state outside
the model space. In a valence-linked diagram the interactions are linked
(via fermion lines) to at least one valence line. Note that a valence-linked
diagram can be either connected (consisting of a single piece) or
disconnected. In the final expansion including folded diagrams as well, the
disconnected diagrams are found to cancel out \cite{ko90}. 
This corresponds to the cancellation of unlinked diagrams
of the Goldstone expansion \cite{ko90}.
We illustrate
these definitions by the diagrams shown in fig.\ 
\ref{fig:diagsexam}, where an arrow pointing upwards
(downwards) is a particle (hole) state.
\begin{figure}[hbtp]
   \setlength{\unitlength}{1mm}
   \begin{picture}(100,50)
   \end{picture}
\caption{Different types of valence-linked diagrams. Diagram (a)
is irreducible and connected, (b) is reducible, while (c) is irreducible
and disconnected.}
\label{fig:diagsexam}
\end{figure}
Particle states outside the model space are given by railed lines. 
Diagram (a) is irreducible, valence linked and connected, 
while (b) is reducible since 
the intermediate particle states belong to the model space. 
Diagram (c) is irreducible, valence linked and disconnected. 

We can then obtain an effective interaction 
$H_{\mathrm{eff}}=H_0+V_{\mathrm{eff}}$ in terms of the $\hat{Q}$-box,
with \cite{ko90,hko94}
\begin{equation}
    V_{eff}^{(n)}=\hat{Q}+{\displaystyle\sum_{m=1}^{\infty}}
    \frac{1}{m!}\frac{d^m\hat{Q}}{d\omega^m}\left\{
    V_{eff}^{(n-1)}\right\}^m . 
    \label{eq:fd}
\end{equation}
Observe also that the
effective interaction $V_{eff}^{(n)}$ 
is evaluated at a given model space energy
$\omega$, as is the case for the $G$-matrix as well. We choose this
starting energy to be $-10$ MeV. 
Moreover, although $\hat{Q}$ and its derivatives contain disconnected
diagrams, such diagrams cancel exactly in each order \cite{ko90}, thus
yielding a fully connected expansion in eq.\ (\ref{eq:fd}).
The first iteration is then given by
\begin{equation}
   V_{eff}^{(0)}=\hat{Q}.
\end{equation}
In this work we define the $\hat{Q}$-box to consist of all two-body
diagrams\footnote{With two-body we also mean one-body diagrams like 
diagram (a) in fig.\ \ref{fig:qbox} with a spectator valence line.}
through third order in the $G$-matrix, as shown in
Ref.\ \cite{hko94}. Typical examples of diagrams which are included
in the $\hat{Q}$-box are shown in fig.\ \ref{fig:qbox}. The summations
over intermediate states is restricted to excitations 
of $2\hbar\Omega$ (\hbar\Omega$ is the oscillator energy) 
in oscillator energy, an approximation which has
been found to be approriate if one uses a potential with a weak
tensor force like the Bonn A potential. The reader should note that
diagram (c) is this figure is not included in the $\hat{Q}$-box. Three-body
and other many-body effective contributions may be of importance 
in the study of spectra of nuclei with more than two valence
nuclei. These contributions will be studied by us in future works.
\begin{figure}[hbtp]
   \setlength{\unitlength}{1mm}
   \begin{picture}(100,50)
   \end{picture}
\caption{Examples of diagrams included in the $\hat{Q}$-box. 
Diagram (a) is a one-body diagram, whereas diagram (b) is a two-body
diagram. Diagram (c) is an effective three-body diagram which is not
included in our definition of the $\hat{Q}$-box.}
\label{fig:qbox}
\end{figure}
Another iterative scheme which has been much favored in the literature
is a method proposed by Lee and Suzuki (LS) \cite{ls80}. The effective interaction
we will employ in this work have been obtained using the LS method, which
gives the following expression for the effective interaction 
\begin{equation}
V_{eff}^{(n)}=\left[1-\hat{Q}_{1}-\sum_{m=2}^{n}\hat{Q}_{m}
\prod_{k=n-m+1}^{n-1}V_{eff}^{(k)}\right]^{-1}\hat{Q},
\end{equation}
where
\begin{equation}
\hat{Q}_{m}=\frac{1}{m!}\frac{d^m\hat{Q}}{d\omega^m}.
\end{equation}

To define the one-body part of the
effective two-body interaction, we use normally the experimental
single-particle energies. These are however not known for $^{101}$Sn, 
and must therefore be estimated from the spectra of odd tin isotopes.
The determination of the single-particle energies is discussed 
in section 4.

We end this section with some calculational details regarding
the effective interaction appropriate for the tin isotopes.
In this work we solve the Bethe-Goldstone equation for the $G$-matrix 
for five starting
energies $\Omega$, by way of the so-called double-partitioning scheme
defined in Ref.\ \cite{kkko76}. The Pauli operator is constructed so as to
prevent scattering into intermediate states with a single
nucleon in any of the
states defined by the orbitals from the $0s$ state, the states 
in the $0p$-shell, the 
$1s0d$-shell and the $1p0f$-shell, in addition we also have the $0g_{9/2}$
orbit. Moreover, the Pauli operator prevents scattering into intermediate
states with two particles in the orbits of the $2s1d0g$-shell (except the $0g_{9/2}$
orbit) and the $2p1f0h$-shell.
A harmonic-oscillator basis was chosen for the
single-particle wave functions, with an oscillator energy $\hbar\Omega$ given
by
$\hbar\Omega = 45A^{-1/3} - 25A^{-2/3}= 8.5~MeV$,  $A=100$ being the mass
number.





\bibitem{br79} G.E.\ Brown and M.\ Rho, Phys.\ Lett.\ {\bf B82} (1979) 177

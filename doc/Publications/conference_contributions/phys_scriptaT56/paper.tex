%	\documentstyle[11pt,epsf,a4]{article}
%%
\documentstyle[11pt,epsf,pstricks,pst-plot,a4]{article}
\newcommand{\be}{\begin{equation}}
\newcommand{\ee}{\end{equation}}
\newcommand{\bra}[1]{\left\langle #1 \right|}
\newcommand{\ket}[1]{\left| #1 \right\rangle}
\newcommand{\braket}[2]{\left\langle #1 \right| #2 \right\rangle}
\newcommand{\OP}[1]{{\bf\widehat{#1}}}
\newcommand{\matr}[1]{{\bf \cal{#1}}}
%%%%%%%%%%%%%%%%
%
\title{Large Shell Model Calculations with Realistic  Effective Interaction}
\author{T. Engeland, M. Hjorth-Jensen, A. Holt and E. Osnes\\[2ex]
Department of Physics, University of Oslo, N-0316 Oslo, Norway}
%

\begin{document}

%
\maketitle
%
\begin{abstract}
In this work we discuss methods to calculate shell model
effective interaction, starting from the basic nucleon--nucleon
interaction. The result is applied in large shell model calculations
using the Lanczos iterative method with m--scheme basic states.
Perspectives for such large--scale shell model calculations with
realistic effective interactions in heavy nuclei are discussed.
Results are shown for the Sn isotopes.
\end{abstract}
%


%%%%%%%%%%%%%%%%%%
%
%%%%%%%%%%%%%%%% file: intro.tex %%%%%%%%%%%%%%%%%%%
%
\section{Introduction.}
%
One of the fundamental and yet unsolved problems in nuclear theory
is to describe the properties of complex nuclei in terms of
the constituent nucleons and the interaction between them.
Firstly, this represents a quantal many--body problem which can only
be approximately solved. Secondly, the basic nucleon--nucleon
interaction is not well known. Thus, it may be difficult to
know whether an eventual failure to solve this problem should be
ascribed to the many--body methods or the interaction model used.

Ideally, one should derive the free nucleon--nucleon interaction
from the interaction between quarks. Although attempted,
this program has not been quantitatively successful. Thus, one would
be  content to start from a nucleon--nucleon interaction
derived from  meson models which reproduces relevant
two--nucleon data. Examples of such potentials are the Paris\cite{lac80},
and Bonn\cite{mac89} potentials.
 Once the basic nucleon--nucleon
interaction has been established, it should be employed in a
quantal many--body approximation to the actual nuclear structure
problem of interest. Here we shall be concerned  with the spherical
nuclear shell model which has provided a successfull microscopic
approach.

%
The non-relativistic shell model problem for
a nucleus $_Z^{A}X_N$ with $Z$ protons and $N$ neutrons
amounts to solving the Schr\"{o}dinger equation
%
\be
H\ket{\Psi_m(A)} = E_m \ket{\Psi_m(A)},
\label{theor1}
\ee
%
with $H=T+V$, $T$ being the kinetic energy operator and $V$
the nucleon--nucleon (NN) potential.
$E_m$ and $\Psi_m$ are the eigenvalues and eigenfunctions
for a state $m$ in the Hilbert space.
Introducing the auxiliary single--particle potential $U$, $H$ can
be rewritten as
%
\be
    \begin{array}{ccc}H=H_{0}+H_1, &H_{0}=T+U, &H_1=V-U\end{array}.
\ee
%
If $U$ is chosen such that $H_1$ becomes small, then $H_1$
can be treated as a perturbation.

The first step in this  $A$--nucleon problem is to reduce
the infinitely many degrees of freedom of the Hibert space
to those represented by a physically motivated subspace.
In shell model calculations one usually refers to spherical closed shells
and considers the valence particles or holes
as the relevant model space degrees of freedom.
More formally, one  defines projection operators
%
\be
P = \sum_{i = 1}^n \ket{\psi_i} \bra{\psi_i} ,
\;\;\;
Q = \sum_{i = n+1}^{\infty} \ket{\psi_i} \bra{\psi_i}.
\label{theor2}
\ee
%
The operator $P$ projects onto the chosen reduced shell model space
(or P--space), $\ket{\psi_i}$ being the basic functions spanning
this space and $n$ the dimension of the space.
The operator $Q$ defines the complement of $P$, $Q = {\bf 1} - P$,
which projects out of the shell model
space. The basic shell model problem is then to obtain an effective
Hamiltonian $H_{eff}$ acting in the P--space and defined by
%
\be
PH_{eff}P\ket{\Psi_m}
= P\left (H_0 + H_1 \right ) P \ket{\Psi_m}
= E_m P\ket{\Psi_m},
\label{theor3}
\ee
%
where again $E_m$ is the m-th eigenvalue and $\ket{\Psi_m}$ the m-th
eigenstate of the original Hamiltonian $H$ in eq.~(\ref{theor1}).
Thus the eigenfunctions of $H_{eff}$ are projections of the true
eigenfunctions of $H$ onto the P--space.
The $H_{eff}$ includes a one--body term $PH_0P$ which represents
the mean--field degrees of freedom of the chosen P--space
and an effective iteraction $PH_1P$  acting between the nucleons
in the P-space.

In eq.~(\ref{theor3}) we face two separate problems. Firstly, we
need to find a relevant expression for $H_1$. Secondly, with
a proper effective interaction between nucleons in the P--space
eq.~(\ref{theor3}) must be solved. In actual cases
the dimension $n$ is not small and the problem to find
eigenvalues and eigenvectors of $PH_{eff}P$ requires
efficient  numerical  methods.
As a typical example, a shell model calculation
for $^{106}$Sn with six valence neutrons
in the P--space (discussed later in this work, see fig.~\ref{res-fig4})
results in a reduced Hilbert space
of dimension $n = 31124$. Heavier isotopes gives $n > 10^{6}$.

Many alternative
approximate solutions to the above two basic problems
can be found in the
literature\cite{vall89}. A typical approach to the question of
$H_{eff}$ is a parametrization followed by a
fitting procedure to the data of interest. In a  work by
R. D. Lawson and J. M. Soper\cite{law65}, they analyzed
a fictitious set of nuclei named pseudonium
and showed how dangerous such a fitting procedure can be
if one wants to obtain  reasonable realistic wave functions
for a nuclear system. An alternative is to calculate
$H_{eff}$ from the basic interaction between nucleons.
There are basically two main approaches in perturbation theory
used to define the effective interation  $H_{eff}$.
One of these is the energy--dependent Brillouin--Wigner(BW)
perturbation theory while the Rayleigh--Schr\"{o}dinger (RS)
perturbation expansion stands for the energy independent approach.
In our work we calculate the effective interaction within the framework
of the time--independent RS perturbation theory and within this
theory we deal with the open--shell many--body problem for a
degenerate model space. Up to now this technique has only
been applied to light nuclei, in the p-, sd- and pf-shells.
We have now generated a computer code which can take these
calculations into medium and heavy nuclei.  At present we have
a restriction of $Z = N$ for the core system, a limitation
which will be eliminated shortly.

The next problem is to solve  eq.~(\ref{theor3}) with a
given $H_{eff}$. That has also produced many
alternative solutions since one  in many nuclei is faced
with an enormous number of degrees of freedom even with a very restricted
choice of P--space. In this context group theoretical methods have been
developed as a mean to pick out the relevant physical
properties of the nuclear system.
On the other hand, there has been made enormous
progress in modern computer technology and this tendency will continue.
We therefore feel that it is an important challenge
to use modern hardware technology coupled to efficient
numerical methods developed for  other fields of science
and technology in nuclear physics structure problems.

To sum up our basic philosophy,
we want to analyze nuclear systems based on eq.~(\ref{theor3})
with a microscopic realistic effective interaction
and perform a  test
on nuclei with many degrees of freedom in the light, medium
and heavy mass regions. The purpose will be to thoroughly test
the theory of the effective interaction which up to now mainly
has been applied to two--particle P--space systems.
The question of the importance of effective three--particle interaction
is also an important problem in this context and reasonably
easy to implement in our shell model code,
although it is not dealt with in the present work.

In sec.~2 we analyze the effective interaction and sec.~3 contains
a discussion of numerical methods which are available
for solving the many--particle
Schr\"{o}dinger equation in a large basis. As a first application
we present in sec.~4 some results for the Sn isotopes and sum up
the situation and discuss further possibilities in sec.5.
%%%%%%%%%%%%%%%%%%%%%%
%
%%%%%%%%%%%%%%% file: proceed.tex %%%%%%%%
%
\section{Effective Interaction.}
%
In this section we briefly sketch how to evaluate an effective
interaction appropriate for the 
low--lying states in a nuclear system, within the framework
of degenerate Rayleigh--Schr\"{o}dinger perturbation theory. Central
here is the summation to all orders of the so--called folded diagrams,
which arise due to the removal of the dependence of the exact energy
in perturbation theory. A major function of the folded diagrams
is to account for the fact that a valence particle is not
in a given single--particle state  all the time.

As discussed in the previous section,
one is only interested in solving eq.~(\ref{theor1})
for certain
low--lying states. We then divided the Hilbert space
in a model space defined by the operators $P$
and $Q$ in eq.~(\ref{theor2}).
The assumption then is that the components of these low--lying
states can be fairly well reproduced by configurations consisting
of few particles and holes occupying physically selected orbits.
These selected orbits then define the model space.
Eq.~(\ref{theor1})  is then  rewritten as the secular
equation given in eq.~(\ref{theor3}),
where $H_{eff}$  now is an effective hamiltonian acting solely
within the chosen model space.
Our scheme to obtain an effective interaction
can be divided into three steps,
outlined below\footnote{A  more detailed
exposition can be found in refs.\ \cite{ko90,hko94}.}.

\subsection{The nucleon--nucleon potential}

First, one needs a free NN interaction $V$ which is
appropriate for nuclear physics at low and intermediate energies. 
Since quantum chromodynamics (QCD) is commonly
accepted as the theory of the strong interaction, the
NN interaction $V$ should be  completely determined by the underlying
quark--quark dynamics in QCD. However, due to the nonperturbative
character of QCD at low energies, one is still far from a
quantitative understanding of the NN interaction from the QCD point of
view. This problem is circumvented by introducing
models containing some of the properties of QCD, such as
confinement and chiral symmetry breaking. One of the most used models
is the so--called bag model, where a crucial question is the
size of the radius ($R$) of the confining bag. If the size of the bag radius
is chosen as in the "Little bag" ($R\leq 0.5$ fm) \cite{br79}, then
low--energy nuclear physics phenomena can be fairly well described
in terms of hadrons like nucleons, isobars and various mesons,
which are to be understood as an effective description of complicated
multiquark interactions.
However, other models which
seek to approximate QCD also indicate that an effective theory in terms
of hadronic degrees of freedom may very well be the most appropriate
picture for low energy nuclear physics.
Although there is no unique prescription for how to construct
a free NN interaction, a description
of the NN interaction in terms of various meson exchanges is presently
the most quantitative representation of the NN interaction
\cite{mac89}
in the energy regime of nuclear physics.

Motivated by these arguments we employ in this work a model for the
NN interaction based on the exchange of six non--strange mesons, including
the pseudo--scalar $\pi$ and $\eta$ mesons, the vector $\rho$ and $\omega$
mesons and the scalar $\delta$ and $\sigma$ mesons. The latter is
a fictitious meson used in one--boson--exchange (OBE) models to provide in an
effective way the intermediate
range attraction of the potential. This attraction
originates from the $2\pi$ exchange processes.
Realistic calculations \cite{mac89}
which include $2\pi$ exchange differ only negligibly from calculations
with OBE models, justifying thereby the introduction of the effective
$\sigma$ meson.
The NN potential V is then a superposition of OBE interactions $V_i$
and is represented as
\[
V=\sum_{i=\pi\eta\rho\omega\delta\sigma}V_i .
\]
The OBE interaction $V_i$, is then constructed using
standard Lagrangian expressions.
To evaluate the OBE amplitudes we use in this work the
Bonn potential as it is defined by the meson parameters of the Bonn A potential
in table
A.2 of ref.~\cite{mac89}. The reason why we favor the Bonn meson--exchange
potential models, is that  they  all exhibit a weak tensor force,
as compared to previous models for the NN potential.
A weak tensor force introduces more binding in systems like nuclear matter
and finite nuclei, yielding a better agreement with the
experimental binding energy data compared with earlier calculations.
Actually, results obtained with the Bonn potentials
for several nuclear systems \cite{mac89,hko94} may indicate that these potential
models allow for a more consistent description of nuclear structure than more
traditional nuclear--force models.
%
\subsection{The nuclear reaction matrix $G$}

In nuclear many--body calculations, the first problem one is
confronted with is the fact that the repulsive core of the NN potential $V$
is unsuitable for perturbative approaches. This problem is overcome
by introducing the reaction matrix $G$ given by the solution of the
Bethe--Goldstone equation
\begin{equation}
    G=V+V\frac{Q}{\omega - H_0}G,
\end{equation}
where $\omega$ is the unperturbed energy of the interacting nucleons,
and $H_0$ is the unperturbed hamiltonian.
The operator $Q$,
is again the projection operator which prevents the
interacting nucleons from scattering into states occupied by other nucleons.
Note that the definition of $Q$ in eq.~(\ref{theor2}) may differ from that
used in the calculations of the $G$--matrix. If this is the case, then
certain diagrams have to be included or excluded from the calculation
of the effective interaction in eq.~(\ref{theor3}). For further details,
see e.g.\ ref.~\cite{hko94}.

The equation for the $G$--matrix is solved in this work using the
double--partitioning scheme discussed in ref.~\cite{kkko76}.

\subsection{Perturbative many--body approaches}

Finally, we briefly sketch how to calculate an effective interaction
in terms of the $G$--matrix.
The first step here is to define the so--called $\hat{Q}$--box given by
\begin{equation}
   P\hat{Q}P=PH_1P+
   P\left(H_1 \frac{Q}{\omega-H_{0}}H_1+H_1
   \frac{Q}{\omega-H_{0}}H_1 \frac{Q}{\omega-H_{0}}H_1 +\dots\right)P,
   \label{eq:qbox}
\end{equation}
where we will replace $H_1$ with $G-U$ ($G$ replaces the free NN interaction
$V$).
The $\hat{Q}$--box\footnote{The $\hat{Q}$--box should not be confused
with the exclusion operator $Q$.}
is made up of non--folded diagrams which are irreducible
and valence linked. A diagram is said to be irreducible if between each pair
of vertices there is at least one hole state or a particle state outside
the model space. In a valence--linked diagram the interactions are linked
(via fermion lines) to at least one valence line. Note that a valence--linked
diagram can be either connected (consisting of a single piece) or
disconnected. In the final expansion including folded diagrams as well, the
disconnected diagrams are found to cancel out \cite{ko90}.
This corresponds to the cancellation of unlinked diagrams
of the Goldstone expansion \cite{ko90}.
We illustrate
these definitions by the diagrams shown in fig.~\ref{fig:diagsexam},
where an arrow pointing upwards
(downwards) is a particle (hole) state.
\begin{figure}[hbtp]
   \setlength{\unitlength}{1cm}
 \begin{center}
   \begin{picture}(7,3.5)
%%%\put(0,0){\framebox(7,3.5){}}
  \put(0,0){\epsfxsize=7cm,\epsfbox{qbox1.eps}}
   \end{picture}
  \end{center}
\caption{Different types of valence--linked diagrams. Diagram (a)
is irreducible and connected, (b) is reducible, while (c) is irreducible
and disconnected.}
\label{fig:diagsexam}
\end{figure}
Particle states outside the model space are given by railed lines.
Diagram (a) is irreducible, valence linked and connected,
while (b) is reducible since
the intermediate particle states belong to the model space.
Diagram (c) is irreducible, valence linked and disconnected.

We can then obtain an effective interaction
$H_{eff}$ in terms of the $\hat{Q}$--box,
with \cite{ko90,hko94}
\begin{equation}
    H_{eff}^{(n)} = \omega + \hat{Q}+{\displaystyle\sum_{m=1}^{\infty}}
    \frac{1}{m!}\frac{d^m\hat{Q}}{d\omega^m}\left\{
    H_{eff}^{(n-1)} - \omega \right\}^m .
    \label{eq:fd}
\end{equation}
Observe also that the
effective interaction $H_{eff}^{(n)}$
is evaluated at a given model space energy
$\omega$, as is the case for the $G$--matrix as well. We choose this
starting energy to be $-20$ MeV.
Moreover, although $\hat{Q}$ and its derivatives contain disconnected
diagrams, such diagrams cancel exactly in each order \cite{ko90}, thus
yielding a fully connected expansion in eq.\ (\ref{eq:fd}).
The first iteration is then given by
\begin{equation}
   H_{eff}^{(0)} =  \omega + \hat{Q}.
\end{equation}
We define the $\hat{Q}$--box to consist of all two--body
diagrams\footnote{With two--body we also mean one--body diagrams like
diagram (a) in fig.~\ref{fig:qbox} with a spectator valence line.}
through third order in the $G$--matrix, as shown in
ref.~\cite{hko94}. Typical examples of diagrams which are included
in the $\hat{Q}$--box are shown in fig.~\ref{fig:qbox}. The summations
over intermediate states is restricted to excitations
of $2\hbar\Omega$ ($\hbar\Omega$ is the oscillator energy)
in oscillator energy, an approximation which has
been found to be appropriate if one uses a potential with a weak
tensor force like the Bonn A potential. The reader should note that
diagram (c) in fig.~\ref{fig:qbox} is not included in
the $\hat{Q}$--box. Three--body
and other many--body effective contributions may be of importance
in the study of spectra of nuclei with more than two valence
nucleons. These contributions will be studied by us in future works.
%
\begin{figure}[hbtp]
   \setlength{\unitlength}{1cm}
  \begin{center}
   \begin{picture}(7,3)
%%     \put(0,0){\framebox(7,3){}}
      \put(0,0){\epsfxsize= 7cm,\epsfbox{qbox2.eps}}
\end{picture}
\end{center}
\caption{Examples of diagrams included in the $\hat{Q}$--box.
Diagram (a) is a one--body diagram, whereas diagram (b) is a two--body
diagram. Diagram (c) is an effective three--body diagram which is not
included in our definition of the $\hat{Q}$--box.}
\label{fig:qbox}
\end{figure}
%

Another iterative scheme which has been much favored in the literature
is a method proposed by Lee and Suzuki (LS) \cite{ls80}. The effective interaction
we will employ in this work has been obtained using the LS method, which
gives the following expression for the effective interaction
%
\begin{equation}
H_{eff}^{(n)} =  \omega + \left[1-\hat{Q}_{1}-\sum_{m=2}^{n-1}\hat{Q}_{m}
\prod_{k=n-m+1}^{n-1} \left ( H_{eff}^{(k)} - \omega \right )
              \right]^{-1}\hat{Q},
\end{equation}
where
\begin{equation}
\hat{Q}_{m}=\frac{1}{m!}\frac{d^m\hat{Q}}{d\omega^m}.
\end{equation}
%
\subsection{Application to the Sn isotopes.}
%
We end this section with some calculational details regarding
the effective interaction appropriate for the Sn isotopes.
The Bethe--Goldstone equation is solved
for the $G$--matrix for five starting
energies $\omega$, by way of the so--called double--partitioning scheme
defined in ref.~\cite{kkko76}. The Pauli operator is constructed so as to
prevent scattering into intermediate states with a single
nucleon in any of the
states defined by the orbitals from the $0s$ state, the states
in the $0p$--shell, the
$1s0d$--shell and the $1p0f$--shell, in addition we also have the $0g_{9/2}$
orbit. Moreover, the Pauli operator prevents scattering into intermediate
states with two particles in the orbits of the $2s1d0g$--shell (except the $0g_{9/2}$
orbit) and the $2p1f0h$--shell.
A harmonic--oscillator basis was chosen for the
single--particle wave functions, with an oscillator energy $\hbar\Omega$ given
by
$\hbar\Omega = 45A^{-1/3} - 25A^{-2/3}= 8.5~$MeV,  $A=100$ being the mass
number.
%%%%%%%%%%%%%%%%%%%%%%%%%%
%%%%%%%%%%%%%%%% file:sh-alg.tex %%%%%%%%%%%%%
%
\section{Shell Model Algorithm.}
%
The shell model problem requires the solution of a real symmetric
$n \times n$ matrix eigenvalue equation
\be
H\ket{vec_k} = E_k \ket{vec_k},
\label{e1}
\ee
with $k = 1,\ldots, K$. The eigenvalues $E_k$ are understood to be
numbered in increasing order. In a typical shell model problem
we are interested in only the lowest eigenstates of eq.~(\ref{e1}),
so $K$ may be of the order of 10 to 50.
The total dimension $n$ of the eigenvalue matrix $H$ is large,
for the Sn isotopes of interest up to $n \approx 2 \times 10^{7}$.

Such large matrix problems are increasingly used in science and
enginering and practical numerical algorithms for determining their
properties are continually being developed\cite{dav89}.
These methods are closely related to the development of modern
computer technology, both  hardware and software.
What was impossible to solve a few years ago may now be within
reach and we should also be prepared for an increased  future
development. To indicate the present possibility,
in a work by J.~Olsen {\sl et al.}\cite{ols90} in a quantum
chemistry configuration interaction calculation (FCI), eq.~(\ref{e1})
was solved in a basis with $n = 10^{9}$.

Different computational approaches to solve eq.(\ref{e1}) can
be distinguished
based on the size of $n$.
For $n$ small, i.e. $10^2 < n < 10^3$ and with  the number of
matrix elements of $H$
less than $10^6$ such  problems can be accomodated within the direct
access memory of a modern work station and can be diagonalized by
standard matrix routines.
In a second domain  with  $ n > 10^3$ but small enough that $H$
has no more than $10^8$ elements. This will require $\approx 1.5$~Gbyte
of storage.Then all matrix elements may be stored in memory
or alternatively on a standard disk.
In these cases the complete diagonalization of
$H$ will not be of physical interest and efficient iteration
procedures have been developed to find the lowest energy eigenvalues
and eigenvectors.

The last domain with $n$ so large that the number of
non--diagonal matrix elements
of $H$ exceed $\approx 10^{10}$, no storage medium is at present large
enough  and the matrix elements have to be calculated when needed.
This requires a fast computer algorithm since the matrix elements of
$H$ have to be calculated repeatedly throughout the iteration process,
again using similar algorithms as in the second domain. Our cases of
special interest are of this last type.

Based on the present computer methods we have developed a code
which is under continuous improvement
to solve the eigenvalue problem given in eq.~(\ref{e1}).
The basic requirement
is to be able to handle problems with $n > 10^6$. In the following
we discuss some of the important elements which enter the algorithm.

We separate the discussion into three parts:
%
\begin{itemize}
%
\item The m--scheme representation of the basic states.
%
\item The Lanczos iteration algorithm.
%
\item The Davidson--Liu iteration technique.
%
\end{itemize}
%
\subsection{The m--scheme representation.}
%
We write the eigenstates in eq.~(\ref{e1})  as  linear combinations
of slater determinants. In a second quantization representation
a slater determinant (SD) is given by
%
\be
\ket{SD_{\nu}(N)} = \prod_{(jm)\in {\nu}} a_{jm}^{\dagger}\ket{0},
\ee
%
and the complete set is generated by distributing the $N$ particles
in all possible ways throughout the basic one--particle states constituting
the P--space. This is a very efficient
representation. A single $\ket{SD}$ requires only one  computer word
(32 or 64 bits) and  in memory a $\ket{SD}$ with $N$ particles
is given by
%
\be
\ket{SD} \longrightarrow (\underbrace{00111101010 \cdots}_{N 1's}),
\ee
%
where each 0 and 1 corresponds to an m--orbit in the valence
P--space. Occupied orbits  have a 1 and empty orbits a 0.
 Furthermore, all important calculations  can
be handled in Boolean algebra which is very efficient on modern computers.
The action  of operators of the form $a_{\alpha}^{\dagger} a_{\beta}$ or
$a_{\alpha}^{\dagger} a_{\beta}^{\dagger} a_{\gamma} a_{\delta}$
acting on an $\ket{SD}$ is easy to perform.

A disadvantage of this m--scheme approach is that the basic states
$\ket{SD}$ does not have the relevant physical properties such as
good total angular momentum $J$ and in case of both protons and neutrons
total isospin $T$. However, after the iterative solutions have been obtained
they will of course reflect all physical
properties of $H$. But the  computational consequence is larger matrices
than in a more conventional j--j coupling calculation since no reduction
in groups of definite $J$ and $T$ is possible.
There is one solution to this problem. We may replace $H$ by
%
\be
H \longrightarrow H_M = H + \alpha \left [ J^2 - M \cdot (M + 1) \right ],
\ee
%
and construct all the basic $\ket{SD}$ with $\sum_i m_i = M$.
Only states with $ J \geq M$ are calculated. Eigenstates
with $J= M$ are the same for $H$ and $H_M$ whereas eigenstates
of $H_M$ with $J > M$ are pushed up in energy.
Thus with $\alpha = 5 - 10$ ~MeV the lowest states of $H_M$
obtained through the iteration process will all have $J = M$.
Our experience with this technique so far is that it is useful
in cases where many states of a given $J$ are needed such as in double
$\beta$--decay processes but is of no help if the total low--energy
spectrum is of interest.
%
\subsection{The Lanczos iteration process.}
%
At present our basic approach to finding solutions to eq.~(\ref{e1})
is the Lanczos algorithm. This method was already applied to nuclear
physics problems by Whitehead {\sl et al.} in 1977.
In a review article \cite{whit77} they describe the technique in detail.
For the present discussion we outline the basic elements
of the method.\\[2ex]
%
The Lanczos method:

\begin{enumerate}
%
\item We choose  an initial lanczos vector $\ket{lanc_0}$ as the zeroth order
approximation to the lowest eigenvector in eq.~(\ref{e1}). Our experience
is that any  reasonable choice  is acceptable as long as the
vector does not have special properties such as good angular momentum.
That would usually terminate the iteration process at too early a
stage.
%
\item The next step involves generating a new  vector
through the process $|new_{p+1}> = H |lanc_p>$.
Throughout this process we construct the energy matrix elements
of $H$ in this lanczos basis. First, the diagonal matrix elements of $H$
are then obtained by

%
\be
\bra{lanc_p} H \ket{lanc_p} = \bra{lanc_p} \left . new_{p+1} \right \rangle,
\label{lanc1}
\ee
%

\item The new vector $\ket{new_{p+1}}$ is then orthogonalized to all
previously calculated lanczos vectors
%
\be
\ket{new_{p+1}^{'}} = \ket{new_{p+1}} - \ket{lanc_p} \cdot
	                \bra{lanc_p} \left . new_{p+1} \right \rangle		 - \sum_{q = 0}^{p-1} \ket{lanc_q} \cdot
	          \bra{lanc_q} \left . new_{p+1} \right \rangle,
\ee
%
and finally normalized

%
\be
\ket{lanc_{p+1}} = \frac{1}{\sqrt{\bra{new_{p+1}^{'}}
                      \left . new_{p+1}^{'} \right \rangle}}
						\ket{new_{p+1}^{'}},
\ee
%
to produce a new lanczos vector.
%
\item The off--diagonal matrix elements of $H$ are calculated by
%
\be
\bra{lanc_{p+1}} H \ket{lanc_p} = \bra{new_{p+1}^{'}}
                                \left . new _{p+1}^{'}\right \rangle,
\label{off1}
\ee

%
and all others are zero.
%
\item After n iterations we have an energy matrix of the form
%
\be
\left \{
\begin{array}{ccccc}
H_{0,0} & H_{0,1} & 0       & \cdots   & 0  \\
H_{0,1} & H_{1,1} & H_{1,2} & \cdots   & 0  \\
0       & H_{2,1} & H_{2,2} & \cdots   & 0  \\
\vdots  & \vdots  & \vdots  & \vdots   & H_{p-1,p}  \\
0       & 0       & 0       & H_{p,p-1}   & H_{p,p}\\
\end{array}
\right \}
\label{matr1}
\ee
as the p'th approximation to the eigenvalue problem in eq.~(\ref{e1}).
The number p is a reasonably small number and we can diagonalize
the matrix by standard methods to obtain eigenvalues and eigenvectors
which are linear combinations of the lanczos vectors.
%
\item This process is repeated until a suitable convergence
criterium has been reached.
%
\end{enumerate}
%
In this method each lanczos vector is a linear combination
of the basic $\ket{SD}$ with dimension $n$. For
$n \approx 10^6$, as in our case of interest, this means 8~Mb
of storage for one  single vector.
Here is  one of the important difficulties associated
with the lanczos method. Large disk storage is needed when
the number of lanczos vector exceeds $\approx 100$.
Another difficulty is found in the calculation of
$|new_{p+1}> = H |lanc_p>$ when $n > 10^6$.
The number of terms needed to be calculated may exceed
$10^{10}$. However, with an  optimistic view we can
certainly expect increasing computer resources in the future
which will reduce these problem to become moderate.

One important objection found in the computer literature\cite{dav89}
to the lanczos method is its slow convergence. This
is also our experience so far and means
that a large number of lanczos vectors have to be calculated
and stored in order to obtain convergence.
One improvement which we have implemented is to
terminate the lanczos process earlier, diagonalize the energy
matrix and choose some
of the lowest thus obtained eigenvectors as a starting point
for a new lanczos iteration.
This modifies the energy matrix in eq.~(\ref{matr1}) slightly,
since the matrix will not be tri--diagonal any more.
It reduces the disk storage requirement, but the convergence
problem is left. A possible way out here is the Davidson--Liu
method.
%

\subsection{The Davidson--Liu method.}
%
We outline the basic elements of this technique and refer to
the literature\cite{dav89} for details.
The first important improvement to the lanczos process given in
eqs.~(\ref{lanc1},\ref{matr1}) is to start with
several orthogonalized  and normalized initial vectors
and the second difference is the  way new additional vectors
are chosen. The method can be viewed
as an improvement of the lanczos technique and can be described
in the following steps:
%
\begin{enumerate}
%
\item Choose a (small) set  of start vectors
$\ket{x_k^{(0)}},\;\; k = 1,\ldots, K$.
%
\item $K$ new vectors are generated through the process
$\ket{new_k^{(0)}} = H \ket{x_k^{(0)}}$.
%
\item The matrix elements of $H$ are again obtained by

%
\be
\bra{x_p^{(0)}} H \ket{x_q^{(0)}}
= \bra{x_p^{(0)}} \left . new_q^{(0)}\right \rangle,
\label{dav1}
\ee
%
and the matrix $H$ is no longer tri--diagonal.
$H$ is diagonalized within the set of states
$\ket{x_k^{(0)}},\;\; k = 1, \ldots, K$
and $K$ eigenvalues and corresponding eigenstates are obtained
as the zero'th approximation to the lowest true eigenstates
of eq.~(\ref{e1}) in the form
%

\be
\ket{x_p^{(1)}} = \sum_{q = 1}^{K} a_{q, p} \ket{x_q^{(0)}}
                = \sum_{\nu = 1}^{n} C_{\nu,p}^{(1)} \ket{SD_{\nu}}.
\label{dav2}
\ee

%
The vectors $\ket{x_p^{(1)}}$ now constitute the basis for a new iteration.
%
\item New correction vectors are calculated through the steps
%
%
\begin{eqnarray}
\ket{r_p} &=&(H - \varepsilon_p^{(n)}) \ket{x_p^{(n)}}
	 =\ket{h_p^{(n)}} - \varepsilon_p^{(n)} \ket{x_p^{(n)}},
			                       \nonumber\\
\ket{\delta_p} &=& (H_{diag}-\varepsilon_p^{(n)})^{-1} \ket{r_p}.
\label{dav3}
\end{eqnarray}
%
Then new additional vectors are obtained by orthogonalizing
to all previous vectors, and finally normalized
%

\be
\ket{x_{k'+p}^{(n)}}
	= \sqrt{\frac{1}{norm}} \{ \ket{\delta_p}
		  -\sum_{q=1}^K \ket{x_q^{(n)}} \cdot
	                    \bra{x_q^{(n)}} \delta_p \rangle 
  -\sum_{q<p}^K \ket{x_{k'+q}^{(n)}} \cdot \bra{x_{K+q}^{(n)}}
		                  \delta_p \rangle  \},
\ee
%
witk $k' = 1, \ldots, K' \leq K$. Thus up to $K$ new vectors
may be generated through one iteration.
Then the diagonalization process in eqs.~(\ref{dav1}--\ref{dav2}) is
repeated and a new iteration to the true eigenvectors is obtained.
%
\item Again this process may be repeated until some convergence
criterium has been reached.
%
\end{enumerate}
%
At present we have only very preliminary experience with
this technique applied to nuclear physics problems, but experience
from other fields, in particular from ref.~\cite{ols90}, should
give hope for essential improvements over the lanczos
method.
%%%%%%%%%%%%%%%%%%%%%%%%%%%%%%
%
%%%%%%%%%%%%%% file: sh-alg.tex  %%%%%%%%%%%%
%
%
 \section{Shell Model Applications.}
 %
 Shell model calculations with realistic effective interactions
 have mainly been applied to light nuclei
 and with a rather small number of particles in the defined P--space.
 This makes the test of the effective interaction methods limited.
 In addition, the shell structure in these regions of the periodic table is
 not well separated. In many cases one struggles with so--called
 intruder states which make the analysis difficult.

 We are now in the position to carry these methods into medium and
 heavy nuclei. The effective interactions for such systems can be
 calculated reliably and  the shell model problem in the defined
 P--space both with many single--particle states and
 large number of  valence particles can be performed.  At present we
 have concentrated on identical particle systems as the first step.

 During the last few years a  rich
 variety of data has become available for nuclei far from the stability line.
 Recently, substantial progress has been made in the spectroscopic approach
 to the neutron deficient doubly magic $^{100}$Sn core. Experimental
 spectroscopic data are presently available down to $^{104}$Sn[13--16]
%%%%%    \cite{schu92--ryk92}.
 Thus theoretical shell model calculations for these nuclei
 are greatly needed. The starting point for such an analysis is the doubly
 magic core $^{100}$Sn. This system has been studied theoretically, using
 either approaches inspired by the relativistic Serot-Walecka \cite{sw86}
 model, such as the calculations of Hirata {\em et al.} \cite{hir91}
 and Nikolaus {\em et al.} \cite{nhm92}, or non-relativistic
 models used by Leander {\em et al.} \cite{lean84}. In both approaches it
 is concluded that $^{100}$Sn is a reasonably stable closed core.
 Thus the Sn isotopes should be well suited for a shell model
 analysis.

 From a theoretical point of view the Sn isotopes are interesting
 to study as they are a unique testing ground for nuclear structure
 calculations. We have a large sequence of nuclei ranging from
 the closed $Z = N = 50$ shell and up to the closed
 $Z = 50$, $N = 82$ shell and they can be described in terms of
 neutron degrees of freedom only.
 In these nuclei it is often assumed that the seniority or
 generalized seniority scheme is a good approximation \cite{tal77}
 to the large shell model problem which these isotopes represents.
 This assumption can now be tested with our present shell model code.
 Another important topic is the question  of an additional effective
 three--body interaction \cite{mfko83}.
 With so many valence particles available this problem can now
 be thoroughly investigated and our shell model code based on the
 Lanczos iteration procedure is well suited for such a study.

 %
 \begin{table}[htbp]
 %
 \begin{center}
 %
 \begin{tabular}{|l|r|| l|r|} \hline
 System & Dimension & System & Dimension\\ \hline
 $^{102}$Sn  & 36&  $^{110}$Sn  & 1,853,256\\
 $^{103}$Sn  & 245&  $^{111}$Sn  & 3,608,550\\
 $^{104}$Sn  & 1504&  $^{112}$Sn  & 6,210,638\\
 $^{105}$Sn  & 7,451&  $^{113}$Sn  & 9,397,335\\
 $^{106}$Sn  & 31,124&  $^{114}$Sn  & 12,655,280\\
 $^{107}$Sn  & 108,297&  $^{115}$Sn  & 15,064,787\\
 $^{108}$Sn  & 323,682&  $^{116}$Sn  & 16,010,204 \\
 $^{109}$Sn  & 828,422&              &           \\ \hline
 \end{tabular}
 \end{center}
 %
 \caption{\label{res-table1}Number of basic states for the shell model
 calculation in the Sn isotopes with $1d_{5/2}$, $0g_{7/2}$,
 $1d_{3/2}$, $2s_{1/2}$ and $0h_{11/2}$ single particle orbits.}
 %
 \end{table}
 %
 Based on a closed $Z = N = 50$ core  we define the model space to
 consist of the spherical single--particle orbits in
 the $N = 4$ oscillator shell ($1d_{5/2}$, $0g_{7/2}$, $1d_{3/2}$,
 $2s_{1/2}$) plus the intruder  $0h_{11/2}$ orbit
 from the $N = 5$ oscillator shell. This means that our P--space
 ranges from the closed $Z = N =50$ core to the closed $Z=50,\; N = 82$ core.
 As shown below, the inclusion of the $h_{11/2}$ orbit to our model space
 is essential  to obtain a quantitative description of the data.
 Within this P--space the total number of basic states for the
 different isotopes are shown in table~\ref{res-table1}.

%
 \begin{table}[htbp]
 %
 \begin{center}
 %
 \begin{tabular}{||r|r||r|r||r|r||r|r||r|r||}\hline
 J&Number&J&Number&J&Number&J&Number&J&Number\\ \hline
 0&18705&6&159972&12&100117&18&19303&24&816\\
 1&53648&7&160803&13&82466&19&12826&25&368\\
 2&86990&8&156674&14&66229&20&8280&26&157\\
 3&114192&9&146505&15&51087&21&5013&27&40\\
 4&136672&10&133449&16&38398&22&2945&28&15\\
 5&151239&11&117043&17&27601&23&1581&29&1\\ \hline
 %
 \end{tabular}
 %
 \end{center}
 %
 \caption{\label{res-table2}Number of
 basic states separated into different groups of
 total angular momentum J for the $^{110}$Sn isotope.}
 %
 \end{table}
%

 To compare with more conventional shell model calculations we list the number
 of basic states for $^{110}$Sn separated into different J--groups in
 table~\ref{res-table2}. This shows how prohibitively difficult
 it would be to perform such an analysis in a conventional j--j
 coupled shell model scheme.
 %
 \begin{figure}[htbp]
 %
 \setlength{\unitlength}{1cm}
 \begin{center}
 %
 \setlength{\unitlength}{1cm}
 \thicklines
 %
%%\begin{picture}(13,5)
%
%%\put(0,0){\framebox(13,5){}}

 \Cartesian(0.75cm,1cm)
 %
 \pspicture(-1,-1)(16,4)
 %
 \psaxes[Ox=100,Dx=2,dx=1,showorigin=false,linewidth=1pt]{->}(0,0)(15.5,3.5)
 %
 \uput[0](0.0,3.5){MeV}
 %
 \uput[90](15.5,0.2){N}
 %
\psline[showpoints=true,linestyle=dotted,dotstyle=*,dotscale=1.2,linewidth=1pt]
 (2,1.942)(3,2.019)(4,2.019)(5,2.197)(6,2.247)(7,2.187)(8,2.390)(9,2.280)
 (10,2.194)(11,2.144)(12,2.109)(13,2.050)(14,2.0)(15,1.996)
 %
 \uput[90](2.1,1.96){$4^{+}$}
 %
\psline[showpoints=true,linestyle=dashed,dotstyle=+,dotangle=45,dotscale=1.2,linewidth=1pt]
(1,1.3)(2,1.259)(3,1.207)(4,1.205)(5,1.212)(6,1.256)(7,1.299)(8,1.293)
(9,1.229)(10,1.171)(11,1.141)(12,1.132)(13,1.141)(14,1.169)(15,1.227)
 %
\uput[90](1.1,1.4){$2^{+}$}
 %
\endpspicture
%
%%\end{picture}
 %
 \end{center}
 %
 \caption{\label{res-fig1}Experimental excitation energies for the lowest
 $2^{+}$ and $4^{+}$ states in the Sn isotopes.}
 %
 \end{figure}
 %

 %
 \begin{figure}[htbp]
 %
 \setlength{\unitlength}{1cm}
 \begin{center}
 \thicklines
%
%%\begin{picture}(11.5,6.5)
%
%%\put(0,0){\framebox(8,5){}}
%
\Cartesian(1.5cm,2cm)
%
\pspicture(0,-0.25)(7.5,3)
%
\psaxes[Ox=103,Dx=2,dx=1,Oy=-0.5,Dy=0.5,dy=0.5,
showorigin=false,linewidth=1pt]{->}(0,-0.25)(7.5,2.5)
%
\uput[0](0.2,2.5){MeV}
\uput[90](7.5,-0.25){N}
\psline[showpoints=true,dotstyle=triangle*,dotscale=1.2]
(1,0.25)(2,0.25)(3,0.25)(4,0.404)(5,0.660)(6,1.237)
%
\uput[270](1,0.24){$5/2^{+}$}
%
\psline[showpoints=true,linestyle=dashed,dotstyle=triangle*,dotscale=1.2]
(1,0.45)(2,0.401)(3,0.25)(4,0.25)(5,0.337)(6,0.863)(7,0.961)
%
\uput[90](1,0.46){$7/2^{+}$}
%
\psline[showpoints=true,linestyle=dotted,dotstyle=triangle*,dotscale=1.2]
(2,1.916)(3,1.506)(4,1.229)(5,0.988)(6,0.913)
%
\uput[270](2,1.9){$11/2^{-}$}
%
\psline[showpoints=true,dotstyle=triangle*,dotscale=1.2]
(4,0.505)(5,0.25)(6,0.25)(7,0.25)
%
\uput[180](4,0.505){$1/2^{+}$}
%
\psline[showpoints=true,linestyle=dashed,dotstyle=triangle*,dotscale=1.2]
(4,0.894)(5,0.748)(6,0.747)(7,0.408)
%
\uput[135](4,0.894){$3/2^{+}$}
%
\endpspicture
%
%%\end{picture}
%
\end{center}
%
\caption{\label{res-fig2}Experimental "one--quasiparticle" states
in the odd Sn isotopes.}
%
\end{figure}
%

 At present rather few energy levels are known for the lightest
 Sn isotopes, a fact which makes a detailed comparison  difficult
between theory and experiment.
 However, some characteristic properties  can be found in the data.
 In the even $N$ isotopes there is a remarkably constant spacing between
 the $0^{+}$ ground state and the first excited $2^{+}$ state,
 see fig.~\ref{res-fig1}, all the way from the 2--particle system $^{102}$Sn
 to the 30--particle or 2--hole system. A nearly similar feature is found
 for the spacing between the $0^{+}$ ground state and the first
 excited $4^{+}$ state. Furthermore,
 in the odd $N$ isotopes the lowest states can easily be described
 as one--quasiparticle states with increasing fermi level as shown in
 fig.~\ref{res-fig2}. These important properties
 indicate a collective pairing effect or a seniority coupling scheme
 for the Sn  nuclei.  It is  essential that our microscopic shell model
 calculation is able to reproduce such phenomena.
 This  was also the basis for the pioneering
 BCS shell model analysis by Kisslinger and Sorensen~\cite{ks60}
 and also in the generalized seniority scheme discussed by
 Talmi~\cite{tal77}.

%
\begin{figure}[htbp]
%
%%%%%%%%%% fra Anne %%%%%%%%%%%
%
\setlength{\unitlength}{1.3cm}

\begin{center}
\begin{picture}(8,3.5)(1,-1.5)

\newcommand{\lc}[1]{\put(-.5,#1){\line(1,0){1}}}
\newcommand{\ls}[2]{\put(.7,#1){\makebox(0,0){{\scriptsize $#2^{+}$}}}}
\newcommand{\lsr}[2]{\put(.9,#1){\makebox(0,0){{\scriptsize $#2^{+}$}}}}

\newcommand{\lcc}[1]{\put(1.5,#1){\line(1,0){1}}}
\newcommand{\lss}[2]{\put(2.7,#1){\makebox(0,0){{\scriptsize $#2^{+}$}}}}
\newcommand{\lssr}[2]{\put(2.8,#1){\makebox(0,0){{\scriptsize $#2^{+}$}}}}

\newcommand{\lccc}[1]{\put(3.5,#1){\line(1,0){1}}}
\newcommand{\lsss}[2]{\put(4.7,#1){\makebox(0,0){{\scriptsize $#2^{+}$}}}}
\newcommand{\lsssr}[2]{\put(4.8,#1){\makebox(0,0){{\scriptsize $#2^{+}$}}}}

\newcommand{\lcccc}[1]{\put(5.5,#1){\line(1,0){1}}}
\newcommand{\lssss}[2]{\put(6.7,#1){\makebox(0,0){{\scriptsize $#2^{+}$}}}}
\newcommand{\lssssr}[2]{\put(6.8,#1){\makebox(0,0){{\scriptsize $#2^{+}$}}}}

\newcommand{\lccccc}[1]{\put(7.5,#1){\line(1,0){1}}}
\newcommand{\lsssss}[2]{\put(8.7,#1){\makebox(0,0){{\scriptsize $#2^{+}$}}}}
\newcommand{\lsssssr}[2]{\put(8.8,#1){\makebox(0,0){{\scriptsize $#2^{+}$}}}}

\newcommand{\lcccccc}[1]{\put(9.5,#1){\line(1,0){1}}}
\newcommand{\lssssss}[2]{\put(10.7,#1){\makebox(0,0){{\scriptsize $#2^{+}$}}}}
\newcommand{\lssssssr}[2]{\put(10.8,#1){\makebox(0,0){{\scriptsize $#2^{+}$}}}}

\put(-.25,2.5){\makebox(0,0){MeV}}
\put(4.5,2.3){\makebox(0,0){{g$_{7/2} -$ d$_{5/2} = 0.0$ MeV}}}

\thicklines
\put(-.75,-.5){\line(0,1){3}}
\multiput(-.75,.0)(0,1){3}{\line(1,0){.1}}
\multiput(-.75,.5)(0,1){2}{\line(1,0){.05}}

\put(-1.,2){\makebox(0,0){2}}
\put(-1.,1){\makebox(0,0){1}}
\put(-1.,0){\makebox(0,0){0}}

%single-particle levels g_7/2 - d_5/2 = 0.0

\lc{.0}      \ls{.0}{0}
\lc{1.194}   \ls{1.194}{2}
\lc{1.478}   \ls{1.478}{4}
\put(0,-.5){\makebox(0,0){{\large $^{102}$Sn}}}

\lcc{.0}      \lss{.0}{0}
\lcc{1.009}   \lss{1.009}{2}
\lcc{1.423}   \lssr{1.423}{4}
\put(2,-.5){\makebox(0,0){{\large $^{104}$Sn}}}

\lccc{.0}      \lsss{.0}{0}
\lccc{1.080}   \lsss{1.080}{2}
\lccc{1.332}   \lsss{1.332}{0}
\lccc{1.443}   \lsssr{1.473}{4}
\put(4,-.5){\makebox(0,0){{\large $^{106}$Sn}}}

\lcccc{.0}      \lssss{.0}{0}
\lcccc{.963}    \lssss{.963}{2}
\lcccc{1.373}   \lssss{1.373}{4}
\lcccc{1.510}   \lssssr{1.510}{0}
\put(6,-.5){\makebox(0,0){{\large $^{108}$Sn}}}

\lccccc{.0}     \lsssss{.0}{0}
\lccccc{.829}   \lsssss{.829}{2}
\lccccc{1.225}  \lsssss{1.225}{4}
\lccccc{1.352}  \lsssssr{1.372}{4}
\put(8,-.5){\makebox(0,0){{\large $^{110}$Sn}}}

\lcccccc{.0}     \lssssss{.0}{0}
\lcccccc{.736}   \lssssss{.736}{2}
\lcccccc{1.104}  \lssssss{1.104}{4}
\put(10,-.5){\makebox(0,0){{\large $^{112}$Sn}}}

\end{picture}
%
\end{center}
%

%
\begin{center}
\begin{picture}(8,2.5)(1,0)

\newcommand{\lc}[1]{\put(-.5,#1){\line(1,0){1}}}
\newcommand{\ls}[2]{\put(.7,#1){\makebox(0,0){{\scriptsize $#2^{+}$}}}}
\newcommand{\lsr}[2]{\put(.8,#1){\makebox(0,0){{\scriptsize $#2^{+}$}}}}

\newcommand{\lcc}[1]{\put(1.5,#1){\line(1,0){1}}}
\newcommand{\lss}[2]{\put(2.7,#1){\makebox(0,0){{\scriptsize $#2^{+}$}}}}
\newcommand{\lssr}[2]{\put(2.8,#1){\makebox(0,0){{\scriptsize $#2^{+}$}}}}

\newcommand{\lccc}[1]{\put(3.5,#1){\line(1,0){1}}}
\newcommand{\lsss}[2]{\put(4.7,#1){\makebox(0,0){{\scriptsize $#2^{+}$}}}}
\newcommand{\lsssr}[2]{\put(4.8,#1){\makebox(0,0){{\scriptsize $#2^{+}$}}}}

\newcommand{\lcccc}[1]{\put(5.5,#1){\line(1,0){1}}}
\newcommand{\lssss}[2]{\put(6.7,#1){\makebox(0,0){{\scriptsize $#2^{+}$}}}}
\newcommand{\lssssr}[2]{\put(6.8,#1){\makebox(0,0){{\scriptsize $#2^{+}$}}}}

\newcommand{\lccccc}[1]{\put(7.5,#1){\line(1,0){1}}}
\newcommand{\lsssss}[2]{\put(8.7,#1){\makebox(0,0){{\scriptsize $#2^{+}$}}}}
\newcommand{\lsssssr}[2]{\put(8.8,#1){\makebox(0,0){{\scriptsize $#2^{+}$}}}}

\newcommand{\lcccccc}[1]{\put(9.5,#1){\line(1,0){1}}}
\newcommand{\lssssss}[2]{\put(10.7,#1){\makebox(0,0){{\scriptsize $#2^{+}$}}}}
\newcommand{\lssssssr}[2]{\put(10.8,#1){\makebox(0,0){{\scriptsize $#2^{+}$}}}}


\put(-.25,2.5){\makebox(0,0){MeV}}
\put(4.5,3){\makebox(0,0){{g$_{7/2} - $ d$_{5/2} = 1.0$ MeV}}}

\thicklines
\put(-.75,-.5){\line(0,1){3}}
\multiput(-.75,.0)(0,1){3}{\line(1,0){.1}}
\multiput(-.75,.5)(0,1){2}{\line(1,0){.05}}

\put(-1.,2){\makebox(0,0){2}}
\put(-1.,1){\makebox(0,0){1}}
\put(-1.,0){\makebox(0,0){0}}

%single-particle levels g_7/2 - d_5/2 = 1.0
%
\lc{.0}      \ls{.0}{0}
\lc{.715}    \ls{.715}{2}
\lc{.944}    \ls{.944}{4}
\put(0,-.5){\makebox(0,0){{\large $^{102}$Sn}}}

\lcc{.0}      \lss{.0}{0}
\lcc{.722}    \lss{.722}{2}
\lcc{.949}    \lss{.949}{4}
\put(2,-.5){\makebox(0,0){{\large $^{104}$Sn}}}

\lccc{.0}      \lsss{.0}{0}
\lccc{1.834}   \lsss{1.834}{4}
\lccc{2.002}   \lsss{2.082}{0}
\lccc{2.1783}  \lsss{2.2}{2}
\put(4,-.5){\makebox(0,0){{\large $^{106}$Sn}}}

\lcccc{.0}      \lssss{.0}{0}
\lcccc{.682}    \lssss{.682}{2}
\lcccc{1.056}   \lssss{1.056}{4}
\put(6,-.5){\makebox(0,0){{\large $^{108}$Sn}}}

\lccccc{.0}     \lsssss{.0}{0}
\lccccc{.636}   \lsssss{.636}{2}
\lccccc{1.006}  \lsssssr{1.006}{4,4}
\lccccc{1.029}  %\lsssssr{1.029}{4}
\put(8,-.5){\makebox(0,0){{\large $^{110}$Sn}}}

\lcccccc{.0}     \lssssss{.0}{0}
\lcccccc{.598}   \lssssss{.598}{2}
\lcccccc{.965}   \lssssss{.965}{4}
\put(10,-.5){\makebox(0,0){{\large $^{112}$Sn}}}
%
\end{picture}
%
\end{center}
%%%%%%%%%%%%%% slutt Anne %%%%%%%%%%%
%

\vspace*{3mm}

\caption{\label{res-fig3}Energy spectra of $^{102\mbox{--}112}$Sn with a
model space consisting of the
$\varepsilon_{d_{5/2}}$ and $\varepsilon_{g_{7/2}}$ orbits only.}
%
\end{figure}
%


\subsection{The single--particle spectrum.}
%
The properties of the mean field defining the energies of the single--particle
states in the P--space are important for our shell model calculation.
At present it seems not possible to calculate these values
along the same lines as for the effective two--body interaction.
The theoretical framework is available but the results are not
accurate enough for our purpose. Futhermore, no experimental information
is available for the $^{101}$Sn  one--neutron system
to establish the single--particle energies.
Thus, these must be estimated theoretically.

As our first guide we have the mean--field work of
Leander {\em et al.} \cite{lean84}.
They obtain nearly degenerate
$\varepsilon_{d_{5/2}}, \varepsilon_{g_{7/2}}$ orbits.
A simplified calculation with only these two orbits in the P--space
are shown in fig.~\ref{res-fig3}. The results 
show that a large energy separation between the two orbits  would produce
a shell closure around $^{106}$Sn. Thus we choose $
\varepsilon_{d_{5/2}} - \varepsilon_{g_{7/2}} = 0.20$~MeV
in agreement with ref.~ \cite{grawe92}. The next important orbit
is $\varepsilon_{h_{11/2}}$. From the experimental data in
fig.~\ref{res-fig1} only a weak shell closure can be seen around
$N = 12\mbox{--}14$. 
This indicates an important contribution from
the $\varepsilon_{h_{11/2}}$ orbit.
Any shell model calculation must include this orbit in the valence 
P--space  if one wants to treat all Sn isotopes from $N = 102$ to 
$N= 130$ on the same footing.
At $^{107}$Sn an experimental level at 1.666~MeV with $J = (11/2)^{-}$
has been found. In a complete  calculation  we adjust the position
of the $\varepsilon_{h_{11/2}}$ to reproduce this level.
That gives
$\varepsilon_{h_{11/2}}-\varepsilon_{d_{5/2}} = 3.0$~MeV.
Then we are left with  the two last orbits
$\varepsilon_{d_{3/2}}$ and $\varepsilon_{s_{1/2}}$ which are of minor
importance for the spectra of the light Sn isotopes.
However, in $^{111}$Sn the low--lying  $(1/2)^{+}$ and $(3/2)^{+}$
are found experimentally at 0.255~MeV and 0.644~MeV
and are expected to be one--quasiparticle states\cite{san-94}.
Based on this information we choose
$\varepsilon_{s_{1/2}}-\varepsilon_{d_{5/2}} = 2.45$~MeV
and $\varepsilon_{d_{3/2}}-\varepsilon_{d_{5/2}} = 2.54$~MeV.
The final single--particle spectrum is shown in fig.~\ref{res-fig4}.


%
\begin{figure}[htbp]
%
\setlength{\unitlength}{1cm}
\begin{center}
%
\thicklines
%
\Cartesian(1cm,1cm)
%
\pspicture(0,-1.5)(6,5)
%
\psline[linewidth=1pt,linestyle=dashed](1,-1)(3,-1)
\psline[linewidth=1pt,linestyle=dashed](1,-1.5)(3,-1.5)
\uput[0](4,-1.5){\Large Q--space}
\uput[0](1.1,-0.5){\Large N = 50}
%
\psline[linewidth=1pt](1,0)(3,0)
%
\uput[0](0,-0.1){\footnotesize 5/2$^{+}$}
\uput[0](3.1,-0.1){\footnotesize 0.0 MeV}
%
\psline[linewidth=1pt](1,0.2)(3,0.2)
\uput[0](0,0.3){\footnotesize 7/2$^{+}$}
\uput[0](3.1,0.3){\footnotesize 0.20 MeV}
%
\psline[linewidth=1pt](1,2.45)(3,2.45)
\uput[0](0,2.3){\footnotesize 1/2$^{+}$}
\uput[0](3.1,2.3){\footnotesize 2.45 MeV}
%
\psline[linewidth=1pt](1,2.54)(3,2.54)
\uput[0](0,2.69){\footnotesize 3/2$^{+}$}
\uput[0](3.1,2.69){\footnotesize 2.54 MeV}
%
\psline[linewidth=1pt](1,3)(3,3)
\uput[0](0,3.1){\footnotesize 11/2$^{-}$}
\uput[0](3.1,3.1){\footnotesize 3.0 MeV}
%
\uput[0](1.1,3.5){\Large N = 82}
%
\uput[0](4,1.5){\Large P--space}
%%%	
\psline[linewidth=1pt,linestyle=dashed](1,4)(3,4)
\psline[linewidth=1pt,linestyle=dashed](1,4.3)(3,4.3)
\uput[0](4,4.3){\Large Q--space}
%
\endpspicture
%
\end{center}
%
\caption{\label{res-fig4}The single--particle spectrum
for the Sn isotopes.}.
%
\end{figure}
%
%
\begin{figure}[htbp]
%
%%%%%%%%%%%%%%%%% even.tex file fra Anne %%%%%%%%%%%%%%%%

\setlength{\unitlength}{1.75cm}

\begin{center}
\begin{picture}(7.9,4.5)(.4,-1.)
\newcommand{\lc}[1]{\put(-.5,#1){\line(1,0){1}}}
\newcommand{\ls}[2]{\put(.7,#1){\makebox(0,0){{\scriptsize $#2^{+}$}}}}
\newcommand{\lsr}[2]{\put(.75,#1){\makebox(0,0){{\scriptsize $#2^{+}$}}}}

\newcommand{\lcc}[1]{\put(1.25,#1){\line(1,0){1}}}
\newcommand{\lss}[2]{\put(2.45,#1){\makebox(0,0){{\scriptsize $#2^{+}$}}}}
\newcommand{\lssr}[2]{\put(2.53,#1){\makebox(0,0){{\scriptsize $#2^{+}$}}}}

\newcommand{\lccc}[1]{\put(3.,#1){\line(1,0){1}}}
\newcommand{\lsss}[2]{\put(4.2,#1){\makebox(0,0){{\scriptsize $#2^{+}$}}}}
\newcommand{\lsssr}[2]{\put(4.28,#1){\makebox(0,0){{\scriptsize $#2^{+}$}}}}

\newcommand{\lcccc}[1]{\put(4.75,#1){\line(1,0){1}}}
\newcommand{\lssss}[2]{\put(5.95,#1){\makebox(0,0){{\scriptsize $#2^{+}$}}}}
\newcommand{\lssssr}[2]{\put(6.03,#1){\makebox(0,0){{\scriptsize $#2^{+}$}}}}

\newcommand{\lccccc}[1]{\put(6.5,#1){\line(1,0){1}}}
\newcommand{\lsssss}[2]{\put(7.7,#1){\makebox(0,0){{\scriptsize $#2^{+}$}}}}
\newcommand{\lsssssr}[2]{\put(7.78,#1){\makebox(0,0){{\scriptsize $#2^{+}$}}}}

\newcommand{\lcccccc}[1]{\put(8.25,#1){\line(1,0){1}}}
\newcommand{\lssssss}[2]{\put(9.45,#1){\makebox(0,0){{\scriptsize $#2^{+}$}}}}
\newcommand{\lssssssr}[2]{\put(9.53,#1){\makebox(0,0){{\scriptsize $#2^{+}$}}}}

\put(-.25,4.4){\makebox(0,0){\large MeV}}
%\put(4,4){\makebox(0,0){\large ($d_{5/2}$, $g_{7/2}$, $d_{3/2}$, $s_{1/2}$,
%$h_{11/2}$) = (0.00, 0.20, 1.50, 2.80, 3.00 MeV)}}

\thicklines
\put(-.75,-.5){\line(0,1){5}}
\multiput(-.75,.0)(0,1){5}{\line(1,0){.1}}
\multiput(-.75,.5)(0,1){4}{\line(1,0){.05}}

\put(-1.,4){\makebox(0,0){4}}
\put(-1.,3){\makebox(0,0){3}}
\put(-1.,2){\makebox(0,0){2}}
\put(-1.,1){\makebox(0,0){1}}
\put(-1.,0){\makebox(0,0){0}}

% Theoretical spect. Sn-102
\lc{.0}      \ls{.0}{0}
\lc{1.3741}   \ls{1.3741}{2}
\lc{1.6447}   \ls{1.5947}{4}
\lc{1.6876}   \ls{1.6976}{6}
\lc{1.8163}   \ls{1.8163}{2}
\lc{1.9923}   \lsr{1.9923}{0,2}
\lc{1.9987}   %\ls{1.9987}{2}
\lc{2.0944}   \ls{2.1044}{4}
\lc{2.2335}   \ls{2.2335}{4}
\lc{2.3865}   \ls{2.3865}{6}
\lc{4.0046}   \ls{3.9646}{8} 
\put(1.4,3.9646){\makebox(0,0){{\scriptsize (6.005 MeV)}}}
\lc{4.0198}   \ls{4.0898}{10} 
\put(1.4,4.0898){\makebox(0,0){{\scriptsize (6.020 MeV)}}}
\put(0,-.3){\makebox(0,0){{\large $^{102}$Sn}}}

%Theoretical spect. Sn-104
\lcc{.0}      \lss{.0}{0}
\lcc{1.2308}   \lss{1.2308}{2}
\lcc{1.6713}   \lss{1.6713}{4}
\lcc{1.8464}   \lssr{1.7964}{0,6}
\lcc{1.8633}   %\lss{1.8633}{6}
\lcc{1.9124}   \lss{1.9124}{2}
\lcc{1.9820}   %\lss{1.9820}{4}
\lcc{2.0067}   \lssr{2.0467}{4,2}
\lcc{2.2854}   \lss{2.2854}{4}
\lcc{2.4038}   \lss{2.4038}{4}
\lcc{2.8394}   \lss{2.8394}{8} 
\lcc{3.1503}   \lss{3.1503}{10}
\put(1.75,-.3){\makebox(0,0){{\large $^{104}$Sn}}}
\put(4.2,4.5){\makebox(0,0){{\Large {\sc Theoretical energy spectra}}}}

%Theoretical spect. Sn106
\lccc{.0}      \lsss{.0}{0}
\lccc{1.2584}   \lsss{1.2584}{2}
\lccc{1.6279}   \lsss{1.6279}{0}
\lccc{1.8436}   \lsss{1.8136}{4}
\lccc{1.9288}   \lsssr{1.9288}{4,6}
\lccc{1.9603}   %\lsss{1.9603}{6}
\lccc{1.9747}   \lsss{2.0247}{2}
\lccc{2.1398}   \lsss{2.1398}{2}
\lccc{2.1809}   \lsss{2.2409}{4} 
\lccc{2.4252}   \lsss{2.4252}{6}
\lccc{3.0478}   \lsss{3.0478}{8}
%\lccc{3.5927}   %\lsss{3.5927}{8}
\lccc{3.1550}   \lsss{3.1650}{10}
\put(3.5,-.3){\makebox(0,0){{\large $^{106}$Sn}}}

%Theoretical spect. Sn108
\lcccc{.0}    \lssss{.0}{0}
\lcccc{1.3164} \lssss{1.3164}{2}
\lcccc{1.7232} \lssss{1.7232}{0}
\lcccc{1.9267} \lssss{1.8567}{2}
\lcccc{1.9863} \lssss{1.9563}{4}
\lcccc{2.0700} \lssss{2.0600}{6}
\lcccc{2.1271} %\lssss{2.1271}{4}
\lcccc{2.1567} \lssssr{2.1767}{4,2}
\lcccc{2.2637} \lssssr{2.3037}{4,6}
\lcccc{2.2640} %\lssss{2.2640}{6}
\lcccc{3.1831} \lssss{3.1931}{8}
\lcccc{3.8796} \lssss{3.8796}{10}
\put(5.25,-.3){\makebox(0,0){{\large $^{108}$Sn}}}

%Theoretical spect. Sn110
\lccccc{0.0}   \lsssss{0.0}{0}
\lccccc{1.2994}  \lsssss{1.2994}{2}
\lccccc{2.0100}  \lsssssr{2.010}{0,4}
\lccccc{2.023}   %\lsssss{2.023}{4}
\lccccc{2.1451}  \lsssssr{2.1251}{2,4}
\lccccc{2.1927}  %\lsssss{2.1927}{4}
\lccccc{2.2079}  \lsssss{2.2479}{6}
\lccccc{2.2711}  %\lsssss{2.2711}{6}
\lccccc{2.3434}  \lsssss{2.3834}{2}
\lccccc{2.4307}  \lsssss{2.4807}{4}
\lccccc{3.443} \lsssss{3.443}{8}
\lccccc{3.956} \lsssss{3.956}{10}
\put(7,-.3){\makebox(0,0){{\large $^{110}$Sn}}}

%Theoretical spect. Sn112
\lcccccc{0,0}     \lssssss{0,0}{0}
\lcccccc{1.3395}  \lssssss{1.3395}{2}
\lcccccc{1.9748}  \lssssss{1.8848}{0}
\lcccccc{1.9806}  \lssssss{1.9806}{2}
\lcccccc{2.0982}  \lssssss{2.0982}{0}
\lcccccc{2.1600}  \lssssssr{2.1800}{4,4}
\lcccccc{2.1989}  %\lssssss{2.1989}{4}
\lcccccc{2.2911}  \lssssss{2.3011}{6}
\lcccccc{2.4614}  \lssssss{2.4614}{2}

\put(8.75,-.3){\makebox(0,0){{\large $^{112}$Sn}}}

\end{picture}
\end{center}

\begin{center}
\begin{picture}(7.9,5)(.4,-.25)

\newcommand{\lc}[1]{\put(-.5,#1){\line(1,0){1}}}
\newcommand{\ls}[2]{\put(.7,#1){\makebox(0,0){{\scriptsize $#2^{+}$}}}}
\newcommand{\lsr}[2]{\put(.9,#1){\makebox(0,0){{\scriptsize $#2^{+}$}}}}

\newcommand{\lcc}[1]{\put(1.25,#1){\line(1,0){1}}}
\newcommand{\lss}[2]{\put(2.45,#1){\makebox(0,0){{\scriptsize $#2^{+}$}}}}
\newcommand{\lssr}[2]{\put(2.55,#1){\makebox(0,0){{\scriptsize $#2^{+}$}}}}

\newcommand{\lccc}[1]{\put(3.,#1){\line(1,0){1}}}
\newcommand{\lsss}[2]{\put(4.2,#1){\makebox(0,0){{\scriptsize $#2^{+}$}}}}
\newcommand{\lsssr}[2]{\put(4.3,#1){\makebox(0,0){{\scriptsize $#2^{+}$}}}}

\newcommand{\lcccc}[1]{\put(4.75,#1){\line(1,0){1}}}
\newcommand{\lssss}[2]{\put(5.95,#1){\makebox(0,0){{\scriptsize $#2^{+}$}}}}
\newcommand{\lssssr}[2]{\put(5.05,#1){\makebox(0,0){{\scriptsize $#2^{+}$}}}}

\newcommand{\lccccc}[1]{\put(6.5,#1){\line(1,0){1}}}
\newcommand{\lsssss}[2]{\put(7.7,#1){\makebox(0,0){{\scriptsize $#2^{+}$}}}}

\newcommand{\lcccccc}[1]{\put(8.25,#1){\line(1,0){1}}}
\newcommand{\lssssss}[2]{\put(9.45,#1){\makebox(0,0){{\scriptsize $#2^{+}$}}}}

\put(-.25,4.4){\makebox(0,0){\large MeV}}

\thicklines
\put(-.75,-.5){\line(0,1){5}}
\multiput(-.75,.0)(0,1){5}{\line(1,0){.1}}
\multiput(-.75,.5)(0,1){4}{\line(1,0){.05}}

\put(-1.,4){\makebox(0,0){4}}
\put(-1.,3){\makebox(0,0){3}}
\put(-1.,2){\makebox(0,0){2}}
\put(-1.,1){\makebox(0,0){1}}
\put(-1.,0){\makebox(0,0){0}}


%Exp. spectra Sn-104
\lcc{.0}      \lss{.0}{0}
\lcc{1.259}   \lss{1.259}{2}
\lcc{1.942}   \lss{1.942}{4}
\lcc{2.257}   \lss{2.257}{6}
\lcc{3.440}   \lss{3.440}{8}
\lcc{3.980}   \lss{3.980}{10}

\put(1.75,-.3){\makebox(0,0){{\large $^{104}$Sn}}}
\put(4.2,5.){\makebox(0,0){{\Large {\sc Experimental energy spectra}}}}

%Exp. spectra Sn-106
\lccc{.0}      \lsss{.0}{0}
\lccc{1.208}   \lsss{1.208}{2}
\lccc{2.020}   \lsss{2.020}{4}
\lccc{2.324}   \lsss{2.324}{6}
\lccc{3.476}   \lsss{3.476}{8}
\lccc{4.128}   \lsss{4.128}{10}
\put(3.5,-.3){\makebox(0,0){{\large $^{106}$Sn}}}

%Exp. spect. Sn-108
\lcccc{.0}      \lssss{.0}{0}
\lcccc{1.206}    \lssss{1.206}{2}
\lcccc{2.111}   \lssss{2.112}{4}
\lcccc{2.365}   \lssss{2.365}{6}
\lcccc{2.700}   \lssss{2.700}{0}
\lcccc{2.817}   \lssss{2.817}{6}
\lcccc{3.561}   \lssss{3.551}{8}
%\lcccc{4.140}   \lssss{4.140}{8}
\lcccc{4.251}   \lssss{4.251}{10}
\put(5.25,-.3){\makebox(0,0){{\large $^{108}$Sn}}}

%Exp. spect. Sn-110
\lccccc{.0}      \lsssss{.0}{0}
\lccccc{1.212}   \lsssss{1.212}{2}
\lccccc{2.197}   \lsssss{2.197}{4}
\lccccc{2.4559}  \lsssss{2.4159}{4}
\lccccc{2.480}   \lsssss{2.5300}{6}
\lccccc{2.753}   \lsssss{2.753}{6}
\lccccc{3.8148}  \lsssss{3.8148}{8}
\lccccc{4.319}   \lsssss{4.319}{10}
\put(7,-.3){\makebox(0,0){{\large $^{110}$Sn}}}

%Exp. spect. Sn112
\lcccccc{0,0}     \lssssss{0,0}{0}
\lcccccc{1.2569}  \lssssss{1.2569}{2}
\lcccccc{2.1511}  \lssssss{2.0911}{2}
\lcccccc{2.1909}  \lssssss{2.1909}{0}
\lcccccc{2.2476}  \lssssss{2.2976}{4}
\lcccccc{2.4762}  \lssssss{2.4262}{2}
\lcccccc{2.5211}  \lssssss{2.5211}{4}
\lcccccc{2.5493}  \lssssss{2.6193}{6}
\lcccccc{4.0779}  \lssssss{4.0779}{8}
\lcccccc{4.8197}  \lssssss{4.8197}{10}
\put(8.75,-.3){\makebox(0,0){{\large $^{112}$Sn}}}

\end{picture}
\end{center}
%%%%%%%%%%%% end fil even.tex  fra Anee %%%%%%%%%%%%%

\caption{\label{res-fig5}Theoretical (upper) and experimental (lower)%%	
energy spectra for the even isoptopes $^{102-112}$Sn.}
%%
\end{figure}
%
%
\begin{figure}[htbp]
%
%%
\setlength{\unitlength}{3.5cm}
\begin{center}
\begin{picture}(4.25,1.5)(-.2,-.25)
\newcommand{\lc}[1]{\put(-.5,#1){\line(1,0){.5}}}
\newcommand{\ls}[2]{\put(.15,#1){\makebox(0,0){{\scriptsize $#2^{+}$}}}}
\newcommand{\lsr}[2]{\put(.2,#1){\makebox(0,0){{\scriptsize $#2^{+}$}}}}

\newcommand{\lcc}[1]{\put(.5,#1){\line(1,0){.5}}}
\newcommand{\lss}[2]{\put(1.15,#1){\makebox(0,0){{\scriptsize $#2^{+}$}}}}
\newcommand{\lssr}[2]{\put(1.2,#1){\makebox(0,0){{\scriptsize $#2^{+}$}}}}

\newcommand{\lccc}[1]{\put(1.5,#1){\line(1,0){.5}}}
\newcommand{\lsss}[2]{\put(2.15,#1){\makebox(0,0){{\scriptsize $#2^{+}$}}}}
\newcommand{\lsssr}[2]{\put(2.2,#1){\makebox(0,0){{\scriptsize $#2^{+}$}}}}
\newcommand{\lsssneg}[2]{\put(2.1,#1){\makebox(0,0){{\scriptsize $#2^{-}$}}}}

\newcommand{\lcccc}[1]{\put(2.5,#1){\line(1,0){.5}}}
\newcommand{\lssss}[2]{\put(3.15,#1){\makebox(0,0){{\scriptsize $#2^{+}$}}}}
\newcommand{\lssssr}[2]{\put(3.25,#1){\makebox(0,0){{\scriptsize $#2^{+}$}}}}

\newcommand{\lccccc}[3]{\put(3.5,#1){\line(1,0){.5}}
\put(4.15,#2){\makebox(0,0){{\scriptsize $#3^{+}$}}}}
\newcommand{\lcccccr}[3]{\put(3.5,#1){\line(1,0){.5}}
\put(4.2,#2){\makebox(0,0){{\scriptsize $#3^{+}$}}}}

\put(-.5,1.75){\makebox(0,0){\large MeV}}
%\put(1.05,1.75){\makebox(0,0){\large ($d_{5/2}$, $g_{7/2}$, $d_{3/2}$,
%$s_{1/2}$, $h_{11/2}$) = (0.00, 0.20, 1.50, 2.80, 3.00 MeV)}}

\thicklines
\put(-.75,-.25){\line(0,1){2.}}
\multiput(-.75,.0)(0,1){2}{\line(1,0){.05}}
\multiput(-.75,.5)(0,1){2}{\line(1,0){.025}}

%\put(-.85,2){\makebox(0,0){2}}
\put(-.85,1){\makebox(0,0){1}}
\put(-.85,0){\makebox(0,0){0}}


% Teoretisk spektra Sn-103
\lc{0}        \ls{0}{5/2}
\lc{0.3075}    \ls{.3075}{7/2}
\lc{0.8073}    \ls{.8073}{3/2}
\lc{1.2095}    \ls{1.1895}{9/2}
\lc{1.2488}    \ls{1.2488}{5/2}
\lc{1.3051}    \ls{1.3051}{9/2}
\lc{1.3542}    \lsr{1.3642}{7/2, 11/2}
\lc{1.3607}    %\ls{1.3607}{11/2}
\lc{1.3986}    \ls{1.4186}{7/2}
\lc{1.4180}    \ls{1.4680}{3/2}

\put(-.25,-.15){\makebox(0,0){{\Large $^{103}$Sn}}}

%Theoretical spect. Sn-105
\lcc{0}    \lss{0}{5/2}
\lcc{0.3288}    \lss{.3288}{7/2}
\lcc{0.7883}    \lss{.7883}{3/2}
\lcc{1.1240}    \lssr{1.1240}{3/2, 5/2}
\lcc{1.1292}    %\lss{1.1292}{5/2}
\lcc{1.2347}    \lssr{1.2357}{9/2, 11/2}
\lcc{1.2379}    %\lss{1.2379}{11/2}
\lcc{1.3494}    \lssr{1.3494}{5/2, 1/2}
\lcc{1.3510}    %\lss{1.3510}{1/2}
\lcc{1.4257}    \lss{1.4257}{7/2}
\lcc{1.4659}    \lss{1.4889}{9/2}
\put(.75,-.15){\makebox(0,0){{\Large $^{105}$Sn}}}
\put(2.,1.75){\makebox(0,0){{\Large {\sc Theoretical energy spectra}}}}


%Theoretical spect. Sn-107
\lccc{0}    \lsss{0}{5/2}
\lccc{0.1697}    \lsss{.1697}{7/2}
\lccc{0.9275}    \lsssr{.9275}{3/2, 5/2}
\lccc{0.9341}    %\lsss{.9341}{5/2}
\lccc{1.1339}    \lsss{1.0939}{1/2}
\lccc{1.1809}    \lsss{1.1409}{9/2}
\lccc{1.2046}    \lsssr{1.2046}{1/2, 3/2}
\lccc{1.2079}    %\lsss{1.2079}{3/2}
\lccc{1.2578}    \lsss{1.2578}{7/2}
\lccc{1.2931}    \lsss{1.3231}{11/2}
%\lccc{1.6}    \lsssneg{1.6}{11/2}
%Theoretical spect. Sn-107
\put(1.75,-.15){\makebox(0,0){{\Large $^{107}$Sn}}}

%Theoretical spect. Sn109
\lcccc{0}       \lssss{0}{7/2}
\lcccc{0.0517}  \lssss{0.0617}{5/2}
\lcccc{0.6091}  \lssss{0.6091}{1/2}
\lcccc{0.6671}  \lssss{0.6671}{5/2}
\lcccc{0.8664}  \lssss{0.8664}{3/2}
\lcccc{1.0440}  \lssss{1.0440}{3/2}
\lcccc{1.1162}  \lssss{1.1162}{7/2}
\lcccc{1.2511}  \lssss{1.2511}{11/2}
\put(2.75,-.15){\makebox(0,0){{\Large $^{109}$Sn}}}

%Theoretical spect. Sn111
\lccccc{0.}{0.}{7/2}
\lcccccr{.2420}{.2420}{5/2, 1/2}
\lccccc{.2504}{.2504}{}
\lccccc{.6368}{.6368}{3/2}
\lccccc{.7002}{.7002}{5/2}
\lccccc{1.1735}{1.1735}{3/2}
\lccccc{1.3796}{1.3796}{9/2}
\put(3.75,-.15){\makebox(0,0){{\Large $^{111}$Sn}}}
\end{picture}
\end{center}


\begin{center}
\begin{picture}(4.25,2.5)(-.2,-.15)

\newcommand{\lc}[1]{\put(-.5,#1){\line(1,0){.5}}}
\newcommand{\ls}[2]{\put(.15,#1){\makebox(0,0){{\scriptsize $#2^{+}$}}}}
\newcommand{\lsr}[2]{\put(.2,#1){\makebox(0,0){{\scriptsize $#2^{+}$}}}}

\newcommand{\lcc}[1]{\put(.5,#1){\line(1,0){.5}}}
\newcommand{\lss}[2]{\put(1.15,#1){\makebox(0,0){{\scriptsize $#2^{+}$}}}}
\newcommand{\lssr}[2]{\put(1.2,#1){\makebox(0,0){{\scriptsize $#2^{+}$}}}}

\newcommand{\lccc}[1]{\put(1.5,#1){\line(1,0){.5}}}
\newcommand{\lsss}[2]{\put(2.15,#1){\makebox(0,0){{\scriptsize $#2^{+}$}}}}
\newcommand{\lsssr}[2]{\put(2.2,#1){\makebox(0,0){{\scriptsize $#2^{+}$}}}}

\newcommand{\lcccc}[1]{\put(2.5,#1){\line(1,0){.5}}}
\newcommand{\lssss}[2]{\put(3.15,#1){\makebox(0,0){{\scriptsize $#2^{+}$}}}}
\newcommand{\lssssr}[2]{\put(3.25,#1){\makebox(0,0){{\scriptsize $#2^{+}$}}}}

\newcommand{\lccccc}[3]{\put(3.5,#1){\line(1,0){.5}}
\put(4.15,#2){\makebox(0,0){{\scriptsize $#3^{+}$}}}}
\newcommand{\lcccccr}[3]{\put(3.5,#1){\line(1,0){.5}}
\put(4.25,#2){\makebox(0,0){{\scriptsize $#3^{+}$}}}}
\put(-.5,1.75){\makebox(0,0){\large MeV}}

\thicklines
\put(-.75,-.25){\line(0,1){2.}}
\multiput(-.75,.0)(0,1){2}{\line(1,0){.05}}
\multiput(-.75,.5)(0,1){2}{\line(1,0){.025}}

%\put(-.85,2){\makebox(0,0){2}}
\put(-.85,1){\makebox(0,0){1}}
\put(-.85,0){\makebox(0,0){0}}

%Exp. spectra Sn-105
\lcc{0.0}        \lss{0.0}{5/2}
\lcc{0.200}      \lss{.200}{7/2}
\lcc{1.195}      \lss{1.195}{9/2}
\lcc{1.393}      \lss{1.393}{11/2}
%\lcc{1.849}      \lss{1.849}{13/2}
%\lcc{2.031}      \lss{2.031}{15/2}

%\lcc{2.204}      \lss{2.204}{17/2}

\put(.75,-.15){\makebox(0,0){{\Large $^{105}$Sn}}}
\put(2.05,1.75){\makebox(0,0){{\Large {\sc Experimental energy spectra}}}}

%Exp. spect. Sn-107
\lccc{0.0}        \lsss{0.0}{5/2}
\lccc{0.151}      \lsss{.151}{7/2}
\lccc{1.221}      \lsss{1.221}{9/2}
\lccc{1.348}      \lsss{1.340}{11/2}
\lccc{1.371}      \lsss{1.401}{9/2}
%\lccc{1.798}      \lsss{1.798}{13/2}
%\lccc{1.943}      \lsss{1.943}{13/2}
%\lccc{2.067}      \lsss{2.067}{15/2}

\put(1.75,-.15){\makebox(0,0){{\Large $^{107}$Sn}}}

%Exp. spect. Sn-109
\lcccc{0.0}     \lssssr{0.0}{5/2,(7/2)}
\lcccc{.5449}   \lssss{.5449}{5/2}
\lcccc{.6645}   \lssss{.6445}{5/2}
\lcccc{.6787}   \lssss{.6987}{3/2}
\lcccc{.8911}   \lssss{.8711}{9/2}
\lcccc{.9256}   \lssssr{.9256}{3/2, 5/2}
\lcccc{.9921}   \lssssr{.9921}{5/2, 7/2, 9/2}
\lcccc{1.0628}  \lssss{1.0728}{5/2}
\put(2.75,-.15){\makebox(0,0){{\Large $^{109}$Sn}}}

%Exp. spect. Sn111
\lccccc{0.0}{0.0}{7/2}
\lccccc{.1545}{.1545}{5/2}
\lccccc{.2547}{.2547}{1/2}
\lccccc{.6435}{.6435}{3/2}
\lccccc{.7554}{.7554}{5/2}
\lcccccr{1.0326}{1.0226}{3/2, 5/2}
\lccccc{1.1001}{1.1001}{1/2}
\lcccccr{1.1517}{1.1617}{(3/2, 5/2)}
\lcccccr{1.2356}{1.2456}{5/2,7/2,9/2}
\put(3.75,-.15){\makebox(0,0){{\Large $^{111}$Sn}}}

\end{picture}
\end{center}
%%%%%%%%%%%%%%% end fil odd.tex fra Anne %%%%%%%%%%%%%%%%%


%
\caption{\label{res-fig6}Theoretical (upper) and experimental (lower)
energy spectra for the odd isoptopes $^{103-111}$Sn.}
%
\end{figure}
%

\subsection{Results and discussion.}
%
The resulting energy spectra obtained with a complete single-particle basis
including the $h_{11/2}$ orbit and the above--mentioned
effective interaction are displayed in fig.~\ref{res-fig5} 
and fig.~\ref{res-fig6}.
It shows all theoretical levels below 2.5~MeV exitation energy.
In addition the lowest $8^{+}$ and $10^{+}$ states in the even isotopes up 
to $^{110}$Sn are displayed.


For the even Sn isotopes (fig.~\ref{res-fig5}) the excitation energies
of the lowest $J=2$ states  are in good agreement with
the experimental values, with a spacing of about 1.2 MeV.
We find that the $h_{11/2}$ is important for this stability
even if the admixture  in wave functions is small. 
The theoretical spectra above the first excited $J = 2$ state
are  more difficult to interpret.
Compared to the levels experimentally found our calculation gives many
more levels in the region between 1 and 2~MeV. However, we believe
that additional experimental levels are to be found
in this region in the future.
The $8^{+}$ and $10^{+}$ yrast states show some interesting and 
challenging behaviour.  In $^{108}$Sn   and $^{110}$Sn our calculated
states are in reasonably good agreement with experiment. But in
the lighter isotopes we obtain  clear disagreements. This should 
be investigated further.
At present we have rather preliminary calculations of transition rates.
But it seems clear that it is not possible to reproduce
the experimentally found E2 transitions from
$6^{+} \longrightarrow 4^{+}$ 
without an  unphysically large  neutron effective charge.
Thus there are still many open questions in our shell model description of 
the light Sn isotopes and more experimental data are  greatly needed.

The odd Sn isotopes are displayed in fig.~\ref{res-fig6}. The two lowest energy
levels $J=5/2^{+}$ and $J=7/2^{+}$ are reproduced in correct order
and we find the level crossing between $J=5/2^{+}$ and $J=7/2^{+}$ around
$^{109}$Sn in agreement with experiment.
In $^{111}$Sn all states below 1~MeV are well reproduced. However, one
should keep in mind that the single--particle energies of the
$J=1/2^{+}$ and $J=3/2^{+}$ orbits are adjusted to reproduce these
levels. In $^{109}$Sn a tentative third $J=5/2^{+}$ at $\approx 0.6$~MeV is 
missing in our theoretical description compared to experiment. 
The next theoretical level with $J=5/2^{+}$ is
well above 1.5~MeV. Thus  theory will have large difficulties to
reduce the energy by as much as 1.0~MeV.
In the lighter Sn isotopes the experimental data are so limited that 
a further comparision must be postponed.
%
\subsection{Generalized seniority.}
%
The near constancy of the $2^{+}$ -- $0^{+}$ spacing is remarkable and may
indicate that the generalized seniority is approximately conserved\cite{tal77}.
In order to test this we construct the pair correlation operator
%
\be
S^{\dagger} = \sum_j C_j   \sum_{m > 0} (-)^{j - m} 
             a_{j m}^{\dagger} a_{j m}^{\dagger}
\label{sen}
\ee
where the coefficients $C_j$ are determined from the $^{102}$Sn
ground state. With our wave functions we then evaluate the square overlap
$|\bra{^{A}\makebox{Sn}} S^{\dagger} \ket{^{A - 2}\makebox{Sn}}|^{2}$.
The results are given in Table~\ref{tab-sen}.

%
\begin{table}[htbp]
%%
%
\begin{center}
\begin{tabular}{|r|ccccc||r|cccc|} \hline
 $J^{\pi}$ & A = 104 &  106 & 108 & 110 & 112 &
$J^{\pi}$ & A = 103 & 105 & 107 & 109\\
\hline
$0^{+}_{1}$ & 0.949       & 0.939       & 0.956       & 0.972       & 0.976&
$5/2^{+}_{1}$ & 0.967       & 0.899       & 0.911       & 0.940             \\
$2^{+}_{1}$   & 0.898       & 0.878       & 0.825       & 0.880       & 0.897&
$7/2^{+}_{1}$ & 0.935       & 0.912       & 0.897       & 0.939            \\
$4^{+}_{1}$   & 0.901       & 0.519       & 0.627       & 0.632       & 0.850&
$1/2^{+}_{1}$ & $\approx 0$ & 0.010       & 0.515       &             \\
$4^{+}_{2}$   & $\approx 0$ & 0.293       & 0.132    & 0.069 &$\approx 0 $ &
$1/2^{+}_{2}$ & $\approx 0$ & 0.680       & 0.346       &             \\
{$4^{+}_{3}$} & $\approx 0$ & 0.067       & 0.100       & 0.129 &$\approx 0 $
&$1/2^{+}_{3}$ & 0.828       & 0.187       & $\approx 0$ &             \\
$6^{+}_{1}$   & 0.922       & 0.879       & 0.895    & $\approx 0$ & 0.849& 
$3/2^{+}_{1}$ & $\approx 0$ & $\approx 0$ &$\approx 0$  &             \\
$6^{+}_{2}$ & $\approx 0$   & $\approx 0$ &$\approx 0$  & 0.893       
&$\approx 0 $     &
$3/2^{+}_{5}$ & 0.888       & 0.174       & $\approx 0$ &             \\ 
\hline
\end{tabular}
%%%%%
\end{center}
%%%%%%%%  end tabular %%%%%%%%%%%%%%%%%%%%%%






%%
\caption{\label{tab-sen}The seniority zero overlap (see eq.~\ref{sen})
for the lowest calculated eigenstates in $^{103}$Sn -- $^{112}$Sn.}
%%
\end{table}
%%%%%%%%%%%%%%%%%%%%%%%%%%%

For the even systems our shell model wave functions for 
$0^{+}$ ground state and the first $2^{+}$ excited state
 are reasonably well described in a generalized seniority scheme.
More than 80~\% of the wave functions for the $^{A}$Sn system
is given as the $^{A-2}$Sn system pluss a seniority zero pair. Similar feature
are seen for the lowest $(5/2)^{+}$ and $(7/2)^{+}$ states.
On the other hand, higher--lying states, both in even and odd cases
do not show  this simple behaviour. This may  indicate
that the pairing picture and the seniority coupling scheme 
does not give a proper description of these nuclei.

%%%%%%%%%%%%%%  Summary %%%%%%%%%%%%%%%%%%
\section{Conclusion.}
%
We have presented the basic elements and methods
 for a  large--basis shell model calculation with  a realistic microscopic
effective interation.
The potential is derived from a modern meson--exchange NN potential using
many--body perturbation theory. 
An application to the Sn isotopes ranging from
$A = 102$ to $A = 112$ are shown.  
In view of the fact that this is a truly realistic and microscopic calculation,
with very few parameters, the overall agrement with data is indeed remarkable.

In order to make a more complete assessment of the calculation, however,
we need additional experimental data, in particular on non--yrast states.
Electromagnetic transition rates are also needed for a more 
stringent test of the wave functions than obtained from the 
energy spectra only.


On the theoretical side, we are proceeding towards higher mass numbers.
The rather slow convergence of the Lanczos process is a difficulty and 
we plan to use the Davidson--Liu methods for the heavier Sn isotopes.
One should also include effective three--body forces.
Although these effects are expected to be small, they 
may give a sizeable effect for systems with many valence nucleons. 
It would also be useful to compare our calculations with
more phenomenological   calculations in order to facilitate a 
more direct physical interpretation of our  results.
%%%%%%%%%%%%%%%%% end %%%%%%%%%%%%%%%%%%



%%%%%%%%%%%%%%%%%%%%%%%%%%%
%%       Reference list
%%%%%%%%%%%%%%%%%%%%%%%%%%%


\begin{thebibliography}{9}
%
%%%%%%%%%%%%%%%%%%%%%%%%% file: intro.tex %%%%%%%%%%%%%%%%%%
%
\bibitem{lac80} M. Lacombe et al.,Phys. Rev. {\bf C21} (1980) 861
%
\bibitem{mac89} R.\ Machleidt, Adv.\ Nucl.\ Phys.\ {\bf 19} (1989)  189
%
\bibitem{vall89} M. Valli\`{e}res, " Computational Atomic and Nuclear
Physics", Proc. Summer school, Oak Ridge 1989, page 302,
ed. C. Bottcher et al., World Scientific, Singapore 1989,
ISBN 981--02--0125--7.
\bibitem{law65} R. D. Lawson and J. M. Soper, Intern. Nucl. Phys.
Conf., Gatlinburg 1966, page 511, eds. R. Becker, C. D. Goodman.
%

%%%%%%%%%%%%%%%%%%%%%%% file sect. 2 proceed.tex %%%%%%%%%

\bibitem{ko90} T.T.S.\ Kuo and E.\ Osnes, Folded-Diagram Theory
of the Effective Interaction in Atomic Nuclei, Springer Lecture
Notes in Physics, (Springer, Berlin, 1990) Vol. 364
%
\bibitem{hko94} M.\ Hjorth-Jensen, T.T.S.\ Kuo and
E.\ Osnes, submitted to Phys.\ Reports; M.\ Hjorth-Jensen,
E.\ Osnes and H.\ M\"{u}ther, Ann.\ of Phys.\  {\bf 213} (1992) 108;
M.\ Hjorth-Jensen, T.\ Engeland, A.\ Holt and E.\ Osnes, Phys.\ Reports
{\bf 242} (1994) 37
%
\bibitem{br79} G.E.\ Brown and M.\ Rho, Phys.\ Lett.\ {\bf B82} (1979) 177
%
\bibitem{kkko76}  E.M.\ Krenciglowa, C.L.\ Kung, T.T.S.\ Kuo
and E.\ Osnes, Ann.\ of Phys.\   {\bf 101} (1976) 154
%
\bibitem{ls80} S.Y.\ Lee and K.\ Suzuki, Prog.\ Theor.\ Phys.\
{\bf 64} (1980) 2091
%
%%%%%%%%%%%%%%%%%%%% file sh-alg.tex %%%%%%%%%%%%%%%%%%%%
\bibitem{dav89} E. R. Davidson, Comp. Phys. Comm. {\bf 53} (1989) 49\\
                E. R. Davidson, Comp. in  Phys. {\bf No.~5} (1993) 519
%
\bibitem{ols90} J. Olsen, P.J{\o}rgensen and J. Simons, Chem. Phys. Lett.
{\bf vol 169} (1990) 463
%
\bibitem{whit77} R. R. Whitehead, A. Watt, B. J. Cole and I Morrison,
Advances in Nuclear Physics, {\bf vol 9} (1977) New York, Plenum.
%

%%%%%%%%%%%%% file result.tex %%%%%%%%%%%%%%%%%%
\bibitem{schu92}  R. Schubart {\em et al.}, Z. Phys. {\bf A343} (1992) 123
\bibitem{schu91}  R. Schubart {\em et al.}, Z. Phys. {\bf A340} (1991) 109
\bibitem{grawe92}  H. Grawe {\em et al.}, Prog. Part. Nucl. Phys. {\bf 28}
(1992) 281
\bibitem{ryk92}  A. Plochocki {\em et al.}, Z. Phys. {\bf A342} (1992) 43
\bibitem{sw86}  S.D. Serot and J.D. Walecka, Adv. Nucl. Phys. {\bf 16} (1986)
1
\bibitem{san-94} N. Sandulesco, private communication
\bibitem{hir91}  D. Hirata {\em et al.}, Phys. Rev. {\bf C 44}, (1991) 1467
\bibitem{nhm92}  T. Nikolaus, T. Hoch and D.G. Madland,
Phys. Rev. {\bf C~46} (1992) 1757
\bibitem{lean84}  G.A. Leander, J. Dudek, W. Nazarewicz, J.R. Nix and Ph.
Quentin, Phys. Rev. {\bf C 30} (1984) 416
\bibitem{tal77} I. Talmi in Elementary Modes of Excitation in Nuclei,
ed. A. Bohr and R.A. Broglia, (North-Holland, Amsterdam, 1977);
private communication
\bibitem{mfko83} H. M\"{u}ther, A. Faessler, T.T.S. Kuo and E. Osnes,
Nucl.\ Phys. {\bf 401} (1983) 124;
H. M\"{u}ther, A. Polls and T.T.S. Kuo, Nucl.\ Phys. {\bf 435} (1985) 548
\bibitem{ehho93}  T. Engeland, M. Hjorth-Jensen, A. Holt and E. Osnes,
Phys. Rev. {\bf C 48} (1993) 535
%
\bibitem{wwcm} R.R. Whitehead, A. Watt, B.J. Cole and I. Morrison,
Adv.\ Nucl.\ Phys.\ {\bf 7} (1977) 123
\bibitem{ks60} L.S. Kisslinger and R.A. Sorensen, Mat.\ Fys.\ Medd.\ Dan.\
Vid.\ Selsk.\ {\bf 32} (1960) 1
\bibitem{blom93} J. Blomqvist, private communication
\bibitem{ehho94} T. Engeland, M. Hjorth-Jensen, A. Holt and E. Osnes,
unpublished
\end{thebibliography}
%
\end{document}

%

%%%%%%%%%%%%%%%%  fig.1 %%%%%%%%%%%%%%

\begin{figure}[hbtp]
   \setlength{\unitlength}{1cm}
 \begin{center}
   \begin{picture}(7,3.5)
%%%\put(0,0){\framebox(7,3.5){}}
  \put(0,0){\epsfxsize=7cm,\epsfbox{qbox1.eps}}
   \end{picture}
  \end{center}
\caption{Different types of valence--linked diagrams. Diagram (a)
is irreducible and connected, (b) is reducible, while (c) is irreducible
and disconnected.}
\label{fig:diagsexam}
\end{figure}

%%%%%%%%%%%%%%%%%%%%%%%%%%%%%%%%%%%%%%%%%

%%%%%%%%%%  fig.2 %%%%%%%%%%%%%%%%%%%

%
\begin{figure}[hbtp]
   \setlength{\unitlength}{1cm}
  \begin{center}
   \begin{picture}(7,3)
%%     \put(0,0){\framebox(7,3){}}
      \put(0,0){\epsfxsize= 7cm,\epsfbox{qbox2.eps}}
\end{picture}
\end{center}
\caption{Examples of diagrams included in the $\hat{Q}$--box.
Diagram (a) is a one--body diagram, whereas diagram (b) is a two--body
diagram. Diagram (c) is an effective three--body diagram which is not
included in our definition of the $\hat{Q}$--box.}
\label{fig:qbox}
\end{figure}

%%%%%%%%%%%%%%%%%%%%%%%%

%
 \begin{figure}[htbp]
 %
 \setlength{\unitlength}{1cm}
 \begin{center}
 %
 \setlength{\unitlength}{1cm}
 \thicklines
 %
%%\begin{picture}(13,5)
%
%%\put(0,0){\framebox(13,5){}}

 \Cartesian(0.75cm,1cm)
 %
 \pspicture(-1,-1)(16,4)
 %
 \psaxes[Ox=100,Dx=2,dx=1,showorigin=false,linewidth=1pt]{->}(0,0)(15.5,3.5)
 %
 \uput[0](0.0,3.5){MeV}
 %
 \uput[90](15.5,0.2){N}
 %
\psline[showpoints=true,linestyle=dotted,dotstyle=*,dotscale=1.2,linewidth=1pt]
 (2,1.942)(3,2.019)(4,2.019)(5,2.197)(6,2.247)(7,2.187)(8,2.390)(9,2.280)
 (10,2.194)(11,2.144)(12,2.109)(13,2.050)(14,2.0)(15,1.996)
 %
 \uput[90](2.1,1.96){$4^{+}$}
 %
\psline[showpoints=true,linestyle=dashed,dotstyle=+,dotangle=45,dotscale=1.2,linewidth=1pt]
(1,1.3)(2,1.259)(3,1.207)(4,1.205)(5,1.212)(6,1.256)(7,1.299)(8,1.293)
(9,1.229)(10,1.171)(11,1.141)(12,1.132)(13,1.141)(14,1.169)(15,1.227)
 %
\uput[90](1.1,1.4){$2^{+}$}
 %
\endpspicture
%
%%\end{picture}
 %
 \end{center}
 %
 \caption{\label{res-fig1}Experimental excitation energies for the lowest
 $2^{+}$ and $4^{+}$ states in the Sn isotopes.}
 %
 \end{figure}
 %
%%%%%%%%%%%%%%%%%%%%%%%% fig.3 %%%%%%%%%%%%%%%%%
 %
 \begin{figure}[htbp]
 %
 \setlength{\unitlength}{1cm}
 \begin{center}
 \thicklines
%
%%\begin{picture}(11.5,6.5)
%
%%\put(0,0){\framebox(8,5){}}
%
\Cartesian(1.5cm,2cm)
%
\pspicture(0,-0.25)(7.5,3)
%
\psaxes[Ox=103,Dx=2,dx=1,Oy=-0.5,Dy=0.5,dy=0.5,
showorigin=false,linewidth=1pt]{->}(0,-0.25)(7.5,2.5)
%
\uput[0](0.2,2.5){MeV}
\uput[90](7.5,-0.25){N}
\psline[showpoints=true,dotstyle=triangle*,dotscale=1.2]
(1,0.25)(2,0.25)(3,0.25)(4,0.404)(5,0.660)(6,1.237)
%
\uput[270](1,0.24){$5/2^{+}$}
%
\psline[showpoints=true,linestyle=dashed,dotstyle=triangle*,dotscale=1.2]
(1,0.45)(2,0.401)(3,0.25)(4,0.25)(5,0.337)(6,0.863)(7,0.961)
%
\uput[90](1,0.46){$7/2^{+}$}
%
\psline[showpoints=true,linestyle=dotted,dotstyle=triangle*,dotscale=1.2]
(2,1.916)(3,1.506)(4,1.229)(5,0.988)(6,0.913)
%
\uput[270](2,1.9){$11/2^{-}$}
%
\psline[showpoints=true,dotstyle=triangle*,dotscale=1.2]
(4,0.505)(5,0.25)(6,0.25)(7,0.25)
%
\uput[180](4,0.505){$1/2^{+}$}
%
\psline[showpoints=true,linestyle=dashed,dotstyle=triangle*,dotscale=1.2]
(4,0.894)(5,0.748)(6,0.747)(7,0.408)
%
\uput[135](4,0.894){$3/2^{+}$}
%
\endpspicture
%
%%\end{picture}
%
\end{center}
%
\caption{\label{res-fig2}Experimental "one--quasiparticle" states
in the odd Sn isotopes.}
%
\end{figure}
%
%%%%%%%%%%%%%%%%%%%%%%%%%%%%%%%%


%%%%%%%%%%%%%%%%% fig.4 %%%%%%%%%%%%%
%
\begin{figure}[htbp]
%
\setlength{\unitlength}{1cm}
\begin{center}
%
\thicklines
%
\Cartesian(1cm,1cm)
%
\pspicture(0,-1.5)(6,5)
%
\psline[linewidth=1pt,linestyle=dashed](1,-1)(3,-1)
\psline[linewidth=1pt,linestyle=dashed](1,-1.5)(3,-1.5)
\uput[0](4,-1.5){\Large Q--space}
\uput[0](1.1,-0.5){\Large N = 50}
%
\psline[linewidth=1pt](1,0)(3,0)
%
\uput[0](0,-0.1){\footnotesize 5/2$^{+}$}
\uput[0](3.1,-0.1){\footnotesize 0.0 MeV}
%
\psline[linewidth=1pt](1,0.2)(3,0.2)
\uput[0](0,0.3){\footnotesize 7/2$^{+}$}
\uput[0](3.1,0.3){\footnotesize 0.20 MeV}
%
\psline[linewidth=1pt](1,2.45)(3,2.45)
\uput[0](0,2.3){\footnotesize 1/2$^{+}$}
\uput[0](3.1,2.3){\footnotesize 2.45 MeV}
%
\psline[linewidth=1pt](1,2.54)(3,2.54)
\uput[0](0,2.69){\footnotesize 3/2$^{+}$}
\uput[0](3.1,2.69){\footnotesize 2.54 MeV}
%
\psline[linewidth=1pt](1,3)(3,3)
\uput[0](0,3.1){\footnotesize 11/2$^{-}$}
\uput[0](3.1,3.1){\footnotesize 3.0 MeV}
%
\uput[0](1.1,3.5){\Large N = 82}
%
\uput[0](4,1.5){\Large P--space}
%%%	
\psline[linewidth=1pt,linestyle=dashed](1,4)(3,4)
\psline[linewidth=1pt,linestyle=dashed](1,4.3)(3,4.3)
\uput[0](4,4.3){\Large Q--space}
%
\endpspicture
%
\end{center}
%
\caption{\label{res-fig4}The single--particle spectrum
for the Sn isotopes.}.
%
\end{figure}
%
%
\end{document}
%%%%%%%%%%%%%%%%%%%










\documentclass[prc,superscriptaddress,floatfix]{revtex4}
%\documentclass[prc,12pt,superscriptaddress,showpacs,floatfix]{revtex4}
%\documentclass[prl,aps,twocolumn,floatfix]{revtex4}
\usepackage[dvips]{graphicx}
%\usepackage{epsfig}
%\usepackage{pst-plot}
%\usepackage{bm}
%\usepackage{multicol}	% used for the two-column index

\def\be{\begin{equation}}
\def\ee{\end{equation}}
\def\ba{\begin{eqnarray}}
\def\ea{\end{eqnarray}}
\def\bas{\begin{eqnarray*}}
\def\eas{\end{eqnarray*}}



\begin{document}
%
\title{Ab-Initio Coupled Cluster Theory for Nuclear Structure}
%

\author{G.~Hagen}
\affiliation{Physics Division, Oak Ridge National Laboratory,
P.O. Box 2008, Oak Ridge, TN 37831, USA}
\affiliation{Department of Physics and Astronomy, University of
Tennessee, Knoxville, TN 37996, USA}
\affiliation{Centre of Mathematics for Applications, University of Oslo, 
N-0316 Oslo, Norway} 

\author{D.~J.~Dean}
\affiliation{Physics Division, Oak Ridge National Laboratory,
P.O. Box 2008, Oak Ridge, TN 37831, USA}


\author{M.~Hjorth-Jensen}
\affiliation{Department of Physics and Center of Mathematics for Applications, 
University of Oslo, N-0316 Oslo, Norway}
\affiliation{Centre of Mathematics for Applications, University of Oslo, 
N-0316 Oslo, Norway} 


\author{T.~Papenbrock}
\affiliation{Physics Division, Oak Ridge National Laboratory,
P.O. Box 2008, Oak Ridge, TN 37831, USA}
\affiliation{Department of Physics and Astronomy, University of
Tennessee, Knoxville, TN 37996, USA}


\author{A.~Schwenk}
\affiliation{TRIUMF, 4004 Wesbrook Mall, Vancouver, BC V6T 2A3, Canada}

\author{A.~Nogga}
\affiliation{Institut f\"r Kernphysik, Forschungszentrum J\"ulich, D-52428 J\"ulich, Germany}


%
\begin{abstract}
We apply \emph{ab-initio} coupled cluster theory 
to the  description of loosely bound and unbound nuclei starting from 
realistic nucleon-nucleon (NN) forces. 
In order to keep the basis size manageable
we use a renormalized interaction of the low-momentum type derived
from the bare NN interaction.
Coupled cluster calculations with singles and
doubles excitations (CCSD) are performed for the ground states of the
helium isotopes $^{3-10}$He. Loosely bound and unbound nuclei exhibit
strong coupling through continuum degrees of freedom. In order to
account properly for this non-neglible coupling, a Berggren 
basis which treats bound, resonant and non-resonant continuum states 
on equal footing is used. Starting from a Berggren basis, and a
realistic nucleon-nucleon interaction  we
are able to compute lifetimes  and decay widths of a whole isotopic
chain for the first time! Our results for the helium chain are in 
semi-quantitative agreement with experiment. The results show
underbinding of the whole isotopic chain which is an 
indication of missing three-nucleon forces (3NF's).

We have derived and implemented coupled cluster equations 
for three-body Hamiltonians. 
We employ a low-momentum version of the Argonne V18
nucleon-nucleon interaction augmented by a three-body 
interaction and compute the binding energy of $^4$He.
The results show that the main contribution from the three-nucleon
force to the binding energy consists of its density-dependent zero-,
one-, and two-body terms that result from the normal ordering of the
Hamiltonian employed in coupled-cluster theory. The contribution from
the normal ordered three-body part of the 3NF is more than an 
order of magnitude
smaller than corrections due to the perturbative triples (CCSD(T)). 
This finding is very promising for the study of loosely bound and
unbound nuclei and in the study of the role of 3NF's in heavier nuclei. The
effects of 3NF's can be effectively taken into account at the zero-, one-, and
two-body normal ordered level. This renders the equations/calculations 
no more difficult than starting from a Hamiltonian consisting of only
NN-forces. 
\end{abstract}

\maketitle

\end{document}
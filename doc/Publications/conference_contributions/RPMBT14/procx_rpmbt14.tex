%\documentclass[draft]{ws-procs975x65}
%\documentclass[square]{ws-procs975x65}
\documentclass{ws-procs975x65}

\begin{document}


\title{Coupled-cluster approach to an {\it ab-initio} description of nuclei}

\author{D.~J.~Dean$^{1,2}$, G.~Hagen$^{1,2,3}$, M.~Hjorth-Jensen$^{2,4}$, and T.~Papenbrock$^{1,3}$}

\address{$^1$Physics Division, Oak Ridge National Laboratory, \\ P.O. Box 2008,
Oak Ridge, TN 37831, U.S.A.} 

\address{$^2$Center of Mathematics for Applications,
University of Oslo, N-0316 Oslo, Norway }

\address{$^3$Department of Physics and Astronomy, University of Tennessee,  \\
  Knoxville, Tennessee 37996, U.S.A. } 


\address{$^4$Department of Physics, University of Oslo, N-0316 Oslo, Norway}



\begin{abstract}
We presents results from {\it ab-initio} coupled-cluster theory for stable, 
resonant, and weakly bound nuclei. Results for the 
chain of helium isotopes $^{4-10}$He,
$^{16}$O, and $^{40}$Ca are discussed.  
\end{abstract}

\keywords{Nuclear structure calculations; Coupled-cluster theory; Weakly bound nuclei; Ground state resonances}

\bodymatter

\section{Introduction}

The theoretical description of bound, weakly bound, and 
unbound quantum many-body systems, together with present and planned experimental studies
of such systems, represents a  great challenge to our understanding of nuclear systems.
Experiments in nuclear physics will address such important topics
as how shells evolve, the role of many-body correlations, and the position of
the stability lines of nuclei.
The proximity of the scattering continuum
in these systems implies that they should be treated as
open quantum systems where coupling with the scattering
continuum can take place. This means that a many-body formalism
should contain resonant and continuum states in the basis in order to describe
loosely bound systems or unbound systems.
Extending the single-particle basis to include such degrees of freedom results in intractable
dimensionalities for traditional configuration interaction methods (shell-model  in nuclear physics)  approaches.

Shell-model codes tailored to the nuclear 
many-body problem can today reach dimensionalities of approximately $10^{10}$ basis states. Some of the systems studied
here exhibit dimensionalities of some $10^{60}$ basis states.
To circumvent this dimensionality problem, we have built a nuclear many-body
program based on the coupled-cluster methods. Coupled-cluster theories allow for numerical cost-efficient ways of
dealing with large dimensionalities compared with traditional configuration interaction methods.

We report here new results from coupled-cluster theories including  both bound, resonant, and continuum
states\cite{hagen1,hagen2,hagen3,hagen4}.  
We also show that coupled-cluster theories reproduce benchmark results for light nuclei with minimal numerical cost  
and  provide benchmarks for heavier nuclei.


\section{Results and discussions}

In addition to the above dimensionality problems, the nuclear many-body problem is riddled by the fact that
there is no analytic expression for the underlying nucleon-nucleon (NN) interaction. 
Furthermore, three-body interactions are
important in nuclear physics and need to be included in a systematic way in a many-body formalism.
In recent years, quite a lot of progress has been made within  chiral effective field theories
to construct NN and three-nucleon interactions from the underlying symmetries of QCD.  
The starting point is then a chiral effective Lagrangian with nucleons and pions as effective degrees of freedom only.
Three-body interactions emerge naturally and have explicit expressions at 
every order in the chiral perturbation theory expansion. 
In this work we have chosen to work with a nucleon-nucleon interaction derived from 
effective-field theory,
such as the N$^3$LO model of Entem and Machleidt. 
In addition, we have also used the more phenomenological $V_{18}$ 
interaction. 

We renormalize the short-range part of the nucleon-nucleon interaction by a similarity transformation
technique in momentum space.\cite{hagen4}
This renormalized  interaction defines our  Hamiltonian 
which enters the solution of the coupled-cluster equations. 

To obtain ground-state energies of both bound and 
weakly bound systems, we need a many-body scheme which is
(i) fully microscopic and size extensive, (ii) allows for inclusion, in 
a systematic way, various many-body
correlations to be summed to infinite order, (iii) can account for 
the description of both closed-shell systems and valence systems, 
and  (iv)  capable
to describe both bound and weakly bound systems.
Coupled-cluster theories allow for the inclusion of  all these features. 

Our coupled-cluster approaches include $1p-1h$ and $2p-2h$ correlations, normally dubbed 
single and double excitations (CCSD). Correlations of the $3p-3h$ type are included perturbatively
(labelled CCSD(T)) or via other approximations to the full $3p-3h$ correlations (CCSDT). 
Furthermore, for weakly bound systems, we 
employ  complex Gamow-Hartree-Fock single-particle basis and an effective 
interaction  defined by such a single-particle basis \cite{hagen1,hagen2}

In the left panel of Fig.~\ref{fig:he4} we show the 
coupled-cluster results 
for $^4$He and compare them with results from
few-body calculations. There is excellent agreement, 
showing that coupled-cluster results reproduce other
ab initio results with a much smaller numerical cost.  
The right panel shows the corresponding results for
$^{16}$O, providing a benchmark for this nucleus. The results are given as a 
function of the number of oscillator
shells, limited by $2n+l$.\cite{hagen2}. See Ref.~\cite{hagen2} for further details.
\begin{figure}[b]%
\begin{center}
  \parbox{2.1in}{\epsfig{figure=fig1a.eps,width=2in}}
  \hspace*{4pt}
  \parbox{2.1in}{\epsfig{figure=fig1b.eps,width=2in}}
  \caption{Left figure is the binding energy for $^4$He as a function of the number of oscillator shells $N=2n+l$.
The maximum orbital momentum was set to $l=7$. 
The right panel exhibits the corresponing result for $^{16}$O.  Taken from Ref.~\cite{hagen2}.}%
  \label{fig:he4}
\end{center}
\end{figure}
The $^{16}$O results show an overbinding,which most likely is due to omitted three-body interactions.
Fig.~\ref{fig:ox16} shows results for $^{16}$O (left panel) and $^{40}$Ca as a functions of the oscillator
energy $\hbar\omega$ used in computing the oscillator wave 
function and the number of major shells $N$
used in the coupled-cluster calculations. 
\begin{figure}[b]%
\begin{center}
  \parbox{2.1in}{\epsfig{figure=fig2a.eps,width=2in}}
  \hspace*{4pt}
  \parbox{2.1in}{\epsfig{figure=fig2b.eps,width=2in}}
  \caption{Left figure is the binding energy for $^{16}$O as function of the number of oscillator shells $N=2n+l$ and oscillator energy $\hbar\omega$. The maximum orbital momentum was set to $l=7$. 
The right panel is the corresponing result for $^{40}$Ca. Taken from Ref.~\cite{hagen2}.}%
  \label{fig:ox16}
\end{center}
\end{figure}
As expected, with increasing size 
of the model space, the results stabilize as a function of the 
chosen oscillator energy.  Our results are converged with a given two-body Hamiltonian and we can therefore claim
that lack of agreement with experiment is due to missing physics, such as three-body interactions, in our Hamiltonian.

In Fig.~\ref{fig:hechain} we present our recent CCSD results~\cite{hagen3}
for the chain
of helium isotopes using a complex single-particle basis.   
The largest model space has
$850$ single-particle orbitals, distributed among $5s5p5d4f44h4i$ proton orbitals
and  $20s20p5d4f44h4i$ neutron orbitals. For $^{10}$He this results in approximately $10^{22}$ basic states. 
\begin{figure}[t]%
\begin{center}
\epsfig{figure=fig3.eps,width=3in}
\caption{CCSD calculation of the $^{3-10}$He ground states with the
    low-momentum N$^3$LO nucleon-nucleon interaction for an increasing
    number of partial waves. Our calculated width of $^{10}$He is $\approx 0.002$MeV. TNF stands for three-body forces while
triples are three-body correlations not included here. Taken from Ref.~\cite{hagen3}.  \label{fig:hechain}}
\end{center}
\end{figure} 
These are the first ever {\em ab initio} calculations of weakly bound isotopes and we see that with a two-body Hamiltonian
we are able to reproduce correctly the experimental trend and predict correctly which nuclei have bound ground states and 
which are resonances.  The results are converged within our chosen model spaces.
The quantitative lack of agreement with experiment is due to our omission of three-body interactions.

In summary, coupled-cluster theories hold great promise for a quantitative understanding of nuclei. With the possibility
to include three-body interactions\cite{hagen1}, we may be able to 
tell how nuclei evolve as one moves towards the drip line.

\section{Acknowledgments}

This work was supported in part by the U.S.~Department
of Energy under Contract Nos.~DE-AC05-00OR22725 (Oak Ridge National
Laboratory), DE-FG02-96ER40963 (University of Tennessee),
DE-FG05-87ER40361 (Joint Institute for Heavy Ion Research), 
DE-FC02-07ER41457 (University of Washington)
and by the
Research Council of Norway (Supercomputing grant
NN2977K). Computational resources were provided by the Oak Ridge
Leadership Class Computing Facility and the National Energy Research
Scientific Computing Facility.  Discussions with A.~Schwenk are acknowledged.

\begin{thebibliography}{10}
\bibitem{hagen1}
G.~Hagen, T.~Papenbrock, D.~J.~Dean, A.~Schwenk, M.~W{\l}och, P.~Piecuch, and
  A.~Nogga, Phys.~Rev.~C {\bf 76}, 034302 (2007).

\bibitem{hagen2}
G.~Hagen, D.~J.~Dean, M.~Hjorth-Jensen, T.~Papenbrock, and A.~Schwenk, Phys.~Rev.~C {\bf 76}, in press (2007).

\bibitem{hagen3}
G.~Hagen, D.~J. Dean, M.~Hjorth-Jensen, and T.~Papenbrock, Phys.~Lett.~B {\bf 655}, in press (2007).

\bibitem{hagen4}
G.~Hagen, M.~Hjorth-Jensen, and N.~Michel, Phys.~Rev. C {\bf 73}, 064307 (2006).
\end{thebibliography}


\bibliographystyle{ws-procs975x65}
\bibliography{sample}

\end{document}















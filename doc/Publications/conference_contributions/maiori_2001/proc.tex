%%only change the website address - 18/4/2000
%%%%%%%%%%%%%%%%%%%%%%%%%%%%%%%%%%%%%%%%%%%%%%%%%%%%%%%%%%%%%%%%%%%%%%%%%%
%%
%% ws-p8-50x6-00.tex : 20-11-97
%% This Latex2e file rewritten from various sources for use in the
%% preparation of the (smaller [8.50''x6.00'']) single-column proceedings 
%% Volume, latest version by R. Sankaran with acknowledgements to Susan 
%% Hezlet and Lukas Nellen. Please comments to:rsanka@wspc.com.sg
%%
%%%%%%%%%%%%%%%%%%%%%%%%%%%%%%%%%%%%%%%%%%%%%%%%%%%%%%%%%%%%%%%%%%%%%%%%%%
%
\documentclass{ws-p8-50x6-00}
\usepackage{pst-plot}
\usepackage{epsf}
\begin{document}

\title{Large-Scale Realistic Nuclear Structure Studies in the Sn-Region}

\author{T.~Engeland, M.~Hjorth-Jensen, and E.~Osnes}

\address{Department of Physics, University of Oslo, Oslo, Norway}

\author{A.~Holt}

\address{Oslo University College, Oslo, Norway}  

%%%%%%%%%%%%%%%%%%%%%%%%%%%%%%%%%%%%%%%%%%%%%%%%%%%%%%%%%%%%%%
% You may repeat \author \address as often as necessary      %
%%%%%%%%%%%%%%%%%%%%%%%%%%%%%%%%%%%%%%%%%%%%%%%%%%%%%%%%%%%%%%

\maketitle

\abstracts{The Sn isotopes from nucleon number $A = 100$ to $A = 132$, with valence neutrons filling the $N = 4$ major shell, provide a challenging testing ground for large-scale shell-model calculations and the accompanying effective interactions. We report calculations using realistic effective interactions obtained from modern meson-exchange nucleon-nucleon potentials using many-body perturbation theory. The gross properties of the energy spectra, including the approximately constant $2^+ - 0^+$ spacing characteristic of generalized seniority, are generally well reproduced. We have found that this pairing property may be attributed to the strong $^1S_0$ partial wave of the nucleon-nucleon interaction. We also present some results for more complicated neighbouring nuclides in which both valence protons and neutrons are present.}

\section{Introduction}
Large-scale shell-model calculations employing realistic effective 
interactions obtained from many-body theory have been rather successful 
in describing nuclear structure data both near and somewhat away from 
closed shells. In recent years such model calculations have been put 
to crucial test particularly in the Sn region, where experimental 
advances have provided us with a wealth of new data for nuclei 
with unusual proton and neutron numbers. In  fact, level spectra 
are now available for essentially all the Sn isotopes from 
nucleon number $A = 100$ to $A = 132$, obtained by filling valence 
neutrons into the $N = 4$ major shell. This range of nuclei thus provides a 
testing ground for shell-model calculations in which one may study 
the transition from single-particle degrees of freedom to 
more collective ones away from closed shells, as well 
as the origin of possible underlying symmetries such as 
generalized seniority. Similarly, the N = 82 isotones beyond $^{132}$Sn 
provide us with the analogous valence proton spectra. 
In both these cases, however, only like valence nucleons are considered 
and thus only the $T = 1$ part of the effective interaction is at play. 
A further challenge is met when both valence protons and neutrons 
are present. Calculational efforts are under way to 
match the extensive experimental work in progress on 
proton-rich nuclei around $A = 100$ with both kinds of valence nucleons.

The outline of this paper is as follows. In Sect. 2 we briefly 
review our theoretical framework, including the shell model 
and the effective interaction. In Sect. 3 we present selected 
results for the Sn isotopes and discuss the origin of the manifest 
pairing properties of these nuclei. Then, in Sect. 4 we summarize 
briefly results obtained for nuclei in this mass region 
with both kinds of valence particles. 
Outlook and perspectives are presented in the final Sect. 5.

\section{Theoretical framework}
\subsection{The nuclear shell model}
We shall take the nuclear shell model as the microscopic 
model of nuclear structure. The shell model describes the 
properties of nuclei in terms effective interactions 
among constituent nucleons filling the valence shell  
beyond the a closed-shell core. In the present case of the Sn 
isotopes, we assume that the valence particles are filling 
the $N = 4$ major shell consisting of the single-particle orbitals 
$2s_{1/2}$, $1d_{5/2}$, $1 d_{3/2}$, $0g_{7/2}$ along with the intruding 
orbital $0h_{11/2}$ from the $N = 5$ shell. These orbitals may accomodate 
as many as 32 valence nucleons, in which case the next closed-shell 
nucleus $^{132}$Sn is obtained. The maximum dimensionality of the 
shell-model eigenvalue problem is obtained at half filling of 
the $N = 4$ shell, namely for 16 valence nucleons giving rise to $^{116}$Sn. 
Working in an $m$-scheme basis, giving up total angular momentum 
as a good quantum number from the outset, we have to deal with 
approximately 16 million states. This allows us to represent 
each basis Slater determinant with a 32-bits computer word, in which 
an occupied stated is represented by 1 and an empty state by 0. 
Thus, operating on these states with fermion creation and 
annihilation operators appearing in the nuclear Hamiltonian amount 
simply to moving 1's and 0's around. Starting from the original basis 
vectors, we construct, as described elsewhere, new basis vectors 
using the Lanzocs algorithm. This procedure is repeated until the 
Hamiltonian matrix is obtained on tridiagonal form and can be 
diagonalized by conventional methods. Various precautions may be 
taken to speed up convergence when the dimensionality gets big. 
Although the basis vectors are constructed in $m$-schene, 
the final eigenvectors will have good angular momentum.

\subsection{The effective interaction}
Although the shell-model space employed is large, it is 
nevertheless severely truncated and it is necessary to use an 
effective interaction to account for the degrees of freedom 
which are ignored. It is our aim to calculate this as rigourously 
as possible using many-body perturbation theory and then use is 
without further adjustment in the shell model. Acknowledging 
that it is not yet possible to obtain a realistic 
nucleon-nucleon interaction directly from QCD, we have chosen 
to start from modern one-boson-exchange potentials such as the 
Bonn potentials, see Ref. \cite{ruprecht}. These potentials are too large at small distances 
to be applied directly to perturbative calculations. Thus, they have 
to be regularized at short distance by integrating out the 
high-momentum states to yield a well-behaved Bethe-Brueckner-Goldstone 
reaction matrix denoted as $G$-matrix. Then, the long-range 
correlations may be taken into account using perturbative methods. 
This is done by first evaluating the non-folded diagrams 
(Q-box interaction) to third order and then by folding these 
onto themselves to arbitrary order. For further details we 
refer to extensive reviews on the topic, see e.g.,  Ref. \cite{hko95}
and references therein.

\section{Application to Sn isotopes}
The effective interaction obtained from many-body theory 
contains one-body, two-body, three-body etc. terms, up to as 
many valence particles as one considers. It is customary to add 
the one-body term to the unperturbed Hamiltonian and take the 
eigenvalues of this effective one-body Hamiltonian from the 
experimental single-particle energies. Further, one ignores the 
third- and higher-body terms, so that the only piece to calculate 
is the two-body effective interaction (which, of course, is hard enough). 
In order to start from a $^{100}$Sn core, one needs the $^{101}$Sn single-particle 
energies which unfortunately are not known. Thus, one has had to rely 
on a combination of mean-field calculations and on relevant data 
from other nuclei in the vicinity. An alternative is to start from $^{132}$Sn 
as a core and add holes in the $N = 4$ valence shell. This has the 
advantage that the single-hole energies are better, although not 
entirely known. In this case, one has to use an effective 
interaction which is calculated with respect to the $^{132}$Sn core. 
In fact, to eliminate uncertainties and test the consistency of our 
calculation, we have taken both approaches. Starting from the $^{131}$Sn 
single-hole energies, we obtain the $^{101}$Sn single-particle energies 
shown in Fig. 1 and which are in qualitative agreement with 
the "empirical" $^{101}$Sn single-particle energies. 
Then, starting from the latter, we obtain the $^{131}$Sn 
single-hole energies which compare reasonably well with the 
empirical ones, as shown in Fig. 2. 

%
\begin{figure}
\setlength{\unitlength}{1.0cm}
\begin{center}

\Cartesian(1cm,0.9cm)
%
\pspicture(0,1)(12,12)
%
%
%	single-particle spectrum
%
\psline[linewidth=1pt](2,3.5)(4,3.5)
\uput[0](0.7,3.2){$d_{5/2}^{+}$}

%    \uput[0](4.1,3.3){\small 0.00 MeV}

%
\psline[linewidth=1pt](2,3.65)(4,3.65)
\uput[0](0.7,4.2){$g_{7/2}^{+}$}
\uput[0](2.2,3.8){\small 0.08 MeV}
%
\psline[linewidth=1pt](2,8.17)(4,8.17)
\uput[0](0.7,8.0){$s_{1/2}^{+}$}
\uput[0](2.2,7.95){\small 2.45 MeV}
%
\psline[linewidth=1pt](2,8.36)(4,8.36)
\uput[0](0.7,8.7){$d_{3/2}^{+}$}
\uput[0](2.2,8.55){\small 2.55 MeV}
%
\psline[linewidth=1pt](2,9.6)(4,9.6)
\uput[0](0.7,9.6){$h_{11/2}^{-}$}
\uput[0](2.2,9.8){\small 3.2 MeV}
%

\uput[0](1.9,2.5){\small ``Experiment''}
%

%%%%%%%%%%%%%%%%%%%%%%%%%%%

%	single-particle spectrum
%
\psline[linewidth=1pt](8,3.5)(10,3.5)
%	\uput[0](6.7,3.3){$d_{5/2}^{+}$}
\uput[0](10.1,3.6){\small 0.00 MeV}
%
\psline[linestyle=dashed,dotsep=1pt](4.2,3.5)(7.8,3.5)
%


\psline[linewidth=1pt](8,3.19)(10,3.19)
%	\uput[0](6.7,4.0){$g_{7/2}^{+}$}
\uput[0](10.1,3.1){\small -0.16 MeV}
%
\psline[linestyle=dashed,dotsep=1pt](4.2,3.65)(7.8,3.19)
%


\psline[linewidth=1pt](8,6.55)(10,6.55)
%	\uput[0](6.7,6.51){$s_{1/2}^{+}$}
\uput[0](10.1,6.8){\small1.58 MeV}
%
\psline[linestyle=dashed,dotsep=1pt](4.2,8.17)(7.8,6.45)
%

\psline[linewidth=1pt](8,6.45)(10,6.45)
%	\uput[0](6.7,8.6){$d_{3/2}^{+}$}
\uput[0](10.1,6.45){\small 1.57 MeV}
%
\psline[linestyle=dashed,dotsep=1pt](4.2,8.36)(7.8,6.45)
%

\psline[linewidth=1pt](8,6.11)(10,6.11)
%	\uput[0](6.7,9.6){$h_{11/2}^{-}$}
\uput[0](10.1,6.11){\small 1.37 MeV}
%

\psline[linestyle=dashed,dotsep=1pt](4.2,9.6)(7.8,6.11)


\uput[0](7.8,2.5){\small 31 hole-calculation}
%

%%%%%%%%%%%%%%

\endpspicture
\end{center}
\caption{The calculated spectrum of $^{101}$Sn based
on $^{132}$Sn as closed shell core.}
\end{figure}

%
\begin{figure}
\setlength{\unitlength}{1cm}
\begin{center}

\Cartesian(1cm,0.9cm)
%
\pspicture(0,1)(12,12)
%

%	single-particle spectrum
%
\psline[linewidth=1pt](1.7,3.5)(3.45,3.5)
\uput[0](0.45,3.4){$d_{3/2}^{+}$}
\uput[0](2.0,3.7){\small 0.00 MeV}
%

\psline[linewidth=1pt](1.7,4.04)(3.45,4.04)
\uput[0](0.45,4.1){$h_{11/2}^{-}$}
\uput[0](2.0,4.3){\small 0.24 MeV}
%

\psline[linewidth=1pt](1.7,5.46)(3.45,5.46)
\uput[0](0.45,5.46){$s_{1/2}^{+}$}
\uput[0](2.0,5.7){\small 0.82 MeV}
%

\psline[linewidth=1pt](1.7,7.51)(3.45,7.51)
\uput[0](0.45,7.51){$d_{5/2}^{+}$}
\uput[0](2.0,7.8){\small 1.66 MeV}
%

\psline[linewidth=1pt](1.7,9.5)(3.45,9.5)
\uput[0](0.45,9.4){$g_{7/2}^{+}$}
\uput[0](2.0,9.7){\small 2.43 MeV}
%
\uput[0](1.9,2.2){\small Experiment}
%


%%%%%%%%%%%%%%%%%%%%%%%%%%%%%%%%%%

%%	single-particle spectrum
%
\psline[linewidth=1pt](7.7,3.5)(9.45,3.5)
%	\uput[0](6.45,3.45){$d_{3/2}^{+}$}
\uput[0](9.5,3.45){\small 0.00 MeV}
%
\psline[linestyle=dashed,dotsep=1pt](3.5,3.5)(7.6,3.5)

\psline[linewidth=1pt](7.7,3.0)(9.45,3.0)
%	\uput[0](6.45,3.0){$h_{11/2}^{-}$}
\uput[0](9.5,3.0){\small -0.20 MeV}
%
\psline[linestyle=dashed,dotsep=1pt](3.5,4.04)(7.6,3.0)

\psline[linewidth=1pt](7.7,4.64)(9.45,4.64)
%	\uput[0](6.45,5.46){$s_{1/2}^{+}$}
\uput[0](9.5,4.64){\small 0.47 MeV}
%
\psline[linestyle=dashed,dotsep=1pt](3.5,5.46)(7.6,4.64)

\psline[linewidth=1pt](7.7,9.42)(9.45,9.42)
%	\uput[0](6.45,7.51){$d_{5/2}^{+}$}
\uput[0](9.5,9.42){\small 2.44 MeV}
%
\psline[linestyle=dashed,dotsep=1pt](3.5,7.51)(7.6,9.42)

\psline[linewidth=1pt](7.7,8.28)(9.45,8.28)
%	\uput[0](6.45,9.4){$g_{7/2}^{+}$}
\uput[0](9.5,8.28){\small 1.97 MeV}
%
\psline[linestyle=dashed,dotsep=1pt](3.5,9.5)(7.6,8.28)

\uput[0](7.8,2){\small 31 particle-calculation}
%

%%%%%%%%%%%%%%

\endpspicture
\end{center}
\caption{The calculated spectrum of $^{131}$Sn based
on $^{100}$Sn as closed shell core.}
\end{figure}
Similar correspondances are obtained for the two-particle and 
two-hole spectra.

In Fig. 3 we show typical results for many-particle spectra, 
represented by $^{122}$Sn, 
which has 10 valence holes with respect to a $^{132}$Sn core. 
A satisfactory qualitative agreement is indeed obtained.

\begin{figure}
%
\setlength{\unitlength}{1cm}
\begin{center}

\Cartesian(1cm,3cm)
%
\pspicture(0,0)(12,3)

%
%	The experimental positive 122^Sn particle spectrum
%

\psaxes[Ox=0,Dx=2,dx=1,Oy=0, Dy=1,showorigin=false,linewidth=1pt](0,0)(0,3.2)
\psline[linewidth=1pt](-0.05,0)(+0.05,0)
%

\uput[0](2.0,-0.1){\small Experiment}
%
\uput[0](-0.5,3.3){\small MeV}
%
%%%%%%%%   J = 0  %%%%%%%%
\psline[linewidth=1pt](2,0)(4,0)
\uput[0](0.7,0){\small 0$^{+}$}
%

\psline[linewidth=1pt](2,2.09)(4,2.09)
\uput[0](0.7,2.04){\small 0$^{+}$}
%

\psline[linewidth=1pt](2,2.67)(4,2.67)
\uput[0](0.7,2.6){\small 0$^{+}$}

%%%%%%%%   J = 2  %%%%%%%%
\psline[linewidth=1pt](2,1.14)(4,1.14)
\uput[0](0.7,1.14){\small 2$^{+}$}
%
\psline[linewidth=1pt](2,2.15)(4,2.15)
\uput[0](0.8,2.2){\small 2$^{+}$}
%
%%%%%%%%   J = 4  %%%%%%%%
\psline[linewidth=1pt](2,2.14)(4,2.14)
\uput[0](0.4,2.14){\small 4$^{+}$}
%
\psline[linewidth=1pt](2,2.33)(4,2.33)
\uput[0](0.7,2.33){\small 4$^{+}$}
%
%%%%%%%%   J = 6  %%%%%%%%
\psline[linewidth=1pt](2,2.56)(4,2.56)
\uput[0](0.7,2.5){\small 6$^{+}$}
%
%%%%%%%%   J = 8  %%%%%%%%
\psline[linewidth=1pt](2,2.69)(4,2.69)
\uput[0](0.7,2.72){\small 8$^{+}$}
%
%%%%%%%%   J = 10  %%%%%%%%
\psline[linewidth=1pt](2,2.78)(4,2.78)
\uput[0](0.7,2.82){\small 10$^{+}$}
%

%%%%%%%%%%%%%%%%%%%%%%%%%%%

%
%   The calculated positive 122^Sm spectrum 
%

\uput[0](8.0,-0.1){\small Calculation}
%

%%%%%%%%   J = 0  %%%%%%%%
\psline[linewidth=1pt](8,0)(10,0)
%	\uput[0](6.7,3){\small 0$^{+}$}
\psline[linestyle=dashed,dotsep=1pt](4.2,0)(7.8,0)

%
\psline[linewidth=1pt](8,2.41)(10,2.41)
%	\uput[0](6.7,7.88){\small 0$^{+}$}
\psline[linestyle=dashed,dotsep=1pt](4.2,2.09)(7.8,2.41)

%
\psline[linewidth=1pt](8,2.80)(10,2.80)
%	\uput[0](6.7,9.23){\small 0$^{+}$}
\psline[linestyle=dashed,dotsep=1pt](4.2,2.67)(7.8,2.80)

%%%%%%%%   J = 2  %%%%%%%%
\psline[linewidth=1pt](8,1.15)(10,1.15)
%	\uput[0](0.7,5.68){\small 2$^{+}$}
\psline[linestyle=dashed,dotsep=1pt](4.2,1.14)(7.8,1.15)

%
\psline[linewidth=1pt](8,2.15)(10,2.15)
%	\uput[0](0.7,8.02){\small 2$^{+}$}
\psline[linestyle=dashed,dotsep=1pt](4.2,2.15)(7.8,2.15)

%
%%%%%%%%   J = 4  %%%%%%%%
%
\psline[linewidth=1pt](8,2.30)(10,2.30)
%	\uput[0](0.7,8.86){\small 4$^{+}$}
\psline[linestyle=dashed,dotsep=1pt](4.2,2.14)(7.8,2.30)

%

\psline[linewidth=1pt](8,2.51)(10,2.51)
%	\uput[0](0.7,8.86){\small 4$^{+}$}
\psline[linestyle=dashed,dotsep=1pt](4.2,2.33)(7.8,2.51)

%
%%%%%%%%   J = 6  %%%%%%%%
\psline[linewidth=1pt](8,2.78)(10,2.78)
%	\uput[0](0.7,9.49){\small 6$^{+}$}
\psline[linestyle=dashed,dotsep=1pt](4.2,2.56)(7.8,2.78)
%
%%%%%%%%   J = 8  %%%%%%%%
\psline[linewidth=1pt](8,2.88)(10,2.88)
%	\uput[0](0.7,9.72){\small 8$^{+}$}
\psline[linestyle=dashed,dotsep=1pt](4.2,2.69)(7.8,2.88)

%
%%%%%%%%   J = 10  %%%%%%%%
\psline[linewidth=1pt](8,2.95)(10,2.95)
%	\uput[0](0.7,9.88){\small 10$^{+}$}
\psline[linestyle=dashed,dotsep=1pt](4.2,2.78)(7.8,2.95)
%

%%%%%%%%%%%%%%%%%%%%%%%%%%%

\endpspicture
\end{center}

%%%%%%%%%%%%%%%%%%% end figure %%%%%%%%%%%%%%%%%%


%%%%%%%%%%%%%%% New figure %%%%%%%%%%%%%
%
\setlength{\unitlength}{1cm}
\begin{center}

\Cartesian(1cm,3cm)
%
\pspicture(0,0)(12,2.5)

%
%	The experimental negative  122^Sn particle spectrum
%

\psaxes[Ox=0,Dx=2,dx=1,Oy=2, Dy=1,showorigin=false,linewidth=1pt](0,0)(0,2.2)
\psline[linewidth=1pt](-0.05,0)(+0.05,0)
%

\uput[0](2.0,-0.1){\small Experiment}
%
\uput[0](-0.5,2.3){\small MeV}
%
%%%%%%%%   J = 3-  %%%%%%%%
\psline[linewidth=1pt](2,0.49)(4,0.49)
\uput[0](0.7,0.52){\small 3$^{-}$}
%

\psline[linewidth=1pt](2,1.36)(4,1.36)
\uput[0](0.7,1.36){\small 3$^{-}$}
%

%%%%%%%%   J = 5-  %%%%%%%%
\psline[linewidth=1pt](2,0.25)(4,0.25)
\uput[0](0.7,0.25){\small 5$^{-}$}
%
\psline[linewidth=1pt](2,0.75)(4,0.75)
\uput[0](0.8,0.75){\small 5$^{-}$}
%

%%%%%%%%   J = 7-  %%%%%%%%
\psline[linewidth=1pt](2,0.41)(4,0.41)
\uput[0](0.7,0.38){\small 7$^{-}$}
%
%%%%%%%%%%%%%%%%%%%%%%%%%%%

%
%   The calculated negative 122^Sm spectrum 
%

\uput[0](8.0,-0.1){\small Calculation}
%

%%%%%%%%   J = 3-  %%%%%%%%
\psline[linewidth=1pt](8,0.90)(10,0.90)
\psline[linestyle=dashed,dotsep=1pt](4.2,0.49)(7.8,0.90)

%
\psline[linewidth=1pt](8,1.47)(10,1.47)
\psline[linestyle=dashed,dotsep=1pt](4.2,1.36)(7.8,1.47)

%%%%%%%%   J = 5-  %%%%%%%%
\psline[linewidth=1pt](8,0.55)(10,0.55)
\psline[linestyle=dashed,dotsep=1pt](4.2,0.25)(7.8,0.55)

%
\psline[linewidth=1pt](8,0.96)(10,0.96)
\psline[linestyle=dashed,dotsep=1pt](4.2,0.75)(7.8,0.96)

%
%%%%%%%%   J = 7-  %%%%%%%%
%
\psline[linewidth=1pt](8,0.74)(10,0.74)
\psline[linestyle=dashed,dotsep=1pt](4.2,0.41)(7.8,0.74)
%

%%%%%%%%%%%%%%%%%%%%%%%%%%%

\endpspicture
\end{center}
\caption{Experimental and theoretical spectra of $^{122}$Sn based
on $^{132}$Sn as closed shell core.}

\end{figure}




In Table 1  we compare the calculated and experimental $2^+ - 0^+$ 
spacings for the even Sn isotopes with mass numbers from $A=116$
to $A=130$. We reproduce in fact 
the nearly constant spacing, indicating that generalized seniority is a 
good quantum number. This may be further tested by defining 
the two-particle(hole) $0^+$ and $2^+$ states to have generalized 
seniority 0 and 2, respectively. Then, by adding pairs of  
particles (or holes) to these, one can construct seniority 0 and 2 
states in the other isotopes. These are found to have an overlap squared 
from $80$ to $100\%$ with the full shell-model states. Similarly, 
the single-quasiparticle states in the odd-mass isotopes are found to 
have substantial overlaps with the seniority 1 states. However, for the 
higher excited states the seniority approximation quickly breaks down 
and is of little use in characterizing the states.

There are probably two dynamic effects contributing to the 
pairing property of the $2^+ - 0^+$ spacing. One is due to the single-particle 
Hamiltonian providing nearly degenrate single-particle orbits 
with high occupancy, such as $1d_{5/2}$ and  $0g_{7/2}$ at the beginning 
of the shell and $1d_{3/2}$ and $0h_{11/2}$ at the end of the shell. 
The other is due to the two-particle interaction, as can be seen 
from Table 1, where we have calculated the $2^+ - 0^+$ spacing 
for selected partial waves of the nucleon-nucleon interaction. 
It turns out that the $^1S_0$ partial wave plays a crucial role in 
creating the pairing gap. On the contrary, by taking out 
the $^1S_0$ and $^3P_2$ channels, which are known to be responsible for 
pairing in nuclear and neutron matter, 
the pairing gap breaks down and the ground state is not always $0^+$.
\begin{table}[hbt]
\begin{center}
\caption{ $2^+_1-0^+_1$ excitation energy for the 
even tin isotopes $^{130-116}$Sn for various approaches
to the effective interaction. See text for further details. 
Energies are given in MeV. }\footnotesize
\begin{tabular}{lcccccccc}\hline
 & {$^{116}$Sn} & {$^{118}$Sn} & {$^{120}$Sn} &{$^{122}$Sn} & {$^{124}$Sn} & {$^{126}$Sn} & {$^{128}$Sn} & {$^{130}$Sn} \\ \hline
Expt & 1.29 & 1.23 & 1.17 & 1.14 & 1.13 & 1.14 & 1.17 & 1.23 \\
$V_{\mathrm{eff}}$ & 1.17 & 1.15 & 1.14 & 1.15 & 1.14 & 1.21 & 1.28 & 1.46 \\
$G$-matrix &1.14 & 1.12& 1.07 & 0.99 & 0.99 & 0.98 & 0.98 & 0.97  \\
$^1S_0$ $G$-matrix &1.38 &1.36 &1.34 &1.30 & 1.25& 1.21 &1.19 &1.18 \\
No $^1S_0$ \& $^3P_2$ in $G$ &     &     &     &      &0.15 &-0.32 &0.02 &-0.21  \\\hline
\end{tabular}
\end{center}
\label{tab:table1}
\end{table}
We note from this table that the three first cases nearly produce a constant 
$2^+_1-0^+_1$ excitation energy, with our most optimal effective interaction
$V_{\mathrm{eff}}$ being closest the experimental data. The bare $G$-matrix
interaction, with no folded diagrams as well, results in a slightly more compressed
spacing. This is mainly due to the omission of the core-polarization 
diagrams which typically render the $J=0$ matrix elements more attractive.
Such diagrams are included in $V_{\mathrm{eff}}$. 
Including only the $^1S_0$ partial wave in the construction of the  $G$-matrix
(case 3),
yields in turn a somewhat larger spacing. This can again be understood from the
fact that a $G$-matrix constructed with this partial wave  
only does not receive contributions from any entirely repulsive partial wave.
It should be noted that our optimal interaction, as demonstrated in 
Ref.\ \cite{ehho98}, shows a rather good reproduction of the 
experimental spectra for both even and odd nuclei. Although the approximations
made in cases 2 and 3 produce an almost constant $2^+_1-0^+_1$ excitation energy,
they reproduce poorly the properties of odd nuclei and other 
excited states in the even Sn isotopes. 

However, the fact that the first three  approximations result in a such a good
reproduction of the  $2^+_1-0^+_1$ spacing may hint to the fact that the 
$^1S_0$ partial wave is of paramount importance. 
If we now turn the attention to case 4, i.e., we omit the
$^1S_0$ and $^3P_2$ partial waves in the construction of the $G$-matrix,
the results presented  in Table 1  exhibit  a spectroscopic 
catastrophe. We do also not list eigenstates
with other quantum numbers. For e.g., $^{126}$Sn
the ground state is no longer a $0^+$ state, rather it carries $J=4^+$ while for $^{124}$Sn the ground state 
has $6^+$. The first $0^+$ state for this nucleus is given at an excitation
energy of $0.1$ MeV with respect to the $6^+$ ground state.
The general picture for other eigenstates is that of 
an extremely poor agreement
with data.  
Since the agreement is so poor, even the qualitative reproduction of the 
$2^+_1-0^+_1$ spacing, we defer from performing time-consuming shell-model
calculations for $^{116,118,120,122}$Sn.


\section{Shell model studies of the proton drip line nuclei 
$^{105,106,107}$Sb}


Considering valence protons in addition to valence neutrons 
serves to increase substantially the dimensionality of the 
eigenvalue problem. Whereas for example 116Sn with 16 valence 
neutrons has about 16 million basis states in the $m$-scheme, 
$^{116S}$b with one valence proton and 15 valence neutrons has 
two orders of magnitude more states. An additional complexity 
arises from the neutron-proton or $T = 0$ interaction which 
seems to be less well known than the identical-nucleon or $T = 1$ interaction.


We present recent results for $^{105}$Sb, $^{106}$Sb and $^{107}$Sb in
Table 2. The calculations use $^{100}$Sn as
closed shell core with an effective 
interaction for the four, five or six valence neutrons and one valence proton
based on the CD-Bonn nucleon-nucleon interaction \cite{ruprecht}.
The experimental spin assignements for $^{105}$Sb and
$^{106}$Sb are tentative. 
There are also many more theoretical states than 
reported in the enclosed table. 

The high spin level scheme of $^{105}$Sb resembles the level scheme of
$^{107}$Sb up to $J=19/2$ \cite{sb107}. 
When compared to $^{107}$Sb the $^{105}$Sb
level scheme shows similar trends as when going from $^{106}$Sn to
$^{104}$Sn. That means that coupling a $d_{5/2}$ proton to a 
$^{104}$Sn core is appropriate to describe the observed states.
The calculation favors $J^{\pi}$=5/2$^+$ for the ground state in
agreement with the suggestion from proton decay data.
In this state the valence proton is mainly in the $d_{5/2}$
orbit and the two neutron pairs are almost evenly distributed
over the $d_{5/2}$ and $g_{7/2}$ neutron orbits. The situation
is very similar in the 9/2$^+$ and 13/2$^+$ states, while 
the $\nu$$g_{5/2}^3$$g_{7/2}^1$ configuration exhausts the
largest parts of the wave functions of the 15/2$^+$ and 17/2$^+$ 
states. The neutron part of the wave function of the 19/2$^+$
state is almost identical to the 17/2$^+$ state. However,
since 17/2$^+$ is the maximum spin for the 
$\pi$$g_{5/2}^1$$\nu$$g_{5/2}^1$$g_{7/2}^1$ configuration,
the odd proton resides almost exclusively in the $g_{7/2}$
orbit in the 19/2$^+$ state. 
For proton degrees of freedom
the $s_{1/2}$, $d_{3/2}$ and $h_{11/2}$ single-particle 
orbits give essentially negligible
contributions to the wave functions and the energies of the excited
states, as expected. For neutrons, although
the single-particle distribution for a given state is also negligible,
these orbits are important for a good describtion of the energy spectrum,
as also demonstrated in large-scale shell-model calculations of
tin isotopes \cite{ehho98}.
Similar picturer applies to $^{106}$Sb and $^{107}$Sb as well, 
see Refs.~\cite{sb106,sb107}.
The wave functions for the various states are to a large extent
dominated by the $g_{7/2}$ and $d_{5/2}$ single-particle orbits
for neutrons ($\nu$)
and the $d_{5/2}$ single-particle
orbit for protons ($\pi$). 
The $\nu g_{7/2}$ and $\nu d_{5/2}$ single-particle orbits represent
in general more than $\sim 90\%$ of the total neutron single-particle
occupancy, while the  $\pi d_{5/2}$ single-particle orbits stands for 
$\sim 80-90\%$ of the proton single-particle occupancy. The other single-particle
orbits play  an almost negligible role in the structure of the wave functions.
\begin{table}[hbt]
\begin{center}
\caption{ Low-lying states of $^{105,106,107}$Sb, theory and experiment.
Energies in MeV. }\footnotesize
\begin{tabular}{ccc|ccc|ccc}
\hline
\multicolumn{3}{c|}{ $^{105}$Sb} & \multicolumn{3}{c|}{ $^{106}$Sb}& \multicolumn{3}{c}{ $^{107}$Sb} \\ 
{$J^{\pi}_i$} & {Exp} & {Theory} & 
{$J^{\pi}_i$} & {Exp} & {Theory} & 
{$J^{\pi}_i$} & {Exp} & {Theory} \\
\hline 
$5/2^{+}$ & 0 & 0 & $2^{+}$ & 0 & 0 & $5/2^{+}$ & 0 & 0 \\
$9/2^{+}$ & 1.22 & 1.22 & $4^{+}$ & 0.10 & 0.25 & $7/2^{+}$ & 0.77 & 0.69 \\
$13/2^{+}$ & 1.84 & 1.94 & $5^{+}$ & 0.32 & 0.54 & $9/2^{+}$ & 1.06 & 1.08  \\
$15/2^{+}$ & 2.21 & 2.10 & $6^{+}$ & 0.44 & 0.66 & $11/2^{+}$ & 1.79 & 1.80 \\
$17/2^{+}$ & 2.50 & 2.41 & $7^{+}$ & 0.89 & 1.34 & $13/2^{+}$ & 1.90 & 1.94  \\
$19/2^{+}$ & 2.99 & 2.94 & $8^{+}$ & 1.53 & 1.78 &  $15/2^{+}$ & 2.24 & 2.38\\
$23/2^{+}$ & 3.73 & 4.09 & $10^{+}$ & 2.26 & 2.57 & $17/2^{+}$ & 2.75 & 2.83  \\\hline
\end{tabular}
\end{center}
\end{table}
We have further calculated a number of nuclei with both 
proton and neutron valence nucleons near $^{100}$Sn,
such as Ag, Cd and In. These have all been calculated 
with an $^{88}$Sr core, as have a number of Zr isotopes from 
$A = 90$ to $A = 98$. The results of these calculations 
are discussed elsewhere \cite{anne2000,matej}.


\section{Summary and outlook}
The aim of the present work has been to test realistic 
microscopic calculations of the effective interaction 
in large-scale shell-model calculations with many valence nucleons. 
In the whole string of Sn isotopes we have have 
obtained reasonable agreement with the observed energy spectra. 
In particular, we have reproduced the approximate constant $2^+ - 0^+$ 
pairing gap and attributed it to the $^1S_0$ (and $^3P_2$) partial 
wave of the nucleon-nucleon interaction, in consistency with what 
has been found for nuclear and neutron matter. 
Although the energy spectra are well reproduced, the binding 
energies are strongly overestimated. In a previous work we have 
shown that this may be cured by adding a constant repulsive 
term of 150 keV. It is a challenge for the future to establish 
the dynamic basis of such a correction. One track to follow may 
be to evaluate the effective three-body correction. 
Such work is under way and will be reported elsewhere in due time.

We have also started to probe the proton-neutron effective interaction. 
This may be viable as long as the proton and neutrons occupy 
the same valence shell, but seems to be far more difficult 
when they occupy different valence shells. Also, 
the dimensionality of the shell-model problem is quickly 
blowing up when both kinds of valence nucleons 
are present and thus represents a major numerical challenge. 

Further analysis of nuclei such as Ag, Cd, In near $A\sim 100$ and
Sn isotopes near $A\sim 132$ are in progress.
 
We are much indebted to Cyrus Baktash, David Dean, Hubert Grawe
and Matej Lipoglav\v{s}ek
for many discussions on properties of nuclei near $A\sim 100$.

\begin{thebibliography}{200}

\bibitem{ruprecht} D.R.\ Entem and R.\ Machleidt, these proceedings; 
R.\ Machleidt, F.\ Sammarruca and Y.\ Song,
Phys.\ Rev.\ C 53 (1996).

\bibitem{hko95}  M.\ Hjorth-Jensen, T.\ T.\ S.\ Kuo and
E.\ Osnes, Phys.\ Reports 261 (1995) 125.


\bibitem{ehho98} A.~Holt, T.~Engeland, M.~Hjorth-Jensen, and E.~Osnes,
                 Nucl.~Phys.~A634 (1998) 41.

\bibitem{sb107} D.~R.~LaFosse et al., Phys.~Rev.~C62 (2000) 014305.  

\bibitem{sb106} D.~Sohler {\em et al.}, Phys.~Rev.~C59 (1999) 1324; for a
shell-model analysis see e.g., 
T.~Engeland, M.~Hjorth-Jensen and E.~Osnes, 
Phys.~Rev.~C {\bf 61} (2000) 021302(R).


\bibitem{anne2000} A.~Holt, T.~Engeland, M.~Hjorth-Jensen, and E.~Osnes,
                 Phys.~Rev.~C61 (2000) 064318.
\bibitem{matej}M.~Lipoglav\v{s}ek, C.~Baktash,
M.~P.~Carpenter,
D.~J.~Dean,
T.~Engeland,
C.~Fahlander,
M.~Hjorth-Jensen,
R.~V.~F.~Janssens,
A.~Likar,
J.~Nyberg,
E.~Osnes,
S.~D.~Paul,
A.~Piechaczek,
D.~C.~Radford,
D.~Rudolph,
D.~Seweryniak,
D.~G.~Sarantites,
M.~Vencelj,
C.~H.~Yu, Phys.\ Rev.\ C, in press.

\end{thebibliography}



\end{document}






  



\documentclass{ws-procs9x6}

\begin{document}

\title{Coupled Cluster Approaches to Nuclei, Ground States and Excited States}

\author{D.~J.~Dean$^{1,2}$, M.~Hjorth-Jensen$^{2,3,4,5}$,
K.~Kowalski$^{6}$, T.~Papenbrock$^{1,7}$, M.~Wloch$^{6}$, and 
P.~Piecuch$^{5,6}$}


\address{$^1$Physics Division, Oak Ridge National Laboratory,
P.O. Box 2008, Oak Ridge, TN 37831, USA\\
$^2$Center of Mathematics for Applications, University of Oslo, N-0316 Oslo, Norway\\
$^3$Department of Physics, University of Oslo, N-0316 Oslo, Norway\\
$^4$PH Division, CERN, CH-1211 Geneve 23, Switzerland\\
$^5$Department of Physics and Astronomy,
Michigan State University, East Lansing, MI 48824, USA\\
$^6$Department of Chemistry, Michigan State University,
East Lansing, MI 48824, USA\\
$^7$Department of Physics and Astronomy, University of Tennessee,
Knoxville, TN 37996, USA}



\maketitle

\abstracts{We present recent coupled-cluster studies of nuclei, 
with an emphasis on ground state and excited states of 
closed shell nuclei. 
Perspectives for future studies are delineated.}

\section{Introduction}

Physical properties, such as masses and life-times,
of very short-lived, and hence very rare, nuclei are important
ingredients that determine element production mechanisms in
the universe. Given that present nuclear structure research facilities
and the proposed Rare Isotope Accelerator will open significant
territory into regions of medium-mass and heavier nuclei,
it becomes important to investigate theoretical methods that will allow
for a description of medium-mass systems that are involved in such
element production. Such systems pose significant
challenges to existing nuclear structure models, especially since many of
these nuclei will be unstable and short-lived. How to deal with weakly
bound systems and coupling to resonant states is an unsettled problem in
nuclear spectroscopy.

The {\it ab initio} coupled-cluster theory is a particularly promising
candidate for such endeavors due to its enormous success in quantum
chemistry. Here we  describe applications of 
coupled cluster techniques to
nuclear structure. The coupled-cluster methods are
very promising, since they allow one to study ground- and 
excited-state properties
of nuclei
with dimensionalities beyond the capability of present shell-model
approaches, with a much smaller numerical effort when compared to
the more traditional shell-model methods aimed at similar accuracies.
Even though the shell-model combined with appropriate effective interactions
offers in general a very good description of several stable and even weakly bound nuclei, the increasing single-particle level density of weakly bound systems
makes it imperative to 
identify and investigate methods that will extend to unstable systems,
systems whose dimensionality is beyond reach for present shell-model studies, 
typically limited today to  systems with at most $\sim 10^9$ basis states.

The coupled-cluster approach offers several advantages. 
It is fully microscopic and allows one to start with the free nucleon-nucleon interaction, or eventually three-body interaction models.
It contains only linked diagrams, it is size extensive.
and can be improved upon systematically, for example by the inclusion of 
three-body interactions and more complicated correlations. 
One can study both closed-shell systems 
and valence systems and it is possible to derive effective two and three-body 
interactions for open shell systems, with the inclusion of 
complex interactions, 
of great relevance for the study of weakly bound systems. Finally, 
it is amenable to parallel computing.

Here we present several coupled-cluster 
results from recent calculations with singles,
doubles, and noniterative triples and their generalizations to
excited states applied to the $^4$He and $^{16}$O nuclei. A comparison
of coupled cluster results with the results of the exact
diagonalization of the Hamiltonian in the same model space shows that
the quantum chemistry inspired coupled cluster approximations provide
an excellent description of ground and excited states of nuclei.
The results presented here are a summary of recent works 
listed in Refs.~\cite{ref1,ref2}. 

\section{Coupled Cluster approach to Nuclei }



Nuclear many-body theory
often begins with a $G$-matrix interaction which is derived from 
an underlying bare nucleon-nucleon interaction.  
This $G$-matrix can in turn be used in perturbative many-body approaches
in order to derive effective interactions for the nuclear shell model, 
see for example Refs.~\cite{mhj95,dehko04} for recent reviews.
These approaches have shown be to rather successful in shell-model studies of 
several nuclear systems. However, to derive effective interactions within 
the framework of many-body perturbation theory 
is hard to expand upon in a systematic manner by including for example 
three-body diagrams.
In addition, there are
no clear signs of convergence, even  in terms of a weak interaction such as the $G$-matrix. Even in atomic and molecular physics, many-body perturbative
methods are not much favoured any longer, see for example Ref.~\cite{helgaker} for a 
critical discussion. 
The lessons from atomic and molecular many-body systems clearly point to 
the need of non-perturbative resummation techniques of large
classes of diagrams.  

An alternative to such resummation techniques is however offered by the 
so-called no-core approach. There one typically defines a two-body or three-body effective interaction within a large, but limited model space. This is 
parallel to our own approach below, where we limit
the discussion to the no-core $G$-matrix so that all particles are active
within our chosen model space.  Using a given basis expansion
of the many-body wave function we could then solve the nuclear problem by
diagonalization as has been pursued by the no-core shell model collaboration
\cite{bruce1,bruce2,bruce3,petr_erich2002}. 
In fact, the current and most advanced no-core
techniques have approached $^{12}$C, with nearly
converged solutions \cite{hayes03}.

It should be evident, however, that diagonalization procedures scale 
almost combinatorially with the number of particles in a given number of 
single-particle orbitals. Because of this scaling, diagonalization simply
becomes untenable at some point. The efforts to 
expand diagonalization into $p$-shell nuclei with all 
nucleons active, an effort that
spans over ten years, illustrates the problem. The
computational complexity of the nucleus grows dramatically as the size
of the nucleus increases. As a simple example consider 
oscillator single-particle states,
and single-particle spaces consisting of 4 and 7 major
oscillator shells, and compare the number of uncoupled many-body basis states
there are for 4,8,12, and 16 particles. From table \ref{table_1}
we see an enormous growth of the standard shell-model diagonalization
problem within the space. We calculated the number of $M=0$ states for
He and B
within the model space consisting of 4 major shells
and estimated the number of basis states for C and O. Also
indicated are similar estimates for seven major oscillator
shells. The important lesson to learn from these numbers is
that the model-space expansion becomes astronomical quite quickly.
\begin{table}[th]
\tbl{Dimensions of the shell-model problem in four major oscillator
shells  and 7 major oscillator
shells with $M=0$.}
{\footnotesize
\begin{tabular}{@{}ccc@{}}
\hline
System &   4 shells & 7 shells \cr
\hline
$^{4}$He & 4E4  &  9E6 \cr
$^{8}$B  & 4E8  &  5E13 \cr
$^{12}$C & 6E11 &  4E19 \cr
$^{16}$O & 3E14 &  9E24  \cr
\hline
\end{tabular}\label{table_1}}
\vspace*{-13pt}
\end{table}

Yet, because of the advent of radioactive nuclear beam
accelerators, such as the proposed Rare Isotope Accelerator (RIA) in the
U.S., we face the daunting task of moving beyond
$p$-shell nuclei in {\it ab initio} calculations. We should therefore
investigate several ways of approaching the nuclear many-body 
problem in order to successfully make the move into the RIA era.
Here we will discuss the coupled-cluster 
technique which can be used to pursue nuclear many-body calculations to 
heavier systems beyond the $p$-shell. 

Coupled cluster theory originated in nuclear physics
\cite{coester58,coester60} around 1960.  Early studies in the
seventies \cite{klz78} probed ground-state properties in limited
spaces with free nucleon-nucleon interactions available at the
time. The subject was revisited
only recently by Bishop {\it et al.}
\cite{ticcm}, for further theoretical development, and by Mihaila and
Heisenberg \cite{hm99}, for coupled cluster calculations
using realistic
two- and three-nucleon
bare interactions
and expansions in the
inverse particle-hole energy spacings.
However, much of
the impressive development in
coupled cluster theory made in quantum chemistry in
the last 20-30 years
\cite{comp_chem_rev00,Bartlett95,Paldus99,Piecuch02a,Piecuch02b},
after the introduction of coupled-cluster theory to quantum
chemistry by \v{C}\'{\i}\v{z}ek in the 1960's \cite{cizek66,cizek69},
still awaits applications to the nuclear many-body problem.

Many solid theoretical reasons exist that motivate a pursuit of
coupled-cluster methods. First of all, the method is fully
microscopic and is capable of systematic and hierarchical improvements.
Indeed, when one expands the cluster operator in coupled-cluster theory
to all $A$ particles in the system, one exactly produces the fully-correlated
many-body wave function of the system. The only input that the method
requires is the nucleon-nucleon interaction. 
The method may also be extended
to higher-order interactions such as the three-nucleon interaction.
Second, the method is size extensive which means that only linked
diagrams appear in the computation of the  
energy (the expectation value of the Hamiltonian) and amplitude equations.
As discussed, for example, in Refs.~\cite{comp_chem_rev00,Paldus99} all shell model calculations
that use particle-hole truncation schemes
actually suffer from the inclusion of disconnected diagrams
in computations of the energy.
Third, coupled-cluster theory is also size
consistent which means that the energy of two non-interacting fragments
computed separately is the same as that computed for both fragments
simultaneously. In chemistry, where the study of reactions
is quite important, this is a crucial property not available
in the interacting shell model (named configuration interaction in
chemistry).
Fourth, while the theory
is not variational,
the energy behaves as a variational quantity in most instances.
Finally, from a
computational point of view, the practical implementation of coupled
cluster theory is amenable to parallel computing.

We are in the process of applying quantum chemistry inspired coupled cluster
methods
\cite{comp_chem_rev00,Bartlett95,Paldus99,Piecuch02a,Piecuch02b,cizek66,cizek69,Stanton:1993,Piecuch99,Kowalski00,Kowalski03} to
finite nuclei \cite{ref1,ref2}. We show one result
from our current studies, namely the convergence of $^{16}$O
as a function of the model space in which we perform the calculations.

The basic idea of coupled-cluster theory is that the correlated many-body
wave function $\mid \Psi\rangle$ 
may be obtained by application of a cluster operator, 
$T$, such that
\begin{equation}
\mid\Psi \rangle =\exp\left(T\right)\mid\Phi\rangle\;,
\end{equation}
where $\Phi$ is a reference Slater determinant chosen as a convenient starting
point.  For example, we use the filled $0s$ state as the reference 
determinant for $^4$He.

The cluster operator $T$ is given by
\begin{equation}
T=T_1 + T_2 + \cdots T_A\;,
\end{equation}
and represent various
$n$-particle-$n$-hole ($n$p-$n$h) excitation amplitudes such as
\begin{eqnarray}
T_1 &=& \sum_{a\langle\varepsilon_f, i\rangle\varepsilon_f}t^i_a a^\dagger_a a_i\;, \\
T_2 &=& \frac{1}{4}\sum_{i,j\langle\varepsilon_f; ab \rangle \varepsilon_f}t^{ij}_{ab}
a^\dagger_a a^\dagger_b a_j a_i\;,
\end{eqnarray}
and higher-order terms from $T_3$ to $T_A$.  
The basic approximation is obtained by truncating the many-body
expansion of $T$ at the $2p-2h$ cluster component $T_{2}$.
This is 
commonly referred to in the literature as the coupled-cluster singles and
doubles approach (CCSD). 

We compute the ground-state energy from
\begin{equation}
E_{\rm g.s.}=\langle\Phi\mid \exp\left(-T\right) H \exp\left(T\right)
\mid\Phi\rangle\;. 
\end{equation}
The Campbell-Hausdorff-Baker relation may be used to rewrite the similarity
transformation as an expansion in terms of nested commutators. 
The expansion terminates exactly at four nested commutators  when
the Hamiltonian contains, at most, two-body terms, and at six-nested 
commutators when  three-body potentials are present. 
This can also be seen diagrammatically, since $e^{-T} H e^{T}$ is
equivalent to the connected product of the Hamiltonian and $e^T$, which
has to terminate at the quartic terms in $T$ when interactions are
pairwise (the Hamiltonian has at most four lines that can be connected
with the $T$ vertices) and at the $T^{6}$ terms when interactions are
three-body (the Hamiltonian has at most six lines that can be connected
with the $T$ vertices) \cite{Paldus99,cizek66,cizek69}.
The equations for amplitudes are found by left projection of
excited Slater determinants
so that
\begin{eqnarray}
0 &=& \langle\Phi_i^a\mid 
\exp\left(-T\right) H \exp\left(T\right) \mid \Phi\rangle\;,  \nonumber \\ 
0 &=& \langle\Phi_{ij}^{ab}\mid 
\exp\left(-T\right) H \exp\left(T\right) \mid \Phi\rangle \;.
\label{project_eqns}
\end{eqnarray}
The commutators also generate nonlinear terms within these expressions. 
To derive these equations, we use the diagrammatic approach.
In order to obtain the computationally efficient algorithms,
which lead to the lowest operation count and memory requirements,
we use the idea of recursively generated intermediates and
diagram factorization \cite{Bartlett95}.
The resulting equations can be cast into a computationally
efficient form, where diagrams representing intermediates multiply
diagrams representing cluster operators. The resulting equations
can be solved using efficient iterative algorithms, see for example Refs.~\cite{ref1,Bartlett95}.


In our coupled-cluster study of Ref.~\cite{ref1}, 
we performed calculations of 
the $^{16}$O ground state for up to seven major oscillator 
shells as a function of $\hbar\omega$. 
Fig.~\ref{fig_ox_hw} indicates the level of convergence
of the energy per particle for $N=4,5,6,7$ shells. The experimental value
resides at 7.98~MeV per particle.  This calculation is practically converged.
By seven oscillator shells, the $\hbar\omega$ dependence becomes rather
minimal and we find a ground-state binding energy of 7.52 MeV per particle in
oxygen using the Idaho-A potential. Since the Coulomb interaction should give
approximately 0.7 MeV/A of repulsion, and is not included in this
calculation, we actually obtain approximately 6.90 MeV of nuclear binding
in the 7 major shell calculation which is somewhat above the experimental
value (most likely, due to the neglect of three-body interactions in the
calculations). We note that the entire procedure ($G$-matrix plus CCSD) tends to
approach from below converged solutions.
\begin{figure}
\begin{center}
\includegraphics[angle=270, scale=0.35]{figure_ccsd_ox.eps}
\caption{Dependence of the ground-state energy of $^{16}$O  on $\hbar\omega$
as a function of increasing model space.}
\label{fig_ox_hw}
\end{center}
\end{figure}
We have recently performed calculations with eight major shells, and the 
results are practically converged.
 
We also considered chemistry inspired
 noniterative corrections to the CCSD energy due to three-body clusters $T_{3}$
(labelled triples in quantum chemistry).
We performed this study in the model space consisting
of four major oscillator shells, since we can perform exact shell-model
calculations for nuclei such as $^{4}$He. 
Table~\ref{table_ox16_gs} shows the total ground-state energy values
obtained with the CCSD and one of the
triples-correction approaches (labeled CR-CCSD(T) 
\cite{Piecuch02a,Piecuch02b,Kowalski00,Kowalski03}
in the table). Slightly
differing triples-corrections yield similar corrections to the
CCSD energy.
The coupled cluster methods recover the bulk of the correlation
effects, producing the results of the SM-SDTQ, or better, quality.
SM-SDTQ stands for the expensive shell-model (SM) diagonalization in
a huge space spanned by the reference and all
singly (S), doubly (D), triply (T), and
quadruply (Q) excited determinants.
To understand this result, we note that
the CCSD $T_1$ and $T_2$ amplitudes are similar in order of magnitude. (For
an oscillator basis, both $T_1$ and $T_2$ contribute to the first-order
MBPT wave function.)
Thus, the $T_1 T_2$ {\it disconnected} triples are large, much larger than
the $T_3$ {\it connected} triples, and the difference
between the SM-SDT (SM singles, doubles, and triples)
and SM-SD energies is mostly due to $T_1 T_2$.The small $T_3$
effects, as estimated by CR-CCSD(T), are consistent
with the SM diagonalization calculations. If the $T_3$ corrections
were large, we would observe a significant lowering of the
CCSD energy, far below the SM-SDTQ result.
Moreover, the CCSD and CR-CCSD(T) methods
bring the nonnegligible higher-than-quadruple excitations,
such as $T_1^3 T_2$, $T_1 T_2^2$, and $T_{2}^{3}$, which are
not present in SM-SDTQ. It is, therefore, quite likely that the
CR-CCSD(T) results are very close to the results of the exact
diagonalization, which cannot be performed.
\begin{table}[ht]
\tbl{The ground-state energy of $^{16}$O
calculated using various coupled cluster methods
and oscillator basis states.  }
{\footnotesize
\begin{tabular}{@{}cc@{}}
\hline
Method & Energy \cr
\hline
CCSD                       & -139.310 \cr
CR-CCSD(T)                 & -139.467 \cr
SM-SD                        & -131.887 \cr
SM-SDT                       & -135.489 \cr
SM-SDTQ                      & -138.387 \cr
\hline
\end{tabular}\label{table_ox16_gs}}
\vspace*{-13pt}
\end{table}

These results indicate that the bulk of the correlation energy within
a nucleus can be obtained by solving the CCSD equations. This gives us
confidence that we should pursue this method in open shell systems
and to excited states. We have recently 
\cite{ref2} performed excited state calculations on $^{4}$He
using the EOMCCSD (equation of motion CCSD) method.
For the excited
states $|\Psi_{K}\rangle$ and energies $E_{K}^{\rm (CCSD)}$ ($K > 0$),
we apply the EOMCCSD (``equation of motion CCSD'') approximation
\cite{Stanton:1993,Piecuch99} (equivalent to the 
response CCSD method \cite{Monkhorst:1977}),
in which
\begin{equation}  
|\Psi_{K}\rangle=R_{K}^{\rm (CCSD)} \exp(T^{\rm (CCSD)}) |\Phi\rangle .  
\label{eomfun}  
\end{equation}
Here $R_{K}^{\rm (CCSD)} = R_{0}+ R_{1} + R_{2}$ is a sum of the
reference ($R_{0}$), one-body ($R_{1}$), and two-body ($R_{2}$)
components
obtained by diagonalizing
$\bar{H}^{{\rm (CCSD)}}$
in the same space of singly and doubly excited determinants
$|\Phi_{i}^{a}\rangle$ and $|\Phi_{ij}^{ab}\rangle$ as used in the
ground-state CCSD calculations. These calculations may also be 
corrected in a non-iterative fashion using the completely renormalized
theory for excited states 
\cite{Piecuch02a,Piecuch02b,Kowalski00,Kowalski03,Kowalski01}.  
The low-lying
$J=1$ state most likely results from the center-of-mass contamination
which we have removed only from the ground state.  The $J=0$ and $J=2$
states calculated using EOMCCSD and CR-CCSD(T) are in excellent
agreement with the exact results. 
\begin{table}[ht]
\tbl{The excitation energies of $^4$He   
calculated using the  
oscillator basis states (in MeV).}  
{\footnotesize
\begin{tabular}{@{}ccccc@{}}  
\hline
State & EOMCCSD & CR-CCSD(T) & CISD & Exact \cr
\hline
J=1   &  11.791 & 12.044 & 17.515    & 11.465 \cr
J=0   &  21.203 & 21.489 & 24.969    & 21.569 \cr
J=2   &  22.435 & 22.650 & 24.966    & 22.697 \cr
\hline
\end{tabular}\label{table_2}}
\vspace*{-13pt}
\end{table}
We have recently also computed excited states in $^{16}$O, with a particular 
emphasis on the first $3_1^-$ state, which is known to be of a 1p-1h nature.
Our results based on the EOMCCSD method yields 
13.57 MeV for five shells and 12.98 MeV for six shells, to be compared 
with the experimental value of 6.13 MeV. We expect that with seven shells
and the  inclusion of triples to get closer to the experimental value.
For states like this and for two-body interactions it is
well known in quantum chemistry that EOMCCSD is a very accurate
approach, producing excitation energies within 10 \% of the exact values.
Thus, we will be able to predict the result corresponding to
an Idaho-A potential that we used in these calculations once we complete
our work for the seven shells and extrapolate the energies to the complete
basis set limit. These results will be presented elsewhere, see 
Ref.~\cite{marta2004}. 

Our experience thus far with the 
quantum chemistry inspired coupled cluster
approximations to calculate the ground and excited states of the
$^{4}$He and $^{16}$O nuclei indicates that this will be a promising
method for nuclear physics.  By comparing coupled cluster results
with the exact results obtained by diagonalizing the Hamiltonian in
the same model space, we demonstrated that relatively inexpensive
coupled cluster approximations recover the bulk of the nucleon
correlation effects in ground- and excited-state nuclei. These results
are a strong motivation to further develop coupled cluster methods for
the nuclear many-body problem, so that accurate {\it ab initio}
calculations for small- and medium-size nuclei become as routine as
molecular electronic structure calculations.


\section{Perspectives and Future Plans}
The study of exotic nuclei opens new 
challenges to nuclear physics. 
The challenges and the excitement arise 
because exotic nuclei will present new and 
radically different manifestations of nucleonic matter 
that occur near the bounds of nuclear existence, 
where the special features of weakly bound, quantal systems 
come into prominence. Furthermore, many of these
nuclei are key to understanding matter production in the universe.
Given that present and future nuclear structure research facilities
will open significant
territory into regions of medium-mass and heavier nuclei,
it becomes important to investigate theoretical methods that will allow
for a description of medium-mass nuclear systems. 
Such systems pose significant
challenges to existing nuclear structure models, especially since many of
these nuclei will be unstable and short-lived. How to deal with weakly
bound systems and coupling to resonant states is an unsettled problem in
nuclear spectroscopy. 

Many-body methods like the 
coupled cluster theory offer possibilities for extending
microscopic {\it ab initio} calculations to nuclei of the size of $^{40}$Ca.
Especially the coupled-cluster methods are
very promising, since they allow one to study 
ground- and excited-state properties of nuclei
with dimensionalities beyond the capability of present shell-model
approaches. As demonstrated here and in Ref.~\cite{ref2} we 
show for the first time how to calculate 
excited states for a nucleus using coupled cluster 
methods from quantum chemistry.
For the weakly bound nuclei to be produced by future low-energy 
nuclear structure facilities
it is almost imperative to increase the
degrees of freedom under study in order to reproduce
basic properties of these systems. 
We are presently working on deriving complex 
effective interactions, see for example
Ref.~\cite{hvh2004}, 
for weakly bound systems to be used in coupled cluster 
studies of these weakly bound nuclei. 

We have based most of our analysis 
using two-body nucleon-nucleon interactions only. 
We feel this is important since
techniques like the coupled cluster methods 
allow one to include a much larger
class of many-body terms than done earlier. Eventual discrepancies 
with experiment 
such as the missing reproduction of e.g., the first 
excited $2^+$ state in a $1p0f$ calculation
of $^{48}$Ca, can then be ascribed to eventual 
missing three-body forces, as indicated by the studies
in Refs.~\cite{petr_erich2002,bob1,bob2,bob3,apr98,petr_erich2003} 
for light nuclei. 
The inclusion of real three-body interactions belongs 
to our future plans.

 \section*{Acknowledgments}
 Supported by the U.S. Department of Energy
 under
 Contract Nos. DE-FG02-96ER40963 (University of Tennessee),
 DE-AC05-00OR22725 with UT-Battelle, LLC (Oak Ridge
 National Laboratory), and DE-FG02-01ER15228 (Michigan State University),
 the National Science Foundation (Grant No. CHE-0309517; Michigan State University),
 the Research Council of Norway, and the Alfred P. Sloan Foundation.

 \begin{thebibliography}{200}
\bibitem{ref1} D.J.~Dean, and M.~Hjorth-Jensen, 
Phys. Rev. {\bf C69} (2004) 054320.
\bibitem{ref2} K.~Kowalski, D.J.~Dean, M.~Hjorth-Jensen, T.~Papenbrock, 
and P.~Piecuch, Phys. Rev. Lett.~{\bf 92} (2004) 132501.
\bibitem{mhj95} M.\ Hjorth-Jensen, T.T.S.\ Kuo and E.\ Osnes,
Phys.\ Rep.\ {\bf 261} (1995) 125.
\bibitem{dehko04} D.J.~Dean, T.~Engeland, M.\ Hjorth-Jensen, M.P.~Kartamyshev, 
and E.\ Osnes, Prog.~Part.~Nucl.~Phys.~{\bf 53} (2004) 419.
\bibitem{helgaker} T.~Helgaker, P.~J{\o}rgensen, and J.~Olsen, 
 {\em Molecular Electronic Structure Theory. Energy and Wave Functions}, (Wiley, Chichester, 2000).
\bibitem{bruce1} P.~Navr\'atil and B.R.~Barrett, Phys. Rev. {\bf C57} (1998) 562.
\bibitem{bruce2} P.~Navr\'atil, J.P.~Vary, and B.R.~Barrett
Phys.~Rev.~Lett.~{\bf 84} (2000) 5728.
\bibitem{bruce3} P.~Navr\'atil, J.P.~Vary, and B.R.~Barrett, Phys. Rev. {\bf C62 }(2000) 054311.
\bibitem{petr_erich2002} P.~Navr\'atil and W.E.~Ormand, Phys. Rev. Lett.~{\bf 88} (2002) 152502.
\bibitem{hayes03} A.C.~Hayes, P.~Navr\'atil, and J.P.~Vary,  Phys. Rev. Lett.~{\bf 91}
(2003)  012502.
\bibitem{coester58} F.~Coester, Nucl. Phys. {\bf 7} (1958) 421.
\bibitem{coester60} F.~Coester and H.~K\"ummel, Nucl. Phys.~{\bf 17} (1960) 477.\bibitem{klz78} H.\ K\"{u}mmel, K.H.\ L\"{u}hrmann and J.G.\ Zabolitzky, Phys.\ Rep.\
{\bf 36} (1977) 1.
\bibitem{ticcm} R.F.\ Bishop, E.\ Buendia, M.F.\ Flynn and R.\ Guardiola, J.\ Phys.\ G:
Nucl.\ Part.\ Phys.\ {\bf 17} (1991) 857; 
ibid.\ {\bf 18} (1992) 1157; 
ibid.\ {\bf 19} (1993) 1663; 
R.\ Guardiola, P.I.\ Moliner, J.\ Navarro, R.F.\ Bishop, A.\ Puente and
N.R.\ Walet, Nucl.\ Phys.\ {\bf A609} (1996) 218; R.F.\ Bishop and
R.\ Guardiola.
\bibitem{hm99} J.H.~Heisenberg, and B.~Mihaila, Phys. Rev. {\bf C59} (1999) 1440\bibitem{comp_chem_rev00} T.D.~Crawford and H.F.~Schaefer III,  Rev.~Comp.~Chem.~{\bf 14} (2000) 33..
\bibitem{Bartlett95} S.A.~Kucharski, R.J.~Bartlett, Theor.~Chim.~Acta {\bf 80} (1991) 387;
P.~Piecuch, S.A.~Kucharski, K.~Kowalski, and M.~Musia{\l},
Comp.~Phys.~Comm,~{\bf 149} (2002) 72, and references
therein. 
\bibitem{Paldus99} J. Paldus and X. Li, Adv.~Chem.~Phys.~{\bf 110} (1999) 1.
\bibitem{Piecuch02a} P.~Piecuch and K.~Kowalski and I.S.O.~Pimienta and M.J.~McGuire,
 Int. Rev. Phys. Chem. {\bf 21} (2002) 527.
\bibitem{Piecuch02b} P. Piecuch and K. Kowalski and P.-D. Fan and I.S.O. Pimienta,
eds.~J. Maruani, R. Lefebvre and E. Br{\"a}ndas,
{\em Topics in Theoretical Chemical Physics} vol.~{\bf 12},
     series     Progress in Theoretical Chemistry and Physics,
 (Kluwer, Dordrecht, 2004) 119.
\bibitem{cizek66} J. \v{C}\'{\i}\v{z}ek, J. Chem. Phys. {\bf 45} (1966) 4256.
\bibitem{cizek69} J. {\v C}{\'\i}{\v z}ek, Adv. Chem. Phys. {\bf 14} (1969) 35. 
\bibitem{Stanton:1993} J. F. Stanton and R. J. Bartlett, J. Chem. Phys.~{\bf 98} (1993) 7029.
\bibitem{Piecuch99} P. Piecuch and R. J. Bartlett, Adv. Quantum Chem.~{\bf 34} (1999) 295. 
\bibitem{Kowalski00} K. Kowalski and P. Piecuch, J. Chem. Phys.~{\bf 113} (2000) 18.
\bibitem{Kowalski03} K. Kowalski and P. Piecuch, J. Chem. Phys.~{\bf 120} (2004) 1715.
\bibitem{Monkhorst:1977} H.~Monkhorst, Int. J. Quantum Chem. Symp.~{\bf 11} (1977) 421.
\bibitem{Kowalski01} K. Kowalski and P. Piecuch, J. Chem. Phys.~{\bf 115} (2001) 2966.
\bibitem{marta2004} M.~Wloch, D.J.~Dean, J.R.~Gour, M.~Hjorth-Jensen, K.~Kowalski, T.~Papenbrock, and P.~Piecuch, in preparation for Phys.~Rev.~Lett.
\bibitem{hvh2004} G.~Hagen, J.S.~Vaagen, and M.~Hjorth-Jensen, 
J.~Phys.~A:Math.~Gen.~{\bf 37} (2004) 8991.
\bibitem{bob1} S.C.~Pieper, V.R.~Pandharipande, R.B.~Wiringa, and J.~Carlson, Phys.~Rev.~{\bf C64}
(2001) 014001
\bibitem{bob2} S.C.~Pieper, K.~Varga, and R.B.~Wiringa, Phys.~Rev.~{\bf C66}
(2002) 0044310
\bibitem{bob3}  R.B.~Wiringa and S.C.~Pieper, Phys.~Rev.~Lett.~{\bf 89}
(2002) 182501
\bibitem{apr98} A.\ Akmal, V.R.\ Pandharipande and D.G.\
                Ravenhall, Phys.\ Rev.\ {\bf C58}  (1998) 1804.
\bibitem{petr_erich2003} P.~Navr\'atil and W.E.~Ormand, Phys. Rev.~{\bf C68} (2003) 034305.

 \end{thebibliography}

 \end{document}



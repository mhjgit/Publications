\documentstyle[12pt]{article}
\pagestyle{empty}
%
%   LaTeX format for a contributed Abstract to the
%
%
%			North-West Europe
%		    Nuclear Physics Conference
%		   Vrije Universiteit Amsterdam
%		       April 16 - 19, 1996
%
%
\topmargin=0in\headheight=0in\headsep=0in\footheight=1in\oddsidemargin=7.2pt
\evensidemargin=7.2pt\marginparwidth=20mm\marginparsep=5mm
\textheight=23.8cm
\textwidth=16.8cm
%
\begin{document}
\begin{center}{\large {\bf 
Effective interactions for finite nuclei and nuclear matter}}
\end{center}
\begin{center}
M.\ Hjorth--Jensen
\end{center}

\begin{center}
{\small\it
ECT*, European Centre for Theoretical
Studies in Nuclear Physics and Related Areas, Trento, Italy
}
\end{center}



The scope of this talk is to attempt at an overview of {\em applications}
of perturbative many--body techniques to various
nuclear systems.
This many--body scheme can be described in the following three steps:
\newline
First, one needs a nucleon--nucleon (NN) interaction which is
appropriate for nuclear physics at low and
intermediate energies.
Since we deal with strong interactions, a really
microscopic approach would be to derive nuclear expectation values
from the underlying theory of the strong interaction, namely
quantum chromodynamics (QCD). Due to the strongly
non--perturbative character of QCD in the regime of nuclear physics,
such an approach is at present not available. However, there are
both theoretical and experimental indications that interaction
models based on meson--exchange offer a viable scheme at low and
intermediate energies. With microscopic we will therefore
mean a theory which starts from selected baryons and mesons as the
nuclear constituents.
The nucleon-nucleon interaction is then described in terms of the
exchange of selected mesons.\newline
Secondly,
in nuclear many--body calculations, the first problem one is confronted
with is the fact
that the repulsive core of the NN potential is unsuitable for
a perturbative treatment. This problem is however overcome by
introducing the reaction matrix $G$ given by the solution
of Bethe--Goldstone equation.\newline
Finally,
employing the $G$-matrix defined in the previous step and 
perturbation techniques one
can derive expressions for effective transition operators and interactions
in terms of the
$G$-matrix.


The nuclear systems we will try to discuss using the
above procedure are various properties of medium heavy nuclei,
and dense nuclear matter with relevance
for neutron star studies.

For the finite systems, emphasis will be put on the
capability of effective interactions
and operators and shell--model techniques to handle nuclei
with many degrees of freedom. An interesting property of e.g.\ the
Sn isotopes, is the more or less constant spacing, from $A=104$ to
$A=130$,
between the first
excited $2^+$ state and the $0^+$ ground state. This is interpreted
as a good example of pairing between valence nucleons
for even nuclei.

For infinite nuclear matter, we will try to discuss
the equation of state (EOS). The EOS
for dense matter is central to calculations of
neutron--star properties, such as the mass range, the mass--radius
relationship, the crust thickness
and the cooling rate through various neutrino generating processes.
The rate of the latter are  also influenced by the possibility of
superconducting protons and superfluid neutrons in the core or
inner crust of a neutron star. Thus, following the pairing discussion
for finite nuclei (valence nucleons),
we will also attempt at a discussion of pairing
in dense matter.

The hope is also to point to other methods to describe the above systems,
such as
variational and
Shell--model Monte--Carlo approaches.





\end{document}



















\end{document}
















\documentclass[10pt]{article}
\topmargin=0in\headheight=0in\headsep=0in                     
\oddsidemargin=7.2pt\evensidemargin=7.2pt\marginparwidth=20mm 
\marginparsep=5mm\textheight=23.0cm\textwidth=16.0cm          
\pagestyle{empty}                                            
\begin{document}                                          

\begin{center}                                 

{\bf                                                       
CHALLENGES FOR NUCLEAR STRUCTURE: FROM STABLE TO WEAKLY BOUND NUCLEI\\ } 
\vspace{4ex}                                               
D.~J.~Dean$^{1,2}$, J.~R.~Grour$^{3}$, G.~Hagen$^{2}$, M.~Hjorth-Jensen$^{2,4,5,6}$,
K.~Kowalski$^{3}$, T.~Papenbrock$^{1,7}$, P.~Piecuch$^{3,6}$, and M.~W.~W{\l}och$^{3}$\\
\vspace{2ex}                                          
{\sl                                                  
$^1$Physics Division, Oak Ridge National Laboratory,
P.O. Box 2008, Oak Ridge, TN 37831, USA\\
$^2$Center of Mathematics for Applications, University of Oslo, N-0316 Oslo, Norway\\
$^3$Department of Chemistry, Michigan State University,
East Lansing, MI 48824, USA\\
$^4$Department of Physics, University of Oslo, N-0316 Oslo, Norway\\
$^5$PH Division, CERN, CH-1211 Geneva 23, Switzerland\\
$^6$Department of Physics and Astronomy,
Michigan State University, East Lansing, MI 48824, USA\\
$^7$Department of Physics and Astronomy, University of Tennessee,
Knoxville, TN 37996 USA\\

}                                                             
\vspace{1ex}                                                 
{\small \sl                                  
Contact e-mail: morten.hjorth-jensen@fys.uio.no\\} 
\end{center}                             
\vspace{5mm}
Physical properties, such as masses and life-times,
of very short-lived, and hence very rare, nuclei are important
ingredients that determine element production mechanisms in
the universe. Given that present nuclear structure research facilities
and the proposed Rare Isotope Accelerator will open significant
territory into regions of medium-mass and heavier nuclei,
it becomes important to investigate theoretical methods that will allow
for a description of medium-mass systems that are involved in such
element production. Such systems pose significant
challenges to existing nuclear structure models, especially since many of
these nuclei will be unstable and short-lived. How to deal with weakly
bound systems and coupling to resonant states is an unsettled problem in
nuclear spectroscopy. 

The ab initio coupled cluster theory is a particularly promising
candidate for such endeavors due to its enormous success in quantum
chemistry. Refs.~\cite{prl,prc,prl2} 
describe applications of coupled cluster techniques to
nuclear structure. The coupled-cluster methods are
very promising, since they allow one to study ground- and excited-state properties 
of nuclei
with dimensionalities beyond the capability of present shell-model
approaches, with a much smaller numerical effort when compared to
the more traditional shell-model methods aimed at similar accuracies.
For the weakly bound nuclei to be produced by the proposed Rare
Isotope Accelerator it is almost imperative to increase the
degrees of freedom under study in order to reproduce
basic properties of these systems. 

Here we present several results from recent calculations with singles,
doubles, and noniterative triples and their generalizations to
excited states applied to $^{16}$O. A comparison
of coupled cluster results with the results of the exact
diagonalization of the Hamiltonian in the same model space shows that
the quantum chemistry inspired coupled cluster approximations provide
an excellent description of ground and excited states of nuclei.
How to derive effective interactions for weakly bound systems 
to be used in
coupled-cluster calculations will also be addressed \cite{jpga,prc2}. \newline
\begin{small}
\noindent $^{*}$Supported by the U.S. Department of Energy
under
Contract Nos. DE-FG02-96ER40963 (University of Tennessee),
DE-AC05-00OR22725 with UT-Battelle, LLC (Oak Ridge
National Laboratory), DE-FG02-01ER15228 (Michigan State University),
the Research Council of Norway, and the Alfred P. Sloan Foundation.
\end{small}


\begin{enumerate}
\itemsep=-3pt 
\bibitem{prl} K.~Kowalski, D.~J.~Dean, M.~Hjorth-Jensen, T.~Papenbrock, and P.~Piecuch,
Phys.~Rev.~Lett.~{\bf 92}, 132501 (2004).
\bibitem{prc} D.~J.~Dean and M.~Hjorth-Jensen, Phys.~Rev.~{\bf C  69}, 054320 (2004).
\bibitem{prl2} M.~W.~W{\l}och, D.~J.~Dean, J.~R.~Grour, 
M.~Hjorth-Jensen, K.~Kowalski, T.~Papenbrock, and P.~Piecuch,
Phys.~Rev.~Lett., submitted.
\bibitem{jpga} G.~Hagen, J.~S.~Vaagen, and M.~Hjorth-Jensen,
J.~Phys.~{\bf A 37}; Math.~Gen., 8991 (2004).
\bibitem{prc2} G.~Hagen, M.~Hjorth-Jensen, and J.~S.~Vaagen,
nucl-th/0410114, Phys.~Rev.~C, submitted.

\end{enumerate}

\end{document}                                     

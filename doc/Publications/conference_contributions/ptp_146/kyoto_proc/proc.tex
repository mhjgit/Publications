%%%%%%%%%%%%%%%%%%%%%%%%%%%%%%%%%%%%%%%%%%%%%%%%%%%%%%
%%%%%%   template.tex for PTPTeX.sty <ver.1.0>  %%%%%%
%%%%%%%%%%%%%%%%%%%%%%%%%%%%%%%%%%%%%%%%%%%%%%%%%%%%%%
\documentstyle[seceq,epsf,wrapft]{ptptex}
%\documentstyle[seceq,preprint]{ptptex}
%\documentstyle[seceq,letter]{ptptex}
%\documentstyle[seceq,supplement]{ptptex}
%\documentstyle[seceq,addenda]{ptptex}
%\documentstyle[seceq,errata]{ptptex}

%%%%% Personal Macros %%%%%%%%%%%%%%%%%%%


%%%%%%%%%%%%%%%%%%%%%%%%%%%%%%%%%%%%%%%%%
%\pubinfo{Vol. 101, No. 4, Aril 1999}  %Editorial Office use
%\setcounter{page}{}                   %Editorial Office use
%------------------------------------------------------------
%\nofigureboxrule%to eliminate the rule of \figurebox
%\notypesetlogo  %comment in if to eliminate PTPTeX logo
%\subfontMBF     %use if you have not enough fonts when using mbf.sty
%---- When [preprint] you can put preprint number at top right corner.
%\preprintnumber[3cm]{%<-- [..]: optional width of preprint # column.
%KUNS-1325\\ HE(TH)~97/04\\ hep-th/9702083}
%-------------------------------------------

\markboth{%     %running head for odd-page (authors' name)
M.~Hjorth-Jensen
}{%             %running head for even-page (`short' title)
Pairing correlations in nuclear systems, from neutron stars to finite nuclei
}


\title{%        %You can use \\ for explicit line-break
Pairing correlations in nuclear systems, from neutron stars to finite nuclei
}
%\subtitle{This is a Subtitle}    %use this when you want a subtitle

\author{%       %Use \sc for the family name
Morten {\sc Hjorth-Jensen}\footnote{E-mail address: morten.hjorth-jensen@fys.uio.no}
}

\inst{%         %Affiliation, neglected when [addenda] or [errata]
Department of Physics, University of Oslo, N-0316 Oslo, Norway}

%\publishedin{%      %Write this ONLY in cases of addenda and errata
%Prog.~Theor.~Phys.\ {\bf XX} (19YY), page}

\recdate{%      %Editorial Office will fill in this.
%\today
}

\abst{%       %this abstract is neglected when [addenda] or [errata]
In this contribution we attempt at giving an overview
of pairing correlations in nuclear systems with an emphasis
on 
\begin{itemize} 
\item Superfluidity and superconductivity in neutron stars
and the relation to the underlying nucleon-nucleon interaction.
\item Extraction and interpretations 
of pairing correlations in finite nuclei through
large-scale shell model studies of nuclei with mass $A\sim 100-132$.
Effective interactions based on recent nucleon-nucleon interactions
are utilized in the shell-model studies. The partial
waves which lead to superfluid properties in infinite matter are also
crucial for pairing correlations in finite nuclei. 
\item  Discussion of recent experimental and theoretical studies 
of thermodynamical properties of finite nuclei
and their interpretation in terms  of eventual pairing transitions 
in finite nuclei. 
\end{itemize}

}

\begin{document}

\maketitle

\section{Introduction}

Pairing correlations are expected
to play an essential role in nuclear systems, ranging
from the binding energy, excitation spectrum and odd-even effects
in finite nuclei to superfluidity in the interior of neutron stars,
A neutron star is perhaps the largest object in the universe
which exhibits superfluidity in its interior.
An eventual superfluid phase in a neutron star will condition
the neutrino emission and thereby the cooling history of such a
star, in addition to inducing mechanisms such as 
sudden spin ups in the rotational
period of the star.
For an infinite system, such as a neutron star, the nature
of the pairing phase transition is well established as second order.

For finite nuclei there are several interesting manifestations
of pairing correlations.
In e.g., recent theoretical and experimental studies 
of thermodynamical properties of finite nuclei, the heat capacity
has been found to exhibit a non-vanishing bump at temperatures proportional to
half the pairing gap. These bumps have been interpreted as  signs of the
quenching of pair correlations, representing in turn features
of the pairing transition for an infinitely large system.              
Furthermore,
finite nuclei such as those found in the chain of even tin isotopes 
from $^{102}$Sn to $^{130}$Sn,
exhibit a near constancy of the $2^+_1-0^+_1$ excitation energy, 
a constancy which can be related
to strong pairing correlations and the near degeneracy in 
energy of the relevant single particle
orbits. Large shell-model calculations for these isotopes 
reveal that the major contribution to
pairing correlations in the tin isotopes stems from the $^1S_0$ 
partial wave in the nucleon-nucleon
interaction. Omitting this partial wave and the $^3P_2$ wave 
in the construction of an effective
interaction, results in a spectrum which has 
essentially no correspondence with experiment.
These partial wave are also of importance for infinite 
neutron matter and nuclear matter and give
the largest contribution to the pairing interaction and 
energy gap in neutron star matter. 

This contribution falls in four sections. After the above introductory words, we give a 
brief review of pairing in infinite neutron matter. Shell-model
analyses of different approaches to the effective interaction are in turn made in section
\ref{sec:sec3}. In the same section we discuss 
recent experimental and theoretical 
thermodynamical properties of finite nuclei. An
interpretation in terms  of eventual pairing transitions 
in finite nuclei is also presented. Concluding remarks 
are given in section \ref{sec:sec4}.

\section{Pairing in infinite neutron matter}\label{sec:sec2}

The presence of neutron superfluidity in 
the crust and the inner part 
of neutron stars 
are considered well established 
in the physics of these compact stellar objects. 
In the low density outer part of a neutron star, 
the neutron superfluidity is expected 
mainly in the attractive $^1S_0$ channel. 
At higher density, the nuclei in the crust dissolve, and one 
expects a region consisting of a quantum liquid of neutrons and 
protons in beta equilibrium. 
The proton contaminant should be superfluid 
in the $^1S_0$ channel, while neutron superfluidity is expected to  
occur mainly in the coupled $^3P_2$-$^3F_2$ two-neutron channel. 
In the core of the star any superfluid 
phase should finally disappear.
 
The presence of two different superfluid regimes 
is suggested by the known trend of the 
nucleon-nucleon (NN) phase shifts 
in each scattering channel. 
In both the $^1S_0$ and $^3P_2$-$^3F_2$ channels the
phase shifts indicate that the NN interaction is attractive. 
In particular for the $^1S_0$ channel, the occurrence of 
the well known virtual state in the neutron-neutron channel
strongly suggests the possibility of a 
pairing condensate at low density, 
while for the $^3P_2$-$^3F_2$ channel the 
interaction becomes strongly attractive only
at higher energy, which therefore suggests a possible 
pairing condensate
in this channel at higher densities. 
In recent years the BCS gap equation
has been solved with realistic interactions, 
and the results confirm
these expectations. 

The $^1S_0$ neutron superfluid is relevant for phenomena
that can occur in the inner crust of neutron stars, like the 
formation of glitches, which may to be related to vortex pinning  
of the superfluid phase in the solid crust \cite{glitch}. 
The results of different groups are in close agreement
on the $^1S_0$ pairing gap values and on 
its density dependence, which
shows a peak value of about 3 MeV at a Fermi momentum close to
$k_F \approx 0.8\; {\rm fm}^{-1}$ \cite{bcll90,kkc96,eh98,sclbl96}. 
All these calculations adopt the bare
NN interaction or effective interactions without screening 
corrections as the pairing force. It has been pointed out
that the screening by the medium of the interaction 
could strongly reduce
the pairing strength in this channel \cite{sclbl96,chen86,ains89}. 
However, the issue of the 
many-body calculation of the pairing 
effective interaction is a complex
one and still far from a satisfactory solution.

The precise knowledge of the $^3P_2$-$^3F_2$ pairing gap is of 
paramount relevance for, e.g.,  the cooling of neutron stars, 
and different values correspond to drastically
different scenarios for the cooling process.
Generally, the gap suppresses the cooling by a factor
$\sim\exp(-\Delta/T)$ (where $\Delta$ is the energy gap)
which is severe for
temperatures well below the gap energy.
Unfortunately, only few and partly
contradictory calculations of the pairing gap exist in the literature, 
even at the level of the bare NN interaction 
\cite{amu85,bcll92,taka93,elga96,khodel97}. 
However, when comparing the results, one should note that the  
NN interactions used in these calculations are not phase-shift 
equivalent, i.e.,  they do not 
predict exactly the same NN phase shifts.  
Furthermore, for the interactions used in 
Refs.~\cite{amu85,bcll92,taka93,elga96} the predicted 
phase shifts do not agree accurately with modern phase shift 
analyses, and the fit of the NN data has typically 
$\chi^2/{\rm datum}\approx 3$.  
Progress has 
however been made not only in the accuracy and the consistency of the 
phase-shift analysis, but also in the fit of realistic NN interactions 
to these data.  As a result, several new NN interactions have 
been constructed which fit the world data for $pp$ and $np$ scattering 
below 350 MeV with high precision.  Potentials like the recent 
Argonne $V_{18}$ \cite{v18}, the CD-Bonn \cite{cdbonn} 
or the new Nijmegen potentials \cite{nim} yield a 
$\chi^2/{\rm datum}$ of about 1 and may be called phase-shift 
equivalent.  
In Table \ref{tab:pgaps} we show the recent non-relativistic
pairing gaps for the $^3P_2$-$^3F_2$ partial waves, where
effective nucleon masses from the  lowest-order Brueckner-Hartree-Fock
calculation have been 
employed, see Ref.\ \cite{beehs98} for more details.
These results are for pure neutron matter and we observe that
up to $k_F\sim 2$ fm$^{-1}$, the various potentials
give more
or less the same pairing gap. Above this Fermi momentum, which
corresponds to a lab energy of $\sim 350$ MeV, the results start
to differ. This is simply due to the fact that the potentials
are basically fit to reproduce scattering data up to this
lab energy. Beyond this energy, the potentials predict rather
different phase shifts for the 
$^3P_2$-$^3F_2$ partial waves, see e.g.,  Ref.\ \cite{beehs98}.
\begin{table}[hbtp]
\begin{center}
\caption{Collection of $^3P_2$-$^3F_2$ energy gaps (in MeV) for the 
modern potentials discussed.  
BHF single-particle energies have been used. In case of no results,
a vanishing gap was found.\label{tab:pgaps}}
\begin{tabular}{ccccc}\hline 
\multicolumn{1}{c}{$k_F\;({\rm fm}^{-1})$}& 
\multicolumn{1}{c}{CD-Bonn}&\multicolumn{1}{c}{$V_{18}$}&
\multicolumn{1}{c}{Nijm I}&\multicolumn{1}{c}{Nijm II} \\ \hline  
     1.2  & 0.04 & 0.04 & 0.04  & 0.04  \\
     1.4  & 0.10 & 0.10 & 0.10  & 0.10  \\
     1.6  & 0.18 & 0.17 & 0.18  & 0.18  \\
     1.8  & 0.25 & 0.23 & 0.26  & 0.26  \\
     2.0  & 0.29 & 0.22 & 0.34  & 0.36  \\
     2.2  & 0.29 & 0.16 & 0.40  & 0.47  \\
     2.4  & 0.27 & 0.07 & 0.46  & 0.67  \\
     2.6  & 0.21 &      & 0.47  & 0.99  \\
     2.8  & 0.17 &      & 0.49  & 1.74  \\
     3.0  & 0.11 &      & 0.43  & 3.14  \\ \hline
\end{tabular}

\end{center}
\end{table} 
Thus, before a precise calculation of $^3P_2$-$^3F_2$ energy gaps
can be made, one needs NN interactions that fit the scattering
data up to lab energies of $\sim 1$ GeV. This means 
in turn that the interaction models have to 
account for, due to the opening
of inelasticities above $350$ MeV, the
$N\Delta$ channel.


The reader should however note that the above results are
for pure neutron matter. We end therefore this section
with a discussion of the pairing gap for $\beta$-stable
matter of relevance for the neutron star cooling, see e.g.,
Ref.\ \cite{report}.
We will also omit a discussion on neutron pairing gaps in the
$^1S_0$ channel, since these appear at densities corresponding 
to the crust of the neutron star. The gap in the crustal material 
is unlikely
to have any significant effect on cooling processes \cite{pr95}, 
though
it is expected to be important in the explanation 
of glitch phenomena.
Therefore, the relevant pairing gaps for neutron star cooling
should stem from the 
the proton contaminant 
in the $^1S_0$ channel, and superfluid neutrons yielding energy gaps 
in the coupled $^3P_2$-$^3F_2$ two-neutron channel. 
If in addition one studies closely the phase shifts for
various higher partial waves of the NN interaction, one notices
that at the densities which will correspond to the  
core of the star, any superfluid 
phase should eventually disappear. This is due to the fact that
an attractive NN interaction is needed in order to
obtain a positive energy gap.
\begin{wrapfigure}{r}{6.6cm}   % r: RIGHT, 6.6cm: WIDTH  
          {\epsfxsize=10pc \epsfbox{fig1.eps}}
          \caption{Proton pairing in $\beta$-stable matter for 
          the $^1S_0$ partial wave.}
     \label{fig:figgap}
        \end{wrapfigure}
\begin{wrapfigure}{r}{6.6cm}   % r: RIGHT, 6.6cm: WIDTH  
          {\epsfxsize=10pc \epsfbox{fig2.eps}}
          \caption{Neutron pairing in 
          $\beta$-stable matter for the $^3P_2$
          partial wave.}
     \label{fig:figgap2}
        \end{wrapfigure}
Since the relevant total baryonic densities for these types of
pairing will be higher than the saturation
density of nuclear matter, we will account for relativistic
effects as well in the calculation of the pairing gaps.
As an example, consider the evaluation of the proton
$^1S_0$ pairing gap using a Dirac-Brueckner-Hartree-Fock  approach.
In Fig.\ \ref{fig:figgap} we plot as function of the total baryonic 
density the pairing gap for protons in the $^1S_0$
state, together with the results from the non-relativistic 
approach discussed in  Refs.\
\cite{elga96,eeho96}. 
These results are all 
for matter in $\beta$-equilibrium. In Fig.\ \ref{fig:figgap} 
we plot also the 
corresponding relativistic 
results for the neutron energy gap in the $^3P_2$ channel. 
For the $^3P_0$ and the $^1D_2$ channels we found both 
the non-relativistic and the relativistic
energy gaps to vanish. 
As can be seen from Fig.\ \ref{fig:figgap}, there are only small
differences (except for higher densities) between the non-relativistic
and relativistic proton gaps in the $^1S_0$ wave.
This is expected since the proton fractions (and their respective Fermi
momenta) are rather small.
For neutrons however, 
the Fermi momenta are larger, and we would 
expect relativistic effects to be important. At Fermi momenta
which correspond to the
saturation point of nuclear matter, $k_F=1.36$ fm$^{-1}$,
the lowest relativistic correction to the kinetic energy per 
particle is of the order of 2 MeV. 
At densities higher than the saturation
point, relativistic effects should be even 
more important.
Since we are dealing with
very small proton fractions, a Fermi momentum
of $k_F=1.36$ fm$^{-1}$, would correspond to a total baryonic 
density $\sim 0.09$  fm$^{-3}$. Thus, at larger densities 
relativistic effects for neutrons should
be important.
This is also reflected in Fig.\ \ref{fig:figgap2} for the pairing
gap in the $^3P_2$ channel.
The relativistic $^3P_2$ gap is less  than half
the corresponding non-relativistic one, and the 
density region is also much smaller, see Ref.\ \cite{eeho96} for further details.
This discussion  can be summarized as follows.
The $^1S_0$ proton gap in $\beta$-stable matter
            is $ \le 1$ MeV, and if polarization
            effects were taken into account \cite{sclbl96},
            it could be further reduced by a factor 2-3.

The $^3P_2$ gap is also small, of the order
            of $\sim 0.1$ MeV in $\beta$-stable matter.
            If relativistic effects are taken into account,
            it is almost vanishing. However, there is
            quite some uncertainty with the value for this
            pairing gap for densities above $\sim 0.3$
            fm$^{-3}$ due to the fact that the NN interactions
            are not fitted for the corresponding lab energies. 

Higher partial waves give essentially vanishing
            pairing gaps in $\beta$-stable matter.

Thus, the $^1S_0$ and $^3P_2$ partial waves are crucial for our
understanding of superfluidity in neutron star matter. 

We have not mentioned recent developments beyond the BCS approach,
nor have we discussed results for proton-neutron pairing in symmetric
or asymmetric matter. Such topics are addressed in the recent works
of Lombardo, Schulze and collaborators, see e.g., Refs.\ 
\cite{ls2000,ls2001} and references therein. 

\section{Pairing in finite nuclei} \label{sec:sec3}

\subsection{Tin isotopes}
We turn the attention to finite nuclei. Here we focus on the chain of 
tin isotopes. 
Of interest in this study is the fact that 
the chain of even tin isotopes from $^{102}$Sn to $^{130}$Sn 
exhibits a near constancy of the 
$2^+_1-0^+_1$ excitation energy, a constancy which can be related
to strong pairing correlations and the near degeneracy in energy 
of the relevant single particle orbits. As an example, we show the 
experimental\footnote{We will limit our discussion to even nuclei
from  $^{116}$Sn to $^{130}$Sn, since a qualitatively similar picture
is obtained from $^{102}$Sn to $^{116}$Sn.}
$2^+_1-0^+_1$ excitation energy 
from  $^{116}$Sn to $^{130}$Sn in Table \ref{tab:table1}. 
Our aim is to see whether the partial waves which played such a crucial
role in neutron star matter, viz., $^1S_0$ and $^3P_2$, are equally
important in reproducing the near constant spacing in the chain
of even tin isotopes shown in  Table \ref{tab:table1}. 

To achieve this, we mount a large-scale shell-model calculation in 
a model space relevant for the description of tin isotopes. 
In order to test the dependence on the above partial waves in
the NN interaction, different effective interactions are employed.
 
Our scheme to obtain an effective two-body interaction for 
the tin isotopes
starts with a free nucleon-nucleon  interaction $V$ which is
appropriate for nuclear physics at low and intermediate energies. 
In this work we will thus choose to work with the charge-dependent
version of the Bonn potential models, see \mbox{Ref. \cite{cdbonn}}.
With this interaction, we compute  
effective two-particle matrix elements based on 
a $Z = 50, \quad N = 82$ asymmetric core and with the active $P$-space for holes
based on the $2s_{1/2}$, $1d_{5/2}$, $1d_{3/2}$, $0g_{7/2}$ and $0h_{11/2}$
hole orbits, see e.g., Refs.~\cite{hko95,ehho97} for details.
The corresponding single-hole energies are
$\varepsilon(d_{3/2}^{+}) = 0.00$~MeV, 
 $\varepsilon(h_{11/2}^{-}) = 0.242$~MeV, $\varepsilon(s_{1/2}^{+}) = 0.332$~MeV,
$\varepsilon(d_{5/2}^{+}) = 1.655$~MeV and  $\varepsilon(g_{7/2}^{+}) = 2.434$~MeV
and the shell model calculation amounts to studying
valence neutron holes outside this core.
The shell model problem requires the solution of a real symmetric
$n \times n$ matrix eigenvalue equation
$\widetilde{H}\left | \Psi_k\right\rangle  = E_k \left | \Psi_k\right\rangle$ .
where for the present cases the dimension of the $P$-space reaches $n \approx 2 \times 10^{7}$.
At present our basic approach in finding solutions to shell-model problem
is the Lanczos algorithm; an iterative method
which gives the solution of the lowest eigenstates. This method was 
already applied to nuclear physics problems by Whitehead {\sl et al.} 
in 1977. The technique is described in detail in Ref.\ \cite{whit77}, 
see also Ref.\ \cite{ehho95}. 

In order to test whether the $^1S_0$ and $^3P_2$ partial waves are equally
important in reproducing the near constant spacing in the chain
of even tin isotopes as they are for the superfluid properties of infinite matter,
we study four different approximations to the shell-model
effective interaction, viz.,
\begin{enumerate}
  \item Our best approach to the effective interaction, $V_{\mathrm{eff}}$, contains
        all one-body and two-body diagrams through third order in the $G$-matrix, 
        see Ref.\ \cite{ehho97}. 
  \item The effective interaction is given by the $G$-matrix only and inludes
        all partial waves up to $l=10$.
  \item We define an effective  interaction based on a $G$-matrix which now includes
        only the $^1S_0$ partial wave.
  \item Finally, we use an effective interaction based on a $G$-matrix which does
        not contain the  $^1S_0$ and $^3P_2$ partial waves, but all other waves
        up to $l=10$.  
\end{enumerate}
In all four cases the same NN interaction is used, viz., 
the CD-Bonn interaction described in Ref.\ \cite{cdbonn}.
Table \ref{tab:table1} lists the results obtained for the three first cases.  
\begin{table}[t]
\begin{center}
\caption{ $2^+_1-0^+_1$ excitation energy for the 
even tin isotopes $^{130-116}$Sn for various approaches
to the effective interaction. See text for further details. 
Energies are given in MeV. \label{tab:table1}}
\begin{tabular}{lcccccccc}\hline
 & {$^{116}$Sn} & {$^{118}$Sn} & {$^{120}$Sn} &{$^{122}$Sn} & {$^{124}$Sn} & {$^{126}$Sn} & {$^{128}$Sn} & {$^{130}$Sn} \\ \hline
Expt & 1.29 & 1.23 & 1.17 & 1.14 & 1.13 & 1.14 & 1.17 & 1.23 \\
$V_{\mathrm{eff}}$ & 1.17 & 1.15 & 1.14 & 1.15 & 1.14 & 1.21 & 1.28 & 1.46 \\
$G$-matrix &1.14 & 1.12& 1.07 & 0.99 & 0.99 & 0.98 & 0.98 & 0.97  \\
$^1S_0$ $G$-matrix &1.38 &1.36 &1.34 &1.30 & 1.25& 1.21 &1.19 &1.18 \\\hline
\end{tabular}
\end{center}

\end{table}

We note from this Table that the three first cases nearly produce a constant 
$2^+_1-0^+_1$ excitation energy, with our most optimal effective interaction
$V_{\mathrm{eff}}$ being closest the experimental data. The bare $G$-matrix
interaction, with no folded diagrams as well, results in a slightly more compressed
spacing. This is mainly due to the omission of the core-polarization 
diagrams which typically render the $J=0$ matrix elements more attractive.
Such diagrams are included in $V_{\mathrm{eff}}$. 
Including only the $^1S_0$ partial wave in the construction of the  $G$-matrix
case 3,
yields in turn a somewhat larger spacing. This can again be understood from the
fact that a $G$-matrix constructed with this partial wave  
only does not receive contributions from any entirely repulsive partial wave.
It should be noted that our optimal interaction, as demonstrated in Ref.\ \cite{ehho97}, shows a rather good reproduction of the 
experimental spectra for both even and odd nuclei. Although the approximations
made in cases 2 and 3 produce an almost constant $2^+_1-0^+_1$ excitation energy,
they reproduce poorly the properties of odd nuclei and other 
excited states in the even Sn isotopes. 

However, the fact that the first three  approximations result in a such a good
reproduction of the  $2^+_1-0^+_1$ spacing may hint to the fact that the 
$^1S_0$ partial wave is of paramount importance. 
If we now turn the attention to case 4, i.e., we omit the
$^1S_0$ and $^3P_2$ partial waves in the construction of the $G$-matrix,
the results presented  in Table \ref{tab:table2} exhibit  a spectroscopic 
catastrophe. In this Table we do also not list eigenstates
with other quantum numbers. For e.g., $^{126}$Sn
the ground state is no longer a $0^+$ state, rather it carries the quantum numbers
$4^+$ while for $^{124}$Sn the ground state 
has $6^+$. The first $0^+$ state for this nucleus is given at an excitation
energy of $0.1$ MeV with respect to the $6^+$ ground state.
The general picture for other eigenstates is that of an extremely poor agreement
with data.  
\begin{table}[t]
\begin{center}
\caption{ $2^+_1-0^+_1$ excitation energy for the 
even tin isotopes $^{130-124}$Sn obtained with a $G$-matrix
effective interaction which excludes the important
pairing waves $^1S_0$ and $^3P_2$. See text for further details. 
Energies are given in MeV. \label{tab:table2}}
\begin{tabular}{lcccc}\hline
& {$^{124}$Sn} & {$^{126}$Sn} & {$^{128}$Sn} & {$^{130}$Sn} \\ \hline
No $^1S_0$ and $^3P_2$ in $G$-matrix &0.15  &-0.32  &0.02 &-0.21  \\\hline
\end{tabular}
\end{center}
\end{table}
Since the agreement is so poor, even the qualitative reproduction of the 
$2^+_1-0^+_1$ spacing, we defer from performing time-consuming shell-model
calculations for $^{116,118,120,122}$Sn.

\subsection{Thermodynamic properties of rare earth nuclei}
The thermodynamical properties of nuclei deviate from infinite systems. 
While the quenching of pairing in superconductors is well described as a 
function of temperature, the nucleus represents a finite many body system 
characterized by large fluctuations in the thermodynamic observables. A 
long-standing problem in experimental nuclear physics has been to observe the 
transition from strongly paired states, at around $T=0$, to unpaired states at 
higher temperatures. 

In nuclear theory, the pairing gap parameter $\Delta$ can be studied as 
function of temperature using the BCS gap equations \cite{SY63,Go81}. From this
simple model the gap decreases monotonically to zero at a critical temperature 
of $T_c\sim 0.5\,\Delta$. However, if particle number is projected out 
\cite{FS76,DK95}, the decrease is significantly delayed. The predicted decrease
of pair correlations takes place over several MeV of excitation energy 
\cite{DK95}. Recently \cite{MB99}, structures in the level 
densities in the 1--7~MeV region were reported, structures which 
probably are due to the breaking of 
nucleon pairs and a gradual decrease of pair correlations. 

Experimental data on the quenching of pair correlations are important as a 
test for nuclear theories. Within finite temperature BCS and RPA models, level 
density and specific heat are calculated for e.g., $^{58}$Ni \cite{Ng90}; 
within the shell model Monte Carlo method (SMMC) \cite{LJ93,KD97} one is now 
able to estimate level densities \cite{Or97} in heavy nuclei \cite{WK98} up to 
high excitation energies. 
Here we report on the observation of the gradual 
transition from strongly paired states to unpaired states in rare earth nuclei 
at low spin. The canonical heat capacity is used as a thermometer. Since only 
particles at the Fermi surface contribute to this quantity, it is very 
sensitive to phase transitions. It has been demonstrated from SMMC calculations
in the Fe region \cite{RH98,AL99,Al99}, that breaking of only one nucleon pair 
increases the heat capacity significantly. 

The experiments were carried out with 45~MeV $^3$He projectiles from the MC-35 
cyclotron at the University of Oslo. In that experiment, one could
extract level densities and $\gamma$
strength functions for the $^{161,162}$Dy and $^{171,172}$Yb nuclei. The data 
for the even nuclei are published recently \cite{MB99,schiller2001}.

The partition function in the canonical ensemble 
$Z(T)=\sum_{n=0}^\infty\rho(E_n)e^{-E_n/T} $
is determined by the measured level density of accessible states $\rho(E_n)$ in
the present nuclear reaction. Strictly, the sum should run from zero to 
infinity. Here  we calculate $Z$ for temperatures up to $T=1$~MeV. 
However, the experimental level
densities only cover the excitation region up 
close to the neutron binding energy of about 6 and 8~MeV for odd and even mass 
nuclei, respectively. For higher energies it is reasonable to assume Fermi gas 
properties, since single particles are excited into the continuum region with 
high level density. Therefore, due to lack of experimental data, the level 
density is extrapolated to higher energies by the shifted Fermi gas model 
expression \cite{GC65}. 
The extraction of the microcanonical heat capacity $C_V(E)$ gives large 
fluctuations which are difficult to interpret \cite{MB99}. Therefore, the heat 
capacity $C_V(T)$ is calculated within the canonical ensemble, where $T$ is a 
fixed input value in the theory, and a more appropriate parameter,
see e.g., Schiller {\em et al.} \cite{schiller2001} for further details.
\begin{wrapfigure}{r}{8cm}
          {\epsfxsize=14pc \epsfbox{fig3.ps}}
          \caption{Heat capacity for iron isotopes, see Ref.\ \cite{Al99}, 
and for $^{161,162}$Dy. See text for further details. \label{fig:heatcapacity}}
\end{wrapfigure}
The deduced heat capacities for the $^{161,162}$Dy nuclei 
are shown in Fig.~\ref{fig:heatcapacity} together with the SMMC
results of Liu and Alhassid \cite{Al99} for various iron isotopes.
The results labelled 'model' are discussed further in Refs.\ 
\cite{magne1,magne2}. We note that both the theoretical and 
experimental results exhibit 
S-shaped $C_V(T)$-curves.
The S-shaped curve is interpreted as a 
fingerprint of a phase transition in a finite system from a phase with strong 
pairing correlations to a phase without such correlations. Due to the strong 
smoothing introduced by the transformation to the canonical ensemble, we do not
expect to see discrete transitions between the various quasiparticle regimes, 
but only the transition where all pairing correlations are quenched as a whole.
It is worth noticing that the S-shape is much less 
pronounced for the odd system,
again a possible indication of the importance of pairing correlations. 
This can also be seen from Fig.\ \ref{fig:entropy}, taken from Ref.\
\cite{magne3}.
\begin{wrapfigure}{r}{8cm}
          {\epsfxsize=14pc \epsfbox{fig4.ps}}
          \caption{Experimental entropy in the canonical ensemble for $^{161,162}$Dy and
for $^{171,172}$Yb. \label{fig:entropy}}
\end{wrapfigure}
Here we notice that the entropy of the even and odd systems merge
at a temperature $T\approx 0.5$ MeV, in close agreement with the point where
the S-shape of the heat capacity of 
the $^{161,162}$Dy nuclei appears  
in Fig.~\ref{fig:heatcapacity}. The temperature where 
the experimental entropies merge,
could in turn be interpreted as the point where other degrees of freedom
than pairing take over. A theoretical interpretation
in terms of the vanishing of pairing correlations is given 
in Refs.\ \cite{magne3,bdh2001}. 
\section{Conclusion} \label{sec:sec4}


In summary, the $^1S_0$ and $^3P_2$ partial waves are crucial for our
understanding of superfluidity in neutron star matter. In addition, viewing
the results of Table \ref{tab:table2}, one sees that 
pairing correlations, being important for the 
reproduction of the $2^+_1-0^+_1$ excitation energy of
the even Sn isotopes, depend strongly on the same  partial waves
of the NN interaction. Omitting these waves, especially the  $^1S_0$ wave,
results in a spectrum which has essentially no correspondence with experiment.

Furthermore, we have also discussed 
recent experimental and theoretical studies 
of thermodynamical properties of finite nuclei
and their interpretation in terms  of eventual pairing transitions 
in finite nuclei. 


\section*{Acknowledgements}
I am much indebted to Alexandar Beli{\'c} (Belgrade), 
David Dean (ORNL), Torgeir Engeland (Oslo), \O ystein Elgar\o y (Cambridge),
Magne Guttormsen (Oslo), Eivind Osnes (Oslo)  
and Andreas Schiller (LLNL) 
for the many discussion on the topics addressed here.

\begin{thebibliography}{99}
\bibitem{glitch} 
J.A.\ Sauls, in: Timing Neutron Stars, 
eds.\ H. \"{O}gelman, and E.P.J.\ van den Heuvel,   
(Dordrecht, Kluwer, 1989) p.\ 457. 
\bibitem{bcll90}
  M.\ Baldo, J.\ Cugnon, A.\ Lejeune, and U.\ Lombardo,
  Nucl.\ Phys.\ A {\bf 515}, 409 (1990).
\bibitem{kkc96} 
  V.A.\ Khodel, V.V.\ Khodel, and J.W.\ Clark,
  Nucl. Phys. A {\bf 598}, 390 (1996).
\bibitem{eh98}
  \O. Elgar\o y and M. Hjorth-Jensen,
  Phys. Rev.  C {\bf 57}, 1174 (1998). 
\bibitem{sclbl96} 
  H.-J.\ Schulze, J.\ Cugnon, A.\ Lejeune, M.\ Baldo, 
  and U.\ Lombardo, 
  Phys.\ Lett.\  { \bf B375}, 1 (1996). 
\bibitem{chen86}
  J.M.C.\ Chen, J.W.\ Clark, E.\ Krotschek and R.A.\ Smith,
  Nucl.\ Phys.\ A {\bf 451}, 509 (1986);
  J.M.C.\ Chen, J.W.\ Clark, R.D.\ Dave, and V.V.\ Khodel,
  Nucl.\ Phys.\ A {\bf 555}, 59 (1993).
\bibitem{ains89}
  T.L.\ Ainsworth, J.\ Wambach, and D.\ Pines,
  Phys.\ Lett.\ B {\bf 222}, 173 (1989);
  J.\ Wambach, T.L.\ Ainsworth, and D.\ Pines,
  Nucl.\ Phys.\ A {\bf 555}, 128 (1993).
\bibitem{amu85}
  L.\ Amundsen and E.\ \O stgaard, 
  Nucl.\ Phys.\ A {\bf 437}, 487 (1985).
\bibitem{bcll92}
  M.\ Baldo, J.\ Cugnon, A.\ Lejeune, and U.\ Lombardo,
  Nucl.\ Phys.\ A {\bf 536}, 349 (1992).
\bibitem{taka93}
  T.\ Takatsuka and R.\ Tamagaki,
  Prog. Theor. Phys. Suppl. {\bf 112}, 27 (1993).
\bibitem{elga96}
  \O.\ Elgar\o y, L.\ Engvik, M.\ Hjorth-Jensen, and E.\ Osnes,
  Nucl.\ Phys.\ A {\bf 607}, 425 (1996).
\bibitem{khodel97} 
  V.V.\ Khodel, PhD.\ Thesis, Washington University, St. Louis, 
  unpublished (1997); 
  V.A.\ Khodel, V.V.\ Khodel, and J.W.\ Clark, 
  Phys.\ Rev.\ Lett.\ {\bf 81}, 3828 (1998); 
  V.A.\ Khodel, V.V.\ Khodel, and J.W.\ Clark, 
  Nucl.\ Phys.\ A {\bf  679}, 827 (2001);
  V.A.\ Khodel, V.V.\ Khodel, and J.W.\ Clark, 
  Phys.\ Rev.\ Lett.\ {\bf 81}, 3828 (2001).


\bibitem{v18} R.B.\ Wiringa, V.G.J.\ Stoks, and R.\ Schiavilla, 
Phys.\ Rev.\ C {\bf 51}, 38 (1995).
\bibitem{cdbonn}     R.\ Machleidt, F.\ Sammarruca, and Y.\ Song,
                     Phys.\ Rev.\ C {\bf 53}, 1483  (1996).
\bibitem{nim} V.G.J.\ Stoks, R.A.\ M.\ Klomp, C.P.F.\ Terheggen, 
and J.J.\
de Swart, Phys.\ Rev.\ C {\bf 49}, 2950  (1994).
\bibitem{beehs98} M.\ Baldo, \O.\ Elgar\o y, L.\ Engvik, 
                  M.\ Hjorth-Jensen, and H.-J.\ Schulze,
                  Phys.\ Rev.\ C {\bf 58}, 1921 (1998).
\bibitem{report} H.\ Heiselberg and M.\ Hjorth-Jensen, Phys.\ Rep.\ {\bf 328},
237 (2000).
                 and nucl-th/9902033.
\bibitem{pr95} C.J.\ Pethick and D.G.\ Ravenhall, 
Ann.\ Rev.\ Nucl.\ Part.\ Sci.\ {\bf 45}, 429 (1995).

\bibitem{eeho96} \O.\ Elgar\o y, L.\ Engvik, 
                  M.\ Hjorth-Jensen, and E.\ Osnes, 
                  Nucl.\ Phys.\ A {\bf 604}, 466 (1996).
\bibitem{ls2000} U.\ Lombardo and H.-J.\ Schulze, astro-ph/0012209 and references therein.
\bibitem{ls2001} U.\ Lombardo, P.\ Nozieres, P.\ Schuck, H.-J.\ Schulze,
and A.\ Sedrakian, Phys.\ Rev.\ C {\bf 64}, 064314 (2001).
\bibitem{hko95}  M.\ Hjorth-Jensen, T.\ T.\ S.\ Kuo, and
E.\ Osnes, Phys.\ Reports {\bf 261}, 125 (1995).

\bibitem{ehho97}     A.\ Holt, T.\ Engeland, M.\ Hjorth-Jensen, and E.\ Osnes,
                     Nucl.\ Phys.\ A {\bf 634}, 41 (1998) .


\bibitem{whit77}     R.\ R.\ Whitehead, A.\ Watt, B.\ J.\ Cole, and I.\
                     Morrison, 
                     Adv.\ Nucl.\ Phys.\ {\bf 9}, 123 (1977).

\bibitem{ehho95}     T.\ Engeland, M.\ Hjorth-Jensen, A.\ Holt, and E.\ Osnes,
                     Phys.\ Scripta {\bf T56}, 58  (1995).
\bibitem{SY63}M. Sano and S. Yamasaki, Prog.\ Theor.\ Phys.\ \bf 29\rm, 397 
(1963). 
\bibitem{Go81}A.L. Goodman, Nucl.\ Phys.\ A \bf 352\rm, 45 (1981) .
\bibitem{FS76}A. Faessler \sl et al.\rm, Nucl.\ Phys.\ A\bf 256\rm, 106 (1976). 
\bibitem{DK95}T. D{\o}ssing \sl et al.\rm, Phys.\ Rev.\ Lett.\ \bf 75\rm, 1276
(1995).  
\bibitem{MB99}E. Melby \sl et al.\rm, Phys.\ Rev.\ Lett.\ \bf 83\rm, 3150 
(1999).
\bibitem{schiller2001}A. Schiller \sl et al.\rm, Phys.\ Rev.\ C \bf 63 \rm, 021306(R) 
(2001).
\bibitem{Ng90}Nguyen Dinh Dang, Z. Phys.\ A \bf 335\rm, 253 (1990).
\bibitem{LJ93}G.H. Lang, C.W. Johnson, S.E. Koonin, and W.E. Ormand, Phys.\ 
Rev.\ C \bf 48\rm, 1518 (1993).
\bibitem{KD97}S.E. Koonin, D.J. Dean, and K. Langanke, Phys.\ Rep.\ \bf 278\rm,
1 (1997). 
\bibitem{Or97}W.E. Ormand, Phys.\ Rev.\ C \bf 56\rm, R1678 (1997).
\bibitem{WK98}J.A. White, S.E. Koonin, and D.J. Dean, Phys.\ Rev.\ C \bf 61 \rm, 034303 (2000). 
\bibitem{RH98}S. Rombouts, K. Heyde, and N. Jachowicz, Phys.\ Rev.\ C \bf 58\rm,
3295 (1998).
\bibitem{AL99}Y. Alhassid, S. Liu, and H. Nakada, Phys.\ Rev.\ Lett.\ \bf 
83\rm, 4265 (1999).
\bibitem{Al99} S. Liu and Y. Alhassid, Phys.\ Rev.\ Lett.\ \bf 
87\rm,  022501 (2001).
\bibitem{Be36}H.A. Bethe, Phys.\ Rev.\ \bf 50\rm, 332 (1936). 
\bibitem{GC65}A. Gilbert and A.G.W. Cameron, Can.\ J. Phys.\ \bf 43\rm, 1446
(1965).
\bibitem{magne1}M. Guttormsen, M. Hjorth-Jensen, E. Melby, J. Rekstad, A. 
Schiller, and S. Siem, Phys.\ Rev.\ C \bf 63\rm, 044301 (2001).
\bibitem{magne2}M. Guttormsen, M. Hjorth-Jensen, E. Melby, J. Rekstad, A. 
Schiller, and S. Siem, Phys.\ Rev.\ C \bf 64\rm, 034319 (2001).
\bibitem{magne3}M. Guttormsen, A. Bjerve, M. Hjorth-Jensen, E. Melby, J. Rekstad,
A. Schiller, S. Siem, and A. Beli{\'c}, Phys. Rev. C \bf 62\rm, 024306 (2000).
\bibitem{bdh2001}A. Beli{\'c}, D. J. Dean, and M. Hjorth-Jensen, preprint 
cond-mat/0104138.


\end{thebibliography}



\end{document}



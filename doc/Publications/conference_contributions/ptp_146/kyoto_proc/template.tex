%%%%%%%%%%%%%%%%%%%%%%%%%%%%%%%%%%%%%%%%%%%%%%%%%%%%%%
%%%%%%   template.tex for PTPTeX.sty <ver.1.0>  %%%%%%
%%%%%%%%%%%%%%%%%%%%%%%%%%%%%%%%%%%%%%%%%%%%%%%%%%%%%%
\documentstyle[seceq]{ptptex}
%\documentstyle[seceq,preprint]{ptptex}
%\documentstyle[seceq,letter]{ptptex}
%\documentstyle[seceq,supplement]{ptptex}
%\documentstyle[seceq,addenda]{ptptex}
%\documentstyle[seceq,errata]{ptptex}

%%%%% Personal Macros %%%%%%%%%%%%%%%%%%%


%%%%%%%%%%%%%%%%%%%%%%%%%%%%%%%%%%%%%%%%%
%\pubinfo{Vol. 101, No. 4, Aril 1999}  %Editorial Office use
%\setcounter{page}{}                   %Editorial Office use
%------------------------------------------------------------
%\nofigureboxrule%to eliminate the rule of \figurebox
%\notypesetlogo  %comment in if to eliminate PTPTeX logo
%\subfontMBF     %use if you have not enough fonts when using mbf.sty
%---- When [preprint] you can put preprint number at top right corner.
%\preprintnumber[3cm]{%<-- [..]: optional width of preprint # column.
%KUNS-1325\\ HE(TH)~97/04\\ hep-th/9702083}
%-------------------------------------------

\markboth{%     %running head for odd-page (authors' name)
S.-I.~Tomonaga and H.~Yukawa
}{%             %running head for even-page (`short' title)
Instruction for Making \LaTeX\ Compuscripts Using \protect\PTPTeX
}


\title{%        %You can use \\ for explicit line-break
Instruction for Making \LaTeX\ Compuscripts Using \PTPTeX
}
%\subtitle{This is a Subtitle}    %use this when you want a subtitle

\author{%       %Use \sc for the family name
Shin-Ichiro {\sc Tomonaga}\footnote{A friend of Schwinger 
because of bearing `swing' commonly in their names.} 
and Hideki {\sc Yukawa}$^{*,}$\footnote{A common friend of Fermi and
Bose. E-mail address: yukawa@yukawa.kyoto-u.ac.jp}
}

\inst{%         %Affiliation, neglected when [addenda] or [errata]
Physics Department, Tokyo Bunrika University, Tokyo 113
\\
$^*$Yukawa Institute for Theoretical Physics, 
Kyoto University, Kyoto 606-8502}

%\publishedin{%      %Write this ONLY in cases of addenda and errata
%Prog.~Theor.~Phys.\ {\bf XX} (19YY), page}

\recdate{%      %Editorial Office will fill in this.
%\today
}

\abst{%       %this abstract is neglected when [addenda] or [errata]
Write your ABSTRACT here.
}

\begin{document}

\maketitle

\section{Introduction}
Start your paper from here.


\section{Equations}

\subsection{Fractions}


\section{References}


\section*{Acknowledgements}
We would like to thank ...........

\appendix
\section{First Appendix} %Empty argument \section{} yields `Appendix'. 

\section{Second Appendix}


\begin{thebibliography}{99}
%%%%%%%%%%%%%%%%%%%%%%%%%%%%%%%%%%%%%%%%%%%%%%%%%%%%%%%%%%%%%
% Some macros are available for the bibliography:
%   o for general use
%      \JL : general journals          \andvol : Vol (Year) Page
%   o for individual journal 
%      \PR  : Phys. Rev.               \PRL : Phys. Rev. Lett.
%      \NP  : Nucl. Phys.              \PL  : Phys. Lett.
%      \JMP : J. Math. Phys.           \CMP : Commun. Math. Phys.
%      \PTP : Prog. Theor. Phys.       \JPSJ: J. Phys. Soc. Jpn.
%      \JP  : J. of Phys.              \NC  : Nouvo Cim.
%      \IJMP: Int. J. Mod. Phys.       \ANN : Ann. of Phys.
% Usage:
%   \PR{D45,1990,345}            ==> Phys.~Rev.\ {\bf D45} (1990), 345
%   \JL{Phys.~Lett.,A30,1981,56} ==> Phys.~Lett.\ {\bf A30} (1981), 56
%   \andvol{B123,1995,1020}      ==> {\bf B123} (1995), 1020
%%%%%%%%%%%%%%%%%%%%%%%%%%%%%%%%%%%%%%%%%%%%%%%%%%%%%%%%%%%%%
\bibitem{rf:NJL}
Y.~Nambu and G.~Jona-Lasinio, 
        Phys.~Rev.\ {\bf 122} (1961), 345.   %End with period .(not ;)
\\
M.~Kobayashi and T.~Maskawa, \JL{Prog.~Theor.~Phys.,49,1973,652}.
\bibitem{rf:PW}
A.~Polyakov and P.B.~Wiegmann, 
        Phys.~Lett.\ {\bf B131} (1983), 121.
\\
P.B.~Wiegmann, \PL{B141,1984,217}; \andvol{B142,1984,173}.
\bibitem{rf:SW}
R.~Streater and A.S.~Wightman, {\it PCT, Spin \& Statistics, and All
That} (W.A.~Benjamin, New York, 1964), p.~117.
\end{thebibliography}

\end{document}







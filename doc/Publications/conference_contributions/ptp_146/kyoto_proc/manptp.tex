%%%%%%%%%%%%%%%%%%%%%%%%%%%%%%%%%%%%%%%%%%%%%%%%%%%%%%
%%%%%%   template.tex for PTPTeX.sty <ver.0.8>  %%%%%%
%%%%%%%%%%%%%%%%%%%%%%%%%%%%%%%%%%%%%%%%%%%%%%%%%%%%%%
\documentstyle[seceq,mbf,wrapft]{ptptex}
%\documentstyle[seceq,preprint,mbf,wrapft]{ptptex}
%\documentstyle[seceq,letter]{ptptex}
%\documentstyle[seceq,supplement]{ptptex}
%\documentstyle[seceq,addenda]{ptptex}
%\documentstyle[seceq,errata]{ptptex}
%%%%% Personal Macros %%%%%%%%%%%%%%%%%%%
\def\BS{\tt\symbol{"5C}}        %backslash
\def\PTP{{\sl Progress of Theoretical Physics}}        %backslash
\def\ttmac#1{{\tt \BS #1}}
\def\boxmac#1{\fbox{\ttmac{#1}}}
\def\boxenv#1{\fbox{\tt #1}}

%%%%%%%%%%%%%%%%%%%%%%%%%%%%%%%%%%%%%%%%%
\pubinfo{Vol.~95, No.~4, April 1996}  %Editorial Office use
%\setcounter{page}{}                 %Editorial Office use
%------------------------------------------------------------
%\nofigureboxrule%to eliminate the rule of \figurebox
%\notypesetlogo  %comment in if to eliminate PTPTeX logo
%\subfontMBF     %use if you have not enough fonts when using mbf.sty
%---- When [preprint] you can put preprint number at top right corner.
\preprintnumber[3cm]{%<-- [..]: optional width of preprint # column.
KUNS-1325\\PTPTeX ver.0.8\\ August, 1997}
%-------------------------------------------

\markboth{%     %running head for odd-page (authors' name)
S.-I.~Tomonaga and H.~Yukawa
}{%             %running head for even-page (`short' title)
Instruction for Making \LaTeX\ Compuscripts Using \protect\PTPTeX}

\title{%        %You can use \\ for explicit line-break
Instruction for Making \LaTeX\ Compuscripts Using \PTPTeX
}
%\subtitle{This is a Subtitle}    %use this when you want a subtitle

\author{%       %Use \sc for the family name
Shin-Ichiro {\sc Tomonaga}\footnote{A friend of Schwinger 
because of bearing `swing' commonly in their names.} 
and Hideki {\sc Yukawa}$^{*,}$\footnote{A common friend of Fermi and
Bose. E-mail address: yukawa@yukawa.kyoto-u.ac.jp}
}

\inst{%         %Affiliation, neglected when [addenda] or [errata]
Physics Department, Tokyo Bunrika University, Tokyo 113
\\
$^*$Yukawa Institute for Theoretical Physics, 
Kyoto University, Kyoto 606-01
}

%\publishedin{%         %Write this ONLY in cases of addenda and errata
%Prog.~Theor.~Phys.\ {\bf XX} (19YY), page}

\recdate{%      %Editorial Office will fill in this.
April 2, 1996}

\abst{%         %this abstract is neglected when [addenda] or [errata]
This is a manual for making \LaTeX\ compuscripts for 
\PTP\ using the style file \PTPTeX.sty. In particular
we explain some useful macros and environments which are 
specially prepared in \PTPTeX.sty. 
The source file of this manual itself is designed to give a template 
which can be used for writing your own compuscripts.
}

\begin{document}

\maketitle

%\tableofcontents

\makeatletter
\if 0\@prtstyle
\def\asp{.3em} \def\bsp{.26em}
\else
\def\asp{.3em} \def\bsp{.3em}
\fi \makeatother

\section{Introduction}

We have in part started the printing of \PTP\ (PTP) 
directly using \LaTeX\cite{rf:1} manuscripts from April, 1996. 
When you prepare \LaTeX manuscript (compuscript) to submit to PTP, 
please use our style file \PTPTeX.sty. 
In \PTPTeX, you can use all the usual \LaTeX\ commands of course, 
and also some useful macros and environments in addition which are 
specially prepared. This manual explains about those commands special 
to \PTPTeX\ from Sect.2 on.

\subsection{Submission}

Use mainly {\bf E-mail} to submit the \LaTeX\ compuscript written by 
using \PTPTeX\ style file. The address is the Editorial Office of PTP:
{\vskip \asp
\begin{verbatim}
             ptp@yukawa.kyoto-u.ac.jp
\end{verbatim}\vskip\bsp}
\noindent
If the compuscript consists of two or more files, {\it e.g.}, 
by containing 
ps/eps figure files other than \LaTeX\ file itself, please first 
combine and pack them into a single file in either way of the 
following two and send it to the above address:
\begin{enumerate}
\item[1)] Use (P.~Ginsparg's) {\tt uufiles} command on UNIX, or, 
\newline 
do `manually' \ {\tt tar+gzip(compress)+uuencode};
\item[2)] Use \ {\tt lha+ish}.
\end{enumerate}

Don't forget to write your name, institution and address in the mail.
Generally when you send E-mail message to PTP Editorial Office, 
please indicate, in the Subject line of the mail header, 
Subject: $<$type of message$>$ $<$manuscript Ref.Number$>$ $<$first author's last name$>$; 
for instance,
\begin{tabbing}
WW\={\tt submit PTP Kobayashi} \= : Final manuscript after accepted for publication  \kill   %setting width
\>{\tt submit PTP Kobayashi} \>: Submission of a new manuscript \\
\>{\tt final 5Z03L Sato}
\>: Final manuscript after accepted for publication \\
\>{\tt status 5X31 Maskawa} \>  : Status query \\
\>{\tt report 6425 Yamada}\>   : Referee report
\end{tabbing}

In case E-mail is not available, we also accept floppy disk. 
In this case, however, do the initial submission by sending 
two hard copies of the manuscript and a letter by conventional mail 
as before. Indicate ``\LaTeX\ submission" clearly in your letter. 
At the stage when the paper is accepted for publication, we ask you 
to send your final \LaTeX\ compuscript by floppy disk to the PTP 
office. Use then {\bf 3.5 inch} floppy disk with DOS/V or NEC formatted.

In either case, 
the communication between Editorial Office and the author such as 
the information of referee reports or the author's reply to it, 
is performed via {\it conventional mail} using the {\it hard copy} of 
the manuscript as before. If the author makes a revised manuscript 
in this process, send it by {\it conventional mail} together with the 
old manuscript.

When the paper is accepted for publication, the 
PTP Editorial Office makes corrections/additions to 
the {\bf hard copy} of the manuscript and send it back to the author. 
The author is asked to examine the suggested corrections/additions 
there and to make necessary corrections to the \LaTeX\ source file. 
Then send the completed \LaTeX\ file as the final manuscript to 
the PTP office either by E-mail or by conventional mail (floppy disk).

\subsection{How to get \protect\PTPTeX\ style file}

\PTPTeX\ style file can be obtained by accessing to
WWW Home Page of Progress of Theoretical Physics: 
{\vskip \asp \baselineskip 1.1em
\begin{verbatim}
       PTP Home Page:  http://www2.yukawa.kyoto-u.ac.jp/~ptpwww/
\end{verbatim}\vskip\bsp}\noindent
Please try to get the most recent version since it is revised from
time to time.

The file {\tt ptptex08.uu} placed there is a 
`uuencoded-gzipped-tar' file, which is automatically unpacked and yield 
the following eight files if you simply execute 
\linebreak {\tt \% csh ptptex.uu} 
%\vskip -.2em
%\begin{verbatim}
%         \% csh ptptex.uu 
%\end{verbatim}\vskip\bsp
%\noindent
on your UNIX machine.
(In the above directories, you can also find 
the other forms of packed files, 
`lha+ish'-ed file {\tt ptptex08.ish} and 
`gzipped-tar' file {\tt ptptex08.tar.gz}, 
as well as the un-packed files themselves. Take suitable ones depending 
on your system environment.)
\begin{center}
\let\tabularsize\normalsize
\begin{tabular}{lp{.7\textwidth}} 
1. manptp.tex & Source file of this manual.\\
2. template.tex & Template for making your \PTPTeX\ compuscript.\\
3. ptptex.sty & Main style file of \PTPTeX.\\
4. ptp-text.sty & Style file for making PTP text style output.\\
5. ptp-prep.sty & Style file for making preprint style output.\\
6. wrapft.sty & Macro for wrapfigure and wraptable environments. \\
7. mbf.sty & Macro for using math-boldface ({\tt \BS mbf}).\\
8. epsf.sty & Macro for putting ps/eps figure files in 
\LaTeX\ manuscript. \\
\end{tabular}\end{center}
The macros 6 -- 8 are optional. 
The last {\tt epsf.sty} is a style file attached to the standard
\LaTeX\ system. It is, however, put here for reference since there
seems to exist another different style file bearing the same name.



\section{Style and preamble}

When you write a compuscript of your paper using \PTPTeX.sty, 
make use of {\bf template.tex} file which is prepared for the users' 
convenience as a separate file. By filling in it, 
you can easily make your compuscript a la \PTPTeX.sty format. 

The default style of \PTPTeX.sty is designed for Regular 
Articles in {\sl Progress of Theoretical Physics}. 
You can simply start your \LaTeX\ source file with the line 
{\vskip \asp \baselineskip 1.1em
\begin{verbatim}
        \documentstyle{ptptex}
\end{verbatim}%\vskip \asp
or, %\vskip \asp
\begin{verbatim}
        \documentstyle[seceq]{ptptex}
\end{verbatim}\vskip\bsp}
\noindent
Generally the argument of \LaTeX\ command put in a square bracket
{\tt [ ]} is called an optional argument and can be omitted. 
The optional argument {\tt [seceq]} in the second is a declaration 
to put the equation numbers with 
section numbers like (2$\cdot$15).  If you omit the option, then the 
sequential numbers are put to the equations throughout the paper, like
(1),\ (2),\ $\cdots$.

In the usual \LaTeX's \verb+article.sty+, this first line is like 
{\vskip \asp
\begin{verbatim}
        \documentstyle[12pt]{article}
\end{verbatim}\vskip\bsp}
\noindent
If you rewrite it as above, then the output automatically becomes of 
the Article form of {\sl Progress of Theoretical Physics}. 
Since \PTPTeX.sty is designed to be as compatible with the usual 
\LaTeX\ \verb+article.sty+ as possible, the source file which is 
written with \LaTeX\ \verb+article.sty+ can also be compiled with
\PTPTeX.sty as well.

If you want an output in a preprint form 
for your private distribution, you should start with the line
{\vskip \asp
\begin{verbatim}
        \documentstyle[seceq,preprint]{ptptex}
\end{verbatim}\vskip\bsp}
\noindent

If you want to write Letter article, Supplement article, Addenda or 
Errata, instead of the regular PTP article, please start with the 
line, respectively,
{\vskip \asp \baselineskip 1.1em
\begin{verbatim}
        \documentstyle[letter]{ptptex}
        \documentstyle[seceq,supplement]{ptptex}
        \documentstyle[seceq,addenda]{ptptex}
        \documentstyle[seceq,errata]{ptptex}
\end{verbatim}\vskip\bsp}
\noindent

The part of the source file from the first \verb+\documentstyle+ line 
to the declaration line of
{\vskip \asp
\begin{verbatim}
        \begin{document}
\end{verbatim}\vskip\bsp}
\noindent
is called {\it preamble}. One usually put in the preamble 
the definitions of personal macros and some style specifications. 
But, since the style specifications are quite unnecessary 
(and even harmful) in \PTPTeX, please do not write them. 
In \PTPTeX, you should fill in the following in the preamble:
{\vskip \asp \baselineskip 1.1em
\begin{verbatim}
        \markboth{ }{ }    : Running head
        \title{ }          : Title of the paper
        \author{ }         : Author's name
        \inst{ }           : Institution (address)
        \abst{ }           : Abstract
        \recdate{ }        : Received date
\end{verbatim}\vskip\bsp}
\noindent
You can see how to write these if you use {\tt template.tex} 
where an example is written. 
(Beware of the fact that the commands from \verb+\author+ to 
\verb+\recdate+ are macros particular to \PTPTeX.sty. 
So, if you compile the file 
with those commands by \LaTeX\ \verb+article.sty+, you will get 
an error message\  `{\tt ! Undefined control sequence}'.)


In the preamble of {\tt template.tex}, the following 
commands other than these are written (with comment out symbol 
{\tt \%} in front):
%\begin{center}
{\let\tabularsize\normalsize
\begin{tabular}{lcp{.5\textwidth}} 
{\tt \BS notypesetlogo}
&:& Not to print out `{\sf typeset using \PTPTeX.sty}' \\
{\tt \BS subfontMBF}
&:& To use subfont for \verb+\mbf+ (explained later)\\
{\tt \BS publishedin\{ \}}
&:& Vol/Year/Page of the paper for which Addenda or Errata is written.\\
{\tt \BS preprintnumber}[{\it width}]{\tt\{ \}}
&:& To write preprint numbers when [preprint]. [{\it width}] stands
for the width of the preprint number column. \\
\end{tabular}}%\end{center}
\vskip \bsp
\noindent
Use these when necessary. The usage will also be clear in 
the {\tt template.tex}.
%        \nofigureboxrule  : Cmmand to delete \figurebox's rule

\section{Equations}

Here we explain a few macros and environments 
which can be used in \PTPTeX\ in addition to the usual ones in \LaTeX.

\subsection{Mathematical italic bold}

A macro \boxmac{mib} is available to obtain 
math-bold (mathematical italic bold) fonts which are used, 
for instance, to denote 3-vectors. 
This macro is a very simple one which is defined as 
{\vskip \asp
\begin{verbatim}
        \def\mib#1{\mbox{\boldmath $#1$}} 
\end{verbatim}\vskip\bsp}
\noindent
If you type, for instances, \ttmac{mib\{\BS alpha\}} and 
\ttmac{mib\{kx\}} in math-mode, you can obtain math-bold outputs 
$\mib{\alpha}$,\ $\mib{kx}$. 

This macro has, however, a trouble that it
does not make the letter in an appropriate size when used in a suffix, 
and moreover you cannot make it small even writing 
\ttmac{small} or \ttmac{scriptsize}.

We prepared a new font command \boxmac{mbf} free from this trouble 
which can be used just like other font commands \ttmac{it}, \ttmac{bf}
and so on. If you write \ttmac{mbf}, then you enter in the math-bold 
mode and all the letters following are printed in math-bold fonts. 
(Therefore you have to specify the region by sandwiching 
by \verb+{ }+ like \verb+{\mbf \alpha kx}+.) \ 
The suffices in the math-bold mode automatically becomes 
small. (Like in the usual boldface mode by {\tt \BS bf}, 
the symbols {\tt  + - < / > ( ) [ ]} do not become math-bold face. 
If you want to make them also math-bold, add \boxmac{mbfplus} and 
write like {\tt \BS mbf\BS mbfplus}.) \ 
This command \ttmac{mbf} is realized by the \verb+mbf.sty+ file. 
So you have to include it in the optional argument of the 
\ttmac{documentstyle} in the first line:
{\vskip \asp
\begin{verbatim}
        \documentstyle[seceq,mbf]{ptptex}
\end{verbatim}\vskip\bsp}
\noindent 
By this you can write equations like
\begin{equation}
\phi({\mbf x}) = \sum_{\mbf k}\left[ {\mbf a_k} e^{i{\mbf kx}} + 
      {\mbf a_k}^\dagger e^{-i{\mbf kx}}\right]
\label{eq:3.1}
\end{equation}

This command also has a trouble in fact. 
%First, {\it \ttmac{mbf} does
%not work in case of \LaTeX$2\epsilon$}. (An method to write math-bold 
%letters in the \LaTeX$2\epsilon$ case will be explained in the final 
%section.) \ Secondly, 
If you are working on a 
machine which does not possess many fonts, you will sometimes be 
warned as \ {\tt No Font: Any of the above.} and cannot do preview 
nor print. In such a case, write 
%\boxmac{subfontMBF}
{\vskip \asp
\begin{verbatim}
        \subfontMBF
\end{verbatim}\vskip\bsp}
\noindent
in the preamble. (This command is already written with comment-out 
symbol {\tt \%} in the template.tex.)  This makes the preview and print 
possible by substituting the fonts by available ones temporarily. 
When you submit your \LaTeX\ file to PTP or do preview and print on 
machines possessing enough fonts, therefore, please comment-out 
that line with {\tt \%}.

\subsection{Fractions}

By the usual \verb+\frac{numerator}{denominator}+ command, the 
numerator and the denominator are printed in a smaller font than usual.
If you want the output in the usual size, you can use 
\boxmac{dfrac} command. The usage is quite the same as \ttmac{frac},
and you can write complex fractions such as 
\[
   a_0 + \dfrac{b_1}
  {a_1 + \dfrac{b_2}%{a_2}
  {a_2 + \dfrac{b_3}{a_3}}} \ .
\]

\subsection{Subeqnarray and subequations}

When you want to number the equations in the form like (3$\cdot$2a), 
(3$\cdot$2b), $\cdots$ in a set of arrayed equations, you can use 
\boxenv{subeqnarray} environment. The usage is quite the same as the
usual \verb+eqnarray+ environment: you can simply write 
\verb+\begin{subeqnarray}+ $\cdots$ \verb+\end{subeqnarray}+ in place
of \verb+\begin{eqnarray}+ $\cdots$ \verb+\end{eqnarray}+.

If you want to do the same thing for some equations apart from 
one another, 
you can use the \boxenv{subequations} environments.  The usage will be 
clear from the following simple example:
\vskip .5em
\begin{center}
\begin{minipage}[t]{6.5cm}
\centerline{\bf Input}
\vskip .4em
\baselineskip 1.1em 
\begin{verbatim}
\begin{subequations}
  \label{eq:1}
  An example of subequations:
  \begin{equation}
    \alpha + 2\beta + \gamma = 2
    \label{eq:1a}
  \end{equation}
  Here come some sentences,
  which can be of any length.
  \begin{eqnarray}
    \gamma &=& \nu (2-\eta) \\
    \delta &=& \mu (1+\rho)
  \end{eqnarray}
\end{subequations}
\end{verbatim}
\end{minipage}
\begin{minipage}[t]{.8cm}
~\vspace{5.3\baselineskip}
\begin{tabular}{c}
$\Longrightarrow$ \\
gives
\end{tabular}
\end{minipage}
\hspace{.3cm}
\begin{minipage}[t]{.4\textwidth}
\centerline{\bf Output}
\vskip 22pt
\fbox{
\begin{minipage}[t]{.95\textwidth}
\begin{subequations}
  \label{eq:1}
  An example of subequations:
  \begin{equation}
    \alpha + 2\beta + \gamma = 2
    \label{eq:1a}
  \end{equation}
  Here come some sentences, which can be of any length.
  \begin{eqnarray}
    \gamma &=& \nu (2-\eta) \hspace{4em} \\
    \delta &=& \mu (1+\rho)
  \end{eqnarray}
\end{subequations}
%\vskip 0em%\baselineskip
\end{minipage}}
\end{minipage}
\end{center}
\vskip 1em
\noindent
Here note that the \verb+\label{eq:1}+ just after the 
\verb+\begin{subequations}+ defines the group equation name 
so that \verb+(\ref{eq:1})+ gives 
``(\ref{eq:1})", while \verb+\label{eq:1a}+ refers to the first 
equation (\ref{eq:1a}). On the other hand, in the previous 
\verb+subeqnarray+ environment, the \verb+\label+ command refers only 
to the group equation name wherever it is placed inside the 
\verb+subeqnarray+ environment. In order to refer to the individual 
equation there, a command \boxmac{slabel} is available and is used 
in quite the same way as the usual \verb+\label+ command.

\section{References}

References are cited using \boxmac{cite} command.  All references 
should be listed continually within curly brackets using commas
such as \verb+\cite{rf:1,rf:3,rf:4,rf:5}+. 
(Don't put any space after the commas here. Otherwise \LaTeX\ will 
regard the reference label as containing the space also.)
When three or more consecutive reference numbers appear as in this 
example, the numbers are automatically printed in a compressed 
way like~\cite{rf:1,rf:3,rf:4,rf:5}. For cases of very long 
consecutive references, it would be very tedious to type this way. 
In such cases you can type as 
\verb+\cite{rf:3}\tocite{rf:5}+; namely, you write the first reference 
using \ttmac{cite} and succeedingly cite the last reference using 
\boxmac{tocite} command. 
Then you obtain an output like this.\cite{rf:3}\tocite{rf:5} \ 
\boxmac{citen} command can be used to obtain citation numbers: 
{\it e.g.,} 
\verb+Ref.~\citen{rf:3}+ gives an output ``Ref.~\citen{rf:3}". 

\def\spacesymb{\raisebox{-1pt}[0pt][0pt]{$\sqcup$}}
When you use \ttmac{cite} command at the end of sentences, 
put always the \ttmac{cite} command {\it after period, comma, colon or 
semicolon}, like \verb+something.\cite{rf:5}+. 
If you continue the next sentence, type 
 \spacesymb{\tt \BS}\spacesymb\ (\spacesymb\ is [space] key) to give a 
suitable space between sentences. For instance, type 
{\vskip \asp \baselineskip 1.1em
\begin{verbatim}
   This is $\cdots$ something.\cite{rf:5} \ Therefore we can $\cdots$ 
\end{verbatim}\vskip\bsp}\noindent
then the output becomes 

This is $\cdots$ something.\cite{rf:5} \  Therefore we can $\cdots$ 

\noindent

The references are listed at the end of the file by using the 
\boxmac{thebibliography} environment and \boxmac{bibitem} command 
as usual. For example the references at the end of this manual are 
typed as follows:
{\vskip \asp \baselineskip 1.1em
\begin{verbatim}
 \begin{thebibliography}{99}
  \bibitem{rf:1} 
     Leslie Lamport, {\it LaTeX{\rm :} A Document Preparation System} 
     (Addison-Wesley, New York, 1986).
  \bibitem{rf:2} 
     S.~Weinberg, Phys.\ Rev.\ Lett.\ {\bf 19} (1967), 1264.
  \bibitem{rf:3} 
     M.~Kobayashi and T.~Maskawa, \JL{Prog.\ Theor.\ Phys.,49,1973,652}.
  \bibitem{rf:4} 
     D.~Gross and F.~Wilczek, \PRL{30,1973,1343}.
    \\
     H.~D.~Politzer, \PRL{30,1973,1346}.
  \bibitem{rf:5} 
     G.~'t~Hooft, \NP{B33,1971,173}; \andvol{B35,1971,167}.
 \end{thebibliography}
\end{verbatim}\vskip\bsp}
\noindent
Please write your bibitems properly according to these examples 
in the PTP format. In particular, the volume numbers should be in 
boldface, the (year) should be followed by a comma `{\tt ,}' and 
the line should end with a period `{\tt .}'.  Even when you cite 
two or more references of different authors in an item 
as in {\tt rf:4}, end with a {\it period} and continue with a 
line-feed by {\tt \BS\BS}.
When you cite two or more references with the same author as in 
{\tt rf:5}, continue with a semicolon `{\tt ;}' without line-feed.
To make it easy to write the bibitem following this PTP format, 
we prepared macros \boxmac{JL}\ , \boxmac{andvol}\ , \boxmac{PRL}\ , 
$\cdots$, as used in {\tt rf:3},\ {\tt rf:4} and {\tt rf:5}: 
{\vskip \asp
\begin{verbatim}
   o for general use
      \JL : general journals          \andvol : Vol (Year), Page
   o for individual journal 
      \PR  : Phys. Rev.               \PRL : Phys. Rev. Lett.
      \NP  : Nucl. Phys.              \PL  : Phys. Lett.
      \JMP : J. Math. Phys.           \CMP : Commun. Math. Phys.
      \PTP : Prog. Theor. Phys.       \JPSJ: J. Phys. Soc. Jpn.
      \JP  : J. of Phys.              \NC  : Nouvo Cim.
      \IJMP: Int. J. Mod. Phys.       \ANN : Ann. of Phys.
\end{verbatim}\vskip\bsp}
\noindent
The usage is as follows:
\vskip .1em
\begin{tabbing}
 A \= \verb+\JL{Phys.\ Lett.,A30,1981,56}+ \= $\Rightarrow$ \=
  \verb+Phys.\ Lett.\ {\bf A30} (1981), 56+ \kill
\> \verb+\PR{D45,1990,345}+ \> $\Rightarrow$ \>
  \verb+Phys.\ Rev.\ {\bf D45} (1990), 345+ \\
\> \verb+\JL{Phys.\ Lett.,A30,1981,56}+ \> $\Rightarrow$ \>
  \verb+Phys.\ Lett.\ {\bf A30} (1981), 56+ \\
\> \verb+\andvol{B123,1995,1020}+ \> $\Rightarrow$ \>
  \verb+{\bf B123} (1995), 1020+
\end{tabbing}
\vskip .1em
Here also you should not put any space after commas separating 
the volume, year and page numbers in the argument.

\section{Figures}

\subsection{Post-Script (ps/eps) figure files}

It is best fitted to compuscripts to prepare figures in the 
form of epsf (Encapsulated PostScript Files). 
To put the epsf figures into the text, you should add {\tt epsf.sty} 
as an input file by writing in the first line like 
{\vskip \asp
\begin{verbatim}
        \documentstyle[seceq,epsf]{ptptex}
\end{verbatim}\vskip\bsp}\noindent 
then write for instance as follows at the place 
where you want to put the figure:
{\vskip \asp
\begin{verbatim}
        \begin{figure}
            \epsfxsize = WIDTH cm   %or \epsfysize = HEIGHT cm
            \centerline{\epsfbox{FILENAME.eps}}
        \caption{Explanation of the figure.}
        \label{fig:1}
        \end{figure}
\end{verbatim}\vskip\bsp}
\noindent

Here {\tt \BS epsfxsize = WIDTH cm} \ specifies the horizontal size 
and the vertical size is automatically calculated by the proportion 
of height to width of the original epsf figure. (You can conversely 
specify the height by writing {\tt  \BS epsfysize = HEIGHT cm}, then 
the width is calculated.) \ 
{\tt \BS epsfbox\{FILENAME.eps\}} in the next line is the command 
to place the figure with file name {\tt FILENAME.eps}. 
There seems to exist a different style file bearing the same name 
{\tt epsf.sty}. 
If you cannot get the desired result with these commands, the 
{\tt epsf.sty} file may be such a different one. 
Then try to input the attached (world-standard) {\tt epsf.sty} file.

Instead, if you prepare conventional figures drawn by ink,
then send the original ones to PTP editorial office by conventional 
mail separately. In that case, 
to allocate a space for the figure in the \LaTeX\ source file, 
the width and height of the figure should be given by 
\boxmac{figurebox} command.  Here is an example:
{\vskip\asp
\begin{verbatim}
        \begin{figure}
        \figurebox{WIDTH}{HEIGHT}
        \caption{Here is the caption}
        \label{fig:1}
        \end{figure}
\end{verbatim}\vskip\bsp}
\noindent
If you want to delete the frame rule of this {\tt figurebox}, 
use the \boxmac{nofigureboxrule} command in the preamble written in 
{\tt template.tex} by eliminating {\tt \%} in front.


\subsection{Wrapfigure environment}

\begin{wrapfigure}{r}{6.6cm}
  \figurebox{60mm}{3cm}
\caption{A figure given by {\tt wrapfigure} environment.}
\label{fig:2}
\end{wrapfigure}
Narrow figures should be put in the half text-width (7cm). 
\boxenv{wrapfigure} environment is available for this purpose. 
We give an example of the \verb+wrapfigure+ in Fig.~\ref{fig:2}. 
Beware that this \verb+wrapfigure+ environment is {\it not a floating} 
one but gives the figure exactly at the place specified 
(irrespectively of whether the space exists enough or not). 
For instance, 
the Fig.~\ref{fig:2} on the right hand side is obtained by putting the 
following commands on the top of this paragraph as shown shortly below.
Be careful, therefore, that this environment should not be used 
at the edge of section or subsection or page, since it does not work 
properly at such boundaries. Because of this ``non-floating'' nature, 
it is better to put 
those \verb+wrapfigure+'s at the final stage where no more additions
will be made to the text. 
{\vskip\asp
\begin{verbatim}
        \begin{wrapfigure}{r}{6.6cm}   % r: RIGHT, 6.6cm: WIDTH  
          \figurebox{60mm}{3cm}
          \caption{A figure given by {\tt wrapfigure} environment.}
          \label{fig:2}
        \end{wrapfigure}
        Narrow figures should be put in the half text-width (7cm).... 
\end{verbatim}\vskip\bsp}

To use this environment, you have to add {\tt wrapft.sty} as an input
file by writing in the first line like 
{\vskip \asp
\begin{verbatim}
        \documentstyle[seceq,epsf,wrapft]{ptptex}
\end{verbatim}\vskip\bsp}\noindent
The basic format of the \verb+wrapfigure+ environment is:
{\vskip \asp
\begin{verbatim}
      \begin{wrapfigure}[number]{position}{width}
          <figure> etc.                          
          \caption{ <caption> }                  
      \end{wrapfigure}                           
\end{verbatim}\vskip\bsp}
\noindent
Here the optional argument {\tt [number]} is to specify the number of
the narrow text lines wrapping around the figure. Since it is
automatically calculate well, skip it usually. Write it explicitly
only when you want to specify the number by force. (For instance, when
the figure is placed at the bottom of a page, then the text lines at
the beginning of the next page are sometimes made narrow. You can
avoid such a thing by explicitly specifying the {\tt [number]}.) \ The
argument {\tt \{width\}} specifies the width of the space for the
figure (and of the figure caption). PTP admits only the size of half
textwidth (7cm) from aesthetic viewpoint. So write {\bf always
\{6.6cm\}} or {\tt \{\BS halftext\}} as this width by taking a space
between the text and figure). \boxmac{halftext} is defined to be 0.471
times {\tt \BS textwidth} and becomes 6.6cm in the case of PTP text
style. The argument {\tt \{position\}} is {\tt \{r\}} or {\tt \{l\}} 
when you put the
figure on the right or left hand side. You can write {\tt \{c\}}
(center) there when you dislike the floating nature of the usual
\verb+figure+ environment: if you write
\verb+\begin{wrapfigure}{c}{width}+ with putting \verb+{c}+, then it
works just like the \verb+figure+ environment except that it gives the
figure exactly at that place. The {\tt \{width\}} specifies the
caption width in this case. In the {\tt <figure>} part can
come epsf figures like in the above example, {\tt picture} environment
or anything else.

When you want to put a figure at the beginning of a paragraph 
like in the above example of Fig.~\ref{fig:2}, place 
the commands of \verb+wrapfigure+ environment before the 
first sentence of the paragraph. If you want to put it in the middle of 
a paragraph, it can be done rather manually; First comment out 
the \verb+wrapfigure+ environment part from the file and compile it.
Then, in the preview screen, find and remember the word 
which comes at the end of the line below which you want to put the 
figure. Going back to the file, place the \verb+wrapfigure+ environment 
just after that word. 
A more detailed usage can be found in the beginning of the 
{\tt wrapft.sty} file. 

If you do not like any such complications, then put all the figures
simply by using the usual \verb+figure+ environment alone. 
The editorial office will remake them in a most appropriate form.

\subsection{Putting figures side by side}

Either in {\tt figure} or {\tt wrapfigure} environment, 
it is easy to put two or more figures in a column (vertically). 
Repeat simply the {\tt <figure>} part in the environment. 
Repeat also the {\tt \BS caption} part if you want 
caption for each figure.

It is not so trivial to put two figures side by side. There would be 
many ways to do this. We here cite an example using 
{\tt \BS parbox} command:
\begin{verbatim}
        \begin{figure}[htb]
            \parbox{\halftext}{%   %\def\halftext{.471\textwidth}
                \figurebox{6cm}{2cm}
                \caption{The first figure on the left.}}
            \hspace{8mm}
            \parbox{\halftext}{
                \figurebox{6cm}{2cm}
                \caption{The second figure on the right.}}
        \end{figure}
\end{verbatim}
This gives the following output:
\begin{figure}[htb]
 \parbox{\halftext}{\figurebox{6cm}{2cm}
                \caption{The first figure on the left.}}
 \hspace{8mm}
 \parbox{\halftext}{\figurebox{6cm}{2cm}
                \caption{The second figure on the right.}}
\end{figure}


\section{Tables}

\begin{wraptable}{l}{\halftext}
\caption{An example of small table given by 
        {\tt \BS begin\{wraptable\}\{l\}\{\BS halftext\}}.}
\label{table:1}
\begin{center}
\begin{tabular}{ccc} \hline \hline
temperature & energy  & specific heat \\ \hline
0.1     & 0.24  & 2.46\\
0.2     & 0.80  & 4.62\\
0.3     & 1.11  & 3.27\\ \hline
\end{tabular}
\end{center}
\end{wraptable}
The standard \LaTeX\ \verb+table+ environment can of course be used.
We here give only an example of small table  
using \boxenv{wraptable} environment in Table~\ref{table:1}. 
This environment is also supported by the {\tt wrapft.sty} file
explained above for figures and the usage is quite the same as
\verb+wrapfigure+ environment. The format for Tables in PTP is, as in
this example, to put {\bf caption above the table} and to draw {\bf
double line only for the top line} of the table using {\tt \BS hline\
\BS hline}. Please follow this way of writing tables. For reference we
cite the source for the Table~\ref{table:1}: 
{\vskip.4em\baselineskip 1.1em
\begin{verbatim}
      \begin{wraptable}{l}{\halftext}
        \caption{An example of small table given by 
                 {\tt \BS begin\{wraptable\}\{l\}\{\BS halftext\}}.}
        \label{table:1}
        \begin{center}
          \begin{tabular}{ccc} \hline \hline
           temperature & energy  & specific heat \\ \hline
               0.1     &  0.24   &  2.46  \\
               0.2     &  0.80   &  4.62  \\
               0.3     &  1.11   &  3.27  \\ \hline
          \end{tabular}
        \end{center}
      \end{wraptable}
      The standard \LaTeX\ \verb+table+ environment can ......
\end{verbatim}\vskip.8em}


The original \verb+tabular+ environment is rewritten in \PTPTeX.sty 
so as to print in the footnotesize fonts suitable for tables. 
If you want to use it in the usual text, therefore, please add 
a line of a sentence %\verb+\let\tabularsize\normalsize+ 
{\vskip \asp
\begin{verbatim}
        \let\tabularsize\normalsize
\end{verbatim}\vskip.15em}
\noindent 
to obtain normalsize
output. But take care that the size change affects only {\it locally}. 

\section{Comments}

An error which sometimes occurs but is difficult to find the cause is
the case that one cites equations by {\tt \BS ref\{..\}} in the
captions of Figure and Tables, for instance. This type of error can 
often be avoided if you put {\tt \BS protect} in front of 
{\tt \BS ref\{..\}}. The {\tt \BS protect} command is also worth
trying when you meet some unknown errors about the argument of 
{\tt \BS section} and {\tt \BS title}.

When you use \PTPTeX, you may find possible bugs in {\tt ptptex.sty}. 
If you find such bugs, please inform us of them to the address
{\vskip \asp
\begin{center}
        {\tt ptp@yukawa.kyoto-u.ac.jp} \ \ or \ \ 
        {\tt kugo@gauge.scphys.kyoto-u.ac.jp} 
\end{center}\vskip\bsp}\noindent
with writing {\tt ptptex.sty bug} at the Subject column of your mail
header. 

\vskip3mm
\noindent
{\it A comment on} \LaTeX2$\epsilon$
\vskip2mm

Our math-bold font command {\tt \BS mbf} is now made to work also 
in \LaTeX2$\epsilon$. 
In \LaTeX2$\epsilon$ case, however, the standard 
command \boxmac{boldsymbol} can also be used to obtain the math-bold 
fonts. The command {\tt \BS boldsymbol} becomes available 
by adding {\tt amsbsy.sty} as an input style file. 
Then, an input like 
\verb+$\boldsymbol{a_k}$+, for instance,  
would give an output ${\mbf a_k}$.

%does not work and yields an error message. \LaTeX2$\epsilon$ adopts 
%NFSS (New Font Selection Scheme), and so you can get an output 
%`This is $\mbf E_C$' by simply typing 
%\verb+{\boldmath This is  $E_C$}+, for instance. 
%However, note that this {\tt \BS boldmath} is a font-changing command 
%which can be used only {\it outside} the math-mode, and it will make 
%the whole equation math-bold face. Therefore it will be useless if you 
%want math-bold font only for a part of the equation. Moreover, you
%will want that the font automatically becomes of suitable size
%depending where you use it. In such a case, you have a suitable 
%command \boxmac{boldsymbol}, which becomes available in 
%\LaTeX$2\epsilon$ by adding {\tt amsbsy.sty} as 
%an input style file. So, write in the first line like
%{\vskip \asp
%\begin{verbatim}
%        \documentstyle[seceq,amsbsy]{ptptex}
%\end{verbatim}\vskip\bsp}
%\noindent 
%Then, writing in the source file, for instance,
%\begin{verbatim}
%       \begin{equation}
%         \phi(\boldsymbol{x}) = \sum_{\boldsymbol{k}}\left[ 
%            \boldsymbol{a_k} e^{i\boldsymbol{kx}} + 
%              \boldsymbol{a_k}^\dagger e^{-i\boldsymbol{kx}} \right]
%       \end{equation}
%\end{verbatim}
%you will get the same output as Eq.~(\ref{eq:3.1}). Note that this
%\ttmac{boldsymbol}, contrary to \ttmac{mbf}, is a command taking an
%argument and the argument is made math-bold face.
 

\section*{Acknowledgements}
Acknowledgements can be written using the LaTeX standard command
 \newline
\verb+\section*{Acknowledgements}+.

The style files and the guides of JJAP and JPSJ were of much help in
making \PTPTeX\ style file and writing this manual. 
We acknowledge the staffs of JJAP and JPSJ making those.
At the final part of our style file {\tt ptptex.sty}, 
we have included free-ware
style files {\tt subeqn.sty} (by Stephen Gildea), 
{\tt subeqna.sty} (by Johannes Braams), 
{\tt cite.sty} (by Donald Arseneau) as they stand. And 
{\tt wrapft.sty} is merely a modified version of 
Tetsuo Iwakuma's extension {\tt wrapfloat.sty} of the original 
{\tt wrapfig.sty} by Donald Arseneau. We express sincere thanks to
those authors.  


\appendix
\section{How to write Appendix}
The appendices can be written by the standard \LaTeX\ commands as
{\vskip.4em\baselineskip 1.1em
\begin{verbatim}
    \appendix
    \section{How to write Appendix}
      ...
    \section{Second Appendix}
      ...
\end{verbatim}\vskip.4em}
\noindent
If you write two or more appendices, the appendix numbers automatically
become {\bf A}, {\bf B}, {\it etc}.  Of course, the equations can be 
used and look like

\begin{equation}
S_q^z=\frac{1}{L}\sum_{j=1}^LS_j^ze^{iqj}.
\end{equation} 


\section{Converting PS file to EPS file}

EPSF (Encapsulated PostScript Files) are best suited for the figures
to be put in the \TeX (\LaTeX) compuscripts. 
If you have only the usual PS (PostScript) files, you can easily
convert them into EPSF as follows.

EPS file is merely a PS file which has an additional information on the
region where the figure is drawn. This specification of the figure
region is done by the command line
``\verb+%%BoundingBox: x1 y1 x2 y2+'', which is usually written within 
ten lines from the top of the EPS file. The four numbers here
represent two coordinates $(x1, y1)$ and $(x2, y2)$ standing
respectively for the bottom left and top right corners of the figure
region. (Even the ordinary PS file has usually this
command line \verb+%%BoundingBox:+. But, then, the coordinates 
$(x1,y1)$ and $(x2,y2)$ represent the region of the whole page. If
such a line is missing, you have to add one to convert it into EPSF.)

To know the correct values for the coordinates $(x1, y1)$ and
$(x2,y2)$, open the PS figure first using some viewer like ghostview 
(or GSview). If you move the cursor on the viewer screen, the
coordinate of the cursor will be shown on the top (left) part of the
viewer every moment. So you can read the desired coordinates $(x1,
y1)$ and $(x2,y2)$ by putting your cursor on the bottom left and top
right corners of the figure region.

Next, open the PS file by your editor software. You will see 
the top parts written like
{\vskip.4em\baselineskip 1.1em
\begin{verbatim}
        %!PS-Adobe-2.0
        %%Creator: gnuplot
        %%DocumentFonts: Helvetica
        %%BoundingBox: 50 50 410 302
        %%EndComments
\end{verbatim}\vskip.4em}
\noindent
Now rewrite the four numbers (50 50 410 302 in this example) in the 
command \verb+%%BoundingBox:+ into the coordinates $(x1,y1)$ and 
$(x2,y2)$ which you have read above. If they are  
$(70, 80)$, $(100,120)$, for instance, you will have to rewrite as
{\vskip.4em\baselineskip 1.1em
\begin{verbatim}
        %!PS-Adobe-2.0 EPSF-2.0
        %%Creator: gnuplot
        %%DocumentFonts: Helvetica
        %%BoundingBox: 70 80 100 120
        %%EndComments
\end{verbatim}\vskip.4em}
\noindent
We have also changed the first line adding \verb+EPSF-2.0+. Although
this change seems irrelevant to the use in the \TeX\ with the style
file \verb+epsf.sty+, it will cause a 
difference when viewing the file by ghostview in showing the whole
page or the figure region only. It will also be better to change the 
file name from  \verb+xxx.ps+ to \verb+xxx.eps+. That is all for the 
change of PS into EPS files.

%\section{Second Appendix}
%The section number of the second appendix will be {\bf B}.  Thus, the 
%equation number will be like this:
%\begin{equation}
%S_q^z=\sum_{l}|\langle {\rm GS}|S_q^z|l;q\rangle|^2
%e^{-\tau[(E_{1}(k+q)-E_{\rm g}]},
%\end{equation} 

\section{ }
If you want to write an appendix without title like this appendix, 
simply write as 
\verb+\section{}+ leaving the argument empty or blank. Even then, the 
appendix section counter is working and the equation look like
\begin{equation}
A=B.
\end{equation}
So, the use of appendix without title is not recommended exept for the
case having a single appendix.

\begin{thebibliography}{99}
%%%%%%%%%%%%%%%%%%%%%%%%%%%%%%%%%%%%%%%%%%%%%%%%%%%%%%%%%%%%%
% Some macros are available for the bibliography:
%   o for general use
%      \JL : general journals          \andvol : Vol (Year) Page
%   o for individual journal 
%      \PR  : Phys. Rev.               \PRL : Phys. Rev. Lett.
%      \NP  : Nucl. Phys.              \PL  : Phys. Lett.
%      \JMP : J. Math. Phys.           \CMP : Commun. Math. Phys.
%      \PTP : Prog. Theor. Phys.       \JPSJ: J. Phys. Soc. Jpn.
%      \JP  : J. of Phys.              \NC  : Nouvo Cim.
%      \IJMP: Int. J. Mod. Phys.       \ANN : Ann. of Phys.
% Usage:
%   \PR{D45,1990,345}            ==> Phys.~Rev.\ {\bf D45} (1990), 345
%   \JL{Phys.~Lett.,A30,1981,56} ==> Phys.~Lett.\ {\bf A30} (1981), 56
%   \andvol{B123,1995,1020}      ==> {\bf B123} (1995), 1020
%%%%%%%%%%%%%%%%%%%%%%%%%%%%%%%%%%%%%%%%%%%%%%%%%%%%%%%%%%%%%
\bibitem{rf:1} 
    Leslie Lamport, {\it LaTeX{\rm :} A Document Preparation System} 
    (Addison-Wesley, New York, 1986).
\bibitem{rf:2} 
    S.~Weinberg, Phys.\ Rev.\ Lett.\ {\bf 19} (1967), 1264.
\bibitem{rf:3} 
    M.~Kobayashi and T.~Maskawa, \JL{Prog.\ Theor.\ Phys.,49,1973,652}.
\bibitem{rf:4} 
    D.~Gross and F.~Wilczek, \PRL{30,1973,1343}.
  \\
    H.~D.~Politzer, \PRL{30,1973,1346}.
\bibitem{rf:5} 
    G.~'t~Hooft, \NP{B33,1971,173}; \andvol{B35,1971,167}.
\end{thebibliography}

\end{document}


% The CRCKAPB.STY should be in your LaTeX directory.

% Begin your text file with:

\documentstyle[editedvolume,numreferences,psfig,epsf]{crckapb} 

% Alternatives:
%    \documentstyle[proceedings]{crckapb} 
%    \documentstyle[monograph]{crckapb} 
%    \documentstyle[nato]{crckapb} 

\newcommand{\stt}{\small\tt}

% This document needs the CRCKAPB.STY file to create a 
% document with font size 12pts. 
% The title, subtitle, author's name(s) and institute(s) 
% are handled by the `opening' environment.

\begin{opening}
\title{Effective interactions for finite nuclei}

% You can split the title and subtitle by putting 
% two backslashes at the appropriate place. 

\author{M.~Hjorth-Jensen}
\institute{Department of Physics, University of Oslo, Oslo, Norway}
% If there are more authors at one institute, you should first
% use \author{...} for each author followed by \institute{...}.

\end{opening}

\runningtitle{Effective interactions for finite nuclei}

\begin{document}

% The \begin{document} command comes after the \end{opening}
% command.

\section{Introduction}
Large-scale shell-model calculations employing realistic effective 
interactions obtained from many-body theory have been rather successful 
in describing nuclear structure data both near and somewhat away from 
closed shells. In recent years such model calculations have been put 
to crucial test particularly in the Sn region, where experimental 
advances have provided us with a wealth of new data for nuclei 
with unusual proton and neutron numbers. In  fact, level spectra 
are now available for essentially all the Sn isotopes from 
nucleon number $A = 100$ to $A = 132$, obtained by filling valence 
neutrons into the $N = 4$ major shell. This range of nuclei thus provides a 
testing ground for shell-model calculations in which one may study 
the transition from single-particle degrees of freedom to 
more collective ones away from closed shells, as well 
as the origin of possible underlying symmetries such as 
generalized seniority. Similarly, the N = 82 isotones beyond $^{132}$Sn 
provide us with the analogous valence proton spectra. 

Furthermore, 
doubly-magic nuclei and their nearby neighbors are of great experimental
and theoretical interest. Being the heaviest particle-stable
self-conjugate nucleus,
$^{100}$Sn occupies a unique place among them.
An important property of $^{100}$Sn is the degree of rigidity
of its spherical equilibrium shape. This is directly related to
the excitation energy and transition rates of the lowest 2$^+$
state. The main wave function component of this state in a
microscopic description is an isoscalar mixture of proton and
neutron excitations 2d$_{5/2}$1g$_{9/2}^{-1}$ across the $N=Z=50$
shell closures. This state is not experimentally known and
intense radioactive ion beams will most likely be needed for
its identification.

The outline of this paper is as follows. In Sect. 2 we briefly 
review our theoretical framework, including the shell model 
and the effective interaction. In Sect. 3 we present selected 
results for  nuclei around  $^{100}$Sn while Sect. 4 deals 
with selected results for Sb isotopes.
A brief summary is presented in Sect. 5.

\section{Theoretical framework}
\subsection{The nuclear shell model}
We shall take the nuclear shell model as the microscopic 
model of nuclear structure. The shell model describes the 
properties of nuclei in terms effective interactions 
among constituent nucleons filling the valence shell  
beyond the a closed-shell core. In the present case of the Sn 
isotopes, we assume that the valence particles are filling 
the $N = 4$ major shell consisting of the single-particle orbitals 
$2s_{1/2}$, $1d_{5/2}$, $1 d_{3/2}$, $0g_{7/2}$ along with the intruding 
orbital $0h_{11/2}$ from the $N = 5$ shell. These orbitals may accomodate 
as many as 32 valence nucleons, in which case the next closed-shell 
nucleus $^{132}$Sn is obtained. The maximum dimensionality of the 
shell-model eigenvalue problem is obtained at half filling of 
the $N = 4$ shell, namely for 16 valence nucleons giving rise to $^{116}$Sn.
Another alternative is to use 
$^{88}$Sr as closed shell core.
With this core, valence neutrons are again in the
single-particle orbits $2s_{1/2}$, $1d_{5/2}$, $1d_{3/2}$,
$0g_{7/2}$ and $0h_{11/2}$ while  valence protons are in the single-particle
orbits $0g_{9/2}$ and $1p_{1/2}$.
To represent a given shell-model state, we
work in the $m$-scheme basis.
This allows us to represent 
each basis Slater determinant with a 32-bits computer word, in which 
an occupied stated is represented by 1 and an empty state by 0. 
Thus, operating on these states with fermion creation and 
annihilation operators appearing in the nuclear Hamiltonian amount 
simply to moving 1's and 0's around. Starting from the original basis 
vectors, we construct, as described elsewhere, new basis vectors 
using the Lanzocs algorithm. This procedure is repeated until the 
Hamiltonian matrix is obtained on tridiagonal form and can be 
diagonalized by conventional methods. Various precautions may be 
taken to speed up convergence when the dimensionality gets big. 
Although the basis vectors are constructed in $m$-schene, 
the final eigenvectors will have good angular momentum.

\subsection{The effective interaction}
Although the shell-model space employed is large, it is 
nevertheless severely truncated and it is necessary to use an 
effective interaction to account for the degrees of freedom 
which are ignored. It is our aim to calculate this as rigourously 
as possible using many-body perturbation theory and then use it 
without further adjustment in the shell model. Acknowledging 
that it is not yet possible to obtain a realistic 
nucleon-nucleon interaction directly from QCD, we have chosen 
to start from modern one-boson-exchange potentials such as the 
Bonn potentials, see Ref. \cite{cdbonn}. These potentials are too large at small distances 
to be applied directly to perturbative calculations. Thus, they have 
to be regularized at short distance by integrating out the 
high-momentum states to yield a well-behaved Bethe-Brueckner-Goldstone 
reaction matrix denoted as $G$-matrix. Then, the long-range 
correlations may be taken into account using perturbative methods. 
This is done by first evaluating the non-folded diagrams 
(Q-box interaction) to third order and then by folding these 
onto themselves to arbitrary order. For further details we 
refer to extensive reviews on the topic, see e.g.,  Ref. \cite{hko95}
and references therein.

\section{Application to Sn isotopes}
The effective interaction obtained from many-body theory 
contains one-body, two-body, three-body etc. terms, up to as 
many valence particles as one considers. It is customary to add 
the one-body term to the unperturbed Hamiltonian and take the 
eigenvalues of this effective one-body Hamiltonian from the 
experimental single-particle energies. Further, one ignores the 
third- and higher-body terms, so that the only piece to calculate 
is the two-body effective interaction (which, of course, is hard enough). 
In order to start from a $^{100}$Sn core, one needs the 
$^{101}$Sn single-particle 
energies which unfortunately are not known. Thus, one has had to rely 
on a combination of mean-field calculations and on relevant data 
from other nuclei in the vicinity. An alternative is to start from $^{132}$Sn 
as a core and add holes in the $N = 4$ valence shell. This has the 
advantage that the single-hole energies are better, although not 
entirely known. In this case, one has to use an effective 
interaction which is calculated with respect to the $^{132}$Sn core. 

However, if we wish to study nuclei like e.g., $^{99}$Cd or 
$^{101}$In 
a better closed shell core is probably $^{88}$Sr.  

The calculation denoted with SMH in Fig.~\ref{fig:cd99}
uses $^{88}$Sr as closed shell core with an effective
interaction based on the CD-Bonn nucleon-nucleon interaction \cite{cdbonn},
using the model space discussed in the previous section.
The single-particle energies
for the $^{88}$Sr core were taken from Ref.~\cite{anne}.
In the above model space $^{99}$Cd has 1 valence neutron and 10
valence protons.
The calculation favors a J$^{\pi}$=5/2$^+$ assignment for
the ground state, in
agreement with the systematics of odd-A N=51 isotones.
The wave functions of the
states with J$^{\pi}$=5/2$^+$,9/2$^+$,13/2$^+$,17/2$^+$,19/2$^+_2$
and 21/2$^+_1$ all have main configurations with the valence
neutron in the d$_{5/2}$ orbit, while for the
7/2$^+$,11/2$^+$,15/2$^+$,19/2$^+_1$,21/2$^+_2$
and 23/2$^+$ states the neutron occupies mainly the g$_{7/2}$ orbit.
The two proton holes always remain in the g$_{9/2}$ orbit.
The single-particle energy of the $\nu$g$_{7/2}$ orbit relative
to the $^{88}$Sr core was raised from 2.63 to 2.9 MeV in order to
reproduce the
experimental excitation energy of the 7/2$^+$ state. This
yielded excellent agreement between calculation and
experiment up to the J$^{\pi}$=23/2$^+$ state, which is the level
wuth highest spin and positive parity that can be obtained
with 1 neutron and 10 protons outside the $^{88}$Sr core
(or 1 neutron and 2 proton holes in the $^{100}$Sn core) and
the model space used in the calculation. Another shell model calculation
was therefore performed to study higher spin states.
\begin{figure}
   \setlength{\unitlength}{1mm}
   \begin{picture}(100,100)
   \put(0,0){\epsfxsize 14cm \epsfbox{cd99.ps}}
   \end{picture}
\caption{Experimental (EXP) and calculated (SMH, SMG) level
schemes of $^{99}$Cd. The widths of the arrows are proportional
to the intensity of a $\gamma$ ray observed in the experiment.\label{fig:cd99}}
\end{figure}
The
calculation denoted with SMG in Fig.~\ref{fig:cd99} used $^{78}$Sr as closed
shell core. The same model space as in the above calculation was
used except for the $\nu$h$_{11/2}$ orbit was replaced by the
$\nu$g$_{9/2}$ orbit. 
Single-particle energies with respect to the $^{78}$Sr core are
not known and were kept the same as in Ref.~\cite{anne} for the
$^{88}$Sr core. The energy of the g$_{9/2}$ orbit was put at
5.0 MeV below the d$_{5/2}$ orbit. The excitation energies of the
states with J$^\pi$$\geq$25/5$^+$ are very sensitive to the position
of the g$_{9/2}$ orbit. By fitting this single particle energy we,
therefore, indirectly deduced the size of the $N=50$ shell gap.
The wave functions of the states below the 25/2$^+$ level are very
similar to the wave functions of the SMH calculation described
earlier. For these states the $\nu$g$_{9/2}$ orbit is completely filled.
The 25/2$^+$ level and the higher lying states have 9 neutrons in the
g$_{9/2}$ orbit while the remaining neutron pair is almost
evenly distributed over the d$_{5/2}$ and g$_{7/2}$ orbits.
Calculations with the same model spaces as for $^{99}$Cd were
also done for $^{101}$In with a similar agreement, see Ref.~\cite{matej}
for further references.
The interaction from the SMH and SMG calculations
was used in shell model calculations of lowest lying
levels in $^{101}$Sn.
Both SMH and SMG calculations favor J$^{\pi}$=5/2$^+$ for the gound state,
while the 7/2$^+$ state lies only $\sim$100 keV above. This is in excellent
agreement with the extrapolation from odd-A Sn isotopes down to
$^{103}$Sn.

\section{Shell model studies of the proton drip line nuclei 
$^{105,106,107}$Sb}


We present recent results for $^{105}$Sb, $^{106}$Sb and $^{107}$Sb in
Table \ref{tab:tab1}. The calculations use $^{100}$Sn as
closed shell core with an effective 
interaction for the four, five or six valence neutrons and one valence proton
based on the CD-Bonn nucleon-nucleon interaction \cite{cdbonn}.

The high spin level scheme of $^{105}$Sb resembles the level scheme of
$^{107}$Sb up to $J=19/2$ \cite{sb107}. 
When compared to $^{107}$Sb the $^{105}$Sb
level scheme shows similar trends as when going from $^{106}$Sn to
$^{104}$Sn. That means that coupling a $d_{5/2}$ proton to a 
$^{104}$Sn core is appropriate to describe the observed states.
The calculation favors $J^{\pi}$=5/2$^+$ for the ground state in
agreement with the suggestion from proton decay data.
In this state the valence proton is mainly in the $d_{5/2}$
orbit and the two neutron pairs are almost evenly distributed
over the $d_{5/2}$ and $g_{7/2}$ neutron orbits. The situation
is very similar in the 9/2$^+$ and 13/2$^+$ states, while 
the $\nu$$g_{5/2}^3$$g_{7/2}^1$ configuration exhausts the
largest parts of the wave functions of the 15/2$^+$ and 17/2$^+$ 
states. The neutron part of the wave function of the 19/2$^+$
state is almost identical to the 17/2$^+$ state. However,
since 17/2$^+$ is the maximum spin for the 
$\pi$$g_{5/2}^1$$\nu$$g_{5/2}^1$$g_{7/2}^1$ configuration,
the odd proton resides almost exclusively in the $g_{7/2}$
orbit in the 19/2$^+$ state. 
For proton degrees of freedom
the $s_{1/2}$, $d_{3/2}$ and $h_{11/2}$ single-particle 
orbits give essentially negligible
contributions to the wave functions and the energies of the excited
states, as expected. For neutrons, although
the single-particle distribution for a given state is also negligible,
these orbits are important for a good describtion of the energy spectrum,
as also demonstrated in large-scale shell-model calculations of
tin isotopes \cite{ehho98}.
Similar picturer applies to $^{106}$Sb and $^{107}$Sb as well, 
see Refs.~\cite{sb106,sb107}.
The wave functions for the various states are to a large extent
dominated by the $g_{7/2}$ and $d_{5/2}$ single-particle orbits
for neutrons ($\nu$)
and the $d_{5/2}$ single-particle
orbit for protons ($\pi$). 
The $\nu g_{7/2}$ and $\nu d_{5/2}$ single-particle orbits represent
in general more than $\sim 90\%$ of the total neutron single-particle
occupancy, while the  $\pi d_{5/2}$ single-particle orbits stands for 
$\sim 80-90\%$ of the proton single-particle occupancy. 
The other single-particle
orbits play  an almost negligible role in the structure of the wave functions.
\begin{table}[hbt]
\begin{center}
\caption{ Low-lying states of $^{105,106,107}$Sb, theory and experiment.
Energies in MeV. \label{tab:tab1}}\footnotesize
\begin{tabular}{ccc|ccc|ccc}
\hline
\multicolumn{3}{c|}{ $^{105}$Sb} & \multicolumn{3}{c|}{ $^{106}$Sb}& \multicolumn{3}{c}{ $^{107}$Sb} \\ 
{$J^{\pi}_i$} & {Exp} & {Theory} & 
{$J^{\pi}_i$} & {Exp} & {Theory} & 
{$J^{\pi}_i$} & {Exp} & {Theory} \\
\hline 
$5/2^{+}$ & 0 & 0 & $2^{+}$ & 0 & 0 & $5/2^{+}$ & 0 & 0 \\
$9/2^{+}$ & 1.22 & 1.22 & $4^{+}$ & 0.10 & 0.25 & $7/2^{+}$ & 0.77 & 0.69 \\
$13/2^{+}$ & 1.84 & 1.94 & $5^{+}$ & 0.32 & 0.54 & $9/2^{+}$ & 1.06 & 1.08  \\
$15/2^{+}$ & 2.21 & 2.10 & $6^{+}$ & 0.44 & 0.66 & $11/2^{+}$ & 1.79 & 1.80 \\
$17/2^{+}$ & 2.50 & 2.41 & $7^{+}$ & 0.89 & 1.34 & $13/2^{+}$ & 1.90 & 1.94  \\
$19/2^{+}$ & 2.99 & 2.94 & $8^{+}$ & 1.53 & 1.78 &  $15/2^{+}$ & 2.24 & 2.38\\
$23/2^{+}$ & 3.73 & 4.09 & $10^{+}$ & 2.26 & 2.57 & $17/2^{+}$ & 2.75 & 2.83  \\\hline
\end{tabular}
\end{center}
\end{table}

\section{Summary and outlook}
The aim of the present work has been to test realistic 
microscopic calculations of the effective interaction 
in large-scale shell-model calculations with many valence nucleons. 
In general we obtain a  nice agreement with the experimental data
for the excited states. The main problem is still however a proper
reproduction of the binding energy, indicating the need of including
realistic three-body forces as well.
 
We are much indebted to Cyrus Baktash, David Dean, Hubert Grawe
and Matej Lipoglav\v{s}ek
for many discussions on properties of nuclei near $A\sim 100$.

\begin{thebibliography}{200}
\bibitem{cdbonn} R.\ Machleidt, F.\ Sammarruca and Y.\ Song,
Phys.\ Rev.\ C 53 (1996).
\bibitem{hko95}  M.\ Hjorth-Jensen, T.\ T.\ S.\ Kuo and
E.\ Osnes, Phys.\ Reports 261 (1995) 125.
\bibitem{anne} A.~Holt, T.~Engeland, M.~Hjorth-Jensen, and E.~Osnes,
                 Phys.~Rev.~C61 (2000) 064318.
\bibitem{ehho98} A.~Holt, T.~Engeland, M.~Hjorth-Jensen, and E.~Osnes,
                 Nucl.~Phys.~A634 (1998) 41.
\bibitem{sb107} D.~R.~LaFosse et al., Phys.~Rev.~C62 (2000) 014305.  
\bibitem{sb106} D.~Sohler {\em et al.}, Phys.~Rev.~C59 (1999) 1324; for a
shell-model analysis see e.g., 
T.~Engeland, M.~Hjorth-Jensen and E.~Osnes, 
Phys.~Rev.~C {\bf 61} (2000) 021302(R).

\bibitem{matej}M.~Lipoglav\v{s}ek {\em et al.}, Phys.\ Rev.\ Lett., submitted.

\end{thebibliography}



\end{document}






  


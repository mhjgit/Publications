\documentclass[a4paper]{jpconf}
\usepackage{graphicx,amsmath}
\newcommand\Heff{H_{\mathrm{eff}}}
\newcommand\heff{h_{\mathrm{eff}}}
\newcommand\Veff{V_{\mathrm{eff}}}
\newcommand\Vlowk{V_{\mathrm{low}k}}
\newcommand\Vsrg{V_{\mathrm{SRG}}}
\def\Nu#1#2#3{{}^{#1}_{#3}\mathrm{#2}}

\def\bra#1{\langle #1 \rvert}
\def\ket#1{\lvert #1 \rangle}
\def\braket#1#2{\langle #1\lvert#2\rangle}
\def\~#1{\tilde{#1}}
\def\1by2#1#2{
  \left(
  \begin{array}{c}
    #1  \\
    #2  \\
  \end{array}
  \right)
}
\def\2by2#1#2#3#4{
  \left(
  \begin{array}{cc}
    #1  & #2  \\
    #3  & #4  \\
  \end{array}
  \right)
}
\def\3j#1#2#3#4#5#6{
  \left(
  \begin{array}{ccc}
    #1  & #2  & #3 \\
    #4  & #5  & #6 \\
  \end{array}
  \right)
}
\def\W#1#2#3#4#5#6{
  \left\{
  \begin{array}{ccc}
    #1  & #2  & #3 \\
    #4  & #5  & #6 \\
  \end{array}
  \right\}
}
\def\9j#1#2#3#4#5#6#7#8#9{
  \left\{
  \begin{array}{ccc}
    #1  & #2  & #3 \\
    #4  & #5  & #6 \\
    #7  & #8  & #9 \\
  \end{array}
  \right\}
}
\def\trb#1#2#3#4#5#6#7#8#9{
  \left[
  \begin{array}{ccc}
    #1  & #2  & #3 \\
    #4  & #5  & #6 \\
    #7  & #8  & #9 \\
  \end{array}
  \right]
}
\newcommand\fmi{\mathrm{fm}^{-1}}
\newcommand\TSO{{}^3S_1}
\newcommand\TDO{{}^3D_1}
\newcommand\TE{{}^3 E}
\newcommand\TO{{}^3 O}
\newcommand\SE{{}^1 E}
\newcommand\SO{{}^1 O}

\begin{document}
\title{Tensor force in effective interaction of nuclear force}

\author{Naofumi Tsunoda, Takaharu Otsuka, Koshiroh Tsukiyama}

\address{Department of physics, the University of Tokyo, 7-3-1 Hongo,
Bunkyo-ku, Tokyo, Japan}

\author{Morten Hjorth-Jensen}
\address{Department of Physics and Center of Mathematics for Applications, University of Oslo, N-0316 Oslo, Norway}
\ead{tsunoda@nt.phys.s.u-tokyo.ac.jp}

\begin{abstract}
The tensor force in effective interaction is discussed.
We show that the tensor force in low-momentum effective interactions
 $\Vlowk$ and effective interactions for shell model studies are almost identical
 to that of the bare nucleon-nucleon interaction.
We use realistic nuclear forces and microscopic theories of effective interactions.
We calculate $\Vlowk$ by the technique of similarity transformation. The
 effective interaction for the nuclear shell model is calculated by folded
 diagram theory.
The monopole contribution from the tensor-force is analyzed by a spin-tensor decomposition.
\end{abstract}

 \section{Introduction}

 Radioactive beam accelerator facilities  make it possible to perform experiments 
that test the limits of nuclear stability by studying exotic nuclei.
The  nuclear shell model, which was proposed by Mayer and Jensen, 
 has been one of the most successful models for describing nuclei.
 However, recent studies of neutron rich nuclei indicate that the so-called nuclear magic
 numbers may only be  valid close to stable nuclei.
 Otsuka et al. showed that the tensor force varies the spin-orbit splitting in exotic
 nuclei, where the  proton number and the neutron number are very different.
 The effective single particle energy of protons change as follows
  \begin{align}
   \label{eq:ESPE}
   \Delta \epsilon_p(j)=\frac{1}{2}
   \left(V_{jj'}^{T=0}+V_{jj'}^{T=1}\right)
   n_n(j'),
  \end{align}
  where $\Delta \epsilon_p(j)$ represents the  change in the  effective single
  particle energy and $n_n(j')$ is the occupation number of neutrons in orbit
  $j'$.
 Here $V_{jj'}$ represents the 
 monopole part of the Hamiltonian which is defined as 
 \begin{eqnarray}
  V_{a,b}^T=\frac{\sum_{J}(2J+1) \bra{ab} V \ket{ab}_{JT}}{\sum_{J} (2J+1)}.
 \end{eqnarray}
 This is nothing but the angular averaged interaction between two orbits
 $j$ and $j'$.
The  tensor force monopole has opposite sign between
  a pair of spin-orbit partners. For example, $V_{j_{>}j'}$ and
  $V_{j_{<}j'}$ have opposite sign, where $j_{>}=l + 1/2$ and $j_{<} = l
  -1/2$.
  This effect breaks the conventional magic numbers believed to be
  universal during the last  50 years \cite{PhysRevLett.95.232502}.
  
  In Ref.~\cite{PhysRevLett.95.232502}, the tensor force in effective interactions for the nuclear shell model
  is taken as the exchange of $\pi+\rho$ mesons for simplicity. This is qualitatively the
  same as the tensor force in the bare realistic nuclear force. The tensor force of realistic interactions is fitted to
  reproduce  phase shifts of NN scattering data and ground state properties of the deuteron.
  However, the effective interaction for the shell model is normally different
  from the bare nuclear force.
  First,the  bare nuclear force has strong coupling between low-momentum and
  high-momentum space generated from short range details of the interaction.
  For example, the Argonne interaction is defined by using a local operator
  form in coordinate space and has a strong short range repulsion.
  Furthermore, the effective interaction for the shell model is defined in
  a restricted configuration space, normally called the model space.
  For that reason, effective interactions for the shell model should include
  the effect of restriction to the model space.
  Therefore, the assumption of extracting the tensor force from an effective
  interaction for the shell model is far from trivial.
  To understand the  validity of this assumption, theoretical studies based on
  realistic nuclear forces and the microscopic
  theory of effective interactions are needed.
  
  In this work, we construct an effective interaction for the shell model
  starting from realistic nuclear forces and using microscopic theory of
  effective interaction. Then, we analyze the tensor component to see of the 
  monopole term from tensor-force survives under the renormalization.
  
 \section{How to construct an effective interaction from nuclear
 forces~\cite{HjorthJensen1995125}} 
 \label{sec:effective}

  The original eigenvalue problem is
  \begin{align}
   \label{eq:eigen}
   H \ket{\Psi_i}= E_i \ket{\Psi_i}.
  \end{align}
  Here $\ket{\Psi_i}$ and $E_i$ are ith eigenvector and eigenvalue, respectively.
  We divide the Hilbert space into a model space, the so-called $P$-space and an excluded space, the  $Q$-space for short.
  Here, $\hat{P}$ and $\hat{Q}$ are projection operators onto the $P$-space and the $Q$-space,
  respectively, i.e. $\hat{P}^2=\hat{P},\,\hat{Q}^2=\hat{Q},\, \hat{P}+\hat{Q}={\bf 1}$.
  The basic problem is to find an effective interaction $\tilde{H}$ that
  reproduces some of the 
  eigenvalues of the original eigenvalue problem (\ref{eq:eigen})
  \begin{align}
   \label{eq:eigenP}
   \hat{P}\tilde{H}\hat{P} \ket{\phi_i} = E_i \ket{\phi_i}
  \end{align}
   where $\ket{\phi_i}=\hat{P} \ket{\Psi_i}$.
  This can be accomplished by the following similarity transformation,
  \begin{align}
   {\label{eq:transformation}}
   \tilde{H} = e^{- \omega} H  e^{\omega} \,\,\,\,\,\,\,\,\,\,\,\,\,
   \hat{Q}\omega \hat{P} = \omega.
  \end{align}
  Then it follows immediately $\omega^2=\omega^3=0$ and $e^{\omega} = 1+\omega$.
  If $\hat{P}\tilde{H}\hat{P}$ is an effective interaction, the decoupling condition
  $\hat{Q}\tilde{H}\hat{P}=0$ must be fulfilled.
  
  Therefore, the central problem is now reduced to 
  how to obtain $\omega$ which
  fulfills the decoupling condition.
  The physical meaning of $\omega$ can be understood as a mapping from the $P$-space
  wavefunction to the corresponding $Q$-space component.
  In general, there is no unique solution of the decoupling equation and
  $\omega$ depends on the set of original eigenfunctions which are
  selected to reproduce the corresponding eigenvalues. In this sense, the effective
  interaction is said to be state dependent.

  If one chooses a given set $d$ of eigenfunctions,
  the decoupling equation has 
 the  following formal solution
  \begin{align}
   {\label{eq:formal}}
   \omega =\sum_{i=1}^d \hat{Q} \ket{\Psi _i}\bra{\tilde{\phi_i}} \hat{P}\,\,\, ,
  \end{align}
  where $\bra{\tilde{\phi_i}}$ refers to the bi-orthogonal state of
  $\ket{\phi_i}$, which is defined to be
  $\braket{\tilde{\phi_i}}{\phi_j}=\delta_{ij}$. The original eigenfunction
  $\ket{\Psi}$ is orthogonal
  because of the hermiticity of the Hamiltonian. However, its projection onto
  the $P$-space $\ket{\phi}=\hat{P}\ket{\Psi}$ is not orthogonal in general.
  
  If we take the $P$-space as the low-momentum space and the $Q$-space as the high-momentum
  space and choose $d$ low-energy eigenstates,
  we will obtain a low-momentum interaction by the above procedure.
  We call this potential $\Vlowk$ and the cutoff parameter $\Lambda$ is
  defined as the boundary between the $P$-space and the $Q$-space~\cite{Bogner20031}.
  Using $\Vlowk$ can be interpreted as rewriting our problem from
  $H=T-T_{\mathrm{CM}}+\sum V_{\mathrm{bare}}$ to
  $H=T-T_{\mathrm{CM}}+\sum \Vlowk$.
  This means that we change our resolution of the problem and solve the
  problem not seeing high-momentum details. The problem becomes much easier
  to solve. In this work, we will neglect induced many-body force.

  Nevertheless, if we need to solve a many-body problem, it is still
  difficult to solve the Schrodinger equation directly even if we use
  $\Vlowk$.
  For example, if one wants to obtain the energy levels of $\Nu{18}{O}{}$ with 18 interacting particles,
  solving the whole 18-body problem is too large a calculation to perform
  and we need to calculate a two-body effective interaction defined for say a smaller space. A typical example is the $sd$-shell.
  However, to determine of $\omega$ by Eq.(\ref{eq:formal}) one needs
  the complete knowledge of the $d$ true eigenfunctions $\ket{\Psi_i}$,
  which we do not know.
  Hence,
  what we
  want to do is to obtain the eigenvalues of the original problem
  without solving the original problem.
  In this case,
  one cannot know $\omega$ from Eq.(\ref{eq:formal}) and have to
  obtain it by an alternative procedure.
  An efficient way to do this is the so-called $\hat{Q}$-box expansion.
  
  If the unperturbed Hamiltonian is degenerate in the model space
  \footnote{This assumption is natural when we consider the nuclear
  many-body problem, where valence particles are naturally considered to
  be a member of a set of harmonic oscillator eigenfunctions with same
  single particle energy.}
  , that is,$H_0 \ket{\phi_i} = E_0 \ket{\phi_i}$,
  the effective interaction defined in $P$-space can be calculated by the
  following iterative formula
  \if0
  then the decoupling equation (\ref{eq:decoupling}) can be written as
  \begin{align}
   \omega = \frac{1}{E_0 - QHQ} QH_1 P - \frac{1}{E_0 - QHQ} \omega
    (PH_1 P+ PH_1 Q\omega P).
  \end{align}
  Therefore, from Eq.(\ref{eq:effective}) the decoupling equation reads,
  \begin{align}
   {\label{eq:w-interation.orig}}
   \omega = \frac{1}{E_0 - QHQ}QH_1 P - \frac{1}{E_0 - QHQ}Q\omega P
   \Veff .
  \end{align}
  Multiplying with $PH_1$ and adding $PH_1 P$ we obtain
  \begin{align}
   {\label{eq:Q-box}}
   &\Veff = \hat{Q}(E_0) - PH_1 \frac{1}{E_0 -QHQ} \omega \Veff
   \intertext{where $\hat{Q}(E_0)$ is defined as,}
   &\hat{Q}(E_0) \equiv  PH_1 P + PH_1 \frac{1}{E_0 - QHQ}QH_1 P.
  \end{align}
  here $\hat{Q}(E_0)$ is the so-called $Q$-box.\\
  Note that $Q$-box is defined as an operator acting in $P$-space only.
  The first term is simply the $P$-space component of the interaction.
  The second term represents processes that link model space states with states outside the model space.
The effective interaction in the $P$-space should contain all these
  processes to infinite order.
  Thus it should be written as a sum of $Q$-boxex and repetition of it.
  We can obtain such an expression by the following iterative procedure.
  From Eq.(\ref{eq:w-interation.orig}) and Eq.(\ref{eq:Q-box}),
  we can construct the following set of iterative
  equations. This iterative scheme is not unique.
  \begin{align}
   {\label{eq:iteration}}
   &\Veff^{(n)} = \hat{Q}(E_0) - 
    PH_1 \frac{1}{E_0 - QHQ} \omega _n \Veff^{(n-1)}  \notag \\
   &\omega _n = \frac{1}{E_0 - QHQ} QH_1 P 
   - \frac{1}{E_0 -QHQ } \omega _n \Veff ^{(n-1)}
  \end{align}
We obtain finally
  \begin{align}
   \label{eq:formal_iterative_solution}
   \Veff^{(n)} 
   &= 
   \hat{Q}(E_0) - PH_1  \left( \frac{1}{E_0 - QHQ} \right)^2 QH_1 P \Veff ^{(n-1)}
   + PH_1 \left( \frac{1}{E_0 -QHQ}\right)^2 \omega _n
   \{\Veff^{(n-1)}\}^2                          \notag \\
   &=
   \hat{Q}(E_0) + \hat{Q}_1(E_0) \Veff^{(n-1)} + \cdots \notag\\
   &=
   \hat{Q}(E_0) + \sum _m \hat{Q}_m(E_0) \{ \Veff^{(n-1)}\}^{m}
   \intertext{where,}
   &\hat{Q}_m(E_0) = \frac{1}{m!}\frac{d^m \hat{Q}(E_0)}{d E_0^m}
  \end{align}
  \fi
  \begin{align}
   \label{eq:formal_iterative_solution}
   \Veff^{(n)} 
   &= 
   \hat{Q}(E_0) + \sum _m \hat{Q}_m(E_0) \{ \Veff^{(n-1)}\}^{m}
   \intertext{where,}
   &\hat{Q}(E_0) \equiv  PH_1 P + PH_1 \frac{1}{E_0 - QHQ}QH_1 P\\
   &\hat{Q}_m(E_0) = \frac{1}{m!}\frac{d^m \hat{Q}(E_0)}{d E_0^m}.
  \end{align}
  where the derivatives correspond to folded diagrams.
  If we set $n \rightarrow \infty$, this is equivalent to Brandow's formal
  solution~\cite{RevModPhys.39.771}.
  By this prescription, we can obtain effective interaction for the shell
  model, which gives equivalent results to the original eigenvalue problem
  within the range of approximation. Again, induced many-body forces are
  neglected.
  
  \subsection{Tensor force in low-momentum interaction $\Vlowk$}
  As  mentioned above, the tensor force in the 
  bare realistic nuclear force mainly comes from $\pi
  +\rho$ meson exchange.
  In this section, we discuss whether how the  tensor force 
  renormalization of the short range part of the interaction is incorporated in an effective interaction.
  For this purpose, we have performed a spin-tensor decomposition of the  $\Vlowk$  interaction
  with various values of cutoff $\Lambda$.

  \begin{figure}[h]
   \includegraphics[width=14pc]{Vlowk_tensor.eps}\hspace{2pc}%
    \begin{minipage}[b]{14pc}\caption{\label{fig:vlowk_ten}Tensor-force
     monopole of low-momentum interaction $\Vlowk$. Cutoff parameter
     $\Lambda$ of $\Vlowk$
     is taken as from 1.0 $\fmi$ to 5.0 $\fmi$}
    \end{minipage}
  \end{figure}

  Figure \ref{fig:vlowk_ten} shows the tensor-force monopole part of $\Vlowk$
  derived using the  Argonne V8'(AV8') potential.
  Except for very low cutoff values like $\Lambda = 1.0 \fmi$, the tensor-force monopole part of $\Vlowk$ has
  almost no cutoff dependence. A low value $\Lambda = 1.0 \fmi$ in momentum space
  corresponds to a distance in coordinate space of approximately  $1.0 \mathrm{fm}$.
  Since the Compton length of a pion is approximately $0.7 \mathrm{fm}$, such
  a low cutoff $\Lambda = 1.0 \fmi$ can be interpreted as too low. With
  such a cutoff, the renormalization induces
  much higher-body force, which makes the problem much more complicated. 
  Therefore, this value is not appropriate for our purpose.
 
 In conclusion, the tensor force survives in a low-momentum interaction
 $\Vlowk$ with usual cutoff values,
 at least for its  monopole part. 
 
   \subsubsection{Effect of renormalization of short-range tensor force}
   
    In this subsection, we will focus on the effect of the renormalization procedure on the
    tensor force.
    In our calculation of low-momentum interaction $\Vlowk$, we are
    considering a two-nucleon system.
    The tensor force plays a crucial role in the case of the deuteron.
    The deuteron has isospin $T=0$ and a non-negligible ${}^3S_1-{}^3D_1$ mixing. There is no experimental 
data which indicates an existence
    of excited states or any other bound state of two nucleons.
    The mixing of ${}^3S_1-{}^3D_1$ is due to tensor force.
    
    The Schr\"odinger equation for the  deuteron can be written as the following coupled set of equations,
    \begin{align}
     \label{eq:deuteron}
     &-\frac{\hbar^2}{M}\frac{d^2u(r)}{dr^2}+V_Cu(r)+\sqrt{8}V_T w(r)=E_d u(r)
     \notag \\
     &-\frac{\hbar^2}{M}\frac{d^2w(r)}{dr^2}+\left(\frac{6\hbar^2}{Mr^2} +
     V_C - 2V_T -3V_{LS}\right)w(r)+\sqrt{8}V_T u(r)=E_d w(r),
    \end{align}
    where $u(r)$ and $w(r)$ are the radial wave functions of the $S$-wave and the $D$-wave, respectively
    Knowing the solution of Eq.~(\ref{eq:deuteron}) 
    , we integrate out the $D$-wave degrees of freedom
    to the  obtain following effective central force
    \begin{align}
     {\label{eq:Veff_central}}
     \Veff(r;\TSO) = V_C(r;\TSO) + \Delta \Veff(r;\TSO) \notag \\
     \Delta \Veff(r;\TSO) \equiv \sqrt{8}V_T(r)\frac{w(r)}{u(r)}.
    \end{align} Here
    $\Delta \Veff$ is comparable to $V_C$ in strength.
    This effective central force makes the deuteron bound.
    In this sense, the tensor force plays a crucial role in making the  deuteron
    bound.
    This effect is at least a second order effect in terms of the tensor force,
    since both the initial and the final state have
    orbital angular momentum $0$.
    To see how the tensor force is renormalized into a central force, we derive
    $\Vlowk$ starting from the full Argonne V8' potential(AV8' full) and tensor
    subtracted Argonne V8'(AV8' TS) potential.
    \begin{figure}[h]
     \includegraphics[width=14pc]{Av8_with_and_without_tensor.eps}\hspace{2pc}%
     \begin{minipage}[b]{14pc}\caption{\label{fig:ren_ten}Effect of
      renormalization of short-range tensor force.}
     \end{minipage}
    \end{figure}
     In Fig.\ref{fig:ren_ten}, the red line and the blue line
 represent the monopole part of $\Vlowk$ from AV8' full and AV8' TS,
 respectively, and the black line represents the  bare AV8' potential.
 %The difference indicates of these two potentials shows the effect of
 %renormalization of tensor force. 
 The difference between $\Vlowk$ from AV8'full and $\Vlowk$ from AV8'TS
 shows the effect of renormalization of short-range tensor force.
 The effect is strongly attractive in the $T=0$ channel,
 which is consistent with the considerations from
 Eq.(\ref{eq:Veff_central}).

 To understand this, consider the 
 Schr\"odinger written as 
 \begin{align}
  \2by2{PHP}{PHQ}{QHP}{QHQ} \1by2{P\ket{\Psi}}{Q\ket{\Psi}}
  = E \1by2{P\ket{\Psi}}{Q\ket{\Psi}}.
 \end{align}

 By integrating out the $Q$-space degrees of freedom, 
 we obtain
 \begin{align}
  PHP \ket{\Psi}+ PHQ\frac{1}{E-QHQ}QHP\ket{\Psi} = E\ket{\Psi}.
 \end{align}
 From this equation, we can consider an effective interaction defined in
 the $P$-space as
 \begin{align}
  {\label{eq:Feshbach}}
  \Heff = PHP + PHQ\frac{1}{E-QHQ}QHP.
 \end{align}
 
 Eq.(\ref{eq:Feshbach}) is so general that we can choose any definition
 of the $P$-space and the $Q$-space.
 If we choose the $P$-space and the $Q$-space as $\TSO$ and $\TDO$
 respectively, the resulting
 $\Heff$ is exactly $\Veff(r;\TSO)$ of Eq.(\ref{eq:Veff_central}),
 that is the effective
 central force in the $\TSO$ channel.
 On the other hand, if we choose $P$-space and $Q$-space as
 low-momentum and high-momentum space, we obtain the low-momentum
 interaction $\Vlowk$.
  


  \subsection{Tensor force in effective interaction for the shell model}
 
 In this section we discuss the tensor force derived from the 
 effective interaction for the shell model.
 We have calculated shell model effective interactions for the $sd$ shell
 and the $pf$ shell, by considering the Q-box to 2nd and 3rd order, with folded
 diagram correction as explained in
 Section {\ref{sec:effective}},
 starting from $\Vlowk$ obtained in the previous section.
 
 The Q-box is calculated by considering valence-linked and connected
 diagrams with unperturbed single particle energy of the harmonic oscillator.
 Folded diagrams are obtained by numerical derivatives of the Q-box with
 respect to the starting energy, the sum of single particle energies of two
 valence particles, as given in Eq.(\ref{eq:formal_iterative_solution}).
 
  Since the $Q$-space is defined as the complement of the $P$-space, intermediate state
 should be taken up to infinitely high oscillator shells. As an
 approximation, we define a  sufficiently large $Q$-space in the evaluation of
 the Q-box.
 Here the Q-box is calculated by 
 $\Vlowk$ with cutoff $\Lambda = 2.1\, \fmi$.
  
   \begin{figure}[h]
    \includegraphics[width=14pc]{Qbox_tensor.eps}\hspace{2pc}%
    \begin{minipage}[b]{14pc}\caption{\label{fig:Qbox_ten}
     Tensor-force monopole part of the effective interaction for the shell model
     calculated with the Q-box expansion.}
    \end{minipage}
   \end{figure}


 Figure \ref{fig:Qbox_ten} shows the tensor-force monopole part of the effective
 interaction for the $sd$ shell and the $pf$ shell. We label this
 effective interaction as $\Veff$.
 One can see again that the tensor-force monopole part of $\Veff$ is fairly similar to that of
 the bare realistic nuclear force both for the $sd$ shell and for the $pf$ shell.
 These results are not trivial again, because they mean that the effects
 of the renormalization of the medium contributions and the truncation of the
 model space are not predominantly affecting the tensor force.
 The first order Q-box is exactly the $\Vlowk$, whose tensor force is
 almost equal to that of the bare nuclear force.
 Therefore, the results mean that the tensor-force monopole is dominated by
 the first order Q-box term and the contributions from the other terms are
 comparatively small.

 \section{conclusion}
 We can conclude that the tensor force survives from both two steps of 
 renormalization, that is the renormalization of the high-momentum component of
 realistic nuclear forces and  the 
 renormalization of the medium effects into the effective interaction of 
 two valence particles.
 %This fact is assumed in several works, and 
% This work is
% the first quantitative justification of the modeling of taking the
% tensor force as that of the bare realistic nuclear force in the shell
% model.

  \subsection{Acknowledgments}

  We are very grateful to Professors R.~Okamoto and H.~Feldmeier
  for valuable discussions.
  This work is supported in part by Grant-in-Aid for Scientific
  Research (A) 20244022 and also by Grant-in-Aid for 
  JSPS Fellows (No.~228635), and by the JSPS Core to Core 
  program ``International Research Network for Exotic Femto Systems''
  (EFES).

\section*{References}
\begin{thebibliography}{99}
\addcontentsline{toc}{chapter}{Reference}

 \bibitem{PhysRevLett.95.232502}
T.~Otsuka et al., Phys. Rev. Lett. \textbf{95}, 232502 (2005).
 \bibitem{HjorthJensen1995125}
M.~Hjorth-Jensen, T.~T.~S.~Kuo and E.~Osnes, Phys.
Repts. \textbf{261}, 125 (1995).

\bibitem{Bogner20031}
S.K. Bogner, T.T.S. Kuo, and A. Schwenk, Phys. Rep.
\textbf{386}, 1 (2003).

 \bibitem{RevModPhys.39.771}
B.H.~Brandow, Rev. Mod. Phys. \textbf{39}, 771-828 (1967).

 \bibitem{Brown1988191}
B.A. Brown and B.H. Wildenthal, Ann. Rev. Nucl. Part.
Sci. \textbf{38}, 29 (1988).


\end{thebibliography}
\end{document}



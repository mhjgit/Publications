%\documentstyle[//saphire/user/mhjensen/text/style/aps]{revtex}
\documentstyle[/local/tex/inputs/fig]{article}



%               Preamble

%               definition of various commands

\setlength{\hoffset}{-0.5in}
\setlength{\textwidth}{6in}
\setlength{\voffset}{-0.5in}
\setlength{\textheight}{8.5in}
%\renewcommand{\baselinestretch}{1,5}
%\APLfig{<filnavn> } {hoyde [cm]}   
\newcommand{\be}{\begin{equation}}
\newcommand{\ee}{\end{equation}}
\newcommand{\bra}[1]{\left\langle #1 \right|}
\newcommand{\ket}[1]{\left| #1 \right\rangle}

%               end of definitions and begin of document

\begin{document}
\pagestyle{plain}
\title{Isobar contributions to the optical-model potential for $^{16}$O.}
\author{M.\ Borromeo$^1$, M. Hjorth-Jensen$^2$, H.\ M\"{u}ther$^3$
and A.\ Polls$^4$\\\\
$^1$ Dipartimento di Fisica, Universita' di Milano, Italia
\\
$^2$ Fysisk Institutt, Universitetet i Oslo, N-0316 Oslo, Norway\\
$^3$ Institut f\"{u}r Theoretische Physik, Universit\"{a}t T\"{u}bingen,\\
D-7400 T\"{u}bingen, Germany
\\ $^4$ Departament d'Estructura i Constituentes de la Materia,\\
Universitat de Barcelona, E-08028 Barcelona, Spain}
%\address{}
\maketitle

%\date{\today}

\begin{abstract}
In this work we evaluate the optical-model potential for $^{16}$O
by accounting for the effect of $\Delta$ isobars  contributions to
both the imaginary and real
part of the optical potential.
We also investigate the role played
by a
finite $\Delta$ isobar self-energy
in the evaluation of the two-particle-one-hole (2p1h)
and three-particle-two-hole (3p2h)
diagrams. Here we employ the energy-dependent parametrization 
of the self-energy of Oset and co-workers.

The effect of the finite isobar self-energy is to redistribute
the strength of the imaginary parts.
However, the total integrated strength with and without a finite 
$\Delta$ isobar self-energy, is conserved.
Thus, in the
determination of quantities like the occupation probabilities for
$l=1$ and $l=2$ states, the largest uncertainty due to isobar contributions
lies in the choice of coupling constants for the meson-nucleon-$\Delta$ 
vertices.
\end{abstract}

%\pacs{}


\clearpage

\section{Introduction}

In this work we derive the self-energy for nucleons with
orbital momentum $l=1,2$ in $^{16}$O from a microscopic point 
of view. With microscopic we will mean a perturbative
many-body  approach which departs from the underlying
nucleon-nucleon interaction defined through the exchange of 
selected mesons. This interaction is renormalized in a
nuclear medium through the introduction of the Brueckner reaction matrix
$G$. The $G$-matrix can in turn be used to calculate
higher-order contributions to the self-energy, such as diagrams
(b) and (c) in fig.\ \ref{fig:diagrams}. Diagram (b) is the 
two-particle-one-hole (2p1h) diagram while (c) represents the
three-particle-two-hole (3p2h) contribution to the self-energy.

The self-energy is intimately related to the optical-model
potential, and provides a microscopic justification 
for the real and imaginary parts of the scattering potential of a free
nucleon interacting with a nucleus. The wealth of data which exist 
on elastic nucleon-nucleus scattering and (e, e') experiments
provide significant information about excitation modes of nuclei. The 
theoretical description of such properties are linked to the self-energy.
As an example, 
the self-energy can be used to define properties like the 
spectral functions.
The integration of the spectral function with respect to the energy yields
information about the e.g.\ the occupation probability of a single-particle
configuration. In the simple
shell-model picture for a closed-shell nucleus like $^{16}$O, the
occupation probabilities for single-particle states below the Fermi energy
are one and zero for single-particle states above. 
As such, the self-energy determines the deviation from the
naive single-particle picture due to nuclear correlations. 



The motivation behind this work is twofold:\newline
Firstly, we wish to extend the analysis of the optical-model potential
for $^{16}$O of refs.\ [1-2]
to
incoming nucleons with orbital angular momentum $l=1,2$. The emphasize in
this work is on the contribution from isobars $\Delta$. 
In ref.\ \cite{hbmp93} we found $\Delta$ excitations to yield a
depletion of the occupation probability of the $s_{1/2}$ state to be around
$2.5\%$, whereas contributions from short-range nucleon-nucleon (NN) 
correlations accounted for $8.5\%$. Our results 
in \cite{hbmp93} were however to be viewed as a qualitative estimate of
isobar contributions, since there is a large uncertainty in the values
of both coupling constants and energy cutoffs used in the evaluation of
$N\Delta$ potentials based on meson exchange ($\pi$ and $\rho$ mesons
in our case). Furthermore, in the analysis of ref.\ \cite{hbmp93} we set
the self-energy of the isobar equal zero, whereas a more realistic 
analysis should account for a finite and medium dependent self-energy.
It was pointed out to us by Oset \cite{os93}, that
the inclusion of a finite self-energy would not only shift the peaks
of the isobar contributions to lower energies, but also the width of these
contributions may get modified. With a constant self-energy, the 
total integrated strengths of the isobar contributions are 
conserved, though, whether a more realistic energy-dependent self-energy
conserves
the total strength or not is an open question. 
These remarks lead us to the second point we wish to address:
Using the parametrization of Oset and co-workers \cite{os87,nieves93},
we investigate the role played by a finite and energy-dependent 
self-energy in the evaluation of the 2p1h and 3p2h diagrams. 

The reason why we focus on isobar contributions is due to the fact
that we relate the real part of the optical potential to the imaginary
part through a dispersion relation. The latter implies that, in order
to have a real part at low energies, we need the imaginary part at both 
low energies and at energies
of several hundreds of MeV (see e.g. the discussion in [1-2]). At high
energies  nucleonic resonances such as the isobar $\Delta$ become of
importance. The $\Delta$ resonance has a mass of 1232 MeV, roughly 300
MeV higher than that of the nucleon. Other resonances come also
into play at higher energies, or meson production not described in terms
of resonances. However, we wish here to focus on the isobar 
$\Delta$ only, since it is the lightest pion-nucleon resonance. 

This work falls in four sections. The next section is devoted to a review
of our approach
and a discussion on the self-energy of the isobar
We recast there the necessary formulae for calculating the 
nucleon self-energy, for more details we will refer the reader
to refs.\ [1-2]. In the
subsequent section we present and discuss our results for $^{16}$O. Finally,
our conclusions are drawn in section four.


\section{A microscopic approach to the optical-model potential}

\subsection{Construction of the $G$-matrix for nucleons and isobars}


For a nucleon and an isobar $\Delta$
interacting through the exchange of $\pi$ plus $\rho$ mesons,
the transition potential $V_{NNN\Delta}$ depicted in fig. \ref{fig:deltrans} 
is usually written, in the static
 non-relativistic
limit, as
\cite{ew88,tow87,mut85}
\be
V_{NNN\Delta}({\bf k})={\displaystyle -\left\{
D_{\pi}^{N\Delta}({\bf k})\frac{f_{\pi NN}f_{\pi
 N\Delta}}{m_{\pi}^{2}}
\mbox{\boldmath $\sigma$} \cdot {\bf k}{\bf S}\cdot
{\bf k}
+D_{\rho}^{N\Delta}({\bf k})\frac{f_{\rho NN}f_{\rho N\Delta}}{m_{\rho}^{2}}
\mbox{\boldmath $\sigma$}\times {\bf k}\cdot {\bf S}\times {\bf k}
\right\}
\mbox{\boldmath $\tau$}{\bf T},}   \label{eq:vndel}
\ee
with $S(T)$ the transition matrix which creates a spin (isospin)
$3/2$ object from a spin (isospin) $1/2$ one.
The meson propagator $D_{\pi , \rho}({\bf k})$ is defined as
\[
D_{\pi , \rho}^{N\Delta}({\bf k})=
\frac{1}{2}\left(\frac{1}{m_{\pi ,\rho}^{2}+{\bf k} ^{2} }+
\frac{1}{m_{\pi , \rho}^{2}+{\bf k} ^{2} +
m_{\pi , \rho}(m_{\Delta}-m_{N})}\right).
\]
Similarly, the $V_{NN\Delta\Delta}$ potential is easily
evaluated by simply replacing $\sigma \rightarrow S$,
$\tau\rightarrow T$
and $f_{\pi,\rho NN}\rightarrow f_{\pi,\rho N\Delta}$ in eq.\
(\ref{eq:vndel}), yielding
\be
V_{NN\Delta\Delta}({\bf k})={\displaystyle -\left\{
D_{\pi}^{\Delta\Delta}({\bf k})\frac{f_{\pi
 N\Delta}^{2}}{m_{\pi}^{2}}
{\bf S}_1 \ \cdot {\bf k}{\bf S}_2 \cdot
{\bf k}
+D_{\rho}^{\Delta\Delta}({\bf k})\frac{f_{\rho N\Delta}^{2}}{m_{\rho}^{2}}
{\bf S}_1 \times {\bf k}\cdot {\bf S}_2 \times {\bf k}
\right\}
{\bf T}_1{\bf T}_2,}   \label{eq:vdeldel}
\ee
The  meson propagator for the
$V_{\Delta\Delta}$ transition potential is given as
\[
D_{\pi , \rho}^{\Delta\Delta}({\bf k})=
\frac{1}{m_{\pi ,\rho}^{2}+{\bf k} ^{2}+m_{\pi ,\rho}(m_{\Delta}-m_{N})} .
\]
The coupling constant for the $\pi$
meson is given by a relation obtained from
the non-relativistic quark model \cite{bw75}
\be
f_{\pi N\Delta} = \frac{6}{5}\sqrt{2}f_{\pi NN}=
\frac{6}{5}\sqrt{2}g_{\pi NN}\frac{m_{\pi}}{2m_{N}},
\ee
and similarly for the $\rho$ meson we have
\be
f_{\rho N\Delta} =\frac{f_{\pi N\Delta}}{f_{\pi NN}}
g_{\rho NN}\frac{m_{\rho}}{4m_{N}}\left(
1+\frac{f_{\rho NN}}{g_{\rho NN}}\right),
\ee
with
\be
f_{\rho NN} = \sqrt{4\pi}g_{\rho NN}\frac{m_{\rho}}{m_{N}}\left(
1+\frac{f_{\rho NN}}{g_{\rho NN}}\right),
\ee
where we have taken the values of vector (tensor) coupling constants
$g_{\rho NN}(f_{\rho NN})$ from table A.2 of reference \cite{mac89}
implying $g_{\rho NN}=0.95$ and $\frac{f_{\rho NN}}{g_{\rho NN}} =6.1$.
In addition we need to include form factors in order to regularize the
potentials at short distances.
Our choice for the relation $f_{\pi N\Delta}/f_{\pi NN}=1.7$ follows that 
of the Bonn NN potential model employed here, though it is smaller than
the experimental value of $2.15$. This difference  leaves  therefore room
for large uncertainties ( of the order $50\%$)  in the evaluation of 
isobar contributions. As to be discussed below in connection with both
the strength of the Wood-Saxon potential and occupation probabilities, our
results are to be viewed as  mere qualitative estimates of isobar contributions. 
In this work we take the cutoff masses which enter the form factors 
and the remaining meson coupling constants
for the $NN$ potential model from the potential model B of table
A.2 of ref.\ \cite{mac89}.
The cutoff parameters  for the $N\Delta$ vertices will be discussed in section 3.

Having defined the various potentials we are now able to write down the corresponding
expressions for the reactions matrices $G$. For nucleons only we have, see also
the discussion in [1-2],
\be
G_{NN}(\Omega) = V_{NN}+V_{NN}Q\frac{1}{\Omega - QTQ}QG_{NN}(\Omega),
\label{eq:gnn}
\ee
where $T$ is the kinetic energy for two particles and $Q$ the Pauli
exclusion operator. The angle-averaged Pauli operator defined in
ref.\ \cite{bbmp91} is used here as well.
The $NN$ potential is conventionally determined in terms of partial waves
and the coordinates of the relative and center of mass motion (RCM),
which means that the $G_{NN}$-matrix in eq.\ (\ref{eq:gnn}) reads
\be
\bra{klKLJST}G_{NN}\ket{k'l'KLJST}.
\ee
The variables
$k$, $k'$ denote the relative momentum, and $l$ and $l'$ the orbital
angular momentum for the relative motion. The quantities
$K$ and $L$ are the corresponding
quantum numbers of the center of mass motion. Finally, $J$, $S$
and $T$ represent the total angular momentum, spin and isospin, respectively.
Transformations from the RCM system to the lab system will be discussed
below.
Starting from the transition potentials $V_{NNN\Delta}$ and
$V_{NN\Delta\Delta}$
we are now able to construct the corresponding $G$-matrices $G_{N\Delta}$
and $G_{\Delta\Delta}$, respectively, by defining
\be
G_{N\Delta}(\Omega)
= V_{NNN\Delta}+G_{NN}(\Omega)
Q\frac{1}{\Omega - QTQ}Q V_{NNN\Delta},
\label{eq:gndel}
\ee
and
\be
G_{\Delta\Delta}(\Omega)
= V_{NN\Delta\Delta}+G_{NN}(\Omega)
Q\frac{1}{\Omega - QTQ}Q V_{NN\Delta\Delta}.
\label{eq:gdeldel}
\ee
Observe that the above equations
are approximations to full the $G_{N\Delta}$- or $G_{\Delta\Delta}$-matrices,
since we omit terms involving the transitions potentials such as 
$V_{\Delta N\Delta N}$ shown in fig.\ \ref{fig:deltrans}. 
One of the problems with this transition potential is the fact that it contains
$\Delta\Delta$-meson vertex on which very little is known.
The attidude in the literature has been either to ignore this potential
or to include only the exchange of $\pi$ and $\rho$ mesons. As in ref.\ 
\cite{hbmp93} we choose here to ignore terms arising from diagrams
(c-e) in fig.\ \ref{fig:deltrans}.

In defining the $G$-matrices we include partial waves for $J,l\leq 4$.
This approximation was found to be sufficient in ref.\ \cite{hbmp93}.




\subsection{Evaluation of the 2p1h and 3p2h contributions to the
self-energy}

The evaluation of the imaginary part arising from the 2p1h and the 3p2h
diagrams with only nucleons or isobars as intermediate states have already 
been discussed in [1-2]. Here we briefly recast the necessary
equations.
In the evaluation of the optical-model potential, we will need to
describe two-particle wave functions in a mixed representation
where differing single-particle wave functions define a two-particle
state.
This leads
to complications in the evaluation of the above diagrams, since
a two-particle wave function may be represented by a particle in a bound
orbital with e.g., the quantum numbers of the harmomic oscillator and the
other particle in the continuum, conveniently given by the quantum numbers
of plane wave functions.
Depending on the choice of single-particle basis, the $G$-matrix elements
needed in the evaluation of the above expressions take different forms.
The most commonly employed sp basis is the harmonic oscillator, which
in turn means that e.g., the symbolic notation for the
two-particle wave function
$\ket{p_1 p_2}$ can be expressed as \cite{law80}
\be
\begin{array}{ll}
&\\
\ket{(n_{1}l_{1}j_{1})(n_{2}l_{2}j_{2})JT}=&
{\displaystyle
\frac{1}{\sqrt{(1+\delta_{12})}}
\sum_{\lambda S{\cal J}}\sum_{nNlL}}
F\times\left\{\begin{array}{ccc}l_{1}&l_{2}&\lambda\\s_1&s_2&S\\
j_{1}&j_{1}&J\end{array}
\right\}\\&\\
&\times (-1)^{\lambda +{\cal J}-L-S}\hat{{\cal J}}\hat{\lambda}^{2}
\hat{j_{1}}\hat{j_{2}}\hat{S}
\left\{\begin{array}{ccc}L&l&\lambda\\S&J&{\cal J}
\end{array}\right\}\\&\\
&\times \left\langle nlNL| n_{1}l_{1}n_{2}l_{2}\right\rangle
\ket{nlNL({\cal J})SJT},\end{array}\label{eq:hoho}
\ee
where the term
$\left\langle nlNL| n_{1}l_{1}n_{2}l_{2}\right\rangle$
is the familiar Moshinsky bracket, see e.g. ref.\ \cite{law80}. Here
we use $\hat{x} = \sqrt{2x +1}$.
The factor $F$ is defined as $F=\frac{1-(-1)^{l+S+T}}{\sqrt{2}}$ when
$s_1 = s_2$, i.e. if we have either two nucleons or two isobars
which define   $\ket{p_1 p_2}$, or $F=\sqrt{2}$ if we have one nucleon
and one isobar. The latter is introduced in order to obtain a normalized
wave function. This is of importance, since we want to employ
anti-symmetrized matrix elements in the evaluation of either the 2p1h or the
3p2h diagrams.
Moreover, the wave function $\ket{nlNL({\cal J})SJT}$, with $nNlL$ the
oscillator quantum numbers of the relative motion and center of mass (RCM),
is related
to $\ket{klKL({\cal J})SJT}$ by \cite{bm89}
\[
\ket{nlNL({\cal J})SJT}= \int k^{2}K^{2}dkdKR_{nl}(\sqrt{2}\alpha k)
R_{NL}(\sqrt{1/2}\alpha K)
\ket{klKL({\cal J})SJT},
\]
with $\alpha$ being the oscillator length and $R_{nl}$ and $R_{NL}$ the HO
functions in momentum space.

A representation for the two-particle wave function were both particles
are represented by plane waves will also be needed. In this case we have
\cite{bm89,kkr79,wc72}
\be
\begin{array}{ll}
&\\
\ket{(k_{1}l_{1}j_{1})(k_{2}l_{2}j_{2})JT}=&
{\displaystyle \sum_{lL\lambda S{\cal J}}\int k^{2}dk\int K^{2}dK}
F\times\left\{\begin{array}{ccc}l_{1}&l_{2}&\lambda\\s_1&s_2&S\\
j_{1}&j_{2}&J\end{array}
\right\}\\&\\
&\times (-1)^{\lambda +{\cal J}-L-S}
\hat{{\cal J}}\hat{\lambda}^{2}
\hat{j_{1}}\hat{j_{2}}\hat{S}
\left\{\begin{array}{ccc}L&l&\lambda\\S&J&{\cal J}
\end{array}\right\}\\&\\
&\times \left\langle klKL| k_{1}l_{1}k_{2}l_{2}\right\rangle
\ket{klKL({\cal J})SJT},\end{array}\label{eq:kk}
\ee
where the term $\left\langle klKL| k_{a}l_{a}k_{b}l_{b}\right\rangle$
is the vector bracket defined in refs.\ \cite{kkr79,wc72}.


Finally, two-particle wave functions with
$\ket{(n_{1}l_{1}j_{1})(k_{2}l_{2}j_{2})JT}$,
can be obtained from the latter equation by noting that
\be
\ket{(n_{a}l_{a}j_{a})(k_{b}l_{b}j_{b})JT}=
\int k_{a}^{2}dk_{a}R_{n_{a}l_{a}}(\alpha k_{a})
\ket{(k_{a}l_{a}j_{a})(k_{b}l_{b}j_{b})JT}.    \label{eq:kho}
\ee

The diagrams involved in the evaluation of the imaginary part of self-energy
are displayed in fig.\ \ref{fig:diagrams} and the corresponding analytical
expressions are obtained using
\[
\frac{1}{\omega\pm i\eta}=P\frac{1}{\omega}\mp i\pi\delta (\omega),
\]
$P$ referring to the principal value, which 
gives for the imaginary part of the 2p1h diagram \cite{hbmp93}, diagrams
(a) in fig.\ \ref{fig:diagrams},
\be
\begin{array}{ll}
{\cal W}_{2p1h}^{BB'}(j_cl_ck_{c}k_{a}\omega) = &{\displaystyle -\frac{1}
{2(2j_c+1)}\sum_{n_{h}l_{h}j_{h}}
\sum_{JT}\sum_{lLS{\cal J}}\int k^{2}dk\int K^{2}dK\hat{J}\hat{T}}\\&\\
&\times \bra{k_{a}l_{c}j_{c}n_{h}l_{h}j_{h}JT}G_{BB'}\ket{klKL({\cal J})SJT}\\&\\
&\times\bra{klKL({\cal J})SJT}G_{BB'}\ket{k_{c}l_{c}j_{c}n_{h}l_{h}j_{h}JT}\\&\\
&\times\pi\delta(\omega + \varepsilon_{h}-\varepsilon_{B}
-\varepsilon_{B'}),
\end{array} \label{eq:2p1hbb}
\ee
where $BB'=NN$, $BB' = N\Delta$ or
$BB' = \Delta\Delta$. 
The single-hole energy $\varepsilon_{h}$ is given by the eigenvalues
of the harmonic oscillator.
Only positive energies $\omega$ contribute, as can be deduced from the
$\delta$ function in eq.\ (\ref{eq:2p1hbb}). Note that we have
assumed that
the self-energy of the isobar is set equal to zero in order to obtain
the above result. The latter implies that we set the energy of
the isobar to be given by a pure kinetic energy term plus the mass
difference between the isobar and the nucleon, i.e.,
\[
\omega_{\Delta} = \frac{k_{\Delta}^{2}}{2m_{\Delta}}
+m_{\Delta}-m_{N}.
\]
The energies of the intermediate states are further given in terms
of $k$ and $K$ using an angle average prescription such that 
$<{\bf k}{\bf K}>=0$.

Correspondingly, the imaginary contribution arising from the 3p2h
diagram, (c) in fig.\ \ref{fig:diagrams}, with either one nucleon particle
$BB'=NN$ or one intermediate $\Delta$ state $BB'=N\Delta$ reads
\be
\begin{array}{ll}
{\cal W}_{3p2h}^{BB'}(j_cl_ck_{c}k_{a}\omega) = &{\displaystyle -\frac{1}
{2(2j_c+1)}\sum_{l_{B'}j_{B'}}
\sum_{JT}\sum_{nNlLS{\cal J}}\int k_{B'}^{2}dk_{B'}\hat{J}\hat{T}}\\&\\
&\times \bra{k_{a}l_{c}j_{c}k_{B'}l_{B'}j_{B'}JT}
G_{BB'}\ket{nlNL({\cal J})SJT}\\&\\
&\times\bra{nlNL({\cal J})SJT}G_{BB'}\ket{k_{c}l_{c}j_{c}
k_{B'}l_{B'}j_{B'}JT}\\&\\
&\times\pi\delta(-\omega + \varepsilon_{nNlL}-\varepsilon_{B'}),
\end{array} \label{eq:3p2hbb}
\ee
where $\varepsilon_{nNlL}$ is the total energy of the two holes in the center
of mass and relative coordinate system represented in this
work in terms of a  HO basis.



\subsubsection{Contributions to the isobar self-energy}

In the previous subsection we assumed that the isobar energy
was given by a pure kinetic energy term, plus the difference in mass
between the isobar and the nucleon.
If we now take  the self-energy of the isobar to be
composed of both a real and  an
imaginary part, the isobar energy reads
\[
\omega_{\Delta} = \frac{k_{\Delta}^{2}}{2m_{\Delta}}
+m_{\Delta}-m_{N} + \Sigma_{\Delta}(\omega_{\Delta},k_{\Delta})
\]
with
\[
\Sigma_{\Delta}(\omega_{\Delta},k_{\Delta})=
Re\Sigma_{\Delta}(\omega_{\Delta},k_{\Delta})
+iIm\Sigma_{\Delta}(\omega_{\Delta},k_{\Delta}).
\]

As there exist no direct experimental measurements of the $\Delta$
self-energy, which, combined with the fact that the coupling constants
which determine e.g., the $V_{N\Delta N\Delta}$ potential
discussed in subsection 2.1, are unknown, one has in principle
little control on
whether a theoretically derived self-energy is reliable or not.
Typical contributions to $\Sigma_{\Delta}(\omega_{\Delta},k_{\Delta})$
are displayed in fig.\ \ref{fig:delself}. Diagram (a)
involves the decay of the isobar into a nucleon and pion,
and gives rise to the free $\Delta$ width if the nucleon and the pion
were free particles. However, the intermediate nucleon states are partly
blocked for the decaying isobar, since some of these states are occupied.
These contributions lead to an energy dependent correction due to the
Pauli principle (Pauli blocking) of the free width \cite{ow79}. Here we
approximate diagram (a) with the empirical free width. 
Diagrams (b) and (c) represent further $\Delta$-nucleus interactions.
Experimental data for protons derived by pions at various energies covering
the isobar resonance region indicate that
the dominant process for pion absorption
is represented by  diagram (b), which couples the isobar to two-nucleon
one-hole states ($\Delta + N \rightarrow 2N$).
Within the terminology of the isobar-hole model \cite{ew88,otw83},
diagram (c) is then supposed to represent the rescattering of e.g., a real
pion, so-called reflection contributions to quasi-free scattering.
Diagram (d) is an example of a
self-energy contribution arising from three-body
absorption mechanisms. 


Albeit one in principle is not able to derive a theoretical
consistent $\Delta$ self-energy, indirect information about the self-energy
can be derived from e.g.\ pion-nucleus scattering. In the extensive
analyses of ref.\ \cite{hirata80}, contributions arising from diagrams
like (b) and (c) in fig.\ \ref{fig:delself}, are represented by way
of a $\Delta$-spreading potential, fitted to provide best results for
pion-nucleus elastic scattering.

Typical contributions
to both the imaginary and real parts of the self-energy arise
from topologies like those diplayed in fig.\ \ref{fig:delself}.
In ref.\ \cite{os87}, a parametrization for the imaginary part of the
$\Delta$ self-energy is obtained in nuclear matter by considering the
contributions from the diagrams  in fig.\ \ref{fig:delself}, accounting
thus for quasielastic corrections, two-body and three-body
absorption. The nuclear matter results are then compared to the
$\Delta$-spreading potential from the empirical determination in
ref.\ \cite{hirata80} by allowing for a density dependent self-energy,
given by the approximate analytical expression
\be
Im\Sigma_{\Delta} =
Im\Sigma_{\Delta}^{A2} +Im\Sigma_{\Delta}^{A3} +Im\Sigma_{\Delta}^{Q}, 
\label{eq:oset}
\ee
The term ${Q}$ accounts for the quasielastic part and goes to zero at
$T_{\pi}\approx 80$ MeV, while ${A2}$ is the 
two-body absorption part, which is zero at $\omega_{\Delta} =0$. Finally, 
${A3}$ represents three-body absorption, going to zero at $T_{\pi}=0$. 
A numerical parametrization for these terms was given by Oset and Salcedo
\cite{os87}, while analytical expressions based on the results of ref.\
\cite{os87} were recently presented by Nieves {\em et al.} \cite{nieves93}. 
In the latter work the terms of eq.\ (\ref{eq:oset}) are given as
\be
Im\Sigma_{\Delta}^{A2}=Im\hat{\Sigma}_{\Delta}(x)\frac{1}{a(x)}
arctg\left(a(x)(\rho/\rho_0)\right),
\ee
with $x=\frac{T}{m_{\pi}}$, $T$ and $m_{\pi}$ being the energy and
mass of the pion, respectively. $\rho$ is the density of particles and
we set the relation $\rho/\rho_0=0.75$ \cite{os87}. Further, the function 
$\hat{\Sigma}_{\Delta}$  is 
\be 
Im\hat{\Sigma}_{\Delta}(x)=-38.3\left(1-0.85x+0.54x^2\right),
\ee
and
\be
a(x)=2.72-4.07x+3.07x^2.
\ee
The three-body therm $A3$ is parametrized as
\be
Im\Sigma_{\Delta}^{A3}=-C_{A3}(x)\left(\rho_/\rho_0\right)^{\alpha (x)},
\ee
with 
\be
C_{A3}(x)=\left\{
\begin{array}{cc} x\left(-7.08+27.4x-9.49x^2\right)& x\geq 0.62\\
                  \frac{3.7xm_{\pi}}{85}
              & x\leq 0.62 \end{array}\right. ,
\ee
and
\be
\alpha (x)= 1+x\left(0.984-0.512x+0.1x^2\right).
\ee
Finally, the quasielastic part is given as
\be
Im\Sigma_{\Delta}^{Q}=-C_{Q}(x)\left(\rho/\rho_0\right)^{\beta (x)},
\ee
with
\be
C_{Q}(x) = x\left(20.2 -8.58x+0.702x^2\right),
\ee
and
\be
\beta (x)=1+x\left(-0.309-0.315x+0.151x^2\right).
\ee



For the real part, which in principle can be deduced from the imaginary part
through a dispersion relation, we also choose to adopt the  phenomenological
parametrization of ref.\ \cite{os87}
\[
Re\Sigma_{\Delta}(\omega_{\Delta},k_{\Delta})\approx -35\rho/\rho_0 MeV. 
\]

The parametrization of refs.\ \cite{os87,nieves93} is also of great interest because
it provides information about the different sources to the imaginary part
of the $\Delta$ spreading potential. In this work we employ the ansatz
defined by eq.\ (\ref{eq:oset}) in order to determine the self-energy of the
isobar, in addition to the terms provided by the free width and the Pauli
correction to the free width.
Finally, it is worth noting that both the imaginary and real parts 
in eq.\ (\ref{eq:oset}) 
depend only on the
energy of the isobar, whereas, see the discussion below, 
the self-energy which enter our calculations depends, in principle,
on the momentum of the isobar and the  orbital and angular momenta
as well.

\subsubsection{The 2p1h and 3p2h diagrams with a finite isobar self-energy}

Here we list the final expressions for the 2p1h and 3p2h diagrams including
one or two isobars $\Delta$ as intermediate states allowing for a 
finite isobar self-energy.

With one isobar as intermediate state we obtain for the imaginary contribution
to the 2p1h diagram
\be
\begin{array}{ll}
{\cal W}_{2p1h}^{N\Delta}(j_cl_ck_{c}k_{a}\omega) = &{\displaystyle -\frac{1}
{2(2j_c+1)}\sum_{n_{h}l_{h}j_{h}}
\sum_{JT}\sum_{lLS{\cal J}}\int k^{2}dk\int K^{2}dK\hat{J}\hat{T}}\\&\\
&\times \bra{k_{a}l_{c}j_{c}n_{h}l_{h}j_{h}JT}G_{N\Delta}
\ket{klKL({\cal J})SJT}\\&\\
&\times\bra{klKL({\cal J})SJT}G_{N\Delta}
\ket{k_{c}l_{c}j_{c}n_{h}l_{h}j_{h}JT}\\&\\
&\times\frac{Im\Sigma_{\Delta}
(\omega_{\Delta},k_{\Delta})}{\alpha^{2}
+Im\Sigma_{\Delta}^2
(\omega_{\Delta},k_{\Delta})},
\end{array} \label{eq:2p1himdel}
\ee
while with two isobars we have
\be
\begin{array}{ll}
{\cal W}_{2p1h}^{N\Delta}(j_cl_ck_{c}k_{a}\omega) = &{\displaystyle -\frac{1}
{2(2j_c+1)}\sum_{n_{h}l_{h}j_{h}}
\sum_{JT}\sum_{lLS{\cal J}}\int k^{2}dk\int K^{2}dK\hat{J}\hat{T}}\\&\\
&\times \bra{k_{a}l_{c}j_{c}n_{h}l_{h}j_{h}JT}G_{\Delta\Delta}
\ket{klKL({\cal J})SJT}\\&\\
&\times\bra{klKL({\cal J})SJT}G_{\Delta\Delta}
\ket{k_{c}l_{c}j_{c}n_{h}l_{h}j_{h}JT}\\&\\
&\times\frac{2Im\Sigma_{\Delta}
(\omega_{\Delta},k_{\Delta})}{\beta^{2}
+4Im\Sigma_{\Delta}^2
(\omega_{\Delta},k_{\Delta})}.
\end{array} \label{eq:2p1him2del}
\ee
It ought to be remarked that
with our RCM representation for the intermediate states,
we are not able to define directly an energy and momentum for e.g.\ the isobar
intermerdiate state(s). Further, the isobar self-energy depends also
on the orbital and angular momentum. This would however require that we sum
explicitely over the quantum numbers of the intermediate states in the 
lab frame, a procedure which would blow up the computational expenditure since
there is no restriction on e.g. angular momentum for the intermediate states
in the lab frame. We will therefore stick to our representation of the 
intermediate states in the RCM representation and pragmatically adopt
the parametrization of Oset and co-workers \cite{os87,nieves93}.   
Furthermore, the energy and momentum of the intermediate isobar
determines in turn the self-energy $Im \Sigma_{\Delta}$. We can however
circumvent the latter deficiency by the following prescription ( we discuss
here eqs.\ (\ref{eq:2p1himdel}) and (\ref{eq:2p1him2del})):
\begin{itemize}
\item We can determine $k_{\Delta}$ and $k_p$ from momentum
conservation (using an angle average $<{\bf k}{\bf K}>=0$) such that
\[
k_{\Delta}^2=k_p^2 = \frac{K^2}{4}+k^2.
\]
With two isobars as intermediate states we simply replace $k_p$ with
$k_{\Delta}$.
\item
The energy of the intermediate isobar can be found from
\[
\omega_{\Delta} = \omega + \varepsilon_h - \varepsilon_p
=\omega + \varepsilon_h -\frac{K^2 /4+k^2}{2m_N}.
\]
Similar to the previous point, with two isobars we replace $\varepsilon_p$
with $\frac{K^2 /4+k^2}{2m_{\Delta}}+(m_{\Delta}-m_n)+Re\Sigma_{\Delta}$.
The energy $\omega_{\Delta}$ is used to determine the energy in the parametrization
of the isobar self-energy, see eq.\ (\ref{eq:oset}).
\item 
The term $\alpha$ in the energy denominator of eq.\ (\ref{eq:2p1himdel}) reads
\[
\alpha=\omega+\varepsilon_{h}
-m_{\Delta}+m_{N}-
Re\Sigma_{\Delta}(\omega_{\Delta},k_{\Delta})
-\frac{K^{2}}{8\mu_{\Delta N}}-
\frac{k^{2}}{2\mu_{\Delta N}},
\]
with $\mu_{\Delta N}=\frac{m_{\Delta}m_{N}}{m_N + m_{\Delta}}$. Similarly, we have
that $\beta$ in eq.\ (\ref{eq:2p1him2del}) is
\[
\beta=\omega+\varepsilon_{h}-
-2(m_{\Delta}-m_{N})-
2Re\Sigma_{\Delta}(\omega_{\Delta},k_{\Delta})
-\frac{K^{2}}{4m_{\Delta}}-
\frac{k^{2}}{m_{\Delta}}.
\]

\end{itemize}

For the 3p2h diagram we have
\be
\begin{array}{ll}
{\cal W}_{3p2h}^{N\Delta}(j_cl_ck_{c}k_{a}\omega) = &{\displaystyle -\frac{1}
{2(2j_c+1)}\sum_{l_{\Delta}j_{\Delta}}
\sum_{JT}\sum_{nNlLS{\cal J}}\int k^{2}_{\Delta}dk_{\Delta}
\hat{J}\hat{T}}\\&\\
&\times \bra{k_{a}l_{c}j_{c}k_{\Delta}l_{\Delta}j_{\Delta}JT}
G_{N\Delta}\ket{nlNL({\cal J})SJT}\\&\\
&\times\bra{nlNL({\cal J})SJT}G_{N\Delta}
\ket{k_{c}l_{c}j_{c}k_{\Delta}l_{\Delta}j_{\Delta}JT}\\&\\
&\times\frac{Im\Sigma_{\Delta}
(\omega_{\Delta},k_{\Delta})}{\delta^{2}
+Im\Sigma_{\Delta}^2
(\omega_{\Delta},k_{\Delta})},
\end{array} \label{eq:3p2him}
\ee
with the isobar in the lab system and
\[\delta=-\omega+\varepsilon_{nNlL}-
\frac{k_{\Delta}^{2}}{2m_{\Delta}}
-m_{\Delta}+m_{N}-
Re\Sigma_{\Delta}(\omega_{\Delta},k_{\Delta}).
\]

\subsection{Isobar contributions to the real part of the 
optical-model potential}
After calculating the imaginary parts for the nucleon self-energy terms
displayed in fig.\ \ref{fig:diagrams}  one may also calculate 
the corresponding real
parts by means of the dispersion relations \cite{bhmp93a,bbmp91}
\be
{\cal V}_{2p1h}(j_cl_ck_ck_a\omega)=\frac{P}{\pi} \int_{-\infty}^{\infty}
\frac{{\cal W}_{2p1h}^{NN}(j_cl_ck_ck_a\omega')}{\omega'-\omega} d\omega',
\ee
where $P$ means a principal value integral. Since  ${\cal W}_{2p1h}^{NN}$
is different from zero only for positive values of
$\omega'$ and as its diagonal matrix elements are negative,
this dispersion relation  implies that the diagonal elements of
${\cal V}_{2p1h}$ will be attractive for
negative energies $\omega$. This attraction should increase for small
positive energies. It will decrease and will change sign only for large
positive values for the energy of the interacting nucleon. This analysis 
was performed in the preceeding paper \cite{bhmp93a}. Here
we  wish also to account for the presence of isobars in the
evaluation of the above equation. This is simply achieved by replacing
${\cal W}_{2p1h}^{NN}(j_cl_ck_ck_a\omega')$ with either
${\cal W}_{2p1h}^{N\Delta}(j_cl_ck_ck_a\omega')$ or
${\cal W}_{2p1h}^{\Delta\Delta}(j_cl_ck_ck_a\omega')$. 

Similar dispersion
relations hold for the real (${\cal V}_{3p2h}^{N\Delta}$) and imaginary part (${\cal
W}_{3p2h}^{N\Delta}$) of the 3p2h contribution  to the self-energy
\be
{\cal V}_{3p2h}^{N\Delta}(j_cl_ck_ck_a\omega)=-\frac{P}{\pi} \int_{-\infty}^{\infty}
\frac{{\cal W}_{3p2h}^{N\Delta}(j_cl_ck_ck_a\omega')}{ \omega'-\omega} d\omega' .
\ee
Since ${\cal W}_{3p2h}^{NN}$ is positive (at least its diagonal matrix
elements) and
different from zero for negative energies $\omega'$ only, it is evident
that ${\cal V}_{3p2h}^{N\Delta}$ is repulsive for positive energies and decreases
with increasing energy. Only for large and negative energies it will become
attractive. 


\section{Results and discussion}

\subsection{The imaginary part}
Several of the discussions of ref.\ \cite{hbmp93} on isobar contributions
for $l=0$, apply to the present study as well.
We will therefore 
not repeat some of the arguments presented in \cite{hbmp93}, rather, 
in this subsection, 
we wish first to
investigate the dependence of the imaginary part
on the cutoff for the $\pi$ and $\rho$ mesons, $\Lambda_{\pi ,\rho}$.
Secondly, we focus on the role played by a finite and energy-dependent
$Im\Sigma_{\Delta}$. Furthermore, in this subsection, 
we will limit the discussion to
eq.\ (\ref{eq:2p1himdel}) only, with $l=1$ and $j=1/2$ for the incoming
nucleon. The reason being that eqs.\ (\ref{eq:2p1him2del}) and 
(\ref{eq:3p2him}) exhibit a qualitative similar behavior.

In the determination of cutoffs which enter the evaluation of the
$V_{NNN\Delta}$ and $V_{NN\Delta\Delta}$ potentials, we choose, in order
to be consistent, the same
parameters as those which define the Bonn B potential of ref.\ \cite{mac89},
table A.2. This leads to $\Lambda_{\pi}=1.2$ and $\Lambda_{\rho}=1.3$ GeV.
These parameters lie in the higher region of accepted values, though other
workers in the field even choose $\Lambda_{\rho}=2.0-2.5$ GeV
\cite{garc91,ost92} and $\Lambda_{\pi}=0.8$ GeV
\cite{ost92}. Thus, although it is not consistent with the 
parameters of the Bonn B potential, we display in fig.\ \ref{fig:test1}
results for eq.\ (\ref{eq:2p1himdel}) with $\Sigma_{\Delta}=0$ for three
choices of $\Lambda_{\pi ,\rho}$, i.e. (i) $\Lambda_{\pi}=1.2$ GeV
and $\Lambda_{\rho}=1.3$ GeV, (ii) $\Lambda_{\pi}=1.2$ GeV and
$\Lambda_{\rho}=2.5$ GeV, (iii) $\Lambda_{\pi}=0.8$ GeV and 
$\Lambda_{\rho}=2.0$ GeV and (iv)
$\Lambda_{\pi}=0.8$ GeV and 
$\Lambda_{\rho}=1.3$ GeV.
Here we used $k=100$ MeV and $l=1$ and 
$j=1/2$. We see from fig.\ \ref{fig:test1} 
that  
with $\Lambda_{\rho}\geq 2.0$ GeV we obtain results which are quite different
at high energies than the results obtained with the parameters of the
Bonn B potential. One possible explanation to these differing behaviors
is simply that we are not being consistent with the parameters which define
the NN potential. Thus, results with other parameters than those of
the potential  may give rather unrealistic imaginary parts. Another
possibility, is that
with $\Lambda_{\rho}\geq 2.0$ GeV, we
are probing the short range part of the potential, which is known
to diverge at small distances. If we choose $\Lambda_{\rho}\geq 2.0$ GeV,
or even 1.5 GeV, we are actually pushing the meson-exchange picture
beyond its range of validity, yielding unrealistic results for the
self-energy. If we choose $\Lambda_{\pi}=0.8$ GeV and $\Lambda_{\rho}=1.3$ GeV
(not shown in fig.\ \ref{fig:test1}), we obtain a reduction 
of the total integrated strength of about $50\%$ 
compared to the results with the parameters of the Bonn B potential.
A similar qualitative behavior as that discussed above, apply to the 
3p2h diagram with one isobar and the 2p1h diagram with two isobars
as intermediate states. 
Here we wish to conclude that one has to be consistent with choice of
parameters which define the NN potential, although the results of fig.\
\ref{fig:test1} leave a large degree of uncertainty in the determination
of $\Delta$ contributions.



If we now include a finite isobar self-energy,
a useful numerical test on the double integrations in eqs.\ (\ref{eq:2p1himdel})-
(\ref{eq:3p2him}) is to first consider a constant imaginary part
$Im\Sigma_{\Delta}$. Application of
\[
\int_{-\infty}^{\infty}dx\frac{Im\Sigma_{\Delta}}{x^2
+Im\Sigma_{\Delta}^2} =\pi,
\label{eq:const}
\]
shows then that the total strength should be conserved. Note also
that in limit $\Sigma_{\Delta}\rightarrow 0$, one obtains a
$\delta$-function
\[
{\displaystyle lim_{\Sigma_{\Delta}\rightarrow 0}\frac{Im\Sigma_{\Delta}}{x^2
+Im\Sigma_{\Delta}^2}=\pi \delta (x)}.
\]
In connection with the latter, it is important to note
that with our choice of imaginary part described by eq.\
(\ref{eq:oset}), we have that  due to the two-body absorption term
the imaginary part is zero when $\omega_{\Delta}$ is zero. This leads in turn
to an integrand in eqs. (\ref{eq:2p1himdel}), (\ref{eq:2p1him2del}) 
and (\ref{eq:3p2him}) which never gives
rise to a $\delta$-function singularity when $x$ occasionally becomes
equal zero. This observation should be compared to the
recent nuclear matter analysis of Davies \cite{dav92}, where 
$Im\Sigma_{\Delta}$ is set equal the free width, which occurs at 
$\omega_{\Delta}>0$. Several of the numerical instabilities reported in the work of
Davies \cite{dav92} could  then be retraced to the use of a less realistic
$Im\Sigma_{\Delta}$. 

In fig. \ref{fig:test2}
we compare the imaginary part arising from eq.\ (\ref{eq:2p1himdel}), 
for an incoming
nucleon with $l=1$, $j=1/2$  and momentum $k=100$ MeV, upper part,
and $k=200$ MeV, lower part,
using (i) no finite self-energy of the isobars (solid line) and
(ii)
employing the parametrization of Oset and Salcedo of eq.\ (\ref{eq:oset})
for $\Sigma_{\Delta}$ (dashed line).
AS can be seen from this figure, the total integrated areas with 
$\Sigma_{\Delta}=0$ and $\Sigma_{\Delta}\neq 0$ are roughly conserved (a
numerical integration over $\omega$ confirms this). The strenghts are however
redistributed and the positions of the peaks shift slightly. 
The qualitative pattern we exhibit in fig.\ \ref{fig:test2}
pertains to other values of $k$ and orbital and angular momentum
of the incoming nucleon. This result is however not surprising as can 
easily be understood from the following argument. Consider a simple
shell-model example with only one sp state $\alpha$ 
and two configurations of
a more complicated character. The latter represent the admixture of
$2p1h$ and $\Delta ph$ configurations. The eigenvalue problem is then
\be
\left(\begin{array}{ccc}\varepsilon_{\alpha}&V_{2p1h}&V_{\Delta ph}\\
                        V_{2p1h}&\varepsilon_{2p1h}&0\\
                        V_{\Delta ph}&0&\varepsilon_{\Delta ph}
      \end{array}\right)
\left(\begin{array}{c}X_{\alpha}\\X_{2p1h}\\X_{\Delta ph}
      \end{array}\right)=
E\left(\begin{array}{c}X_{\alpha}\\X_{2p1h}\\X_{\Delta ph}
      \end{array}\right),
\ee
where we have set
\[
\varepsilon_{2p1h}=\varepsilon_{p1}+\varepsilon_{p2}-\varepsilon_h,
\]
and 
\[
\varepsilon_{\Delta p1h}=\varepsilon_{\Delta}+\varepsilon_{p}-\varepsilon_h,
\]
with the $\varepsilon$s being the unperturbed sp energies. The overlap
between the $2p1h$ and $\Delta ph$ configurations is obviously zero.
We assume first that both the interaction terms  $V$ 
and $\varepsilon_{\Delta}$  are real. The above eigenvalue problem is
trivially solved
\be
\left(\varepsilon_{\alpha}+\frac{V_{2p1h}^2}{E-\varepsilon_{2p1h}}
\frac{V_{\Delta ph}^2}{E-\varepsilon_{\Delta ph}}\right)X_{\alpha}
=EX_{\alpha}.
\ee
The normalization condition for the expansion coefficients $X$ is
\be
1=X_{\alpha}^2+X_{2p1h}^2+X_{\Delta ph}^2 =
\left(1+\frac{V_{2p1h}^2}{(E-\varepsilon_{2p1h})^2}
\frac{V_{\Delta ph}^2}{(E-\varepsilon_{\Delta ph})^2}\right)X_{\alpha}^2.
\ee



Thus, since the shape of the curves exhibited
in fig.\ \ref{fig:test2}, their widths and position of maxima, are not so
different, we claim that the largest uncertainty in the evaluation
of isobar contributions to the 2p1h and 3p2h terms lies in the choice of
coupling constants $f_{\pi N\Delta}$. Actually, a contribution like
eq.\ (\ref{eq:2p1himdel}) gives rise to a difference between our choice
of $f_{\pi N\Delta  }$ and the experimental value of $\approx 40\%$, while a
difference of $\approx 60\% $ occurs in eq.\ (\ref{eq:2p1him2del}).
Our choice of $f_{\pi N\Delta}$ and $f_{\rho N\Delta}$ follows that of the
Bonn B potential \cite{mac89}.
The difference between a calculation which
accounts for a finite isobar self-energy and one with $\Sigma_{\Delta}=0$, is
therefore couched by the choice of coupling constants.
Finally, the quantities we are looking for, e.g. the occupation
probabilities and the real part of self-energy, are given in terms of integrals
over the energy. Since the positions of the peaks are not so different, 
and the total integrated
strenghts are roughly the same, we don't expect large differences between
a calculation with a finite $\Sigma_{\Delta}$ or one without. Also, the 
numerical expenditures favor the $\Sigma_{\Delta}=0$ approximation
\footnote{A note on
technicalities: If one in addition to the above observation
adds the fact that, in order to obtain
a stable result with a finite isobar width, we needed
$40-50$ mesh points in the integration over $k$-relative. With
12 mesh points needed in the integration of $K$-center of mass and the hole
momentum $k_h$, the computational time expenditure grows tremendously.
To obtain results for e.g. eq.\ (\ref{eq:2p1himdel}) with $l=1$ and $j=1/2$,
and 8 values of the incoming momentum $k$ and 25 values of the energy $\omega$,
we needed several
days (CPU time) on a Convex supercomputer. With $\Sigma_{\Delta}=0$,
the corresponding number is one CPU hour.}.







\subsection{Real part and occupation probabilities for $l=1,2$ states}




\section{Conclusions}
Many discussions with Prof.\ Eulogio Oset (Valencia) are gratefully
acknowledged.
This work has been supported by the Norwegian Research
Council for Science and the Humanities (NAVF), the
``Bundesministerium f\"ur Forschung und
Technologie'' (?? T\"u ??) (Germany) and
by DGICYT-grant No.\ PB???? (Spain).


%        bibliography

%\begin{references}
\footnotesize{\begin{thebibliography}{99}
\bibitem{bbmp91} M.\ Borromeo, D.\ Bonatsos, H.\ M\"{u}ther and A.\ Polls,
Nucl. Phys. {\bf A539} (1992) 189
\bibitem{hbmp93} M.\ Hjorth-Jensen, M.\ Borromeo, H.\ M\"{u}ther
and A.\ Polls,
Nucl.\ Phys.\ {\bf  A551} (1993) 580
\bibitem{os93} E.\ Oset, private communication
\bibitem{os87} E.\ Oset and L.L.\ Salcedo, Nucl.\ Phys.\ {\bf A468} (1987) 631
\bibitem{nieves93} J.\ Nieves, E.\ Oset and C.\ Garcia-Recio,
Nucl. Phys. {\bf A554} (1993) 554
\bibitem{ew88} T.E.O.\ Ericson and W.\ Weise, Pions and Nuclei
(Clarendon press, Oxford, 1988)
\bibitem{tow87} I.S.\ Towner, Phys.\ Rep.\ {\bf 155} (1987) 263
\bibitem{mut85} H.\ M\"{u}ther, Prog.\ Part.\ Nucl.\ Phys.\ {\bf 14} (1985) 123
\bibitem{bw75} G.E.\ Brown and W.\ Weise, Phys.\ Rep.\ {\bf 22} (1975) 279
\bibitem{mac89} R.\ Machleidt, Adv.\ Nucl.\ Phys.\ {\bf 19} (1989) 189
\bibitem{law80} R.D.\ Lawson, Theory of the Nuclear Shell Model
(Clarendon Press, Oxford, 1980), p.\ 208
\bibitem{bm89} D.\ Bonatsos and H.\ M\"{u}ther, Nucl.\ Phys.\ {\bf A496} (1989) 23
\bibitem{kkr79} C.L.\ Kung, T.T.S.\ Kuo and K.F.\ Ratcliff, Phys. Rev.
{\bf C19} (1979) 1063
\bibitem{wc72} C.W.\ Wong and D.M.\ Clement, Nucl.\ Phys.\ {\bf A183} (1972) 210
\bibitem{ow79} E.\ Oset and W.\ Weise, Nucl.\ Phys.\ {\bf A319} (1979) 477;
W.\ Weise, Nucl.\ Phys.\ {\bf A278} (1977) 402
\bibitem{otw83} E. Oset, H. Toki and W. Weise, Phys.\ Rep.\ {\bf 83} (1982) 281
\bibitem{hirata80} M.\ Hirata, F.\ Lenz and K.\ Yazaki, Ann. of Phys. 
{\bf 108} (1977) 108; M.\ Hirata, J.H.\ Koch and E.J.\ Moniz, Ann.\ of Phys.\
{\bf 120} (1979) 205; Y.\ Horikawa, M.\ Thies and F.\ Lenz, Nucl.\ Phys.\
{\bf A345} (1980) 386
\bibitem{ost92} F.\ Osterfeld, Rev. Mod. Phys. {\bf 64} (1992) 491
\bibitem{garc91} C.\ Garcia-Recio, E.\ Oset and L.L.\ Salcedo, D.\ Strottman
and M.J.\ Lopez, Nucl.\ Phys.\ {\bf A526} (1991) 685
\bibitem{dav92} K.T.R.\ Davies, Ann.\ of Phys.\ {\bf 215} (1992) 386
\end{thebibliography}}
%\end{references}

\begin{table}[hbtp]
\caption{The coefficients of eq.\ (15) used in the calculation
of the imaginary part of the isobar self-energy in the case of pion
nuclear scattering. Taken from ref.\ [4].}
\begin{center}
\begin{tabular}{ccccrrr}
&&&&&&\\ \hline\hline
&&&&&&\\
\multicolumn{1}{c}{$T_{\pi}$(MeV)}&\multicolumn{1}{c}{$C_Q$(MeV)}&
\multicolumn{1}{c}{$C_{A2}$(MeV)}&\multicolumn{1}{c}{$C_{A3}$(MeV)}&
\multicolumn{1}{c}{$\alpha$}&\multicolumn{1}{c}{$\beta$}
&\multicolumn{1}{c}{$\gamma$}\\
&&&&&&\\ \hline
&&&&&&\\
85&9.7&18.9&3.7&0.79&0.72&1.44\\
125&11.9&17.7&8.6&0.62&0.77&1.54\\
165&12.0&16.3&15.8&0.42&0.80&1.60\\
205&13.0&15.2&18.0&0.31&0.83&1.66\\
245&14.3&14.1&20.2&0.36&0.85&1.70\\
315&9.8&13.1&14.7&0.42&0.88&1.76\\
&&&&&&\\ \hline\hline
\end{tabular}
\end{center}
\label{tab:delparam}
\end{table}
%\begin{figure}[hbtp]
\PSfigure{fig5.fig }{10}
%\caption
{Diagram (a) is the energy independent 
Hartree-Fock contribution. The 2p1h diagrams with nucleons only, 
one isobar and two isobars as intermediate state(s)
are depicted in (b). The corresponding 3p2h term with nucleons only
and one isobar as intermediate state are shown 
in (c).}
%\label
{fig:diagrams}
%\end{figure}
\clearpage


%\begin{figure}[hbtp]
\PSfigure{fig4.fig }{10}
%\caption
{Transitions potentials for the $V_{N\Delta NN}$ (a), $V_{\Delta\Delta NN}$
(b), $V_{\Delta N\Delta N}$ (c), $V_{\Delta\Delta\Delta N}$ (d) and
$V_{\Delta\Delta\Delta\Delta}$ (e) processes. Since there is no empirical 
knowledge on the meson-$\Delta\Delta$ vertices we choose to omit the contributions
from diagrams (c-e) in the evaluation of the $G_{N\Delta}$- and
$G_{\Delta\Delta}$-matrices discussed in eqs.\ (8) and (9).}
%\label
{fig:deltrans}
%\end{figure}

\clearpage

%\begin{figure}[hbtp]
\PSfigure{fig6.fig }{10}
%\caption
{Example of diagrams which arise in the evaluation
of the self-energy of the isobar. (a) is the free width which is
accompanied with a Pauli blocking term. (b) is the two-body absorption term, while
(c) is an example of a so-called reflection contribution to quasi-free
scattering. Finally, (d) is a three-body contribution.}
%\label
{fig:delself}
%\end{figure}
\clearpage
\begin{figure}[hbtp]
\vspace{10cm}
\caption{The parametrization of the imaginary part
of the isobar self-energy
from ref.\ [4] . $A2$ stands for two-body absorption, $A3$ for three-body
absorption while $Q$ is the quasielastic contribution. The bars are from the
empirical determination of ref.\ [18].}
\label{fig:osetdel}
\end{figure}

\clearpage
%\begin{figure}[hbtp]
\PSfigure{lambda.fig }{15}
%\caption
{Role of various cutoff choices for $\Lambda_{\pi ,\rho}$
of eq.\ (\ref{eq:2p1himdel}). Here we
have chosen $k=100$ MeV and $l=1$ and $j=1/2$.
The solid line represents the results
with the parameters of the Bonn B potential, $\Lambda_{\pi}=1.2$ GeV
and $\Lambda_{\rho}=1.3$ GeV. The dashed line represent the results with
$\Lambda_{\rho}=2.5$ GeV and $\Lambda_{\pi}=1.2$, while the dotted line
is for $\Lambda_{\pi}=0.8$ GeV and $\Lambda_{\rho}=2.0$ GeV.}
%\label
{fig:test1}
%\end{figure}


\clearpage
%\begin{figure}[hbtp]
\PSfigure{k100.fig }{15}
%\caption
{Here
we compare the imaginary part arising from eq.\ (\ref{eq:2p1himdel})
for an incoming
nucleon with $l=1$, $j=1/2$  and momentum $k=100$ MeV, upper part,
and $k=200$ MeV, lower part. Two cases were
studied; (i) no finite self-energy of the isobars (solid line)
and (ii)
employing the parametrization of Oset and Salcedo of eq.\ (\ref{eq:oset})
for $\Sigma_{\Delta}$ (broken line).}
%\label
{fig:test2}
%\end{figure}

\end{document}

















      
      

      



      
      
      
      
      
      
      
      
      
      

      

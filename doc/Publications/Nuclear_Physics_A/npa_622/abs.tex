
    
\title{Nuclear and Neutron Matter Calculations with Different Model 
Spaces}

\author{L.\ Engvik, E.\ Osnes}
\address{Department of Physics, University of Oslo, N-0316 Oslo, 
Norway}
\author{M.\ Hjorth-Jensen}
\address{Nordita, Blegdamsvej 17, DK-2100 K\o benhavn \O, Denmark}
\author{T.T.S.\ Kuo}
\address{Department of Physics, State University of New York
at Stony Brook, NY 11794, USA }
\begin{keyword}
Nuclear Matter;
Many-body correlations
\end{keyword}
\maketitle

\begin{abstract}
In this work we investigate the so-called model-space 
Brueckner-Hartree-Fock 
(MBHF) approach for nuclear matter as well as for neutron matter 
and the extension of this which  
includes  the particle-particle and hole-hole (PPHH) diagrams.
A central ingredient in the model-space approach for nuclear matter
is the boundary momentum $k_M$ 
beyond which the single-particle potential energy is set equal to zero.
This is also the boundary of the model space within which 
the PPHH diagrams are calculated.
It has been rather uncertain which value  should be used for $k_M$.
We have carried out model-space nuclear matter and neutron matter 
calculations with and without PPHH diagrams for various 
choices of $k_M$
and using several modern nucleon-nucleon potentials.
Our results exhibit a saturation region where  the 
nuclear and neutron matter 
matter energies  are quite stable  as $k_M$ varies.
The location of this region may serve to determine an "optimum" choice
for $k_M$.
However, we find that the strength of the tensor force has a 
significant influence on binding energy variation with $k_M$.
The implications for nuclear and neutron matter calculations are 
discussed.
\end{abstract}
\end{frontmatter}



\title{Effective interactions and shell model studies of heavy tin isotopes}


\author[oslo]{A.\ Holt},
\author[oslo]{T.\ Engeland}, 
\author[nordita]{M.\ Hjorth-Jensen} and
\author[oslo]{E.\ Osnes}
\address[oslo]{Department of Physics,
         University of Oslo, N-0316 Oslo, Norway}
\address[nordita]{Nordita, Blegdamsvej 17, DK-2100 K\o benhavn \O, Denmark}

\maketitle

\begin{abstract}
We calculate the low-lying spectra of heavy tin isotopes from $A=120$ to 
$A=130$ using the $2s1d0g_{7/2}0h_{11/2}$ shell to define the model
space. An effective interaction has been derived using $^{132}$Sn as
closed core  employing perturbative many-body techniques.
We start from a nucleon--nucleon potential derived from modern meson exchange
models. This potential is in turn renormalized for the given medium, 
$^{132}$Sn, yielding the nuclear reaction matrix, which is then used in 
perturbation theory to obtain the shell model effective interaction.

\noindent {\it PACS:\ } 21.60.-n; 21.60.Cs; 24.10.Cn; 27.60.+j 

\noindent {\it Keywords:\ }  Shell model; Effective interactions

\end{abstract}

\end{frontmatter}


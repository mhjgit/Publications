\documentstyle[12pt]{article}
\setlength{\textheight}{23cm}
\begin{document}
\begin{center}
  {\bf REFEREE REPORT on \#2055 "Hyperon properties in finite nuclei 
  using realistic YN interactions"}\\
   (authors: I. Vidana, A. Polls, A. Ramos and M. Hjorth-Jensen )
\end{center}
\vspace{0.5cm} 
In this paper the authors summarize the theoretical study of 
single-particle energies of $\Lambda$ and $\Sigma$ in finite 
nuclei which have been calculated in the framework of the second 
order G-matrix formalism with the J\"ulich and Nijmegen YN potentials. 
They investigated the dependence of the results on the starting-energy 
and on the nuclear Fermi momentum, and showed the HF and 2p1h 
contributions to the hyperon s.p. energies separately. They found 
that the latter contribution stabilizes the results. 

On the basis of the careful reading, the referee remarks the following 
points which should be taken into account in improving the way of 
presentation:

\begin{enumerate}
\item 
The author followed their previous formalism which seems one of the 
acceptable standard ways of calculating the G-matrix up to the second 
order. The treatment of Eq.(18) in the nuclear matter does not cause 
any problem, while the most essential point in this paper is to take 
the finite nuclear closed-core into account in terms of the R-CM 
momentum representation. In this respect, the authors should describe 
more clearly how they perform the practical integrations over $k$ an 
$K$ in Eq.(16), in order to treat the finite Q-operator as faithfully 
as possible. If the procedure is OK, then another question arizes 
why the agreement in the $\Lambda$ s.p. energy becomes worse as going 
to the heavier systems where the difference between $(Q/e)_{FN}$ and 
$(Q/e)_{NM}$ should decreases. 

\item
The results for the $\Sigma$ s.p. energy is essentially new, although 
the calculated binding energies seem extremely unrealistic (too 
attractive) in view of the experimental indications. It is natural to 
stop the application to the other mass regions, but the readers like 
to know the essential reason for this overbinding. Of course, the 
authors mention the strong $\Sigma N$-$\Lambda N$ coupling nature 
involved in the YN potentials. The referee recommends them to add 
the calculated $\Sigma$ widths so that one can compare the results 
meaningfully with the other works published before. 

\item 
As for the calculated $\Lambda$ s.p. energies, the paper reports that 
the Nijmegen soft core model leads to too shallow binding, while the 
J\"ulich-B interaction gives the reasonable values with respect to the 
experimental ones. However such results have been already revealed 
in the other works done previously with slightly different 
approximations (e.g. Refs.40, 41, 42, etc). Moreover these works seem 
to have the similar quality in getting the theoretical s.p. energies, 
so that there is no specially better merit of the present treatment. 
This fact is not described clearly in the manuscript. 

\item 
The authors repeat criticism to the YNG prescription of Refs.37-40 for 
the $k_F$ parametrization and/or the averaged treatment such as the 
local density approximation (for example, the fourth sentence of 
sect. III-A, the fifth sentence on p.14, the fourth sentence in the  
second paragraph on p.14). If it were the case, then normal criticism 
should be welcome in general. In this paper, however, the referee could 
not find any better quality or better agreement with respect to the 
existing experiments. In spite of such criticism, on the other hand, 
the author agree that the YNG treatment was successful in getting a 
constraint on the nature of the YN potential (in the second sentence 
on p.16, for example). In the present situation of the 
experimental quality, it is not always easy to discriminate various 
theoretical treatments, so that one should avoid very crude 
expressions (such as ``...may not lead to...", and ``... cannot be 
achieved...") without showing any clear evidence. 

\item 
The sign convention of the binding energy $B_Y$ is confusing. 
The authors mostly use the negative value for a bound state, while 
in Fig. 3 they use the opposite sign. The referee recommends the 
authors to use the traditional sign convention to avoid unnecessary 
confusion for the readers.  

\item
The discussion on the comparison with the Woods-Saxon systematics 
seems lengthy. What is the main purpose of this systematics ? 
There are three sets of $\Lambda$ s.p. energies: the calculated ones, 
the Woods-Saxon parametrization and the experimental ones. 
Do the authors like to show that the calculated ones can (OR cannot)
be reproduced by the adjusted W-S ?  Where are the experimental values 
when they make such comparison ? 

\item 
In the headings of Tables I-IV, there is no description that the 
hyperon is in the $1s_{1/2}$ state.

\end{enumerate}

In conclusion, the manuscript should be revised by taking the above 
problems carefully into account before it is reconsidered for 
publication. Finally the referee lists the trivial typographical 
errors as follows:

\vspace{0.5cm}
\noindent 
--- p. 6, Eq.(5): $\xi_F$ should read $\xi_Y$. \\
--- p. 6, Eq.(6): it might be better to use $-B_Y(k_Y)$ instead 
  of $B_Y(k_Y)$, as in the traditional convention. \\
--- p. 11, line 7 of III-A: ``a effective" should read ``an effective".\\ 
--- p. 14, line 19: ``believe" should read ``belief". \\
--- p. 17, line 5: ``than" might be ``as" (?) \\
--- p. 30, the Fig. 3 caption: ``single-partkce" should read 
     ``...particle" \\

\end{document}






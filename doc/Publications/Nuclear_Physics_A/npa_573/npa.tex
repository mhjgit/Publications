\documentstyle[12pt,//selenium/user/mhjensen/text/style/mhj]{article}
\begin{document}
\setlength{\parindent}{0cm} 
\setlength{\parskip}{2.0ex} 
\setlength{\parsep}{0ex} 
\pagestyle{empty}
\uioENGhead%
{Morten Hjorth-Jensen}             % Senders name
{\today}                          % Date
{ }                               % Our ref.
{ }                               % Doc ref.
{Revised MS 1466}                               % Case ref.
\sadr%
{Nuclear Physics {\bf A}}                               % Adr 1
{}                               % Adr 2
{  }                               % Adr 5
{}
{}
{ }                               % Adr 6
\header{10mm}{             }{10mm}% Header
\uiologo                          % University logo
\baselineskiptj{6}                % Baselineskip in cm.
%           < Write LETTER here >

Dear Editors,

please find enclosed a revised version of the
manuscript --MS 1466-- entitled 
``Convergence properties of the effective interaction'', 
co-authored with Ellis, Engeland, Holt and Osnes.

As with the previous case,
I have enclosed an MS-DOS 3.5 inch diskette which contains the manuscript.
In this diskette, the manuscript file is named \newline
\begin{center}CONVERGE.TEX\end{center}
and it employs the Elsevier Latex style espart.sty. 
In addition to the manuscript,
the three figures of the manuscript, are included as postscript files. These
files are named
\begin{center}FIG1.PS\end{center}
\begin{center}FIG2.PS\end{center}
\begin{center}FIG3.PS\end{center}
Thus, there are four files in the diskette. 
I do not include hard copies of the figures, as I guess you already 
have them, since they followed the first manuscript.

I have also included 
two hard-copies of the manuscript (with figs as well), 
which could e.g.\ be used by the referee. Our changes, following the
suggestions of the referee, are marked with red in the manuscript.

Enclosed you will also find a letter to the referee, where we explain
our changes.
%           < End of LETTER here >
\greet{Yours sincerely}                  % Salutation 
{Morten Hjorth-Jensen}                    % Senders name
{ }         



\end{document}





\title{Modern nucleon-nucleon potentials and 
       symmetry energy in infinite matter}
 
\author[oslo]{L.\ Engvik}, 
\author[kbh]{M.\ Hjorth-Jensen}, 
\author[idaho]{R.\ Machleidt},  
\author[tuebingen]{H.\ M\"{u}ther},  
\author[barcelona]{A.\ Polls}
\address[oslo]{Department of Physics, University of Oslo, N-0316 Oslo, Norway}
\address[kbh]{Nordita, Blegdamsvej 17, DK-2100 K\o benhavn \O, Denmark}
\address[idaho]{Department of Physics, University of Idaho, Moscow, 
         ID 83844, U.S.A.}
\address[tuebingen]
         {Institut f\"ur Theoretische Physik, Universit\"at T\"ubingen,
         D-72076 T\"ubingen, Germany}
\address[barcelona]{Departament d'Estructura i Costituents de la Mat\`eria,
         Universitat de Barcelona, E-08028 Barcelona, Spain}

\maketitle

\begin{abstract}

We study the symmetry energy in infinite nuclear
matter employing a non-relativistic Brueckner-Hartree-Fock approach
and using various new nucleon-nucleon (NN) potentials, 
which fit np and pp scattering
data very accurately. The potential models we employ
are the recent
versions of the Nijmegen group, Nijm-I, Nijm-II and Reid93, 
the Argonne $V_{18}$ potential and the CD-Bonn potential.
All these potentials yield a symmetry energy which increases with density,
resolving a discrepancy that existed for older
NN potentials. The origin of remaining differences is
discussed.

\end{abstract}

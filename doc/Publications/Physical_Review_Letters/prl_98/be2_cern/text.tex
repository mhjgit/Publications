\documentclass[prc]{revtex4}

\begin{document}

We compare measured B(E2$\uparrow$) value in the
unstable $^{110}$Sn to the results of the two 
large-scale shell model calculations discussed in Ref.~\cite{banu}. 
These shell-model calculations were carried out by the Oslo and Strasbourg groups,
using effective interactions defined for two different cores, namely 
$^{90}$Zr and $^{100}$Sn, using the same nucleon-nucleon interaction
as starting point. For details on how to derive these effective
interactions, see for example Refs.~\cite{banu,hko95}.

The $^{90}$Zr case includes protons in the 
$1d_{5/2}$,
$0g_{7/2}$, $0g_{9/2}$, 
$1d_{3/2}$ and $2s_{1/2}$ single-particle orbits and neutrons
in the $1d_{5/2}$,
$0g_{7/2}$, $1d_{3/2}$, $2s_{1/2}$ and $0h_{11/2}$ single-particle orbits. 
For a $^{100}$Sn core, neutrons confined to the  
$1d_{5/2}$,
$0g_{7/2}$, $1d_{3/2}$, $2s_{1/2}$ and $0h_{11/2}$ single-particle orbits 
define the shell-model space. 
In the calculation of 
the B(E2) systematic, an effective neutron charge of $0.5e$ and proton charge $1.5e$ were used for the 
$^{90}$Zr core while an effective neutron  charge of $1.0e$ was used for the $^{100}$Sn case.
The results with a $^{90}$Zr core, taken from Ref.~\cite{banu},  are displayed in Fig.~3.

In the case of a $^{100}$Sn core, the experimental B(E2$\uparrow$)
values are reproduced with the above mentioned effective charge
for all isotopes from $^{116}$Sn to $^{130}$Sn. 
For the lighter nuclei, the theoretical results display the expected parabolic behavior,
see Ref.~\cite{banu} for further details.
These results are however at askance with the present experimental result for $^{110}$Sn and 
the recent experiments  from
Ref.~\cite{banu} on $^{108}$Sn ($0.230\pm 0.057$ e$^2$b$^2$) and Vaman {\em et al} \cite{vaman} 
for $^{106-112}$Sn. The latter results for the reduced 
B(E2;0$^+_{\text{g.s.}}$$\to$2$^{+}_1$) transitions are $0.240\pm 0.02$ e$^2$b$^2$,
$0.240\pm 0.02$ e$^2$b$^2$, $0.230\pm 0.03$ e$^2$b$^2$, and $0.240\pm 0.06$ e$^2$b$^2$ for
$^{112}$Sn, $^{110}$Sn, $^{108}$Sn, $^{106}$Sn, respectively.
To reproduce these experimental values one needs a larger effective charge. Furthermore, the experimental
values seem to indicate a deviation from a good seniority picture for the lighter tin isotopes,
with transition rates almost independent of the mass number $A$.
This indicates that
the effective charges for the lighter Sn isotopes show stronger
renormalization effects for the lighter nuclei, implying larger core polarization due to
particle-hole excitations, and a different character of core
excitations in the $N = Z$ and $N >> Z$ regions of the Sn isotopic
chain.

To further investigate the variation and intrinsic $ph$ structure
of the polarization charge in the pure neutron space, 
Ref.~\cite{banu} included also a calculation with $^{90}$Zr as core, displayed
in Fig.~3.  In these calculations, which include up to four-particle-four-hole
proton excitations (the $g_{9/2}$ orbit can at most have four holes, being the computational
limit of Ref.~\cite{banu}), 
one can reproduce the same trend as 
for the $^{100}$Sn core but, with an effective charge for neutrons of $0.5e$ and
protons of $1.5e$. These are 
the so-called non-renormalized charges, see for example Bohr and Mottelson \cite{bm69}.  

However, even with an increased model space, the enlarged calculations deviate from the new
experimental data for lighter tin isotopes, in particualr for 
$^{106,108}$Sn. This indicates that further core-polarization effects may be
needed and/or a better effective interaction.
The proton-neutron interaction plays here an essential role. In
particular the
$\pi(0g_{9/2})-\nu(0g_{7/2}1d_{5/2}1d_{3/2}2s_{1/2})$ monopoles,
responsible for the evolution of the spectroscopy between
$^{91}$Zr and $^{101}$Sn, govern the evolution of the proton
$Z=50$ gap with the neutron filling. These monopoles were fitted to reproduce
the experimental spectra of nuclei around $A~100$, leading to some ambiguity,
since in particular 
the $\pi (0g_{9/2}) - \nu (1h_{11/2})$ monopole reflects
experimental uncertainties in the $11/2^-$ states in the $N=51$
nuclei. A more in depth  theoretical analysis of these isotopes will be presented elsewhere 
\cite{joakim2007}.


\begin{thebibliography}{100}
\bibitem{hko95} M.~Hjorth-Jensen, T.~T.~S.~Kuo, and E.~Osnes,
Phys.~Rep:~{\bf 261}, 125 (1995).
\bibitem{vaman} C. Vaman {\em et al}, 
e-Print Archive: nucl-ex/0612011.
\bibitem{bm69} Aa.~Bohr and B.~Mottelson, {\em Nuclear Structure}, (W.~A.~Benjamin, New York, 1969), Vol.~1.
\bibitem{joakim2007} J.~Cederk\"all, T.~Engeland, A.~Ekstr\"om, and M.~Hjorth-Jensen, unpublished.
\end{thebibliography}
\end{document}

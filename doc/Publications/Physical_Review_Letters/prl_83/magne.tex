\documentstyle[preprint,aps,multicol]{revtex}

\draft
\begin{document}

\title{Observation of Thermodynamical Properties in the $^{162}$Dy,
$^{166}$Er and $^{172}$Yb Nuclei}


\author{E.~Melby, L.~Bergholt, M.~Guttormsen, M.~Hjorth-Jensen,
F.~Ingebretsen, S.~Messelt, J.~Rekstad, A.~Schiller, S.~Siem, and
S.W.~{\O}deg{\aa}rd}
\address{Department of Physics, University of Oslo,
Box 1048 Blindern, N-0316 Oslo, Norway}

\maketitle

\begin{abstract}
The density of accessible levels at low spin in the ($^3$He,$\alpha
\gamma$) reaction has been extracted for the $^{162}$Dy, $^{166}$Er and
$^{172}$Yb nuclei. The nuclear temperature is measured as a function of
excitation energy in the region of 0 -- 6 MeV. The temperature curves
reveal structures indicating the onset of new degrees of freedom. The heat
capacity of the nuclear system is
discussed within the framework of a canonical ensemble. \end{abstract}

\pacs{ PACS number(s): 21.10.Ma, 24.10.Pa, 25.55.Hp, 27.70.+q}

%\begin{multicols}{2}
A challenging goal in nuclear physics is to trace thermodynamical
quantities as functions of excitation energy. These quantities depend on
statistical properties in the nuclear many body system and may reveal phase
transitions. Unfortunately, it is difficult to investigate these aspects -
both experimentally and theoretically.

The density of levels as a function of excitation energy is the starting
point to extract quantities like entropy, temperature and heat capacity. In
the pioneer work of Bethe \cite{1} the level density was described within
the Fermi gas model using a partition function for the grand-canonical
ensemble. This picture of the nucleus as a gas of non-interacting fermions
confined to the nuclear volume has later been modified. The
phenomenological back-shifted Fermi gas model \cite{2} is a popular
extension of the model, which simulates shell and pair correlation effects.
This model works well at excitation energies where the level density is
high, typically above the neutron binding energy.

In addition to reveal statistical properties of nuclear matter, knowledge
of the level density is important in nuclear astrophysics. The level
density is essential for the understanding of the nucleosynthesis in stars,
where thousands of cross-sections have to be included in the calculations
\cite{3}. In parallel with these applications new theoretical approaches
are emerging. Recent microscopic model calculations \cite{3} include pair
correlations as well as shell effects. With the recent shell model Monte
Carlo method \cite{4,5,6,7} one is able to estimate level densities in
heavy nuclei up to high excitation energies.

The theoretical progress has so far not been followed by new experimental
data. The straightforward way to determine level densities is by counting
discrete levels. However, this technique is restricted to light nuclei
and/or low excitation energies where the experimental resolution is high
enough to resolve individual lines in the spectra. In rare earth nuclei,
the estimates obtained by counting levels are only valid up to $\sim$2 MeV
of excitation energy. A very useful experimental quantity is the average
level spacing observed in slow neutron resonance capture \cite{2}. From
these spacings rather accurate level densities for a certain spin value can
be determined at the neutron binding energy region. In addition, the level
density can be extracted from the shape of continuum particle spectra.
However, if pre-equilibrium particle emission takes place, this procedure
may be doubtful since high-energy particles reveal high temperatures and,
thus, too low level densities.

Recently, the Oslo group has presented a new way of extracting level
densities at low spin from measured $\gamma$-ray spectra \cite{8,9}. The
main advantage of this method is that the nuclear system is very likely
thermalized prior to the $\gamma$-ray emission. In addition, the method
allows the simultaneous extraction of level density and the
$\gamma$-strength function over a wide energy region. In this letter we
report for the first time on experimentally deduced temperatures and heat
capacities of rare earth nuclei in the 0 -- 6 MeV excitation energy region.

The experiments were carried out with 45 MeV $^3$He-projectiles at the Oslo
Cyclotron Laboratory (OCL). The experimental data are obtained with the
CACTUS multidetector array \cite{10} using the ($^3$He,$\alpha \gamma$)
reaction on $^{163}$Dy, $^{167}$Er and $^{173}$Yb self-supporting targets.
The charged ejectiles were detected with eight particle telescopes placed
at an angle of 45$^{\circ}$ relative to the beam direction. An array of 28
NaI $\gamma$-ray detectors with a total efficiency of $\sim$15\% surrounded
the target and particle detectors.

The assumptions behind the method of data analysis and techniques are
described in Refs.~\cite{8,9} and only a few comments pertinent to the
present work are made here. The experimental level density is deduced from
$\gamma$-ray spectra recorded at a number of initial excitation energies
$E$, determined by the measured $\alpha$-energy. These data are the basis
for making the first-generation (or primary) $\gamma$-ray matrix, which is
factorized according to the Brink-Axel hypothesis \cite{11,12} as
\begin{equation}
P(E,E_{\gamma}) = \sigma (E_{\gamma}) \rho (E -E_{\gamma}).
\end{equation}
>From this expression the $\gamma$-ray energy dependent function $\sigma$ as
well as the level density $\rho$ is deduced by an iteration procedure [9],
using the same 0th-order trial functions as described in Ref.~[9], and the
average slopes and the absolute density depend on this starting point. The
extraction technique takes the advantage that when $\sigma$ is known,
$\rho$ can be calculated from Eq.~(1), and visa versa. With proper trial
functions the converging $\sigma$ and $\rho$ functions can be calculated in
every second iteration step. For each iteration the $\rho$ ($\sigma$)
function is averaged over all $E_{\gamma}$ ($E$). In the following we
concentrate only on the level density and its fine structure, which is
assumed to be independent of particular $\gamma$-ray decay routes.

The extracted level densities for the $^{162}$Dy, $^{166}$Er and $^{172}$Yb
nuclei are shown as data points in Fig.~1. The data for $^{162}$Dy and
$^{172}$Yb deviate slightly from the results previously published~\cite{9}.
In the present work the extraction procedure has been refined by omitting
data with $E_{\gamma} <$ 1 MeV, where the first-generation $\gamma$-spectra
exhibit methodical uncertainties. By excluding these data points, the error
bars for the resulting level density are reduced.

The level density $\rho(E)$ is proportional to the number of states
accessible to the nuclear system at excitation energy $E$. Thus, the
entropy in the microcanonical ensemble is given by \begin{equation}
S(E)=S_0 + \ln \rho(E),
\end{equation}
where, for convenience, the Boltzmann constant is set to unity ($k_B$ =1).
The normalization constant $S_0$ is not important in this discussion, since
it vanishes in the evaluation of the temperature \begin{equation}
T(E)=\frac{1}{(\partial S/\partial E)_V}. \end{equation}

Small statistical deviations in the entropy $S$ may give rise to large
contributions in the temperature $T$. In order to reduce this sensitivity,
the differentiation of $S$ is performed by a least square fit of a straight
line to five adjacent data points at a time. The slope of the straight line
is taken as the differential of $S$ at that energy. Thus, an effective
smoothing of about 0.5 MeV is performed through this procedure. Since the
energy particle resolution is around 0.3 MeV, this differentiation
procedure will not significantly reduce the potential experimental
information.

The temperatures deduced are shown as discrete data points in Fig.~2. The
data reveal several broad structures in the 1 -- 5 MeV region, which are
not explained in a Fermi gas description. The most pronounced bumps are
located at 1.8 MeV and 3.2 MeV in $^{162}$Dy, at 2.5 MeV and 3.7 MeV in
$^{166}$Er and at 1.8 MeV and 3.0 MeV in $^{172}$Yb. These structures are
interpreted as being the breaking of nucleonpairs and, at higher energies,
the possible quenching of pair correlations. Similar behaviour has recently
been described for $^{192}$Hg within the BCS model \cite{13}, where kinks
in the level density curve is interpreted as the onset of different
quasiparticle excitations. This corresponds to a decrease in temperature
for $^{192}$Hg at 1.4 and 3 MeV of excitation energy.

The extraction of specific heat capacity is given by \begin{equation}
C_V(E) =\frac{1}{(\partial T/\partial E)_V}, \end{equation}
which effectively means that the entropy $S(E)$ has to be differentiated
twice. From the scattered data points of Fig.~2, one immediately sees that
this will not give meaningful information for $C_V$ with the present
experimental statistics.

At this point one could introduce a strong smoothing of the temperature
curve in order to extract $C_V$. However, it seems rather incidental how
this should be done. One way to proceed is to introduce the canonical
ensemble in the description of the nuclear system. Since the temperature
enters as a fixed parameter in this formalism, the statistical uncertainty
of $T$ does not introduce any additional fluctuations. The cost of this
nice feature is that one has to make an average over a wide excitation
energy region.

The partition function in the canonical ensemble \begin{equation}
Z(T)=\sum_{n=0}^{\infty}\rho (E_n)e^{-E_n/T} \end{equation}
is determined by the multiplicity of states at a certain energy.
Experimentally, the multiplicity corresponds to the level density of
accessible states, $\rho(E_n)$, in the present nuclear reaction at energy
bin $E_n$. The widths of the energy bins are 120, 120 and 96 keV for the
$^{162}$Dy, $^{166}$Er and $^{172}$Yb nuclei, respectively.

The mathematical justification of Eq.~(5) is that the sum is performed from
zero to infinity. The experimental information on $\rho$ covers the
excitation region 0 -- 7 MeV, only. For the three nuclei investigated, the
proton and neutron binding energies are $\sim$8 MeV. For excitation
energies above this binding energy it is reasonable to assume Fermi gas
properties, since single particles may be excited into a region with a very
high level density (continuum). Therefore, due to lack of experimental data
above 7 MeV, the level density is extrapolated from 7 MeV to higher
energies by the Fermi gas model expression \cite{14} \begin{equation}
\rho_{\scriptscriptstyle {FG}}(E)=CE^{-5/4}e^{2 \sqrt{aE}}, \end{equation}
where $C$ is a normalization factor and $a$ is the level density parameter.
The best fits of $\rho_{\scriptscriptstyle {FG}}$ to data in the 3.5 -- 6.0
MeV excitation region give a level density parameter of $a$ =17.8, 19.0,
and 18.7 MeV$^{-1}$ for the $^{162}$Dy, $^{166}$Er and $^{172}$Y nuclei,
respectively. These values are in agreement with data from slow neutron
resonances \cite{2} and the semiempirical formula for $a$ with values in
between $A/10$ and $A/8$ MeV$^{-1}$. The theoretical level density
functions are displayed as solid lines in Fig.~1 and are drawn from 3.5 MeV
in order to visualize the fits to data.


In order to determine how far the sum of Eq.~(5) has to be performed, the
probability density function
\begin{equation}
p (E)=\frac{\rho_{\scriptscriptstyle {FG}} (E)e^{-E/T}}
{\int{\rho_{\scriptscriptstyle {FG}}
(E)e^{-E/T}dE}}
\end{equation}
is shown in Fig.~3 for three typical temperatures. At low temperatures, say
$T$ = 0.3 MeV, the nucleons are scattered no higher than $E$ =5 MeV.
However, this upper limit increases very rapidly for higher temperatures.
For the typical temperatures studied in this letter, $E$ =30 MeV has been
tested and found to be a sufficiently high upper limit for the summing in
Eq.~(5).

The excitation energy in the canonical
ensemble is given by the thermal average \begin{equation}
<E(T)>=Z^{-1}\sum_{n=0}^{\infty}E_n\rho (E_n)e^{-E_n/T}, \end{equation}
where $\rho$ is the level density (shown in Fig.~1) composed of an
experimental and a theoretical part for the excitation regions below and
above $\sim$7 MeV, respectively.

The smoothing effect implied by the canonical ensemble can be investigated
by calculating the standard deviation $\sigma_{\scriptscriptstyle E}$ for
the thermal average of the energy
\begin{equation}
\sigma_{\scriptscriptstyle E}= \sqrt{<E^2>-<E>^2}, \end{equation}
giving e.g. $\sigma_{\scriptscriptstyle E}$ = 3 MeV at $E$ =7 MeV. This
again shows that the energy in the canonical ensemble is strongly smoothed
for a given temperature. Thus, one cannot expect to find abrupt changes in
the thermodynamical quantities.

>From Eq.~(8) the temperature in the canonical ensemble can be studied as a
function of $<E>$. In Fig.~2 the canonical ensemble gives a temperature
dependence (solid lines) that coincides well with the average values found
in the microcanonical ensemble (data points).

This gratifying behaviour of the canonical temperature encourages us to use
the canonical ensemble to estimate the heat capacity $C_V$ as well. The
heat capacity can be deduced by simply calculating the increase in the
thermal average of the energy $<E>$ with respect to $T$ \begin{equation}
C_V(T)=\frac{\partial <E>}{\partial T}.
\end{equation}

The deduced heat capacities for the $^{162}$Dy, $^{166}$Er and $^{172}$Yb
nuclei as functions of $<E(T)>$ are shown in Fig.~4. All nuclei display
similar dependencies, reflecting that the thermal averages of the
excitation energies smear out structures seen in the experimental level
densities. The heat capacities are mainly following the theoretical values
obtained by using Eq.~(6) with the proper values of $a$ deduced from the
fits of Fig.~1. For comparison, we also show the simplified expression
$C_V=2\sqrt{aE}$ assuming $\rho \sim \exp (2 \sqrt{aE})$ with $a =18.5$
MeV$^{-1}$. The canonical heat capacity shows no traces of the fine
structures found in the level density below $\sim$4 MeV of excitation
energies, as also was the case for the canonical temperature.

In conclusion, for the first time temperature and heat capacity based on
$\gamma$-ray spectra have been extracted for rare earth nuclei. The
extracted temperature curves deduced for the microcanonical ensemble reveal
structures in the 1 -- 5 MeV excitation region, which are tentatively
interpreted as the breaking of nucleon pairs and quenching of the pair
correlations. The heat capacity could only be extracted by the use of the
canonical ensemble. These semiexperimental values, which are not expected
to show any fine structures, are in agreement with the Fermi gas
predictions. It would be very interesting to see if more realistic
theoretical calculations can describe the fine structures observed in the
microcanonical ensemble.

The authors are grateful to E.A.~Olsen and J.~Wikne for providing the
excellent experimental conditions. We wish to acknowledge the support from
the Norwegian Research Council (NFR).

\begin{references}
\bibitem{1} H.~A.~Bethe, Phys.~Rev.~{\bf 50}, 332 (1936).
\bibitem{2} A.~Gilbert and A.G.W.~Cameron, Can. J. Phys. {\bf 43}, 1446
(1965). \bibitem{3} S.~Goriely, Nucl.~Phys.~{\bf A605}, 28 (1996).
\bibitem{4} G.~H.~Lang, C.~W.~Johnson, S.~E.~Koonin and W.~E.~Ormand,
Phys.~Rev.~C {\bf 48}, 1518 (1993); S.~E.~Koonin, D.~J.~Dean, and
K.~Langanke, Phys.~Rept.~{\bf 278}, 2 (1997).
\bibitem{5} H.~Nakada and Y.~Alhassid, Phys. Rev. Lett. {\bf 79}, 2939
(1997). \bibitem{6} W.~E.~Ormand, Phys. Rev. {\bf C} 56, R1678 (1997).
\bibitem{7} J.~A.~White, S.~E.~Koonin and D.~J.~Dean, preprint
nucl-th/9812044. \bibitem{8} L.~Henden, L.~Bergholt, M.~Guttormsen,
J.~Rekstad and T.~S.~Tveter, Nucl.~Phys. {\bf A589}, 249 (1995).
\bibitem{9} T.~S.~Tveter, L.~Berholt, M.~Guttormsen, E.~Melby and J.~Rekstad,
Phys.~Rev.~Lett. {\bf 77}, 2404 (1996).
\bibitem{10} M.~Guttormsen, A.~Atac, G.~L{\o}vh{\o}iden, S.~Messelt,
T.~Rams{\o}y, J.~Rekstad, T.~F.~Thorsteinsen, T.~S.~Tveter and Z.~Zelazny,
Phys.~Scripta {\bf T32}, 54 (1990).
\bibitem{11} D.~M.~Brink, Doctorial Thesis, Oxford University (1955).
\bibitem{12} P.~Axel, Phys. Rev. {\bf 126}, 671 (1962).
\bibitem{13} T.~D{\o}ssing {\sl et al.}, Phys.~Rev.~Lett. {\bf 75}, 1276
(1995).
\bibitem{14} A.~Bohr and B.~Mottelson, {\em Nuclear Structure}, (Benjamin,
New York, 1969), Vol. I, p. 289. \end{references}

%\end{multicols}

\begin{figure}
%\includegraphics[totalheight=17.5cm,angle=0,bb=0 80 350 730]{rho.ps}
\caption{ Extracted level density (points) for $^{162}$Dy, $^{166}$Er and
$^{172}$Yb. The error bars show the statistical uncertainties. The solid
lines are extrapolations based on the Fermi gas model. The curves are in
arbitrary units and separated with a factor of $\sim$10 for better
visualization.} \end{figure}

\begin{figure}
%\includegraphics[totalheight=19cm,angle=0,bb=0 20 350 730]{temp.ps}
\caption{Observed temperatures as functions of the excitation energy $E$
(data points with statistical error bars). The solid lines are temperatures
as functions of average excitation energies $<E>$ deduced within the
canonical ensemble.}
\end{figure}

\begin{figure}
%\includegraphics[totalheight=17.5cm,angle=0,bb=0 80 350 730]{Z.ps}
\caption{The Fermi gas model energy distribution $p(E)$ of the canonical
ensemble.}
\end{figure}

\begin{figure}
%\includegraphics[totalheight=17.5cm,angle=0,bb=0 80 350 730]{Cv.ps} %
\setlength{\unitlength}{1mm}
 %\begin{picture}(140,180)
 %\put(-30,0){\epsfbox{new.ps}}
 %\end{picture}
\caption{The heat capacities $C_V$ extracted within the canonical ensemble.
The dashed curve displays the simplified Fermi gas expression for $a$ =18.5
MeV$^{-1}$.}
\end{figure}

\end{document}


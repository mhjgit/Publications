\documentclass[twoside,12pt]{article}
\usepackage{epsfig}
\usepackage{pst-plot}


\def\Journal#1#2#3#4{{#1} {#2} (#4) #3 }
\def\NCA{{\em Nuovo Cimento} A}
\def\PHYS{{\em Physica}}
\def\NPA{{\em Nucl. Phys.} A}
\def\MATH{{\em J. Math. Phys.}}
\def\PRO{{\em Prog. Theor. Phys.}}
\def\NPB{{\em Nucl. Phys.} B}
\def\PLA{{\em Phys. Lett.} A}
\def\PLB{{\em Phys. Lett.} B}
\def\PLD{{\em Phys. Lett.} D}
\def\PL{{\em Phys. Lett.}}
\def\PRL{\em Phys. Rev. Lett.}
\def\PREV{\em Phys. Rev.}
\def\PREP{\em Phys. Rep.}
\def\PRA{{\em Phys. Rev.} A}
\def\PRD{{\em Phys. Rev.} D}
\def\PRC{{\em Phys. Rev.} C}
\def\PRB{{\em Phys. Rev.} B}
\def\ZPC{{\em Z. Phys.} C}
\def\ZPA{{\em Z. Phys.} A}
\def\ANNP{\em Ann. Phys. (N.Y.)}
\def\RMP{{\em Rev. Mod. Phys.}}
\def\CHEM{{\em J. Chem. Phys.}}
\def\INT{{\em Int. J. Mod. Phys.} E}
\def\r{\vec r}
\def\R{\vec R}
\def\p{\vec p}
\def\P{\vec P}
\def\q{\vec q}
\def\ss{\mbox{\boldmath $\sigma$}}
\newcommand{\be}{\begin{equation}}
\newcommand{\ee}{\end{equation}}
\newcommand{\bea}{\begin{eqnarray}}
\newcommand{\eea}{\end{eqnarray}}
\newcommand{\nn}{\nonumber}
\newcommand{\bra}[1]{\left\langle #1 \right|}
\newcommand{\ket}[1]{\left| #1 \right\rangle}
\topmargin-2.8cm
\oddsidemargin-1cm
\evensidemargin-1cm
\textwidth18.5cm
\textheight25.0cm
\begin{document}

\title{Effective interactions and the nuclear shell-model}
\author{D.~J.~Dean,$^{1,2}$ T.~Engeland,$^{2,3}$ M.~Hjorth-Jensen,$^{2,3}$\\
M.~P.~Kartamyshev$^{2,3}$, and E.~Osnes$^{2,3}$\\
\\
$^{1}$Physics Division, Oak Ridge National Laboratory,
P.O. Box 2008,\\  Oak Ridge, TN 37831-6373, U.S.A.\\
$^{2}$Center of Mathematics for Applications, University of Oslo, Norway\\
$^{3}$Department of Physics, University of Oslo, Norway}

\maketitle


\begin{figure}
\setlength{\unitlength}{1cm}
\begin{center}

\Cartesian(1cm,0.9cm)
%
\pspicture(0,1)(12,12)
%
\psframe[linewidth=0.0pt,fillstyle=solid,fillcolor=gray](6,1.5)(12,3)
%
\psframe*[linecolor=white](7,2)(11,2.6)
%
\uput[0](6.9,2.3){ Q--space: $N \leq 50$}
%
%	single-particle spectrum
%
\psline[linewidth=1pt](8,3.5)(10,3.5)
\uput[0](6.7,3.3){$d_{5/2}^{+}$}
\uput[0](10.1,3.3){\small 0.00 MeV}
%
\psline[linewidth=1pt](8,5.65)(10,5.65)
\uput[0](6.7,5.5){$s_{1/2}^{+}$}
\uput[0](10.1,5.5){\small 1.26 MeV}
%
\psline[linewidth=1pt](8,8.17)(10,8.17)
\uput[0](6.7,7.8){$d_{5/2}^{+}$}
\uput[0](10.1,7.9){\small 2.23 MeV}
%
\psline[linewidth=1pt](8,8.36)(10,8.36)
\uput[0](6.7,8.6){$g_{7/2}^{+}$}
\uput[0](10.1,8.5){\small 2.63 MeV}
%
\psline[linewidth=1pt](8,9.6)(10,9.6)
\uput[0](6.7,9.6){$h_{11/2}^{-}$}
\uput[0](10.1,9.6){\small 3.5 MeV}
%
%
\uput[0](8.0,6.0){ P--space for particles}


\psframe[linewidth=0.0pt,fillstyle=solid,fillcolor=lightgray](6,10)(12,11.5)
%
\psframe*[linecolor=white](7,10.4)(11,11)
%
\uput[0](6.9,10.7){ Q--space: $N > 82$}
%


%
\psframe[linewidth=0.0pt,fillstyle=solid,fillcolor=gray](0,1.5)(6,3)
%
\psframe*[linecolor=white](1,2)(5.8,2.6)
%
\uput[0](0.9,2.3){ Q--space: $Z < 38 $}
%
%	single-particle spectrum
%
\psline[linewidth=1pt](2,3.5)(4,3.5)
\uput[0](0.7,3.3){$p_{1/2}^{-}$}
\uput[0](4.1,3.3){\small 0.00 MeV}
%
\psline[linewidth=1pt](2,8.17)(4,8.17)
\uput[0](0.7,7.8){$g_{9/2}^{+}$}
\uput[0](4.1,7.9){\small  0.9 MeV}
%
\uput[0](1.0,6.0){ P--space for particles}




%%%%%%%%%%%%%%
\psframe[linewidth=0.0pt,fillstyle=solid,fillcolor=lightgray](0,10)(6,11.5)
%
\psframe*[linecolor=white](1,10.4)(5.8,11)
%
\uput[0](0.9,10.7){ Q--space: $Z > 50$}
\endpspicture
\caption{Possible shell-model space for nuclei around $A\sim 100$ using 
$^{88}$Sr as closed-shell core with the neutron orbitals $2s_{1/2}$, 
$1d_{5/2}$, $1d_{3/2}$, $0g_{7/2}$ and $0h_{11/2}$ and 
proton single-particle orbitals  $0g_{9/2}$ and $1p_{1/2}$ defining the model space.
Their respective single-particle energies are displayed as well. \label{fig:sr88ms}}
\end{center}
\end{figure}







\begin{figure}
\setlength{\unitlength}{1cm}
\begin{center}

\Cartesian(1cm,0.9cm)
%
\pspicture(0,1)(12,12)
%
\newgray{whitegray}{.9}

\psframe[linewidth=0.0pt,fillstyle=solid,fillcolor=gray](0,1.5)(6,3)
%
\psframe*[linecolor=white](1,2)(5,2.6)
%
\uput[0](1.0,2.3){ Q--space: $Z \leq 50$}
%
\psframe[linewidth=0.0pt,fillstyle=solid,fillcolor=gray](6,1.5)(12,10)
%
\psframe*[linecolor=whitegray](6.5,3)(11.7,9.8)
%
\psframe*[linecolor=white](7,2)(11,2.6)
%
\uput[0](7.0,2.3){ Q--space: $N \leq 82$}
%
%	single-particle spectrum
%
\psline[linewidth=1pt](7.7,3.5)(9.45,3.5)
\uput[0](6.45,3.40){$d_{3/2}^{+}$}
\uput[0](9.5,3.45){\small 0.00 MeV}
%
\psline[linewidth=1pt](7.7,4.09)(9.45,4.09)
\uput[0](6.45,4.05){$h_{11/2}^{-}$}
\uput[0](9.5,4.05){\small 0.24 MeV}
%
\psline[linewidth=1pt](7.7,4.32)(9.45,4.32)
\uput[0](6.45,4.70){$s_{1/2}^{+}$}
\uput[0](9.5,4.42){\small 0.33 MeV}
%
\psline[linewidth=1pt](7.7,7.6)(9.45,7.6)
\uput[0](6.45,7.6){$d_{5/2}^{+}$}
\uput[0](9.5,7.6){\small 1.66 MeV}
%
\psline[linewidth=1pt](7.7,9.5)(9.45,9.5)
\uput[0](6.45,9.4){$g_{7/2}^{+}$}
\uput[0](9.5,9.4){\small 2.43 MeV}
%
%
\uput[0](7,6.5){ P--space for holes }


%%%%%%%%%%%%%%
\psframe[linewidth=0.0pt,fillstyle=solid,fillcolor=lightgray](0,3)(6,11.5)
%
\psframe*[linecolor=white](1,8)(5,8.6)
%
\uput[0](1.0,8.3){ Q--space: $Z > 50$}
%

\psframe[linewidth=0.0pt,fillstyle=solid,fillcolor=lightgray](6,10)(12,11.5)
%
\psframe*[linecolor=white](7,10.4)(11,11)
%
\uput[0](7.0,10.7){ Q--space: $N > 82$}
%
\endpspicture
\caption{Shell-model space for tin isotopes for $100 \le A \le  132$. The valence neutrons are hole states.\label{fig:sn132sm}}
\end{center}
\end{figure}

\end{document}





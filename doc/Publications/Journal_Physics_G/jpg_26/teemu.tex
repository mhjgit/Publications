
\documentclass[12pt]{iopart}
\jl{4}
\begin{document}

\title{Shell-Model Effective Operators for Muon Capture in $^{20}$Ne}

\author{T Siiskonen\dag, J Suhonen\dag and M Hjorth-Jensen\ddag}

\address{\dag Department of Physics, University of Jyv\"askyl\"a, FIN-40351
Jyv\"askyl\"a, Finland}

\address{\ddag Department of Physics, University of Oslo, N-0316 Oslo, Norway}

\begin{abstract}
It has been proposed that
the discrepancy between the partially-conserved axial-current prediction
and the nuclear shell-model calculations of the ratio $C_{\rm P}/C_{\rm A}$
in the muon-capture reactions can be solved in the case of $^{28}$Si by
introducing effective transition operators. Recently there has been
experimental interest in measuring the needed angular correlations also in
$^{20}$Ne. Inspired by this, we have performed a shell-model analysis employing
effective transition operators in the shell-model formalism
for the transition $^{20}{\rm Ne}(0^+_{\rm g.s.})+\mu^-
\to {}^{20}{\rm F}(1^+;\, 1.057\, {\rm MeV})+\nu_\mu$. Comparison of the
calculated capture rates with existing data supports the
use of effective transition operators. Based on our calculations, as soon as
the experimental
anisotropy data becomes available, the limits for the ratio $C_{\rm P}/
C_{\rm A}$ can be extracted.
\end{abstract}

%\begin{keyword}Shell model; Muon capture; Effective operators
%\end{keyword}
\pacs{23.40.Bw, 23.40.Hc, 21.60.Cs}


\submitted

\maketitle

The large energy release in the ordinary (non-radiative) capture of stopped
negative muons by atomic nuclei probes the hadronic current much deeper than
ordinary beta decay or electron capture. In particular, the role
of the induced pseudoscalar coupling $C_{\rm P}$ becomes important in muon
capture. 
Based on this, there have been many attempts in the past to extract the ratio
of the induced pseudoscalar
and axial-vector coupling constants, $C_{\rm P}/C_{\rm A}$, from measured
capture rates (see e.g.\ \cite{mor60,cie76,gmi90,kuz94,joh96,sii98}) as well as
from angular correlations of the gamma emission
following the capture reaction ${}^A_Z{\rm X}^{}_N+\mu^-\to{}^{\phantom{AA}A}
_{Z-1}{\rm X}_{N+1}^{'*}+\nu_\mu$ of polarized muons (see e.g.\
\cite{mof97,bru95,sii99}).

The
angular correlation data, available for muon capture in $^{28}$Si, has been
in a key role in pointing out discrepancies in the shell-model calculations
of $C_{\rm P}/C_{\rm A}$. In various shell-model  calculations (see e.g.\ \cite{mof97,sii99}
and references therein) anomalously small values of this ratio 
($C_{\rm P}/C_{\rm A}\sim 0$) have been obtained. 
In \cite{sii99} we have proposed a method
which,
at least partly, lifts this discrepancy. This method is based on the use of
effective transition operators in the shell-model formalism. Unfortunately, the
anisotropy data is available only for $^{28}$Si and thus further testing of the
effective-operator method has to be done in the context of measured capture
rates or future experiments on angular correlations in the capture of polarized
muons. At this point it is worth pointing out that the matrix elements of muon
capture
can also be applied to various other problems, one of the most interesting
being the search of the scalar coupling of the hadronic current \cite{ego88}.

A new measurement of the correlation coefficients of $\gamma$-radiation
following the capture reaction in $^{28}$Si has been reported recently \cite{bru99},
confirming the earlier results of Refs.\ \cite{mof97} and \cite{bru95}.
So far $^{28}$Si has
been the only nucleus where this angular correlation data exists. In these
experiments the parameter
        \begin{equation}
        \label{x}
        x\equiv M_1(2)/M_1(-1)
        \end{equation}
can be extracted via the scaling coefficient $\alpha$ of the angular correlation
between the emitted $\gamma$-radiation and the muon neutrino. This scaling
coefficient is related to $x$ as \cite{bru95}
        \begin{equation}
        \label{alpha}
        \alpha\equiv\frac{\sqrt2 x-x^2/2}{1+x^2}.
        \end{equation}
The quantities $M_1(-1)$ and $M_1(2)$ are linear combinations of reduced nuclear
matrix elements and are given by
        \begin{equation}\label{M1}\fl
        M_1(-1)=\sqrt{2\over3}\left\{\left({1\over3}G_{\rm P}-G_{\rm A}
        \right)[101]+G_{\rm P}{\sqrt2\over3}[121]{C_{\rm A}
        \over M}[011p]+{C_{\rm V}\over M}\sqrt{2\over3}[111p]\right\},
        \end{equation}
        \begin{equation}\label{M2}\fl
        M_1(2)=\sqrt{2\over3}\left\{\left(G_{\rm A}-{2\over3}G_{\rm P}\right)
        [121]-G_{\rm P}{\sqrt2\over3}[101]+
        {C_{\rm A}\over M}\sqrt2
        [011p]+{C_{\rm V}\over M}\sqrt{2\over3}[111p]\right\},
        \end{equation}
where $M$ is the nucleon mass. The definitions of the reduced nuclear matrix
elements $[\dots ]$ can be found e.g.\ from \cite{sii98}.
%$\int U_{J_fM_f}^\dagger\sum_{s=1}^A{\rm e}^{-\alpha Zm_\mu'r_s}\Psi_s
%\tau_-^sU_{J_iM_i}d{\bf r}_1\cdots d{\bf r}_A={\cal M}[k\,w\,u\,(
%{\pm\atop p})](J_i\,M_i\,u\,M_f-M_i|J_f\,M_f)$
%are listed in Table \ref{tab:operators}.
%The functions $\cal Y$ are given by
%${\cal Y}^{\mu'-\mu}_{0\nu u}(\Omega)={1\over\sqrt{4\pi}}Y_{\nu,
%\mu'-\mu}(\Omega)$, ${\cal Y}^{\mu'-\mu}_{1\nu u}(\Omega,\sigma)=
%\sum_m(1\ -m\ \nu\ m-\mu+\mu'|u\ \mu'-\mu)Y_{\nu,m+\mu'-\mu}(\Omega){\cal Y}_{1,-m}(\sigma)$,
%where ${\cal Y}_{1m}(\sigma)=\sqrt{3\over4\pi}\sigma_m$.
%The functions $Y$ are the spherical harmonics, and $j_w(qr_s)$
%are spherical Bessel functions
%\begin{table}
%\caption{Definition of reduced matrix elements for muon capture. See text for more details.}
%\begin{tabular}{ll}
%        Matrix element & $\Psi_s$\cr
%        \hline
%        $[0wu]$ & $j_w(qr_s){\cal Y}_{0wu}^{M_f-M_i}(\hat r_s)\delta_{wu}$\cr
%        $[1wu]$ & $j_w(qr_s){\cal Y}_{1wu}^{M_f-M_i}(\hat r_s,\sigma_s)$\cr
%        $[0wu\pm]$ & $\left[j_w(qr_s)\pm\alpha Z(m_\mu'/p_\nu
%           )j_{w\mp1}(qr_s)\right]{\cal Y}_{0wu}^{M_f-M_i}
 %          (\hat r_s)\delta_{wu}$\cr
%        $[1wu\pm]$ & $\left[j_w(qr_s)\pm\alpha Z(m_\mu'/p_\nu
%           )j_{w\mp1}(qr_s)\right]{\cal Y}_{1wu}^{M_f-M_i}
%           (\hat r_s,\sigma_s)$\cr
%        $[0wup]$ & $ij_w(qr_s){\cal Y}_{0wu}^{M_f-M_i}(\hat
%           r_s)\sigma_s\cdot{\bf p}_s\delta_{wu}$\cr
%        $[1wup]$ & $ij_w(qr_s){\cal Y}_{1wu}^{M_f-M_i}(\hat r_s,{\bf p}_s)$\cr\hline
%\end{tabular}
%\label{tab:operators}
%\end{table}
The constants
$G_{\rm P}$ and $G_{\rm A}$ are related to the weak-interaction coupling
constants as
        \begin{eqnarray}
        G_{\rm P}&=&(C_{\rm P}-C_{\rm A}-C_{\rm V}-C_{\rm M}){E_\nu\over2M},\\
        G_{\rm A}&=&C_{\rm A}-(C_{\rm V}+C_{\rm M}){E_\nu\over2M}.
        \end{eqnarray}

Using the expressions of Eqs.\ (\ref{M1}) and (\ref{M2}), combined with Eq.\
(\ref{x}), the value
of $C_{\rm P}/C_{\rm A}$ can be extracted if the experimental value of $x$ is
known. However, calculations with different nuclear models give very different
predictions for this ratio. In Ref.\ \cite{bru95} the values
$C_{\rm P}/C_{\rm A}=3.4\pm1.0$ and $C_{\rm P}/C_{\rm A}=2.0\pm1.6$ were
extracted using the matrix elements of Refs.\ \cite{cie76} and \cite{par81},
respectively. In addition, the measurement of Ref.\ \cite{mof97} gives the
estimates $C_{\rm P}/C_{\rm A}=5.3\pm2.0$ using the matrix elements of
\cite{cie76} and $C_{\rm P}/C_{\rm A}
=4.2\pm2.5$ using the matrix elements of \cite{par81}.

The more realistic matrix elements, obtained
from the full 1s0d shell calculation utilizing Wildenthal's USD
interaction \cite{wil84}, yield the value of $C_{\rm P}/C_{\rm A}=0.0\pm3.2$
\cite{mof97,jun96}, far from the value $C_{\rm P}/C_{\rm A}\approx7$ given by
the nuclear-model independent Goldberger-Treiman relation (see e.g.\
\cite{com83})
obtained by using the partially-conserved axial-current hypothesis (PCAC).
The estimate given by the shell-model matrix elements is very surprising,
since the
USD interaction is fitted to a selected set of the 1s0d-shell spectroscopic
data, reproducing various spectroscopic quantities like
energy spectra, Gamow-Teller decay properties and electromagnetic properties
(see e.g.\ \cite{car86,bro87}) very well.

This anomaly, present in the shell-model calculations of $^{28}{\rm Si}(0^+
_{\rm g.s.})+\mu^-\to{}^{28}{\rm Al}(1^+_3)+\nu_\mu$,
can be, at least partly, avoided by using renormalized one-body
transition operators in the context of the shell model. In the work of
\cite{sii99} the USD and effective interactions based on the 
recent CD-Bonn \cite{mac96} and 
Nijmegen \cite{nim94} nucleon-nucleon (NN) interaction models, yielded the interval
$0.4\le C_{\rm P}/C_{\rm A}\le 2.7$, whereas with the renormalized transition
operators the interval $3.4\le C_{\rm P}/C_{\rm A}\le 5.4$ was obtained, closer
to the PCAC-prediction of this ratio. 
This value agrees also with the recent analysis of Brudanin et al.\ \cite{bru99}.
Moreover, of special interest are the recent plans for the angular-correlation
measurements following the capture reaction ${}_{10}^{20}{\rm Ne}(0^+_{\rm
g.s.})+\mu^- \to{}_{\phantom{0}9}^{20}{\rm F}(1^+_1)+\nu_\mu$, as  announced
in \cite{bru99}. In
the present Letter we investigate this particular reaction in the shell-model
framework with and without effective operators, and give predictions for
the ratio $C_{\rm P}/C_{\rm A}$ using
different sets of two-body interactions. The needed muon-capture formalism is
treated in great detail in Ref.\ \cite{mor60} and reviewed in the shell-model
context e.g.\ in Ref.\ \cite{sii98}. 

In the present shell-model calculation
we have employed three different two-body interactions.
In addition to the abovementioned USD interaction \cite{wil84}, we have 
derived microscopic effective interactions and operators based on
the recent CD-Bonn meson-exchange NN interaction model  of Machleidt {\em et al.}\
\cite{mac96} and the Nijm-I NN interaction model of the Nijmegen group \cite{nim94}.
These are the same interactions which were 
 employed by us in Ref.\  \cite{sii99}.
In order to obtain effective interactions, see Ref.\ \cite{hko95} for more
details,
and operators for the muon capture studies,
we use $^{16}$O as a closed-shell nucleus and define the 1s0d shell as
the shell-model space for which the effective interactions and operators are
derived. Based on a $G$-matrix derived for  $^{16}$O, we include all diagrams
through third-order in $G$ and sum folded diagrams to infinite order
employing the so-called 
$\hat{Q}$-box approach described in e.g.\ Ref.\ \cite{hko95}, in order
to derive an effective two-body interaction for the 1s0d shell.
In the discussions below, we will refer to these effective two-body
interactions simply as CD-Bonn and Nijm-I interactions.
%Employing the abovementioned NN interactions, we derive first the so-called
%$G$-matrix, which sums ladder diagrams to infinite order and renormalizes thereby
%the short-range part of the NN interaction.
%A harmonic oscillator basis was used in our calculation
%of the $G$-matrix with an oscillator parameter $b=1.72$ fm.
%This $G$-matrix is in turn used in a
%perturbative summation of higher-order terms using the
%so-called $\hat{Q}$-box approach described in e.g., Ref.\ \cite{hko95}.
%All diagrams through third-order in perturbation
%theory were used to define the $\hat{Q}$-box, while folded
%diagrams were summed to infinite order, see Ref.\  \cite{hko95}
%for further details.

The effective single-particle operators are calculated along the same lines 
as the effective interactions. In  
nuclear transitions, the quantity of
interest is the transition matrix element between an initial state
$\left|\Psi_i\right\rangle$ and a final state $\left|\Psi_f\right\rangle$
of an operator ${\cal O}$ defined as
\begin{equation}
               {\cal O}_{fi}=
               \frac{\left\langle\Psi_f\right|
               {\cal O}\left|\Psi_i\right\rangle }
               {\sqrt{\left\langle\Psi_f | \Psi_f \right\rangle
               \left\langle \Psi_i | \Psi_i \right\rangle}}.
               \label{eq:effop1}
\end{equation}
Since we perform our calculation in a reduced space, the exact
wave functions $\Psi_{f,i}$ are not known, only their
projections $\Phi_{f,i}$ onto the model space. We are then confronted with the
problem of how to evaluate ${\cal O}_{fi}$ when only the model
space wave functions are known. In treating this problem, it is usual
to introduce an effective operator
${\cal O}_{fi}^{\mathrm{eff}}$, defined by
requiring
\begin{equation}
           {\cal O}_{fi}=\left\langle\Phi_f\right |{\cal O}_{\mathrm{eff}}
           \left|\Phi_i\right\rangle.
\end{equation}
Observe that ${\cal O}_{\mathrm{eff}}$
is different from the original operator ${\cal O}$. The standard
scheme is then to employ a 
perturbative expansion for the effective operator, see e.g.\ Refs.\ 
\cite{towner87,eo77}.

To obtain effective one-body transition operators for muon capture, we
evaluate all effective operator diagrams through second-order in the
$G$-matrix obtained with  the CD-Bonn and Nijm-I interactions. Such diagrams
are discussed in the reviews by Towner \cite{towner87}
and Ellis and Osnes \cite{eo77}.  
Terms arising from meson-exchange currents have
been neglected, similarly, also the possibility
of having isobars $\Delta$ as intermediate states are omitted
since the focus here is  on nucleonic degrees
of freedom only. Moreover, the nucleon-nucleon potentials
we are employing do already contain such intermediate states.
Including $\Delta$ degrees of freedom may thus lead to a possible
double-counting.
Intermediate-state excitations in each diagram
up to $6-8\hbar\omega$ in oscillator energy were included
in order to achieve a converged result. This is also in line
with studies of effective interactions with weak tensor
forces \cite{sommerman},
such as the CD-Bonn potential employed here.

The energy spectrum of $^{20}$F, emerging from our full 1s0d-shell calculation
using $^{16}$O
as closed-shell core, is shown in Fig.\ \ref{fig:spec}. 
\begin{figure}
       \begin{center}
%       {\centering
%       \mbox{\psfig{figure=??.ps,height=11cm,width=12cm,angle=-90}}}
        \caption{Calculated and experimental \protect{\cite{CD}} energy spectra
        of $^{20}$F.}
        \label{fig:spec}
        \end{center}
\end{figure}
The agreement with experiment is good.
In particular, both the CD-Bonn and Nijm-I results are very close to
the USD ones, and the energy of the $1^+_1$ final state of the capture reaction is
reproduced almost exactly. The description of the spectrum of the double-even
$^{20}$Ne nucleus by shell-model is more trivial than the description of the
spectrum of the double-odd $^{20}$F. For this reason the agreement between the
calculated and measured \cite{CD} energy spectra of $^{20}$Ne is excellent
for all interactions and thus we refrain from a detailed comparison of the
$^{20}$Ne spectra. The shell-model calculations were performed using
the code OXBASH \cite{oxb88}. The reader should note that 
since the USD interaction is an effective
interaction operating in the 1s0d shell only, it is not possible to calculate
with this interaction
the corresponding effective operators which connect to states outside the
1s0d model space. Therefore, we have employed the effective operators obtained
with the CD-Bonn interaction 
for the USD calculation as well. Employing those from the 
Nijm-I interaction gives similar results.

The renormalization effects on the one-body transition matrix elements are
of the order of $10-30\%$, and in almost all cases we get reduction in the
absolute value.
In particular, the Gamow--Teller-type single-particle matrix elements,
corresponding to the matrix element $[101]$, reduce roughly by 10\%. However, it
should be noted, that the radial dependence in the $[101]$ matrix element
differs from the radial dependence of the pure Gamow--Teller matrix element.
The resulting nuclear
matrix elements for the transition $^{20}{\rm Ne}(0^+_{\rm g.s.})+\mu^-
\to {}^{20}{\rm F}(1^+;\, 1.057\, {\rm MeV})+\nu_\mu$
are shown in Table \ref{nme}, obtained by combining the one-body transition
matrix elements with the corresponding one-body transition densities of the
shell-model calculation.
\begin{table}
\caption{The values of the reduced nuclear matrix elements (RNME). The recoil
        matrix elements $[\dots p]$ are given in units of fm$^{-1}$.}
\begin{indented}\item[]
\begin{tabular}{lrrrrrr}\hline
             &\multicolumn{2}{c}{USD} &\multicolumn{2}{c}{CD-Bonn}&
             \multicolumn{2}{c}{Nijm-I}\\\hline
        RNME & bare & renorm & bare & renorm& bare & renorm \\
        \hline
        $[101]$  &  0.0165 &  0.0178 &  0.0200 &  0.0205 &0.0203 &0.0209 \\
        $[121]$  &  0.0037 &  0.0023 &  0.0032 &  0.0020 &0.0035 &0.0024 \\
        $[101-]$ &  0.0157 &  0.0171 &  0.0190 &  0.0196 &0.0194 &0.0201\\
        $[121+]$ &  0.0045 &  0.0028 &  0.0039 &  0.0024 &0.0042 &0.0028\\
        $[111p]$ &  0.0247 &  0.0189 &  0.0233 &  0.0179 &0.0229 &0.0180\\
        $[011p]$ & -0.0111 & -0.0075 & -0.0135 & -0.0090 &-0.0133 &-0.0089\\
        \hline
\end{tabular}
\end{indented}
\label{nme}
\end{table}
The corresponding capture rates 
obtained using the formalism of Ref.\ \cite{mor60} are shown
in Fig.\ \ref{rates} with the experimental value of Ref.\ \cite{fil98}.
The capture rates $W$ are calculated according to
        \begin{equation}\label{W}
        W=4P(\alpha Zm'_\mu)^3\frac{2J_f+1}{2J_i+1}\left(1-\frac{Q}{m_\mu+
        AM}\right)Q^2,
        \end{equation}
where $\alpha$ is the fine-structure constant, $m'_\mu$ is the reduced muon
mass, and $Q$ is the $Q$-value of the nuclear transition. The reduced
nuclear matrix
elements are included in $P$ (see Ref.\ \cite{mor60} for further details).
Instead of renormalizing the axial vector coupling constant, the
corrections are included in the effective operators.
Therefore, the calculations are performed using the bare value $C_{\rm A}/
C_{\rm V}=-1.251$.

>From Fig.\ \ref{rates} it can be seen that the renormalization increases the
capture rate for all interactions, pushing it
closer to the experimental value for both the USD, CD-Bonn and Nijm-I
interactions, when $C_{\rm P}/C_{\rm A}$ is close to the PCAC value.
%Interestingly enough, the ratio $-0.9\le C_{\rm P}/C_{\rm A}\le 0.7$, obtained
%with the renormalized USD interaction, is included in the interval
%$C_{\rm P}/C_{\rm A}=0.0\pm3.2$ of Ref.\ \cite{mof97} obtained from the angular
%correlation analysis in $^{28}$Si using the USD matrix
%elements of Junker {\em et al}.\ \cite{jun96}. On the other hand, t
The USD
calculation with the bare operators yields an interval far from a reasonable
expectation for the value of the ratio $C_{\rm P}/C_{\rm A}$. Although the
result with the renormalized operators does not overlap with experiment near
the PCAC region, the correction shifts the values to right direction.
The ratio $C_{\rm P}/C_{\rm A}$ calculated with the renormalized CD-Bonn and
Nijm-I one-body operators agrees slightly better
with the PCAC prediction. For the PCAC prediction $C_{\rm P}/C_{\rm A}\approx7$,
all the calculations yield a capture rate below the experimental
window.
\begin{figure}
       \begin{center}
%       {\centering
%       \mbox{\psfig{figure=nerate.ps,height=11cm,width=12cm,angle=-90}}}
        \caption{Capture rates leading to the $1^+_1$ (1.057 MeV) final state
        in $^{20}$F.}
        \label{rates}
     \end{center}
\end{figure}

\begin{figure}
       \begin{center}
%       {\centering
%       \mbox{\psfig{figure=nex.ps,height=11cm,width=12cm,angle=-90}}}
        \caption{Parameter $x$ of Eq.\ (\protect\ref{x}) plotted as a function of
        the ratio $C_{\rm P}/C_{\rm A}$.}
        \label{xfig}
        \end{center}
\end{figure}
As soon as the angular-correlation data on the muon
capture in $^{20}$Ne are published, the predictions of Fig.\ \ref{xfig} can
be used for the extraction of the ratio $C_{\rm P}/C_{\rm A}$. At this point
we can observe that the general trend is very similar to the $^{28}$Si case
of Ref.\ \cite{sii99}. The renormalized calculations reduce the magnitude of
$x$ for a given $C_{\rm P}/C_{\rm A}$ ratio, and the
behaviour is very similar for the USD, CD-Bonn \cite{mac96} and Nijm-I \cite{nim94}
interactions. This supports
the conclusion of Ref.\ \cite{sii99}, where the qualitative effects of the
renormalization on the $x$ were found to be interaction independent.

In conclusion, our calculations support the near interaction indepedence of the
effects of the
renormalization of the one-body transition operators involved in the shell-model
calculation of the muon-capture rates and the angular-correlation parameter $x$.
This renormalization is introduced by replacing the bare transition
operators, operating in the full Hilbert space, by effective ones, calculated
with the CD-Bonn and Nijm-I interactions and now 
operating in the shell-model valence space.
In the present work we found that these effective operators give very
satisfactory results when compared to the experimental data. This is
confirmed by the capture rates, where the agreement with experiment is
better with the effective operators. We have also given predictions for the ratio
$x=M_1(2)/M_1(-1)$, which
can be used for the determination of the ratio $C_{\rm P}/C_{\rm A}$ as soon as
the experimental anisotropy data becomes available. If $C_{\rm P}/C_{\rm A}\approx7$,
as predicted by PCAC and as seen in the capture rate
calculations, then $x\sim 0.35$ for all interactions employed. The results from
$^{28}$Si indicate however \cite{bru99,sii99} that $C_{\rm P}/C_{\rm A}\sim 5$.
The latter value would yield $x\sim 0.30$ for
the present reaction. With  $C_{\rm P}/C_{\rm A}\sim 5$,
the capture rates reported in Fig.\ \ref{rates} will clearly deviate
from experiment. How this deviation is related to the underlying one-body
transition densities and their relative magnitudes is hard to tell. An
experimental determination of $x$ would help in clarifying this point.

\section*{References}
\begin{thebibliography}{99}
\bibitem{mor60} Morita M and Fujii A 1960 {\em Phys.\ Rev.}  {\bf 118} 606
\bibitem{cie76} Ciechanowicz S 1976 {\em Nucl.\ Phys.}  {\bf A267} 472
\bibitem{gmi90} Gmitro M, Kamalov S S, \v{S}imkovic F and
        Ovchinnikova A A 1990 {\em Nucl.\ Phys.} {\bf A 507} 707
\bibitem{kuz94} Kuz'min V A, Ovchinnikova A A and Tetereva T V 1994
        {\em Physics of Atomic Nuclei} {\bf 57} 1881
\bibitem{joh96} Johnson B L, Gorringe T P, Armstrong D S,
        Bauer J, Hasinoff M D, Kovash M A, Measday D F,
        Moftah B A, Porter R and Wright D H 1996
        {\em Phys.\ Rev.} {\bf C 54} 2714
\bibitem{sii98} Siiskonen T, Suhonen J, Kuz'min V A and Tetereva T V 1998
        {\em Nucl.\ Phys.}  {\bf A635} 446
\bibitem{mof97} Moftah B A, Gete E, Measday D F, Armstrong D S,
        Bauer J, Gorringe T P, Johnson B L, Siebels B and
        Stanislaus S 1997 {\em Phys.\ Lett.} {\bf B395} 157
\bibitem{bru95} Brudanin V {\it et al} 1995 {\em Nucl.\ Phys.} {\bf A587} 577
\bibitem{sii99} Siiskonen T, Suhonen J and Hjorth-Jensen M
        {Phys.\ Rev.\ C, in press}
\bibitem{ego88} Egorov V {\em et al.} {\em PSI Annual Report 1998}
\bibitem{bru99} Brudanin V,
        Egorov V, Filipova T, Mamedov T,
        Salamatin A, Shitov Yu, Vylov Ts, Yutlandov I, Zaparov Sh,
        Deutsch J, Prieels R, Brian{\c c}on Ch, Kudoyarov M,
        Lobanov V and Pasternak A
        {\em submitted to Nucl.\ Phys.\  A}
\bibitem{par81} Parthasarathy R and Sridhar V N 1981 {\em Phys.\ Rev.}
        {\bf C23} 861
\bibitem{wil84} Wildenthal B H 1984 {\em Prog.\ Part.\ Nucl.\ Phys.}  {\bf 11} 5
\bibitem{jun96} Junker K, Kuz'min V A, Ovichinnikova A A and
        Tetereva T V 1995 {\em Proc.\ IV Int.\ Symp.\ on Weak and
        Electromagnetic Interactions in Nuclei (WEIN'95)} (Osaka, Japan)
        eds. Ejiri H, Kishimoto T and Sato T (Singapore: World Scientific) p 394
\bibitem{com83} Commins E D and Bucksbaum P H 1983 {\em Weak Interactions
        of Leptons and Quarks} (Cambridge: Cambridge University Press)
        Ch 4.11
\bibitem{car86} Carchidi M, Wildenthal B H and Brown B A 1986 {\em Phys.\ Rev.}
        {\bf C34} 2280
\bibitem{bro87} Brown B A and Wildenthal B H 1987 {\em Nucl.\ Phys.} {\bf A474}
        29
\bibitem{mac96} Machleidt R, Sammarruca F and Song Y 1996
        {\em Phys.\ Rev.} {\bf C53} R1483
\bibitem{nim94} Stoks V G J, Klomp R A M, Terheggen C P F and de Swart J J 1994
        {\em Phys.\ Rev.} {\bf C49} 2950
\bibitem{hko95} Hjorth-Jensen M, Kuo T T S and Osnes E 1995 {\em Phys.\ Rep.}
        {\bf 260} 125
\bibitem{towner87} Towner I S 1987 {\em Phys.\ Rep.} {\bf 155} 263; Castel B
        and Towner I S 1990 {\em Modern Theories
        of Nuclear Moments} (Oxford: Clarendon Press) p 55
\bibitem{eo77} Ellis P J and Osnes E 1997 {\em Rev.\ Mod.\ Phys.} {\bf 49} 777
\bibitem{sommerman} Sommerman H M, M\"uther H,
        Tam K C, Kuo T T S and Faessler A 1981 {\em Phys.\ Rev.} {\bf C23} 1765
\bibitem{CD} Firestone R B, Shirley V S, Chu S Y F,
        Baglin C M and Zipkin J 1996 {\em Table of Isotopes CD-ROM} Eighth Edition,
        Version 1.0 (New York: Wiley-Interscience)
\bibitem{oxb88} Brown B A, Etchegoyen A and Rae W D M 1988
        {\em The computer code OXBASH} MSU-NSCL report 524
\bibitem{fil98} Filipova T 1998 {\em private communication}
\end{thebibliography}

\end{document}


--------------9743CA4128534D98942CBF49--



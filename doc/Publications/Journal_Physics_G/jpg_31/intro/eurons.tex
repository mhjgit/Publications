 
\section{Challenges for the Nuclear Many-Body Problem} 

Intricate nuclear forces, which
have yet to be completely determined, two different fermionic species 
(protons and neutrons), and the lack of an external force, generate a
range and diversity of behaviors that make the nucleus a truly unique 
quantum many-body system. One major goal of the physics of nuclei is to
develop a unified predictive theory of nuclei and nuclear matter 
that can describe the diverse phenomena found in nuclei. 
Furthermore, physical properties, such as masses and life-times,
of very short-lived, and hence very rare, nuclei are important
ingredients that determine element production mechanisms in
the universe. 

While nuclear scientists have developed many excellent descriptions
that embody various properties of the nucleus, a full microscopic
understanding is lacking. The challenge is therefore to use 
the properties of unstable and short-lived nuclei to obtain that
fully microscopic description. How to deal with weakly
bound systems and coupling to resonant states is an unsettled problem in
nuclear spectroscopy. Similarly,  a description of  larger nuclei using a
fully microscopic first principles ("ab initio") approach, starting from the
fundamental laws of quantum theory is another unresolved problem
in nuclear physics that awaits a satisfactory, computationally tractable,
solution.


\subsection{From Free Interactions to Few- and Many-Body Systems}  

The connection between QCD and the free two-body and three-body interactions
is still an open and unresolved problem in nuclear physics. 
It is crucial to the dialectics of all {\it ab initio} methods, which start 
typically with a two-body interaction and eventually a three-body interaction 
fitted to reproduce low-energy scattering data and properties
of selected light nuclei. 
With petascale computers available within two-three years, there is considerable hope that
models for the nucleon-nucleon interaction can be better constrained from Lattice QCD calculations.
The latter will have  top priority in the future, namely to develop nuclear interactions constrained from
first principle.  These interactions will in turn be used in a nuclear many-body environment.

\subsection{'Ab Initio' Many-Body Methods}

In order to move beyond the standard shell model, and beyond the
lightest of nuclei, it will be necessary to
identify methods which can complement the shell model for heavier 
systems and preserve an {\it ab initio} philosophy. Presently, 
the Green's function Monte Carlo using realistic
and bare two- and three-body potentials, 
and the No-Core shell-model 
approaches offer very precise few and many-body 
calculations of systems with 
$A\le 16$.  The Coupled Cluster approach, inspired 
from the success in quantum chemistry, have shown a 
potential for performing similar calculations
for nuclei with $A\le 16$ and $A> 16$. 
Other methods based on Green's function approaches and 
Block-Horowitz are interesting candidates for 'ab initio' studies.

Another important topic is to constrain density-functional approaches from ab initio
methods.


\subsection{Methods for Unstable Systems} 

It is important to identify and investigate 
methods that will extend to unstable 
systems, where we especially face the problem of an 
increasing single-particle 
level density and likely resonant states.
Presently, there are several attempts at  
combining shell-model 
technologies with  studies of weakly bound systems. 
One of the promising approaches is 
the so-called Gamow shell model. This method
is based on complex scaling techniques 
developed in quantum chemistry and atomic physics.


\subsection{Shell Model and Effective Interactions} 

For heavier systems, one will most likely need to 
complement the shell model with methods which allow 
for precise derivations of effective
interactions including both two-body and three-body effective interactions.
Presently, the  shell model with effective two-body interactions based on 
e.g., perturbative many-body methods,
offers a very good description of the 
excited spectra of several nuclei.  However, this approach faces 
a number of challenges and fails in reproducing 
binding energies, single-particle energies and basically  
all known shell closures. Recent Green's function Monte Carlo 
and No-Core shell-model 
calculations demonstrate the need of three-body 
interactions.  But for heavier nuclei
such calculations are not feasible due to the 
large dimensionalities involved.
One needs, therefore, in parallel more phenomenologically based 
effective interactions. These play a crucial role
since, being fitted to reproduce the available body of data, 
they may be of great help in clarifying
which matrix elements are of importance for shell 
closures and other features of 
excited states.   

\subsection{Computational and Algorithmic Issues}

Here one needs to single out important computational and 
algorithmic developments.  
Parallel technology offers, for example, a totally new paradigm
for many-body theories. It is imperative, therefore, 
to develop methods which are capable
of utilizing these advances fully.  


\section{Experimental Challenges and Many-Body Methods} 


Nuclear many-body theory is now in the position
where precise and benchmark calculations can be performed with a given
two-body and/or three-body Hamiltonian  for light nuclei.. 
For mass $A\le 16$, the Green's function Monte Carlo
approach, the No-Core shell model and Coupled 
Cluster approaches,
represent in principle {\it ab initio} methods. 
This offers wide perspectives for studies
of various Hamiltonians used in  solving the 
non-relativistic Schr\"odinger
equation. An eventual disagreement with data can 
then be retraced to our Hamiltonian.
The calculations with three-body interactions of 
clearly demonstrate this point. Unless a three-body 
interaction is included, one cannot  reproduce the binding 
energy or some excited states for nuclei with $A\le 16$. 

For stable nuclei, important for the $s$-process and Big Bang nucleosynthesis,
we have typically proton and neutron separation energies 
$S_p$ and $S_n$ of the  order of $S_p\sim S_n \sim 8$ MeV,
and the excited states are dominated by an interplay 
between collective and single-particle  degrees of freedom. 
However, for stable nuclei with $A > 16$, we are 
not able to describe properties such as shell 
closures or the binding energy based on a microscopic approach,
unless we employ a fitted two-body interaction in shell-model studies. 
Well-known cases are the shell-closures in $^{48}$Ca or 
the chain of oxygen isotopes.
We may, however, be able to reproduce several excited 
states of stable nuclei starting from a perturbative 
many-body approach.  It is, on the other hand, difficult 
to extend such approaches in 
order to include three-body interactions or go 
beyond a certain order in perturbation theory. 
This poses serious challenges to nuclear many-body 
theories when we move to proton-rich
or neutron rich nuclei. Such nuclei have been produced recently 
or will be studied in the near future. As an example, neutron-rich nuclei, with 
$S_p\sim 15$ MeV and $S_n\sim 0$ MeV and crucial 
for the $r$-process, supernovae and 
star formation, have been studied extensively in 
the last years. They exhibit features 
like halos and resonances which put a strong pressure on existing   
many-body techniques. How to deal with weakly bound states and 
an increased number of single-particle degrees of 
freedom is not easy to account for in present shell-model 
approaches. Other features like the predictions of 
new shell-closures and their origin
are also properties we would expect a many-body approach to deal with. 





Summarizing, it is our firm belief 
that new developments in many-body theories
for nuclear problems should contain as many as possible of the 
following ingredients:
\begin{itemize}
\item
It should be fully microscopic and start with present two- and three-body
interactions derived from {\it e.g.,} effective field theory and/or constrained from Lattice QCD calculations;
\item It can be improved upon systematically, e.g., by inclusion of
three-body interactions and more complicated correlations;
\item It allows for description of both closed-shell 
systems and valence systems;
\item For nuclear systems where shell-model studies are the only feasible ones,
viz., a small model space requiring an effective interaction, 
one should be able to
derive  effective two and three-body 
equations and interactions for the shell
model;
\item It is amenable to parallel computing;
\item It can be used to generate excited spectra for nuclei like 
where many shells are involved (It is hard for the traditional shell model
to go beyond one major shell.  The inclusion of several shells may imply 
the need of  complex effective interactions
needed in studies of weakly bound systems); and
\item Allow for studies of  the interface between density functional theories and ab initio methods for both  stable and unstable nuclei; and  
\item Finally, nuclear structure results should be used in marrying microscopic 
many-body results with reaction studies. This will be another hot topic
of future {\it ab initio} research.
\end{itemize}


